% ref: https://tikz.net/venn-diagrams/
% author: Jake

\documentclass{article}
\usepackage{tikz}
\usetikzlibrary{calc}

\begin{document}
\tikzstyle{reverseclip}=[insert path={(current page.north east) --
  (current page.south east) --
  (current page.south west) --
  (current page.north west) --
  (current page.north east)}
]

\tikzset{
    venn0/.code={
        \begin{scope}
            \begin{pgfinterruptboundingbox}
            \path  [clip] (-90:0.7cm) circle [radius=1cm] [reverseclip];
            \path  [clip] (30:0.7cm) circle [radius=1cm] [reverseclip];
            \path  [clip] (-210:0.7cm) circle [radius=1cm] [reverseclip];
            \end{pgfinterruptboundingbox}
            \fill [orange] (-2,-2) rectangle (2,2);
        \end{scope}
    },
    venn1/.code={
        \begin{scope}
            \begin{pgfinterruptboundingbox}
            \path  [clip] (-90:0.7cm) circle [radius=1cm];
            \path  [clip] (30:0.7cm) circle [radius=1cm] [reverseclip];
            \path  [clip] (-210:0.7cm) circle [radius=1cm] [reverseclip];
            \end{pgfinterruptboundingbox}
            \fill [orange] (-2,-2) rectangle (2,2);
        \end{scope}
    },
    venn2/.code={
        \begin{scope}
            \begin{pgfinterruptboundingbox}
            \path  [clip] (-90:0.7cm) circle [radius=1cm] [reverseclip];
            \path  [clip] (30:0.7cm) circle [radius=1cm];
            \path  [clip] (-210:0.7cm) circle [radius=1cm] [reverseclip];
            \end{pgfinterruptboundingbox}
            \fill [orange] (-2,-2) rectangle (2,2);
        \end{scope}
    },
    venn3/.code={
        \begin{scope}
            \begin{pgfinterruptboundingbox}
            \path  [clip] (-90:0.7cm) circle [radius=1cm] [reverseclip];
            \path  [clip] (30:0.7cm) circle [radius=1cm] [reverseclip];
            \path  [clip] (-210:0.7cm) circle [radius=1cm];
            \end{pgfinterruptboundingbox}
            \fill [orange] (-2,-2) rectangle (2,2);
        \end{scope}
    },
    venn4/.code={
        \begin{scope}
            \begin{pgfinterruptboundingbox}
            \path  [clip] (-90:0.7cm) circle [radius=1cm] [reverseclip];
            \path  [clip] (30:0.7cm) circle [radius=1cm] ;
            \path  [clip] (-210:0.7cm) circle [radius=1cm] ;
            \end{pgfinterruptboundingbox}
            \fill [orange] (-2,-2) rectangle (2,2);
        \end{scope}
    },
    venn5/.code={
        \begin{scope}
            \begin{pgfinterruptboundingbox}
            \path  [clip] (-90:0.7cm) circle [radius=1cm];
            \path  [clip] (30:0.7cm) circle [radius=1cm] [reverseclip];
            \path  [clip] (-210:0.7cm) circle [radius=1cm];
            \end{pgfinterruptboundingbox}
            \fill [orange] (-2,-2) rectangle (2,2);
        \end{scope}
    },
    venn6/.code={
        \begin{scope}
            \begin{pgfinterruptboundingbox}
            \path  [clip] (-90:0.7cm) circle [radius=1cm];
            \path  [clip] (30:0.7cm) circle [radius=1cm] ;
            \path  [clip] (-210:0.7cm) circle [radius=1cm] [reverseclip];
            \end{pgfinterruptboundingbox}
            \fill [orange] (-2,-2) rectangle (2,2);
        \end{scope}
    },
    venn7/.code={
        \begin{scope}
            \begin{pgfinterruptboundingbox}
            \path  [clip] (-90:0.7cm) circle [radius=1cm];
            \path  [clip] (30:0.7cm) circle [radius=1cm] ;
            \path  [clip] (-210:0.7cm) circle [radius=1cm];
            \end{pgfinterruptboundingbox}
            \fill [orange] (-2,-2) rectangle (2,2);
        \end{scope}
    },
    vennoutlines/.code={
        \draw (-90:0.7cm) circle [radius=1cm];
        \draw (30:0.7cm) circle [radius=1cm];
        \draw (-210:0.7cm) circle [radius=1cm];
    }
}

\noindent%
\foreach \a in {0,1}
\foreach \b in {0,1}
\foreach \c in {0,1}
\foreach \d in {0,1}
\foreach \e in {0,1}
\foreach \f in {0,1}
\foreach \g in {0,1}
\foreach \h in {0,1}{%
\begin{tikzpicture}[remember picture, scale=0.2]
\ifnum\a=0
    \tikzset{venn0}
\fi
\ifnum\b=0
    \tikzset{venn1}
\fi
\ifnum\c=0
    \tikzset{venn2}
\fi
\ifnum\d=0
    \tikzset{venn3}
\fi
\ifnum\e=0
    \tikzset{venn4}
\fi
\ifnum\f=0
    \tikzset{venn5}
\fi
\ifnum\g=0
    \tikzset{venn6}
\fi
\ifnum\h=0
    \tikzset{venn7}
\fi
\tikzset{vennoutlines}
\end{tikzpicture}
}

\end{document}