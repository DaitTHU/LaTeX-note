\chapter{参数估计}

\begin{definition}{总体}{population}
	总体(population)是指与所研究的问题有关的对象(个体)的全体的某个数值特征$X$的概率分布。
\end{definition}
总体分为有限总体和无限总体。
\begin{definition}{统计模型}{model}
	统计模型(model)是一族概率分布。
\end{definition}
可以通过若干个参数表达出来的模型是参数模型,否则是非参数模型。
\begin{definition}{样本}{sample}
	样本(sample)是随机变量序列$X_1,X_2,\ldots,X_n$,$X_i$取自总体,$n$称为样本容量。
\end{definition}
样本的获取方式:试验、观测。
\begin{definition}{简单随机抽样}{simple random sampling}
	简单随机抽样(simple random sampling)总体个数$N$有限,无放回,任意容量相同的样本都有相同的发生概率。
	\[
		p=\division{1}{\binom Nn}.
	\]
\end{definition}
\begin{definition}{随机样本}{random sample}
	$X_1,\ldots,X_n$ iid,比如有放回抽样
\end{definition}
不当抽样。
\begin{definition}{统计量}{statistic}
	统计量(statistic)是完全由样本决定的量,比如样本期望$\avg X$和样本方差$S^2$
	\[
		\avg X:=\frac1n\sum_{i=1}^nX_i,\quad S^2:=\frac1{n-1}\sum_{i=1}^n(X_i-\avg X)^2.
	\]
\end{definition}
总体决定样本,用样本来推断总体,这就是统计推断。
\section{矩估计}
随机样本$X_1,X_2,\ldots$ iid
\begin{definition}{样本矩}{}
	样本$k$阶原点矩
	\[
		a_k:=\frac1n\sum_{i=1}^nX_i^k\pto\E(X^k),
	\]
	$k$阶中心矩
	\[
		m_k:=\frac1n\sum_{i=1}^n(X_i-\avg X)^k\pto\E\bigfkh{(X-\mu)^k},
	\]
\end{definition}
矩估计就是认为总体矩$\doteq$样本矩,从而得到参数的估计。
\begin{example}{正态分布和Poisson分布的矩估计}{}
	iid $X_i\sim\Norm(\mu,\sigma^2)$用样本均值$\avg X$和样本2阶中心矩$m_2$估计
	\[
		\begin{cases}
			\mu=\avg X,\\
			\sigma^2=m_2
		\end{cases}
	\]
	\tcblower
	iid $X_i\sim\Expo(\lambda)$的参数$\lambda$有不止一种估计方法
	\[
		\lambda=\frac1{\avg X}\enspace\text{or}\enspace\lambda=\frac1{\sqrt{m_2}}.
	\]
	但是我们采用低阶矩$\avg X$,因为其受噪声的影响更小。
\end{example}
\section{极大似然估计}
假设$X_1,\ldots,X_n$的联合分布为 
\[
	f(x_1,\ldots,x_n;\theta),
\]
其中$\theta$可为标量,也可为向量。
\begin{definition}{似然函数}{likelihood function}
	对于观测$(X_1,\ldots,X_n)$的似然函数(likelihood function)
	\begin{equation}
		L(\theta):=f(X_1,\ldots,X_n;\theta).
	\end{equation}
\end{definition}

\begin{remark}
	观测数据通常记为$x_1,\ldots,x_n$,称为随机变量$X_1,\ldots,X_n$的一个实现。
	% 离散情形,$L(\theta)$即为实现$X_1,\ldots,X_n$的概率。
\end{remark}

\begin{corollary}
	若$X_1,\ldots,X_n$ iid,总体分布为$g(x;\theta)$,则
	\[
		L(\theta)=g(X_1;\theta)\cdots g(X_n;\theta).
	\]
\end{corollary}

\begin{definition}{极大似然估计}{maximum likelihood estimation}
	$\theta$的极大似然估计(maximum likelihood estimation, MLE)为\index{MLE, 极大似然估计}
	\begin{equation}
		\theta^\ast:=\mathop{\arg\max}_\theta L(\theta).
	\end{equation}
\end{definition}
\begin{example}{正态分布的MLE}{MLE of Normal distribution}
	iid $X_i\sim\Norm(\mu,\sigma^2)$,$\mu,\sigma^2$未知,则其似然函数为
	\[
		L(\mu,\sigma^2)=\prod_{i=1}^n\fkh{\frac1{\sqrt{2\pi}\sigma}\exp\kh{-\frac{(X_i-\mu)^2}{2\sigma^2}}}
	\]
	要求$L$的最大值点,可对$L$求导,然而更方便的做法是求对数似然:
	\[
		\ln L(\mu,\sigma^2)=-\frac n2\ln2\pi-n\ln\sigma-\frac1{2\sigma^2}\sum_{i=1}^n(X_i-\mu)^2.
	\]
	最大值点处满足
	\[
		\begin{cases}
			\pv{\ln L}\mu=0\\[1ex]
			\pv{\ln L}{(\sigma^2)}=0
		\end{cases}\implies
		\begin{cases}
			\mu^\ast=\avg X\\
			(\sigma^2)^\ast=m_2=\frac1n\sum_{i=1}^n(X_i-\avg X)^2.
		\end{cases}
	\]
	经验证,$\mu^\ast,(\sigma^2)^\ast$为所求MLE。
\end{example}
\begin{remark}
	MLE具有不变性:
	\begin{equation}
		\bigkh{g(\theta)}^\ast=g(\theta^\ast).
	\end{equation}
\end{remark}
\begin{example}{均匀分布的MLE}{MLE of Uniform distribution}
	iid $X_i\sim\Unif(0,\theta)$,$\theta$未知
	\[
		L(\theta)=\begin{cases}
			\frac1{\theta^n},&0<X_i\leqslant\theta\\
			0,&\text{elsewhere}
		\end{cases}
	\]
	$\theta^\ast=\max(X_1,\ldots,X_n)$
\end{example}
\begin{example}{Cauchy分布的估计}{estimation of Cauchy distribution}
	估计Cauchy分布的参数$\theta$
	\[
		f(x;\theta)=\frac1\pi\frac1{1+(x-\theta)^2},\quad x\in\RR
	\]
	期望不存在,无矩。
	
	似然方程
	\[
		\dv{\ln L}\theta=0,\implies\sum_{i=1}^n\frac{X_i-\theta}{1+(X_i-\theta)^2}=0.
	\]
	当$n>2$时很难解。回到参数的意义,$\theta$可用样本中位数估计。
\end{example}
估计方法不唯一,即使MLE也不一定唯一,MLE需要$f$ (分布的参数形式),算法。

\section{Bayes估计}

\begin{definition}
	{Bayes估计}{Bayes estimation}
	矩估计和MLE都是点估计,将参数$\theta$被视为未知的数(组)。
	若对参数$\theta$有先验知识,可用随机变量$\varTheta$的概率分布来刻画,称之为先验分布(priori distribution)
	\[
		f_\varTheta(\theta)
	\]
	$\theta$是$\varTheta$的实现值;$X$是试验(观测)结果,样本分布为
	\[
		f_{X|\varTheta}(x|\theta)
	\]
	由Bayes公式,后验分布(posterior distribution)为
	\begin{equation}
		f_{\varTheta|X}(\theta|X)=\frac{f_\varTheta(\theta)f_{X|\varTheta}(x|\theta)}{f_X(x)}=\frac{f_\varTheta(\theta)f_{X|\varTheta}(x|\theta)}{\textstyle\int f_\varTheta(\theta)f_{X|\varTheta}(x|\theta)\d\theta}.
	\end{equation}
	若$\varTheta\sim\Unif$为均匀分布,称为Bayes法则,也叫同等无知原则。
\end{definition}

\begin{example}{二项分布的Bayes估计}{}
	% $X=n$次掷币正面向上次数,
	当$\varTheta=\theta$时,$X\sim\Bino(n,\theta)$
	\[
		f_{X|\varTheta}(x|\theta)=\binom nx\theta^x(1-\theta)^{n-x},\quad x=0,1,\ldots,n.
	\]
	采用Bayes法则,先验分布$\theta\sim\Unif(0,1)$,则$(X,\varTheta)$的联合分布PDF为
	\[
		f_{X,\varTheta}(x,\theta)=f_{\varTheta}(\theta)f_{X|\varTheta}(x|\theta)=\binom nx\theta^x(1-\theta)^{n-x},\quad\theta\in(0,1).
	\]
	所以
	\[
		f_X(x)=\binom nx\int_0^1\theta^x(1-\theta)^{n-x}\d\theta=\binom nx\mathrm B(x+1,n-x+1)=\frac1{n+1}.
	\]
	故后验概率
	\[
		f_{\varTheta|X}(\theta|x)=\frac{(n+1)!}{x!(n-x)!}\theta^x(1-\theta)^{n-x}.
	\]
	故$\theta\sim\mathrm{Be}(x+1,n-x+1)$,
	
	\[
		\hat\theta=\E(\varTheta|x)=\frac{x+1}{n+2}.
	\]
	或
	\[
		\theta^\ast=\arg\max f_{\varTheta|X}(\theta|x)=\frac xn.
	\]
	\tcblower
	Bayes法则的先验分布为$\Unif(0,1)=\mathrm{Be}(1,1)$,若先验分布取为$\mathrm{Be}(a,b)$,则后验分布为$\mathrm{Be}(a+x,b+n-x)$
\end{example}

\begin{remark}
	先验分布并不是唯一指定的。
\end{remark}

\section{无偏性}

\begin{definition}{偏差}{bias}
	统计量
	\[
		\hat\theta=\hat\theta(X_1,\ldots,X_n).
	\]
	是对$\theta$的一个估计,定义偏差(bias)%\index{bias, 偏差}
	\begin{equation}
		\E(\hat\theta-\theta)=\E(\hat\theta)-\theta.
	\end{equation}
	若$\forall\theta,\enspace\E(\hat\theta)=\theta$,则$\hat\theta$是$\theta$的一个无偏估计(unbiased estimation)。
\end{definition}
$\hat g(X_1,\ldots,X_n)$是$g(\theta)$的无偏估计当
\[
	\E(\hat g)=g(\theta),\quad\forall\theta.
\]
若无偏,由LLN
\[
	\frac1{N}\sum_{n=1}^N\hat g(X_1^{(n)},\ldots,X_n^{(n)})\asto\E(\hat g)=g(\theta).
\]
\begin{example}{样本期望、方差的无偏性}{unbiasedness sample expectation and variance}
	样本均值$\avg X$的期望和方差
	\begin{subequations}
		\begin{align}
			\label{eq:E(avg X)}
			\E(\avg X)&=\frac1n\sum_{i=1}^n\E(X_i)=\mu,\\
			\label{eq:Var(avg X)}
			\Var(\avg X)&=\frac1{n^2}\sum_{i=1}^n\Var(X_i)=\frac{\sigma^2}n.
		\end{align}
	\end{subequations}
	样本方差
	\[
		S^2=\frac1{n-1}\sum_{i=1}^n(X_i-\avg X)^2=\frac1{n-1}\kh{\sum_{i=1}^nX_i^2-n\avg X^2}
	\]
	的期望
	\begin{equation}
		\label{eq:E(S2)}
		\E(S^2)=\frac1{n-1}\fkh{\sum_{i=1}^n\E(X_i^2)-n\E(\avg X^2)}=\frac n{n-1}\fkh{\sigma^2+\mu^2-\kh{\frac{\sigma^2}n+\mu^2}}=\sigma^2.
	\end{equation}
	样本方差$S^2$的方差与样本分布的峰度系数$\nu_4=\E\bigfkh{(X_i-\mu)^4/\sigma^4}$有关,
	记$Y_i:=X_i-\mu$,
	\begin{align*}
		(n-1)^2\E(S^4)&=\E\biggfkh{\biggkh{\sum_{i=1}^nY_i^2}^2}-2n\sum_{i=1}^n\E(Y_i^2\avg Y^2)+n^2\E(\avg Y^4)\\
		&=\biggfkh{n\nu_4+n(n-1)-2n\cdot\frac{\nu_4+(n-1)}n+n^2\cdot\frac{\nu_4+3(n-1)}{n^3}}\sigma^4.
	\end{align*}
	由此
	\begin{equation}
		\label{eq:Var(S2)}
		\Var(S^2)=\biggfkh{\frac{\nu_4}n-\frac{n-3}{n(n-1)}}\sigma^4.
	\end{equation}
	比如正态分布的峰度系数$\nu_4=3$,$\Var(S^2)=2\sigma^4/(n-1)$。
	% 特别地,正态分布的$\Var(S^2)$也可以通过$(n-1)S^2/\sigma^2\sim\chi^2(n-1)$来方便计算。
\end{example}
\begin{example}{均匀分布的无偏估计}{unbiased estimation of Uniform distribution}
	在\exmref{exm:MLE of Uniform distribution} 中给出了均匀分布$\Unif(0,
	\theta)$的MLE 
	\[
		\theta^\ast=\max(X_1,\ldots,X_n).
	\] 
	这种估计是有偏的,($\E(\theta^\ast)=n\theta/(n+1)$)。而
	以下四种估计都是无偏的:
	\begin{subequations}
		\begin{align}
			\hat\theta_1&:=\frac{n+1}n\max(X_1,\ldots,X_n).\\
			\hat\theta_2&:=\max(X_1,\ldots,X_n)+\min(X_1,\ldots,X_n),\\
			\hat\theta_3&:=2\avg X,\enspace\text{(矩估计)}\\
			\hat\theta_4&:=(n+1)\min(X_1,\ldots,X_n).
		\end{align}
	\end{subequations}
	但其精确度越来越差:$\Var(\hat\theta_1)\leqslant\Var(\hat\theta_2)\leqslant\Var(\hat\theta_3)\leqslant\Var(\hat\theta_4)$,具体而言,
	\begin{subequations}
		\begin{align}
			\Var(\hat\theta_1)&=\frac{\theta^2}{n(n+2)},\\
			\Var(\hat\theta_2)&=\frac{2\theta^2}{(n+1)(n+2)},\\
			\Var(\hat\theta_3)&=\frac{\theta^2}{3n},\\
			\Var(\hat\theta_4)&=\frac{n\theta^2}{n+2}.
		\end{align}
	\end{subequations}
	\tcblower
	下面推导这四个估计的方差。首先是最平凡的,
	\[
		\Var(\hat\theta_3)=4\Var(\avg X)=\frac4n\Var(X_i)=\frac4n\cdot\frac{\theta^2}{12}=\frac{\theta^2}{3n}.
	\]
	令$Y:=\min(X_1,\ldots,X_n),\enspace Z=\max(X_1,\ldots,X_n)$,则由\eqref{eq:min-max distribution}可得PDF
	\begin{subequations}
		\begin{align}
			f_Y(y)&=\dd y\biggkh{1-\frac y\theta}^n=\frac n\theta\biggkh{1-\frac y\theta}^{n-1},\\
			f_Z(z)&=\dd z\biggkh{\frac z\theta}^n=\frac n\theta\biggkh{\frac z\theta}^{n-1}.
		\end{align}
	\end{subequations}
	继而
	\begin{align*}
		\E(Y^2)&=\frac n\theta\int_0^\theta y^2\biggkh{1-\frac y\theta}^{n-1}\d y=\frac{2\theta^2}{(n+1)(n+2)},\\
		\E(Z^2)&=\frac n{\theta^n}\int_0^\theta y^{n+1}\d y=\frac{n\theta^2}{n+2}.
	\end{align*}
	可得 
	\begin{equation}
		\Var(Y)=\Var(Z)=\frac{n\theta^2}{(n+1)^2(n+2)}.
	\end{equation}
	因此
	\begin{align*}
		\Var(\hat\theta_4)&=(n+1)^2\Var(Y)=\frac{n\theta^2}{n+2};\\
		\Var(\hat\theta_1)&=\biggkh{\frac{n+1}n}^2\Var(Z)=\frac{\theta^2}{n(n+2)}.
	\end{align*}
	最后求$\Var(\hat\theta_2)=\Var(Y+Z)$,这需要知道$Y,Z$的联合分布,由\eqref{eq:min-max joint distribution},
	\begin{equation}
		f_{Y,Z}(y,z)=\frac{n(n-1)}{\theta^n}(z-y)^{n-2},\quad 0<y<z<\theta.
	\end{equation}
	则
	\[
		\E(YZ)=\int_0^\theta\int_y^\theta yzf_{Y,Z}(y,z)\d z\d y=\frac1{\theta^n}\int_0^\theta y\Bigfkh{\Bigkh{(n-1)z+y}(z-y)^{n-1}}_y^\theta\d y=\frac{\theta^2}{n+2}.
	\]
	故
	\begin{equation}
		\Cov(Y,Z)=\E(YZ)-\E(Y)\E(Z)=\frac{\theta^2}{n+2}-\frac{n\theta^2}{(n+1)^2}=\frac{\theta^2}{(n+1)^2(n+2)}.
	\end{equation}
	因此
	\[
		\Var(\hat\theta_2)=\Var(Y+Z)=\Var(Y)+\Var(Z)+2\Cov(Y,Z)=\frac{2\theta^2}{(n+1)(n+2)}.
	\]
\end{example}

\section{均方误差准则}

\begin{definition}{均方误差}{mean square error}
	定义均方误差(mean square error, MSE)\index{MES, 均方误差}
	\begin{equation}
		\E\bigfkh{(\hat\theta-\theta)^2}=\Var(\hat\theta)+\E^2(\hat\theta-\theta).
	\end{equation}
	由两部分构成,$\Var(\hat\theta)$称为精确度(precision),$\E^2(\hat\theta-\theta)$代表准确度(accuracy)。后者在无偏估计中为0。
\end{definition}
%由定义可以看出,
\begin{definition}{有效性}{}
	假定$\hat\theta_1,\hat\theta_2$是$\theta$的无偏估计,若
	\[
		\Var(\hat\theta_1)\leqslant\Var(\hat\theta_2),\quad\forall\theta
	\]
	且存在一个分布(i.e.一个$\theta$值)使得$<$成立,则称在MSE意义下,$\hat\theta_1$比$\hat\theta_2$更有效。
	
	如果所有无偏估计中存在最小的方差,则称为一致最小方差无偏估计(uniformly minimum variance unbiased estimator, UMVUE)。
\end{definition}
\begin{theorem}{Cramer-Rao下限}{}
	任何$\theta$的无偏估计量满足
	\begin{equation}
		\Var(\hat\theta)\geqslant\frac1{I_n(\theta)}.
	\end{equation}
	其中$I_n(\theta)$是Fisher信息量,见\dfnref{dfn:Fisher information}
\end{theorem}
\begin{example}{低偏倚换方差}{}
	总体$X\sim\Norm(\mu,\sigma^2)$,样本方差$S^2$是无偏的,而样本二阶矩$m_2$是有偏的,
	但MSE
	\[
		\E\bigfkh{(m_2-\sigma^2)^2}<\E\bigfkh{(S^2-\sigma^2)^2}.
	\]
	由于正态分布的样本方差满足$(n-1)S^2/\sigma^2\sim\chi^2(n-1)$,故
	\begin{equation}
		\E\bigfkh{(S^2-\sigma^2)^2}=\Var(S^2)+0=\biggkh{\frac{\sigma^2}{n-1}}^2\cdot 2(n-1)=\frac{2}{n-1}\sigma^4.
	\end{equation}
	可得
	\[
		\E\bigfkh{(m_2-\sigma^2)^2}=\biggkh{\frac{\sigma^2}{n}}^2\cdot 2(n-1)+\biggkh{\frac{n-1}n-1}^2\sigma^4=\frac{2n-1}{n^2}\sigma^4<\frac{2}{n-1}\sigma^4.
	\]
	\tcblower
	一般地,由\exmref{exm:unbiased estimation of Uniform distribution},
	\begin{align*}
		\E\bigfkh{(S^2-\sigma^2)^2}&=\frac{(\nu_4-1)n+3-\nu_4}{n(n-1)}\sigma^4,\\
		\E\bigfkh{(m_2-\sigma^2)^2}&=\frac{(\nu_4-1)n^2+(5-2\nu_4)n+\nu_4-3}{n^3}\sigma^4.
	\end{align*}
	二者的差
	\[
		\E\bigfkh{(S^2-\sigma^2)^2}-\E\bigfkh{(m_2-\sigma^2)^2}
		% =\frac{(3-2\nu_4)n^2+(3\nu_4-8)n+3-\nu_4}{n^3(n-1)}.
		=\frac{(2n^2-3n+1)\nu_4-(3n^2-8n+3)}{n^3(n-1)}\sigma^4.
	\]
	可证当峰度系数$\nu_4\geqslant 3/2$时,均有$\E\bigfkh{(S^2-\sigma^2)^2}>\E\bigfkh{(m_2-\sigma^2)^2}$。
\end{example}

\section{大样本性质}

$\hat\theta=\hat\theta(X_1,\ldots,X_n)$当样本容量$n\to\infty$时的性质,

\begin{definition}{渐进无偏性}{}
	$n\to\infty$时,
	\begin{equation}
		\E(\hat\theta)\to\theta.
	\end{equation}
\end{definition}
\begin{definition}{相合性}{consistent}
	称$\hat\theta$为$\theta$的(弱)相合估计(consistent estimate)当
	\begin{equation}
		\hat\theta\pto\theta.
	\end{equation}
\end{definition}
WLLN指出,$\avg X$为$\mu$的一个相合估计。相合性为良好点估计的自然要求。
\begin{example}{二阶中心矩}{}
	二阶中心矩$m_2$是总体方差$\sigma^2$的相合估计。
	\[
		m_2=\frac1n\sum_{i=1}^n(X_i-\avg X)^2=\frac1n\sum_{i=1}^n(X_i-\mu)^2-(\avg X-\mu)^2\pto\sigma^2+0.
	\]
\end{example}
\begin{definition}{渐进正态性}{}
	若$\exists\sigma_n>0$使得
	\begin{compactenum}
		\item $\lim_{n\to\infty}\sigma_n=0;$
		\item $\lim_{n\to\infty}\P\biggl(\frac{\hat\theta-\theta}{\sigma_n}\leqslant x\biggr)=\Phi(x).$
	\end{compactenum}
	则称$\hat\theta$为$\theta$的相合近似正态估计。
\end{definition}
\begin{compactenum}
	\item 有时,取$\sigma_n^2=\Var(\hat\theta)$,比如用$\avg X$估计$\mu$;
	\item $n\gg 1$时,$\hat\theta\dotsim\Norm(\theta,\sigma_n^2).$;
	\item 原始分布(即$X_i$的分布) $f$满足光滑性条件,则$\exists\sigma_n>0$使得
	\begin{equation}
		\label{MLE-dto-N01}
		\frac{\theta^\ast-\theta}{\sigma_n}\dto\Norm(0,1).
	\end{equation}
\end{compactenum}
\begin{definition}{Fisher信息量}{Fisher information}
	$X_1,\ldots,X_n$ iid,对数似然函数
	\[
		\ell(\theta):=\ln f(X;\theta)=\sum_{i=1}^n\ln f(X_i;\theta).
	\]
	定义Fisher信息量
	\begin{equation}
		I_n(\theta):=\E(\ell'(\theta)^2).
	\end{equation}
	Fisher信息量是衡量观测所得的随机变量$X$携带的关于未知参数$\theta$的信息量,其中$X$的概率分布依赖于参数$\theta$。
\end{definition}

$\ell(\theta)$导数的期望
\[
	\E(\ell'(\theta))=\int\frac1f\pv f\theta\cdot f\d x=\pp\theta\int f\d x\equiv 0.
\]
二阶导期望
\[
	\E(\ell''(\theta))=\E\biggl[\frac1f\pv[2]f\theta-\biggl(\frac1f\pv f\theta\biggr)^2\biggr]=0-\E(\ell'(\theta)^2)=-\Var(\ell'(\theta)).
\]
Fisher信息量与样本容量有关:
\[
	I_n(\theta)=\sum_{i=1}^n\E\biggl[\biggl(\pp\theta\ln f(X_i;\theta)\biggr)^2\biggr]=nI_1(\theta)=:nI(\theta).
\]
\begin{example}{Fisher信息量证明式\eqref{MLE-dto-N01}}{}
	由MLE的性质,$\ell'(\theta^\ast)=0$,做Talyor展开,
	\[
		0=\ell'(\theta^\ast)=\ell'(\theta)+(\theta^\ast-\theta)\ell''(\theta)+\cdots
	\]
	故
	\[
		\theta^\ast-\theta\doteq-\frac{\ell'(\theta)}{\ell''(\theta)}.
	\]
	设$\ell'(\theta)=n\avg Y$,
	\[
		Y_i:=\pp\theta\ln f(X_i;\theta),\quad\E(Y_i)=0,\enspace\Var(Y_i)=I(\theta).
	\]
	CLT说明,$\avg Y$是渐进正态的,故分子
	\[
		\frac1{\sqrt n}\ell'(\theta)=\sqrt n\,\avg Y\dto\Norm\bigkh{0,I(\theta)}.
	\]
	分母
	\[
		\frac1n\ell''(\theta)\pto I(\theta).
	\]

	故$\theta^\ast-\theta$是渐进正态的:
	\[
		\sqrt n(\theta^\ast-\theta)\dto\Norm\kh{0,\frac1{I(\theta)}}.
	\]
	式\eqref{MLE-dto-N01}中的 
	\[
		\sigma_n=\frac1{\sqrt{nI(\theta)}}~\text{或}~\frac1{\sqrt{nI(\theta^\ast)}}
	\]
\end{example}
\paragraph{Review}
\begin{compactenum}
	\item 参数估计:样本$X_1,\ldots,X_n$,通常假设iid
	
	样本分布$f(x_1,\ldots,x_n;\theta)$即$X_1,\ldots,X_n$的联合分布

	估计$\hat\theta=\hat\theta(X_1,\ldots,X_n)$是统计量,样本的函数

	标准误差的估计
	% \[
	% 	se(\hat\theta):=\sqrt{\Var(\hat\theta)}.
	% \]
	$\hat{se}=\hat{se}(\hat\theta)$
	\item 经典估计的优良性:
	
	$n$固定:无偏性、有效性;

	$n\to\infty$:渐进无偏性、相合性、渐进正态性。

	\item Bayes估计:将$\theta$理解成一个随机变量$\varTheta$的实现值。$\varTheta$刻画了对$\theta$的认知
	\[
		f_\varTheta(\theta)\enspace\to\enspace f_{\varTheta|X}(\theta|x),
	\]
	后验众数v.s. MLE。Bayes假设二者相等。
\end{compactenum}
\section{区间估计}
\begin{definition}{置信区间}{}
	给定$\alpha\in(0,1)$,$\forall\theta$可能值,有$\hat\theta_1,\hat\theta_2$使得
	\begin{equation}
		\P(\hat\theta_1<\theta<\hat\theta_2)\geqslant1-\alpha.
	\end{equation}
	这里$\hat\theta_i=\hat\theta_i(X_1,\ldots,X_n)$,则称$(\hat\theta_1,\hat\theta_2)$为$\theta$的$(1-\alpha)$置信的区间估计。
\end{definition}
$\alpha$通常取0.05, 0.01, 0.1,可靠性优先。
\begin{definition}{枢轴变量}{}
	仅含一个待估参数的样本的连续函数,且分布不依赖于未知参数。
\end{definition}
\begin{method}{枢轴变量法}{}
	\begin{compactenum}
		\item 确定$\hat\theta$;
		\item 找到枢轴变量$H(\hat\theta,\theta)$的分布可用;
		\item 确定$(\hat\theta_1,\hat\theta_2)$
	\end{compactenum}
\end{method}
\begin{example}{正态分布的区间估计}{}
	$X\sim\Norm(\mu,\sigma^2)$的$\sigma^2$已知,$\mu$未知,由$\avg X\sim\Norm(\mu,\sigma^2/n)$,
	取枢轴变量
	\[
		Z:=\frac{\avg X-\mu}{\sigma/\sqrt n}\sim\Norm(0,1)
	\]
	取对称的上下限
	\[
		\P\kh{\abs Z\geqslant z_{\alpha/2}}=\alpha,\implies\Phi(z_{\alpha/2})=1-\alpha/2
	\]
	其中$z_{\alpha/2}$是$\Norm(0,1)$的上$\alpha/2$分位数。
	故$\mu$的$(1-\alpha)$置信区间为
	\begin{equation}
		\kh{\avg X\pm z_{\alpha/2}\frac\sigma{\sqrt n}}.
	\end{equation}
	%若$\alpha=0.05$,则$z_{0.025}=1.96$
	%若要求误差不超过$\varepsilon$,则
	%n=\frac{z_{\alpha/2}^2\sigma^2}{\varepsilon^2}
	\tcblower
	若$\mu,\sigma^2$均未知,需要用$S^2$估计$\sigma^2$,由式(\ref{sample-chi2})
	\[
		\chi^2:=\frac{(n-1)S^2}{\sigma^2}\sim\chi^2(n-1),
	\]
	卡方分布并不对称,我们选择等尾置信区间,即
	\[
		\P\bigkh{\chi^2<\chi_{\alpha/2}^2(n-1)}=\P\bigkh{\chi^2>\chi_{1-\alpha/2}^2(n-1)}=\frac\alpha{2},
	\]
	$\sigma^2$的$(1-\alpha)$置信区间为
	\begin{equation}
		\kh{\frac{(n-1)S^2}{\chi_{\alpha/2}^2(n-1)},\frac{(n-1)S^2}{\chi_{1-\alpha/2}^2(n-1)}};
	\end{equation}

	对$\mu$的估计,由式\eqref{sample-t}
	\[
		t:=\frac{\avg X-\mu}{S/\sqrt n}\sim\mathrm t(n-1).
	\]
	%\P\bigkh{\abs G\geqslant t_{\alpha/2}(n-1)}=\alpha.
	可得$\mu$的$(1-\alpha)$置信区间为
	\begin{equation}
		\kh{\avg X\pm t_{\alpha/2}(n-1)\frac{S}{\sqrt n}}.
	\end{equation}
	\begin{center}
		\captionof{table}{区间估计}
		\begin{tabular}{cccc}
			\toprule
			条件&估计&枢轴量&置信区间\\
			\midrule
			$\sigma^2$已知&$\mu$&$\frac{\avg X-\mu}{\sigma/\sqrt n}$&$\kh{\avg X\pm z_{\alpha/2}\frac\sigma{\sqrt n}}$\\[2ex]
			$\sigma^2$未知&$\mu$&$\frac{\avg X-\mu}{S/\sqrt n}$&$\kh{\avg X\pm t_{\alpha/2}(n-1)\frac S{\sqrt n}}$\\[2ex]
			$\mu$未知&$\sigma^2$&$\frac{(n-1)S^2}{\sigma^2}$&$\biggl(\frac{(n-1)S^2}{\chi_{\alpha/2}^2(n-1)},\frac{(n-1)S^2}{\chi_{1-\alpha/2}^2(n-1)}\biggr)$\\
			\bottomrule
		\end{tabular}
	\end{center}
	同理可求单侧置信限,如$\sigma^2$已知时$\mu$的上、下单侧置信限分别为
	\[
		\mu^\pm=\avg X\pm z_\alpha\frac\sigma{\sqrt n}
	\]
\end{example}
\begin{example}{两个正态总体}{}
	$X\sim\Norm(\mu_1,\sigma^2),Y\sim\Norm(\mu_2,\sigma^2)$独立,$\mu_1,\mu_2,\sigma^2$未知,估计$\mu_1-\mu_2$
	\begin{align*}
		\avg X-\mu_1-(\avg Y-\mu_2)&\sim\Norm\Bigkh{0,\frac{\sigma^2}{n_1}+\frac{\sigma^2}{n_2}};\\
		\frac{(n_1-1)S_1^2}{\sigma^2}+\frac{(n_2-1)S_2^2}{\sigma^2}&\sim\chi^2(n_1+n_2-2).
	\end{align*}
	利用t分布的定义,构造枢轴变量:
	\[
		H:=\frac{\avg X-\mu_1-(\avg Y-\mu_2)}{S_w\sqrt{\frac1{n_1}+\frac1{n_2}}}\sim\mathrm t(n_1+n_2-2)
	\]
	其中 
	\[
		S_w:=\sqrt{\frac{(n_1-1)S_1^2+(n_2-1)S_2^2}{n_1+n_2-2}}
	\]
	$\mu_1-\mu_2$的$(1-\alpha)$置信区间为
	\[
		\kh{\avg X-\avg Y\pm t_{\alpha/2}(n_1+n_2-2)S_w\sqrt{\frac1{n_1}+\frac1{n_2}}}
	\]
\end{example}
\paragraph{大样本方法}
渐进置信区间
\begin{example}{选举问题的置信区间估计}{}
	真实支持率$p$未知,$n=1200$,观测比例
	\[
		\frac{684}{1200}\doteq 0.57
	\]
	给出$p$的一个95\% 置信区间。

	近似有放回:$X_i\sim\Bino(p)$ iid,$P_n=\avg X$,故
	\[
		\E(P_n)=p,\quad\Var(P_n)=\frac{p(1-p)}n
	\]
	由CLT,
	\[
		\frac{P_n-p}{\sqrt{p(1-p)/n}}\dotsim\Norm(0,1).
	\]
	由于$p$未知,这个方法不能直接用,需要进行一个好的估计。
	\begin{compactenum}[I]
		\item 用$S^2$估计$\sigma^2=p(1-p)$,则标准误差
			\[
				\hat{se}=\sqrt{\frac{S^2}n}=0.2475.
			\]
			可认为
			\[
				\frac{P_n-p}{\hat{se}}\dotsim\Norm(0,1).
			\]
			$p$的95\%置信区间
			\[
				(P_n\pm z_{\alpha/2}\hat{se})=(0.542,0598)
			\]
		\item 用$m_2$估计$\sigma^2$,
			\begin{align*}
				m_2&=\frac1n\sum_{i=1}^n(X_i-\avg X)^2\\
				&=P_n(1-P_n)^2+(1-P_n)(0-P_n)^2=P_n(1-P_n).
			\end{align*}
			相当于用$P_n$估计$p$,估计结果与I相同。
		\item (Naïve)用$1/4$估计$p(1-p)$,此时$\hat{se}=1/\sqrt{4n}$,估计结果为$(0.542,0.599).$
		\item (不具有推广性)由于 
		\[
			\abs{P_n-p}=z_{\alpha/2}\sqrt{\frac{p(1-p)}n},
		\]
		可以解出$p_\pm$,区间$(p_-,p_+)=(0.542,0598).$
	\end{compactenum}
\end{example}
\begin{example}{相合估计}{}
	上例,极大似然估计$P^\ast=P_n$,由MLE的渐进正态性
	\[
		\frac{P^\ast-P}{\sigma_n^2}\dotsim\Norm(0,1)
	\]
	$f(X_i;p)=p^{X_i}(1-p)^{1-X_i}$,故
	\[
		\pp p\ln f(X_i;p)=\frac{X_i}p-\frac{1-X_i}{1-p}=\frac{X_i-p}{p(1-p)}.
	\]
	Fisher信息量
	\[
		I(p)=\E\biggfkh{\biggkh{\frac{X_i-p}{p(1-p)}}^2}=\frac1{p(1-p)}.
	\]
	则
	\[
		\frac{P^\ast-p}{1/\sqrt{nI(p)}}\dotsim\Norm(0,1).
	\]
	可以用$I(P^\ast)$估计$I(p)$,即
	\[
		\frac{P_n-p}{\sqrt{P_n(1-P_n)/n}}\dotsim\Norm(0,1)
	\]
\end{example}
\begin{example}{两个正态总体的相合估计}{}
	$X\sim\Norm(\mu_1,\sigma_1^2),Y\sim\Norm(\mu_2,\sigma_2^2)$独立,自然
	\[
		\frac{\avg X-\mu_1-(\avg Y-\mu_2)}{\sqrt{\frac{\sigma_1^2}{n_1}+\frac{\sigma_2^2}{n_2}}}\sim\Norm(0,1).
	\]
	可估计
	\[
		\frac{\avg X-\mu_1-(\avg Y-\mu_2)}{\sqrt{\frac{S_1^2}{n_1}+\frac{S_2^2}{n_2}}}\dotsim\Norm(0,1).
	\]
\end{example}
\section{Beyas区间估计}
有了$\varTheta$的后验分布$f_{\varTheta|X}(\theta|x)$
\[
	\P(a<\varTheta<b|x)=1-\alpha.
\]

最大后验区间满足
\[
	f_{\varTheta|X}(a|x)=f_{\varTheta|X}(b|x).
\]
等尾可信区间满足
\[
	\P(\varTheta<a|x)=\P(\varTheta>b|x)=\frac\alpha{2}.
\]
\paragraph{Review}
\begin{compactenum}
	\item 不管是连续还是离散分布,它们的参数都是连续的。
	\item 置信区间 v.s. Bayes区间(可信区间)
	\[
		\P(\hat\theta_1<\theta<\hat\theta_2)\geqslant 1-\alpha,\quad\hat\theta_i=\hat\theta_i(X_1,\ldots,X_n)\text{——统计量}
	\]
	$(\hat\theta_1,\hat\theta_2)$——随机区间。当得到样本具体观测值$(x_1,\ldots,x_n)$代入$\hat\theta_1,\hat\theta_2$得到具体区间。
	\[
		\P(a<\varTheta<b|x)=1-\alpha.
	\]
	后验分布
	\item 小样本方法v.s.大样本方法:精确分布v.s.近似分布
\end{compactenum}