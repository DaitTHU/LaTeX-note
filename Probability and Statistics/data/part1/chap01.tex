% 引言:Monty Hall问题(三门问题);% ——考虑最开始的选择:1/3选中汽车,则换门必得羊;2/3选中羊,因为另一羊已被排除,换门必得车——因此不换门得车的概率只有1/3,而换门得车的概率为2/3。
% 你说你说分数怎么停留一直在停留谁让它停留的,为什么我女朋友场外加油你却还让我出糗。
% 一种疾病,10000人中约有一人发病,患病者检测阳性的比例为99\%,未患病者检测阳性的比例为0.1\%。则阳性报告者患病率
%\[
%	\frac{0.99}{0.99+9.999}\doteq 9\%
%\]
%因此
% 对阳性报告者进行二次检测是有必要的;	
% 敏感问题调查:敏感问题+保护性问题。
	
\chapter{概率}
\paragraph{概率的发展史}
概率的发展经历了如下几个过程:
\begin{compactitem}
	\item de M\'er\'e问题
	\item Pascal, Fermat首创概率的数学理论(初等数学方法)
	\item Laplace创立分析概率论(微积分分析方法)
	\item Kolmogorov发展现代理论(测度论方法)
\end{compactitem}
\section{试验与事件}
\begin{definition}{试验与样本空间}{experiment and sample space}
    概率论研究试验(experiment),试验具有如下性质:
    \begin{compactenum}
    	\item 不能预先确知结果;
    	\item 试验之前可预知所有可能结果,其集合构成样本空间(sample space) $\Omega$。
    \end{compactenum}
\end{definition}
\begin{definition}{事件}{event}
    事件(event)是样本空间的子集(well-defined select)。试验的单一结果称为基本事件。

    特别地,$\Omega$是必然事件,$\varnothing$是不可能事件。
\end{definition}

\paragraph{事件的运算}借助集合语言以及Venn图
\begin{compactitem}
	\item 余 $A\c = \Omega\backslash A$,对立$A=B\c$
	\item 和与差 $A+B, A-B$
	\item 积 $AB = A\cap B$,互斥$AB=\varnothing$
	\item de Morgan定律\footnote{此处省略了上下标$i=1$和$\infty$,后同。}
	\[
		\kh{\sum A_i}\c = \prod A_i\c,
	\]
\end{compactitem}

\paragraph{概率的解释}古典解释,等可能性,Bertrand悖论;频率解释,主观解释。

\section{公理化定义}
定义样本空间$\Omega$的幂集$2^\Omega$为$\Omega$所有子集构成的集合,则
\begin{definition}{事件集类}{event set}
	事件集类$\cF\subset 2^\Omega$是$\Omega$的$\sigma$-代数,满足事件运算的封闭性:
	\begin{compactenum}
		\item $\Omega \in \cF;$
		\item $\forall A \in \cF, A\c \in \cF;$
		\item 特别地,若$\forall A_i \in \cF$,则
		$
			\sum A_i \in \cF.
		$
	\end{compactenum}
\end{definition}
\begin{example}{$\sigma$-代数}{sigma-Algebra}
	若$\Omega=\{a,b,c,d\}$,则一个平凡的$\sigma$-代数
	\[
		\cF_1=\{\varnothing, \Omega\},
	\]
	另一个$\sigma$-代数可以为
	\[
		\cF_2=\{\varnothing, \{a\},\{b,c,d\},\Omega\}.
	\]
\end{example}
\begin{definition}{概率}{probability}
	定义概率(probability)~$\P:\cF\to\RR$,满足以下公理:
	\begin{compactenum}
		\item $\forall A\in\cF,\P(A)\geqslant 0;$
		\item $\P (\Omega)=1;$
		\item (\textbf{加法公理})~$\forall A_i \in \cF,A_iA_j=\varnothing(i\neq j),$
		\begin{align}
			\P\kh{\sum A_i}=\sum\P(A_i).
		\end{align}
	\end{compactenum}
	$(\Omega,\cF,\P)$构成了概率空间(probability space)。
\end{definition}
由定义,可推出以下命题:
\begin{compactenum}
	\item $\forall A\in\cF,\P(A)\leqslant 1;$
	\item $\P(\varnothing)=0;$
	\item $\forall A_i \in \cF,A_iA_j=\varnothing(i\neq j),$
	\[
		\P\kh{\sum_{i=1}^n A_i}=\sum_{i=1}^n\P(A_i)
	\]
	特别的,$\P(A)+\P(A\c)=1;$
	\item $\forall A\subset B,\P(A)\leqslant\P(B);$
	\item $\P(A+B)=\P(A)+\P(B)-\P(AB).$推广之,得到
\end{compactenum}
\begin{theorem}{容斥恒等式}{inclusion and exclusion identities}
	容斥恒等式(inclusion and exclusion identities)
	\begin{align}\notag
		\P\kh{\sum_{i=1}^nA_i}&=\sum_{i=1}^n\P(A_i)-\sum_{i<j}^n\P(A_iA_j)+\cdots+\P\kh{\prod_{i=1}^nA_i}\\
		&=\sum_{r=1}^n(-)^{r+1}\sum_{i_1<\cdots<i_r}\P\kh{\prod_{k=1}^rA_{i_k}}.
	\end{align}
\end{theorem}
\begin{example}{乱序}{out of order}
	$n$个人每人一个帽子,离开时随机取帽子,问无人拿到自己帽子的概率为多少?

	记$A_i$表示第$i$个人拿到自己的帽子,则
	\[
		\P(A_i)=\frac1n\equiv\frac{(n-1)!}{n!}.
	\]
	至少一个人拿到自己的帽子:
	\begin{align*}
		\P\kh{\sum_{i=1}^nA_i}&=\sum_{r=1}^n(-)^{r+1}\sum_{i_1<\cdots<i_r}\P\kh{A_{i_1}\cdots A_{i_r}}\\
		&=\sum_{r=1}^n(-)^{r+1}\binom nr\frac{(n-r)!}{n!}=\sum_{r=1}^n(-)^{r+1}\frac1{r!}\\
		&=1-\frac1{2!}+\frac1{3!}+\cdots+(-)^{n+1}\frac1{n!}.
	\end{align*}
	故所求概率即
	\[
		\P(A_1\c\cdots A_n\c)=1-1+\frac1{2!}-\frac1{3!}+\cdots+(-)^n\frac1{n!}\to\frac1{\e{}}.
	\]
\end{example}
\section{条件概率}
\begin{definition}{条件概率}{conditional probability}
	定义给定$B$事件发生的条件下,$A$事件发生的条件概率(conditional probability)为
	\begin{align}
		\P(A|B):=\frac{\P(AB)}{\P(B)}
	\end{align}
	其中$\P(B)>0.$
\end{definition}
条件概率可用于缩小样本空间。

事实上,给定$B$且$\P(B)>0$,则$\P(\cdot|B):\cF\to\RR$是概率函数,$\bigkh{\Omega,\cF,\P(\cdot|B)}$仍为概率空间。
\begin{theorem}{乘法法则}{Multiplication Rule}
	由条件概率的定义可以直接导出:
	\begin{align}
		\P(AB)=\P(B)\P(A|B)=\P(A)\P(B|A).
	\end{align}
	一般推广:
	\[
		\P(A_1\cdots A_n)=\P(A_1)\P(A_2|A_1)\cdots\P(A_n|A_1\cdots A_{n-1}).
	\]
	加多限制条件以后,算概率可能会变简单。
\end{theorem}
称$\P(A)$为\textbf{先验概率}(priori probability),$\P(A|B)$为\textbf{后验概率}(posterior probability)。

已观测到$A$事件,等价于$\P(A|A)\equiv 1$,绝非$\P(A)=1$。

\section{独立事件}
\begin{definition}{独立事件}{independent event}
	定义$A,B$相互独立(independent)当
	\begin{align}
		\P(AB)=\P(A)\P(B).
	\end{align}
\end{definition}
即,$B$事件发生与否,不影响$A$事件发生的概率:
\[
	\P(A|B)=\P(A)\iff\frac{\P(AB)}{\P(B)}=\frac{\P(A\Omega)}{\P(\Omega)}.
\]
易证,若$A,B$独立,则$A\c,B$独立,$A,B\c$独立,$A\c,B\c$独立。
\begin{definition}{三个事件的独立}{}
	$A,B,C$相互独立等价于:
	\begin{compactenum}
		\item $A,B,C$两两独立;
		\item $\P(ABC)=\P(A)\P(B)\P(C)$
	\end{compactenum}
\end{definition}
需要注意,只知其一不可推出另一个条件。

进而定义可数个事件$A_1,A_2,\ldots$相互独立:任取有限个事件$A_{i_1},\ldots,A_{i_n}$都有
\[
	\P(A_{i_1}\cdots A_{i_n})=\P(A_{i_1})\cdots\P(A_{i_n}).
\]
\begin{definition}{条件独立}{conditional independent}
	若
	\[
		\P(AB|E)=\P(A|E)\P(B|E),
	\]
	则$A,B$关于事件$E$条件独立(conditional independent)。
\end{definition}
$A,B$关于$E$条件独立与$A,B$独立无关,二者之间既不充分也不必要。
\section{Bayes公式}
\begin{theorem}{全概率公式}{total probability formula}
	定义$\Omega$的一个分割$\{B_i\}$,满足:
	\begin{compactenum}
		\item $\sum B_i=\Omega;$
		\item $\forall i\neq j,\enspace B_iB_j=\varnothing;$
		\item $\P(B_i)>0.$
	\end{compactenum}
	则$A$事件的概率
	\begin{align}
		\P(A)=\sum\P(B_i)\P(A|B_i)
	\end{align}
\end{theorem}
\begin{example}{假阳性悖论}{false positive paradox}
	$B=$患病,$A=$阳性,$\P(B)=10^{-4},\P(A|B)=0.99,\P(A|B\c)=10^{-3}$
	\[
		\P(B|A)=\frac{\P(AB)}{\P(A)}=\frac{\P(B)\P(A|B)}{\P(B)\P(A|B)+\P(B\c)\P(A|B\c)}\doteq 9\%.
	\]
\end{example}
\begin{example}{赌徒}{gambler}
	两个人赌博。假设甲的赌本为$i$元,乙的赌本为$n-i$。甲赢的概率为$p$。每赌一局输家给赢家1元,其中一人输光游戏结束。求甲成为最终赢家的概率$Q_i$。

	$A=$甲最终赢,$B=$甲本局赢,由全概率公式
	\[
		\P(A)=\P(B)\P(A|B)+\P(B\c)\P(A|B\c).
	\]
	即
	\[
		Q_i=pQ_{i+1}+(1-p)Q_{i-1},%\implies p(Q_{i+1}-Q_i)=(1-p)(Q_i-Q_{i-1}).
	\]
	且有边界条件:$Q_0=0,Q_n=1$,
	\[
		Q_i=\begin{cases}
			\frac in,&p=1/2\\[2ex]
			\frac{1-\lambda^i}{1-\lambda^n},&p\neq 1/2,\enspace\lambda:=\frac{1-p}p.
		\end{cases}
	\]
\end{example}
\begin{theorem}{Beyas公式}{Beyas formula}
	Beyas公式即
	\begin{align}
		\P(B_i|A)=\frac{\P(B_i)\P(A|B_i)}{\P(A)}=\frac{\P(B_i)\P(A|B_i)}{\sum_j\P(B_j)\P(A|B_j)}
	\end{align}
\end{theorem}
这个公式的重要性不仅在数学意义上,还在于先验概率$\P(B_i)$ v.s.后验概率$\P(B_i|A)$

\begin{remark}
	计算正确的概率、正确计算概率、正确使用概率。
\end{remark}
