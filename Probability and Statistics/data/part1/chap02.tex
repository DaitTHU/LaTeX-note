\chapter{随机变量及其分布}

\begin{definition}{(一维)随机变量}{random variable}
	定义随机变量(random variable)~$X:\Omega\to\RR$是样本空间上的实值函数。有
	\begin{compactitem}
		\item 离散型(discrete):至多可数个取值;
		\item 连续型(continuous):区间型取值(不严格);
		\item 其他
	\end{compactitem}
\end{definition}
$\forall I\subset\RR$可测,记原像集$X\inv(I)\in\cF$,定义\footnote{好看的鱼 v.s.好吃的鱼}
\[
	\P_X(X\in I):=\P(X\inv(I)),\quad\forall I\subset\RR~\text{可测}
\]
一般记$\P_X$为$\P$。
\begin{definition}{累计分布函数}{cumulative distribution function}
	记$X$的累计分布函数(cumulative distribution function, CDF)\index{CDF, 累计分布函数}
	\[
		\CDF(x):=\P(X\leqslant x),\quad\forall x\in\RR
	\]
	则$\P(a<X\leqslant b)\equiv\CDF(b)-\CDF(a)$。
\end{definition}

\begin{theorem}{PDF的性质}{property of PDF}
	\begin{compactenum}
		\item $\CDF(x)$单调递增(不严格单调);
		\item $\lim_{x\to+\infty}\CDF(x)=1,\lim_{x\to-\infty}\CDF(x)=0;$
		\item $\CDF(x)$右连续,不一定左连续。
	\end{compactenum}
\end{theorem}
\begin{remark}~
	\begin{compactenum}
		\item 随机要素来自样本点$\omega$的随机选择;
		\item $X,Y$同样本空间时,一般地,$aX+bY$等$g(X,Y)$也是随机变量;
		\item 随机变量同分布$\iff$ CDF相同;但不代表变量相同。
	\end{compactenum}
\end{remark}

\section{离散分布}

\begin{definition}{离散分布}{discrete distribution}
	离散分布可由分布列(probability distribution)表示概率在所有的可能发生的情况中的分布
	\begin{center}
		\begin{tabular}{ccccc}
			\toprule
			$X$&$x_1$&$x_2$&$\cdots$&$x_n$\\
			\midrule
			$p$&$p_1$&$p_2$&$\cdots$&$p_n$\\
			\bottomrule
		\end{tabular}
	\end{center}
	概率质量函数(probability mass function, PMF)\index{PMF, 概率质量函数}
	\[
		f(x)=\P(X=x),\quad\forall x\in\RR
	\]
\end{definition}
离散分布的CDF为阶梯函数。

\begin{definition}{期望和方差}{expectation and variance}
	期望(expectation)即均值
	\begin{align}
		\E(X):=\sum x_ip_i,
	\end{align}
	期望存在要求级数绝对收敛。

	方差(variance)
	\begin{align}
		\Var(X):=\sum\bigkh{x_i-\E(X)}^2p_i\equiv\E(X^2)-\E(X)^2.
	\end{align}
\end{definition}
\begin{example}{}{}
	随机变量$X$的函数$g(X)$的期望
	\[
		\E(g(X))=\sum g(x_i)p_i.
	\]
\end{example}
\begin{definition}{Bernoulli分布}{Bernoulli distribution}
	Bernoulli分布也称0-1分布,$p$为成功概率,记作$X\sim\Bern(p)$,其分布列为
	\begin{center}
		\begin{tabular}{ccc}
			\toprule
			$X$&0&1\\
			\midrule
			$p$&$1-p$&$p$\\
			\bottomrule
		\end{tabular}
	\end{center}
\end{definition}
\begin{definition}{二项分布}{binominal distribution}
	$n$次独立Bernoulli试验的成功次数$X$服从二项分布(binominal distribution),记作$X\sim\Bern(n,p)$,其分布列为:
	\[
		\P(X=k)=\binom nk p^k(1-p)^{n-k},\quad k=0,\ldots,n
	\]
	$\E(X)=np,\quad\Var(X)=np(1-p).$
\end{definition}
\begin{definition}{Poisson分布}{Poisson distribution}
	一段时间或一定空间内出现的小概率事件次数$X$服从Poisson分布,记作$X\sim\Pois(\lambda)$,其分布列为:
	\[
		\P(X=k)=\frac{\lambda^k}{k!}\e{-\lambda},\quad k=0,1,2,\ldots
	\]
	$\E(X)=\lambda,\quad\Var(X)=\lambda.$
\end{definition}
\begin{theorem}{二项分布趋于Poisson分布}{}
	$n\to\infty$时,二项分布$\Bern(n,\lambda/n)$趋于Poisson分布$\Pois(\lambda)$
	\begin{align*}
		\binom nk p^k&(1-p)^{n-k}=\frac{n!}{k!(n-k)!}\kh{\frac\lambda{n}}^k\kh{1-\frac\lambda{n}}^{n-k}\\
		&=\frac{\lambda^k}{k!}\kh{1-\frac\lambda{n}}^n\cancel{\frac{n!}{n^k(n-k)!}\kh{1-\frac\lambda{n}}^{-k}}\to\frac{\lambda^k}{k!}\e{-\lambda}.
	\end{align*}
	用$\Pois(np)$近似$\Bern(n,p)$的误差最多为$\min(p,np^2)$,证略。
\end{theorem}

若试验不独立,但满足弱相依条件下,Poisson分布仍为较好近似。
\begin{example}{弱相依条件举例:配对问题}{weak dependence condition}
	在\exmref{exm:out of order} 中,尽管$A_i$和$A_j$并不独立,但弱相依
	\[
		\P(A_i)=\frac1n\doteq\P(A_i|A_j)=\frac1{n-1}.
	\]
	记$X=$拿到自己帽子的人数,当$n\to\infty$时,$X\sim\Pois(1)$
	\[
		\P(X=k)=\frac{\e{-1}}{k!},\quad k=0,1,2,\ldots
	\]
	下面用常规做法检验。
	\tcblower
	$E=$指定的$k$个人拿到自己的帽子,
	
	$F=$余下$n-k$个人都未拿到自己的帽子
	\[
		\P(EF)=\P(E)\P(F|E)=\frac{(n-k)!}{n!}f(n-k),
	\]
	其中$f(n)$表示\exmref{exm:out of order} 中的结果,故
	\[
		\P(X=k)=\binom nk\P(EF)=\frac1{k!}f(n-k)\to\frac{\e{-1}}{k!}.
	\]
\end{example}
\section{连续分布}
\begin{definition}{连续分布}{continuous distribution}
	若存在$f(x)\geqslant 0$,使得$\forall I\subset\RR$可测,都有
	\[
		\P(X\in I)=\int_If(x)\d x
	\]
	则称$X$为连续型随机变量,服从连续分布,$f(x)$为概率密度函数(probability density function, PDF)。\index{PDF, 概率密度函数}
\end{definition}
\begin{theorem}{连续分布性质}{property of continuous distribution}
	\begin{itemize}
		\item $\forall x\in\RR,\enspace\P(X=x)\equiv 0;$
		\item 归一性:
			\begin{align}
				\int\iti f(x)\d x\equiv 1;
			\end{align}
		\item 期望和方差
			\begin{align}
				\E(X)&=\int\iti xf(x)\d x,\\
				\Var(X)&=\int\iti(x-\E(X))^2f(x)\d x.
			\end{align}
			若期望存在,则要求上积分式\textbf{绝对收敛}。
		\item 连续分布的CDF $\CDF(x)$连续且可导,且
			\begin{align}
				\CDF'(x)=f(x).
			\end{align}
			若$\CDF(x)$严格递增,则$\CDF^{-1}(y)$存在;但若其不严格递增,也可well define
			\begin{align}
				\CDF^{-1}(y):=\inf\set x{\CDF(x)\geqslant y}.
			\end{align}
	\end{itemize}
\end{theorem}
\iffalse
\begin{example}{常见连续分布}{}
	\begin{compactenum}
		\item 均匀分布$f(x;a,b)=\frac1{b-a},a\leqslant x\leqslant b;$
		\item 指数分布$f(x;\lambda)=\lambda\e{-\lambda x},x\geqslant 0;$
		\item 正态分布$f(x;\mu,\sigma^2)=\frac1{\sqrt{2\pi}\sigma}\e{-(x-\mu)^2/2\sigma^2},-\infty<x<\infty;$
		\item 伽玛分布$f(x;\alpha,\lambda)=\frac{\lambda^\alpha}{\Gamma(\alpha)}x^{\alpha-1}\e{-\lambda x}.x\geqslant 0;$
		\item 卡方分布$f(x;n)=\frac{x^{n/2-1}}{2^{n/2}\Gamma(n/2)}\e{-x/2},x\geqslant 0;$
	\end{compactenum}
\end{example}
\fi
\begin{definition}{均匀分布}{uniform distribution}
	均匀分布(uniform distribution)记作$X\sim\Unif(a,b)$,%随机数$\Unif(0,1)$
	\begin{align}
		f(x)=\begin{cases}
			\frac1{b-a},&x\in(a,b)\\
			0,&\text{elsewhere}
		\end{cases}
	\end{align}
	$\E(X)=\frac{a+b}2,\quad\Var(X)=\frac{(b-a)^2}{12}.$
	%$\CDF(x)=\frac{x-a}{b-a},\quad\E(X)=\frac{a+b}2,\quad\Var(X)=\frac{(b-a)^2}{12};$
\end{definition}
\begin{definition}{正态分布}{normal distribution}
	正态分布(normal distribution)记作$X\sim\Norm(\mu,\sigma^2)$,%标准正态$\Norm(0,1)$
	\begin{align}
		f(x)=\frac1{\sqrt{2\pi}\sigma}\exp\biggfkh{-\frac{(x-\mu)^2}{2\sigma^2}},\quad x\in\RR
	\end{align}
	$\CDF(x)=:\Phi(x),\quad\E(X)=\mu,\quad\Var(X)=\sigma^2;$
\end{definition}
\begin{definition}{指数分布}{exponential distribution}
	指数分布(exponential distribution)通常刻画寿命、等待时间,记作$X\sim\Expo(\lambda)$,
	\begin{align}
		f(x)=\begin{cases}
			\lambda\e{-\lambda x},&x>0\\
			0,&x\leqslant 0
		\end{cases}
	\end{align}
	定义平均寿命
	\[
		\beta:=\E(X)=\frac1\lambda,\quad\Var(X)=\beta^2.
	\]
	也有教材因此记作$\Expo(\beta)$。
\end{definition}

指数分布的尾概率
\[
	\P(X>x)=\e{-\lambda x},%\equiv 1-\CDF(x)
\]
与Poisson分布有关。

\begin{example}{指数分布的导出}{}
	失效率
	\begin{align*}
		\P(x<X<x+\d x|X>x)&=\frac{\P(x<X<x+\d x)}{\P(X>x)}\\
		&=\frac{\CDF(x+\d x)-\CDF(x)}{1-\CDF(x)}\doteq\frac{\CDF'(x)}{1-\CDF(x)}\d x
	\end{align*}
	令失效率为$\lambda(x)\d x$
	\[
		\frac{\CDF'(x)}{1-\CDF(x)}=\lambda(x),\implies\CDF(x)=1-\exp\biggfkh{-\int_0^x\lambda(t)\d t}.
	\]
	若假设无老化:$\lambda(t)\equiv\lambda$,则分布为指数分布:
	\begin{align}
		\CDF(x)=1-\e{-\lambda x},\quad x>0,
	\end{align}

	这体现出指数分布的无记忆性:
	\[
		\P(X>t+\tau|X>\tau)=\frac{\e{-\lambda(t+\tau)}}{\e{-\lambda\tau}}=\e{-\lambda t}=\P(X>t)
	\]
	与$\tau$无关。

	\tcblower
	
	改进$\lambda(x)=\alpha x^{\alpha+1}/\beta^\alpha$,得到Weibull分布
	\begin{align*}
		\CDF(x)=1-\e{-(x/\beta)^\alpha},\quad x>0.
	\end{align*}
\end{example}

\section{随机变量的函数}

设$Y=g(X)$,$X$离散可推出$Y$离散。

\begin{theorem}{连续型随机变量的函数}{}
	若$g$处处可导且严格单调,则$Y=g(X)$的PDF为
	\begin{align}
		f_Y(y)=f_X\bigkh{g^{-1}(y)}\abs{\fkh{g^{-1}(y)}'}.
	\end{align}
	其本质是
	\begin{align}
		\CDF_Y(y)=\CDF_X\bigkh{g^{-1}(y)}.
	\end{align}
\end{theorem}
\begin{example}{生成随机变量}{}
	服从CDF $\CDF(y)$的随机变量$Y$可由随机数$X\sim\Unif(0,1)$生成:
	\begin{align}
		Y=\CDF\inv(X).
	\end{align}
\end{example}
\section*{Review}
\begin{compactenum}
	\item PMF/PDF, CDF
	\item 期望$\mu$、标准差$\sigma$;标准化$\frac{X-\mu}\sigma$
	\item 参数的意义:位置、尺度、形状
	\item $Y=g(X)$
\end{compactenum}