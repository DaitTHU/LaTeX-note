\chapter{典型分布}

\section{离散分布}

\paragraph{二项分布}

$\Bino(n,p)$:进行$n$次独立\Bern 试验中,共成功了$X$次
\begin{equation}
	\P(X=k)=\binom nk p^k(1-p)^{n-k},\quad k=0,\ldots,n
\end{equation}

\paragraph{Pascal分布}

$\mathrm{Pa}(r,p)$:进行$X$次独立\Bern 试验,直到成功了$r$次
\begin{equation}
	\P(X=k)=\binom{k-1}{r-1}p^r(1-p)^{k-r},\quad k=r,r+1,\ldots
\end{equation}

\paragraph{几何分布}

$\mathrm{Ge}(p)=\mathrm{Pa}(1,p)$。

\paragraph{负二项分布}

$\mathrm{NB}(r,p)$:进行独立\Bern 试验,直到成功了$r$次。之前共失败了$X$次
\footnote{
	显然,Pascal分布和负二项分布只是平移的区别:$\P_{\mathrm{Pa}}(X=k+r)=\P_{\mathrm{NB}}(X=k)$。负二项分布名称中负的来源是
	\begin{equation}
		\binom{k+r-1}{r-1}=(-)^k\binom{-r}{k}.
	\end{equation}
}
\begin{equation}
	\P(X=k)=\binom{k+r-1}{r-1}p^r(1-p)^k,\quad k=0,1,\ldots
\end{equation}

\paragraph{超几何分布}

$\mathrm H(n,N,M)$:$N$个球中有$M$个有标记,从中不放回地随机抽取$n$个,其中$X$个球有标记
\begin{equation}
	\P(X=k)=\frac{\binom Mk\binom{N-M}{n-k}}{\binom N n},\quad k=0,1,\ldots,\min(M,n).
\end{equation}

\paragraph{负超几何分布}

$\mathrm{NH}(r,N,M)$:$N$个球中有$M$个有标记,从中不放回地随机抽取$X$个,
直到取到$r$个有标记的球
\begin{equation}
	\P(X=k)=\frac{\binom{k-1}{r-1}\binom{N-k}{M-r}}{\binom N M},\quad k=r,\ldots,r+n-M.
\end{equation}

\paragraph{Poisson分布}

$\Pois(\lambda)$:$\Bino(n,\lambda/n)$当$n\to\infty$的极限
\begin{equation}
	\P(X=k)=\frac{\lambda^k}{k!}\e{-\lambda},\quad k=0,1,2,\ldots
\end{equation}

\subsection{期望、方差}

\begin{center}
	\begin{tabular}{ccc}
		\toprule
		Distribution&$\E(X)$&$\Var(X)$\\
		\midrule
		$\Bino(n,p)$&$np$&$np(1-p)$\\[1ex]
		$\mathrm{Pa}(r,p)$&$\frac rp$&$\frac rp\kh{\frac1p-1}$\\[2ex]
		% $\mathrm{Ge}(p)$&$\frac1p$&$\frac1p\kh{\frac1p-1}$\\
		$\mathrm H(n,N,M)$&$\frac{nM}N$&$\frac{n(N-n)M(N-M)}{N^2(N-1)}$\\[2ex]
		$\mathrm{NH}(r,N,M)$&$\frac{r(N+1)}{M+1}$&$\frac{r(N+1)(N-M)(M-r+1)}{(M+1)^2(M+2)}$\\
		$\Pois(\lambda)$&$\lambda$&$\lambda$\\
		\bottomrule
	\end{tabular}
\end{center}

\paragraph{二项分布}

$X\sim\Bino(n,p)$可以视作$X_1,\ldots,X_n$ iid $\sim\Bino(p)$的和:$X=X_1+\cdots+X_n$,故
期望和方差
\begin{subequations}
	\begin{align}
		\E(X)&=n\E(X_i)=np,\\
		\Var(X)&=n\Var(X_i)=np(1-p).
	\end{align}
\end{subequations}
众数
\[
	\frac{\P(X=k)}{\P(X=k-1)}=\frac{\binom nkp}{\binom n{k-1}(1-p)}=\frac{n-k+1}{k}\frac p{1-p}>1,
	\implies k<(n+1)p.
\]
故
\begin{equation}
	\text{Mode}=\begin{cases}
		n,&p>\frac{n}{n+1}\\
		(n+1)p-1\text{~or~}(n+1)p,&(n+1)p\in\NN\\
		\floor{(n+1)p},&(n+1)p\notin\NN
	\end{cases}
\end{equation}

% \paragraph{Pascal分布}

% 众数
% \[
% 	\frac{\P(X=k)}{\P(X=k-1)}=\frac{k+r-1}{k}(1-p)>1,
% 	\implies k<\frac{(r-1)(1-p)}{p}.
% \]
% 故
% \begin{align*}
% 	\text{Mode}=\begin{cases}
% 		r,&p>\frac{r-1}{r+1}\\
% 		(n+1)p-1\text{~or~}(n+1)p,&(n+1)p\in\NN\\
% 		\floor{(n+1)p},&(n+1)p\notin\NN
% 	\end{cases}
% \end{align*}

% \paragraph{几何分布}

% \paragraph{超几何分布}

% \paragraph{Poisson分布}

\subsection{PGF, MGF和CF}

\begin{center}
	\begin{tabular}{ccc}
		\toprule
		Distribution&$\MGF(X)$&$\CF(X)$\\
		\midrule
		$\Bino(n,p)$&$(q+p\e t)^n$&$(q+p\e{\i t})^n$\\
		$\mathrm{NB}(r,p)$&$\kh{\frac p{q+p\e{t}}}^r$&$\kh{\frac p{q+p\e{\i t}}}^r$\\
		$\mathrm H(n,N,M)$&-&-\\
		$\Pois(\lambda)$&$\e{\lambda(\e t-1)}$&$\e{\lambda(\e{\i t}-1)}$\\
		\bottomrule
	\end{tabular}
\end{center}

\section{连续分布}

\paragraph{均匀分布}

$\Unif(a,b)$
\begin{equation}
	f(x)=\frac1{b-a},\quad x\in(a,b).
\end{equation}

\paragraph{正态分布}

$\Norm(\mu,\sigma^2)$
\begin{equation}
	f(x)=\frac1{\sqrt{2\pi}\sigma}\exp\fkh{-\frac{(x-\mu)^2}{2\sigma^2}},\quad x\in\RR.
\end{equation}

\paragraph{指数分布}

$\Expo(\lambda)$
\begin{equation}
	f(x)=\lambda\e{-\lambda x},\quad x>0.
\end{equation}

\paragraph{Gamma分布}

$\mathrm{Ga}(\alpha,\lambda)$
\begin{equation}
	f(x)=\frac{\lambda^\alpha}{\Gamma(\alpha)}x^{\alpha-1}\e{-\lambda x},\quad x>0.
\end{equation}
其中$\alpha,\lambda>0$,Gamma函数
\[
	\Gamma(\alpha)=\int\zti x^{\alpha-1}\e{-x}\d x,
\]
特别地,$\Gamma(x+1)=x\Gamma(x),\,\Gamma(1)=1,\,\Gamma(1/2)=\sqrt\pi,\,\Gamma(n)=(n-1)!$,且
\begin{gather*}
	\mathrm{Ga}(1,\lambda)=\Expo(\lambda),\,\mathrm{Ga}(n/2,1/2)=\chi^2(n)\\
	f(x;\alpha,\lambda)\d x=f(\lambda x;\alpha,1)\d\lambda x
\end{gather*}

\paragraph{卡方分布}

$\chi^2(n)$
\begin{equation}
	f(x)=\frac{x^{n/2-1}}{2^{n/2}\Gamma(n/2)}\e{-x/2},\quad x>0.
\end{equation}
其中$n\in\mathbb N_+$,特别地,$\chi^2(1)$
\[
	f(x)=\frac1{\sqrt{2\pi x}}\e{-x/2},\quad\lim_{x\to 0^+}f(x)=+\infty.
\]

\paragraph{Beta分布}

$\mathrm{Be}(a,b)$
\begin{equation}
	f(x)=\frac1{\mathrm B(a,b)}x^{a-1}(1-x)^{b-1},\quad x\in(0,1).
\end{equation}
其中$a,b>0$,Beta函数
\[
	\mathrm B(a,b)=\int_0^1x^{a-1}(1-x)^{b-1}\d x\equiv\frac{\Gamma(a)\Gamma(b)}{\Gamma(a+b)},
\]

\paragraph{Cauchy分布}

\begin{equation}
	f(x)=\frac1\pi\frac1{1+x^2},\quad x\in\RR
\end{equation}

\paragraph{Landau分布}

略

\subsection{众数、期望、方差}

\begin{center}
	\begin{tabular}{cccc}
		\toprule
		Distribution&Mode&$\E(X)$&$\Var(X)$\\
		\midrule
		$\Unif(a,b)$&-&$\frac{a+b}2$&$\frac{(b-a)^2}{12}$\\
		$\Norm(\mu,\sigma^2)$&$\mu$&$\mu$&$\sigma^2$\\
		$\Expo(\lambda)$&0&$\frac1\lambda$&$\frac1\lambda$\\
		$\mathrm{Ga}(\alpha,\lambda)$&-&&\\
		\bottomrule
	\end{tabular}
\end{center}

\section{简单随机抽样}
\label{sec:simple random sampling}

总体$Y_1,\ldots,Y_N$均值和方差为$\mu,\sigma^2$,简单随机样本为$X_1,\ldots,X_n$。

$A:=$样本中含有$Y_j$
\[
	\P(X_i=Y_j)=\P(X_i=Y_j|A)\P(A)+0=\frac1n\cdot\frac{\binom{N-1}{n-1}}{\binom{N}{n}}=\frac1N.
\]
故
\begin{subequations}
	\begin{align}
		\E(X_i)&=\sum_{j=1}^N\P(X_i=Y_j)Y_j=\frac1N\sum_{j=1}^NY_j=\mu;\\
		\Var(X_i)&=\sum_{j=1}^N\P(X_i=Y_j)\bigkh{Y_j-\E(Y_j)}^2=\sigma^2.\\
		\E(\avg X)&=\frac1n\sum_{i=1}^n\E(X_i)=\mu.
	\end{align}
\end{subequations}
$B:=$样本中同时含有$Y_k,Y_\ell(k\neq\ell)$
\[
	\P(X_i=Y_k,X_j=Y_\ell)=\frac1{n(n-1)}\cdot\frac{\binom{N-2}{n-2}}{\binom{N}{n}}=\frac1{N(N-1)}.
\]
故
\begin{align*}
	\E(X_iX_j)&=\frac1{N(N-1)}\sum_{k=1}^N\sum_{\ell\neq k}Y_kY_\ell=\frac1{N(N-1)}\sum_{k=1}^NY_k(N\mu-Y_k)\\
	&=\frac{N^2\mu^2-N(\sigma^2+\mu^2)}{N(N-1)}=\mu^2-\frac{\sigma^2}{N-1}.\\
	\Cov(X_i,X_j)&=\E(X_iX_j)-\E(X_i)\E(X_j)=-\frac{\sigma^2}{N-1}.
\end{align*}
故
\begin{align*}
	\Var(\avg X)&=\frac1{n^2}\kh{\sum_{i=1}^n\Var(X_i)+\sum_{i\neq j}\Cov(X_i,X_j)}\\
	&=\frac1{n^2}\kh{n\sigma^2-n(n-1)\cdot\frac{\sigma^2}{N-1}}=\frac{\sigma^2}n\frac{N-n}{N-1}.\\
	\E(S^2)&=\frac n{n-1}\kh{\frac1n\sum_{i=1}^n\E(X_i^2)-\E(\avg X^2)}\\
	&=\frac n{n-1}\fkh{\sigma^2+\mu^2-\kh{\frac{\sigma^2}n\frac{N-n}{N-1}+\mu^2}}=\frac{N\sigma^2}{N-1}.
\end{align*}
