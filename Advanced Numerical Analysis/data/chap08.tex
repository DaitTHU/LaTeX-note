\chapter{矩阵特征值问题的数值方法}

% \section{矩阵特征值问题}

% \begin{definition}
%     {矩阵特征值问题}{}
%     给定$n\times n$矩阵$A$,寻找$\lambda\in\CC$和非零向量$x\in\CC^n$,使得
%     \[Ax=\lambda x.\]
%     $\lambda$称为$A$的特征值,$x$称为对应的特征向量。
% \end{definition}

% \begin{theorem}
%     {特征值性质}{}
%     \begin{enumerate}
%         \item $A$的特征值$\lambda$满足$\det(A-\lambda I)=0$,其中$I$是单位矩阵。
%         \item $A$的特征值的代数重数$\le n$。
%         \item $A$的特征值的几何重数$\le$代数重数。
%         \item $A$的特征值的代数重数等于其特征向量的个数。
%     \end{enumerate}
% \end{theorem}

% \begin{definition}
%     {谱半径}{}
%     $A$的谱半径$\rho(A)=\max\abs{\lambda}$,其中$\lambda$是$A$的特征值。
% \end{definition}

% \section{幂法}

% \begin{definition}
%     {幂法}{}
%     给定$n\times n$矩阵$A$,初始向量$x^{(0)}$,迭代
%     \[x^{(k+1)}=\frac{Ax^{(k)}}{\norm{Ax^{(k)}}},\]
%     直到收敛。
% \end{definition}

% \begin{theorem}
%     {幂法收敛}{}
%     若$A$有特征值$\lambda_1>\lambda_2\ge\cdots\ge\lambda_n$,则幂法收敛到$\lambda_1$对应的特征向量。
% \end{theorem}

% \begin{example}
%     {幂法}{}
%     计算矩阵
%     \[A=\begin{pmatrix}
%         1&2&3\\
%         4&5&6\\
%         7&8&9
%     \end{pmatrix}\]
%     的最大特征值。
% \end{example}

% \section{反幂法}

% \begin{definition}
%     {反幂法}{}
%     给定$n\times n$矩阵$A$,初始向量$x^{(0)}$,迭代
%     \[x^{(k+1)}=\frac{(A-\sigma I)^{-1}x^{(k)}}{\norm{(A-\sigma I)^{-1}x^{(k)}}},\]
%     直到收敛。
% \end{definition}

% \begin{theorem}
%     {反幂法收敛}{}
%     若$A$有特征值$\lambda_1>\lambda_2\ge\cdots\ge\lambda_n$,则反幂法收敛到$\lambda_1$对应的特征向量。
% \end{theorem}

% \begin{example}
%     {反幂法}{}
%     计算矩阵
%     \[A=\begin{pmatrix}
%         1&2&3\\
%         4&5&6\\
%         7&8&9
%     \end{pmatrix}\]
%     的最大特征值。
% \end{example}

% \section{QR方法}

% \begin{definition}
%     {QR分解}{}
%     给定$n\times n$矩阵$A$,QR分解为$A=QR$,其中$Q$是正交矩阵,$R$是上三角矩阵。
% \end{definition}

% \begin{theorem}
%     {QR方法}{}
%     给定$n\times n$矩阵$A$,QR分解为$A=QR$,迭代
%     \[A^{(k+1)}=R^{(k)}Q^{(k)},\]
%     直到收敛。
% \end{theorem}

% \begin{example}
%     {QR方法}{}
%     计算矩阵
%     \[A=\begin{pmatrix}
%         1&2&3\\
%         4&5&6\\
%         7&8&9
%     \end{pmatrix}\]
%     的特征值。
% \end{example}

% \section{广义幂法}

% \begin{definition}
%     {广义幂法}{}
%     给定$n\times n$矩阵$A$和$m\times m$矩阵$B$,初始向量$x^{(0)}$,迭代
%     \[x^{(k+1)}=\frac{B^{-1}Ax^{(k)}}{\norm{B^{-1}Ax^{(k)}}},\]
%     直到收敛。
% \end{definition}

