\chapter{数值积分}
\label{chap:numerical integration}

给定函数$f\in\cont[a,b]$和权函数$\rho(x)$,计算积分
\begin{equation}
    I(f)\equiv\int_a^bf(x)\rho(x)\d x,
\end{equation}
但很多情况下,我们只知道关于$f$的部分信息,不能用Newton-Leibniz公式。因此我们希望构造一种不依赖于$f$具体表达式的近似积分方法。

\begin{definition}
    {数值积分}{numerical integration}
    给定系列积分节点$x_0,\ldots,x_n$,数值积分$\tilde I(f)$一般具有如下形式:
    \begin{equation}
        \tilde I(f)=\sum_{k=0}^nA_kf(x_k).
    \end{equation}
    其中$A_k$称为求积系数。积分方法$\tilde I$的误差为
    \begin{equation}
        E(f):=I(f)-\tilde I(f).
    \end{equation}
\end{definition}

给定一种积分方法$\tilde I$,如何评价其优劣?

\begin{definition}
    {代数精确度}{}
    给定积分法$\tilde I$,若$\forall f\in\poly_n$均有$E(f)=0$;但$\exists f\in\poly_{n+1}$使得$E(f)\neq 0$,则$\tilde I$的代数精确度为$n$。
\end{definition}

\begin{definition}
    {插值型求积公式}{interpolation integration}
    给定系列插值节点%(不一定全在$[a,b]$内)
    $x_0<x_1<\cdots<x_n$,
    由式\eqref{eqn:Lagrange polynomial},函数$f$的$n$次插值多项式为
    \[
        p_n(x)=\sum_{i=0}^nf(x_i)\ell_i(x)\approx f(x),
    \]
    选定
    \begin{equation}
        I_n(f)\equiv I(p_n)=\sum_{i=0}^nI(\ell_i)f(x_i)\equiv\sum_{i=0}^nA_i^{(n)}f(x_i),
    \end{equation}
    其中$A_k^{(n)}$与$f$无关,称为求积系数。
\end{definition}

\section{Newton-Cotes公式}
\label{sec:Newton-Cotes formula}

考虑$\rho\equiv 1$的情形。

\subsection{闭型Newton-Cotes公式}

\begin{theorem}
    {(闭型) Newton-Cotes公式}{Newton-Cotes formula}
    取等距节点$a=x_0<x_1<\cdots<x_n=b$,节点间距$h:=(b-a)/n$,有
    \begin{equation}
        I_n(f)=(b-a)\sum_{k=0}^nC_k^{(n)}f(x_k),
    \end{equation}
    其中$C_k^{(n)}$称为求积Cotes系数,由式\eqref{eqn:Lagrange base}
    \begin{equation}
        C_k^{(n)}=\frac{(-1)^{n-k}}{k!(n-k)!}\frac1n\int_0^n\prod_{j\neq k}(t-j)\d t.
    \end{equation}
\end{theorem}

\begin{remark}
    Cotes系数不但与$f$无关,也与积分区间$[a,b]$无关。且$C_k^{(n)}=C_{n-k}^{(n)}$。
\end{remark}

\begin{example}
    {前几个Cotes系数}{}
    \begin{center}
        \begin{tabular}{cccc}
            \toprule
            $n$&$C_k^{(n)}$\\%\quad k=0,1,\ldots,n
            \midrule
            1&1\quad1&/&2\\[2ex]
            2&1\quad4\quad1&/&6\\[2ex]
            3&1\quad3\quad3\quad1&/&8\\[2ex]
            4&7\quad32\quad12\quad12\quad7&/&90\\
            \bottomrule
        \end{tabular}
    \end{center}
\end{example}

\begin{example}
    {$n=1$:梯形公式}{trapezoidal integration}
    $n=1$时,只有$a,b$两个求积节点,得到梯形公式(trapezoidal)
    \begin{equation}
        \label{eqn:trapezoidal integration}
        I_1(f)=\frac{b-a}2[f(a)+f(b)].
    \end{equation}
    % 余项$R_1(x)=f[a,b,x](x-a)(x-b)$。
    若$f\in\cont^2[a,b]$,则$\exists\xi\in[a,b]$,积分误差
    \begin{equation}
        \label{eqn:trapezoidal error}
        E_1(f)=-\frac{h^3}{12}f''(\xi).
    \end{equation}
    梯形公式的代数精确度为1。
\end{example}

\begin{example}
    {$n=2$:Simpson公式}{Simpson integration}
    $n=2$时,有$a,x_1=(a+b)/2,b$三个求积节点,得到Simpson公式:
    \begin{equation}
        \label{eqn:Simpson integration}
        I_2(f)=\frac{b-a}6\biggfkh{f(a)+4f\biggkh{\frac{a+b}2}+f(b)}.
    \end{equation}
    % 余项$R_2(x)=f[a,x_1,b,x](x-a)(x-x_1)(x-b)$。
    若$f\in\cont^4[a,b]$,则$\exists\xi\in[a,b]$,积分误差
    \begin{equation}
        \label{eqn:Simpson error}
        E_2(f)=-\frac{h^5}{90}f^{(4)}(\xi).
    \end{equation}
    Simpson公式的代数精确度为3。
\end{example}

\begin{remark}
    $n=3$时称为Simpson 3/8公式
    \[
        E_3(f)=-\frac{3h^5}{80}f^{(4)}(\xi),
    \]
    $n=4$时称为Boole公式或Cotes公式
    \[
        E_4(f)=-\frac{8h^7}{945}f^{(6)}(\xi).
    \]
\end{remark}

\begin{theorem}
    {Newton-Cotes公式的代数精确度}{}
    当$n$为奇数时,代数精确度为$n+1$;当$n$为偶数时,代数精确度为$n$。
\end{theorem}

\subsection{开型Newton-Cotes公式}

\begin{theorem}
    {开型Newton-Cotes公式}{}
    取等距节点$a=x_{-1}<x_0<\ldots<x_n<x_{n+1}=b$,以$x_0,\ldots,x_n$为插值节点。
\end{theorem}

\begin{example}
    {$n=0$:中点公式}{midpoint integration}
    $n=0$的开型Newton-Cotes公式称为中点公式:
    \begin{equation}
        \label{eqn:midpoint integration}
        I_0=(b-a)f\biggkh{\frac{a+b}2}.
    \end{equation}
    若$f\in\cont^2[a,b]$,则$\exists\xi\in[a,b]$,积分误差
    \begin{equation}
        \label{eqn:midpoint error}
        E_0(f)=\frac{h^3}3f''(\xi).
    \end{equation}
\end{example}

\begin{remark}
    一般来说,闭型Newton-Cotes公式的结果比相应的开型Newton-Cotes公式的结果要好。
\end{remark}

\subsection{积分法的一致稳定性}

由于积分节点$f(x_i)$可能存在误差,需要考虑随着渐进参数$n$的增大,积分法的稳定性。

\begin{definition}
    {一致稳定}{}
    对于线性积分方法$\tilde I_n$,若$\forall n\in\NN$,$\exists M>0$使得
    \begin{equation}
        \sum_{k=0}^n\abs{A_k^{(n)}}<M,
    \end{equation}
    则称$\tilde I_n$是一致稳定的。
\end{definition}

\begin{corollary}
    若$\norm{f-g}_\infty\leq\delta$,且$\tilde I_n$一致稳定,则
    \[
        \abs{\tilde I_n(f)-\tilde I_n(g)}=\abs{\tilde I_n(f-g)}\leq\sum_{k=0}^n\abs{A_k^{(n)}}\norm{f-g}_\infty\leq M\delta.
    \]
\end{corollary}

\begin{theorem}
    {一致稳定的判定}{}
    若$\forall n,k$均有$A_k^{(n)}>0$且$\tilde I_n$的代数精确度至少为0,则$\tilde I_n$一致稳定。
\end{theorem}

\begin{remark}
    $n\geq 8$时,Newton-Catos积分法存在$A_k^{(n)}<0$,故不是一致稳定的。
\end{remark}

\section{复合求积公式}
\label{sec:composite quadrature rule}

对于一些函数$f$来说,低阶Newton-Catos公式很不准确,而高阶又存在数值不稳定问题。为此需要新的求积方法,一种思路是利用积分运算关于测度的可加性,将积分区间分成若干小的区间进行分片求积。

\begin{definition}
    {复合求积公式}{}
    将积分区间$[a,b]$划分成若干子区间,再在每个子区间上采用低阶Newton-Catos公式。
\end{definition}

\begin{example}
    {复合梯形公式}{composite trapezoidal formula}
    取节点$a=x_0<x_1<\ldots<x_n=b$,子区间为$[x_k,x_{k+1}]$,区间间隔$h_k=x_{k+1}-x_k$,对每个子区间套用梯形积分公式\eqref{eqn:trapezoidal integration},得到复合梯形公式为:
    \begin{equation}
        \label{eqn:composite trapezoidal integration}
        \begin{aligned}
            T_n(f)&=\sum_{k=0}^{n-1}\frac{h_k}2[f(x_k)+f(x_{k+1})]\\
            &=\frac{h_0}2f(x_0)+\frac{h_0+h_1}2f(x_1)+\cdots+\frac{h_{n-2}+h_{n-1}}2f(x_{n-1})+\frac{h_{n-1}}2f(x_n).
        \end{aligned}
    \end{equation}
    % 是一致稳定的。
    由式\eqref{eqn:trapezoidal error}和中值定理知,$\exists\xi_k\in[x_k,x_{k+1}],\eta\in[a,b]$使得
    \begin{equation}
        E_n(f)=-\frac1{12}\sum_{k=0}^{n-1}h_k^3f''(\xi_k)=-\frac1{12}\sum_{k=0}^{n-1}h_k^3f''(\eta),
    \end{equation}
    当等距均分时,$h_0=h_1=\cdots=h_{n-1}=h=(b-a)/n$,有
    \begin{equation}
        E_n(f)=-\frac{b-a}{12}h^2f''(\eta).
    \end{equation}
\end{example}

\begin{example}
    {复合中点公式}{composite midpoint formula}
    对每个子区间套用中点积分公式\eqref{eqn:midpoint integration},得到复合中点公式为:
    \begin{equation}
        H_n(f)=\sum_{k=0}^{n-1}h_kf(x_{k+1/2}),\quad x_{k+1/2}:=\frac{x_k+x_{k+1}}2.
    \end{equation}
\end{example}

\begin{corollary}
    考虑每个子区间上增加一个节点$x_{k+1/2}$,得到递推公式:
    \begin{equation}
        T_{2n}(f)=\frac{T_n(f)+H_n(f)}2.
    \end{equation}
\end{corollary}

\begin{example}
    {复合Simpson公式}{composite Simpson formula}
    对每个子区间套用Simpson积分公式\eqref{eqn:Simpson integration},得到复合Simpson公式为:
    \begin{equation}
        S_{2n}(f)=\sum_{k=0}^{n-1}\frac{h_k}6[f(x_k)+4f(x_{k+1/2})+f(x_{k+1})].
    \end{equation}
    若最大区间长度$h:=\max_kh_k$,则$\exists\eta\in[a,b]$使得
    \begin{equation}
        \abs{E_{2n}(f)}\leq\frac{b-a}{2880}h^4f^{(4)}(\eta).
    \end{equation}
\end{example}

\begin{remark}
    复合梯形公式、复合中点公式、复合Simpson公式都是一致稳定的。
\end{remark}

\begin{corollary}
    递推关系:
    \begin{equation}
        S_{2n}(f)=\frac{T_n(f)+2H_n(f)}3=\frac{4T_{2n}(f)-T_n(f)}3.
    \end{equation}
\end{corollary}

\section{Romberg求积方法}
\label{sec:Romberg's method}

\subsection{Euler-Maclaurin求积公式}

\begin{definition}
    {Bernuoulli多项式}{}
    Bernoulli多项式递归定义:$B_0(x)=1$,
    \begin{equation}
        B_k(x)=k\int_0^xB_{k-1}(t)\d t-k\int_0^1\int_0^xB_{k-1}(t)\d t\nd x
    \end{equation}
    Bernoulli数$B_k:=B_k(0)$。
\end{definition}

\begin{example}
    {前几个Bernoulli多项式}{}
    \begin{equation*}
        \begin{aligned}
            B_1(x)&=x-\frac12,&B_2(x)&=x^2-x+\frac16,\\
            B_3(x)&=x^3-\frac32x^2+\frac12x,&B_4(x)&=x^4-2x^3+x^2-\frac1{30},\\
            B_5(x)&=x^5-\frac52x^4+\frac53x^3-\frac16x,&B_6(x)&=x^6-3x^5+\frac52x^4-\frac12x^2+\frac1{42}.
        \end{aligned}
    \end{equation*}
\end{example}

\begin{theorem}
    {Euler-Maclaurin求积公式}{Euler-Maclaurin integration}
    若$f\in\cont^{2m+2}[a,b]$,则有
    \begin{equation}
        I(f)=T_n(f)+\sum_{\ell=1}^m\frac{B_{2\ell}}{(2\ell)!}[f^{(2\ell-1)}(a)-f^{(2\ell-1)}(b)]h^{2\ell}+r_{m+1},
    \end{equation}
    其中$T_n$是复合梯形公式,
    \begin{equation}
        r_{m+1}=-\frac{B_{2m+2}}{(2m+2)!}(b-a)f^{(2m+2)}(\eta)h^{2m+2},\quad\exists\eta\in(a,b).
    \end{equation}
\end{theorem}

\subsection{Richardson外推方法}

% \paragraph{如何缩小数值积分的误差}

由Euler-Maclaurin求积公式,考虑复合梯形公式关于$h<1$的渐进级数
\[
    T_n(f)=I(f)+\tau_1h^2+\tau_2h^4+\cdots+\tau_mh^{2m}+\bigo(h^{2m+2}).
\]
当区间无限小时,$h\to0$,$T_n(f)\to I(f)$是准确积分。
为了得到$I(f)$的一个近似,考虑
\[
    h_i=\frac{b-a}{n_i},\quad i=0,1,\ldots
\]
且$n_0<n_1<\cdots<n_m<\cdots$,记关于$h^2$的插值多项式$p_{i,j}\in\poly^{j-i}$满足插值条件:
\[
    p_{i,j}(h_i)=T_{n_i}(f),\enspace
    p_{i,j}(h_{i+1})=T_{n_{i+1}}(f),\enspace
    \ldots,\enspace
    p_{i,j}(h_j)=T_{n_j}(f),
\]
由Neville算法\eqref{eqn:Neville},有递推关系:$p_{i,i}(h)=T_{n_i}(f)$,
\[
    p_{i,j}(h)=\frac{(h^2-h_i^2)p_{i+1,j}(h)+(h_j^2-h^2)p_{i,j-1}(x)}{h_j^2-h_i^2},
\]

\begin{theorem}
    {Richardson外推方法}{}
    将外推值$p_{0,i}(0)$作为$I(f)$的一个近似,记$T_{i,j}:=p_{i,j}(0)$,有递推关系
    \begin{equation}
        \label{eqn:Richardson}
        T_{i,j}=T_{i+1,j}+\frac{T_{i+1,j}-T_{i,j-1}}{n_j^2/n_i^2-1}.
    \end{equation}
\end{theorem}

\subsection{Romberg求积方法}

\begin{theorem}
    {Romberg求积方法}{}
    % 记$T_j^k:=T_{j-k,j}$,令$n_k=2^k$,即子区间长度$h/2^k$,式\eqref{eqn:Richardson}变为
    % \begin{equation}
    %     \label{eqn:Romberg}
    %     T_i^k=T_i^{k-1}+\frac{T_i^{k-1}-T_{i-1}^{k-1}}{4^k-1}.
    % \end{equation}
    对区间$[a,b]$进行$n_k=2^k$等分($k=0,1,\ldots$),利用复合梯形公式计算$T_1^k$,再递推
    \begin{equation}
        T_{i+1}^k=\frac{4^iT_i^{k+1}-T_i^k}{4^i-1}.
    \end{equation}
    直到$T_{i+1}^0$满足精度要求。
\end{theorem}

特别地,$T_1^0$为梯形公式,$T_2^k$为复合Simpson公式。

% \begin{example}
%     {$j=i+1$时的Romberg方法}{}
%     当$j=i+1$时,有
%     \[
%         T_{i,i+1}=\frac{4T_{i+1,i+1}-T_{i,i}}3,
%     \]
%     得到Simpson公式。
% \end{example}

% \begin{theorem}
%     {一般外推方法}{}

% \end{theorem}

\section{Gauss求积公式}
\label{sec:Gauss-type quadrature rules}

考虑一般的带权积分$\rho\not\equiv 1$。

\begin{theorem}
    {}{}
    给定插值节点$x_0,\ldots,x_n$,$\ell_0,\ldots,\ell_n$是插值基函数,则当且仅当
    \[
        A_k^{(n)}=I(\ell_k)=\int_a^b\ell_k(x)\rho(x)\d x
    \]
    时$I_n(\cdot)$的代数精确度至少为$n$。
\end{theorem}

\begin{proof}
    充分性:$\forall f\in\poly_n$,有 
    \[
        f(x)=\sum_{k=0}^nf(x_k)\ell_k(x),
    \]
    于是
    \[
        I(f)=\sum_{k=0}^nf(x_k)I(\ell_k)=\sum_{k=0}^nA_k^{(n)}f(x_k)=I_n(f);
    \]
    必要性:若$\forall f\in\poly_n$均有$I(f)=I_n(f)$,则
    \[
        I(f)-I_n(f)=\sum_{k=0}^n[I(\ell_k)-A_k^{(n)}]f(x_k)=0,
    \]
    特别地,取$f(x)=1,x,\ldots,x^n$,得到一个Vandermonde矩阵,从而$A_k^{(n)}=I(\ell_k)$。
\end{proof}

\begin{theorem}
    {}{}
    $n+1$个求积节点的求积公式$I_n(\cdot)$代数精确度不超过$2n+1$。
\end{theorem}

\begin{proof}
    给定节点$x_0,\ldots,x_n$定义
    \[
        \omega_{n+1}(x):=\prod_{k=0}^n(x-x_k),
    \]
    则$I(\omega_{n+1}^2)>0=I_n(\omega_{n+1}^2)$,说明代数精确度$<2n+2$。
\end{proof}

\paragraph{如何达到最大可能代数精确度}

由\eqref{eqn:Lagrange remainder},余项
\[
    R_n(x)=\frac{f^{(n+1)}(\xi(x))}{(n+1)!}\omega_{n+1}(x).
\]
故积分误差为
\[
    E_n(f)=\frac1{(n+1)!}I\bigkh{f^{(n+1)}(\xi(x))\omega_{n+1}(x)}.
\]
若$I_n$具有$2n+1$次代数精确度,即$\forall f\in\poly_{2n+1}$均有$E_n(f)=0$,由于$f^{(n+1)}\in\poly_n$,故
\begin{equation}
    \omega_{n+1}\perp\poly_n.
\end{equation}
即$\omega_{n+1}$为$[a,b]$上权$\rho$的$n+1$次正交多项式,这可以通过选取节点得到。

\begin{theorem}
    {Gauss求积公式}{Gauss quadrature rule}
    取$[a,b]$上权$\rho$的$n+1$次正交多项式的根$x_0,\ldots,x_n$作为求积节点,则
    \[
        I_n(f)=\sum_{k=0}^nI(\ell_k)f(x_k),
    \]
    具有$2n+1$阶代数精确度,称为Gauss公式,节点称为Gauss点。
\end{theorem}

\begin{theorem}
    {}{}
    Gauss求积公式是一致稳定的。
\end{theorem}

\begin{proof}
    略
\end{proof}
