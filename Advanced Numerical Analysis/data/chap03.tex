\chapter{函数逼近}
\label{chap:optimal approximation}

\begin{definition}
    {函数逼近}{optimal approximation}
    给定函数$f\in\cont[a,b]$和子集$\Phi\subset\cont[a,b]$ (如多项式函数),寻找最佳逼近:
    \begin{equation}
        \varphi^*=\arg\min_{\varphi\in\Phi}\norm{f-\varphi}.
    \end{equation}
\end{definition}

\begin{remark}
    ~
    \begin{itemize}
        \item $\Phi$一般是简单函数集合,如多项式函数。但$\Phi$未必是线性空间。
        \item $f\notin\Phi$,且关于$f$的信息可能有误差;
        \item 近似程度的度量:平方(squares)逼近$\norm\cdot_2$和一致(uniform)逼近$\norm\cdot_\infty$。
    \end{itemize}
\end{remark}

\section{最佳平方逼近}
\label{sec:optimal squares approximation}

\subsection{法方程}

最佳平方逼近中,$\norm\cdot_2$是由$\cont[a,b]$的某个内积$\inp\cdot\cdot$诱导出的范数:
\[
    \norm f_2:=\sqrt{\inp ff}.
\]
给定$\Phi$的一组基$\varphi_0,\ldots,\varphi_n$,则
\[
    \varphi=\sum_{i=0}^na_i\varphi_i,
\]
实函数情况,简单计算得到
\[
    \norm{f-\varphi}_2^2=\sum_{i,j=0}^n\inp{\varphi_i}{\varphi_j}a_ia_j-2\sum_{i=0}^n\inp{\varphi_i}fa_i+\inp ff.
\]
是一个关于$a_0,\ldots,a_n$的二次函数,故Hess矩阵
\begin{equation}
    \nabla^2\norm{f-\varphi}_2^2=2\begin{bmatrix}
        \inp{\varphi_0}{\varphi_0}&\cdots&\inp{\varphi_0}{\varphi_n}\\
        \vdots&\ddots&\vdots\\
        \inp{\varphi_n}{\varphi_0}&\cdots&\inp{\varphi_n}{\varphi_n}
    \end{bmatrix}
\end{equation}
为对称正定矩阵,因而$\norm{f-\varphi}_2^2$有最小值。由
\begin{equation*}
    \pp{a_i}\norm{f-\varphi}_2^2=2\fkh{\sum_{j=0}^n\inp{\varphi_i}{\varphi_j}a_j-\inp{\varphi_i}f}=0,
\end{equation*}
得到
\begin{theorem}
    {法方程}{}
    \begin{equation}
        \begin{bmatrix}
            \inp{\varphi_0}{\varphi_0}&\cdots&\inp{\varphi_0}{\varphi_n}\\
            \vdots&\ddots&\vdots\\
            \inp{\varphi_n}{\varphi_0}&\cdots&\inp{\varphi_n}{\varphi_n}
        \end{bmatrix}\begin{bmatrix}
            a_0\\\vdots\\a_n
        \end{bmatrix}=\begin{bmatrix}
            \inp{\varphi_0}f\\\vdots\\\inp{\varphi_n}f
        \end{bmatrix}.
    \end{equation}
\end{theorem}

\begin{remark}
    最佳平方逼近$\varphi^*$满足$f-\varphi^*\perp\Phi$,有
    \begin{equation}
        \norm f_2^2=\norm{\varphi^*}_2^2+\norm{f-\varphi^*}_2^2.
    \end{equation}
\end{remark}

\begin{example}
    {}{}
    给定$\Phi=\lspan(1,x^2)$,求$f=x$在$[0,1]$上的最佳平方逼近。
    \[
        \begin{bmatrix}
            \inp11&\inp1{x^2}\\
            \inp{x^2}1&\inp{x^2}{x^2}
        \end{bmatrix}\begin{bmatrix}
            a\\b
        \end{bmatrix}=\begin{bmatrix}
            \inp1x\\\inp{x^2}x
        \end{bmatrix}
    \]
    解得$\varphi^*(x)=a+bx^2$的系数
    \[
        \begin{bmatrix}
            a\\b
        \end{bmatrix}=\begin{bmatrix}
            1&1/3\\1/3&1/5
        \end{bmatrix}\iv\begin{bmatrix}
            1/2\\1/4
        \end{bmatrix}=\begin{bmatrix}
            3/16\\15/16
        \end{bmatrix}.
    \]
\end{example}

\subsection{正交多项式}

\begin{definition}
    {Hilbert矩阵}{Hilbert matrix}
    考虑$f\in\cont[0,1]$在$\poly_n$上的最佳平方逼近,$1,x,\ldots,x^n$作为基,法方程的系数矩阵为Hilbert矩阵
    \begin{equation}
        H_n:=\begin{bmatrix}
            1&1/2&\cdots&1/(n+1)\\
            1/2&1/3&\cdots&1/(n+2)\\
            \vdots&\vdots&\ddots&\vdots\\
            1/(n+1)&1/(n+2)&\cdots&1/(2n+1)
        \end{bmatrix}
    \end{equation}
    这个矩阵是严重病态的。
\end{definition}

\begin{remark}
    解决方法:将$\poly_n$的基改进为正交基$\varphi_0,\ldots,\varphi_n$,即
    \[
        \inp{\varphi_i}{\varphi_j}=\norm{\varphi_i}_2^2\delta_{ij},
    \]
    则系数矩阵是对角的,
    \begin{equation}
        \varphi^*=\sum_{i=0}^n\frac{\inp f{\varphi_i}}{\norm{\varphi_i}_2^2}\varphi_i.
    \end{equation}
\end{remark}

\begin{definition}
    {广义Fourier级数}{}
    给定$\cont[a,b]$上的一组正交归一函数$\varphi_0,\varphi_1,\ldots$,$f$在$\Phi_n:=\lspan(\varphi_0,\ldots,\varphi_n)$中的最佳平方逼近为
    \[
        f_n=\sum_{i=0}^na_i\varphi_i,\quad a_i:=\inp f{\varphi_i},
    \]
    由于
    \[
        \sum_{i=0}^n\abs{a_i}^2=\norm{f_n}_2^2\leq\norm f_2^2,\implies\sum_{i=0}^\infty\abs{a_i}^2<\infty,
    \]
    故$f_\infty$收敛且$f_\infty\in\overline{\inp{\cont[a,b]}{\norm\cdot_2}}$,称为广义Fourier级数。
\end{definition}

\begin{remark}
    若$\norm\cdot_2$是权系数$\rho$的内积$\inp\cdot\cdot_\rho$诱导的范数,则
    \begin{equation}
        \overline{\inp{\cont[a,b]}{\norm\cdot_2}}=\mathcal L_\rho^2[a,b]
    \end{equation}
\end{remark}

\begin{theorem}
    {}{}
    给定有界闭区间$[a,b]$,若$f\in\cont[a,b]$在$\poly_n$上的最佳平方逼近为$f_n$,则
    \begin{equation}
        \lim_{n\to\infty}\norm{f-f_n}_2=0.
    \end{equation}
    这说明多项式函数是完备的。
\end{theorem}

% \begin{proof}
%     由
%     \[
%         \norm{f-f_n}_2^2=\norm f_2^2-\norm{f_n}_2^2=\norm f_2^2-\sum_{i=0}^n\abs{a_i}^2
%     \]
% \end{proof}

\begin{theorem}
    {Legendre多项式作最佳平方逼近}{}
    若内积为$[-1,1]$上的$\rho\equiv 1$内积,若$f\in\cont^2[-1,1]$,则$\forall\epsilon>0$,$\exists N>0$使得$\forall n\geq N$
    \begin{equation}
        \norm{f-f_n}_\infty\leq\frac{\epsilon}{\sqrt n}.
    \end{equation}
\end{theorem}

\begin{theorem}
    {零平方误差最小}{}
    在所有首项系数为1的$n$次多项式中,Legendre多项式$P_n(x)$在$[-1,1]$与0的平方误差最小。
\end{theorem}

\begin{proof}
    $\forall f\in\poly_n$且首项系数为1,有
    \[
        f=P_n+a_{n-1}P_{n-1}+\cdots+a_0P_0,
    \]
    则
    \[
        \norm f_2^2=\norm{P_n}_2^2+\abs{a_{n-1}}^2\norm{P_{n-1}}_2^2+\cdots+\abs{a_0}^2\norm{P_0}_2^2\geq\norm{P_n}_2^2.
        \qedhere
    \]
\end{proof}

\section{最小二乘法}
\label{sec:least squares algorithm}

给定$\Phi=\lspan\{\varphi_0,\ldots,\varphi_n\}$

\begin{definition}
    {Haar条件}{Haar condition}
    $\forall \varphi_i\in\Phi$且$\varphi_i\neq 0$在$x_0,\ldots,x_m$上不同时为0,则称$\Phi$满足Haar条件。
\end{definition}

\begin{example}
    {}{}
    若$m\geq n$,则$\poly_n$在$x_0,\ldots,x_m$上满足Haar条件。
\end{example}

\section{最佳一致逼近}
\label{sec:optimal uniform approximation}

考虑用多项式函数进行最佳一致逼近:
\begin{equation}
    \varphi^*=\arg\inf_{\varphi\in\poly_n}\norm{f-\varphi}_\infty.
\end{equation}

\begin{definition}
    {}{}
    称$\norm{f-p_n}_\infty$为$p_n$关于$f$的偏差。
    若$\exists x$使得
    \begin{equation}
        f(x)-p_n(x)=\pm\norm{f-p_n}_\infty.
    \end{equation}
    则称$x$为$p_n$关于$f$的正(负)偏差点。
\end{definition}

\begin{theorem}
    {最佳一致逼近的存在性}{}
    $\poly_n$关于$f\in\cont[a,b]$的最小偏差可以达到。
\end{theorem}

\begin{proof}
    记$\varphi(f,p_n)=\norm{f-p_n}_\infty$,则$\varphi(f,\cdot)$是关于$p_n$系数$a_0,\ldots,a_n$的连续函数,且$\varphi(0;p_n)$在单位球面$a_0^2+\cdots+a_n^2=1$上达到正最小值$\mu$。
    \[
        \norm{f-p_n}_\infty\geq\norm{p_n}_\infty-\norm f_\infty\geq\mu\sqrt{\sum_{i=0}^na_i^2}-\norm f_\infty.
    \]
    当
    \[
        \sum_{i=0}^na_i^2\geq\frac4{\mu^2}\norm f_\infty^2
    \]
    时,$\norm{f-p_n}_\infty\geq\norm f_\infty$,故最佳一致逼近存在。
\end{proof}

\begin{lemma}
    最佳一致逼近多项式$p_n^*$关于$f$的正负偏差点同时存在。
\end{lemma}

\begin{definition}
    {Chebyshev交错点}{}
    若$x_1,\ldots,x_m$是$p_n$关于$f$的偏差点且轮流为正负,则称其为一组Chebyshev交错点。
\end{definition}

\begin{theorem}
    {}{}
    $p_n^*$是$f$的最佳一致逼近多项式$\iff$存在$n+2$个Chebyshev交错点。
\end{theorem}

\begin{corollary}
    最佳一致逼近多项式是一个Lagrange插值多项式,其插值点在$(a,b)$内。
\end{corollary}

\begin{theorem}
    {最佳一致逼近的唯一性}{}
    最佳一致逼近多项式是唯一的。
\end{theorem}

\begin{theorem}
    {零偏差最小}{}
    在所有首项系数为1的$n$次多项式中,Chebyshev多项式$T_n(x)$在$[-1,1]$与0的偏差最小。
\end{theorem}

\begin{example}
    {最佳一次一致逼近多项式的求法}{}
    给定$f\in\cont^2[a,b]$且$f''(x)\neq 0$,求$f$的最佳一次一致逼近多项式$g=a_0+a_1x$。

    由$f''$恒正或恒负可知$f-g$在$(a,b)$上的极值点是唯一的,记作$x^*$,
    \[
        f'(x^*)-g'(x^*)=0,
    \]
    且$\{a,x^*,b\}$是一组Chebyshev交错点,故
    \[
        f(a)-g(a)=-[f(x^*)-g(x^*)]=f(b)-g(b)
    \]
    可确定$a_1,x^*,a_0$:
    \[
        \begin{cases}
            a_1=\frac{f(b)-f(a)}{b-a},\\
            x^*={f'}\iv(a_0),\\
            a_0=\frac12[f(a)+f(x^*)-a_1(a+x^*)]
        \end{cases}
    \]
\end{example}
