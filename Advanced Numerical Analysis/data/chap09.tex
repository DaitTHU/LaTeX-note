\chapter{常微分方程初值问题的数值解法}
\label{chap:ordinary differential equation}

\section{常微分方程初值问题}

\begin{definition}
    {一阶非线性常微分方程初值问题}{first-order nonlinear ODE}
    考虑函数$y:[x_0,b]\to\RR^d$满足
    \begin{subequations}
        \label{eqn:ode}
        \begin{align}
            \label{eqn:ode eq}
            &y'(x)=f(x,y(x)),\\
            &y(x_0)=y_0.
        \end{align}
    \end{subequations}
    特别地,若$f(x,y)=f(y)$与$x$无关,则称为自治问题(autonomous problem)。
\end{definition}

\begin{example}
    {Hamilton方程}{Hamiltonian equation}
    给定相空间上的Hamilton量$H:\RR^d\times\RR^d\to\RR$,广义坐标$q\in\RR^d$和广义动量$p\in\RR^d$随时间的演化满足Hamilton方程(或正则方程, canonial equations):
    \begin{subequations}
        \begin{align}
            \dot q_i&=\pv H{p_i},\\
            \dot p_i&=-\pv H{q_i}.
        \end{align}
    \end{subequations}
    比如调和谐振子$H=kx^2/2+p^2/2m$,令角频率$\omega=\sqrt{k/m}$,则Hamilton方程解得
    \[
        \lhkh{\begin{aligned}
            \dot x&=\frac pm,\\
            \dot p&=-kx.
        \end{aligned}}\implies
        \lhkh{\begin{aligned}
            x&=x_0\cos(\omega t)+\frac{p_0}m\frac{\sin(\omega t)}\omega,\\
            p&=p_0\cos(\omega t)-kx_0\frac{\sin(\omega t)}\omega.
        \end{aligned}}
    \]
\end{example}

\begin{definition}
    {半线性高阶常微分方程}{half-linear nth-order ODE}
    考虑函数$y:[x_0,b]\to\RR^d$满足
    \begin{subequations}
        \begin{align}
            &y^{(n)}=f(x,y,y',\dots,y^{(n-1)}),\\
            &y(x_0)=y_0,\enspace y'(x_0)=y_1,\enspace\ldots,\enspace y^{(n-1)}(x_0)=y_{n-1}.
        \end{align}
    \end{subequations}
    可转化为一阶常微分方程:
    \begin{equation}
        Y:=\begin{bmatrix}
            y\\y'\\\vdots\\y^{(n-1)}
        \end{bmatrix},\quad
        Y'=\begin{bmatrix}
            O&I\\ &\ddots&\ddots\\ &&\ddots&I\\ &&&O
        \end{bmatrix}Y+\begin{bmatrix}
            O\\\vdots\\O\\-f(x,Y)
        \end{bmatrix}
    \end{equation}
\end{definition}

\begin{theorem}
    {Lipschitz条件}{Lipschitz condition}
    若$f(x,y)$在$[a,b]\times\RR^d$上连续,且存在不依赖$x$的常数$L>0$使得$\forall y_1,y_2\in\RR^d$,
    \begin{equation}
        \norm{f(x,y_1)-f(x,y_2)}\leq L\norm{y_1-y_2},
    \end{equation}
    则$\forall y_0$,初值问题存在唯一解。
\end{theorem}

\begin{remark}
    我们总是假设$f(x,y)$满足Lipschitz条件,以保证初值问题存在唯一解。
\end{remark}

\begin{definition}
    {适定性}{well-posedness}
    若初值问题\eqref{eqn:ode}存在唯一解$y$,
    考虑其扰动
    \begin{subequations}
        \label{eqn:ode perturbed}
        \begin{align}
            &y'(x)=f(x,y(x))+\D f(x),\\
            &y(x_0)=y_0+\D y_0,
        \end{align}
    \end{subequations}
    若$\forall\epsilon>0$,$\exists\delta$使得当$\norm{\D f}<\delta,\;\norm{\D y_0}<\delta$时,扰动问题\eqref{eqn:ode perturbed}存在唯一解$\tilde y$且满足
    \begin{equation}
        \norm{\tilde y-y}<\epsilon,
    \end{equation}
    则称初值问题\eqref{eqn:ode}是适定的(well-posed)。
\end{definition}

\begin{remark}
    % 零稳定是指扰动量$\delta\to0$时的渐进稳定性。
    适定性说明初值问题\eqref{eqn:ode}的解存在唯一且连续依赖于$f$和初值$y_0$。
\end{remark}

\section{Euler方法}

\begin{theorem}
    {Euler方法}{Euler method}
    考虑区间$[x_0,b]$的一个划分:
    \[
        a=x_0<x_1<\cdots<x_N=b
    \]
    其中$h_n:=x_n-x_{n-1}$称为步长。对于初值问题\eqref{eqn:ode},若$h:=\max_n h_n$充分小,则可用差商近似代替导数:
    \[
        \frac{y(x_{n+1})-y(x_n)}{h_{n+1}}\approx y'(x_n)=f(x_n,y(x_n)).
    \]
    即
    \begin{equation}
        \label{eqn:Euler}
        y_{n+1}=y_n+h_{n+1}f(x_n,y_n).
    \end{equation}
    这称为Euler方法,是一种一阶显式单步法。
\end{theorem}

\begin{remark}
    必须关注的三个问题:
    \begin{enumerate}
        \item 收敛性:当$h\to0$时,$\sup_n\norm{y_n-y(x_n)}\to0$;
        \item 收敛速度;
        \item 稳定性:舍入误差的影响能否控制。
    \end{enumerate}
\end{remark}

\begin{theorem}
    {初值问题的等价积分形式}{}
    考虑划分区间上的初值问题\eqref{eqn:ode}的等价积分形式
    \[
        y(x)=y_n+\int_{x_n}^xf(t,y(t))\d t.
    \]
    即可通过数值积分近似计算$y(x_{n+1})$作为$y_{n+1}$。
\end{theorem}

\begin{corollary}
    用左矩形法则近似积分,即
    \[
        y(x_{n+1})=y(x_n)+h_{n+1}f(x_n,y(x_n))+R_{n+1},
    \]
    舍去积分余项$R_{n+1}$,即得Euler公式\eqref{eqn:Euler}。
\end{corollary}

\begin{example}
    {隐式Euler方法}{implicit Euler method}
    用右矩形法则近似积分,即
    \[
        y(x_{n+1})=y(x_n)+h_{n+1}f(x_{n+1},y(x_{n+1}))+R_{n+1},
    \]
    舍去积分余项$R_{n+1}$,即得隐式Euler公式
    \begin{equation}
        \label{eqn:implicit Euler}
        y_{n+1}=y_n+h_{n+1}f(x_{n+1},y_{n+1}).
    \end{equation}
\end{example}

\begin{example}
    {梯形公式}{}
    用梯形法则近似积分,即得梯形公式
    \begin{equation}
        \label{eqn:trapezoidal ode}
        y_{n+1}=y_n+\frac{h_{n+1}}2[f(x_n,y_n)+f(x_{n+1},y_{n+1})].
    \end{equation}
\end{example}

\begin{definition}
    {单步法的一般形式}{}
    结合\eqref{eqn:Euler},\eqref{eqn:implicit Euler}和\eqref{eqn:trapezoidal ode},初值问题\eqref{eqn:ode}的单步法一般形式为
    \[
        y_{n+1}=y_n+h_{n+1}\varphi(x_n,x_{n+1},y_n,y_{n+1}).
    \]
    % 增量函数$\varphi$与$f$有关。%且$\varphi(x_n,x_{n+1},y_n,y_{n+1})=f(x_n,y_n)$时即为Euler方法。
    若$\varphi$显含$y_{n+1}$,则称为隐式方法;否则称为显式方法。
    % $\varphi$应满足:
    % \[
    %     \norm{\varphi(x_n,x_{n+1},y_n,y_{n+1})-\varphi(x_n,x_{n+1},\tilde y_n,\tilde y_{n+1})}\leq L_f(\norm{y_n-\tilde y_n}+\norm{y_{n+1}-\tilde y_{n+1}}).
    % \]
    % 增量函数$\varphi$满足$\varphi(x_n,x_{n+1},y_n,y_{n+1};0)=0$且
    % 称一步误差为:
    % \[
    %     R_{n+1}=y(x_{n+1})-y_{n+1}
    % \]
\end{definition}

\begin{definition}
    {局部截断误差}{local truncation error}
    单步法在$x_{n+1}$处的局部截断误差定义为积分余项:
    \[
        R_{n+1}=y(x_{n+1})-y(x_n)-h_{n+1}\varphi(x_n,x_{n+1},y(x_n),y(x_{n+1})).
    \]
\end{definition}

\begin{definition}
    {相容性}{consistency}
    若$\lim_{h\to0}R=0$,则称单步法是相容的。
    \tcblower
    若存在不依赖$x$的常数$M>0$和整数$p\geq 1$,使得
    \begin{equation}
        \norm{y(x+h)-y(x)-h\varphi(x,x+h,y(x),y(x+h))}\leq Mh^{p+1},
    \end{equation}
    则称单步法至少$p$阶相容的。若局部截断误差可以展开成
    \[
        R=\psi(x,y)h^{p+1}+\bigo(h^{p+2}),
    \]
    称$\psi(x,y)h^{p+1}$为局部截断误差的主项。
\end{definition}

\begin{example}
    {Euler方法的局部截断误差}{}
    对于Euler方法\eqref{eqn:Euler},局部截断误差为
    \begin{align*}
        R_{n+1}&=y(x_{n+1})-y(x_n)-h_{n+1}f(x_n,y(x_n))\\
        &=y(x_n)+h_{n+1}y'(x_n)+\frac{h_{n+1}^2}2y''(x_n)+\bigo(h_{n+1}^3)-y(x_n)-h_{n+1}y'(x_n)\\
        &=\frac{h_{n+1}^2}2y''(x_n)+\bigo(h_{n+1}^3).
    \end{align*}
    因此Euler方法是一阶相容的,局部截断误差的主项为$h_{n+1}^2y''(x_n)/2$。
\end{example}

\begin{example}
    {梯形方法的局部截断误差}{}
    对于梯形方法\eqref{eqn:trapezoidal ode},局部截断误差为
    \begin{align*}
        R_{n+1}&=y(x_{n+1})-y(x_n)-\frac{h_{n+1}}2[f(x_n,y(x_n))+f(x_{n+1},y(x_{n+1}))]\\
        &=-\frac{h_{n+1}^3}{12}y'''(x_n)+\bigo(h_{n+1}^4).
    \end{align*}
    因此梯形方法是二阶相容的,局部截断误差的主项为$-h_{n+1}^3y'''(x_n)/12$。
\end{example}

\begin{theorem}
    {单步法相容}{}
    单步法相容$\iff f(x,y)=\varphi(x,x,y,y)$。
\end{theorem}

