\documentclass[a4paper, 11pt]{article}

\usepackage[UTF8]{ctex}

\usepackage[T1]{fontenc}								% 字体
\catcode`\。=\active
\newcommand{。}{.} % {\ifmmode\text{.}\else .\fi}
\catcode`\(=\active
\catcode`\)=\active
\newcommand{(}{(}
\newcommand{)}{)}

% \usepackage{zhlineskip}

\usepackage{nicematrix}
% \usepackage{setspace}
% \linespread{1}						% 一倍行距
\setlength{\headheight}{14pt}			% 页眉高度
% \setlength{\lineskip}{0ex}			% 行距
\renewcommand\arraystretch{.82}		% 表格

\usepackage{amssymb, amsmath, amsfonts, amsthm}			% 数学符号,公式,字体,定理环境
\everymath{\displaystyle}			% \textstyle \scriptstyle \scriptscriptstyle
\allowdisplaybreaks[4]      		% 使用行间公式格式
% \makeatletter
% \renewcommand{\maketag@@@}[1]{\hbox{\m@th\normalsize\normalfont#1}}
% \makeatother
\newif\ifcontent\contenttrue		% if 显示目录
\newif\ifparskip\parskipfalse		% if 增加目录后的行距
\newif\ifshowemail\showemailfalse	% if 显示 email
\def\firstandforemost{
	\maketitle
	%\thispagestyle{empty}\clearpage
	\ifcontent
		\renewcommand{\contentsname}{目录}
		\tableofcontents
		\thispagestyle{empty}
		\clearpage
	\fi
	\ifparskip
		\setlength{\parskip}{.8ex}	% 设置额外的段距,目录后
	\fi								% 在 \firstandforemost 前设置 \parskiptrue
	\makenomenclature
	\printnomenclature
	\setcounter{page}{1}
}

\usepackage{mathtools}									% \rcase 环境等

% \usepackage{physics}

\usepackage[]{siunitx}									% 国际制单位
\sisetup{
	inter-unit-product = \ensuremath{{}\cdot{}},
	per-mode = symbol,
	per-mode = reciprocal-positive-first,
	range-units = single,
	separate-uncertainty = true,
	range-phrase = \ifmmode\text{\;-\;}\else\;-\;\fi
}
\DeclareSIUnit\angstrom{\text{Å}}
\DeclareSIUnit\atm{\text{atm}}
% SIunits 额外定义了一个 \square 表示平方,
% 还会把 \cdot 空格加大,真有够无语的 😅

\usepackage{authblk}									% 作者介绍
\ifx \coursefullname\undefined
	\ifx \coursename\undefined
		\def\coursename{笔记}
	\fi
	\def\coursefullname{\coursename}
\fi
\ifx \authorname\undefined
	\def\authorname{Dait}
\fi
\ifx \departmentname\undefined
	\def\departmentname{THU}
\fi
\ifx \emailaddress\undefined
	\def\emailaddress{daiyj20@mails.tsinghua.edu.cn}
\fi
\ifx \beginday\undefined
	\def\beginday{2021}
\fi
\ifx \endday\undefined
	\def\endday{\number\year/\number\month/\number\day}
\fi
\ifx \titleannotation\undefined
	\ifx \teachername\undefined
		\title{\textbf{\coursefullname}}
	\else
		\title{\textbf{\coursefullname}\\\small\textit{主要整理自\teachername 老师讲义}}
	\fi
\else
	\title{\textbf{\coursefullname}\\\small\textit{(\titleannotation)}}
\fi
\newif\ifdefaultauthor\defaultauthortrue
\ifdefaultauthor
	\author{by~\authorname~at~\departmentname}
	\ifshowemail
		\affil{\emailaddress}
	\fi
\fi
\ifx \endday\beginday
	\date{\beginday}
\else
	\date{\beginday~-~\endday}
\fi

\usepackage{hyperref}									% 链接
\ifx \courseEnglishname\undefined
	\def\courseEnglishname{Note}
\fi
\ifx \authorEnglishname\undefined
	\def\authorEnglishname{Dait}
\fi
\hypersetup{
	% dvipdfm								% 表示用 dvipdfm 生成 pdf
	pdftitle={\coursename},
	pdfauthor={\authorname},
	colorlinks=true, breaklinks=true,		% 超链接设置
	linkcolor=black, citecolor=black, urlcolor=blue
}

\usepackage[british]{babel}								% 长单词自动连字符换行
\hyphenation{long-sen-ten-ce}				% 自定义拆分方式

\usepackage{tikz}
\usetikzlibrary{quotes, angles}
\usepackage{pgfplots}
\pgfplotsset{compat=1.17}								% TikZ
\newcommand{\coor}[5][0]{
	\draw[thick,latex-latex](#1,#3)node[left]{$#5$}--(#1,0)node[shift={(-135:7pt)}]{$O$}--(#1+#2,0)node[right]{$#4$}
}			% 坐标轴

\usepackage{enumerate}									% 编号
\usepackage{paralist}
\setlength{\pltopsep}{1ex}
\setlength{\plitemsep}{1ex}
\ifx \eqnrange\undefined
	\numberwithin{equation}{section}
\else
	\numberwithin{equation}{\eqnrange}
\fi

\renewcommand{\thempfootnote}{\Roman{mpfootnote}}
\renewcommand{\thefootnote}{\Roman{footnote}}		% 注释上标 I, II,...
\newcommand{\sectionstar}[1]{
	\section[\hspace{-.8em}*\hspace{.3em}#1]{\hspace{-1em}*\hspace{.5em}#1}
}
\newcommand{\subsectionstar}[1]{					% 带星号的 section 和 subsection
	\subsection{\hspace{-1em}*\hspace{.5em}#1}
}
\newcommand{\subsubsectionstar}[1]{					% 带星号的 section 和 subsection
	\subsubsection{\hspace{-1em}*\hspace{.5em}#1}
}
\newcommand*{\appendiks}{
	\appendix
	\part*{附录}
	\addcontentsline{toc}{part}{附录}
}
\iffalse			% 不清楚
	\newcommand{\varsection}[1]{
		\refstepcounter{section}
		\section*{\thesection\quad #1}
		\addcontentsline{toc}{section}{\makebox[0pt][r]{*}\thesection\quad #1}
	}
\fi

\usepackage{fancyhdr}									% 页眉页脚
\ifx \coursename\undefined
	\def\coursename{笔记}
\fi
\fancyhf{}\pagestyle{fancy}
\fancyhead[L]{\coursename\rightmark}
\fancyhead[R]{by~\authorname}
\fancyfoot[C]{-~\thepage~-} 			%页码

\usepackage{colortbl, booktabs}							% 表

\usepackage{graphicx}
\usepackage{float}
\usepackage{caption}									% 图
\captionsetup{
	margin=20pt, format=hang,
	justification=justified
}
\newcounter{tikzpic}
\def\tikzchap{
	\stepcounter{tikzpic}\\
	\small 图~\thetikzpic\quad
}
\newcounter{linetable}
\newcommand{\tablechap}[1]{
	\stepcounter{linetable}
	{\small 表~\thelinetable\quad #1}\\[1em]
}

\usepackage{tcolorbox}									% 盒子
\tcbuselibrary{theorems, skins, breakable}
\definecolor{MatchaGreen}{HTML}{73C088}		% 抹茶绿B7C6B3
\newtcbtheorem[number within = subsection]{example}{例}{
	enhanced, breakable, sharp corners,
	attach boxed title to top left = {yshifttext = -1mm},
	before skip = 2ex,
	colback = MatchaGreen!5,				% 文本框内的底色
	colframe = MatchaGreen,					% 文本框框沿的颜色
	fonttitle = \bfseries,					% 标题字体用粗体	coltitle 默认 white,
	boxed title style = {
			sharp corners, size = small, colback = MatchaGreen,
		}
}{exm}
\definecolor{MelancholyBlue}{HTML}{9EAABA}	% melancholy: 沮丧
\newcounter{pslt}
\setcounter{pslt}{-1}
\newtcbtheorem[use counter = pslt]{posulate}{假设}{
	enhanced, breakable, sharp corners,
	attach boxed title to top left = {yshifttext = -1mm}, before skip = 2ex,
	colback = MelancholyBlue!5, colframe = MelancholyBlue, fonttitle = \bfseries,
	boxed title style = {
			sharp corners, size = small, colback = MelancholyBlue,
		}
}{psl}
\definecolor{PureBlue}{HTML}{80A3D0}
\newtcbtheorem[number within = subsection]{definition}{定义}{
	enhanced, breakable, sharp corners,
	attach boxed title to top left = {yshifttext = -1mm}, before skip = 2ex,
	colback = PureBlue!5, colframe = PureBlue, fonttitle = \bfseries,
	boxed title style = {
			sharp corners, size = small, colback = PureBlue,
		}
}{dfn}
\definecolor{PeachRed}{HTML}{EA868F}
\newtcbtheorem[number within = subsection]{theorem}{定理}{
	enhanced, breakable, sharp corners,
	attach boxed title to top left = {yshifttext = -1mm}, before skip = 2ex,
	colback = PeachRed!5, colframe = PeachRed, fonttitle = \bfseries,
	boxed title style = {
			sharp corners, size = small, colback = PeachRed,
		}
}{thm}
\definecolor{SchembriumYellow}{HTML}{fbd26a}	% 申博太阳城黄
\newtcbtheorem[number within = section]{method}{方法}{
	enhanced, breakable, sharp corners,
	attach boxed title to top left = {yshifttext = -1mm}, before skip = 2ex,
	colback = SchembriumYellow!5, colframe = SchembriumYellow, fonttitle = \bfseries,
	boxed title style = {
			sharp corners, size = small, colback = SchembriumYellow,
		}
}{mtd}
% 保留颜色
\definecolor{fadedgold}{HTML}{D9CBB0}		% 褪色金
\definecolor{saturatedgold}{HTML}{F0E0C2}	% staurated: 饱和
\definecolor{elegantblue}{HTML}{C4CCD7}		% elegant: 优雅
\definecolor{ivory}{HTML}{F1ECE6}			% 象牙
\definecolor{gloomypruple}{HTML}{CCC1D2}	% 阴沉紫
% \textcolor[HTML]{FFC23A}					% 石板灰

\definecolor{Green}{rgb}{0,.8,0}

\usepackage{imakeidx}								% 索引

\usepackage{nomencl}								% 关键词
%\setlength{\nomitemsep}{0.2cm}							% 设置术语之间的间距
\renewcommand{\nomentryend}{.}							% 设置打印出术语的结尾的字符
\renewcommand{\eqdeclaration}[1]{见公式:(#1)}			% 设置打印见公式的样式
\renewcommand{\pagedeclaration}[1]{见第 (#1) 页}		% 设置打印页的样式
\renewcommand{\nomname}{术语表} 						% 修改术语表标题的名称。

\usepackage{array}
\usepackage{booktabs} % 三线表
\usepackage{multirow}