\chapter{曲线积分和曲面积分}
\begin{definition}{第一类、第二类曲线积分和曲面积分}{}
	第一类曲线积分
	\[
		\int_Lf(X)\d\ell=\int_\alpha^\beta f(x(t),y(t))\sqrt{x'(t)^2+y'(t)^2}\d t.
	\]
	第一类曲面积分
	\[
		\int_Sf(X)\d\sigma=\int_Df(x(u,v),y(u,v),z(u,v))\norm{\vec u\times\vec v}\d u\nd v.
	\]
	第二类曲线积分
	\[
		\int_A^B\vec F\cdot\d\vec\ell=\int_A^B X(x,y)\d x+Y(x,y)\d y.
	\]
	第二类曲面积分
	\[
		\int_{S^+}\vec V\cdot\d\vec S=\int_{S^+}X\d y\wedge\d z+Y\d z\wedge\d x+Z\d x\wedge\d y.
	\]
\end{definition}
第二类曲面积分可以化为第一类曲面积分
\[
	\int_{S^+}\vec V\cdot\d\vec S=\int_S\vec V\cdot\hat n\d\sigma.
\]
从计算方面来说
\[
	\int_{S^+}X\d y\wedge\d z+Y\d z\wedge\d x+Z\d x\wedge\d y=\pm\iint_DXA+YB+ZC\d u\nd v.
\]
\section{Green, Gauss, Stokes公式}
\begin{theorem}{Green公式}{}
	$X(x,y),Y(x,y)$在有界单连通闭区域$D$上连续可微
	\begin{align*}
		\color{green}\iint_D\pv Xx+\pv Yy\d x\nd y=\oint_{\p D}X\d y-Y\d x, \\
		\color{green}\iint_D\pv Yx-\pv Xy\d x\nd y=\oint_{\p D}X\d x+Y\d y.
	\end{align*}
\end{theorem}
\begin{theorem}{Gauss公式}{}
	$X(x,y),Y(x,y)$在有界单连通闭区域$D$上连续可微
	\begin{align*}
		\iiint_\Omega\nabla\cdot\vec V\d x\nd y\nd z=\iint_{\p\Omega}\vec V\cdot\d\vec S.
	\end{align*}
\end{theorem}
