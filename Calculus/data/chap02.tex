\chapter{含参积分及广义含参积分}
\begin{definition}{含参积分}{}
	含参积分
	\[
		I(y)=\int_a^bf(x,y)\d x,\quad y\in\fkh{c,d}.
	\]
	广义含参积分包括无穷积分和瑕积分,有统一形式
	\[
		I(y)=\int_a^\omega f(x,y)\d x\equiv\lim_{A\to\omega^-}\int_a^Af(x,y)\d x.
	\]
	无穷积分$\omega=+\infty$;瑕积分$\omega\in\RR$但$f$在$\omega$邻域内无界(奇点).
\end{definition}
\begin{definition}{一致连续}{}
	$\forall\varepsilon>0,\exists\delta>0$ 使 $\forall X,Y$ 满足 $\vert X-Y\vert<\delta$,有 $|f(X)-f(Y)|<\varepsilon.$
	\tcblower
	\paragraph{否定形式:}存在两点列$\{X_k\}\{Y_k\}$使得$\forall k\geqslant 1,|f(X_k)-f(Y_k)|\geqslant\varepsilon_0>0.$
\end{definition}
\begin{theorem}{一致连续的判定}{}
	函数$f\in C(\Omega)$在有界闭集$\Omega$上连续$\implies$一致连续.
\end{theorem}
\begin{definition}{一致收敛}{}
	$\forall\varepsilon>0,\exists N>a$ 使 $\forall A>N,\forall y\in[c,d]$,有 $\left|\int_a^Af(x,y)\d x-I(y)\right|<\varepsilon.$
\end{definition}
\begin{theorem}{一致收敛的Weierstrass判别法}{}
	\begin{center}
		$|f(x,y)|\leqslant F(x),\int_a^\omega F(x)\d x$收敛.
		\vthus
		$\int_a^\omega f(x,y)\d x$对$y$一致收敛.
	\end{center}
\end{theorem}
\begin{theorem}{一致收敛的Dirichlet判别法}{}
	\begin{center}
		$\int_a^Af(x,y)\d x$有界;\\
		$g(x,y)$关于$x$单调且$\lim_{x\to\omega^-}g(x,y)=0$
		\vthus
		$\int_a^\omega f(x,y)g(x,y)\d x$对$y$一致收敛.
	\end{center}
\end{theorem}
\begin{theorem}{一致收敛的Abel判别法}{}
	\begin{center}
		$\int_a^\omega f(x,y)\d x$对$y$一致收敛;\\
		$g(x,y)$关于$x$单调且有界
		\vthus
		$\int_a^\omega f(x,y)g(x,y)\d x$对$y$一致收敛.
	\end{center}
\end{theorem}
\section{积分号的可交换性}
\begin{theorem}{连续性}{}
	$f(x,y)$连续$\implies I(y)$连续,即
	\[
		\lim_{y\to y_0}\int_a^bf(x,y)\d x=\int_a^b\lim_{y\to y_0}f(x,y)\d x.
	\]
	\tcblower
	$f(x,y)$连续且$\int_a^\omega f(x,y)\d x$关于$y$一致收敛.$\implies I(y)$连续,即
	\[
		\lim_{y\to y_0}\int_a^\omega f(x,y)\d x=\int_a^\omega\lim_{y\to y_0}f(x,y)\d x.
	\]
\end{theorem}
\begin{theorem}{可微性}{}
	$f(x,y),\pv fy(x,y)$连续$\implies I(y)$可微,且
	\[
		I'(y)=\frac{\d}{\d y}\int_a^bf(x,y)\d x=\int_a^b\pp yf(x,y)\d x.
	\]
	\tcblower
	$f(x,y),\pv{f}{y}(x,y)$连续且$\int_a^\omega\pv{f}{y}(x,y)\d x$关于$y$一致收敛$\implies I(y)$可微,且
	\[
		I'(y)=\frac{\d}{\d y}\int_a^\omega f(x,y)\d x=\int_a^\omega\pp yf(x,y)\d x.
	\]
\end{theorem}
公式
\[
	\dv{}y\int_{\alpha(y)}^{\beta(y)}f(x,y)\d x=\int_{\alpha(y)}^{\beta(y)}\pv fy(x,y)\d x+f(\beta(y),y)\beta'(y)-f(\alpha(y),y)\alpha'(y).
\]
\begin{theorem}{积分性}{}
	$f(x,y)$连续$\implies I(y)$可积,且
	\[
		\int_c^dI(y)\d y=\int_c^d\int_a^b f(x,y)\d x\d y=\int_a^b\int_c^d f(x,y)\d y\d x.
	\]
	\tcblower
	$f(x,y)$连续且$\int_a^\omega f(x,y)\d x$关于$y$一致收敛$\implies I(y)$可积,且
	\[
		\int_c^dI(y)\d y=\int_c^d\int_a^\omega f(x,y)\d x\d y=\int_c^\omega\int_a^b f(x,y)\d x\d y
	\]
\end{theorem}
*对于含两个广义积分的交换条件更严格
\begin{compactenum}
	\item $f(x,y)$连续
	\item $\int_a^{+\infty}f(x,y)\d x,\int_c^{+\infty}f(x,y)\d y$分别关于$y\in[c,C],x\in[a,A]$一致收敛
	\item $\int_c^{+\infty}\int_a^{+\infty}|f(x,y)|\d x\d y,\int_a^{+\infty}\int_c^{+\infty}|f(x,y)|\d y\d x$至少一个存在
\end{compactenum}
则
\[
	\int_c^{+\infty}\int_a^{+\infty}f(x,y)\d x\d y=\int_a^{+\infty}\int_c^{+\infty}f(x,y)\d y\d x
\]

\begin{example}{Gamma函数}{}
	\begin{center}
		$\Gamma(x):=\int\zti t^{x-1}\e{-t}\d t.$
	\end{center}
	递推公式
	\begin{align*}
		\Gamma(1)   & =\int\zti\e{-t}\d t=1.                                              \\
		\Gamma(x+1) & =\int\zti t^x\cdot\e{-t}\d t                                        \\
					& =\edg{-t^x\e{-t}}\zti+\int\zti\e{-t}\cdot xt^{x-1}\d t=x\Gamma(x).
	\end{align*}
	对于$n\in\mathbb N,\Gamma(n)=(n-1)!.$
	% $\Gamma(x+1)=x\Gamma(x),\Gamma(1)=1$,对正整数$n$,
	% \[
	% 	\Gamma(n)=\int_0^{+\infty}t^{n-1}\e{-t}\d t=(n-1)!
	% \]

	\textbf{余元公式}
	\[
		\Gamma(x)\Gamma(1-x)=\frac{\pi}{\sin\pi x},\quad x\in(0,1).
	\]
	证明略,有$\Gamma(1/2)=\sqrt\pi$
\end{example}
\begin{example}{Beta函数}{}
	\begin{center}
		$\text B(p,q):=\int_0^1t^{p-1}(1-t)^{q-1}\d t.$ % =2\int_0^{\pi/2}\cos^{2p-1}\theta\sin^{2q-1}\theta\d\theta.$
	\end{center}
	与Gamma函数的关系为$\text B(p,q)=\frac{\Gamma(p)\Gamma(q)}{\Gamma(p+q)}.$
\end{example}
\begin{example}{Possion积分}{}
	\begin{center}
		$\int\zti\e{-t^2}\d t=\frac{\sqrt\pi}2.$
	\end{center}
	\textbf{证明:}考虑积分的平方
	\begin{align*}
		\int\zti\e{-x^2}\d x\int\zti\e{-y^2}\d y=\int\zti\int\zti\e{-x^2-y^2}\d x\nd y \\
		=\int_0^{\pi/2}\int\zti\e{-r^2}\cdot r\d r\nd\theta=\frac\pi{2}\cdot\frac12\int\zti\e{-s}\d s=\frac\pi{4}.
	\end{align*}
	\iffalse
		由$\ex>0,\forall x\in\RR$
		\[
			\iint_A\e{-x^2-y^2}\d x\nd y\leqslant\iint_B\e{-x^2-y^2}\d x\nd y\leqslant\iint_C\e{-x^2-y^2}\d x\nd y
		\]
		其中
		\begin{align*}
			B & =\{(x,y)~|~0\leqslant x\leqslant R,0\leqslant y\leqslant R\}, \\
			A & =\{(x,y)~|~x^2+y^2\leqslant R^2,x\geqslant 0,y\geqslant 0\},  \\
			C & =\{(x,y)~|~x^2+y^2\leqslant 2R^2,x\geqslant 0,y\geqslant 0\}.
		\end{align*}
		又
		\begin{align*}
			\iint_A\e{-x^2-y^2}\d x\nd y & =\int_0^{\pi/2}\int_0^R\e{-\rho^2}\rho\d\rho\d\varphi=\frac{\pi}4\left(1-\e{-R^2}\right), \\
			\iint_B\e{-x^2-y^2}\d x\nd y & =\int_0^R\int_0^R\e{-x^2}\e{-y^2}\d x\nd y=\left(\int_0^R\e{-t^2}\d t\right)^2,          \\
			\iint_C\e{-x^2-y^2}\d x\nd y & =\frac{\pi}4\left(1-\e{-2R^2}\right).
		\end{align*}
		当$R\to+\infty$,即得$\int\zti\e{-t^2}\d t=\frac{\sqrt\pi}2$.
	\fi
\end{example}
\begin{example}{Dirichlet积分}{}
	\begin{center}
		$\int\zti\frac{\sin x}x\d x=\frac{\pi}2.$
	\end{center}
	\textbf{证明:}令
	\[
		F(y)=\int\zti\frac{\sin x}x\e{-xy}\d x,\quad y\geqslant 0.
	\]
	由$\frac{\sin x}x\e{-xy}$可被延拓为$\RR_{\geqslant 0}^2$上的连续函数,故$F(y)$在$\RR_{\geqslant 0}$上连续.

	注意到,积分
	\[
		\int\zti\frac{\sin x}x\d x
	\]
	收敛;$\forall x,y\geqslant 0,|\e{-xy}|\leqslant 1$且$\e{-xy}$关于$x$递减,由Abel判别准则可知$F(y)$关于$y\in[0,+\infty)$一致收敛.

	任取$a>0,\forall x\geqslant 0,y\geqslant a,$可知$\left|-\sin x\e{-xy}\right|\leqslant \e{-xy}\leqslant \e{-ax}$有界,由Weierstrass判别准则知,积分
	\[
		\int\zti\pv {}y\frac{\sin x}x\e{-xy}\d x=-\int\zti\sin x\e{-xy}\d x
	\]
	一致收敛,因此$F(y)$在$[a,+\infty)$上可导
	\begin{align*}
		F'(y) & =-\int\zti\sin x\e{-xy}\d x=\int\zti\Im\e{-xy-\i x}\d x                                                \\
			  & =-\Im\left.\frac{\e{-(y+\i)x}}{y+\i}\right|\zti=-\left.\frac{\e{-yx}}{y^2+1}\Im(y-\i)\e{-\i x}\right|\zti \\
			  & =\left.\frac{\e{-yx}}{y^2+1}\left(y\sin x+\cos x\right)\right|\zti=-\frac1{1+y^2}.
	\end{align*}
	因为$a>0$任意,故$\forall y>0,$
	\[
		F'(y)=-\frac1{1+y^2},\quad F(y)=-\arctan y+C.
	\]
	又$y\to+\infty,$
	\[
		|F(y)|\leqslant\int\zti\frac{|\sin x|}x\e{-xy}\d x\leqslant\int\zti\e{-xy}\d x=\frac1y\to0.
	\]
	故$\lim_{y\to+\infty}F(y)=C-\frac{\pi}2=0$,即$C=\frac{\pi}2.$

	又$F(y)$在$[0,+\infty)$上连续,故$\forall y\geqslant 0,F(y)=\frac{\pi}2-\arctan y.$特别的
	\[
		F(0)=\int\zti\frac{\sin x}x\d x=\frac{\pi}2.
	\]
\end{example}
\newpage
