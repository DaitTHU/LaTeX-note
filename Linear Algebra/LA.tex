\documentclass{../thunote}

\theoremstyle{definition}
\newtheorem*{corollary}{推论}
\newtheorem*{remark}{注}
\newtheorem*{lemma}{引理}

%% text
\newcommand*{\vs}{~\text{-}~}
\newcommand*{\const}{\text{const}}
\newcommand*{\otherwise}{\text{otherwise}}
\newcommand*{\plusc}{{\color{lightgray}\,+\,\const}}

%% roman
\newcommand*{\e}{\mathop{}\!\mathrm{e}^}  % exp
\let\accenti\i
\renewcommand*{\i}{\mathrm{i}}

\usepackage{bm}
\newcommand{\hatbm}[1]{\hat{\bm{#1}}}  % bm with hat, maybe Fourier Transform
\newcommand{\nvec}[1]{\hat{\bm{#1}}}  % normalized vector, without space
\newcommand{\uvec}[1]{\mathop{}\!\nvec{#1}}  % unit vector, with possible space
\newcommand{\dotbm}[1]{\dot{\bm{#1}}}

\usepackage{mathrsfs}  % \mathscr

%% blackboard bold: number sets
\newcommand*{\NN}{\mathbb N}    % natural
\newcommand*{\ZZ}{\mathbb Z}    % integer
\newcommand*{\QQ}{\mathbb Q}    % rational
\newcommand*{\RR}{\mathbb R}    % real
\newcommand*{\CC}{\mathbb C}    % complex
\newcommand*{\FF}{\mathbb F}    % any number field
\newcommand*{\PP}{\mathbb P}    % prime

\usepackage{cancel}

%% vector operator
\let\divides\div
\newcommand*{\grad}{\nabla}         % gradient
\renewcommand*{\div}{\nabla\cdot}   % divergence
\newcommand*{\curl}{\nabla\times}   % curl
\newcommand*{\lap}{\nabla^2}        % Laplacian

%% differential
\let\accentd\d
\renewcommand*{\d}{\mathop{}\!\mathrm{d}}   % differential
\newcommand*{\nd}{\mathrm{d}}               % differential without space
\newcommand*{\vd}{\mathop{}\!\delta}    % delta: δ
\newcommand*{\D}{\Delta}                % Delta: Δ
\newcommand*{\p}{\partial}              % partial: ∂
\newcommand{\dd}[2][{}]{\frac{\nd^{#1}}{\nd{#2}^{#1}}}      % d/dx
\newcommand{\dv}[3][{}]{\frac{\nd^{#1}#2}{\nd{#3}^{#1}}}    % df/dx
\newcommand{\pp}[2][{}]{\frac{\p^{#1}}{\p{#2}^{#1}}}    % ∂/∂x
\newcommand{\pv}[3][{}]{\frac{\p^{#1}#2}{\p{#3}^{#1}}}  % ∂f/∂x
\newcommand{\pw}[3]{\frac{\p^2{#1}}{\p{#2}\p{#3}}}      % ∂^2f/∂x∂y

%% integral limits
\newcommand*{\zti}{_0^{+\infty}}            % zero to infinity
\newcommand*{\iti}{_{-\infty}^{+\infty}}    % -infinity to +infinity

% brackets with auto size
\newcommand{\abs}[1]{\left\lvert#1\right\rvert}     % absolute value: |x|
\newcommand{\norm}[1]{\left\lVert#1\right\rVert}    % norm: ||x||
\newcommand{\edg}[1]{\left.#1\right\rvert}          % edge line: f|
\newcommand{\kh}[1]{\left(#1\right)}                % parentheses: (x)
\newcommand{\fkh}[1]{\left[#1\right]}               % square brackets: [x]
\newcommand{\hkh}[1]{\left\{#1\right\}}             % braces: {x}
% \newcommand{\ang}[1]{\left\langle #1\right\rangle}  % angle brackets: <x>
\newcommand{\floor}[1]{\left\lfloor#1\right\rfloor} % floor
\newcommand{\ceil}[1]{\left\lceil#1\right\rceil}    % ceil
\newcommand{\ave}[1]{\left\langle #1\right\rangle}  % average: <x>
\newcommand{\set}[2]{\left\{#1\,\middle\vert\,#2\right\}}   % set: {x|x1,x2,...}
\newcommand{\bra}[1]{\left\langle #1\right\vert}    % bra: <ψ|
\newcommand{\ket}[1]{\left\vert #1\right\rangle}    % ket: |ψ>
\newcommand{\brkt}[2]{\left\langle #1\middle\vert #2\right\rangle}  % inner product of bra-ket: <φ|ψ>
\newcommand{\ktbr}[2]{\left\vert#1\right\rangle\hspace{-3pt}\left\langle #2\right\vert} % ket-bra: |ψ><φ|
\newcommand{\division}[2]{\left.{#1}\middle/{#2}\right.}    % division: A/B
\newcommand{\inp}[2]{\left\langle #1,#2\right\rangle}       % inner product: <u,v>

% brackets with fixed size
\newcommand{\nnorm}[1]{\lVert#1\rVert}
\newcommand{\nset}[2]{\{#1\,|\,#2\}}
\newcommand{\bigkh}[1]{\bigl(#1\bigr)}
\newcommand{\Bigkh}[1]{\Bigl(#1\Bigr)}
\newcommand{\biggkh}[1]{\biggl(#1\biggr)}
\newcommand{\bigfkh}[1]{\bigl[#1\bigr]}
\newcommand{\Bigfkh}[1]{\Bigl[#1\Bigr]}
\newcommand{\biggfkh}[1]{\biggl[#1\biggr]}

%% math operator
\let\Real\Re
\let\Imaginary\Im
\let\Re\relax
\let\Im\relax
\DeclareMathOperator{\Re}{Re}  % real part
\DeclareMathOperator{\Im}{Im}  % imaginary part
\DeclareMathOperator{\sech}{sech}
\DeclareMathOperator{\csch}{csch} 
\DeclareMathOperator{\arcsec}{arcsec}
\DeclareMathOperator{\arccot}{arccot} 
\DeclareMathOperator{\arccsc}{arccsc} 
\DeclareMathOperator{\arsinh}{arsinh} 
\DeclareMathOperator{\arcosh}{arcosh} 
\DeclareMathOperator{\artanh}{artanh} 
\DeclareMathOperator{\sinc}{sinc}
\DeclareMathOperator{\sgn}{sgn}     % sign function
\DeclareMathOperator{\id}{id}       % identity mapping
\DeclareMathOperator{\Res}{Res}     % residue
\DeclareMathOperator{\supp}{supp}   % support set

%% linear algebra
\DeclareMathOperator{\rank}{rank}   % rank
\DeclareMathOperator{\diag}{diag}   % diagonal
\DeclareMathOperator{\tr}{tr}       % trace
\newcommand*{\tp}{^\top}    % transpose: A^T
\newcommand*{\cj}{^\ast}    % conjugate: A*
\newcommand*{\dg}{^\dagger} % conjugate transpose/Hermite: A†
\newcommand*{\iv}{^{-1}}    % inverse: A^-1

%% physicists
\newcommand*{\Schr}{Schrödinger}
\newcommand*{\Legd}{Legendre}
\newcommand*{\deB}{de Broglie}
\newcommand*{\Rayl}{Rayleigh}
\newcommand*{\Lande}{Landé}

%% particles
\newcommand*{\elc}{\mathrm e}
\newcommand*{\pton}{\mathrm p}
\newcommand*{\nton}{\mathrm n}
\newcommand*{\mol}{\mathrm m}

%% physical constants/notation
\newcommand*{\NA}{N_{\mathrm A}}    % Avogadro constant
\newcommand*{\kB}{k_{\mathrm B}}    % Boltzmann constant
\newcommand*{\muB}{\mu_\mathrm B}   % Bohr magne
\newcommand*{\Ek}{E_{\mathrm k}}    % kinetic energy
\newcommand*{\FWHM}{\mathrm{FWHM}}  % full width at half maximum

%% subscript/superscript 
\newcommand*{\eff}{_\mathrm{eff}}   % effective
\newcommand*{\tot}{_\mathrm{tot}}   % total
\newcommand*{\maxi}{_\mathrm{max}}  % maximum
\newcommand*{\mini}{_\mathrm{min}}  % minimum

%% unit tag
\newcommand*{\lSI}{\tag{SI}}    % le SI
\newcommand*{\CGS}{\tag{CGS}}   % cm, g, s system

%% other
\newcommand*{\qqquad}{\qquad\quad}
\newcommand*{\qqqquad}{\qquad\qquad}

\newcommand{\notimplies}{\hspace{1ex}\not\hspace{-1ex}\implies}

\let\geq\geqslant
\let\leq\leqslant

\newcommand*{\avg}[1]{\overline{#1}}

\newcommand*{\bigo}{\mathcal O}     % big O notation
\newcommand*{\degree}{^\circ}       % degree

\newcommand{\fracdisp}[2]{\frac{\displaystyle #1}{\displaystyle #2}}

\newcommand{\lhkh}[1]{\left\{#1\right.} % left brace: {x


\newcommand*{\FF}{\mathbb F}
\newcommand*{\PP}{\mathbb P}
\newcommand*{\Lm}{\varLambda}
\newcommand{\inp}[2]{\left\langle #1,#2\right\rangle}  % inner product: <f,g>
\newcommand{\notimplies}{\hspace{1ex}\not\hspace{-1ex}\implies}
\DeclareMathOperator{\row}{row}
\DeclareMathOperator{\col}{col}
\DeclareMathOperator{\cof}{cof}
\let\oldC\C
\let\C\relax
\DeclareMathOperator{\C}{C}
\DeclareMathOperator{\N}{N}
% \DeclareMathOperator{\rem}{ref}
\DeclareMathOperator{\rref}{rref}
\DeclareMathOperator{\spn}{span}
\DeclareMathOperator{\adj}{adj}
\DeclareMathOperator{\tr}{tr}
\DeclareMathOperator{\im}{im}
\DeclareMathOperator{\ord}{ord}
\DeclareMathOperator{\sym}{sym}
\DeclareMathOperator{\GL}{GL}  % general linear
\DeclareMathOperator{\SL}{SL}  % special linear

\begin{document}

\title{线性代数\\Linear Algebra}
\maketitle

\frontmatter
\tableofcontents

\mainmatter
% \setcounter{section}{-1}

\chapter{热力学基本定律}

%\section*{宏观}

宏观物质可以用很少的量表征。这种特性源于:
\textit{宏观测量与原子时间尺度相比极其缓慢,与原子空间尺度相比十分粗糙。}
宏观体系忽略了系统内部每个粒子的具体运动,正如Anderson所说:\textbf{\textit{More is different.}}
而热力学便是\textit{唯象}地描述多粒子行为的宏观理论。

\section{热力学第零定律}

\begin{definition}{热力学系统}{thermal system}
	热力学系统(thermal system)是大量微观粒子组成的有限宏观体系。
\end{definition}

平衡态指宏观性质不随时间改变的状态。

\begin{theorem}{热力学第零定律:热平衡定律}{thermal equilibrium}
	若系统A和系统B热平衡,且系统A和系统C也热平衡,则B和C热平衡。
\end{theorem}

\begin{corollary}
	互为热平衡的体系有一共同的物理性质,称为温度$T$。
\end{corollary}

\begin{definition}{物态方程}{state equation}
	物态方程(state equation)是温度$T$与其它状态参量间的关系。
\end{definition}

\begin{example}
	{理想气体物态方程}{}
	理想气体(ideal gas)的压强$p$、体积$V$、温度$T$和物质的量$n$之间的关系为:
	\begin{equation}
		pV=nRT.
	\end{equation}
	其中$R=\SI{8.314}{J/K.mol}$是理想气体常数。
\end{example}

\begin{example}
	{Van der Waals气体物态方程}{}
	Van der Waals气体考虑了分子间的相互作用和分子体积:
	\begin{equation}
		\biggkh{p+a\frac{n^2}{V^2}}(V-nb)=nRT.
	\end{equation}
	其中$a$与分子间的相互作用有关,$b$与分子体积有关。
\end{example}

\section{热力学第一定律}

热力学系统在外界影响下,会从一个平衡态过渡到另一个平衡态,在这个过程中的任一时刻,系统的状态都不是平衡态。
但是如果这个过程中的变化速度足够慢,每一瞬时都可以无限接近平衡态,我们就可以当做平衡态去处理这个过程。

\begin{definition}{准静态过程}{quasistatic process}
	准静态过程(quasistatic process)指每一瞬时,系统状态都无限接近平衡态的过程。
\end{definition}

系统的能量包括内能$U$和整体运动能量。对于封闭系统,能量交换有功$W$和热量$Q$两种方式。准静态过程中,
\begin{align}
	\vd W=\sum_iY_i\d y_i,
\end{align}
其中$(Y_i,y_i)$分别是广义力和广义坐标,如$(-p,V),(\mu_0H,M)$等。

\begin{theorem}{热力学第一定律:能量守恒定律}{Energy Conservation Law}
	一个热力学系统的内能增量$\d U$等于外界对它所做的功$\vd W$与外界向它传递的热量$\vd Q$的和:
	\begin{align}
		\d U=\vd W+\vd Q.
	\end{align}
\end{theorem}

\begin{remark}
	如果系统是\textbf{绝热}($\vd Q\equiv 0$)的,我们便可以用机械功$\vd W$测量内能的变化$\D U$,通过指定基准态的内能$U_0$就可以得出任意状态的内能$U$。进而我们可以测量导热系统的传热$\vd Q$。
\end{remark}

\begin{definition}{热容}{heat capacity}
	定义热容(heat capacity)是物质在单位温度变化下所吸收或放出的热量:
	\begin{equation}
		C:=\lim_{\D T\to0}\frac{\D Q}{\D T}.
	\end{equation}
	比热容(specific heat capacity)是单位质量的热容。
\end{definition}

\begin{remark}
	显然,热容与过程相关,可定义等容热容$C_V$和等压热容$C_p$。
\end{remark}

内能标准全微分式:将$U$全微分式中各变量微分前的系数用可测量表达出来。
\begin{example}{静流体系统}{static fluid system}
	以$T,V$为变量
	\[
		\d U=\underset{C_V}{\underline{\kh{\pv UT}_V}}\d T+\kh{\pv UV}_T\d V.
	\]
	已知 
	\begin{align*}
		C_p&=\kh{\frac{\vd Q}{\d T}}_p=\kh{\frac{\d U+p\d V}{\d T}}_p=\kh{\pv UT}_p+p\kh{\pv VT}_p\\
		&=\underset{C_V}{\underline{\kh{\pv UT}_V}}+\underset{\text{target}}{\underline{\kh{\pv UV}_T}}\kh{\pv VT}_p+p\kh{\pv VT}_p.
	\end{align*}
	因此
	\begin{align}
		\d U=C_V\d T+\fkh{(C_p-C_V)\kh{\pv TV}_p-p}\d V.
	\end{align}
\end{example}
\section{热力学第二定律}
\begin{theorem}{热力学第二定律}{Second Law of Thermodynamics}
	宏观的自发过程是不过逆的。
	\begin{compactitem}
		\item Clausius表述:不可能把热量从低温物体传到高温物体,而不引起其它变化。
		\item Kelvin表述:不可能从单一热源吸热,使之完全变成有用功,而不引起其它变化。
	\end{compactitem}
\end{theorem}
\begin{theorem}{Carnot定理}{Carnot's Theorem}
	在相同高、低温热源之间工作的热机中,可逆机的效率最高:
	\begin{align}
		\eta=1-\frac{Q_2}{Q_1}=1-\frac{T_2}{T_1}.
	\end{align}
	可逆机效率只与热源温度有关,与工作物质无关。
\end{theorem}
\paragraph{热力学温标}借助Carnot机可实现绝对温标。
\begin{theorem}{Clausius不等式}{Clausius inequality}
	在热力学循环中,系统热的变化及温度之间的关系:
	\begin{equation}
		\oint\frac{\vd Q}T\leqslant 0.
	\end{equation}
	当且仅当为可逆热机时取等号,此过程定义为可逆过程。
\end{theorem}
进而定义可逆过程中的熵
\begin{align}
	\d S:=\frac{\vd Q}T.
\end{align}
热力学第二定律的熵表述:孤立系统的熵不减,熵是热运动混乱程度的量度。
\begin{example}{熵的计算}{Calculating Entropy}
	将质量相同而温度分别为$T_1$和$T_2$的两杯水在等压下绝热的混合,求熵变。

	\textbf{解:}终态温度$T=(T_1+T_2)/2$,第一杯水的熵变为
	\[
		\D S_1=\int_{T_1}^T\frac{C_p\d T}T=C_p\ln\frac{T_1+T_2}{2T_1},
	\]
	第二杯水的熵变$\D S_2$同理可求,
	总熵增
	\[
		\D S_1+\D S_2=C_p\ln\frac{(T_1+T_2)^2}{4T_1T_2}\geqslant 0.
	\]
	取等号当且仅当$T_1=T_2$。
\end{example}
\section{热力学第三定律}
\begin{theorem}{热力学第三定律}{Third Law of Thermodynamics}
	% Nernst定理:
	$T\to 0$时,等温过程的熵变$\D_TS\to 0$

	Nernst原理:不可能使一个物体冷却到绝对温度的零度。
\end{theorem}
\chapter{随机变量及其分布}

\begin{definition}{(一维)随机变量}{random variable}
	定义随机变量(random variable)~$X:\Omega\to\RR$是样本空间上的实值函数。有
	\begin{compactitem}
		\item 离散型(discrete):至多可数个取值;
		\item 连续型(continuous):区间型取值(不严格);
		\item 其他
	\end{compactitem}
\end{definition}
\begin{definition}
	{随机变量的概率}{}
	$\forall I\subset\RR$可测,记原像集$X\inv(I)\in\cF$,定义\footnote{这种定义属于好看的鱼(形式简洁),而不属于好吃的鱼(实用)。}
	\[
		\P_X(X\in I):=\P(X\inv(I)),\quad\forall I\subset\RR~\text{可测}
	\]
	一般记$\P_X$为$\P$。
\end{definition}
\begin{definition}{累计分布函数}{cumulative distribution function}
	记$X$的累计分布函数(cumulative distribution function, CDF)\index{CDF, 累计分布函数}
	\[
		\CDF(x):=\P(X\leqslant x),\quad\forall x\in\RR
	\]
	则$\P(a<X\leqslant b)\equiv\CDF(b)-\CDF(a)$。
\end{definition}

\begin{corollary}
	CDF的性质:
	\begin{itemize}
		\item $\CDF(x)$单调递增(不严格单调);
		\item $\lim_{x\to+\infty}\CDF(x)=1,\lim_{x\to-\infty}\CDF(x)=0;$
		\item $\CDF(x)$右连续,不一定左连续。
	\end{itemize}
\end{corollary}
\begin{remark}~
	\begin{compactenum}
		\item 随机要素来自样本点$\omega$的随机选择;
		\item $X,Y$同样本空间时,一般地,$aX+bY$等$X,Y$的函数也是随机变量;
		\item 随机变量同分布$\iff$ CDF相同;但不代表变量相同。
	\end{compactenum}
\end{remark}

\section{离散分布}

\begin{definition}{离散分布}{discrete distribution}
	离散分布可由分布列(probability distribution)表示概率在样本空间中的分布
	\begin{center}
		\begin{tabular}{cccccc}
			\toprule
			$X$&$x_1$&$x_2$&$\cdots$&$x_i$&$\cdots$\\
			\midrule
			$p$&$p_1$&$p_2$&$\cdots$&$p_i$&$\cdots$\\
			\bottomrule
		\end{tabular}
	\end{center}
	% 概率质量函数(probability mass function, PMF)\index{PMF, 概率质量函数}
	% \[
	% 	f(x)=\P(X=x),\quad\forall x\in\RR
	% \]
\end{definition}

\begin{corollary}
	离散分布的CDF为阶梯函数。
\end{corollary}

\begin{definition}{期望和方差}{expectation and variance}
	期望(expectation)即均值
	\begin{equation}
		\E(X):=\sum\nolimits_{i\in I}x_ip_i,
	\end{equation}
	方差(variance)
	\begin{equation}
		\Var(X):=\sum\nolimits_{i\in I}\bigkh{x_i-\E(X)}^2p_i\equiv\E(X^2)-\E(X)^2.
	\end{equation}
\end{definition}

\begin{corollary}
	随机变量$X$的函数$g(X)$的期望
	\[
		\E(g(X))=\sum\nolimits_{i\in I}g(x_i)p_i.
	\]
\end{corollary}

\begin{remark}
	期望存在要求级数绝对收敛。
\end{remark}

\paragraph{二项分布}

~

\begin{definition}{\Bern 分布}{Bernoulli distribution}
	\Bern 分布也称0-1分布,$p$为成功概率,记作$X\sim\Bino(p)$,其分布列为
	\begin{equation}
		\P(X=k)=\begin{cases}
			1-p,&k=0\\
			p,&k=1
		\end{cases}
	\end{equation}
	% \begin{center}
	% 	\begin{tabular}{ccc}
	% 		\toprule
	% 		$X$&0&1\\
	% 		\midrule
	% 		$p$&$1-p$&$p$\\
	% 		\bottomrule
	% 	\end{tabular}
	% \end{center}
\end{definition}

\begin{definition}{二项分布}{binominal distribution}
	$n$次独立\Bern 试验的成功次数$X$服从二项分布(binominal distribution),记作$X\sim\Bino(n,p)$,
	\begin{equation}
		\P(X=k)=\binom nk p^k(1-p)^{n-k},\quad k=0,\ldots,n
	\end{equation}
	\begin{center}
		\includegraphics[width=.95\textwidth]{figures/pdf_bin.pdf}
		\captionof{figure}{不同$n,p$下的二项分布}
	\end{center}
\end{definition}

\begin{corollary}
	% 显然\Bern 分布就是二项分布的特例。
	二项分布的期望和方差为
	\begin{subequations}
		\begin{align}
			\E(X)&=np,\\
			\Var(X)&=np(1-p).
		\end{align}
	\end{subequations}
\end{corollary}

\paragraph{Poisson分布}

我们考虑一个时间段内的事件次数$X\sim\Bino(n,p)$,其中$n$是时间段的切片份数。
保持其期望$\E(X)=np=:\lambda$不变,令$n\to\infty$,即事件可能发生在任一时刻,则
\[
	\lim_{n\to\infty}\binom nk\kh{\frac\lambda{n}}^k\kh{1-\frac\lambda{n}}^{n-k}=\lim_{n\to\infty}\frac{\lambda^k}{k!}\kh{1-\frac\lambda{n}}^n\cancel{\frac{n!}{n^k(n-k)!}}\cancel{\kh{1-\frac\lambda{n}}^{-k}}=\frac{\lambda^k}{k!}\e{-\lambda}.
\]
% 我们便得到了一个新的分布,称为Poisson分布。

\begin{definition}{Poisson分布}{Poisson distribution}
	小概率事件在一段时间内发生的次数$X$服从Poisson分布,记作$X\sim\Pois(\lambda)$,
	\[
		\P(X=k)=\frac{\lambda^k}{k!}\e{-\lambda},\quad k=0,1,2,\ldots
	\]
	\begin{center}
		\includegraphics[width=.9\textwidth]{figures/pdf_poission.pdf}
		\captionof{figure}{不同$\lambda$下的Poisson分布}
	\end{center}
\end{definition}
\begin{corollary}
	Poisson分布的期望和方差为
	\begin{subequations}
		\begin{align}
			\E(X)&=\lambda,\\
			\Var(X)&=\lambda.
		\end{align}
	\end{subequations}
\end{corollary}

\begin{theorem}{Poisson分布近似二项分布}{}
	用$\Pois(np)$近似$\Bino(n,p)$的误差最多为$\min(p,np^2)$。
\end{theorem}

\begin{remark}
	若试验不独立,但满足弱相依条件下,Poisson分布仍为较好近似。
\end{remark}

\begin{example}{弱相依条件举例:配对问题}{weak dependence condition}
	在\exmref{exm:derangement} 中,尽管$A_i$和$A_j$并不独立,但弱相依
	\[
		\P(A_i)=\frac1n\approx\P(A_i|A_j)=\frac1{n-1}.
	\]
	记$X=$拿到自己帽子的人数,当$n\to\infty$时,$X\sim\Pois(1)$
	\[
		\P(X=k)=\frac{\e{-1}}{k!},\quad k=0,1,2,\ldots
	\]
	\tcblower
	下面用常规做法检验:指定$k$个人,记
	\begin{itemize}
		\item $E=$这$k$个人都拿到自己的帽子;
		\item $F=$余下$n-k$个人都未拿到自己的帽子。
	\end{itemize}
	则$\P(F|E)$就是$n-k$个人的错排问题,根据\exmref{exm:derangement} 中的结果:
	\[
		\P(E\cap F)=\P(E)\P(F|E)=\frac{(n-k)!}{n!}\cdot\frac{!(n-k)}{(n-k)!},
	\]
	对于$\P(X=k)$来说,由于这$k$个人是任选的,故
	\[
		\P(X=k)=\binom nk\P(E\cap F)=\frac1{k!}\cdot\frac{!(n-k)}{(n-k)!}\to\frac{1}{k!}\cdot\frac1{\e{}}.
	\]
\end{example}

\section{连续分布}

\begin{definition}{连续分布}{continuous distribution}
	若存在$f(x)\geqslant 0$,使得$\forall I\subset\RR$可测,都有
	\[
		\P(X\in I)=\int_If(x)\d x
	\]
	则称$X$为连续型随机变量,服从连续分布,$f(x)$为概率密度函数(probability density function, PDF)。\index{PDF, 概率密度函数}
\end{definition}

\begin{corollary}
	连续分布的性质:
	\begin{itemize}
		\item $\forall x\in\RR,\enspace\P(X=x)\equiv 0;$
		\item 归一性:
			\begin{equation}
				\int\iti f(x)\d x\equiv 1;
			\end{equation}
		\item 期望和方差(要求积分\textbf{绝对收敛})
			\begin{subequations}
				\begin{align}
					\E(X)&=\int\iti xf(x)\d x,\\
					\Var(X)&=\int\iti(x-\E(X))^2f(x)\d x.
				\end{align}
			\end{subequations}
		\item 连续分布的CDF $\CDF(x)$连续且可导,且
			\begin{equation}
				\CDF'(x)=f(x).
			\end{equation}
			若$\CDF(x)$严格递增,则$\CDF^{-1}(y)$存在;但若其不严格递增,也可well define
			\begin{equation}
				\label{eq:CDF-1}
				\CDF^{-1}(y):=\inf\set x{\CDF(x)\geqslant y}.
			\end{equation}
	\end{itemize}
\end{corollary}
\iffalse
\begin{example}{常见连续分布}{}
	\begin{compactenum}
		\item 均匀分布$f(x;a,b)=\frac1{b-a},a\leqslant x\leqslant b;$
		\item 指数分布$f(x;\lambda)=\lambda\e{-\lambda x},x\geqslant 0;$
		\item 正态分布$f(x;\mu,\sigma^2)=\frac1{\sqrt{2\pi}\sigma}\e{-(x-\mu)^2/2\sigma^2},-\infty<x<\infty;$
		\item 伽玛分布$f(x;\alpha,\lambda)=\frac{\lambda^\alpha}{\Gamma(\alpha)}x^{\alpha-1}\e{-\lambda x}.x\geqslant 0;$
		\item 卡方分布$f(x;n)=\frac{x^{n/2-1}}{2^{n/2}\Gamma(n/2)}\e{-x/2},x\geqslant 0;$
	\end{compactenum}
\end{example}
\fi
\begin{definition}{均匀分布}{uniform distribution}
	均匀分布(uniform distribution)记作$X\sim\Unif(a,b)$,%随机数$\Unif(0,1)$
	\begin{equation}
		f(x)=\begin{cases}
			\frac1{b-a},&x\in(a,b)\\
			0,&\text{elsewhere}
		\end{cases}
	\end{equation}
	特别地,服从标准均匀分布的随机变量$X\sim\Unif(0,1)$也称为随机数。
\end{definition}
\begin{corollary}
	均匀分布的期望和方差为
	\begin{subequations}
		\begin{align}
			\E(X)&=\frac{a+b}2,\\
			\Var(X)&=\frac{(b-a)^2}{12}.
		\end{align}
	\end{subequations}
\end{corollary}
\begin{definition}{正态分布}{normal distribution}
	正态分布(normal distribution)记作$X\sim\Norm(\mu,\sigma^2)$,%标准正态$\Norm(0,1)$
	\begin{equation}
		f(x)=\frac1{\sqrt{2\pi}\sigma}\exp\biggfkh{-\frac{(x-\mu)^2}{2\sigma^2}},\quad x\in\RR
	\end{equation}
	特别地,标准正态分布$\Norm(0,1)$的CDF记作$\Phi(x)$。
	\begin{center}
		\includegraphics[width=.9\textwidth]{figures/pdf_normal.pdf}
		\captionof{figure}{不同$\mu,\sigma$下的正态分布}
	\end{center}
\end{definition}
\begin{corollary}
	正态分布的期望和方差为
	\begin{subequations}
		\begin{align}
			\E(X)&=\mu,\\
			\Var(X)&=\sigma^2.
		\end{align}
	\end{subequations}
\end{corollary}
\begin{definition}{指数分布}{exponential distribution}
	指数分布(exponential distribution)通常刻画寿命、等待时间,记作$X\sim\Expo(\lambda)$,
	\begin{equation}
		f(x)=\begin{cases}
			\lambda\e{-\lambda x},&x>0\\
			0,&x\leqslant 0
		\end{cases}
	\end{equation}
	\begin{center}
		\includegraphics[width=.9\textwidth]{figures/pdf_exp.pdf}
		\captionof{figure}{不同$\lambda$下的指数分布}
	\end{center}
\end{definition}

\begin{corollary}
	指数分布的期望和方差为
	\begin{subequations}
		\begin{align}
			\E(X)&=\frac1\lambda,\\
			\Var(X)&=\frac1{\lambda^2}.
		\end{align}
	\end{subequations}
	其期望往往被定义为平均寿命$\beta:=\E(X)\equiv 1/\lambda$,因此也有教材将指数分布记作$\Expo(\beta)$。
\end{corollary}

\begin{corollary}
	指数分布的尾概率与Poisson分布有关:
	\[
		\P_{\Expo(\lambda)}(T>\tau)=\e{-\lambda\tau}=\P_{\Pois(\lambda\tau)}(N=0).
	\]
	这说明:如果时间$(0,\tau)$内事件的发生次数$N\sim\Pois(\lambda\tau)$,则相邻事件的时间间隔$\D T\sim\Expo(\lambda)$。
\end{corollary}

\begin{example}{指数分布的导出}{}
	产品在$(x,x+\d x)$时间内的失效率为
	\[
		\P(x<X<x+\d x|X>x)=\frac{\P(x<X<x+\d x)}{\P(X>x)}=\frac{\CDF(x+\d x)-\CDF(x)}{1-\CDF(x)}=\frac{\CDF'(x)}{1-\CDF(x)}\d x
	\]
	令失效率为$\lambda(x)\d x$
	\[
		\frac{\CDF'(x)}{1-\CDF(x)}=\lambda(x),\implies\CDF(x)=1-\exp\biggfkh{-\int_0^x\lambda(t)\d t}.
	\]
	若假设无老化:$\lambda(t)\equiv\lambda$,则分布为指数分布:
	\begin{equation}
		\CDF(x)=1-\e{-\lambda x},\quad x>0,
	\end{equation}

	这体现出指数分布的无记忆性:$\forall\tau>0$
	\[
		\P(X>t+\tau|X>\tau)=\frac{\e{-\lambda(t+\tau)}}{\e{-\lambda\tau}}=\e{-\lambda t}=\P(X>t)
	\]

	\tcblower
	
	改进$\lambda(x)=\alpha x^{\alpha+1}/\beta^\alpha$,得到Weibull分布
	\begin{equation}
		\label{eq:Weibull CDF}
		\CDF(x)=1-\e{-(x/\beta)^\alpha},\quad x>0.
	\end{equation}
\end{example}

\section{随机变量的函数}

\begin{theorem}
	{离散型随机变量的函数}{}
	$X$离散$\implies Y:=g(X)$离散。
\end{theorem}

\begin{theorem}{连续型随机变量的函数}{}
	若$g$处处可导且严格单调,则$Y=g(X)$的PDF为
	\begin{equation}
		f_Y(y)=f_X\bigkh{g^{-1}(y)}\abs{\fkh{g^{-1}(y)}'}.
	\end{equation}
	其本质是
	\begin{equation}
		\CDF_Y(y)=\CDF_X\bigkh{g^{-1}(y)}.
	\end{equation}
\end{theorem}
\begin{example}{生成随机变量}{}
	服从CDF $\CDF(y)$的随机变量$Y$可由随机数$X\sim\Unif(0,1)$生成:
	\begin{equation}
		Y=\CDF\inv(X).
	\end{equation}
\end{example}
\section*{Review}
\begin{compactenum}
	\item PMF/PDF, CDF
	\item 期望$\mu$、标准差$\sigma$;标准化$\frac{X-\mu}\sigma$
	\item 参数的意义:位置、尺度、形状
	\item $Y=g(X)$
\end{compactenum}
\chapter{Bose系统和Fermi系统}
理想气体满足非简并条件$\e\alpha\gg 1$,即
\[
	\e\alpha=\frac1n\kh{\frac{2\pi m\kB T}{h^2}}^{3/2}=\frac1{n\lambda_T^3}\gg 1,
\]
对应$\kB T$时的de Broglie波长
\begin{align}\label{T-deB}
	\lambda_T=\frac{h}{\sqrt{2\pi m\kB T}}.
\end{align}
而气体不满足非简并条件,$n\lambda_T^3\not\ll 1$,其分布就是Bose分布和Fermi分布%,对应平衡分布
\begin{align}
	a_i=\frac{\omega_i}{\e{\alpha+\beta\varepsilon_i}\pm 1},
\end{align}
$\pm$号的($+$)对应Fermi分布,($-$)对应Bose分布。
\begin{definition}{巨配分函数}{Grand Partition Function}
	定义巨配分函数的对数
	\begin{align}
		\ln\Xi(\alpha,\beta,y):=\pm\sum\omega_i\ln\kh{1\pm\e{-\alpha-\beta\varepsilon_i}},
	\end{align}
\end{definition}
可得
\begin{gather}
	N=-\pv{\ln\Xi}{\alpha};\\ % =\sum\frac{\omega_i}{\e{\alpha+\beta\varepsilon_i}\pm 1}
	E=-\pv{\ln\Xi}{\beta}.
\end{gather}
物态方程
\begin{align}
	Y_k=-\frac1\beta\pv{\ln\Xi}{y_k}.
\end{align}

再来确定熵,由
\[
	\d E=T\d S+\sum Y_k\d y_k+\mu\d N,
\]
于是
\[
	T\d S=-\d\kh{\pv{\ln\Xi}\beta}+\frac1\beta\sum\pv{\ln\Xi}{y_k}\d y_k-\mu\d N.
\]
利用
\[
	\d\ln\Xi=\pv{\ln\Xi}\alpha\d\alpha+\pv{\ln\Xi}\beta\d\beta+\sum\pv{\ln\Xi}{y_k}\d y_k
\]
消去求和项,可得
\[
	T\d S=\frac1\beta\d\kh{\ln\Xi-\alpha\pv{\ln\Xi}\alpha-\beta\pv{\ln\Xi}\beta}-\kh{\mu+\frac\alpha\beta}\d N.
\]
对封闭系统,$\d N\equiv 0$
\[
	\d S=\frac1{\beta T}\d\kh{\ln\Xi-\alpha\pv{\ln\Xi}\alpha-\beta\pv{\ln\Xi}\beta},
\]
系数对应Boltzmann常数$\kB$,
因而对开放系统
\begin{align}\label{Bose-mu}
	\alpha=-\beta\mu=-\frac\mu{\kB T}.
\end{align}
因此熵
\begin{align}
	S=\kB\kh{\ln\Xi-\alpha\pv{\ln\Xi}\alpha-\beta\pv{\ln\Xi}\beta}{\color[gray]{.8}-\,S'},
\end{align}

另一方面,由Boltzmann关系
\begin{align*}
	S & =\kB\ln\Om[F;B]\simeq\kB\sum\fkh{a_i\ln\kh{\frac{\omega_i}{a_i}\mp 1}\mp\omega_i\ln\kh{1\mp\frac{a_i}{\omega_i}}} \\
	  & =\kB\sum\fkh{a_i\kh{\alpha_\beta\varepsilon_i}\pm\omega_i\ln\kh{1\pm\e{-\alpha-\beta\varepsilon_i}}}                             \\
	  & =\kB\kh{\alpha N+\beta E+\ln\Xi}=\kB\kh{\ln\Xi-\alpha\pv{\ln\Xi}\alpha-\beta\pv{\ln\Xi}\beta}.
\end{align*}
因此$S'=0$。
\section{弱简并理想Bose气体和Fermi气体}
弱简并条件$n\lambda_T^3<1$,宏观量可对$n\lambda_T^3\equiv\e{-\alpha}$展开。\exmref{exm:Monatomic Molecule} 已给出单原子气体平动:
\begin{align}
	g(\varepsilon)\d\varepsilon=g_s\frac{2\pi V}{h^3}(2m)^{3/2}\sqrt\varepsilon\d\varepsilon,
\end{align}
因此弱简并单原子Bose气体和Fermi气体的巨配分函数的对数
\begin{align*}
	\ln\Xi(\alpha,\beta,V) & =\pm\int\zti g(\varepsilon)\ln\kh{1\pm\e{-\alpha-\beta\varepsilon}}\d\varepsilon\\
	&=\pm 2\pi g_sV\kh{\frac{2m}{h^2}}^{3/2}\int\zti\sqrt\varepsilon\ln\kh{1\pm\e{-\alpha-\beta\varepsilon}}\d\varepsilon,
\end{align*}

由于$\e{-\alpha}<1$,用展开式
\[
	\ln(1+ x)=\sum_{n=1}^\infty\frac{x^n}n,\quad\abs x<1,
\]
展开
\begin{align*}
	\int\zti\sqrt\varepsilon\ln\kh{1\pm\e{-\alpha-\beta\varepsilon}}\d\varepsilon=\pm\sum_{n=1}^\infty\frac{(\mp)^{n-1}}n\int\zti\sqrt\varepsilon\e{-n(\alpha+\beta\varepsilon)}\d\varepsilon \\
	=\pm\sum_{n=1}^\infty\frac{(\mp )^{n-1}}n\e{-n\alpha}\frac{\sqrt\pi}{2(n\beta)^{3/2}}=\pm\frac{\sqrt\pi}{2\beta^{3/2}}f(\alpha).
\end{align*}
其中
\[
	f(\alpha)=\sum_{n=1}^\infty(\mp)^{n-1}n^{-5/2}\e{-n\alpha}=\e{-\alpha}\mp 2^{-5/2}\e{-2\alpha}+\cdots.
\]
故
\begin{align}
	\ln\Xi=\pm g_sV\kh{\frac{2\pi m}{h^2\beta}}^{3/2}f(\alpha)=\frac{g_sV}{\lambda_T^3}f(\alpha).
\end{align}

反解出$\alpha$
\[
	N=-\pv{\ln\Xi}\alpha=-\frac{g_sV}{\lambda_T^3}f'(\alpha),
\]
因此
\[
	\xi:=\frac{n\lambda_T^3}{g_s}=-f'(\alpha)=\e{-\alpha}\mp 2^{-3/2}\e{-2\alpha}+\cdots
\]
进而
\(\e{-\alpha}=\xi\pm{2^{-3/2}}\xi^2+\cdots,\)
%\[\alpha=-\ln\frac{n\lambda_T^3}{g_s}\mp\e{-3/2}\frac{n\lambda_T^3}{g_s}+\cdots\]
\[
	f(\alpha)=\xi\pm 2^{-5/2}\xi^2+\cdots.
\]
宏观量
\begin{align}\notag
	E & =-\pv{\ln\Xi}\beta=-\ln\Xi\pv{\ln\ln\Xi}\beta=\frac32\frac{\ln\Xi}\beta \\
	  & =\frac32N\kB T\kh{1\pm 2^{-5/2}\xi+\cdots}.
\end{align}
比热
\begin{align}
	C_V=\kh{\pv ET}_V=\frac32N\kB\kh{1\mp 2^{-7/2}\xi+\cdots}.
\end{align}
物态方程
\begin{align}
	p=\frac1\beta\pv{\ln\Xi}V=\frac{\ln\Xi}{\beta V}=\frac23\frac EV=\frac{N\kB T}V\kh{1\pm 2^{-5/2}\xi+\cdots}.
\end{align}
熵
\begin{align}\notag
	S&=\kB\kh{\ln\Xi-\alpha\pv{\ln\Xi}\alpha-\beta\pv{\ln\Xi}\beta}=\kB\kh{\frac53\beta E+N\alpha}\\
	&=N\kB\fkh{\kh{\frac52-\ln\xi}\pm 2^{-7/2}\xi+\cdots}.
\end{align}
\paragraph{讨论}弱简并条件($n\lambda_T^3<1$)下,$E,p,S$:Fermi $>$半经典$>$ Bose;$C_V$反之。而强简并条件下,Bose气体和Fermi气体性质完全不同。
\section{Bose-Einstein凝聚}
Bose气体的化学势满足
\[
	a_i=\frac{\omega_i}{\e{\beta(\varepsilon_i-\mu)}-1}.
\]
$a_i\geqslant 0$,故$\varepsilon_i\geqslant \mu$。
取$\varepsilon_0=0$,则$\mu\leqslant 0$
\[
	N=\sum_i\frac{\omega_i}{\e{\beta(\varepsilon_i-\mu)}-1}.
\]
随着温度的降低,化学势增加。直到相变点$T_\crt$,$\mu=0$.

计算$T_\crt$,单原子分子能量准连续
\begin{align*}
	N&=\int\zti\frac{g(\varepsilon)\d\varepsilon}{\e{\beta_\crt\varepsilon}-1}=2\pi g_sV\kh{\frac{2m}{h^2}}^{3/2}\int\zti\frac{\sqrt\varepsilon\d\varepsilon}{\e{\beta_\crt\varepsilon}-1}\\
	&=2\pi g_sV\kh{\frac{2m}{h^2\beta_\crt}}^{3/2}\cdot\Gamma\kh{\frac32}\zeta\kh{\frac32}.
\end{align*}
由
% \[\int\zti\frac{x^{n-1}}{\e x-1}\d x=\Gamma(n)\zeta(n).\]
$\Gamma(3/2)=\sqrt\pi/2,\;\zeta(3/2)=2.612$可得
\begin{align}\label{Bose-TC}
	T_\crt=\frac{h^2}{2\pi m\kB}\kh{\frac n{2.612g_s}}^{2/3}.
\end{align}
$T\to T_\crt$时,$\mu\to0$,基态上的粒子数显著增加;另一方面,准连续近似时$g(\varepsilon)\propto\sqrt\varepsilon$忽略了$\varepsilon=0$态。

故激发态中应将$N$分为基态$N_0$和激发态$N_+$两部分,$N_+$部分推导与之前相同
\[
	\frac{N_+}N=\kh{\frac{\beta_\crt}\beta}^{3/2}=\kh{\frac T{T_\crt}}^{3/2}.
\]
因此基态
\begin{align}
	N_0=N\fkh{1-\kh{\frac T{T_\crt}}^{3/2}}.
\end{align}
当$T<T_\crt$降低时,$N_0$不断增多;$T\to 0$时$N_0\to N$,越来越多的粒子处于基态,称为\textbf{Bose-Einstein凝聚}。这个凝聚可看做动量空间的凝聚。

\paragraph{凝聚后的宏观现象}$\varepsilon=0$粒子
\[
	E=0,\;p=0,\;G=N\mu=E+pV-TS=0.
\]
对$E$等无贡献,起粒子源作用;宏观量子态。

$\varepsilon>0$粒子的贡献,注意$\alpha=0$
\begin{align}\notag
	\ln\Xi&=-\int\zti g(\varepsilon)\ln\kh{1-\e{-\beta\varepsilon}}\d\varepsilon\\\notag
	%=\frac23CV\beta^{-3/2}\int\zti\frac{x^{3/2}\d x}{\e{-x}-1}
	&=2\pi g_sV\kh{\frac{2m}{h^2\beta}}^{3/2}\cdot\frac23\int\zti\frac{x^{3/2}\d x}{\e x-1}\\
	&=\zeta\kh{\frac52}\cdot g_sV\kh{\frac{2\pi m}{h^2\beta}}^{3/2}.% \cdot\frac{3\sqrt\pi}4\cdot 1.341.
\end{align}
故
\begin{gather}
	E=-\pv{\ln\Xi}\beta=0.770N\kB T\kh{\frac T{T_\crt}}^{3/2},\\
	p=\frac1\beta\pv{\ln\Xi}V\propto m^{3/2}g_sT^{5/2},\\
	S=\kB\kh{\ln\Xi+N\alpha+\beta E}\propto m^{3/2}g_sVT^{3/2},\\
	C_V=\kh{\pv ET}_V=1.926N\kB\kh{\frac T{T_\crt}}^{3/2}.
\end{gather}
\paragraph{讨论:}
\begin{compactenum}
	\item $T\to 0$时,$E,p,S\to0$
	\item $C_V$在相变点前后的变化
	\begin{center}
		\begin{tikzpicture}
			\coor 5{4.5}{T/T_\crt}{C_V/N\kB};
			\draw[thick,dashed](0,2*1.926)node[left]{1.926}--(1,2*1.926)--(1,0)node[below]{1};
			\draw[thick,dashed](0,3)node[left]{1.5}--(5,3);
			\draw[thick,domain=0:1]plot(\x,{2*1.926*\x^1.5});
			\draw[thick,domain=1:5]plot(\x,{3+(2*1.926-3)/(\x^1.5)});
		\end{tikzpicture}
		\captionof{figure}{热容$C_V$随温度的变化}
	\end{center}
	\item $p\vs  V$
		\subitem 半经典极限$pV=N\kB T$;
		\subitem 凝聚时$p\propto T^{3/5}$与$V$无关。
	\item 凝聚体积$V_\crt$,由式\eqref{Bose-TC}知
	\[
	T=\frac{h^2}{2\pi m\kB}\kh{\frac{N/V_\crt}{2.612g_s}}^{2/3}.
\]
	因此 
	\begin{align}
		V_\crt=\frac{N}{2.612g_s}\kh{\frac{h^2}{2\pi m\kB T}}^{3/2}=\frac{N\lambda_T^3}{2.612g_s}.
	\end{align}
\end{compactenum}
由于历史条件,当时还不知道全同多粒子系存在(量子起源的)统计关联:对Bose子是有效吸
引;而Fermi子是有效排斥。因此,即使没有动力学相互作用,仍可在一定条件下由于有效相互
作用而发生凝聚现象。这是一种纯粹量子起源的相变。

实现Bose-Einstein凝聚极其困难,原则上要使气体冷却至
$\lambda_T\geqslant\avg d$,
但大多数情况下,在远高于BEC的$T_\crt$到达以前,已发生液化甚至固化的相变。为了实现原子气体的BEC,必须用极稀薄的气体,且要求
\begin{center}
	二体弹性碰撞的弛豫时间$\ll$形成分子集团的非弹性碰撞的弛豫时间
\end{center}
对于碱金属原子气体,前者$\sim\SI{10}\ms$,而后者有几秒至几分钟。%$\tau_\mathrm{elas}\ll$$\tau_\mathrm{inelas}$

BEC-BCS Crosssover Fermionic condensation.
\paragraph{液He}$T_\crt=\SI{2.17}\K$,$T<T_\crt$时的液He II具有超流性。

$T=T_\crt$时,比热趋于无穷,$C_T\vs T$曲线形似$\lambda$,故称$\lambda$相变。
\section{光子气体}
光子是一种特殊的Bose子,严格来说,光子没有Bose-Einstein凝聚\footnote{广义上来说,赋予光子以质量是可以发生BEC的。}。讨论黑体辐射,$T,V$给定,满足相对论关系
\begin{align}
	\varepsilon=h\nu=cp.
\end{align}
光子间无相互作用,符合理想气体。%光子自旋$s=1$,简并度$g_s=2$;
光子质量为0,因此$\lambda_T\to\infty$,且光子数不守恒,没有$\alpha$
\[
	a_i=\frac{\omega_i}{\e{\beta\varepsilon_i}-1}.
\]
\paragraph{黑体辐射公式}能完全吸收照射到它上面的各种波长的电磁波的物体,称为黑体。当$V$很大时,能量准连续,$(\nu,\nu+\d\nu)$内状态数
\[
	g(\nu)\d\nu=\frac{g_sV}{h^3}4\pi\kh{\frac{h\nu}c}^2\frac{h\d\nu}{c}=\frac{4\pi g_sV}{c^3}\nu^2\d\nu;
\]
光子数 
\[
	n(\nu)\d\nu=\frac{g(\nu)\d\nu}{\e{\beta h\nu}-1},
\]
光子$g_s=2$,能量 
\begin{align}
	u(\nu)\d\nu=\frac{n(\nu)}Vh\nu\d\nu=\frac{8\pi\nu^2}{c^3}\frac{h\nu}{\e{\beta h\nu}-1}\d\nu.
\end{align}
上式即Planck定律。

低频高温下,$h\nu\ll\kB T$,变为经典的\Rayl-Jeans定律
\[
	u(\nu)\d\nu\simeq\frac{8\pi\nu^2}{c^3}\kB T\d\nu.
\]
高频低温极限,变成Wein定律
\[
	u(\nu)\d\nu\simeq\frac{8\pi h\nu^3}{c^3}\e{-\beta h\nu}\d\nu.
\]

辐射场总能量
\[
	u=\int\zti u(\nu)\d\nu=\frac{8\pi\kB^4}{h^3c^3}\int\zti\frac{x^3\d x}{\e x-1}=\frac{8\pi^5\kB^4}{15h^3c^3}T^4.
\]
辐射通量密度
\begin{align}
	J=\frac c4u=\sigma T^4.
\end{align}
其中$\sigma=\SI{5.6704e-8}{\W\per\m\squared\per\K\squared}.$及Stefan-Boltzmann定律。

若将能量密度按波长分布
\[
	u(\lambda)\d\lambda=\frac{8\pi h}{\lambda^3}\frac{1}{\e{\beta hc/\lambda}-1}\frac c{\lambda^2}\d\lambda.
\]
其极大值满足Wein位移定律
\begin{align}
	\lambda_\mathrm mT=\frac{hc}{4.96\kB}=\SI{2.89777}{\mm\K}.
\end{align}
\paragraph{热力学}
\[
	g(\varepsilon)\d\varepsilon=2\cdot\frac{4\pi V}{h^3}\frac{\varepsilon^2\d\varepsilon}{c^3}.
\]
配分函数
\begin{align*}
	\ln\Xi(\beta,V)&=-\int\zti g(\varepsilon)\ln\kh{1-\e{-\beta\varepsilon}}\d\varepsilon\\
	&=-\frac{8\pi V}{h^3c^3\beta^3}\int\zti x^2\ln(1-\e x)\d x=\frac{8\pi^5V}{45h^3c^3\beta^3}.
\end{align*}

能量
\[
	E=-\pv{\ln\Xi}\beta=\frac{8\pi^5V}{15h^3c^3\beta^4}=:bVT^4.
\]
与前面一致。而比热
\[
	C_V=\kh{\pv ET}_V=4bVT^3,
\]
随着温度上升而增加,因为光子数不守恒。

压强
\[
	p=\frac1\beta\pv{\ln\Xi}V=\frac13\frac EV=\frac13bT^4.
\]
熵等热力学量
\begin{gather*}
	S=\kB\kh{\ln\Xi-\beta E}=4\kB\ln\Xi=\frac43bVT^3;\\
	F=U-TS=-\frac13U;\\
	G=F+pV=0,\implies\mu=0.
\end{gather*}
与光子数不守恒对应。
\section{声子气体}
在Einstein模型中,我们将固体晶格振动简谐近似为独立的简谐振子,频率$\nu$,量子数为$n$的振子激发态相当于产生了$n$个能量为$h\nu$的粒子,称为声子。

声子气体不可分,符合Bose分布,且声子数不守恒
\[
	a_i=\frac{\omega_i}{\e{\beta h\nu_i}-1}.
\]

Einstein模型定量不符,因为忽略了低频振动,而低温下的热激发主要在低频(长波)部分,当波长$\gg$原子间距时,可看做$0-\omega_\Db$的连续谱。


声波分为横波(transverse)和纵波(longitudinal),速度分别为$v_\tv$和$v_\lt$;横波有两种振动方式,纵波只有一种。
\[
	\varepsilon=\hbar\omega,\quad p=\hbar k;\quad \omega=kv.
\]
纵波声子状态数
\[
	\frac V{h^3}\cdot 4\pi p_\lt^2\d p_\lt=\frac V{2\pi^2v_\lt^3}\omega^2\d\omega.
\]
横波同理,故总状态数
\[
	g(\omega)\d\omega=\frac V{2\pi^2}\kh{2v_\tv^{-3}+v_\lt^{-3}}\omega^2\d\omega=:B\omega^2\d\omega.
\]
由
\[
	3	N=\int_0^{\omega_\Db}g(\omega)\d\omega=\frac B3\omega_\Db^3,\implies B=\frac{9N}{\omega_\Db^3}.
\]
可得
\begin{align*}
	g(\omega)=\begin{cases}
		9N\omega^2/\omega_\Db^3,&0\leqslant\omega\leqslant\omega_\Db\\
		0,&\omega>\omega_\Db
	\end{cases}
\end{align*}
能量 
\begin{align*}
	E&=E_0+\int\zti\frac{ g(\omega)\hbar\omega}{\e{\beta\hbar\omega}-1}\d\omega=E_0+\frac{9N\hbar}{\omega_\Db^3}\int_0^{\omega_\Db}\frac{\omega^3\d\omega}{\e{\beta\hbar\omega}-1}
\end{align*}
取Debye温度
\[
	\theta_\Db:=\frac{\hbar\omega_\Db}\kB\sim\SI{200}\K
\]
并取$y=\theta_\Db/T=\beta\hbar\omega_\Db$
\begin{gather}
	E=E_0+3N\kB T\Debye(y).\\
	C_V=3N\kB\fkh{4\Debye(y)-\frac{3y}{\e{y}-1}}.
\end{gather}
其中Debye函数
\[
	\Debye(y)=\frac 3{y^3}\int_0^{y}\frac{x^3\d x}{\e x-1}.
\]

高温极限$y\ll 1$
\begin{align*}
	\Debye(y)=\frac3{y^3}\int_0^yx^2-\frac{x^3}2+\bigo(x^4)\d x=1-\frac38y+\bigo(y^2).
\end{align*}
\[
	E\simeq E_0+3N\kB T,\quad C_V\simeq 3N\kB.
\]
低温极限$y\gg 1$,可认为
\begin{gather}
	\notag
	\Debye(y)=\frac3{y^3}\int\zti\frac{x^3\d x}{\e x-1}=\frac{\pi^4}{5y^3}.\\
	\label{eqn:CV-Debye}
	C_V=3N\kB\frac{4\pi^4}5\kh{\frac T{\theta_\Db}}^3\propto T^3.
\end{gather}
与试验符合。
\begin{compactenum}
	\item 固体中原子作用强,不能直接用近独立粒子统计。$T$较低时,简谐近似成立——原子集体振动的简正模式。
	相互独立:近独立的理想声子气体。
	\item 声子是准粒子,与振动激发态等效的粒子,有能量、动量等,
	%但不同于电子等,
	只存在于固体中,$\varepsilon$与$p$的关系(色散关系)可不同于普通粒子。
	\item 实际固体比热:金属、自由电子气贡献。

	化合物的分子间振动为声频,适用Debye模型;分子内振动为光频,适用Einstein模型。
\end{compactenum}
\section{Fermi气体}
讨论简并费米气体的低温性质,$n\lambda_T^3\geqslant 1$,相互作用弱。
\[
	a_i=\frac{\omega_i}{\e{\alpha+\beta\varepsilon_i}+1}.
\]
能级$\varepsilon_i$的每个量子态上的平均粒子数
\[
	f_i:=\frac{a_i}{\omega_i}=\frac1{\e{\alpha+\beta\varepsilon_i}+1}.
\]
\paragraph{完全Fermi气}由Pauli原理,粒子不能都处于$\varepsilon=0$态,但尽可能低,即存在$\varepsilon_\Fm$:当$\varepsilon<\varepsilon_\Fm$时,各量子态各有一个粒子;而$\varepsilon>\varepsilon_\Fm$时,态无粒子
\[
	\lim_{T\to0}f_i=\lim_{T\to0}\frac1{\e{(\varepsilon-\mu)/\kB T}+1}=\begin{cases}
	1,&\varepsilon<\mu(T=0)\equiv\varepsilon_\Fm\\0,&\varepsilon>\varepsilon_\Fm
\end{cases}
\]

单原子为例,能量准连续
\[
	g(\varepsilon)\d\varepsilon=2\pi g_s\kh{\frac{2m}{h^2}}^{3/2}V\sqrt\varepsilon\d\varepsilon=:CV\sqrt\varepsilon\d\varepsilon.
\]
有
\[
	N=\int_0^{\varepsilon_\Fm}g(\varepsilon)\d\varepsilon=\frac23CV\varepsilon_\Fm^{3/2}.
\]
故
\begin{align}
	\varepsilon_\Fm=\frac{h^2}{2m}\kh{\frac{3N}{4\pi g_sV}}^{2/3}.
\end{align}
零点能
\begin{align}
	U_0=\int_0^{\varepsilon_\Fm}g(\varepsilon)\varepsilon\d\varepsilon=\frac35N\varepsilon_\Fm.
\end{align}
零点压强
\[
	p_0=-\kh{\pv FV}_T=-\pv{U_0}V=-\dv{\varepsilon_\Fm}V\dv{U_0}{\varepsilon_\Fm}=\frac23\frac{U_0}V.
\]
熵
\[
	S=\kB\ln\Om[F]=0.
\]
\begin{example}{金属中的电子气}{Electron Gas in Metals}
	电子$m_\elc\sim\SI{e-30}\kg$,数密度$\sim\SI{e28}{\per\m\cubed}$, % \tothe{3}
	自旋$g_s=2$,故$\varepsilon_\Fm\sim\SI1\eV$,
	\[
	v_\Fm\sim\sqrt{\frac{2\varepsilon_\Fm}m}\sim\SI[per-mode=symbol]{e6}{\m\per\s}
\]
	压强$p_0\sim\SI{e4}\atm$,这是纯粹的量子效应。
\end{example}
\paragraph{强简并Fermi气}Fermi温度
\[
	T_\Fm:=\frac{\varepsilon_\Fm}\kB.
\]
对于金属电子气,$T_\Fm\sim\SI{e4}\K$。

低温情形$T\ll T_\Fm$,热运动能量小,粒子分布基本不变,只有$\varepsilon_\Fm$附近的粒子可能是跳到高能级态上:
\begin{center}
	\begin{tikzpicture}
		\coor 43{\varepsilon}{f_i};
		\draw[thick,dashed](2,1.25)--(2,0); % node[below]{$\mu$};
		\draw[thick,dashed](1,2.5)--(1,0)node[below]{$\mu-\kB T$};
		\node[left]at(0,2.5){1};
		\draw[thick,domain=0:3]plot(\x,{2.5/(e^(6*\x-12)+1)})node[below]{$\mu+\kB T$};
	\end{tikzpicture}
	\captionof{figure}{强简并Fermi气粒子分布}
\end{center}
定性估计比热$C_V$:相对$T=0$时,能量增量
\[
	\D E\simeq N\frac{\kB T}{\varepsilon_\Fm}\D\varepsilon,\quad\D\varepsilon=\kB T.
\]
比热
\[
	C_V\simeq 2\kB N\frac{\kB T}{\varepsilon_\Fm}\sim T.
\]

单原子,能量准连续,需计算积分
\[
	Q_\ell:=\int\zti f(\varepsilon)\varepsilon^\ell\d\varepsilon.
\]
注意到$f$的特点,可在$\varepsilon=\mu$展开
\begin{align}\notag
	Q_\ell&=\cancel{\edg{\frac{\varepsilon^\ell}{\ell+1}f(\varepsilon)}\zti}-\frac1{\ell+1}\int\zti f'(\varepsilon)\varepsilon^{\ell+1}\d\varepsilon % =:\int\zti f'(\varepsilon)\upsilon(\varepsilon)
	\d\varepsilon\\
	&=\sum_{n=0}^\infty\frac{\upsilon^{(n)}(\mu)}{n!}\int\zti f'(\varepsilon)(\varepsilon-\mu)^n\d\varepsilon,\quad\upsilon(\varepsilon):=-\frac{\varepsilon^{\ell+1}}{\ell+1}.
\end{align}
令$\eta:=\beta(\varepsilon-\mu)$,则
\[
	f(\varepsilon)=\frac1{\e\eta+1},\quad f'(\varepsilon)=-\frac{\beta\e\eta}{(\e\eta+1)^2}
\]
故
\[
	Q_\ell=-\sum_{n=0}^\infty\frac{\upsilon^{(n)}(\mu)}{n!\beta^n}\int_{-\beta\mu}^{+\infty}\frac{\eta^n\e\eta}{(\e\eta+1)^2}\d\eta.
\]

低温下,积分下限$-\beta\mu\to-\infty$
\begin{align*}
	Q_\ell&\simeq-\sum_{n=0}^\infty\frac{\upsilon^{(n)}(\mu)}{n!\beta^n}\int\iti\frac{\eta^n\e\eta}{(\e\eta+1)^2}\d\eta\\
	&=-\fkh{\upsilon(\mu)+\frac{\upsilon''(\mu)}{2\beta^2}\frac{\pi^2}3+\cdots}.
\end{align*}
故
\begin{align}
	N=CVQ_{1/2}&=\frac23CV\mu^{3/2}\fkh{1+\frac{\pi^2}8\alpha^{-2}+\bigo\!\kh{\alpha^{-4}}};\\
	U=CVQ_{3/2}&=\frac25CV\mu^{5/2}\fkh{1+\frac{5\pi^2}8\alpha^{-2}+\bigo\!\kh{\alpha^{-4}}}.
\end{align}
其中$\alpha^{-1}(T)=-\frac1{\beta\mu}=\frac{\kB T}{\mu}$。

巨配分函数$\ln\Xi=\frac23\beta U$,压强$p=\frac1\beta\pv{\ln\Xi}V$,
熵
\begin{align}\notag
	S&=\kB\kh{\ln\Xi-\alpha\pv{\ln\Xi}\alpha-\beta\pv{\ln\Xi}\beta}\\\notag
	&=\kB\frac4{15}CV\beta^{-3/2}(-\alpha)^{5/2}\fkh{0+\frac{5\pi^2}4\alpha^{-2}+\bigo\!\kh{\alpha^{-4}}}\\
	&=\frac{\pi^2}3CV\mu^{1/2}\kB^2T\fkh{1+\bigo\!\kh{\alpha^{-2}}}.
\end{align}

利用
\[
	N=\frac23CV\varepsilon_\Fm^{3/2}.
\]
结合$\varepsilon_\Fm=\mu_0$反解出$\mu$
\[
	\mu=\mu_0\fkh{1-\frac{\pi^2}{12}\alpha^{-2}+\bigo\!\kh{\alpha^{-4}}},
\]
不同于Bose气体,$\mu$可正可负。

宏观量用可观测量表示
\begin{gather}
	U=U_0\fkh{1+\frac{5\pi^2}{12}\kh{\frac T{T_\Fm}}^2+\bigo(T^4)},\\
	\label{eqn:CV-Fermi}
	C_V=N\kB\cdot\frac{\pi^2}{2}\frac T{T_\Fm}\fkh{1+\bigo(T^2)}.
\end{gather}
电子气对金属热容量的贡献首先由Sommerfeld解决。

因此低温下金属比热的实验值是电子气和晶格振动(Debye模型)共同贡献
\[
	C_V\sim \underset{\text{Fermi}}{c_\elc T}+\underset{\text{Debye}}{c_\vb T^3}.
\]
与实验符合得很好。
\begin{example}{电子比热vs.晶格比热}{}
	低温下,式\eqref{eqn:CV-Debye}给出晶格比热和式\eqref{eqn:CV-Fermi}给出电子气比热分别为
	\[
	C_V^\vb=N\kB\frac{12\pi^4}5\kh{\frac T{\theta_\Db}}^3,\quad C_V^\elc=N\kB\frac{\pi^2}2\frac T{T_\Fm}.
	\]
	对铜,$\theta_\Db\sim\SI{300}\K,\;T_\Fm\sim\SI{8e4}\K$,二者比值
	\[
	\frac{C_V^\elc}{C_V^\vb}=\frac5{24\pi^2}\frac T{T_\Fm}\kh{\frac{\theta_\Db}T}^3\sim\frac8{T^2}.
	\]
\end{example}
\chapter{原子核反应}
通过研究核衰变来认识原子核具有相当大的局限性:
\begin{compactitem}
	\item 只涉及不稳定核素向稳定核素的转变,大量稳定核素并不发生衰变;
	\item 是自发过程,不涉及核与核、核与其它粒子的相互作用;
	\item 核衰变仅限于几个MeV的低激发能级,而在高能量范围的核现象会更加丰富多彩……
\end{compactitem} 

\begin{definition}{原子核反应}{nuclear reaction}
	用具有一定能量的粒子(核子、原子核、$\gamma$射线或电子)轰击靶核,使其组成或能量状态发生变化,成为不稳定核素,并放出粒子的过程。
\end{definition}
与自发的衰变不同,核反应是被诱发的过程。

考虑一个典型的核反应
\begin{align}\label{AabB}
	a+\nuc A\to b+\nuc B
\end{align}
其中$a$是入射粒子,$\nuc A$是靶核,$\nuc B$是余核,$b$是出射粒子,这个反应我们可以记作
\[
	\reac AabB,
\]
其中$(a,b)$可以代表一类反应:$(\alpha,\nton),(\nton,\gamma)$等。

实现核反应的粒子来源:
\begin{compactitem}
	\item 放射源:$\alpha,\beta,\gamma,\nton$
	\item 宇宙射线:高能质子、$\nucli4{He}$、中子;
	\item 加速器:质子、中子、重离子、X/$\gamma$射线;
	\item 反应堆:中子、$\gamma$、中微子等。
\end{compactitem}
\section{原子核反应概况}
1919年,Rutherford实现了历史上第一个人工核反应
\[
	\alpha+\nucli{14}N\to\nucli{17}O+\pton.
\]

1932年,John. Cockcroft和Ernest Walton加速质子轰击锂靶,实现了第一个在加速器上实现的核反应
\[
	\pton+\nucli 7{Li}\to\alpha+\alpha.
\]

1934年,Curie夫妇产生第一个人工放射性核素
\begin{gather*}
	\alpha+\nucli{27}{Al}\to\nucli{30}P+\nton,\\
	\nucli{30}P\decayto{\SI{2.5}{min}}\nucli{30}{Si}+\elc^++\nu_\elc,\\
	\elc^++\elc^-\to\gamma+\gamma.
\end{gather*}

1932年,Chadwick发现中子
\[
	\alpha+\nucli9{Be}\to\nucli{12}C+\nton.
\]
\paragraph{核反应分类}
按出射粒子分类:
\begin{compactitem}
	\item 核散射:出射粒子和入射粒子是同种粒子;
	
	分为弹性散射$\reac AaaA$和非弹性散射$\reac Aa{a'}{A^\ast}$。
	\item 核转变:出射粒子和入射粒子不同。
	
	若出射粒子有$\gamma$,也叫辐射俘获;若入射粒子有$\gamma$,则称之为光核反应。
\end{compactitem}
按入射粒子分类:中子核反应、带电粒子核反应、光核反应。

按能量分类……一般的原子核物理只涉及低能核反应。
\paragraph{反应道}核反应的入射道和出射道统称为反应道。

各反应道的发生几率是不同的:随入射粒子能量变化,与核反应机制、核结构有关,受守恒条件约束。

\paragraph{核反应中的守恒}电荷、核子数、动量、能量、角动量、宇称守恒。
\section{核反应能和\textit{Q}方程}
核反应\eqref{AabB}中能量守恒
\[
	(m_a+m_{\nuc A})c^2+T_a+T_{\nuc A}=(m_b+m_{\nuc B})c^2+T_b+T_{\nuc B}
\]
核反应过程释放出的能量,称为反应能$Q$
\begin{align}\notag
	Q&=(T_b+T_{\nuc B})-(T_a+T_{\nuc A})\\
	&=(m_a+m_{\nuc A})c^2-(m_b+m_{\nuc B})c^2.
\end{align}
\paragraph{$Q$方程}假设靶核静止$T_{\nuc A}=0$,由动量守恒
\[
	\bm P_a=\bm P_b+\bm P_{\nuc B}.
\]
\begin{center}
	\includegraphics[page=8]{figures/tikz/layouts.pdf}
	\captionof{figure}{核反应动量守恒}
\end{center}
$b$的出射角$\theta$,由余弦定理,不易测量的$T_\nuc B$可被$P_a,P_b,\theta$表示
\[
	P_\nuc B^2=P_a^2+P_b^2-2P_aP_b\cos\theta.
\]
对于低能核反应,用非相对论公式$P^2=2mT$
%Q=\kh{1+\frac{m_b}{m_{\nuc B}}}T_b-\kh{1-\frac{m_a}{m_{\nuc B}}}T_a-\frac{2\sqrt{m_am_bT_aT_b}}{m_{\nuc B}}\cos\theta
\begin{align}\notag
	Q&=T_b+T_{\nuc B}-(T_a+0)\\
	&=\kh{1+\frac{m_b}{m_{\nuc B}}}T_b-\kh{1-\frac{m_a}{m_{\nuc B}}}T_a-\frac{2\sqrt{m_am_bT_aT_b}}{m_{\nuc B}}\cos\theta
\end{align}
进而可得到出射粒子$b$的能量-角度关系
\begin{align}
	T_b=\hkh{\frac{\sqrt{m_am_bT_a}}{m_{\nuc B}+m_b}\cos\theta\pm\sqrt{\kh{\frac{m_{\nuc B}-m_a}{m_{\nuc B}+m_b}+\frac{m_am_b\cos^2\theta}{(m_{\nuc B}+m_b)^2}}T_a+\frac{m_{\nuc B}}{m_{\nuc B}+m_b}Q}}^2.
\end{align}
这就是$Q$方程,\index{$Q$方程}将$T_a,T_b,\theta,Q$四个变量联系起来。

当余核B处于激发态的时候,反应能$Q$也会发生相应的改变。

\section{核反应的阈值}
\paragraph{实验室系与质心系}
入射粒子$a$和靶核A相对于质心C运动的动能之和,称为$a$的相对运动动能$T'$
\begin{align}
	T'=\frac{m_{\nuc A}}{m_a+m_{\nuc A}}T_a
\end{align}
由于A相对实验室L是静止的,$a$在L系下的动能$T_a$,只有一部分构成了内能项$T'$,能够参与核反应,而整个质心系的动能不能参与核反应。

\paragraph{核反应的阈值}
在L系中能够引起核反应的$a$的最低能量$T_\thres$
\begin{align}
	T_\thres=\frac{m_a+m_{\nuc A}}{m_{\nuc A}}\abs Q.
\end{align}
由于质心系要带走动能,$T_\thres$必然比$\abs Q$大。
\paragraph{出射角转换}
出射粒子在L系和C系中速度的关系
\[
	\bm v_b=\bm v'_b+\bm v\CM
\]
\begin{center}
	\includegraphics[page=9]{figures/tikz/layouts.pdf}
	\captionof{figure}{L系和C系速度关系}
\end{center}
由正弦定理、余弦定理等关系
\[
	\begin{cases}
		\frac{v_b'}{\sin\theta\LAB}=\frac{v\CM}{\sin\theta\CM-\sin\theta\LAB}\\
		v_b\cos\theta\LAB=v\CM+v_b'\sin\theta\CM\\
		v_b^2=v\CM^2+v_b'^2+2v\CM v_b'\cos\theta\CM
	\end{cases}
\]
定义
\begin{align}
	\gamma:=\frac{v\CM}{v_b'}.
\end{align}
则
\begin{align}
	\begin{cases}
		\theta\CM=\theta\LAB+\arcsin(\gamma\sin\theta\LAB)\\
		\cos\theta\LAB=\frac{\gamma+\cos\theta\CM}{(1+\gamma^2+2\gamma\cos\theta\CM)^{1/2}}
	\end{cases}
\end{align}
不难用$T'$和$Q$解出$\gamma$
\[
	\gamma^2=\frac{m_am_b}{m_\nuc Am_\nuc B}\frac{m_b+m_\nuc B}{m_a+m_\nuc A}\frac{T'}{T'+Q}
\]
由于$m_a+m_\nuc A\doteq m_b+m_\nuc B$
\begin{align}
	\gamma\doteq\sqrt{\frac{m_am_b}{m_\nuc Am_\nuc B}\frac{T'}{T'+Q}}.
\end{align}
\subparagraph{弹性散射}$Q=0,a=b,\nuc A=\nuc B$,故 
\[
	\gamma=\frac{m_a}{m_\nuc A}.
\]
当$m_\nuc A\gg m_a$时,$\gamma\doteq 0$,$\theta\CM=\theta\LAB$;当$m_\nuc A= m_a$时,$\gamma=1$,$\theta\CM=2\theta\LAB$。
\subparagraph{一般情况}当$\gamma<1$时,$v_b'>v\CM$

当$\theta\CM=\theta\LAB=0$时,$v_b$取最大值$v_{b\max{}}=v_b'+v\CM$;当$\theta\CM=\theta\LAB=\pi$时,$v_b$取最小值$v_{b\min{}}=v_b'-v\CM$。
\begin{center}
	\includegraphics[page=10]{figures/tikz/layouts.pdf}
	\captionof{figure}{$\gamma<1$}
\end{center}

当$\gamma>1$时,$v_b'<v\CM$。会出现圆锥效应:
一个$\theta\LAB$对应两个$\theta\CM$
\[
	\theta_{\mathrm L\max{}}=\arcsin\gamma\iv.
\]
\begin{center}
	\includegraphics[page=11]{figures/tikz/layouts.pdf}
	\captionof{figure}{$\gamma>1$圆锥效应}
\end{center}

\section{核反应截面和产额}
核反应截面$\sigma$的物理意义为一个入射粒子与单位面积上的靶核发生反应的概率。截面的大小与$a,\nuc A$和$T_a$有关。\index{反应截面}

截面的量纲为面积,单位为$\si{barn}$
\begin{align}
	\SI{1}{barn}=\SI{e-24}{cm^2}
\end{align}
与原子核的截面大小相当。

核反应产物的出射可能各向异性,这就会定义微分截面,
\begin{align}
	\sigma(\theta,\phi):=\dv\sigma\Omega(\theta,\phi)
\end{align}
实验测量微分截面,积分可得到总截面。
\[
	\sigma=\int_0^{2\pi}\int_0^\pi\sigma(\theta,\phi)\sin\theta\d\theta\nd\phi=2\pi\int_0^\pi\sigma(\theta)\sin\theta\d\theta.
\]
\paragraph{反应截面转换}出射粒子数不随坐标系的选择而改变
\[
	\sigma\CM(\theta\CM)\d\Omega\CM=\sigma\LAB(\theta\LAB)\d\Omega\LAB
\]
可得
\[
	\sigma\LAB(\theta\LAB)=\frac{(1+\gamma^2+2\gamma\cos\theta\CM)^{3/2}}{1+\gamma\cos\theta\CM}\sigma\CM(\theta\CM).
\]
\paragraph{核反应产额}定义:入射粒子在靶中引起的核反应数$N'$与入射粒子数$I_0$之比,称为核反应产额(yield) $Y$。\index{反应产额}
\begin{align}
	Y:=\frac{N'}{I_0}.
\end{align}
与反应截面$\sigma$和数密度$N$有关。

通过厚度为$D$的靶后,未发生反应的入射粒子数
\[
	-\frac{\d I}{I}=\sigma N\d x,\implies I=I_0\e{-\sigma ND}
\]
故透射率$T=\e{-\sigma ND},$
\begin{align}
	Y=1-\e{-\sigma ND}.
\end{align}
宏观截面($\si{1/cm}$) $\Sigma:=N\sigma$,平均自由程$\lambda=1/\Sigma$,厚靶$D\gg\lambda,Y=1.$

\sectionstar{核反应中的分波分析}
尽管核反应截面的单位$\si{barn}$与原子核截面相近,但是有时候二者也可以差异巨大。入射粒子带来的轨道角动量有不同的组成,可以根据不同的轨道角动量来分析核反应截面。

\paragraph{半经典的分波分析}入射粒子$a$速度$v_a$,相对质心的运动动能
\[
	T'=\frac12\mu v_a^2,\quad \mu:=\frac{m_am_{\nuc A}}{m_a+m_{\nuc A}}.
\]
相对运动动量
\[
	p=\mu v_a=\frac\hbar\barlambda,\quad\lambdabar:=\frac\hbar{p}.
\]
式中$\lambdabar$是约化de Broglie波长。\\
相对运动角动量
\[
	L=\rho p=\frac\rho\barlambda\hbar=\ell\hbar,\implies\rho=\ell\lambdabar,\quad\ell=0,1,\ldots
\]%ħ

$(a,\nuc A)$的碰撞过程,可以被分解为对应于轨道角动量$L=0,\hbar,2\hbar,\ldots$的部分。轨道角动量为$\ell\hbar$的入射粒子与靶核作用的截面为
\[
	S_\ell=\pi(\rho_{\ell+1}^2-\rho_\ell^2)=(2\ell+1)\pi\lambdabar^2.
\]
最大半径$\rho\maxi=R=R_a+R_{\nuc A}$,进而得到反应截面
\begin{align}
	\sigma=\sum_{\ell=0}^{R/\barlambda}(2\ell+1)\pi\lambdabar^2=\pi(R^2+\lambdabar^2),
\end{align}
核的尺寸和粒子的波动性,都对截面有贡献。
\paragraph{量子力学的分波分析}
向$x$方向入射的粒子束可用平面波表示,在有心力场中,球面波分解更合适
\[
	\psi_\i=\e{\i kx}=\e{\i kr\cos\theta}=\sum_{\ell=0}^\infty(2\ell+1)\i^\ell j_\ell(kr)P_\ell(\cos\theta),
\]
波函数远离原子核时,$kr\gg\ell$,
\[
	j_\ell(kr)\doteq\frac{\sin(kr-\pi\ell/2)}{kr}=\i\frac{\e{-\i(kr-\pi\ell/2)}-\e{\i(kr-\pi\ell/2)}}{kr}.
\]
故
\[
	\psi_\i=\frac1{2kr}\sum_{\ell=0}^\infty\i^{\ell+1}(2\ell+1)\fkh{\e{-\i(kr-\pi\ell/2)}-\e{\i(kr-\pi\ell/2)}}P_\ell(\cos\theta).
\]
其中$\e{-\i(kr-\pi\ell/2)}$代表入射球面波,$\e{\i(kr-\pi\ell/2)}$代表出射球面波。
由于原点上有靶核,散射会导致出射波函数的变化,
\[
	\psi=\frac1{2kr}\sum_{\ell=0}^\infty\i^{\ell+1}(2\ell+1)\fkh{\e{-\i(kr-\pi\ell/2)}-\eta_\ell\e{\i(kr-\pi\ell/2)}}P_\ell(\cos\theta).
\]
其中$\eta_\ell$是出射波系数。
故靶核导致的散射
\begin{align}
	\psi_\sca=\psi-\psi_\i=\frac1{2kr}\sum_{\ell=0}^\infty\i^{\ell+1}(2\ell+1)(1-\eta_\ell)\e{\i(kr-\pi\ell/2)}P_\ell(\cos\theta).
\end{align}

散射截面等于散射粒子数比上入射粒子注量率,因此
\[
	\d\sigma_\sca=\frac{j_\sca r^2\d\Omega}{j_\i},
\]
量子力学中
\[
	j=\frac{\hbar}{2m\i}\kh{\psi\cj\pv\psi{r}-\psi\pv{\psi\cj}r}.
\]
以$\ell=0$为例,计算得到
\[
	j_\sca=\frac{\hbar k}m\frac{\abs{1-\eta_0^2}}{4k^2r^2},\quad j_\i=\frac{\hbar k}m.
\]
故
\[
	\dv{\sigma_{\mathrm{sc},0}}\Omega=\frac{\abs{1-\eta_0^2}}{4k^2}=\frac{\barlambda^2}4\abs{1-\eta_0^2},
\]
把所有角动量均考虑进去
\[
	\dv{\sigma_\sca}\Omega=\frac{\barlambda^2}4\abs{\sum_{\ell=0}^\infty\i(2\ell+1)(1-\eta_\ell)P_\ell(\cos\theta)}^2.
\]
对$4\pi$立体角积分得散射截面
\begin{align}
	\sigma_\sca=\pi\lambdabar^2\sum_{\ell=0}^\infty(2\ell+1)\abs{1-\eta_\ell}^2.
\end{align}
对于反应截面,可以认为$a$消失了,类似的推导得出
\[
	\sigma_{\mathrm{r}}=\pi\lambdabar^2\sum_{\ell=0}^\infty(2\ell+1)(1-\abs{\eta_\ell}^2).
\]

\paragraph{低能中子的散射截面}对于低能入射中子,$\ell=0$,波函数简化为 
\[
	u_{\mathrm o}(r)=\frac\i{2k}\kh{\e{-\i kr}-\eta_0\e{\i kr}}.
\]
若入射粒子与核的作用已知,则核内波函数可知,继而可知核边界处的对数导数
\[
	f:=\edg{\frac r{u_{\mathrm o}}\dv{u_{\mathrm o}}r}_{r=R}\implies\eta_0=\frac{f+\i kR}{f-\i kR}\e{-\i2kR}
\]
当$f\to\infty$时,$\eta=\e{-\i2kR}$;当$f\to0$时,$\eta=-\e{-\i2kR}$,慢中子$kR\ll 1$,故 
\begin{align}
	\sigma_{\mathrm{sc},0}=\begin{cases}
		4\pi R^2,&f\to\infty\\
		4\pi\lambdabar^2,&f\to0
	\end{cases}
\end{align}
$4\pi R^2$对应势(形状)弹性散射截面,$4\pi\lambdabar^2$对应共振散射截面。
\paragraph{散射截面vs反应截面}
\begin{center}
	\includegraphics[page=12]{figures/tikz/layouts.pdf}
	\captionof{figure}{散射截面$\sigma_\sca$与反应截面$\sigma_\mathrm r$的允许范围}
\end{center}
允许有纯的散射过程($|\eta_\ell|=1$),但不允许有纯的反应过程。入射道对出射道总是开放的——入射粒子可以沿入射道返回,因此一定存在弹性散射。
\section{核反应机制及核反应模型}
\paragraph{核反应的三阶段描述}Weisskopf将核反应分为了三个阶段:
\begin{compactenum}
	\item \textbf{独立粒子阶段}:
	
	部分入射粒子被吸收,引起核反应;\\
	部分入射粒子被散射,形成弹性散射
	\item \textbf{复合系统阶段}:
	
	入射粒子与靶核交换能量:
	直接作用;形成复合核;中间过程
	\item \textbf{最后阶段}:
	
	复合系统分解为出射粒子和剩余核。
\end{compactenum}
各种截面有关系,间讲义。
\paragraph{复合核模型}核反应被分成相互独立的两个阶段:
\[
	a+\nuc A\to\nuc C^\ast\to\nuc B+b
\]
\begin{compactenum}
	\item 入射粒子射入靶核,与之形成一个复合核,该核处于激发态;
	\item 激发态的复合核可沿入射道弹性散射,也可能开放其它反应道。
\end{compactenum}
复合核的激发能$E^\ast$是相对动能$T'$和结合能$B_{a\nuc A}$的和
\begin{align}
	E^\ast=T'+B_{a\nuc A}=\frac{m_\nuc A}{m_a+m_\nuc A}T_a+B_{a\nuc A}.
\end{align}
复合核发射核子一般需要$\sim\SI{8}{MeV}$的分离能。尽管$E^\ast\sim\SI{20}{MeV}$,但平均到每个核子的能量只有$\SI{0.2}{MeV}$,故复合核的分离需要经过多次碰撞,其寿命$\sim\SIrange{e-18}{e-14}{s}$,对核反应而言,这是较长的时间。

反应截面
\begin{align}
	\sigma_{ab}=\sigma_{\mathrm{CN}}(T_a)W_b(E^\ast),
\end{align}
故复合核如何衰变是与它如何形成无关,只取决于系统现在的能量状态。
\paragraph{共振}只考虑s波,即$\ell=0$的情况,$\eta_0=\e{\i2\delta_0}$
\[
	\sigma_{\mathrm{sc},0}=\pi\lambdabar^2|1-\eta_0|^2=4\pi\lambdabar^2\sin^2\delta_0(T').
\]
当入射中子的能量达到某些值时,$\delta_0(T')=\pi/2$,散射截面达到极大,就出现了共振。

在$\delta_0(T')=\pi/2$处($T'=E_\mathrm R$)做Talyor展开,略去高阶项
\[
	\sigma_{\mathrm{sc},0}=\pi\lambdabar^2\frac{\Gamma^2}{(T'-E_\mathrm R)^2+\Gamma^2/4}.
\]
势弹性散射和共振弹性散射是复数和的关系,因此存在干涉。
\paragraph{慢中子反应}对于慢中子与一个$A>100$的靶核而言,最可几的复合核退激机制是发射$\gamma$射线,$(\nton,\gamma)$反应是个比较慢的过程
\[
	\sigma_{\nton,\gamma}=\pi\lambdabar^2\frac{\Gamma_\nton\Gamma_\gamma}{(T'-E_\mathrm R)^2+\Gamma^2/4}.
\]
由于约化de Broglie波长
\[
	\lambdabar^2=\frac{\hbar^2}{2\mu T'^2}\propto\frac1{v^2}
\]
低能中子进入势阱,$E_\nton\ll V_0$,并不容易
\[
	P\propto k\propto v,\implies\Gamma_\nton\propto v
\]

慢中子的动能,对衰变行为没什么影响,$T'\ll B_{\nton\nuc A}$,故
\[
	E^\ast=\const,\enspace\Gamma_\gamma=\const,
\]

非共振中子,能量的变化对复合核的衰变不再有什么影响了$\Gamma=\Gamma_\nton+\Gamma_\gamma$,$\Gamma_\nton\ll\Gamma_\gamma$,$\Gamma=\const$,故\index{$1/v$规律}
\begin{align}
	\label{slow-neutron}
	\sigma_{\nton,\gamma}\propto\frac1v.
\end{align}
当我们用慢中子来照射靶核时,中子能量的降低将会有助于$(\nton,\gamma)$反应的发生,该反应的截面反比于中子的速度。

这里的讨论是在“中子能量远离共振能量”这个前提下开展的,如果中子的动能碰巧在共振能量附近,则公式中分母的第一项会起作用,使的截面或大或小的偏离$1/v$规律
\paragraph{连续区理论}连续区——$T'$增加,复合核处于高激发态,能级重叠。能级密度加大,能级宽度加大,不再有单能级共振。
\paragraph{总结}截面测量


\chapter{线性方程组的直接解法}

求解线性代数方程组就相当于求解矩阵式$Ax=b$:
\begin{equation}
    \begin{bmatrix}
        a_{11}&\cdots&a_{1n}\\
        \vdots&\ddots&\vdots\\
        a_{n1}&\cdots&a_{nn}
    \end{bmatrix}\begin{bmatrix}
        x_1\\\vdots\\x_n
    \end{bmatrix}=\begin{bmatrix}
        b_1\\\vdots\\b_n
    \end{bmatrix}.
\end{equation}
因此有必要了解一些对矩阵的操作。

\section{矩阵操作}

\begin{definition}
    {稀疏矩阵}{sparse matrix}
    如果一个矩阵绝大多数元素是0,则称其是稀疏的(sparse)。
\end{definition}

\begin{definition}
    {秩一矩阵}{rank-1 matrix}
    若一个矩阵可以表示成$A=uv\dg$,则其秩为一。
\end{definition}

\begin{theorem}
    {奇异值分解}{singular value}
    根据奇异值分解,可以将矩阵表示成秩一矩阵的线性组合:
    \begin{equation}
        A=U\Sigma V\dg=\sum_i\sigma_iu_iv_i\dg,
    \end{equation}
    其中$U,V$为幺正矩阵。
\end{theorem}

\begin{theorem}
    {矩阵的Hierarchical表示}{Hierarchical matrix}
    考虑分块矩阵
    \[
        A=\begin{bmatrix}
            A_{11}&A_{12}\\
            A_{21}&A_{22}
        \end{bmatrix}
    \]
    $A$一般不是稀疏的,但非对角元$A_{12},A_{22}$是稀疏的,则$Ax$可以分块地写成:
    \[
        Ax=\begin{bmatrix}
            A_{11}&A_{12}\\
            A_{21}&A_{22}
        \end{bmatrix}\begin{bmatrix}
            x_1\\x_2
        \end{bmatrix}=\begin{bmatrix}
            A_{11}x_1+A_{12}x_2\\
            A_{21}x_1+A_{22}x_2
        \end{bmatrix},
    \]
    对$A_{11}x,A_{22}x_2$递归处理。
    \begin{center}
        \includegraphics{graphs/Hierarchical.pdf}
        \captionof{figure}{Hierarchical算法示意图,白块表示稀疏部分}
        \label{fig:Hierarchical}
    \end{center}
\end{theorem}

\begin{theorem}
    {矩阵乘法的Strassen算法}{}
    对矩阵乘法$C=AB$分块得:
    \[
        \begin{bmatrix}
            C_{11}&C_{12}\\
            C_{21}&C_{22}
        \end{bmatrix}=\begin{bmatrix}
            A_{11}&A_{12}\\
            A_{21}&A_{22}
        \end{bmatrix}\begin{bmatrix}
            B_{11}&B_{12}\\
            B_{21}&B_{22}
        \end{bmatrix},
    \]
    定义 
    \begin{subequations}
        \begin{align}
            M_1&=(A_{11}+A_{22})(B_{11}+B_{22}),\\
            M_2&=(A_{21}+A_{22})B_{11},\\
            M_3&=A_{11}(B_{12}-B_{22}),\\
            M_4&=A_{22}(B_{21}-B_{11}),\\
            M_5&=(A_{11}+A_{12})B_{22},\\
            M_6&=(A_{21}-A_{11})(B_{11}+B_{12}),\\
            M_7&=(A_{12}-A_{22})(B_{21}+B_{22}),
        \end{align}
    \end{subequations}
    则
    \begin{subequations}
        \begin{align}
            C_{11}&=M_1+M_4-M_5+M_7,\\
            C_{12}&=M_3+M_5,\\
            C_{21}&=M_2+M_4,\\
            C_{22}&=M_1-M_2+M_3+M_6.
        \end{align}
    \end{subequations}
    Strassen算法将分块矩阵的乘法从直接法的8次降低到了7次,由此分而治之,矩阵乘法的时间复杂度便从$\bigo(n^3)$降低到了$\bigo(n^{\log_27})=\bigo(n^{2.807})$。
\end{theorem}

\begin{definition}
    {离散Fourier变换}{discrete Fourier transform}
    式\eqref{eqn:Fourier betaj}中定义了一个线性变换,称为离散Fourier变换(discrete Fourier transform, DFT)
    \[
        X_n=\sum_{m=0}^{N-1}x_m\omega^{mn},
    \]
    其中$\omega:=\e{-\i2\pi/N}$是$N$次单位根,对应的变换矩阵为
    \begin{eqnarray}
        F=\begin{bmatrix}
            1&1&1&\cdots&1\\
            1&\omega&\omega^2&\cdots&\omega^{N-1}\\
            1&\omega^2&(\omega^2)^2&\cdots&(\omega^2)^{N-1}\\
            \vdots&\vdots&\vdots&\ddots&\vdots&\\
            1&\omega^{N-1}&(\omega^{N-1})^2&\cdots&(\omega^{N-1})^{N-1}
        \end{bmatrix}
    \end{eqnarray}
    事实上$F$上只有$n$个不同的元素。其逆变换为
    \begin{equation}
        F\iv=\frac1N\bar F.
    \end{equation}
\end{definition}

\begin{theorem}
    {Cooley-Tukey快速Fourier变换}{Cooley-Tukey fast Fourier transform}
    以基2 (radix-2)的情形为例,即$N=2^M$。
    将$X_n$的求和分成偶数项$E_n$和奇数项$O_n$
    \begin{align*}
        X_n&=\sum_{k=0}^{N/2-1}x_{2k}\omega^{2kn}+\sum_{k=0}^{N/2-1}x_{2k+1}\omega^{(2k+1)n}\\
        &=\sum_{k=0}^{N/2-1}x_{2k}\omega^{2kn}+\omega^n\sum_{k=0}^{N/2-1}x_{2k+1}\omega^{2kn}=:E_n+\omega^nO_n,
    \end{align*}
    由于$\omega^N=1$,注意到
    \begin{align*}
        X_{n+N/2}&=\sum_{k=0}^{N/2-1}x_{2k}\omega^{2k(n+N/2)}+\sum_{k=0}^{N/2-1}x_{2k+1}\omega^{(2k+1)(n+N/2)}\\
        &=\sum_{k=0}^{N/2-1}x_{2k}\omega^{2kn}-\omega^n\sum_{k=0}^{N/2-1}x_{2k+1}\omega^{2k}=E_n-\omega^nO_n,
    \end{align*}
    由此便将$N$个$X_n$求和($N^2$)转化成了$N/2$个$E_n,O_n$求和($N^2/2$)。
    采用分而治之的算法思想,可以将DFT的时间复杂度从矩阵向量乘法的$\bigo(N^2)$优化到$\bigo(N\log N)$,这称为快速Fourier变换(fast Fourier transform, FFT)。
    \begin{center}
        \includegraphics[width=0.8\linewidth]{graphs/FFT.pdf}
        \captionof{figure}{FFT算法示意图($N=2^3=8$)}
        \label{fig:FFT}
    \end{center}
    \tcblower
    对于非基2的情形,$\omega^k$的周期不是基2的,做处理:
    \[
        X_n=\sum_{m=0}^{N-1}x_m\omega^{-mn}=\sum_{m=0}^Nx_m\omega^{[(m-n)^2-m^2-n^2]/2}.
    \]
    定义$\nu_k:=\omega^{k^2/2}$,记$Y_n:=\nu_nX_n,\;z_m:=\nu_m\iv x_m$,则有卷积形式:
    \[
        Y_n=\sum_{m=0}^{N-1}z_m\nu_{m-n}.
    \]
    可以用0将$z_m,\nu_m$延拓,使其周期是一个比$N$大的基2数$N'$。
    再在两端做DFT:
    \begin{align*}
        \sum_{n=0}^{N'-1}Y_n\omega_{N'}^{nk}&=\sum_{n=0}^{N'-1}\sum_{m=0}^{N'-1}z_m\nu_{m-n}\omega_{N'}^{nk}\\
        &=\sum_{m=0}^{N'-1}z_m\omega_{N'}^{mk}\sum_{n=0}^{N'-1}\nu_{m-n}\omega_{N'}^{(n-m)k}=\sum_{m=0}^{N'-1}z_m\omega_{N'}^{mk}\sum_{m=0}^{N'-1}\nu_m\omega_{N'}^{mk}
    \end{align*}
    因此通过对$z_m,\nu_m$做两次DFT、一次向量分量积、一次逆DFT便可得到$Y_n$。
\end{theorem}

\begin{theorem}
    {周期Toeplitz变换}{periodic Toeplitz transform}
    $n$阶矩阵$A$若满足$a_{ij}=c_{i-j}$且序列$c_k$周期为$n$,则$A$称为周期Toeplitz矩阵,且
    \begin{equation}
        A=F\iv\Lambda F,
    \end{equation}
    其中$\Lambda=\diag(\lambda_0,\ldots,\lambda_{n-1})$且
    \begin{equation}
        \lambda_i=\sum_{j=0}^{n-1}c_j\omega^{-ij}.
    \end{equation}
\end{theorem}

\begin{remark}
    $n$阶矩阵$A$若满足$a_{ij}=c_{i-j}$,则该矩阵可以扩展成$2n$阶的周期Toeplitz矩阵。
    \[
        \begin{bmatrix}
            A&*\\ *&*
        \end{bmatrix}
    \]
\end{remark}

\section{Gauss消元法}
\label{sec:Gauss elimination}

% \paragraph{基本思路}

% 将$Ax=b$的求解问题转化为一个与之等价的容易求解的问题$\tilde A\tilde x=\tilde b$。

\subsection{Gauss消元法}

如何求解$n$元线性方程组?
Cramer法则?时间复杂度$\bigo(n\cdot(n+1)!)$这是不可接受的。

\begin{theorem}
    {Gauss消元法}{Gauss elimination}
    线性方程组形如
    \begin{equation*}
        \lhkh{\begin{aligned}
            a_{11}^{(1)}x_1+a_{12}^{(1)}x_2+\cdots+a_{1n}^{(1)}x_n&=b_1^{(1)}\\
            a_{21}^{(1)}x_1+a_{22}^{(1)}x_2+\cdots+a_{2n}^{(1)}x_n&=b_2^{(1)}\\
            &\vdots\\
            a_{n1}^{(1)}x_1+a_{n2}^{(1)}x_2+\cdots+a_{nn}^{(1)}x_n&=b_n^{(1)}
        \end{aligned}}
    \end{equation*}
    如果$a_{11}^{(1)}\neq 0$,可将第一行的$-a_{i1}^{(1)}/a_{11}^{(1)}$倍加到第$i$行($i=2,3,\ldots,n$),得到一个等价方程组
    \begin{equation*}
        \lhkh{\begin{aligned}
            a_{11}^{(1)}x_1+a_{12}^{(1)}x_2+\cdots+a_{1n}^{(1)}x_n&=b_1^{(1)}\\
            a_{22}^{(2)}x_2+\cdots+a_{2n}^{(2)}x_n&=b_2^{(2)}\\
            &\vdots\\
            a_{n2}^{(2)}x_2+\cdots+a_{nn}^{(2)}x_n&=b_n^{(2)}
        \end{aligned}}
    \end{equation*}
    如果$a_{22}^{(2)}\neq 0$,便可以此类推……最终得到一个等价的上三角线性方程组:
    \begin{equation*}
        \lhkh{\begin{aligned}
            a_{11}^{(1)}x_1+a_{12}^{(1)}x_2+\cdots+a_{1n}^{(1)}x_n&=b_1^{(1)}\\
            a_{22}^{(2)}x_2+\cdots+a_{2n}^{(2)}x_n&=b_2^{(2)}\\
            &\vdots\\
            a_{nn}^{(n)}x_n&=b_n^{(n)}
        \end{aligned}}
    \end{equation*}
    便不难从下至上地解得:$x_n=b_n^{(n)}/a_{nn}^{(n)}$,
    \begin{equation}
        x_i=\division{\biggkh{b_{i}^{(i)}-\sum_{j=i+1}^na_{ij}^{(i)}x_j}}{a_{ii}^{(i)}},\quad i=n-1,\ldots,2,1.
    \end{equation}
    Gauss消元法的复杂度为$\bigo(n^3)$。
\end{theorem}

\begin{remark}
    从上面的算法过程中可见,
    一旦第$k$步$a_{kk}^{(k)}=0$,(顺序) Gauss消元法就不能继续进行下去。
    但是当第$k+1,\ldots,n$行中存在$a_{ik}^{k}\neq 0$,就可以交换$k,i$行,从而使算法继续。

    此外,即使$a_{kk}^{(k)}\neq 0$但$\abs{a_{kk}^{(k)}}\ll 1$,也会出现大数除小数导致精度下降的问题。
\end{remark}

\begin{theorem}
    {列主元方法}{pivoting technique}
    在第$k$步消去之前,找到绝对值最大的主元(pivot):
    \begin{equation}
        i_k=\mathop{\arg\max}\limits_{k\leq i\leq n}\abs{a_{ik}^{(k)}},
    \end{equation}
    然后交换第$k,i_k$行。
\end{theorem}

\subsection{LU分解}

下面我们从矩阵角度考察Gauss消元法。

\begin{definition}
    {初等矩阵}{elementary matrix}
    给定(实)向量$u,v$和标量$\sigma$,形如
    \begin{equation}
        E=I-\sigma uv\tp
    \end{equation}
    的称为(实)初等矩阵(elementary matrix)。
\end{definition}

\begin{corollary}
    初等矩阵的逆也是同类型的初等矩阵:
    \begin{equation}
        (I-\sigma uv\tp)\iv=I-\frac{\sigma}{\sigma v\tp u-1}uv\tp.
    \end{equation}
\end{corollary}

\begin{example}
    {}{}
    \begin{itemize}
        \item 初等排列矩阵:
        \[
            P_{ij}=I-(e_i-e_j)(e_i-e_j)\tp=I-e_{ii}-e_{jj}+e_{ij}+e_{ji}.
        \]
        左乘初等排列矩阵即互换第$i,j$行,右乘即互换第$i,j$列;
        \item 倍加矩阵:
        \[
            I+\alpha e_ie_j\tp=I+\alpha e_{ij},
        \]
        左乘即将第$i$行的$\alpha$倍加到第$j$行上。
    \end{itemize}
\end{example}

\begin{definition}
    {初等单位下三角矩阵}{elementary lower-triangle matrix}
    形如
    \begin{equation}
        L_i=I+\ell_ie_i\tp=\begin{bmatrix}
            1\\ &\ddots\\ &&1\\ &&\ell_{i+1,i}&1\\ &&\vdots&&\ddots\\ &&\ell_{ni}&&&1
        \end{bmatrix}
    \end{equation}
    称为第$i$列的初等单位下三角矩阵。其中向量$\ell_i$的前$i$个分量为0。
\end{definition}

\begin{corollary}
    初等单位下三角矩阵的逆也是初等单位下三角矩阵
    \begin{equation}
        (I+\ell_ie_i\tp)\iv=I-\ell_ie_i\tp.
    \end{equation}
\end{corollary}

\begin{corollary}
    对角元均为1的下三角矩阵称为单位下三角矩阵,可以写成
    \[
        L=L_1L_2\cdots L_{n-1}=\begin{bmatrix}
            1\\\ell_{21}&1\\ \vdots&\vdots&\ddots\\
            \ell_{n1}&\ell_{n2}&\cdots&1
        \end{bmatrix}.
    \]
\end{corollary}

\begin{theorem}
    {矩阵的LU分解}{LU decomposition exists}
    根据Gauss消元法的过程,记
    \[
        A^{(k)}:=\begin{bmatrix}
            a_{11}^{(1)}&\cdots&a_{1k}^{(1)}&\cdots&a_{1n}^{(1)}\\
            &\ddots&\vdots&\ddots&\vdots\\
            &&a_{kk}^{(k)}&\cdots&a_{kn}^{(k)}\\
            &&\vdots&\ddots&\vdots\\
            &&a_{nk}^{(k)}&\cdots&a_{nn}^{(k)}
        \end{bmatrix},\quad
        b^{(k)}:=\begin{bmatrix}
            b_1^{(1)}\\\vdots\\b_k^{(k)}\\\vdots\\b_n^{(k)}
        \end{bmatrix},\quad\ell_k:=-\frac1{a_{kk}^{(k)}}\begin{bmatrix}
            0\\\vdots\\0\\a_{k+1,k}^{(k)}\\\vdots\\a_{nk}^{(k)}
        \end{bmatrix}.
    \]
    相应的初等单位下三角矩阵为$L_k=I+\ell_ke_k\tp$。
    增广矩阵有递推关系:
    \begin{equation}
        [A^{(k+1)}\enspace b^{(k+1)}]=L_k[A^{(k)}\enspace b^{(k)}],
    \end{equation}
    即
    \[
        A^{(n)}=L_{n-1}\cdots L_1A^{(1)},\iff A^{(1)}=L_1\iv\cdots L_{n-1}\iv A^{(n)},
    \]
    则$L=L_1\iv\cdots L_{n-1}\iv$为单位下三角矩阵,$U=A^{(n)}$为上三角矩阵,这样就将$A^{(1)}$分解成了下上三角矩阵的乘积。    
\end{theorem}

\begin{remark}
    根据Gauss消元法的过程可见,LU分解的前提是$a_{11}^{(1)},\ldots,a_{nn}^{(n)}$均不为0。
\end{remark}

\begin{theorem}
    {三角分解定理}{LU decomposition}
    给定矩阵$A$,若其顺序主子式
    \begin{equation}
        \Delta_i:=\begin{vmatrix}
            a_{11}&\cdots&a_{1i}\\
            \vdots&\ddots&\vdots\\
            a_{i1}&\cdots&a_{ii}
        \end{vmatrix},\quad i=1,\ldots,n
    \end{equation}
    均不为0,则存在唯一的单位下三角矩阵$L$和上三角矩阵$U$使得$A=LU$。
\end{theorem}
\begin{proof}
    通过数学归纳法,可证明:$a_{11}^{(1)},\ldots,a_{ii}^{(i)}\neq 0\iff\Delta_1,\ldots,\Delta_i\neq 0$,此时
    \begin{equation}
        \Delta_i=a_{11}^{(1)}\cdots a_{ii}^{(i)},
    \end{equation}
    若存在$L_1,U_1$和$L_2,U_2$使得
    \[
        A=L_1U_1=L_2U_2,
    \]
    两边左乘$L_1\iv$,右乘$U_2\iv$得到:
    \[
        U_1U_2\iv=L_1\iv L_2,
    \]
    此式左边为上三角矩阵,右边为单位下三角矩阵,故只能是单位矩阵$I$,即$U_1=U_2,L_1=L_2$。
\end{proof}

\begin{example}
    {LU分解的例子}{}
    \[
        \begin{bmatrix}
            2&1&1&0\\4&3&3&1\\8&7&9&5\\6&7&9&8
        \end{bmatrix}=\begin{bmatrix}
            1\\2&1\\4&3&1\\3&4&1&1
        \end{bmatrix}\begin{bmatrix}
            2&1&1&0\\ &1&1&1\\ &&2&2\\ &&&2
        \end{bmatrix}.
    \]
\end{example}

\begin{remark}
    $L$有$n(n-1)/2$个变量,$U$有$n(n+1)/2$个变量,可以直接由$A$确定$L,U$。
\end{remark}

\begin{theorem}
    {直接LU分解法(Doolittle分解法)}{Doolittle decomposition}
    写成分块矩阵的形式:
    \[
        L^{(k)}=\begin{bmatrix}
            1\\\ell_{n-k+1}&L^{(k-1)}
        \end{bmatrix},\quad
        U^{(k)}=\begin{bmatrix}
            u_{u-k+1,u-k+1}&u_{u-k+1}\tp\\ &U^{(k-1)}
        \end{bmatrix},
    \]
    则
    \[
        A^{(k)}=L^{(k)}U^{(k)}=\begin{bmatrix}
            u_{u-k+1,u-k+1}&u_{u-k+1}\tp\\
            u_{u-k+1,u-k+1}\ell_{n-k+1}&A^{(k-1)}+\ell_{n-k+1}u_{u-k+1}\tp
        \end{bmatrix}.
    \]
    由此可确定$u,\ell$,同时$A^{(k)}$的阶数减1.
\end{theorem}
    
\begin{remark}
    三角分解\thmref{thm:LU decomposition} 给出了LU分解的条件。
    而对于一般的可逆矩阵,也可以通过换行实现LU分解。
\end{remark}

\begin{theorem}
    {一般三角分解定理}{LUP decomposition}
    若$A$可逆,则存在排列矩阵$P$、单位下三角矩阵$L$和上三角矩阵$U$使得
    \begin{equation}
        PA=LU.
    \end{equation}
\end{theorem}

\begin{proof}
    考虑列主元方法的Gauss消元法,第$k$步交换$k,i_k$行,则
    \[
        A^{(k+1)}=L_kI_{ki_k}A^{(k)},
    \]
    即
    \[
        A^{(n)}=L_{n-1}I_{n-1,i_{n-1}}\cdots L_1I_{1i_1}A^{(1)},
        \iff
        A^{(1)}=I_{1i_1}L_1\iv\cdots I_{n-1,i_{n-1}}L_{n-1}\iv A^{(n)},
    \]
    定义
    \[
        P_k=I_{n-1,i_{n-1}}\cdots I_{ki_k},
    \]
    则$P_k\tp P_{k+1}=I_{ki_k}$,进而
    \[
        P_1A^{(1)}=P_2L_1\iv P_2\tp P_3L_2\iv\cdots P_{n-1}L_{n-1}\iv A^{(n)},
    \]
    易得$P_{k+1}e_k=e_k$,故
    \[
        L_k':=P_{k+1}L_k\iv P_{k+1}\tp=P_{k+1}(I-\ell_ke_k\tp)P_{k+1}\tp=I-P_{k+1}\ell_ke_k\tp,
    \]
    仍然是第$k$列的初等单位上三角矩阵,
    令$P=P_1,\;L=L_1'\cdots L_{n-2}'L_{n-1},\;U=A^{(n)}$即得。
\end{proof}

\subsection{Cholesky分解}

下面再看对称矩阵的三角分解。

\begin{theorem}
    {Cholesky分解}{Cholesky decomposition}
    若$A$实对称正定,则存在唯一的对角元素为正的下三角矩阵$L$使得
    \begin{equation}
        A=LL\tp.
    \end{equation}
\end{theorem}

\begin{proof}
    采用加边Cholesky分解法:写成分块矩阵的形式
    \[
        L_i=\begin{bmatrix}
            L_{i-1}\\\ell_{i-1}\tp&\ell_{ii}
        \end{bmatrix},\quad A_i=\begin{bmatrix}
            A_{i-1}&a_{i-1}\\ a_{i-1}\tp&a_{ii}
        \end{bmatrix}
    \]
    满足$A_i=L_iL_i\tp$,
    可得 
    \begin{subequations}
        \begin{align}
            \ell_{i-1}&=L_{i-1}\iv a_{i-1},\\
            \ell_{ii}&=\sqrt{a_{ii}-\ell_{i-1}\tp\ell_{i-1}}.
        \end{align}
    \end{subequations}
    从$\ell_{11}=\sqrt{a_{11}}$出发便可迭代得到整个$L$。
\end{proof}

\begin{remark}
    这种算法特别适合稀疏矩阵。
\end{remark}

\subsection{Thomas方法}

考虑线性方程组
\[
    \begin{cases}
        b_1x_1+c_1x_2=d_1,\\
        a_ix_{i-1}+b_ix_i+c_ix_{i+1}=d_i,&i=2,\ldots,n-1\\
        a_nx_{n-1}+b_nx_n=d_n
    \end{cases}
\]
系数矩阵为三对角矩阵:
\[
    \begin{bmatrix}
        b_1&c_1\\a_2&b_2&\ddots\\ &\ddots&\ddots&c_{n-1}\\ &&a_n&b_b
    \end{bmatrix}.
\]
\begin{theorem}
    {Thomas方法}{Thomas technique}
    容易验证有如下三角分解形式:
    \[
        A=LU=\begin{bmatrix}
            1\\\ell_2&1\\&\ddots&\ddots\\ &&\ell_n&1
        \end{bmatrix}\begin{bmatrix}
            u_1&c_1\\ &u_2&\ddots\\ &&\ddots&c_{n-1}\\ &&&u_n
        \end{bmatrix},
    \]
    可直接乘开得到$u_1=b_1$
    \begin{equation}
        \ell_i=\frac{a_i}{u_{i-1}},\quad u_i=b_i-\ell_ic_{i-1}.
    \end{equation}
\end{theorem}


\section{稳定性分析}
\label{sec:stability analysis}

在用直接法求解$Ax=b$的过程中,由于舍入误差的存在,必然会导致结果产生误差。因而有必要对可能产生的误差作一估计。
% 通常我们假设在数值处理的过程中计算都是精确的.
\begin{example}
    {数据的微小变化导致解的巨大变化}{}
    方程组
    \[
        \begin{bmatrix}
            10&7&8&7\\
            7&5&6&5\\
            8&6&10&9\\
            7&5&9&10
        \end{bmatrix}\begin{bmatrix}
            x_1\\x_2\\x_3\\x_4
        \end{bmatrix}=\begin{bmatrix}
            32\\23\\33\\31
        \end{bmatrix}\implies\begin{bmatrix}
            x_1\\x_2\\x_3\\x_4
        \end{bmatrix}=\begin{bmatrix}
            1\\1\\1\\1
        \end{bmatrix}.
    \]
    对向量$b$数据做微小的修改
    \[
        \begin{bmatrix}
            10&7&8&7\\
            7&5&6&5\\
            8&6&10&9\\
            7&5&9&10
        \end{bmatrix}\begin{bmatrix}
            x_1'\\x_2'\\x_3'\\x_4'
        \end{bmatrix}=\begin{bmatrix}
            32.1\\22.9\\33.1\\30.9
        \end{bmatrix}\implies\begin{bmatrix}
            x_1'\\x_2'\\x_3'\\x_4'
        \end{bmatrix}=\begin{bmatrix}
            9.2\\-12.6\\4.5\\-1.1
        \end{bmatrix}.
    \]
    对系数矩阵$A$数据做微小的修改
    \[
        \begin{bmatrix}
            10&7&8.1&7.2\\
            7.08&5.04&6&5\\
            8&5.98&9.89&9\\
            6.99&4.99&9&9.98
        \end{bmatrix}\begin{bmatrix}
            x_1''\\x_2''\\x_3''\\x_4''
        \end{bmatrix}=\begin{bmatrix}
            32\\23\\33\\31
        \end{bmatrix}\implies\begin{bmatrix}
            x_1''\\x_2''\\x_3''\\x_4''
        \end{bmatrix}=\begin{bmatrix}
            -81\\137\\-34\\22
        \end{bmatrix}.
    \]
    可见数据的微小变化会导致解的巨大变化,这是因为系数矩阵的条件数$\cond(A)=32825/11$很大。
\end{example}

\begin{definition}
    {条件数}{condition number}
    给定诱导的矩阵范数$\norm\cdot$,可逆矩阵$A$的条件数(condition number)为
    \begin{equation}
        \cond(A)\equiv\norm A\nnorm{A\iv}.
    \end{equation}
\end{definition}

\begin{corollary}
    条件数的性质:
    \begin{itemize}
        \item $\cond(A)\geq 1$;
        \item $\cond(A\iv)=\cond(A)$;
        \item $\cond(cA)=\cond(A)$;
        \item 若$U$为正交矩阵,则$\cond_2(U)=1$,且
        \begin{equation}
            \cond_2(A)=\cond_2(AU)=\cond_2(UA);
        \end{equation}
        \item 若$\lambda_1,\lambda_n$是$A$模最大与最小的特征值,则
        \begin{equation}
            \cond(A)\geq\frac{\abs{\lambda_1}}{\abs{\lambda_n}},
        \end{equation}
        若$A$对称,则$\cond_2(A)=\abs{\lambda_1}/\abs{\lambda_n}$;
        \item 由范数的等价性,可知条件数的等价性:
        \begin{subequations}
            \begin{alignat}{4}
                \frac1n&\cond_2&(A)&\leq\cond_1&(A)&\leq n\cond_2&(A),\\
                \frac1n&\cond_\infty&(A)&\leq\cond_2&(A)&\leq n\cond_\infty&(A),\\
                \frac1{n^2}&\cond_1&(A)&\leq\cond_\infty&(A)&\leq n^2\cond_1&(A).
            \end{alignat}
        \end{subequations}
    \end{itemize}
\end{corollary}

\begin{theorem}
    {解的扰动定理}{}
    给定可逆矩阵$A$和微小扰动$\D A$,满足
    \[
        \frac{\norm{\D A}}{\norm A}<\frac1{\cond(A)},
    \]
    则$(A+\D A)$也可逆,考察线性方程组$Ax=b$及其扰动方程组
    \[
        (A+\D A)(x+\D x)=b+\D b,
    \]
    则有
    \begin{equation}
        \frac{\norm{\D x}}{\norm x}\leq\frac{\cond(A)}{1-\norm{A\iv}\norm{\D A}}\biggkh{\frac{\norm{\D A}}{\norm A}+\frac{\norm{\D b}}{\norm b}}.
    \end{equation}
\end{theorem}

\begin{proof}
    由扰动定理\thmref{thm:perturbation theorem II} 知$(A+\D A)$可逆且
    \begin{equation}
        \norm{(A+\D A)\iv}\leq\frac{\norm{A\iv}}{1-\norm{A\iv}\norm{\D A}}.
    \end{equation}
    由
    \begin{align*}
        \D x&=(A+\D A)\iv(b+\D b)-x\\
        &=(A+\D A)\iv(b+\D b-(A+\D A)x)\\
        &=(A+\D A)\iv(\D b-\D Ax),
    \end{align*}
    两边取范数
    \begin{align*}
        \norm{\D x}&\leq\norm{(A+\D A)\iv}(\norm{\D b}+\norm{\D A}\norm x)\\
        &\leq\frac{\norm{A\iv}}{1-\norm{A\iv}\norm{\D A}}\biggkh{\frac{\norm{\D A}}{\norm A}\norm A\norm x+\frac{\norm{\D b}}{\norm b}\norm A\norm x}.
        \qedhere
    \end{align*}
\end{proof}

\begin{remark}
    因此条件数可以看成扰动方程组相对误差的放大倍数。
\end{remark}

\begin{theorem}
    {矩阵相对奇异性的度量}{relative singularity}
    若$A$可逆,定义所有使得$(A+\D A)$不可逆的$\D A$构成集合$S$,则 
    \begin{equation}
        \min_{\D A\in S}\frac{\norm{\D A}_2}{\norm A_2}=\frac1{\cond_2(A)},
    \end{equation}
\end{theorem}

\begin{proof}
    当$\nnorm{A\iv}\norm{\D A}<1$时,$(A+\D A)$可逆,故
    \[
        \min_{\D A\in S}\norm{\D A}_2\geq\frac1{\norm{A\iv}_2}.
    \]
    由$\norm\cdot_2$的定义,$\exists x$且$\norm x_2=1$使得$\norm{A\iv x}_2=\norm{A\iv}_2$,令$y=A\iv x/\nnorm{A\iv}_2$,并取
    \[
        \D A=-\frac{xy\tp}{\norm{A\iv}_2},
    \]
    则$\norm y_2=1$且
    \[
        (A+\D A)y=\frac{x}{\norm{A\iv}_2}-\frac{xy\tp y}{\norm{A\iv}_2}=0,
    \]
    故$(A+\D A)$不可逆,又
    \[
        \norm{\D A}_2=\max_{\norm z_2=1}\norm{\D Az}_2=\frac{\norm x_2}{\norm{A\iv}_2}\max_{\norm z_2=1}\abs{y\tp z}=\frac1{\norm{A\iv}_2}.
        \qedhere
    \]
\end{proof}

\begin{remark}
    因此可逆矩阵到最接近的奇异矩阵的相对距离在2 - 范数意义下就是2 - 条件数的倒数。当条件数很大时,矩阵与奇异矩阵的相对距离很小,称为病态(ill conditioned)。
\end{remark}

\begin{theorem}
    {近似解的相对误差}{}
    若$x,x'$分别是方程组$Ax=b$的精确解和近似解,$x'$的剩余$r=b-Ax'$,则
    \begin{equation}
        \frac1{\cond(A)}\frac{\norm r}{\norm b}\leq\frac{\norm{x'-x}}{\norm x}\leq\cond(A)\frac{\norm r}{\norm b}.
    \end{equation}
\end{theorem}

\begin{proof}
    由$A(x'-x)=-r$和$\norm A\norm x\geq\norm b$可得
    \[
        \norm r\leq\norm A\norm{x'-x}=\frac{\cond(A)}{\norm{A\iv}}\norm{x'-x},
    \]
    两边除$\norm b$,由$x=A\iv b$可得 
    \[
        \frac{\norm r}{\norm b}\leq\cond(A)\frac{\norm{x'-x}}{\norm{A\iv}\norm b}\leq\cond(A)\frac{\norm{x'-x}}{\norm x};
    \]
    另一方面,由$x'-x=-A\iv r$可得
    \[
        \norm{x'-x}=\norm{A\iv r}\leq\norm{A\iv}\norm r=\cond(A)\frac{\norm r}{\norm A}.
    \]
    两边同除$\norm x$,由$Ax=b$可得 
    \[
        \frac{\norm{x'-x}}{\norm x}\leq\cond(A)\frac{\norm r}{\norm A\norm x}\leq\cond(A)\frac{\norm r}{\norm b}.
    \]
    综上,两边不等式均得证。
\end{proof}

\begin{remark}
    这说明当方程组病态时,即使剩余$\norm r$比较小,解的相对误差仍可能很大。
\end{remark}

\begin{example}
    {Hilbert矩阵}{Hilbert matrix}
    Hilbert矩阵
    \begin{equation*}
        H_n:=\begin{bmatrix}
            1&1/2&\cdots&1/(n+1)\\
            1/2&1/3&\cdots&1/(n+2)\\
            \vdots&\vdots&\ddots&\vdots\\
            1/(n+1)&1/(n+2)&\cdots&1/(2n+1)
        \end{bmatrix}
    \end{equation*}
    的条件数增长很快:
    \[
        \cond_2(H_n)=\bigo\biggkh{\frac{(1+\sqrt2)^{4n}}{\sqrt n}}.
    \]
\end{example}

\begin{theorem}
    {条件数与数值精度}{}
    用直接法解方程组$Ax=b$,$A,b$的元素有效位数为$s$而$\cond(A)$的数量级为$t$,则求得$x$分量有效位数约为$s-t$。
\end{theorem}

\paragraph{病态方程组的解法}

除采用更高精度的运算外,另一个更有效的方法是对原方程进行预处理:
\[
    Ax=b,\iff PAQ(Q\iv x)=Pb,
\]
从而降低系数矩阵的条件数:$\cond(PAQ)\ll\cond(A)$。一般$P,Q$可选择为三角矩阵或对角矩阵。

\begin{example}
    {预处理例子}{}
    方程组
    \[
        \begin{bmatrix}
            10&10^5\\1&1
        \end{bmatrix}\begin{bmatrix}
            x_1\\x_2
        \end{bmatrix}=\begin{bmatrix}
            10^5\\2
        \end{bmatrix}\implies\begin{bmatrix}
            x_1\\x_2
        \end{bmatrix}=\frac1{9999}\begin{bmatrix}
            10000\\9998
        \end{bmatrix}.
    \]
    系数矩阵的条件数$\cond_2(A)=100010$很大,左乘$D=\diag(10^{-5},1)$平衡:
    \[
        \begin{bmatrix}
            10^{-4}&1\\1&1
        \end{bmatrix}\begin{bmatrix}
            x_1\\x_2
        \end{bmatrix}=\begin{bmatrix}
            1\\2
        \end{bmatrix},
    \]
    系数矩阵的条件数$\cond_2(DA)=940/359$得到了有效降低。
\end{example}

\chapter{平面电磁波}
\label{chap:plane wave}

本章讨论无限或半无限介质中的平面波。首先论述了平面电磁波在非导电介质中的基本性质:平面电磁波的横向性、线偏振态和圆偏振态。然后推导了至关重要的电磁波在平面分界面上反射和折射的Fresnel公式,并加以应用,接着概括阐述了电介质、导体和等离子体的高频色散关系。%对200赫兹以上各种频率的电磁波画出了液态水的折射率和吸收系数对频率的关系曲线(图7.9),通过全面的图象分析,说明了许多性质。
而后简单地讨论了电磁波在电离层中的传播,和在导电介质或耗散介质中的波。继而介绍了相速度和群速度的概念,以及一个脉冲或波包在色散介质中传播时的扩展。%相当详细地讨论了因果性这一重要课题,以及由此而得到的有关介质色散性质的一些结果,包括克喇末-克朗尼格色散关系以及根据这些关系导出的各种求和法则.本章最后论述了关于信号在色散介质中的传播这一经典问题,此问题最早是由索末菲和布里渊(1914年)讨论过的,但只不过在最近才获得了实验的验证.

\section{非导电介质中的平面波}

电磁场Maxwell方程组的基本特征是存在行波解,这些解描写能量从一点输运到另一点.最简单最基本的电磁波是横平面波。%我们首先考虑在简单的(即$\varepsilon,\mu$是常数)非导电介质这种情形下如何求得这些解。

% 当没有源存在时,无限介质中的Maxwell方程组是:
% \begin{alignat*}{2}
%     \div\bm D&=0,&\qquad\curl\bm E+\pv{\bm B}t&=\bm 0,\\
%     \div\bm B&=0,&\curl\bm H-\pv{\bm D}t&=\bm 0.    
% \end{alignat*}

考虑无源介质($\rho=0,\;\bm J=\bm 0$),并假设场是随时间简谐的($\e{-\i\omega t}$),Maxwell方程为
\begin{subequations}
    \begin{align}
        \div\bm D&=0,\\
        \div\bm B&=0,\\
        \curl\bm E&=\i\omega\bm B,\\
        \curl\bm H&=-\i\omega\bm D. 
    \end{align}
\end{subequations}
对于均匀各向同性的线性介质,
\begin{align*}
    \bm D&=\varepsilon\bm E,\\
    \bm B&=\mu\bm H,
\end{align*}
%进一步假设没有色散,即$\varepsilon,\mu$是不随时间变化的正实数(无损耗),可以得到
可以推出,$\bm E,\bm H$均满足如下的Helmholtz波动方程:
\begin{equation}
    (\lapla+k^2)\begin{Bmatrix}
        \bm E\\\bm H
    \end{Bmatrix}=0,
\end{equation}
其波数(wave number)定义为
\begin{equation}
    k=\sqrt{\mu\varepsilon}\omega.
\end{equation}

Helmholtz方程的一个解是沿$z$方向传播的平面波$\e{\i(kz-\omega t)}$,其
相位为$\varphi=kz-\omega t$,等相位面移动的速度称为
相速度(phase velocity):
\begin{equation}
    v_\varphi:=\frac{\omega}k=\frac1{\sqrt{\mu\varepsilon}}=\frac{c}{\sqrt{\mu_\r\varepsilon_\r}}.
\end{equation}
% 对于非色散介质,$\varepsilon,\mu$与频率无关,有通解
% \[
%     u(x,t)=f(x-vt)+g(x+vt).
% \]

\paragraph{平面电磁波}

频率为$\omega$、波数为$k$、传播方向为$\uvec n$的平面电磁波一般具有如下形式:
\begin{equation}
    \begin{cases}
        \bm E=\bm E_0\e{\i(\bm k\cdot\bm r-\omega t)},\\
        \bm B=\bm B_0\e{\i(\bm k\cdot\bm r-\omega t)},
    \end{cases}
\end{equation}
其中$\bm E_0,\bm B_0$是常矢量,$\bm k:=k\uvec n$为波矢(wave vector)。
\begin{theorem}{}{}
    无源Maxwell方程组与下列方程组等价
    \begin{subequations}
        \begin{align}
            \lapla\bm E+k^2\bm E&=\bm 0,\\
            \div\bm E&=0,\\
            \curl\bm E&=\i\omega\bm B.
        \end{align}
    \end{subequations}
\end{theorem}
\begin{corollary}
    电磁波是横波(transverse wave),即$\bm E,\bm B$均垂直于传播方向$\uvec n$:
    \begin{subequations}
        \label{eqn:transverse}
        \begin{align}
            \bm k\cdot\bm E_0&=0,\\
            \bm k\cdot\bm B_0&=0,\\
            \bm k\times\bm E_0&=\omega\bm B_0.
        \end{align}
    \end{subequations}
    同时$\uvec n\cdot\uvec n=1$,注意:$\uvec n$可以是复矢量,此时$\abs{\uvec n}^2=\uvec n\cdot\uvec n^*\neq 1$。
\end{corollary}

\paragraph{实波矢}

若$\uvec n$是实矢量,则$\bm E_0$和$\bm B_0$有相同的相位:
\begin{subequations}
    \begin{align}
        \bm E_0&=E_0\uvec\epsilon_1,\\
        \bm B_0&=\sqrt{\mu\varepsilon}E_0\uvec\epsilon_2.
    \end{align}
\end{subequations}
其中
$\uvec n,\uvec\epsilon_1,\uvec\epsilon_2$是两两正交的单位矢量。
$\uvec n$称为传播矢量(propagation vector),
$\uvec\epsilon_1,\uvec\epsilon_2$称为极化矢量(polarization vector)。

能流的时间平均%单位时间流过单位面积的能量
由复Poynting矢量的实部给出:
\begin{equation}
    \bm S=\frac12\bm E\times\bm H\cj=\frac12\sqrt{\frac\varepsilon\mu}\abs{\bm E_0}^2\uvec n.
\end{equation}
相应地,电磁场能量密度的时间平均为
\begin{equation}
    u=\frac14\biggkh{\varepsilon\bm E\cdot\bm E\cj+\frac1\mu\bm B\cdot\bm B\cj}=\frac\varepsilon 2\abs{\bm E_0}^2.
\end{equation}
因此,对于实波矢的平面横波,能流的速度等于相速度:$\abs{\bm S}/u=v_\varphi$。

\paragraph{复波矢}

若$\uvec n$是复矢量,
% 分成实部和虚部两部分:
% \[
%     \uvec n=\Re(\uvec n)+\i\Im(\uvec n),
% \]
则电磁波为非均匀平面波,在虚部方向$\Im(\uvec n)$上呈指数衰减(或增长)
\[
    \e{\i(\bm k\cdot\bm r-\omega t)}=\e{-k\Im(\uvec n)\cdot\bm r}\e{\i(k\Re(\uvec n)\cdot\bm r-\omega t)},
\]
等幅面和等相面仍然是平面,但它们不再平行;横波的关系\eqref{eqn:transverse}仍成立,由$\uvec n\cdot\uvec n=1$:
\begin{subequations}
    \begin{align}
        \abs{\Re(\uvec n)}^2-\abs{\Im(\uvec n)}^2&=1,\\
        \Re(\uvec n)\cdot\Im(\uvec n)&=0.
    \end{align}
\end{subequations}

\section{偏振}

前文指出:均匀平面波的电场沿$\uvec\epsilon_1$方向,磁场沿$\uvec\epsilon_2$方向;但显然,一旦确定极化矢量$\uvec\epsilon_1,\uvec\epsilon_2$后,一个均匀平面波的电场当然也可以沿其他方向,一般地,
可以将任意均匀平面波分解成两种偏振方向的波:
%也可以有一种平面波的电场$\uvec\epsilon_2$方向,
%我们分别称其为偏振矢量为$\uvec\epsilon_{1,2}$的线偏振波,这两种偏振波是不同的。一般地,
\[
    \bm E=(E_1\uvec\epsilon_1+E_2\uvec\epsilon_2)\e{\i(\bm k\cdot\bm r-\omega t)},
\]
振幅$E_1,E_2$是复数,$E_2$相对$E_1$的相位差为$\delta$。
% 因为$t$的起始点是任意的,不妨令$E_1$是实数,则$E_2=\abs{E_2}\e{\i\delta}$。
% \[
%     E_1=a_1\e{\i\vd_1},\quad E_2=a_2\e{\i\vd_2}.
% \]
% 若选定$(\uvec n,\uvec\epsilon_1,\uvec\epsilon_2)=(\uvec z,\uvec x,\uvec y)$,取实部得到
% \begin{align*}
%     E_x&=a_1\cos(kz-\omega t+\vd_1),\\
%     E_y&=a_2\cos(kz-\omega t+\vd_2).
% \end{align*}

\paragraph{线偏振}

若$E_1,E_2$同相($\vd=0,180\degree$),则$\bm E$是线偏振的(linear polarized)。%$\bm E$与$\uvec\epsilon_1$的夹角是不变的,
%在$\uvec\epsilon_1,\uvec\epsilon_2$坐标系中,
如\figref{fig:linear polarization},
$\bm E$的终点在一条直线(图中虚线)上简谐运动。
细箭头表示$\bm E$终点运动方向,
\begin{center}
	\includegraphics[page=19]{figures/tikz/layouts.pdf}
    \includegraphics[width=0.5\linewidth]{figures/polarization_line.pdf}
	\captionof{figure}{线偏振($\delta=0$)}
    \label{fig:linear polarization}
\end{center}

\paragraph{圆偏振}

若$E_1,E_2$不同相,但幅值相同($\abs{E_1}=\abs{E_2}$)且相位差$\delta=\pm 90\degree$,则
$\bm E$是圆偏振的(circular polarized),可以写成
\[
    \bm E=E_1(\uvec\epsilon_1\pm\i\uvec\epsilon_2)\e{\i(\bm k\cdot\bm r-\omega t)},
\]
如\figref{fig:circular polarization},向$-\bm k$方向(即垂直纸面向内)看去,($+$)对应电场终点逆时针(counterclockwise)旋转,我们称为左旋(left circular)或正螺旋性(positive helicity),因为其角动量是沿$+\bm k$的;相应的,($-$)对应顺时针旋转,称为右旋或负螺旋性。
\begin{center}
    \includegraphics[page=20]{figures/tikz/layouts.pdf}
    \includegraphics[width=0.5\linewidth]{figures/polarization_circ.pdf}
	\captionof{figure}{圆偏振($\delta=+90\degree$)}
    \label{fig:circular polarization}
\end{center}
依此定义圆偏振基:
\begin{equation}
    \uvec\epsilon_\pm:=\frac1{\sqrt2}(\uvec\epsilon_1\pm\i\uvec\epsilon_2),
\end{equation}
则$\uvec n,\uvec\epsilon_\pm$也是两两正交的单位矢量(复矢量内积)。在这一对坐标下,任意平面波都可以分解成左旋和右旋波:
\[
    \bm E=(E_+\uvec\epsilon_++E_-\uvec\epsilon_-)\e{\i(\bm k\cdot\bm r-\omega t)},
\]
若$E_\pm$幅值相同,则$\bm E$是线偏振的;若$E_\pm$有一个为0,则$\bm E$是圆偏振的。

\paragraph{椭圆偏振}

若$E_\pm$幅值不同,则$\bm E$是椭圆偏振的(elliptically polarized)。$\bm E$的终点在一个椭圆上旋转,如\figref{fig:elliptical polarization} 所示。
\begin{center}
	\includegraphics[page=21]{figures/tikz/layouts.pdf}
    \includegraphics[width=0.5\linewidth]{figures/polarization_elli.pdf}
    \captionof{figure}{椭圆偏振}
    \label{fig:elliptical polarization}
\end{center}
为了方便地描述这个椭圆的方程,记$\Re(\bm E)=E_{\epsilon_1}\uvec\epsilon_1+E_{\epsilon_2}\uvec\epsilon_2$,则
\begin{align*}
    E_{\epsilon_1}&=\abs{E_1}\cos(\bm k\cdot\bm r-\omega t),\\
    E_{\epsilon_2}&=\abs{E_2}\cos(\bm k\cdot\bm r-\omega t+\delta),
\end{align*}
则
\[
    \Bigkh{\frac{E_{\epsilon_1}}{\abs{E_1}}}^2+\Bigkh{\frac{E_{\epsilon_2}}{\abs{E_2}}}^2-2\Bigkh{\frac{E_{\epsilon_1}}{\abs{E_1}}}\Bigkh{\frac{E_{\epsilon_2}}{\abs{E_2}}}\cos\delta=\sin^2\delta.
\]
若$E_-$相对$E_+$的相位差是$\alpha$,
则椭圆半长轴与$\uvec\epsilon_1$轴的夹角为$\alpha/2$,且满足
\[
    \tan\alpha=\frac{2\abs{E_1}\abs{E_2}}{\abs{E_1}^2-\abs{E_2}^2}\cos\delta.
\]
椭圆长短轴之比由$E_\pm$幅值之比给出:
\[
    \frac{E_{\max}}{E_{\min}}=\frac{\abs{E_+}+\abs{E_-}}{\abs{E_+}-\abs{E_-}}.%\quad r:=\frac{\abs{E_-}}{\abs{E_+}};
\]

\paragraph{Stokes参数}

我们如何从观察光束的细节来确定偏振的状态?对于一束光来说,
\[
    \uvec\epsilon_1\cdot\bm E,\quad\uvec\epsilon_2\cdot\bm E;\quad\uvec\epsilon_+\cj\cdot\bm E,\quad\uvec\epsilon_-\cj\cdot\bm E,
\]
分别是沿$\uvec\epsilon_{1,2}$方向线偏振的、左旋和右旋圆偏振的振幅。这些振幅的平方给出了各类偏振强度的一种测量。%为体现幅值和相位因素,记
%我们给出相对于线偏振基和相对于圆偏振基的斯托克斯参数的定义,把这些参数用投影振幅(7.25)来表示,并且还直接用诸分量的幅值和相对位相来表示.为了达到后一目的,我们把(7.19)和(7.24)中的各标量系数定义为一

\begin{definition}
    {Stokes参数}{Stokes parameter}
    在线偏振基$\uvec\epsilon_{1,2}$下,Stokes参数表示为
    \begin{subequations}
        \begin{align}
            s_0&=|\uvec\epsilon_1\cdot\bm E|^2+|\uvec\epsilon_2\cdot\bm E|^2=\abs{E_1}^2+\abs{E_2}^2,\\
            s_1&=|\uvec\epsilon_1\cdot\bm E|^2-|\uvec\epsilon_2\cdot\bm E|^2=\abs{E_1}^2-\abs{E_2}^2,\\
            s_2&=2\Re\bigfkh{(\uvec\epsilon_1\cdot\bm E)\cj(\uvec\epsilon_2\cdot\bm E)}=2\abs{E_1}\abs{E_2}\cos\delta,\\
            s_3&=2\Im\bigfkh{(\uvec\epsilon_1\cdot\bm E)\cj(\uvec\epsilon_2\cdot\bm E)}=2\abs{E_1}\abs{E_2}\sin\delta;
        \end{align}
    \end{subequations}
    在圆偏振基$\uvec\epsilon_\pm$下,Stokes参数表示为
    \begin{subequations}
        \begin{align}
            s_0&=|\uvec\epsilon_+\cj\cdot\bm E|^2+|\uvec\epsilon_-\cj\cdot\bm E|^2=\abs{E_+}^2+\abs{E_-}^2,\\
            s_1&=2\Re\bigfkh{(\uvec\epsilon_+\cj\cdot\bm E)\cj(\uvec\epsilon_-\cj\cdot\bm E)}=2\abs{E_+}\abs{E_-}\cos\alpha,\\
            s_2&=2\Im\bigfkh{(\uvec\epsilon_+\cj\cdot\bm E)\cj(\uvec\epsilon_-\cj\cdot\bm E)}=2\abs{E_+}\abs{E_-}\sin\alpha,\\
            s_3&=|\uvec\epsilon_+\cj\cdot\bm E|^2-|\uvec\epsilon_-\cj\cdot\bm E|^2=\abs{E_+}^2-\abs{E_-}^2.
        \end{align}
    \end{subequations}
\end{definition}
\begin{corollary}
    Stokes参数仅依赖于$\abs{E_1},\abs{E_2},\delta$或$\abs{E_+},\abs{E_-},\alpha$三个参数,因此并不独立:
    \[
        s_0^2=s_1^2+s_2^2+s_3^2.
    \]
\end{corollary}

\begin{table}[h]
    \centering
    \caption{典型偏振的Stokes参数}
    \begin{tabular}{ccc}
        \toprule
        偏振 & 取向 & Stokes参数\\
        \midrule
        \multirow{2}{*}{线偏振} & 水平(horizontal)和垂直(vertical) & $[1\enspace\pm1\enspace0\enspace0]$ \\
         & $\pm 45\degree$ & $[1\enspace0\enspace\pm1\enspace0]$ \\
        圆偏振 & 右旋和左旋 & $[1\enspace0\enspace0\enspace\pm1]$ \\
        \bottomrule
    \end{tabular}
    \label{tab:Stokes parameters}
\end{table}

现实中,即使是%目前应用的
单色性(monochromatic)很好的辐射波束也是由有限波列(finite wave trains)叠加成的,因而根据Fourier定理,它包含一个频率范围,即不完全是单色的。%一种看法认为$a_i,\vd_i$随时间变化缓慢,
这时可观测的Stokes参数就是在一个较长时间尺度上的平均,比如:
\[
    s_2=2\ave{a_1a_2\cos(\vd_2-\vd_1)},
\]
求平均的结果是:准单色波束的Stokes参数满足不等式:
\[
    s_0^2\geqslant s_1^2+s_2^2+s_3^2,
\]
无偏振(unpolarized)光或自然光(natural light):$\bm E$由无数个无规则取向的偏振光叠加,各方向振幅相同,因此$s_1=s_2=s_3=0$,Stokes参数为
\[
    [1\enspace0\enspace0\enspace0].
\]

\begin{definition}{偏振度}{}
    定义偏振度是总光强$I\tot$中椭圆偏振的成分$I_\text{elp}$占比:
    \[
        P:=\frac{I_\text{elp}}{I\tot}=\frac{\sqrt{s_1^2+s_2^2+s_3^2}}{s_0},
    \]
    显然$0\leqslant P\leqslant 1$,
\end{definition}

\begin{corollary}
    依偏振度分类:
    \begin{compactitem}
    	\item $P=1$对应完全偏振光;
    	\item $P=0$对应无偏振光;
    	\item $0<P<1$对应部分偏振光(partially polarized),这时尽管$\bm E$含有各种振动方向的光矢量,但在某一方向更显著,不难表达为无偏振光和完全偏振光的叠加:
        \[
            \begin{bmatrix}
                s_0\\s_1\\s_2\\s_3
            \end{bmatrix}_\text{pp}=(1-P)
            \begin{bmatrix}
                s_0\\0\\0\\0
            \end{bmatrix}_\text{unp}+P
            \begin{bmatrix}
                s_0\\s_1'\\s_2'\\s_3'
            \end{bmatrix}_\text{elp}.
        \]
    \end{compactitem}
\end{corollary}

\paragraph{Stokes参数的测量}

Stokes参数在场强上是二次的,只能通过强度测量来确定,并结合:
\begin{itemize}
    \item 线偏振器(linear polarizer):利用物质的材料特性让透射光只有沿着特定方向偏振的分量透过;
    \item 1/4波片(quarter waveplate):利用了晶体的双折射(birefringence)现象\footnote{一束光线进入某些晶体会产生两束折射光。改变入射角,其中一束光在晶体内的光速各向同性,称为o光(ordinary);而另一束光的光速各向异性,称为e光(extraordinary)。}使得o光和e光产生相位差,控制波片厚度便可以使线偏振光变成圆偏振光。
\end{itemize}
如果线偏振器与入射光偏振方向夹角为$\theta$,波片的相位差为$\phi$,则出射光的光强为
\[
    I(\theta,\phi)=\frac12(s_0+s_1\cos2\theta+s_2\sin2\theta\cos\phi-s_3\sin2\theta\sin\phi),
\]
由此可以利用线偏振器和1/4波片,通过测量4次出射光强来得到4个Stokes参数:
\begin{compactitem}
	\item 线偏振器$\theta=0$:$I(0,0)=(s_0+s_1)/2;$
    \item 线偏振器$\theta=90\degree$:$I(90\degree,0)=(s_0-s_1)/2;$
    \item 线偏振器$\theta=45\degree$:$I(45\degree,0)=(s_0+s_2)/2;$
    \item 线偏振器$\theta=45\degree+1/4$波片:$I(45\degree,45\degree)=(s_0-s_3)/2.$
\end{compactitem}

\begin{example}
    {Stokes参数的应用:蟹状星云辐射的偏振态}{the Crab nebula}
    在天体物理中,
    蟹状星云中的脉冲星发来的可见光频和射频(radiofrequency)辐射可以用Stokes参数描写偏振态。
    光频辐射显示出微小的线偏振,而$\omega\simeq\SI{2.5e9}{s^{-1}}$的射频辐射有高度的线偏振。但两个频段均没有圆偏振的迹象。
    这类资料有助于阐明这些迷人天体发出辐射的机理。
\end{example}

\section{反射与透射}

平面电磁波入射(incident)到不同媒质的分界面上会发生反射(reflect)和透射(transmit)\footnote{也称为折射(refract)。}。下面我们研究这种现象的规律。
\begin{center}
	\includegraphics[page=22]{figures/tikz/layouts.pdf}
	\captionof{figure}{入射波、反射波和透射波}
    \label{fig:refraction}
\end{center}

\paragraph{反射定律和折射定律}

入射波、反射波和透射波的波数分别为$k_\i,k_\r,k_\t$。由前文的结论,边界上电场的切向分量是连续的:
\[
    (\bm E_\i+\bm E_\r-\bm E_\t)\times\uvec n=\bm 0.
\]
上式对任意$t$均成立,因此光在任意介质中都具有相同的频率:
\begin{equation}
    \label{eqn:omegai=omegar=omegat}
    \omega_\i=\omega_\r=\omega_\t=\omega,
\end{equation}
% 式\eqref{eqn:omegai=omegar=omegat}说明,不论光怎样被散射,都具有相同的频率;
又上式对于分界面上的任意一点均成立,故$\bm E_\i,\bm E_\r,\bm E_\t$在分界面上任意一点的相位差均相同,故
% \[
%     \bm k_\i\cdot\bm r=\bm k_\r\cdot\bm r+\delta_\r=\bm k_\t\cdot\bm r+\delta_\t.
% \]
% 从前两项可以得到
% \[
%     (\bm k_\i-\bm k_\r)\cdot\bm r=\delta_\r,
% \]
% 由内积的投影性质,$\bm r$的终点会扫过一个垂直于$(\bm k_\i-\bm k_\r)$的平面,而$\bm r$的终点正好在分界面上,因此$(\bm k_\i-\bm k_\r)$垂直于分界面:
% \[
%     (\bm k_\i-\bm k_\r)\times\uvec n=\bm 0,
% \]
% 上述关系当然对第三项也适用,因此:
$\bm k_\i,\bm k_\r,\bm k_\t$三线共面,且三者的切向分量相等:
\begin{equation}
    k_\i\sin\theta_\i=k_\r\sin\theta_\r=k_\t\sin\theta_\t,
\end{equation}

\begin{theorem}{反射定律}{}
    入射波与反射波在同一介质中,当然有$k_\i=k_\r$,故入射角$\theta_\i$和反射角$\theta_\r$相等:
    \begin{equation}
        \theta_\i=\theta_\r;
    \end{equation}
\end{theorem}
入射波与折射波在不同介质中,角度的正弦与波数成反比:
\[
    \frac{\sin\theta_\i}{\sin\theta_\t}=\frac{k_\t}{k_\i},
\]
但二者频率相同,对波矢同乘$c/\omega$得到一个无量纲参数:
\begin{definition}
    {折射率}{index of refraction}
    将光速与相速度的比值称为折射率(index of refraction)
    \begin{equation}
        n:=\frac c{v_\varphi}\geqslant1,
    \end{equation}
\end{definition}
\begin{theorem}{Snell定律}{Snell's law}
    入射角$\theta_\i$和折射角$\theta_\t$的正弦与折射率成反比:
    \begin{equation}
        \label{eqn:Snell}
        n_\i\sin\theta_\i=n_\t\sin\theta_\t.
    \end{equation}
\end{theorem}

\paragraph{电磁分量的变化}

定义振幅的反射系数(reflection coefficient)和透射系数(transmission coefficient):
\[
    r:=\frac{E_\r}{E_\i},\enspace t:=\frac{E_\t}{E_\i}.
\]

\begin{theorem}
    {Stokes倒逆关系}{Stokes relation}
    定义$r_{ij},t_{ij}$分别表示从介质$i$入射介质$j$的反射系数和透射系数,由光路可逆
    \begin{center}
        \includegraphics[page=23]{figures/tikz/layouts.pdf}
        \includegraphics[page=24]{figures/tikz/layouts.pdf}
        \captionof{figure}{Stokes倒逆关系示意图}
    \end{center}
    可得
    \begin{subequations}
        \begin{gather}
            r_{21}=-r_{12}\\
            r_{12}^2+t_{12}t_{21}=1.
        \end{gather}
    \end{subequations}
\end{theorem}

由边界上:$\bm D$和$\bm B$法向分量连续,$\bm E$和$\bm H$切向分量连续。我们可以求出$r,t$的表达式。
根据入射面(入射光线与法线确定的平面)与$\bm E_\i$的关系可以分成两种独立情形:
\begin{figure}[!htp]
    \centering
    \subcaptionbox{垂直极化波}{\includegraphics[page=25]{figures/tikz/layouts.pdf}}
    \subcaptionbox{平行极化波}{\includegraphics[page=26]{figures/tikz/layouts.pdf}}
    \caption{垂直极化波和平行极化波}
    \label{fig:senkrecht and parallel polarization}
\end{figure}
% 在应用这些边界条件时,可以考虑线偏振的两种独立情况:
\begin{compactenum}
	\item 垂直极化波(senkrecht polarized):$\bm E_\i$与入射面垂直(即$\bm E_\i$垂直纸面向外)%\footnote{在讨论Fresnel公式之前,我们必须指定$\bm E_\i$的方向。}
	\begin{align*}
        E_\i+E_\r&=E_\t,\\
        H_\i\cos\theta_\i-H_\r\cos\theta_\r&=H_\t\cos\theta_\t,
    \end{align*}
    \item 平行极化波(parallel polarized):$\bm E_\i$与入射面平行(或者说$\bm H_\i$垂直纸面向外)
    \begin{align*}
        E_\i\cos\theta_\i-E_\r\cos\theta_\r&=E_\t\cos\theta_\t,\\
        H_\i+H_\r&=H_\t,
    \end{align*}
\end{compactenum}
电介质通常满足$\mu_\i=\mu_\r\approx\mu_\t\approx\mu_0$,因此$r,t$的表达式为:
\begin{theorem}
    {Fresnel公式}{Fresnel formula}
    对于垂直极化波,$1+r_\perp=t_\perp$:
    \begin{subequations}
        \begin{align}
            \label{eqn:Fresnel rperp}
            r_\perp&=\frac{n_\i\cos\theta_\i-n_\t\cos\theta_\t}{n_\i\cos\theta_\i+n_\t\cos\theta_\t},\\%=-\frac{\sin(\theta_\i-\theta_\t)}{\sin(\theta_\i+\theta_\t)},\\
            \label{eqn:Fresnel tperp}
            t_\perp&=\frac{2n_\i\cos\theta_\i}{n_\i\cos\theta_\i+n_\t\cos\theta_\t};%=\frac{2\sin\theta_\i\sin\theta_\t}{\sin(\theta_\i+\theta_\t)};
        \end{align}
    \end{subequations}
    对于平行极化波,$1+r_\parallel=t_\parallel\cdot n_\t/n_\i$:
    \begin{subequations}
        \begin{align}
            \label{eqn:Fresnel rpara}
            r_\parallel&=\frac{n_\t\cos\theta_\i-n_\i\cos\theta_\t}{n_\t\cos\theta_\i+n_\i\cos\theta_\t},\\
            \label{eqn:Fresnel tpara}
            t_\parallel&=\frac{2n_\i\cos\theta_\i}{n_\t\cos\theta_\i+n_\i\cos\theta_\t}.
        \end{align}
    \end{subequations}
\end{theorem}
    
利用Snell定律\eqref{eqn:Snell},$r,t$便完全由折射率之比$n_\i/n_\t$与入射角$\theta_\i$决定。

\begin{example}
    {Brewster角}{Brewster angle}
    对于平行极化波,存在一个特殊的入射角$\theta_\i$使得$r_\parallel=0$,%依\eqref{eqn:Fresnel rpara}
    % \[
    %     n_\t\cos\theta_\i-n_\i\cos\theta_\t=0,\quad n_\i\sin\theta_\i=n_\t\sin\theta_\t.
    %     %n_\t^2\cos i_\text B=n_\i\sqrt{n_\t^2-n_\i^2\sin^2 i_\text B},
    % \]
    称为Brewster角:
    \begin{equation}
        \label{eqn:Brewster}
        \theta_\text B=\arctan\Bigkh{\frac{n_\t}{n_\i}}.
    \end{equation}
    在Brewster角下,平行极化波可以发生全透射,因此加了偏振镜的相机可以拍出更通透的照片:
    \begin{itemize}
        \item 空气折射率$n_\i=1$,玻璃的折射率$n_\t=1.5$,得到$\theta_\text B=56.31\degree$;
        \item 水折射率$n_\t=4/3$,得到$\theta_\text B=53.13\degree$。
    \end{itemize}
    
\end{example}

\paragraph{全反射}

前面的讨论在绝大程度上只限于外反射($n_\i<n_\t$),即光疏介质入射光密介质。相反的情况是内反射($n_\i>n_\t$),这时$\theta_\i<\theta_\t$。
考虑$\theta_\t\geqslant90\degree$,即光疏介质没有折射光,我们称之为全反射(total internal reflection),由Snell定律,全反射的临界角
\begin{equation}
    \label{eqn:total reflect}
    \theta\c=\arcsin\Bigkh{\frac{n_\t}{n_\i}}.
\end{equation}
发生全反射时,$\theta_\i>\theta\c$,折射角的正弦
\[
    \sin\theta_\t=\frac{n_\i}{n_\t}\sin\theta_\i=\frac{\sin\theta_\i}{\sin\theta\c}>1,
\]
则$\cos\theta_\t$是纯虚的,
\[
    \e{\i\bm k'\cdot\bm r}=\e{\i k_\t(x\sin\theta_\t+z\cos\theta_\t)},
\]
此时折射波在光疏介质内仅平行于表面传播,并在垂直方向上呈指数衰减,Poynting矢量的法向分量在时间平均上为0:
\[
    \bm S\cdot\uvec n=\frac12\Re[(\bm E\times\bm H\cj)\cdot\uvec n]=\frac1{2\omega\mu'}\Re[(\bm k'\cdot\uvec n)|\bm E'_0|^2]=0.
\]

\paragraph{Goos-Haenchen效应}

发生全反射时,光疏介质中的
倏逝波(evanescent wave)在垂直方向上以$\e{-z/\delta}$衰减,其中
\[
    \delta\iv:=k'\sqrt{\sin^2\theta_\i-\sin^2\theta\c}.
\]
如果有限横向范围内的光束经历全反射,则反射光将相对几何反射光束出现横向偏移。对于线偏振光,偏移的距离称为Goos-Haenchen偏移
\[
    D\simeq 2\delta\sin\theta_\i.
\]

\paragraph{Fresnel菱体}

发生全反射时,
反射光相对入射光的相位差为$n_\t\cos\theta_\t$,是一个纯虚数,与入射角$\theta_\i$和折射率$n_\i,n_\t$有关,这种相位差可以用来将一种偏振态的光转化为另一种偏振态,Fresnel菱体(rhombus)便是以此为原理,将垂直于表面入射的线偏振光转换为出射的圆偏振光。

经过复杂的计算,如果要实现此效果,需要满足
\[
    \cos^2\theta_\i+\frac{n_\t^2}{n_\i^2}\sin^2\theta_\i-1=\pm\,2(n_\t^2-n_\i^2)\cos\theta_\i^2\sqrt{\frac{n_\t^2}{n_\i^2}\sin^2\theta_\i-1},
\]
玻璃的折射率$n=1.5$,因此内反射角位$50.2\degree$或$53.3\degree$即可满足条件。

% \paragraph{垂直入射}
% 如果光是垂直界面入射的,$\theta_\i=\theta_\t=0$,则 
% \begin{align*}
%     r&=\frac{n_\i-n_\t}{n_\i+n_\t},\\
%     t&=\frac{2n_\i}{n_\i+n_\t}.
% \end{align*}

\section{色散}

所有介质都表现出一定的色散性,只有在有限的频率范围内,或在真空中,光的传播速度才能被视为常量。

色散的应用:能量啁啾(chirped pulse amplification)。
\paragraph{色散的简单模型}
我们忽略外加电场和局部电场之间的区别。因此,该模型只适用于密度相对较低的物质。
忽略磁场力,
由谐波力束缚和由电场作用的电荷电子的运动方程为:
\[
    -m_\elc\omega_0^2\bm r-m_\elc\gamma\dot{\bm r}-e\bm E=m_\elc\ddot{\bm r}.
\]
$-m_\elc\omega_0^2\bm r$是回复力(restoring force),$-m_\elc\gamma\dot{\bm r}$是唯象的阻尼力(damping force)。
如果振荡的振幅足够小,便可以计算电子平均位置的电场。场随频率$\omega$的简谐$\e{-\i\omega t}$变化。
则由一个电子贡献的偶极矩为:
\[
    \bm p=-e\bm r=\frac{e^2}{m_\elc(\omega_0^2-\omega^2-\i\omega\gamma)}\bm E.
\]
从微观到宏观,单位体积内有$N$个分子,每个分子有$Z$个电子;我们将分子中的电子分类,第$j$组有$f_j$个特性相同的电子,其结合频率为$\omega_j$,阻尼常数为$\gamma_j$。
代入$\bm P=\varepsilon_0\chi_\elc\bm E$和$\varepsilon_\r=1+\chi_\elc$得到
\begin{equation}
    \varepsilon_\r(\omega)=1+\frac{Ne^2}{\varepsilon_0m_\elc}\sum_j\frac{f_j}{\omega_j^2-\omega^2-\i\omega\gamma_j},
\end{equation}
在量子力学中,$f_j,\omega_j,\gamma_j$都有合适的定义,因此上面的方程准确地描述了原子对介电常数的贡献。我们称$f_j$为振子强度(oscillator strength),$\omega_j$为共振频率(resonant frequency)。

通常$\gamma_j\ll\omega_j$,上式可约化为
\[
    \varepsilon_\r(\omega)\simeq 1+\frac{Ne^2}{\varepsilon_0m_\elc}\sum_j\frac{f_j}{\omega_j^2-\omega^2},
\]
对于$\omega<\min(\omega_j)$的低频率下,和的贡献是正的,$\varepsilon_\r>1$;随着频率的增加,它将穿过越来越多的$\omega_j$,求和将从正到负,最后$\varepsilon_\r<1$。

\paragraph{反常色散和共振吸收}

当$\omega$在共振频率$\omega_j$附近时,
\[
    \varepsilon_\r(\omega)\simeq 1+\i\frac{Ne^2}{\varepsilon_0m_\elc}\sum_j\frac{f_j}{\omega\gamma_j},
\]
$\varepsilon_\r$的虚部是可观的,由于$\varepsilon_\r$的正虚部表示电磁波向介质中产生的能量耗散,因此$\Im[\varepsilon_\r(\omega)]$较大的区域称为共振吸收(resonant absorption)区域。

回忆\secref{ssec:Poynting in dissipative media}中线性色散介质中的Poynting定律\eqref{eqn:Poynting with loss},考虑平面波的衰减(attenuation),波数是一个复数
\[
    k=\beta+\i\frac\alpha 2.
\]
其中$\alpha$是衰减常数或吸收系数,波的强度随$\e{-\alpha z}$衰减,且$\alpha\ll\beta$。由
\[
    k\simeq\frac\omega c\sqrt{\varepsilon_\r},\implies
    \begin{cases}
        \alpha\simeq\frac{\Im(\varepsilon_\r)}{\Re(\varepsilon_\r)}\beta,\\
        \beta\simeq\frac\omega c\sqrt{\Re(\varepsilon_\r)}.
    \end{cases}
\]

\paragraph{低频表现,电导}

在低频极限$\omega\to0$下,如果每个分子中电子的某些$f_0$是自由的,即$\omega_0=0$,则$\varepsilon_\r$在$\omega=0$处是奇异的。

如果自由电子的贡献被分别显示出来,
\[
    \varepsilon(\omega)=\varepsilon_\text b(\omega)+\i\frac{Ne^2f_0}{m_\elc\omega(\gamma_0-\i\omega)},
\]
其中$\varepsilon_\text b$体现其他偶极子的贡献。

\begin{definition}{电导率}{conductivity}
    引入电导率(conductivity) $\sigma$的概念,体现电流密度与电场强度的关系:
    \begin{equation}
        \bm J=\sigma\bm E,
    \end{equation}
    这也称为Ohm定律($I=U/R$)的微观表达式。
\end{definition}
式\eqref{eqn:Maxwell curlH freq}变为
\[
    \curl\bm H=\sigma\bm E-\i\omega\bm D.
\]
而$\bm D=\varepsilon_\text b\bm E$,
故电导率也是一个复数:
\begin{equation}
    \sigma=\frac{f_0Ne^2}{m_\elc(\gamma-\i\omega)},
\end{equation} 
\begin{example}{铜的电导率}{conductivity of copper}
    铜的参数为
    \begin{align*}
        N&=\SI{8e28}{/m^3},\\
        \sigma&=\SI{5.9e7}{/\ohm.m}.
    \end{align*}
    由此得到$\gamma_0/f_0\simeq\SI{4e13}{s^{-1}}$,而$f_0\sim 1$,因此当$\omega\leqslant\SI{e11}{s^{-1}}$时,铜的导电率是实数,并且与频率无关。
\end{example}

\paragraph{高频极限,等离子体频率}

在$\omega\gg\max(\omega_j)$的高频下,
\[
    \varepsilon_\r(\omega)\simeq 1-\frac{\omega_\text p^2}{\omega^2},
\]
$\omega_\text p$是介质的等离子体频率(plasma frequency)
\begin{equation}
    \omega_\text p^2=\frac{NZe^2}{\varepsilon_0m_\elc},\quad Z=\sum_jf_j.
\end{equation}
仅取决于总电子密度$NZ$。
\begin{example}{地球电离层的短波全反射}{ionisphere}
    电离层中,电子均是自由的且阻尼可忽略不计。对于$\omega<\omega_\text p$的情形,$\varepsilon_\r\simeq 1-\omega_\text p^2/\omega^2$的关系仍成立,此时
    \[
        ck=\omega\sqrt{\varepsilon_\r}=\sqrt{\omega^2-\omega_\text p^2}.
    \]
    是纯虚的。因此低频电磁波会被电离层反射。

    地球电离层的电子密度$NZ=\SIrange{e18}{e22}{/m^3}$,故等离子体频率$\omega_\text p\simeq\SIrange{6e10}{6e12}{s^{-1}}$,$\omega=0$的衰减常数$\alpha^{-1}=\SIrange{0.2}{2e-3}{cm}$。
\end{example}
\begin{example}{金属的紫外透射现象}{ultraviolet transparency}
    对于$\omega\ll\omega_\text p$,光被完全反射;当频率$\omega$增加到$\varepsilon_\r(\omega)>0$的区间,金属便突然可以透射光,其反射率变化得很剧烈(drastically),这称为紫外透射(ultraviolet transparency)现象。
\end{example}

\paragraph{水的折射率和吸收系数随频率的变化}

在可见光范围内,水的折射率$n\sim 1.34$,吸收系数$\alpha$骤降至一个很低的值。

\sectionstar{在电离层和磁层中的传播模型}

电磁波对等离子体中的传播会受到外部磁场的显著影响。

略

\section{群速度}

如果介质是色散的,则波的每个频率分量的相速度都不相同,能流的速度可能与相位的速度有很大的不同,甚至缺乏精确的意义。

\paragraph{一维标量波}

一维中的标量波的通解为
\[
    u(x,t)=a\e{\i(kx-\omega t)}+b\e{\i(-kx-\omega t)},
\]
由色散性质,$\omega=\omega(k)$是波数的函数。将标量波表示成Fourier变换的形式
\[
    u(x,t)=\frac1{\sqrt{2\pi}}\int\iti A(k)\e{\i(kx-\omega t)}\d k,
\]
$A(k)$描述了不同波的线性叠加的性质,
\[
    A(k)=\frac1{\sqrt{2\pi}}\int\iti u(x,0)\e{-\i kx}\d x,
\]
对于一个单色行波,$u(x,0)\sim\e{\i k_0x}$,则$A(k)=\sqrt{2\pi}\vd(k-k_0)$,从而$u(x,t)=\e{\i(k_0x-\omega_0t)}$。

\paragraph{有限波列、群速度}

对于一个有限波列(wave train),$x,k$平均值的均方根偏差$\D x,\D k$满足Fourier带宽(bandwidth)定律
\begin{equation}
    \D x\D k\geqslant\frac12,
\end{equation}
如果一个脉冲的波数谱不太广,或者一个介质中的频率对波数的依赖性较弱,则$\omega(k)$可以被展开
\[
    \omega(k)=\omega_0+\edg{\dv\omega k}_0(k-k_0)+\cdots
\]
得到 
\[
    u(x,t)\simeq u\Bigkh{x-t\edg{\dv\omega k}_0,0}\exp\biggfkh{\i\Bigkh{k_0\edg{\dv\omega k}_0-\omega_0}t},
\]
除了一个整体的相位因子,脉冲以一种没有失真的(undistorted)速度传播,称为群速度(group velocity)
\begin{equation}
    v_\text g:=\edg{\dv\omega k}_0,
\end{equation}
对于光波,以折射率$n(k)$表示更方便:
\[
    \omega(k)=\frac{ck}{n(k)},
\]
相速度
\[
    v_\varphi=\frac{\omega(k)}k=\frac c{n(k)},
\]
群速度
\[
    v_\text g=c\biggkh{\frac1n-\frac k{n^2}\dv nk}=\frac cn-\frac\omega n\dv n\omega v_\text g,
\]
得到 
\begin{equation}
    \label{eqn:gruop v}
    v_\text g=\frac c{\displaystyle n(\omega)+\omega\dv n\omega},
\end{equation}

\paragraph{群速度可以大于光速吗?}

随着频率$\omega$的增大,经过一系列共振频率,折射率$n(\omega)$将在1附近出现上升和下降的变化,我们称上升段为正常色散,下降段为反常色散(anomalous dispersion),由群速度$v_\text g$和$\d n/\d\omega$的关系\eqref{eqn:gruop v},在反常色散中群速度是可以超越光速的,但这并不违反相对论,个人的解释是这种群速度是波包的速度,并不传递信息。\footnote{唐老师课上讲的词忘了。}

\paragraph{进一步讨论群速度的有效性}

考虑频率对波数依赖的特定模型,并在不近似的情况下计算脉冲在该模型介质中的传播。$t=0$时标量波的初始值和初始变化率为$u(x,0),\dot u(x,0)$,则 
\[
    A(k)=\frac1{\sqrt{2\pi}}\int\iti\biggfkh{u(x,0)+\i\frac{\dot u(x,0)}{\omega(k)}}\e{-\i kx}\d x,
\]
\begin{example}{Gauss波包}{Gaussian modulated oscillation}
    考虑初始为Gauss调制振荡:
    \[
        u(x,0)=\e{-x^2/2L^2}\cos(k_0x),
    \]
    若假设$\dot u(x,0)=0$,则
    \[
        u(x,t)=\e{-x^2/2L^2}\cosh(kx-\omega t);
    \]
    \tcblower
    假设 
    \[
        \omega(k)=v\Bigkh{1+\frac{a^2k^2}2},
    \]
    则群速度
    \[
        v_\text g=\dv\omega k(k_0)=va^2k_0.
    \]
\end{example}

\section{\texorpdfstring{$\bm D$}{D}和\texorpdfstring{$\bm E$}{E}之间的因果关系}

静态电磁场
\[
    D_\alpha=\varepsilon_0E_\alpha+P_\alpha-\sum_\beta\pv{\mathcal Q'_{\alpha\beta}}{x_\beta}+\cdots
\]
若考虑时间,
会发生什么变化?

\paragraph{时间非局域性}

在色散介质中,$\varepsilon(\omega)$对频率$\omega$的依赖会导致$\bm D(\bm r,t)$和$\bm E(\bm r,t)$的临时性非局域联系。对于$\omega$的单色成分,
\[
    \hatbm D(\bm r,\omega)=\varepsilon(\omega)\hatbm E(\bm r,\omega),
\]
Fourier变换得到
\[
    \bm D(\bm r,t)=\frac1{2\pi}\int\iti\varepsilon(\omega)\biggfkh{\int\iti\bm E(\bm r,t')\e{\i\omega t'}\d t'}\e{-\i\omega t}\d\omega,
\]
假设积分的次序可以交换,可以得到
\begin{equation}
    \label{eqn:D-E with omega}
    \bm D(\bm r,t)=\varepsilon_0\biggfkh{\bm E(\bm r,t)+\int\iti\bm E(\bm r,t-\tau)G(\tau)\d\tau},
\end{equation}
其中 
\[
    G(\tau)=\frac1{2\pi}\int\iti\bigfkh{\varepsilon_\r(\omega)-1}\e{-\i\omega\tau}\d\omega,
\]
因此$t$时刻的$\bm D(\bm r,t)$依赖除$t$以外的电场$\bm E(\bm r,t-\tau)$。若$\varepsilon$是不依赖于$\omega$的,则 
\[
    G(\tau)=(\varepsilon_\r-1)\vd(\tau),\quad\bm D(\bm r,t)=\varepsilon\bm E(\bm r,t),
\]
$\bm D(\bm r,t)$和$\bm E(\bm r,t)$有瞬时的联系;但如果$\varepsilon(\omega)$随$\omega$变化,则$G(\tau)$在$\tau\neq 0$的一些地方也不会消失。

\paragraph{$G(\tau)$的简单模型}

假设所有的电子都以一个频率$\omega_0$振荡,则 
\[
    \varepsilon_\r(\omega)=1+\frac{\omega_\text p^2}{\omega_0^2-\omega^2-\i\gamma\omega},
\]
即 
\[
    G(\tau)=\frac{\omega_\text p^2}{2\pi}\int\iti\frac{\e{-\i\omega\tau}\d\omega}{\omega_0^2-\omega^2-\i\gamma\omega}.
\]
被积函数在$\omega$的下半平面上有两个奇点
\[
    \omega_\pm=-\i\frac\gamma 2\pm\nu_0,\quad\nu_0^2=\omega_0^2-\frac{\gamma^2}4.
\]
利用复变函数中对留数的处理,得到
\[
    G(\tau)=\omega_\text p^2\frac{\sin\nu_0\tau}{\nu_0}\e{-\gamma\tau/2},\quad\tau>0.
\]
若没有上述假设,$G(\tau)$也仅仅是上式的线性组合。因此,$\bm D(\bm r,t)$和$\bm E(\bm r,t)$之间连接上的非局域性被限制在$\gamma\iv$的时间内。而$\gamma\iv$的典型值$\SI{e-9}s$。

上面的方程在时间(而非空间)上是非局部的。

\paragraph{$\varepsilon(\omega)$的因果性和分析性域}

$\tau<0$时$G(\tau)=0$。说明只有在$t$时间之前的电场$\bm E$进入以确定电位移$\bm D$,这符合我们在物理现象中的因果关系的基本思想。

假设:空间局域、线性、因果关系,均匀各向同性介质。

\newcommand{\PV}{\mathop{}\!\mathrm{P.V.}}

\paragraph{Kramers-Kronig关系}
由Cauchy定理
\begin{align*}
    \varepsilon_\r(\omega)-1&=\frac1{2\pi\i}\oint_C\frac{\varepsilon_\r(\omega')-1}{\omega'-\omega}\d\omega'\\
    &=\frac1{2\pi\i}\lim_{\delta\to0^+}\int\iti\frac{\varepsilon_\r(\omega')-1}{\omega'-\omega-\i\delta}\d\omega'
\end{align*}
由Sokhotski-Plemelj定理
\[
    \lim_{\delta\to0^+}\frac1{x\pm\i\delta}=\PV\Bigkh{\frac1x}\mp\i\pi\vd(x),
\]
故
\[
    \varepsilon_\r(\omega)=1+\frac1{\pi\i}\PV\int\iti\frac{\varepsilon_\r(\omega')-1}{\omega'-\omega}\d\omega'
\]
由$\Re[\varepsilon_\r(\omega)]$为偶,$\Im[\varepsilon_\r(\omega)]$为奇,得到Kramers-Kronig关系
\begin{equation}
    \begin{aligned}
        \Re[\varepsilon_\r(\omega)]&=1+\frac1\pi\PV\int\iti\frac{\Im[\varepsilon_\r(\omega)]}{\omega'-\omega}\d\omega'\\
        \Im[\varepsilon_\r(\omega)]&=-\frac1\pi\mathop{}\!\mathrm{P.V.}\int\iti\frac{\Re[\varepsilon_\r(\omega)]-1}{\omega'-\omega}\d\omega'\\
    \end{aligned}
\end{equation}
或
\begin{equation}
    \begin{aligned}
        \Re[\varepsilon_\r(\omega)]&=1+\frac2\pi\PV\int\zti\frac{\omega'\Im[\varepsilon_\r(\omega)]}{\omega'^2-\omega^2}\d\omega'\\
        \Im[\varepsilon_\r(\omega)]&=-\frac{2\omega}\pi\PV\int\zti\frac{\Re[\varepsilon_\r(\omega)]-1}{\omega'^2-\omega^2}\d\omega'\\
    \end{aligned}
\end{equation}
Kramers-Kronig关系具有非常普遍的有效性,这仅仅是根据极化和电场之间的因果关系\eqref{eqn:D-E with omega}的假设而得出的。

对于$\omega=\omega_0$处的一条非常窄的吸收线或吸收带,由
\begin{align*}
    \Im[\varepsilon_\r(\omega)]&\simeq\frac{\pi K}{2\omega_0}\vd(\omega'-\omega_0)+\cdots\\
    \Re[\varepsilon_\r(\omega)]&\simeq\bar\varepsilon+\frac K{\omega_0^2-\omega^2},
\end{align*}
后略。


\chapter{波导}
\label{chap:waveguide}
本章我们研究金属边界下的电磁场,这是一个相当重要且实用的题材。因为
在波长为米量级(或更短)的高频情况下,只有利用线度与波长相当的金属结构才能有效产生和发送电磁波。
% 在更高(红外)频率下,介电光纤在电信行业中被广泛应用。
我们在本章里先考虑在一个导体附近的场,并讨论场对表面的穿透以及伴随的电阻损失。
然后讨论了一般的波导和谐振腔中的电磁波等问题,并进行了具体说明。
用两种不同观点讨论波导中的衰减和谐振腔的$Q$值。
% 其次把地球-电离层系统当作谐振腔处理,接着简短地讨论电介质波导。我们还阐述波导中任意场的简正模展开,并应用到定域源产生的场。本章最后把简正模展开进一步应用到用变分法处理波导中障碍物问题。

\section{导体表面和内部的场}
理想导体(perfect conductor)和超导体(superconductor)的异同?
最大的不同是不存在理想导体,但存在超导体。

\paragraph{理想导体}

% 首先考虑一个法向量n的表面,从一侧的完美导体向外进入另一侧的非导电介质。

假设理想导体内部的电荷会随着场的变化瞬间移动,则理想导体内部是没有电场的。其表面电荷密度$\varsigma$ (此处符号与电导率$\sigma$区分)和表面电流密度$\bm\jmath$由理想导体外的$\bm D$和$\bm H$给出:
\begin{subequations}
    \begin{align}
        \nvec n\cdot\bm D&=\varsigma,\\
        \nvec n\times\bm H&=\bm\jmath.
    \end{align}
\end{subequations}
法向量$\nvec n$指向理想导体外部。其余两个边界条件为:
\begin{subequations}
    \begin{align}
        \nvec n\cdot(\bm B-\bm B\c)&=0,\\
        \nvec n\times(\bm E-\bm E\c)&=\bm 0.
    \end{align}
\end{subequations}
下标c表示导体。
在理想导体表面,仅存在法向的$\bm E$和切向的$\bm H$,并且这两个场在理想导体内骤降为0。

\paragraph{良导体}

对于非理想的良导体,导体内的场随趋肤深度
\begin{equation}
    \delta:=\sqrt{\frac2{\mu\c\omega\sigma}}.
\end{equation}
指数衰减,且
其电导率$\sigma$是一个有限量,而Ohm定律$\bm J=\sigma\bm E$说明表面没有电流$\bm\jmath$:
% 边界条件应改为
\[
    \uvec n\times(\bm H-\bm H\c)=\bm 0,
\]
由于场在垂直于表面的方向上的空间变化比平行方向快得多,因此和法向导数相比,切向导数可以忽略。若$\xi$表示导体向内的法向坐标,则
\[
    \nabla\simeq-\uvec n\pp\xi.
\]
忽略导体内的位移电流,谐振场的Maxwell方程变为
\begin{align*}
    \bm E\c&\simeq\frac1\sigma\curl\bm H\c\simeq-\frac1\sigma\uvec n\times\pv{\bm H\c}\xi,\\
    \bm H\c&=-\i\frac1{\mu\c\omega}\curl\bm E\c\simeq\i\frac1{\mu\c\omega}\uvec n\times\pv{\bm E\c}\xi.
\end{align*}
变形得到
\begin{align*}
    \Bigkh{\pp[2]\xi+\i\frac2{\delta^2}}(\uvec n\times\bm H\c)&\simeq\bm 0,\\
    \uvec n\cdot\bm H\c&\simeq 0.
\end{align*}
第二个方程说明:导体内$\bm H$平行于表面,这与边界条件相符。解得
\begin{equation}
    \bm H\c=\bm H_\parallel\e{-\xi/\delta}\e{\i\xi/\delta}.
\end{equation}
$\bm H_\parallel$是导体外表面的切向磁场,则导体内电场
\begin{equation}
    \bm E\c\simeq\sqrt{\frac{\mu\c\omega}{2\sigma}}(1-\i)(\uvec n\times\bm H_\parallel)\e{-\xi/\delta}\e{\i\xi/\delta}.
\end{equation}
故导体内的$\bm H\c$和$\bm E\c$与表面平行,大小随指数迅速衰减,二者有相位差,且磁场比电场强得多。在表面外侧,除了法向的$\bm E$和切向的$\bm H$外,还有$\bm E$的一个微小的切向分量存在。这意味着有功率流入导体,每单位面积的吸收对时间平均的功率是
\begin{equation}
    \dv{P_\text{loss}}a=-\frac12\Re[\uvec n\cdot(\bm E\times\bm H\cj)]=\frac{\mu\c\omega\delta}4|\bm H_\parallel|^2.
\end{equation}
可把这个结果简单地解释为导体内的Ohm损失,由Ohm定律,导体表面附件有电流密度$\bm J=\sigma\bm E\c$,而
\[
    \dv{P_\text{loss}}V=\frac12\bm J\cdot\bm E\cj=\frac\sigma 2|\bm J|^2.
\]
定义有效面电流密度
\[
    \bm K\eff:=\int\zti\bm J\d\xi=\uvec n\times\bm H_\parallel.
\]
则
\[
    \dv{P_\text{loss}}a=\frac1{2\sigma\delta}|\bm K\eff|^2.
\]
\section{柱形空腔和波导}
本章研究电磁波在中空金属柱体内的传播和激发。
如果柱体具有端面,就叫做空腔(cavity),否则就叫做波导(waveguide)。
假定边界面是理想导体,在实际应用中发生的能量损失,可用上一节的方法来估计。

假定柱形曲面$S$的截面形状和大小沿轴不变,且柱体内是介电常数$\varepsilon$、磁导率$\mu$的均匀非耗散介质。当柱体内的场具有谐振关系$\e{-\i\omega t}$时,Maxwell方程组可化为以下形式:
\begin{subequations}
    \begin{align}
        \div\bm E&=0,\\
        \div\bm B&=0,\\
        \curl\bm E&=\i\omega\bm B,\\
        \curl\bm B&=-\i\mu\varepsilon\omega\bm E.
    \end{align}
\end{subequations}
得到Helmholtz方程:
\[
    (\lapla+\mu\varepsilon\omega^2)\bm E=\bm 0,
\]
由边界限制,单独考虑轴线($z$轴)方向(称为纵向)上的空间变化:
\[
    \bm E(x,y,z,t)=\bm E(x,y)\e{\i(\pm k_zz-\omega t)},
\]
通过线性组合可以给出纵向的行波或驻波。
目前,纵向波数$k_z$是一个未知参数,它可能是实数也可能是复数。
假定了场对$z$有上述依赖关系后,波动方程就简化为二维形式:
\begin{equation}
    \label{eqn:lapla_tE}
    (\lapla_\t+\mu\varepsilon\omega^2-k_z^2)\bm E=0,
\end{equation}
其中$\lapla_\t$表示Laplace算符的横向部分:
\[
    \lapla=:\lapla_\t+\pp[2]z,
\]
将场分解为横向和纵向两个分量$\bm E=\bm E_\t+\bm E_z$,有
\begin{subequations}
    \begin{align}
        \bm E_\t&=(\uvec z\times\bm E)\times\uvec z,\\
        \bm E_z&=E_z\uvec z,
    \end{align}
\end{subequations}
Maxwell方程组变为
\begin{subequations}
    \label{eqn:EtEzBtBz}
    \begin{align}
        \nabla_\t\cdot\bm E_\t+\pv{E_z}z&=0,\\
        \nabla_\t\cdot\bm B_\t+\pv{B_z}z&=0,\\
        \nabla_\t E_z-\pv{\bm E_\t}z&=\i\omega\uvec z\times\bm B_\t,\\
        \uvec z\cdot(\nabla_\t\times\bm E_\t)&=\i\omega\bm B_z,\\
        \nabla_\t B_z-\pv{\bm B_\t}z&=-\i\mu\varepsilon\omega\uvec z\times\bm E_\t,\\
        \uvec z\cdot(\nabla_\t\times\bm B_\t)&=\i\mu\varepsilon\omega\bm E_z.
    \end{align}
\end{subequations}
结合纵向波数的关系式\eqref{eqn:lapla_tE},$\bm E_\t,\bm B_\t$可以被$E_z,B_z$确定
\begin{subequations}
    \label{eqn:EtBt=EzBz}
    \begin{align}
        \label{eqn:Et=EzBz}
        \bm E_\t&=\i\frac1{\mu\varepsilon\omega^2-k_z^2}(k_z\nabla_\t E_z-\omega\uvec z\times\nabla_\t B_z),\\
        \label{eqn:Bt=EzBz}
        \bm B_\t&=\i\frac1{\mu\varepsilon\omega^2-k_z^2}(k_z\nabla_\t B_z+\mu\varepsilon\omega\uvec z\times\nabla_\t E_z),
    \end{align}
\end{subequations}
\paragraph{TEM模式}
在讨论中空柱体内可以存在的各种场之前,我们先说一说简并型或特殊型解,即所谓横电磁(TEM)波。这种解只有垂直于传播方向的横场分量,$E_z=B_z=0$且
\begin{subequations}
    \begin{align}
        \nabla_\t\cdot\bm E_\text{TEM}&=0,\\
        \nabla_\t\times\bm E_\text{TEM}&=\bm 0,
    \end{align}
\end{subequations}
因此$\bm E_\text{TEM}$是二维静电问题的一个解。纵向波数就是无限介质中的波数
\begin{equation}
    \label{eqn:TEM k0}
    k_z=k_0:=\omega\sqrt{\mu\varepsilon},
\end{equation}
磁场 
\[
    \bm B_\text{TEM}=\pm\sqrt{\mu\varepsilon}\uvec z\times\bm E_\text{TEM},
\]
电导率为无穷大的单个中空柱形导体内不可能存在TEM模。为了运载TEM模,必须有两个及以上的等势面(比如同轴线)。TEM模的一个重要性质是不存在截止频率,即任意$\omega$均可以传播($k_z$均是实数)。
\paragraph{TE模式和TM模式}
横电(TE)波和横磁(TM)波是在中空柱体内(以及高频时的传输线上)出现两类场的构型,
可以通过纵向分量$E_z$和$B_z$满足的波动方程以及边界条件得到。
对于理想导电柱体,边界上电场为法向,磁场为切向:
\begin{subequations}
    \begin{align}
        \uvec n\times\bm E&=\bm0,\\
        \uvec n\cdot\bm B&=0,
    \end{align}
\end{subequations}
% 对于给定频率$\omega$,波数$k$只可能是特定值(波导情况);或对于给定波数$k$,频率$\omega$只能是特定值(谐振腔)。
由于$E_z$和$B_z$边界条件不同,所以本征值一般是不同的,场自然可分为两种模式:
\begin{compactitem}
	\item TM模式:磁场为横向($B_z=0$),边界条件:
    \begin{equation}
        E_z|_S=0;
    \end{equation}
	\item TE模式:电场为横向($E_z=0$),边界条件:
    \begin{equation}
        \edg{\pv{B_z}n}_S=0,
    \end{equation}
\end{compactitem}
\section{波导}
% 当波在一个均匀截面的中空波导中传播时,可得
由\eqref{eqn:EtEzBtBz},
TM波和TE波的横磁场和横电场的关系如下:
\[
    \bm H_\t=\pm\uvec z\times\frac{\bm E_\t}Z,
\]
式中$Z$定义为波阻抗(wave impedance):
\[
    Z:=\begin{cases}
        \frac{k_z}{\varepsilon\omega}=\frac{k_z}{k_0}\sqrt{\frac\mu\varepsilon},&\text{TM}\\
        \frac{\mu\omega}{k_z}=\frac{k_0}{k_z}\sqrt{\frac\mu\varepsilon},&\text{TE}
    \end{cases}
\]
这里$k_0$由\eqref{eqn:TEM k0}给出。
特别地,自由空间阻抗(impedance of free space)为
\begin{equation}
    Z_0=\sqrt{\frac{\mu_0}{\varepsilon_0}}=\mu_0c\simeq 120\pi\,\si{\ohm}
\end{equation}
横向场可由纵向场决定
\begin{subequations}
    \label{eqn:TE/TM field psi}
    \begin{alignat}{2}
        \label{eqn:TM field psi}
        \text{TM}&:\quad&\bm E_\t&=\pm\i\frac{k_z}{k\c^2}\nabla_\t\psi,\quad E_z=\psi\e{\pm\i k_zz};\\
        \label{eqn:TE field psi}
        \text{TE}&:\quad&\bm H_\t&=\pm\i\frac{k_z}{k\c^2}\nabla_\t\psi,\quad H_z=\psi\e{\pm\i k_zz}.
    \end{alignat}
\end{subequations}
其中$k\c^2=\mu\varepsilon\omega^2-k_z^2$。
标量函数$\psi$满足二维波动方程
\begin{equation}
    \label{eqn:laplat+kc2}
    (\lapla_\t+k\c^2)\psi=0,
\end{equation}
根据边界条件:
\begin{alignat*}{2}
    \text{TM}&:&\psi|_S&=0;\\
    \text{TE}&:&\quad\edg{\pv\psi n}_S&=0.
\end{alignat*}
可以确定一系列本征值$k_{\mathrm ci}$和本征函数$\psi_i$,这些不同的解称为波导的模式(mode)。
\begin{definition}
    {截止频率}{cutoff frequency}
    定义单个模式的截止频率(cutoff frequency):
    \begin{equation}
        \label{eqn:cutoff omega}
        \omega\c:=\frac{k\c}{\sqrt{\mu\varepsilon}},
    \end{equation}
    相应的波长$\lambda\c$称为截止波长。
\end{definition}
对于给定的频率$\omega$,单个模式对应的波数
\begin{equation}
    k_z=\sqrt{\mu\varepsilon}\sqrt{\omega^2-\omega\c^2}.
\end{equation}
截止频率的意义为:
\begin{compactitem}
    \item 当$\omega>\omega\c$时,$k_z^2>0$,模式可以在波导中传播;
    \item 当$\omega<\omega\c$时,$k_z^2<0$,模式不能在波导中传播。
\end{compactitem}
因此给定频率上,只有有限个模式可以传播。
可以通过选择波导的尺寸,使得在使用频率上只能出现最低模式的波。

波导中的波长$k_z$总是小于无限空间中的波长$k_0$,因此相速度$v_\varphi$大于无限空间中的光速:
\[
    v_\varphi=\frac\omega{k_z}=\frac1{\sqrt{\mu\varepsilon}}\frac1{\sqrt{1-(\omega\c/\omega)^2}}>\frac1{\sqrt{\mu\varepsilon}}.
\]
\subsection{矩形波导}
\label{ssec:rectangular waveguide}
在内尺寸为$a,b$的矩形波导中(不失一般性取$a>b$),
\begin{center}
    \begin{tikzpicture}[even odd rule]
        \fill[pattern=north east lines](-.5, -.4)rectangle(4.5, 2.4)(0, 0)rectangle(4, 2);
        \draw[thick, fill=gray!50](4, 0)node[below]{$a$}rectangle(0, 2)node[left]{$b$};
        \node at (2, 1) {$\mu,\varepsilon$};
        \draw[<->](0, 3)node[left]{$y$}--(0, 0)node[left]{$O$}--(5, 0)node[right]{$x$};
    \end{tikzpicture}
    \captionof{figure}{矩形波导}
    \label{fig:rectangle waveguide}
\end{center}
波动方程\eqref{eqn:laplat+kc2}为:
\begin{equation}
    \Bigkh{\pp[2]x+\pp[2]y+k\c^2}\psi=0,
\end{equation}
分离变量$\psi(x,y)=X(x)Y(y)$可得
\begin{align*}
    X(x)&=A\cos(k_xx)+B\sin(k_xx),\\
    Y(y)&=C\cos(k_yy)+D\sin(k_yy).
\end{align*}
满足$k\c^2=k_x^2+k_y^2$,故
\begin{equation}
    k_x^2+k_y^2+k_z^2=k_0^2=\mu\varepsilon\omega^2.
\end{equation}

\paragraph{TE模式}

$E_z=0,\,H_z=\psi$,边界上$\p\psi/\p n=0$可得
\begin{equation}
    \psi_{mn}=A\cos(k_xx)\cos(k_yy),\quad k_x=\frac{m\pi}a,\;k_y=\frac{n\pi}b.
\end{equation}
截止频率为
\begin{equation}
    \omega_{mn}=\frac\pi{\sqrt{\mu\varepsilon}}\sqrt{\Bigkh{\frac ma}^2+\Bigkh{\frac nb}^2}.
\end{equation}
则TE模式的最低截止频率为TE$_{10}$,对应的
\[
    \omega_{10}=\frac\pi{\sqrt{\mu\varepsilon}a},\quad\lambda_{10}=2a\frac{\sqrt{\mu\varepsilon}}c.
\]
% 在自由空间(即波导中没有介质)中正好$\lambda_{10}=2a$。
由\eqref{eqn:TE field psi}和\eqref{eqn:Et=EzBz},场的表达式为:
\begin{subequations}
    \begin{align}
        H_x&=-\i\frac{k_zk_x}{k\c^2}H_0\sin(k_xx)\cos(k_yy)\e{\i(k_zz-\omega t)},\\
        H_y&=-\i\frac{k_zk_y}{k\c^2}H_0\cos(k_xx)\sin(k_yy)\e{\i(k_zz-\omega t)},\\
        H_z&=H_0\cos(k_xx)\cos(k_yy)\e{\i(k_zz-\omega t)},\\
        E_x&=-\i\frac{\omega\mu k_y}{k\c^2}H_0\cos(k_xx)\sin(k_yy)\e{\i(k_zz-\omega t)},\\
        E_y&=\i\frac{\omega\mu k_x}{k\c^2}H_0\sin(k_xx)\cos(k_yy)\e{\i(k_zz-\omega t)},\\
        E_z&=0.
    \end{align}
\end{subequations}
虚数单位$\i$说明传播方向上$\bm H_\t,\bm E_\t$与$H_z$有$90\degree$的相位差。

\paragraph{TM模式}
$H_z=0,\,E_z=\psi$,边界上$\psi=0$可得
\begin{equation}
    \psi_{mn}=A\sin(k_xx)\sin(k_yy),\quad k_x=\frac{m\pi}a,\;k_y=\frac{n\pi}b.
\end{equation}
截止频率表达式同TM模,但由于$\psi\not\equiv 0$,故$m,n$均不能为0,最低模式为TM$_{11}$。
\begin{subequations}
    \begin{align}
        H_x&=-\i\frac{\omega\varepsilon k_y}{k\c^2}E_0\sin(k_xx)\cos(k_yy)\e{\i(k_zz-\omega t)},\\
        H_y&=\i\frac{\omega\varepsilon k_x}{k\c^2}E_0\cos(k_xx)\sin(k_yy)\e{\i(k_zz-\omega t)},\\
        H_z&=0,\\
        E_x&=\i\frac{k_zk_x}{k\c^2}E_0\cos(k_xx)\sin(k_yy)\e{\i(k_zz-\omega t)},\\
        E_y&=\i\frac{k_zk_y}{k\c^2}E_0\sin(k_xx)\cos(k_yy)\e{\i(k_zz-\omega t)},\\
        E_z&=E_0\sin(k_xx)\sin(k_yy)\e{\i(k_zz-\omega t)}.
    \end{align}
\end{subequations}

\subsection{圆波导}

在内径为$a$的圆形波导中,
\begin{center}
    \begin{tikzpicture}[even odd rule]
        \coordinate(x)at(2, 0);
        \coordinate(P)at(60:1.5);
        \fill[pattern=north east lines](O)circle(2)(O)circle(2.2);
        \draw[thick, fill=gray!50](O)circle(2);
        \node at (0, -1) {$\mu,\varepsilon$};
        \draw[dashed](O)--(x)node[midway, below]{$a$};
        % \draw(O)--(P)node[midway, left]{$\rho$};
        % \pic["$\phi$", draw, angle radius=20, angle eccentricity=1.5]{angle=x--O--P};
    \end{tikzpicture}
    \captionof{figure}{圆波导}
    \label{fig:circle waveguide}
\end{center}
采用柱坐标,波动方程\eqref{eqn:laplat+kc2}为:
\begin{equation}
    \Bigkh{\pp[2]\rho+\frac1\rho\pp\rho+\frac1{\rho^2}\pp[2]\phi+k\c^2}\psi=0,
\end{equation}
分离变量$\psi(\rho,\phi)=R(\rho)\Phi(\phi)$可得
\begin{align*}
    R(\rho)&=AJ_m(k\c\rho)+BN_m(k\c\rho),\\
    \Phi(\phi)&=\e{\pm\i m\phi}.
\end{align*}
其中$J_m,N_m$分别是第一、二类Bessel函数,由$R(0)<\infty$知$B=0$。由于极轴的选取是任意的,取$D=0$。

\paragraph{TM模式}

$H_z=0,\,E_z=\psi$,边界上$\psi=0$可得
\begin{equation}
    \psi_{mn}=AJ_m(k\c\rho)\e{\pm\i m\phi},\quad k\c=\frac{x_{mn}}a.
\end{equation}
其中$x_{mn}$表示$J_m$的第$n$个正零点。
场的表达式为
\begin{subequations}
    \begin{align*}
        H_\rho&=\pm m\frac{\omega\varepsilon k_z}{k\c^2\rho}E_0J_m(k\c\rho)\e{\i(k_zz-\omega t\pm m\phi)},\\
        H_\phi&=\i\frac{\omega\varepsilon k_z}{k\c}E_0J_m'(k\c\rho)\e{\i(k_zz-\omega t\pm m\phi)},\\
        H_z&=0,\\
        E_\rho&=\i\frac{k_z}{k\c}E_0J_m'(k\c\rho)\e{\i(k_zz-\omega t\pm m\phi)},\\
        E_\phi&=\mp m\frac{k_z}{k\c^2}E_0J_m(k\c\rho)\e{\i(k_zz-\omega t\pm m\phi)},\\
        E_z&=E_0J_m(k\c\rho)\e{\i(k_zz-\omega t\pm m\phi)}.
    \end{align*}
\end{subequations}

\paragraph{TE模式}

$E_z=0,\,H_z=\psi$,边界上$\p\psi/\p n=0$可得
\begin{equation}
    \psi_{mn}=AJ_m(k\c\rho)\e{\pm\i m\phi},\quad k\c=\frac{x_{mn}'}a.
\end{equation}
其中$x_{mn}'$表示$J_m'$的第$n$个正零点。
场的表达式为
\begin{subequations}
    \begin{align*}
        H_\rho&=\i\frac{k_z}{k\c}H_0J_m'(k\c\rho)\e{\i(k_zz-\omega t\pm m\phi)},\\
        H_\rho&=\mp m\frac{k_z}{k\c^2}H_0J_m(k\c\rho)\e{\i(k_zz-\omega t\pm m\phi)},\\
        H_z&=H_0J_m(k\c\rho)\e{\i(k_zz-\omega t\pm m\phi)},\\
        E_\rho&=\pm m\frac{\omega\mu k_z}{k\c^2\rho}H_0J_m(k\c\rho)\e{\i(k_zz-\omega t\pm m\phi)},\\
        E_\phi&=\i\frac{\omega\mu k_z}{k\c}H_0J_m'(k\c\rho)\e{\i(k_zz-\omega t\pm m\phi)},\\
        E_\rho&=0.
    \end{align*}
\end{subequations}

\paragraph{前几个$x_{mn}$和$x_{mn}'$分布}
如图
\begin{figure}[!htp]
    \centering
    \subcaptionbox{$J_m(x)$的零点分布}
        {\includegraphics[width=0.45\linewidth]{graphs/BesselJzero.pdf}}
    \subcaptionbox{$J_m'(x)$的零点分布}
        {\includegraphics[width=0.45\linewidth]{graphs/BesselJpzero.pdf}}
    \caption{$J_m(x)$和$J_m'(x)$的零点分布}
    % \label{fig:Bessel function}
\end{figure}

查表知,TE和TM的最低模式分别为TM$_{01}$和TE$_{11}$。
\begin{table}[!htp]
    \centering
    \caption{$J_m(x)$和$J_m'(x)$的零点}
    \subcaptionbox{$J_m(x)$的零点$x_{mn}$}{
    \begin{tabular}{crrr}
        \toprule
        $x_{mn}$&$n=1$&$n=2$&$n=3$\\
        \midrule
        $m=0$&2.405&5.520&8.654\\
        $m=1$&3.832&7.016&10.173\\
        $m=2$&5.136&8.417&11.620\\
        $m=3$&6.380&9.761&13.015\\
        \bottomrule
    \end{tabular}}
    \quad
    \subcaptionbox{$J_m'(x)$的零点$x_{mn}'$}{
    \begin{tabular}{crrr}
        \toprule
        $x_{mn}'$&$n=1$&$n=2$&$n=3$\\
        \midrule
        $m=0$&3.832&7.016&10.176\\
        $m=1$&1.841&5.331& 8.536\\
        $m=2$&3.054&6.706& 9.970\\
        $m=3$&4.201&8.015&11.346\\
        \bottomrule
    \end{tabular}}
\end{table}

\subsection{介质加载波导}

\paragraph{慢波结构}
带电粒子与电磁波交换能量(加速/减速)要求电磁波的相速度与电子的速度相同。
因此需要波导中的相速度小于等于光速,称为慢波结构。
由于带电粒子在介质中的速度超过介质光速时,会产生Cherenkov辐射,故无法采用全部填充介质的波导。本小节考虑部分填充介质的波导。
\begin{center}
    \begin{tikzpicture}[even odd rule]
        \fill[pattern=north east lines](-.5, -.4)rectangle(4.5, 2.4)(0, 0)rectangle(4, 2);
        \fill[gray!50](0, 0)rectangle(1.6, 2);
        \node at (1.6, 0) [below] {$d$};
        \draw[thick](4, 0)node[below]{$a$}rectangle(0, 2)node[left]{$b$};
        \node at (0.8, 1) {$\mu_1,\varepsilon_1$};
        \node at (2.8, 1) {$\mu_2,\varepsilon_2$};
        \draw[<->](0, 3)node[left]{$y$}--(0, 0)node[left]{$O$}--(5, 0)node[right]{$x$};
    \end{tikzpicture}
    \captionof{figure}{部分填充介质的矩形波导}
    \label{fig:rectangle waveguide part-filled}
\end{center}
% 后面我们将说明不存在TM模式,
考虑TE模式,根据金属边界条件,两部分的场形式为:
\begin{alignat*}{2}
    H_{z1}&=A\cos(k_{x1}x)\cos(k_{y1}y)\e{\i k_zz},&k_{y1}&=\frac{n_1\pi}b,\\
    H_{z2}&=B\cos(k_{x2}(a-x))\cos(k_{y2}y)\e{\i k_zz},&\quad k_{y2}&=\frac{n_2\pi}b.
\end{alignat*}
且
\begin{subequations}
    \begin{align}
        k_{x1}^2+k_{y1}^2+k_z^2=\mu_1\varepsilon_1\omega^2=:k_1^2,\\
        k_{x2}^2+k_{y2}^2+k_z^2=\mu_2\varepsilon_2\omega^2=:k_2^2,
    \end{align}
\end{subequations}
由$H_z$在介质分界$x=d$处连续,可得$n_1=n_2=n$;当$n\neq 0$时,结合
\[
    H_{yi}=\i\frac{k_z}{k_i^2-k_z^2}\pv{H_{zi}}y
\]
在$x=d$处连续,可得$\varepsilon_1\mu_1=\varepsilon_2\mu_2$。即$\varepsilon_1\mu_1\neq\varepsilon_2\mu_2$时,TE$_{mn}$不存在。
% 由$E_y,H_y,H_z$在介质分界$x=d$处连续,可得$n_1=n_2=n$且$n\neq 0$时,
但是TE$_{m0}$可以存在,此时$H_y\equiv 0$,再结合
\[
    E_{yi}=-\i\frac{\omega\mu_i}{k_i^2-k_z^2}\pv{H_{zi}}x
\]
在$x=d$处连续,可得
\begin{equation}
    \frac{\mu_1}{k_{x1}}\tan(k_{x1}d)=\frac{\mu_2}{k_{x2}}\tan(k_{x2}(a-d)),
\end{equation}
再结合
\begin{equation}
    k_{x1}^2-k_{x2}^2=(\mu_1\varepsilon_1-\mu_2\varepsilon_2)\omega^2,
\end{equation}
便可在给定$\omega$下求得$k_{x1},k_{x2}$和$k_z$。
另一方面,也可以令$k_z=0$求得截止频率满足的方程:
\begin{equation}
    \sqrt{\frac{\mu_1}{\varepsilon_1}}\tan(\omega\c\sqrt{\mu_1\varepsilon_1}d)=
    \sqrt{\frac{\mu_2}{\varepsilon_2}}\tan(\omega\c\sqrt{\mu_2\varepsilon_2}(a-d)).
\end{equation}
上式可以通过作图求出交点。显然,TE$_{m0}$的截止频率位于分别完全填充介质1 ($d=a$)和介质2 ($d=0$)的截止频率之间:
\[
    \frac{m\pi}{a\sqrt{\mu_1\varepsilon_1}}<\omega\c<\frac{m\pi}{a\sqrt{\mu_2\varepsilon_2}},
\]

考虑介质2为真空,
当相速度$v_\varphi$等于电子速度(电子在一定能量下速度可以非常接近真空光速)时,即$v_\varphi=\omega/k_z=c$,
有$k_{x2}=0$。

\subsection{波导中的能流和衰减}

我们扩大前面对任意截面形状的柱形波导所作的一般讨论,使它包括沿波导的能流以及波的衰减,后者是当电导率有限时由波导管壁中能量损耗所引起的。我们只限于讨论波导中只存在一种模式的情况;不过将简短提一下简并模式。能流用复Poynting矢量描写;
\begin{equation}
    \bm S=\frac12\bm E\times\bm H\cj=\frac{\omega k}{2\gamma^4}
    \begin{cases}
        \varepsilon\Bigkh{\uvec z|\nabla_\t\psi|^2+\i\frac{\gamma^2}k\psi\nabla_\t\psi\cj},&\text{TM}\\[1ex]
        \mu\Bigkh{\uvec z|\nabla_\t\psi|^2-\i\frac{\gamma^2}k\psi\cj\nabla_\t\psi},&\text{TE}
    \end{cases}
\end{equation}
因为$\psi$一般是实数,\footnote{有可能这样激发一个波导,使得某给定模式或一些模式的线性组合具有复数$\psi$。这时候横向能流对时间的平均值将不为零,但因为它是一个环流,所以实际上只代表贮藏的能量,在实用上不大重要。}
因此$\bm S$的第二项代表无功能流,并且对能流的时间平均没有贡献。总功率流
\[
    P=\int_A\bm S\cdot\uvec z\d a=\frac{\omega k}{2\gamma^4}
    \begin{Bmatrix}
        \varepsilon\\\mu
    \end{Bmatrix}
    \biggkh{\oint_{\p A}\cancel{\psi\cj\pv\psi n}\d\ell-\int_A\psi\cj\nabla_\t^2\psi\d a}.
\]
由特征方程$(\nabla_\t^2+\gamma_\lambda^2)\psi=0$,
\[
    P=\frac1{2\sqrt{\mu\varepsilon}}\biggkh{\frac\omega{\omega_\lambda}}^2\biggkh{1-\frac{\omega_\lambda^2}{\omega^2}}^{1/2}
    \begin{Bmatrix}
        \varepsilon\\\mu
    \end{Bmatrix}
    \int_A\psi\cj\psi\d a.
\]
时间平均能量密度
\[
    u=\frac14\Bigkh{\varepsilon\bm E\cdot\bm E\cj+\frac1\mu\bm B\cdot\bm B\cj},
\]
单位长度的场能量
\[
    U=\frac12\biggkh{\frac\omega{\omega_\lambda}}^2
    \begin{Bmatrix}
        \varepsilon\\\mu
    \end{Bmatrix}
    \int_A\psi\cj\psi\d a,
\]
能流的速度正是群速度
\[
    \frac PU=\frac k\omega\frac1{\mu\varepsilon}=\frac1{\sqrt{\mu\varepsilon}}\sqrt{1-\frac{\omega_\lambda^2}{\omega^2}}=v_\text g.
\]
结合前面说的相速度
\begin{equation}
    v_\text gv_\varphi=\frac1{\mu\varepsilon},
\end{equation}
在无限介质中,$v_\text g$总是小于$v_\varphi$,并且在截止频率时降为零。
\paragraph{良导体的波导}
对于理想导体,$k_\lambda$是实数或纯虚数;而对于电导率有限的良导体来说,$k_\lambda$是一个实数与一个小复数的和。

后面的略。
\paragraph{最小衰减频率}
略。
\section{边界条件的扰动}
略
\section{谐振腔}
一个谐振腔(resonant cavity)可以是任何形状的,具有一个封闭的导体表面。但通常我们将端面放置在一定长度的圆柱形波导上,以产生一个空腔。端面是垂直于圆柱体轴线的平面。

% 由$z=0,d$的边界条件,应有驻波形式:
% \begin{align*}
%     \text{TM}&:&E_z=\psi(x,y)\cos\Bigkh{\frac{p\pi z}d},&&p&=0,1,2,\ldots,\\
%     \text{TE}&:&H_z=\psi(x,y)\sin\Bigkh{\frac{p\pi z}d},&&p&=1,2,3,\ldots,
% \end{align*}
% 特征值频率
% \[
%     \omega_{\lambda_p}^2=\frac1{\mu\varepsilon}\Bigfkh{\gamma_\lambda^2+\Bigkh{\frac{p\pi}d}^2},
% \]

% 对于TM模,$\psi=E_z$,
% \[
%     \psi(\rho,\phi)=E_0J_m(\gamma_{mn}\rho)\e{\pm\i m\phi},\quad\gamma_{mn}=\frac{x_{mn}}R,
% \]
% 其中$x_{mn}$表示$J_m(x)$的第$n$个零点:

% 谐振频率为
% \begin{equation}
%     \omega_{mnp}=\frac1{\sqrt{\mu\varepsilon}}\sqrt{\Bigkh{\frac{x_{mn}}R}^2+\Bigkh{\frac{p\pi}d}^2}.
% \end{equation}

% 对于TE模,$\psi=H_z$,
% \[
%     \gamma_{mn}=\frac{x'_{mn}}R,
% \]
% 其中$x'_{mn}$表示$J'_m(x)$的第$n$个零点:

% 谐振频率为
% \begin{equation}
%     \omega_{mnp}=\frac1{\sqrt{\mu\varepsilon}}\sqrt{\Bigkh{\frac{x'_{mn}}R}^2+\Bigkh{\frac{p\pi}d}^2}.
% \end{equation}
\section{腔内的功率损失}
对于理想导体,$|E(\omega)|$在各个$\omega_i$上是一个个$\delta$函数;而对于良导体,则会有一定的半高宽。

定义品质因子(quality factor)
\begin{equation}
    Q:=\omega_0\frac{\text{stored energy}}{\text{power loss}}.
\end{equation}


\chapter{多极辐射}
\label{chap:multipole radiation}
在第七章和第八章里,讨论了电磁波的性质,以及电磁波在有界的和无界的区域内的传播情况。但是,很少谈到如何产生这些电磁波的问题。本章就转入这个问题,并讨论定域振荡电荷和电流密度系统辐射的电磁波,我们的论述是直截了当的,而不去精心推敲数学形式。%当然,这样做只限于比较简单的辐射系统,我们把求系统辐射的渐近法(用任意$l$的矢量多极场)推迟到第十六章里论述。本章只讨论电偶极子、磁偶极子和电器极子,以及导体上电流的一些简单位形。也论述波导中一个源的简单多极子展开和孔的有效多极矩。

%本章后半部,用较大篇幅讨论散射和衍射。首先解释长波长情形下的散射,包括瑞利对蓝天的解释和有关的论题,然后讨论标量衍射理论和矢量衍射理论,并举了一些例子。最后讨论短波长情形下的散射和重要的光学定理。
\section{局域振荡源}
对于随时间变化的电荷和电流系统,我们都可以将随时间变化的量作Fourier分析,并分别处理各个Fourier分量。因此,不失一般性,我们只考虑随时间作正弦变化的局域电荷和电流系统所产生的势、场和辐射,设
\begin{align*}
    \rho(\bm r,t)&=\rho(\bm r)\e{-\i\omega t},\\
    \bm J(\bm r,t)&=\bm J(\bm r)\e{-\i\omega t},
\end{align*}
矢量势
\begin{align}
    \notag
    \bm A(\bm r,t)&=\frac{\mu_0}{4\pi}\int\frac{\bm J(\bm r',t')}{\abs{\bm r-\bm r'}}\vd\Bigkh{t'-t+\frac{\abs{\bm r-\bm r'}}c}\d t'\nd v'\\
    \label{eqn:A-Jeiwt}
    &=\frac{\mu_0}{4\pi}\int\frac{\bm J(\bm r')}{\abs{\bm r-\bm r'}}\e{\i k\abs{\bm r-\bm r'}}\d V'.
\end{align}
磁场和电场为
\[
    \bm H=\frac1{\mu_0}\curl\bm A,\quad\bm E=\i\frac Zk\curl\bm H.
\]
给定电流分布$\bm J(\bm r')$以后,至少在原则上可以通过计算\eqref{eqn:A-Jeiwt}的积分来定出场的分布。%9.4节将讨论一两个直接计算源积分的实例.
但是,现在我们先确立在电流源被局限于小区域内(与波长相比,确实很小)的极限情形下,场所具有的某些简单而普遍的性质。如果源的线度为$d$,波长为$\lambda=2\pi c/\omega$,并且$d<\lambda$,则有三个令人感兴趣的空间区域:近区、中间区、远区。

近区:$d\ll r\ll\lambda$,\eqref{eqn:A-Jeiwt}中的指数可以用1代替,并用球谐函数\eqref{eqn:1/|r-r'|=YY}展开,
\[
    \lim_{kr\to0}\bm A(\bm r)=\frac{\mu_0}{4\pi}\sum_{\ell=0}^\infty\sum_{m=-\ell}^\ell\frac{4\pi}{2\ell+1}\frac{Y_{\ell m}(\theta,\phi)}{r^{\ell+1}}\int\bm J(\bm r')r'^\ell Y\cj_{\ell m}(\theta',\phi')\d V'.
\]
上式表明,近场是准静态的,除了按$\e{-\i\omega t}$方式作简谐振荡外,其它性质都是静态的。

远区,$d\ll\lambda\ll r$,
\[
    \bm A(\bm r,t)=\frac{\mu_0}{4\pi}\frac{\e{\i kr}}r\int\bm J(\bm r')\e{-\i\bm k\cdot\bm r'}\d V'.
\]
可以对$\e{-\i\bm k\cdot\bm r'}$展开,略。

中间区,$d\ll r\sim\lambda$,近似失效,引入
球Bessel函数和Hankel函数
\begin{align}
    j_\ell(x)&:=\sqrt{\frac\pi{2x}}J_{\ell+1/2}(x),\\
    y_\ell(x)&:=\sqrt{\frac\pi{2x}}Y_{\ell+1/2}(x),\\
    h^\pm_\ell(x)&:=j_\ell(x)\pm\i y_\ell(x).
\end{align}
则
\[
    \bm A(\bm r)=\i\mu_0k\sum_{\ell,m}h_\ell^+(kr)Y_{\ell m}(\theta,\phi)\int\bm J(\bm r')j_\ell(kr')Y\cj_{\ell m}(\theta',\phi')\d V',
\]
推导和后面的略。

标量势与矢量势形式相同,我们考虑电单极的情况
\[
    \Phi(\bm r,t)=\frac1{4\pi\varepsilon_0r}q\Bigkh{t-\frac rc},
\]
式中$q(t)$是源的总电荷。因为电荷是守恒的,且根据定义,局域源是一个没有电荷流入或流出的源,所以总电荷$q$与时间无关。于是,一个定域源的势(和场)的电单极部分必然是静态的。%具有谐和时间依赖关系e-'(w≠0)的场没有单极子项。

%现在讨论w≠0时最低阶多极场。因为这些场可以由矢势通过(9.4)和(9.5)算出,所以我们在下面的论述中不再明显提及标势。
\section{电偶极辐射}
下面我们求电偶极子
\[
    \bm p:=\int\bm r'\rho(\bm r')\d V',
\]
激发的电场和磁场。
通过远区近似,
\[
    \bm A=-\i\frac{\mu_0\omega}{4\pi}\bm p\frac{\e{\i kr}}r.
\]
场
\begin{align*}
    \bm H&=\frac{ck^2}{4\pi}(\uvec n\times\bm p)\frac{\e{\i kr}}r\Bigkh{1+\i\frac1{kr}},\\
    \bm E&=\frac1{4\pi\varepsilon_0}\biggfkh{k^2(\uvec n\times\bm p)\times\uvec n\frac{\e{\i kr}}r+\bigkh{3\uvec n(\uvec n\cdot\bm p)-\bm p}\Bigkh{\frac1{r^3}-\i\frac{k}{r^2}}\e{\i kr}}.
\end{align*}
$r\to 0$时,
\begin{align*}
    \bm H&=\i\frac{\omega}{4\pi}\frac{\uvec n\times\bm p}{r^2},\\
    \bm E&=\frac1{4\pi\varepsilon_0}\frac{3\uvec n(\uvec n\cdot\bm p)-\bm p}{r^3};
\end{align*}
$r\to\infty$时,
\begin{align*}
    \bm H&=\frac{ck^2}{4\pi}(\uvec n\times\bm p)\frac{\e{\i kr}}r,\\
    \bm E&=Z_0\bm H\times\uvec n,
\end{align*}
% $Z_0$是自由空间阻抗。
由振荡偶极子辐射的时间平均功率
\[
    \dv P\Omega=\frac12\Re\bigfkh{r^2(\bm E\times\bm H\cj)\cdot\uvec n}=\frac{c^2Z_0}{32\pi^2}k^4p^2\sin^2\theta,
\]
故
\begin{equation}
    P=\frac{c^2Z_0^2k^4}{12\pi}p^2.
\end{equation}
为什么天空是蓝色的,特别是在垂直于太阳光线的天弧处?因为$\theta=90\degree$时太阳光线所激发的偶极辐射功率为0。
\begin{example}{天线}{center-fed, linear antenna}
    对于一个中间馈送(center-fed)的线性天线(antenna),其电流
    \[
        I_0\Bigkh{1-\frac{2\abs{z}}d}\e{-\i\omega t},
    \]
    则其电偶极矩为
    \[
        p=\i\frac{I_0d}{2\omega},
    \]
    辐射功率
    \[
        P=\frac{Z_0I_0^2}{48\pi}(kd)^2.
    \]
    由于天线向外辐射功率,因此它需要消耗能量,尽管天线是一个理想导体,但仍有辐射阻抗$R_\text{rad}$
    \[
        R_\text{rad}=\frac{Z_0}{24\pi}(kd)^2\simeq 5(kd)^2.
    \]
\end{example}
\section{磁偶极子和电四极子}
\paragraph{磁偶极子}
磁偶极矩对矢量势的贡献
\[
    \bm A=\frac{k\mu_0}{4\pi}(\uvec n\times\bm m)\frac{\e{\i kr}}r\Bigkh{\i-\frac1{kr}},
\]
得到磁场和电场
\begin{align*}
    \bm H&=\frac1{4\pi}\biggfkh{k^2(\uvec n\times\bm m)\times\uvec n\frac{\e{\i kr}}r+\bigkh{3\uvec n(\uvec n\cdot\bm m)-\bm m}\Bigkh{\frac1{r^3}-\i\frac{k}{r^2}}\e{\i kr}},\\
    \bm E&=-\frac{Z_0}{4\pi}k^2(\uvec n\times\bm m)\frac{\e{\i kr}}r\Bigkh{1+\i\frac1{ kr}}.
\end{align*}
与电偶极子形式高度相似。
\paragraph{电四极子}
电四极子对矢量势的贡献
\[
    \bm A=-\frac{\mu_0ck^2}{8\pi}\frac{\e{\i kr}}r\Bigkh{1+\i\frac1{kr}}\int\bm r'(\uvec n\cdot\bm r)\rho(\bm r')\d V'.
\]
磁场
\[
    \bm H=-\i\frac{ck^3}{24\pi}\frac{\e{\i kr}}r\uvec n\times(Q\cdot\uvec n).
\]
辐射功率 
\[
    \dv P\Omega=\frac{c^2Z_0}{1152\pi^2}k^6\abs{\bigfkh{\uvec n\times(Q\cdot\uvec n)}\times\uvec n}^2
\]
从而 
\[
    P=\frac{c^2Z_0k^6}{1440\pi}\sum_{\alpha,\beta}\abs{Q_{\alpha\beta}}^2\sim\omega^6.
\]



\chapter{常微分方程初值问题的数值解法}
\label{chap:ordinary differential equation}

\section{常微分方程初值问题}

\begin{definition}
    {一阶非线性常微分方程初值问题}{first-order nonlinear ODE}
    考虑函数$y:[x_0,b]\to\RR^d$满足
    \begin{subequations}
        \label{eqn:ode}
        \begin{align}
            \label{eqn:ode eq}
            &y'(x)=f(x,y(x)),\\
            &y(x_0)=y_0.
        \end{align}
    \end{subequations}
    特别地,若$f(x,y)=f(y)$与$x$无关,则称为自治问题(autonomous problem)。
\end{definition}

\begin{example}
    {Hamilton方程}{Hamiltonian equation}
    给定相空间上的Hamilton量$H:\RR^d\times\RR^d\to\RR$,广义坐标$q\in\RR^d$和广义动量$p\in\RR^d$随时间的演化满足Hamilton方程(或正则方程, canonial equations):
    \begin{subequations}
        \begin{align}
            \dot q_i&=\pv H{p_i},\\
            \dot p_i&=-\pv H{q_i}.
        \end{align}
    \end{subequations}
    比如调和谐振子$H=kx^2/2+p^2/2m$,令角频率$\omega=\sqrt{k/m}$,则Hamilton方程解得
    \[
        \lhkh{\begin{aligned}
            \dot x&=\frac pm,\\
            \dot p&=-kx.
        \end{aligned}}\implies
        \lhkh{\begin{aligned}
            x&=x_0\cos(\omega t)+\frac{p_0}m\frac{\sin(\omega t)}\omega,\\
            p&=p_0\cos(\omega t)-kx_0\frac{\sin(\omega t)}\omega.
        \end{aligned}}
    \]
\end{example}

\begin{definition}
    {半线性高阶常微分方程}{half-linear nth-order ODE}
    考虑函数$y:[x_0,b]\to\RR^d$满足
    \begin{subequations}
        \begin{align}
            &y^{(n)}=f(x,y,y',\dots,y^{(n-1)}),\\
            &y(x_0)=y_0,\enspace y'(x_0)=y_1,\enspace\ldots,\enspace y^{(n-1)}(x_0)=y_{n-1}.
        \end{align}
    \end{subequations}
    可转化为一阶常微分方程:
    \begin{equation}
        Y:=\begin{bmatrix}
            y\\y'\\\vdots\\y^{(n-1)}
        \end{bmatrix},\quad
        Y'=\begin{bmatrix}
            O&I\\ &\ddots&\ddots\\ &&\ddots&I\\ &&&O
        \end{bmatrix}Y+\begin{bmatrix}
            O\\\vdots\\O\\-f(x,Y)
        \end{bmatrix}
    \end{equation}
\end{definition}

\begin{theorem}
    {Lipschitz条件}{Lipschitz condition}
    若$f(x,y)$在$[a,b]\times\RR^d$上连续,且存在不依赖$x$的常数$L>0$使得$\forall y_1,y_2\in\RR^d$,
    \begin{equation}
        \norm{f(x,y_1)-f(x,y_2)}\leq L\norm{y_1-y_2},
    \end{equation}
    则$\forall y_0$,初值问题存在唯一解。
\end{theorem}

\begin{remark}
    我们总是假设$f(x,y)$满足Lipschitz条件,以保证初值问题存在唯一解。
\end{remark}

\begin{definition}
    {适定性}{well-posedness}
    若初值问题\eqref{eqn:ode}存在唯一解$y$,
    考虑其扰动
    \begin{subequations}
        \label{eqn:ode perturbed}
        \begin{align}
            &y'(x)=f(x,y(x))+\D f(x),\\
            &y(x_0)=y_0+\D y_0,
        \end{align}
    \end{subequations}
    若$\forall\epsilon>0$,$\exists\delta$使得当$\norm{\D f}<\delta,\;\norm{\D y_0}<\delta$时,扰动问题\eqref{eqn:ode perturbed}存在唯一解$\tilde y$且满足
    \begin{equation}
        \norm{\tilde y-y}<\epsilon,
    \end{equation}
    则称初值问题\eqref{eqn:ode}是适定的(well-posed)。
\end{definition}

\begin{remark}
    % 零稳定是指扰动量$\delta\to0$时的渐进稳定性。
    适定性说明初值问题\eqref{eqn:ode}的解存在唯一且连续依赖于$f$和初值$y_0$。
\end{remark}

\section{Euler方法}

\begin{theorem}
    {Euler方法}{Euler method}
    考虑区间$[x_0,b]$的一个划分:
    \[
        a=x_0<x_1<\cdots<x_N=b
    \]
    其中$h_n:=x_n-x_{n-1}$称为步长。对于初值问题\eqref{eqn:ode},若$h:=\max_n h_n$充分小,则可用差商近似代替导数:
    \[
        \frac{y(x_{n+1})-y(x_n)}{h_{n+1}}\approx y'(x_n)=f(x_n,y(x_n)).
    \]
    即
    \begin{equation}
        \label{eqn:Euler}
        y_{n+1}=y_n+h_{n+1}f(x_n,y_n).
    \end{equation}
    这称为Euler方法,是一种一阶显式单步法。
\end{theorem}

\begin{remark}
    必须关注的三个问题:
    \begin{enumerate}
        \item 收敛性:当$h\to0$时,$\sup_n\norm{y_n-y(x_n)}\to0$;
        \item 收敛速度;
        \item 稳定性:舍入误差的影响能否控制。
    \end{enumerate}
\end{remark}

\begin{theorem}
    {初值问题的等价积分形式}{}
    考虑划分区间上的初值问题\eqref{eqn:ode}的等价积分形式
    \[
        y(x)=y_n+\int_{x_n}^xf(t,y(t))\d t.
    \]
    即可通过数值积分近似计算$y(x_{n+1})$作为$y_{n+1}$。
\end{theorem}

\begin{corollary}
    用左矩形法则近似积分,即
    \[
        y(x_{n+1})=y(x_n)+h_{n+1}f(x_n,y(x_n))+R_{n+1},
    \]
    舍去积分余项$R_{n+1}$,即得Euler公式\eqref{eqn:Euler}。
\end{corollary}

\begin{example}
    {隐式Euler方法}{implicit Euler method}
    用右矩形法则近似积分,即
    \[
        y(x_{n+1})=y(x_n)+h_{n+1}f(x_{n+1},y(x_{n+1}))+R_{n+1},
    \]
    舍去积分余项$R_{n+1}$,即得隐式Euler公式
    \begin{equation}
        \label{eqn:implicit Euler}
        y_{n+1}=y_n+h_{n+1}f(x_{n+1},y_{n+1}).
    \end{equation}
\end{example}

\begin{example}
    {梯形公式}{}
    用梯形法则近似积分,即得梯形公式
    \begin{equation}
        \label{eqn:trapezoidal ode}
        y_{n+1}=y_n+\frac{h_{n+1}}2[f(x_n,y_n)+f(x_{n+1},y_{n+1})].
    \end{equation}
\end{example}

\begin{definition}
    {单步法的一般形式}{}
    结合\eqref{eqn:Euler},\eqref{eqn:implicit Euler}和\eqref{eqn:trapezoidal ode},初值问题\eqref{eqn:ode}的单步法一般形式为
    \[
        y_{n+1}=y_n+h_{n+1}\varphi(x_n,x_{n+1},y_n,y_{n+1}).
    \]
    % 增量函数$\varphi$与$f$有关。%且$\varphi(x_n,x_{n+1},y_n,y_{n+1})=f(x_n,y_n)$时即为Euler方法。
    若$\varphi$显含$y_{n+1}$,则称为隐式方法;否则称为显式方法。
    % $\varphi$应满足:
    % \[
    %     \norm{\varphi(x_n,x_{n+1},y_n,y_{n+1})-\varphi(x_n,x_{n+1},\tilde y_n,\tilde y_{n+1})}\leq L_f(\norm{y_n-\tilde y_n}+\norm{y_{n+1}-\tilde y_{n+1}}).
    % \]
    % 增量函数$\varphi$满足$\varphi(x_n,x_{n+1},y_n,y_{n+1};0)=0$且
    % 称一步误差为:
    % \[
    %     R_{n+1}=y(x_{n+1})-y_{n+1}
    % \]
\end{definition}

\begin{definition}
    {局部截断误差}{local truncation error}
    单步法在$x_{n+1}$处的局部截断误差定义为积分余项:
    \[
        R_{n+1}=y(x_{n+1})-y(x_n)-h_{n+1}\varphi(x_n,x_{n+1},y(x_n),y(x_{n+1})).
    \]
\end{definition}

\begin{definition}
    {相容性}{consistency}
    若$\lim_{h\to0}R=0$,则称单步法是相容的。
    \tcblower
    若存在不依赖$x$的常数$M>0$和整数$p\geq 1$,使得
    \begin{equation}
        \norm{y(x+h)-y(x)-h\varphi(x,x+h,y(x),y(x+h))}\leq Mh^{p+1},
    \end{equation}
    则称单步法至少$p$阶相容的。若局部截断误差可以展开成
    \[
        R=\psi(x,y)h^{p+1}+\bigo(h^{p+2}),
    \]
    称$\psi(x,y)h^{p+1}$为局部截断误差的主项。
\end{definition}

\begin{example}
    {Euler方法的局部截断误差}{}
    对于Euler方法\eqref{eqn:Euler},局部截断误差为
    \begin{align*}
        R_{n+1}&=y(x_{n+1})-y(x_n)-h_{n+1}f(x_n,y(x_n))\\
        &=y(x_n)+h_{n+1}y'(x_n)+\frac{h_{n+1}^2}2y''(x_n)+\bigo(h_{n+1}^3)-y(x_n)-h_{n+1}y'(x_n)\\
        &=\frac{h_{n+1}^2}2y''(x_n)+\bigo(h_{n+1}^3).
    \end{align*}
    因此Euler方法是一阶相容的,局部截断误差的主项为$h_{n+1}^2y''(x_n)/2$。
\end{example}

\begin{example}
    {梯形方法的局部截断误差}{}
    对于梯形方法\eqref{eqn:trapezoidal ode},局部截断误差为
    \begin{align*}
        R_{n+1}&=y(x_{n+1})-y(x_n)-\frac{h_{n+1}}2[f(x_n,y(x_n))+f(x_{n+1},y(x_{n+1}))]\\
        &=-\frac{h_{n+1}^3}{12}y'''(x_n)+\bigo(h_{n+1}^4).
    \end{align*}
    因此梯形方法是二阶相容的,局部截断误差的主项为$-h_{n+1}^3y'''(x_n)/12$。
\end{example}

\begin{theorem}
    {单步法相容}{}
    单步法相容$\iff f(x,y)=\varphi(x,x,y,y)$。
\end{theorem}


\chapter{群、环、域}

\section{二元运算}

\begin{definition}{二元运算}{binary operation}
	集合$S$上的一个二元运算(binary operation)是映射$\circ:S\times S\to S$。

	其中$S\times S\equiv S^2$是笛卡尔积(Cartesian product),
	\[
		A\times B:=\set{(a,b)}{a\in A, b\in B}.
	\]
\end{definition}
二元运算在$S$上是封闭的(property of closure)。
\begin{definition}{恒等元}{identity element}
	$e\in S$是恒等元(identity element),若$\forall a\in S,\enspace e\circ a=a\circ e=a.$
\end{definition}
\begin{definition}{可逆}{inversible}
	$a\in S$是可逆的(inversible),若$\exists a\iv\in S,\enspace a\circ a\iv=a\iv\circ a=e.$
\end{definition}
特别的,简记
\[
	a^m\equiv a\circ\cdots\circ a,\quad a^{-m}\equiv a\iv\circ\cdots\circ a\iv.
\]

\section{群与子群}

\begin{definition}{群}{group}
	群(group)是有二元运算$\circ:G\times G\to G$和集合$G$并满足下列性质的组合$(G,\circ)$:
	\begin{compactitem}
		\item 结合律:$(a\circ b)\circ c=a\circ(b\circ c)\equiv a\circ b\circ c;$
		\item 单位元:$\exists e\in G$使得$e\circ a=a\circ e=a;$
		\item 逆:$\forall a\in G,\enspace\exists a\iv$使得$a\iv\circ a=a\circ a\iv=e.$
	\end{compactitem}
	若还满足交换律,则称为交换群或Abel群。

	%若仅满足结合律,则称为半群;有单位元的半群又叫含幺半群。
\end{definition}
群的阶(order) $\ord(G)$ 表示其元素的个数。群可分为有限群和无限群。
\begin{theorem}{单位元和逆元的唯一性}{uniqueness of identity element and inverse}
	在群中只能有一个单位元,而群中的每个元素都正好有一个逆元素。
\end{theorem}
\begin{proof}
	若一个群存在两个单位元$e,e'$,则 
	\[
		e=e\circ e'=e';
	\]
	若一个元素$a$存在两个逆$b,c$,则 
	\[
		b=b\circ e=b\circ(a\circ c)=(b\circ a)\circ c=e\circ c=c.
		\qedhere
	\]
\end{proof}
\begin{example}{群的例子}{example of group}
	\begin{compactitem}
		\item 整数加群$(\ZZ,+)$:单位元0;
		\item 非零实数乘法群$(\RR\backslash\{0\},\times)$:单位元1;
		\item 一般线性(general linear)群$\GL(n)$:所有$n$阶可逆矩阵集合,单位元$I_n$。
	\end{compactitem}
\end{example}
\begin{theorem}{消去律}{cancellation law}
	$\forall a,b,c\in G$,有
	\begin{subequations}
		\begin{gather}
			a\circ b=a\circ c\implies b=c;\\
			b\circ a=c\circ a\implies b=c;\\
			b\circ a=a~\text{或}~a\circ b=a\implies b=e.
		\end{gather}
	\end{subequations}
\end{theorem}
\begin{proof}
	左乘/右乘$a\iv$.
\end{proof}
\begin{remark}
	逆$a\iv$的存在很关键,如果$G$上的运算只是结合的,则$(G,\circ)$是一个半群(semigroup),有单位元的半群又叫幺半群(monoid)。
\end{remark}
\begin{definition}{置换群和对称群}{permutation group and symmetric group}
	给定有限集合$T$,所有可逆映射$f:T\to T$构成一个群$\sym(T)$,运算是映射的复合,称做置换群(permutation group)
	\tcblower
	当$T=\{1,2,\ldots,n\}$时,对应的置换群称为对称群(symmetric group) $S_n$。
\end{definition}
\begin{example}{$S_2$}{symmetric group S2}
	$S_2=\{1,p\}$,其中 
		\begin{align*}
			1=\id:\{1,2\}\to\{1,2\}&,\\
			p:\{1,2\}\to\{1,2\}&,\quad p(1)=2,\enspace p(2)=1.
		\end{align*}
		$S_2$是交换群。可列出Cayley表
		\begin{align*}
			\begin{array}{c|cc}
				&1&p\\
				\hline
				1&1&p\\
				p&p&1
			\end{array}
		\end{align*}
\end{example}
\begin{example}{$S_3$}{symmetric group S3}
	$S_3$:定义生成元$x,y$满足:
	\begin{align*}
		x(1)=2,\quad x(2)=3,\quad x(3)=1;\\
		y(1)=2,\quad y(2)=1,\quad y(3)=3.
	\end{align*}
	可以证明生成元之间的关系:$x^3=1,\enspace y^2=1,\enspace x^2y=yx$,故$S_3$中所有元素都能写成生成元的积:
	\[
		S_3=\{1,x,x^2,y,xy,x^2y\},
	\]
	易知$S_3$不交换。根据生成元,$S_3$还可写为$S_3=\set{x,y}{x^3=1,y^2=1,x^2y=yx}$。
	
	生成元及其关系称作一个群的表现(presentation),一个群的表现不唯一。
\end{example}
\begin{definition}{子群}{subgroup}
	$H\subset G$是$G$的子群(subgroup),若$H$满足
	\begin{itemize}
		\item 封闭性:$\forall a,b\in H$,$a\circ b\in H$;
		\item 单位元:$e\in H$;
		\item 逆元:$\forall a\in H,\enspace\exists a\iv\in H$。
	\end{itemize}
	$\{e\}$和$G$都是平凡的子群,其他子群称为真子群(proper subgroup)。
\end{definition}

\begin{example}{子群的例子}{}
	\begin{itemize}
		\item 圆群:$(\set{z\in\CC}{\abs z=1},\times)\subset(\CC\backslash\{0\},\times)$;
		\item 特殊线性群$\SL(n)\subset\GL(n)$:所有行列式为1的$n$阶方阵;
	\end{itemize}
\end{example}

\begin{definition}{循环群}{cyclic group}
	循环群(cyclic group)是
	\begin{equation}
		Z_n\equiv\set{1,x,\ldots,x^{n-1}}{x^n=1},
	\end{equation}
	其生成元为$x$。
\end{definition}
\begin{example}{}{}
	$S_3$有两个子群是循环群:$\set{x^k}{x^3=1}=Z_3$和$\set{y^k}{y^2=1}=Z_2$。
\end{example}

\section{群同态}

\begin{definition}{群同态}{group homomorphism}
	$(G,\circ),(G',\circ')$是群,映射$\phi:G\to G'$是群同态(group homomorphism)若$\forall a,b\in G$
	\begin{equation}
		\phi(a\circ b)=\phi(a)\circ'\phi(b).
	\end{equation}
	也称映射$\phi$和群上的乘法相容(compatible)。
\end{definition}
\begin{theorem}{群同态下的单位元和逆元}{}
	若$\phi:G\to G'$是群同态,$G,G'$的单位元分别为$1,1'$,$a\in G$,则
	\begin{equation}
		\phi(1)=1',\quad \phi(a\iv)=\phi(a)\iv.
	\end{equation}
\end{theorem}
\begin{proof}
	(1)由$\phi(1)=\phi(1\circ 1)=\phi(1)\circ'\phi(1)$,再运用消去律可得$1'=\phi(1)$;

	(2)由$\phi(a\iv)\circ'\phi(a)=\phi(a\iv\circ a)=\phi(1)=1'$可得$\phi(a\iv)=\phi(a)\iv$。
\end{proof}
\begin{example}
	线性空间和$+$构成一个群,线性映射都是群同态。
\end{example}
\begin{definition}{群同态的像}{image of group homomorphism}
	群同态$\phi:G\to G'$的像(image) $\im\phi$定义为
	\begin{equation}
		\im\phi\equiv\set{x\in G'}{\exists a\in G,\enspace\phi(a)=x}.
	\end{equation}
\end{definition}
\begin{definition}{群同态的核}{kernel of group homomorphism}
	群同态$\phi:G\to G'$的核(kernel) $\ker\phi$定义为
	\begin{equation}
		\ker\phi\equiv\set{a\in G}{\phi(a)=1'}.
	\end{equation}
\end{definition}
\begin{theorem}{}{}
	若$\phi:G\to G'$是群同态,则$\im\phi$是$G'$的子群,$\ker\phi$是$G$的子群。
\end{theorem}
\begin{proof}
	考虑线性空间在$+$下构成的群,此时线性映射作为群同态的像与核同之前线性映射的像与核相同。
\end{proof}
\begin{definition}{左陪集}{left coset}
	$H$是$G$的子群,$a\in G$,则
	\begin{equation}
		a\circ H\equiv\set{g\in G}{\exists h\in H,g=a\circ h}
	\end{equation}
	是$H$在$G$下的一个左陪集(left coset)。同理可定义右陪集。
\end{definition}
\begin{theorem}{}{}
	群同态$\phi:G\to G'$,$a,b\in G$,则以下命题等价:
	\begin{enumerate}
		\item $\phi(a)=\phi(b)$;
		\item $a\iv\circ b\in\ker\phi$;
		\item $b\in a\circ\ker\phi$;
		\item $b\circ\ker\phi=a\circ\ker\phi$。
	\end{enumerate}
\end{theorem}
\begin{proof}
	$(1)\Rightarrow(2)$:
	\[
		\phi(a)=\phi(b)\implies 1'=\phi(a)\iv\circ'\phi(b)=\phi(a\iv\circ b)\implies a\iv\circ b\in\ker\phi;
	\]

	$(2)\Rightarrow(3)$:
	\[
		a\iv\circ b=h\in\ker\phi\implies b=a\circ h\in a\circ\ker\phi.
	\]

	$(3)\Rightarrow(4)$:由$b\in a\circ\ker\phi$,$\exists h\in\ker\phi$使得$b=a\circ h$;
	$\forall b'\in b\circ\ker\phi$,$\exists h'\in \ker\phi$使得$b'=b\circ h'=(a\circ h)\circ h'=a\circ(h\circ h')\in a\circ\ker\phi$,故$b\circ\ker\phi\subset a\circ\ker\phi$;同理由$a=b\circ h\iv$可以证明$a\circ\ker\phi\subset b\circ\ker\phi$,故$a\circ\ker\phi=b\circ\ker\phi.$

	$(4)\Rightarrow(1)$:$\forall h\in\ker\phi,\enspace\exists h'\in\ker\phi$使得$a\circ h=b\circ h'$
	\begin{align*}
		&\implies \phi(a\circ h)=\phi(b\circ h')\implies\phi(a)\circ'\phi(h)=\phi(b)\circ'\phi(h')\\
		&\implies\phi(a)\circ'1'=\phi(b)\circ'1'\implies\phi(a)=\phi(b).
	\end{align*}
	故以上4个命题等价。
\end{proof}
\begin{remark}~
	\begin{itemize}
		\item 群同态的核不仅告诉我们$G$中的哪些元素映射到1,也告诉我们哪些元素的像相同;
		\item 上面的命题在线性方程组中的应用就是$Ax=b$的通解$=$特解$+\,\{Ax=0\}$的通解。
	\end{itemize}
\end{remark}
\begin{corollary}
	群同态$\phi:G\to G'$是单射$\iff\ker\phi=\{1\}$。
\end{corollary}
\begin{definition}{正规子群}{normal subgroup}
	$N\subset G$是$G$的正规子群(normal subgroup)若$\forall a\in N,\forall g\in G$,共轭$g\circ a\circ g\iv\in N$。
\end{definition}
\begin{theorem}{}{}
	$\phi:G\to G'$是群同态,$\ker\phi$是$G$的正规子群。
\end{theorem}
\begin{definition}{中心}{center of group}
	群$G$的中心(center)是
	\begin{equation}
		Z_G\equiv\set{z\in G}{z\circ x=x\circ z,\forall x\in G},
	\end{equation}
	中心$Z_G$总是$G$的正规子群。
\end{definition}
\begin{example}{}{}
	\begin{itemize}
		\item 行列式是GL的群同态:
		\[
			\det:\GL(n)\to(\RR\backslash\{0\},\times)
		\]
		核$\ker\det=\SL(n)$是$\GL(n)$的正规子群。
		\item $Z_{\SL(2)}=\{I,-I\}$;
		\item $Z_{S_n}=\{1\},\enspace n\geq 3$。
	\end{itemize}
\end{example}

\section{群同构}

\begin{definition}{群同构}{isomorphism}
	若群同态$\phi:G\to G'$是双射,则称$\phi$为群同构(isomorphism),称$G,G'$是同构的(isomorphic),记作$G\simeq G'$。
	
	群到自己的同构$\phi:G\to G$也叫自同构。(automorphism)
\end{definition}
恒等映射$\id:G\to G$是自同构。
\begin{example}{群同构的例子}{}
	\begin{itemize}
		\item 指数函数
		\[
			\exp:(\RR,+)\to(\RR_{>0},\times):x\mapsto\e x.
		\]
		\item $P$是投影矩阵:$P^2=P$
		\[
			S_2\to\{I,I-2P\}.
		\]
	\end{itemize}
\end{example}

\section{等价关系}

\begin{definition}{等价关系}{}
	集合$S$上的等价关系$\sim$是$S$中两个元素$a,b$之间的关系,记作$a\sim b$,满足:
	\begin{itemize}
		\item 传递性:$a\sim b,b\sim c\implies a\sim c;$
		\item 对称性:$a\sim b\implies b\sim a;$
		\item 自反性:$\forall a,\enspace a\sim a.$
	\end{itemize}
\end{definition}
\begin{remark}~
	\begin{itemize}
		\item 等价关系可以看成$=$的抽象;
		\item 等价关系可以理解为映射$f:S\times S\to\{0,1\}$满足 
		\[
			a\sim b\iff f(a,b)=1.
		\]
		\item 未竟
	\end{itemize}
\end{remark}




\end{document}
