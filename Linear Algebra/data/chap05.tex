\chapter{行列式}

\section{行列式}

\paragraph{行列式的递归定义}

首先引入代数余子式。

\begin{definition}{代数余子式}{cofactor}
	给定$n$阶方阵$A$,去掉$A_{ij}$所在的第$i$行和第$j$列得到的$(n-1)$阶方阵记作$A_{\neq ij}$,其行列式$\det(A_{\neq ij})$定义为$A_{ij}$的余子式(minor);而
	\begin{align}
		\cof(A)_{ij}:=(-)^{i+j}\det(A_{\neq ij}).
	\end{align}
	定义为$A_{ij}$的代数余子式(cofactor)。
	其表达式的正负号与位置的关系:
	\begin{align}
		\begin{bmatrix}
			+&-&+&\cdots\\
			-&+&-&\cdots\\
			+&-&+&\cdots\\
			\vdots&\vdots&\vdots&\ddots\\
		\end{bmatrix}
	\end{align}
\end{definition}
\begin{definition}{行列式·一}{determinant}
	$n$阶方阵$A$的行列式(determinant)记作$\det(A)$,其值递归地定义为:
	\begin{compactenum}
		\item $n=1$时,$\det(A)=A_{11};$
		\item $n>1$时,任取第$i$行展开或第$j$列展开(Laplace展开)
		\begin{align}
			\det(A) = \sum_{i=1}^nA_{ij}\cof(A)_{ij} = \sum_{j=1}^nA_{ij}\cof(A)_{ij}.
		\end{align}
		由于$\cof(A)$是$n-1$阶的行列式,故递归会随降阶停止。
	\end{compactenum}
\end{definition}

\begin{corollary}
	三角矩阵的行列式等于对角元的乘积。
\end{corollary}

\begin{example}
	{二阶行列式和三阶行列式}{2x2 and 3x3 determinant}
	二阶行列式:
	\begin{equation}
			\begin{vmatrix}
				a_{11}&a_{12}\\a_{21}&a_{22}
			\end{vmatrix}=a_{11}a_{22}-a_{12}a_{21}.
	\end{equation}
	三阶行列式:
	\begin{equation}
		\begin{aligned}
			\begin{vmatrix}
				a_{11}&a_{12}&a_{13}\\
				a_{21}&a_{22}&a_{23}\\
				a_{31}&a_{32}&a_{33}
			\end{vmatrix}={}&a_{11}
			\begin{vmatrix}
				a_{22}&a_{23}\\
				a_{32}&a_{33}
			\end{vmatrix}-a_{12}
			\begin{vmatrix}
				a_{21}&a_{23}\\
				a_{31}&a_{33}
			\end{vmatrix}+a_{13}
			\begin{vmatrix}
				a_{21}&a_{22}\\
				a_{31}&a_{32}
			\end{vmatrix}\\
			={}&a_{11}a_{22}a_{33}+a_{12}a_{23}a_{31}+a_{13}a_{21}a_{32}\\
			&\qquad-(a_{13}a_{22}a_{31}+a_{11}a_{23}a_{32}+a_{12}a_{21}a_{33}).
		\end{aligned}
	\end{equation}
\end{example}

\section{行列式的性质}

\paragraph{行列式作为线性映射的定义}

可以由行列式的性质,给出行列式的另一定义。

\begin{definition}
	{行列式·二}{determinant II}
	行列式还可定义为$n$个$n$维向量到数域$\FF$的映射:
	\[
		\det:\underbrace{\FF^n\times\cdots\times\FF^n}_n\to\FF
	\]
	且满足:
	\begin{itemize}
		\item 多线性:
		\begin{equation}
			\det(\ldots,ka_i+k'a_i',\ldots)=k\det(\ldots,a_i,\ldots)+k'\det(\ldots,a_i',\ldots);
		\end{equation}
		\item 反对称:
		\begin{equation}
			\det(\ldots,a_i,\ldots,a_j,\ldots)=-\det(\ldots,a_j,\ldots,a_i,\ldots);
		\end{equation}
		\item 单位化:$\det(I)=1$。
	\end{itemize}
	可以证明,这个满足此定义的映射存在且唯一,并且与\dfnref{dfn:determinant}~是等价的。
\end{definition}
\begin{lemma}
	初等矩阵的行列式……略
\end{lemma}
\begin{proof}
	思路在于通过将矩阵$A$表示为一系列初等矩阵$E_1,\ldots,E_n$和$\rref(A)$的乘积得到,过程略。
\end{proof}
\begin{corollary}
	~%在以上证明的推导过程中可以得到行列式的如下几个推论:
	\begin{itemize}
		\item 若$A$中有任意两行或两列相同,则$\det(A)=0$;
		\item 若$A$的行/列之间线性相关,或者说秩小于阶,则$\det(A)=0$。
	\end{itemize}
\end{corollary}


\begin{theorem}
	{可逆性与行列式的关系}{invertible <=> determinant}
	$A$可逆$\iff\det(A)\neq 0$。
\end{theorem}

\begin{remark}
	因此任意给定一个方阵,其不可逆的可能是很小的。因为不可逆($\det(A)=0$)是一个额外的约束。
\end{remark}

\begin{theorem}{行列式与矩阵乘法}{determinant of multiplication}
	给定$A,B\in\FF^{n\times n}$,其乘积的行列式等于行列式的乘积:
	\begin{equation}
		\det(AB)=\det(A)\det(B).
	\end{equation}
\end{theorem}
\begin{proof}
	若$A$不可逆,则$AB$不可逆,等式成立:
	\[
		\det(AB)=\det(A)\det(B)=0;
	\]
	若$A$可逆,则可表示为一系列初等矩阵的乘积$A=E_k\cdots E_1$,从而
	\begin{align*}
		\det(AB)&=\det(E_k\cdots E_1B)=\det(E_k)\det(E_{k-1}\cdots E_1B)\\
		&=\det(E_k)\cdots\det(E_1)\det(B)=\det(E_k\cdots E_1)\det(B)=\det(A)\det(B).
		\qedhere
	\end{align*}
\end{proof}
\begin{corollary}
	逆的行列式为原行列式的倒数:
	\begin{align}
		\det(A\iv)=\det(A)\iv.
	\end{align}
\end{corollary}

\begin{remark}
	% 行列式与矩阵加法之间并无必然联系,即
	一般地,$\det(A+B)\neq\det(A)+\det(B)$。
	但是当$A,B$满足一些特定条件时,$\det(A+B)$可以化简,见\thmref{thm:matrix determinant lemma}。
\end{remark}

\paragraph{行列式的完全展开定义}

首先引入排列的逆序数。

\begin{definition}
	{排列}{permutation}
	给定集合$S=\{1,\ldots,n\}$和$n$个数$i_1,\ldots,i_n\in S$,若映射$\sigma$:
	\[
		\sigma:S\to S,\enspace \sigma(k)\mapsto i_k,
	\]
	是一个一一映射,则称$\sigma$是$S$的一个排列(permutation)。
\end{definition}
\begin{definition}
	{逆序}{inversion}
	给定排列$\sigma$,若$i<j$且$\sigma(i)>\sigma(j)$,则称$(i,j)$是$\sigma$的一个逆序(inversion)。
\end{definition}

\begin{remark}
	有的地方将逆序定义为$(\sigma(i),\sigma(j))$,二者是等价的。
\end{remark}

\begin{definition}
	{逆序数与奇偶性}{inversion number and parity}
	$\sigma$中所有逆序组成的集合为逆序集,逆序集中元素的个数称为逆序数$\mathrm{inv}(\sigma)$。排序的奇偶性(parity)便定义为逆序数的奇偶性:
	\begin{equation}
		\sgn(\sigma)=(-)^{\mathrm{inv}(\sigma)}.
	\end{equation}
\end{definition}

\begin{definition}{全反对称张量}{Levi-Civita symbol}
	给定$n$个数$i_1,\ldots,i_n\in S=\{1,\ldots,n\}$,定义全反对称张量(Levi-Civita symbol)
	\begin{equation}
		\epsilon_{i_1\cdots i_n}=\begin{cases}
			\sgn(\sigma),&\text{$i_1,\ldots,i_n$是$S$的一个排列$\sigma$}\\
			0,&\otherwise
		\end{cases}
	\end{equation}
\end{definition}

\begin{definition}
	{行列式·三}{determinant III}
	利用全反对称张量,行列式也可以定义为
	\begin{align}
		\det(A)=\sum_{i_1,\ldots,i_n\in S}\epsilon_{i_1\cdots i_n}a_{1i_1}\cdots a_{ni_n}.
	\end{align}
\end{definition}

\begin{proof}
	可以验证展开式满足\dfnref{dfn:determinant II} 的三条性质。
\end{proof}

\begin{theorem}
	{行列式与矩阵转置}{}
	行列式在矩阵转置下不变
	\[
		\det(A\tp)=\det(A),
	\]
\end{theorem}
\begin{proof}
	由行列式的\dfnref{dfn:determinant III} 立得。
\end{proof}

\subsectionstar{行列式的运算}

下面研究分块矩阵的行列式,显然分块对角矩阵$\det(\diag(A,D))=\det(A)\det(D)$,下面研究更一般的情况。

\begin{theorem}{分块矩阵的行列式·一}{determinant of bolck matrix I}
	若$A$是$m$阶方阵,$D$是$n$阶方阵,$B$是$m\times n$矩阵,则
	\begin{align}
		\det\!\kh{\begin{bmatrix}
			A&B\\ &D
		\end{bmatrix}}=\det(A)\det(D).
	\end{align}
\end{theorem}
\begin{proof}
	对$A,D$进行$LU$分解,$A=L_AU_A,D=L_DU_D$,则
	\[
		\begin{bmatrix}
			A&B\\ &D
		\end{bmatrix}=
		\begin{bmatrix}
			L_A\\ &L_D
		\end{bmatrix}
		\begin{bmatrix}
			U_A&L_A\iv B\\ &U_D
		\end{bmatrix}
	\]
	前者为下三角矩阵,后者为上三角矩阵,故
	\[
		\det\!\kh{\begin{bmatrix}
			A&B\\ &D
		\end{bmatrix}}=\det(L_A)\det(L_D)\det(U_A)\det(U_D)=\det(A)\det(D).
		\qedhere
	\]
\end{proof}
\begin{theorem}{分块矩阵的行列式·二}{determinant of bolck matrix II}
	若$A$是$m$阶方阵,$D$是$n$阶方阵,且$A,D$至少一个可逆,%$B$是$m\times n$矩阵,$C$是$n\times m$矩阵,
	则
	\begin{equation}
		\det\!\kh{\begin{bmatrix}
			A&B\\C&D
		\end{bmatrix}}
		=\begin{cases}
			\det(A)\det(D-CA\iv B),&\text{$A$可逆}\\
			\det(D)\det(A-BD\iv C),&\text{$D$可逆}
		\end{cases}.
	\end{equation}
\end{theorem}

\begin{proof}
	对\thmref{thm:inverse of block matrix} 的证明中的矩阵取行列式即证:
	\begin{align*}
		\begin{bmatrix}
			A&B\\C&D
		\end{bmatrix}&=\begin{bmatrix}
			I\\CA\iv&I
		\end{bmatrix}\begin{bmatrix}
			A\\ &D-CA\iv B
		\end{bmatrix}\begin{bmatrix}
			I&A\iv B\\ &I
		\end{bmatrix}\\
		&=\begin{bmatrix}
			I&BD\iv\\ &I
		\end{bmatrix}\begin{bmatrix}
			A-BD\iv C\\ &D
		\end{bmatrix}\begin{bmatrix}
			I\\D\iv C&I
		\end{bmatrix}.
		\qedhere
	\end{align*}
\end{proof}

\begin{theorem}
	{Weinstein-Aronszajn恒等式}{Weinstein-Aronszajn identity}
	给定$A\in\FF^{m\times n},B\in\FF^{n\times m}$,则 
	\begin{equation}
		\label{eqn:det(I+AB)=det(I+BA)}
		\det(I_m+AB)=\det(I_n+BA).
	\end{equation}
\end{theorem}
\begin{proof}
	考察分块矩阵
	\[
		M=\begin{bmatrix}
			I_m&A\\-B&I_n
		\end{bmatrix},
	\]
	显然$I_m,I_n$都是可逆的,由\thmref{thm:determinant of bolck matrix II},即得。
\end{proof}

\begin{example}
	{Weinstein-Aronszajn恒等式的应用}{}
	\thmref{thm:Weinstein-Aronszajn identity} 适用于求解$m\gg n\sim 1$的情形,例如
	\[
		\begin{vmatrix}
			1+a_1+b_1&a_1+b_2&\cdots&a_1+b_n\\
			a_2+b_1&1+a_2+b_2&\cdots&a_2+b_n\\
			\vdots&\vdots&\ddots&\vdots\\
			a_n+b_1&a_n+b_2&\cdots&1+a_n+b_n
		\end{vmatrix}
	\]
	定义 
	\[
		A=\begin{bmatrix}
			a_1&a_2&\cdots&a_n\\
			1&1&\cdots&1
		\end{bmatrix}\tp,\quad
		B=\begin{bmatrix}
			1&1&\cdots&1\\
			b_1&b_2&\cdots&b_n
		\end{bmatrix}
	\]
	则原式$=\det(I_n+AB)$,而
	\[
		\det(I_n+AB)=\det(I_2+BA)=\biggkh{1+\sum_{i=1}^na_i}\biggkh{1+\sum_{i=1}^nb_i}-n\sum_{i=1}^na_ib_i.
	\]
\end{example}

\begin{theorem}
	{矩阵行列式引理}{matrix determinant lemma}
	%https://en.wikipedia.org/wiki/Matrix_determinant_lemma
	若$A$可逆且$u,v$均为$n$维列向量,则
	\begin{align}
		\det(A+uv\tp)=(1+v\tp A\iv u)\det(A).
	\end{align}
\end{theorem}
\begin{proof}
	先证明命题对于$A=I$成立,事实上
	\begin{align*}
		\begin{bmatrix}
			I\\v\tp&1
		\end{bmatrix}
		\begin{bmatrix}
			I+uv\tp&u\\ &1
		\end{bmatrix}
		\begin{bmatrix}
			I\\-v\tp&1
		\end{bmatrix}=
		\begin{bmatrix}
			I&u\\ &1+v\tp u
		\end{bmatrix}.
	\end{align*}
	故
	\begin{align}
		\det(I+uv\tp)=1+v\tp u
	\end{align}
	进而 
	\[
		\det(A+uv\tp)=\det(A)\det(I+A\iv uv\tp)=(1+v\tp A\iv u)\det(A).
		\qedhere
	\]
\end{proof}
\begin{remark}
	行列式的运算技巧不宜写得过多,因为这远非线性代数的精髓。
\end{remark}
\section{Cramer法则、伴随矩阵}
\begin{theorem}{Cramer法则}{Cramer's rule}
	考虑线性方程组$Ax=b$,记$A$的第$j$列为$a_j=Ae_j$,将$A$的第$j$列$a_j$替换为$b$得到矩阵$B_j$,即
	\[
		B_j:=[a_1\enspace\cdots\enspace a_{j-1}\enspace b\enspace a_{j+1}\enspace\cdots\enspace a_n]
	\]
	Cramer法则给出:$x$的各分量为
	\begin{align}
		x_j=\frac{\det(B_j)}{\det(A)}.
	\end{align}
\end{theorem}

\begin{proof}
	注意到$a_j=Ae_j$,将$I$的第$j$列$e_j$替换为$x$得到矩阵$C_j$,即
	\[
		C_j:=[e_1\enspace\cdots\enspace e_{j-1}\enspace x\enspace e_{j+1}\enspace\cdots\enspace e_n],
	\]
	则$B_j=AC_j$,等号左右取行列式,由$\det(C_j)=x_j$即证。
\end{proof}

\begin{corollary}
	% Cramer法则的应用之一是矩阵求逆,
	利用$AA\iv=I$,把$A$左乘$A\iv$的每一列看做一个线性方程组,解得
	\[
		(A\iv)_{ij}=\frac{\cof(A)_{ji}}{\det(A)}.
	\]
\end{corollary}
\begin{definition}{伴随矩阵}{adjugate matrix}
	方阵$A$的伴随矩阵(adjugate matrix)定义为其代数余子式矩阵的转置:
	\begin{align}
		\adj(A):=\cof(A)\tp.
	\end{align}
\end{definition}

\begin{corollary}
	伴随矩阵的性质:
	\begin{itemize}
		\item $\adj(I)=I$;$\adj(O)=O$,除了$\adj([0])=[1]$;
		\item $\adj(cA)=c^{n-1}\adj(A)$;
		\item $\adj(A\tp)=\adj(A)\tp$;
		\item $\adj(AB)=\adj(B)\adj(A)$,故$\adj(A^k)=\adj(A)^k$。
	\end{itemize}
\end{corollary}

\begin{theorem}
	{伴随矩阵与逆的关系}{}
	根据Cramer法则,
	\begin{align}
		A\iv=\frac{\adj(A)}{\det(A)}.
	\end{align}
\end{theorem}
\begin{remark}
	即使$A$不可逆,$\adj(A)$依然存在。
\end{remark}
\begin{theorem}{伴随矩阵与原矩阵的乘积}{}
	伴随矩阵与原矩阵可交换,其乘积为
	\begin{align}
		A\adj(A)=\adj(A)A=\det(A)I.
	\end{align}
\end{theorem}
\begin{proof}
	根据Laplace展开,
	\begin{align*}
		\bigfkh{A\adj(A)}_{ij}&=\sum_{k}A_{ik}\adj(A)_{kj}=\sum_{k}(-)^{j+k}A_{ik}\det(A_{\neq jk})=\vd_{ij}\det(A);\\
		\bigfkh{\adj(A)A}_{ij}&=\sum_{k}\adj(A)_{ik}A_{kj}=\sum_{k}(-)^{i+k}\det(A_{\neq ki})A_{kj}=\vd_{ij}\det(A).%
	\end{align*}
	对应元素相等,即证。
\end{proof}
\begin{corollary}
	~
	\begin{itemize}
		\item %由
		% \[
		% 	\det(A)\det(\adj(A))=\det(\det(A)I)=\det(A)^n.
		% \]
		%故
		伴随矩阵的行列式(并不要求$A$可逆)
		\begin{align}
			\det(\adj(A))=\det(A)^{n-1};
		\end{align}
		\item 伴随矩阵的伴随矩阵:
		\begin{align}
			\adj(\adj(A))=\det(A)^{n-2}A.
		\end{align}
		由伴随矩阵$\adj(A)$反求矩阵$A$:
		\[
			A=\det(\adj(A))^{-(n-2)/(n-1)}\adj(\adj(A)).
		\]
		\item $A$可逆$\iff\adj(A)$可逆
		\begin{align}
			\adj(A)\iv=\adj(A\iv)=\frac{A}{\det(A)};
		\end{align}
	\end{itemize}
\end{corollary}

\begin{theorem}
	{伴随矩阵的秩}{}
	矩阵$A$的伴随矩阵$\adj(A)$的秩为
	\begin{equation}
		\rank(\adj(A))=\begin{cases}
			n,&\rank(A)=n\\
			1,&\rank(A)=n-1\\
			0,&\rank(A)\leq n-2\\
		\end{cases}
	\end{equation}
\end{theorem}

\begin{proof}
	~
	\begin{itemize}
		\item $\rank(A)=n$时,$A$可逆,$\adj(A)$也可逆;
		
		\item $\rank(A)=n-1$时,$A\adj(A)=O$,故$\C(\adj(A))\subset\N(A)$,即
		\[
			\rank(\adj(A))\leq n-\rank(A)=1,
		\]
		又$A$存在非0的余子式,故$\rank(\adj(A))\geq 1$,即$\rank(\adj(A))=1$;
	
		\item $\rank(A)\leq n-2$时,$A$的余子式均为0,故$\adj(A)=O$。
		\qedhere
	\end{itemize}
\end{proof}

\begin{theorem}
	{Jacobi公式}{Jacobi's formula}
	给定矩阵函数$A:\FF\to\FF^{n\times n},\enspace t\mapsto A(t)$,则 
	\begin{equation}
		\label{eqn:Jacobi det'}
		\dd t\det(A)=\tr\biggkh{\adj(A)\dv At}
	\end{equation}
\end{theorem}

\begin{corollary}
	\begin{equation}
		\dd t\det(A)=\det(A)\tr\biggkh{A\iv\dv At}
	\end{equation}
\end{corollary}

\begin{remark}
	计算机运用Cramer法则解$n$元线性方程组的时间复杂度是$\bigo(n\cdot n!)$,且在数值上不稳定\footnote{Cramer, Gabriel (1750). "Introduction à l'Analyse des lignes Courbes algébriques" (in French). Geneva: Europeana. pp. 656-659. Retrieved 2012-05-18.},这是不可接受的。与其在计算方面的作用相比,其理论价值更为重大,即:研究了方程组的系数与方程组解的存在性与唯一性关系。

	因此在解题中建议用增广矩阵和Gauss-Jordan消元法求解线性方程组$Ax=b$:
	\[
		[A\enspace b]\enspace\to\enspace[I\enspace A\iv b];
	\]
	同时,不推荐用伴随矩阵$\adj(A)$求逆,因为其与Cramer法则等价。仍建议用增广矩阵和Gauss-Jordan消元法求逆:
	\[
		[A\enspace I]\enspace\to\enspace[I\enspace A\iv].
	\]
\end{remark}
