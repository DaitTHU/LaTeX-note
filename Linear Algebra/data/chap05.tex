\chapter{行列式}
在\exmref{exm:inverse of 2x2 matrix}~中,2阶方阵的逆为
\begin{align*}
	\begin{bmatrix}
		a&b\\c&d
	\end{bmatrix}\iv=\frac1{ad-bc}
	\begin{bmatrix}
		d&-b\\
		-c&a
	\end{bmatrix}.
\end{align*}
定义2阶方阵的行列式(determinant)
\begin{align}
	\det (A)\equiv\abs A\equiv\begin{vmatrix}
		a&b\\c&d
	\end{vmatrix}:=ad-bc.
\end{align}
特别的,给出3阶方阵的行列式
\begin{align}
	\begin{vmatrix}
		a_{11}&a_{12}&a_{13}\\
		a_{21}&a_{22}&a_{23}\\
		a_{31}&a_{32}&a_{33}
	\end{vmatrix}={}&a_{11}
	\begin{vmatrix}
		a_{22}&a_{23}\\
		a_{32}&a_{33}
	\end{vmatrix}-a_{12}
	\begin{vmatrix}
		a_{21}&a_{23}\\
		a_{31}&a_{33}
	\end{vmatrix}+a_{13}
	\begin{vmatrix}
		a_{21}&a_{22}\\
		a_{31}&a_{32}
	\end{vmatrix}\\\notag
	={}&a_{11}a_{22}a_{33}+a_{12}a_{23}a_{31}+a_{13}a_{21}a_{32}\\\notag
	&-(a_{13}a_{22}a_{31}+a_{11}a_{23}a_{32}+a_{12}a_{21}a_{33}).
\end{align}
\section{行列式}
下面我们用递归的方式给出行列式的定义。首先引入代数余子式。
\begin{definition}{代数余子式}{cofactor}
	对于$n$阶方阵$A$,记$A_{\neq ij}$为去掉第$i$行第$j$列得到的$(n-1)$阶方阵。则$A$的余子式(minor)定义为
	\begin{align}
		M_{ij}:=\det (A_{\neq ij});
	\end{align}
	代数余子式(cofactor)定义为
	\begin{align}
		C_{ij}:=(-)^{i+j}M_{ij}.
	\end{align}
\end{definition}
代数余子式前的正负号与矩阵中元素所在位置的关系:
\begin{align}
	\begin{bmatrix}
		+&-&+&\cdots\\
		-&+&-&\cdots\\
		+&-&+&\cdots\\
		\vdots&\vdots&\vdots&\ddots\\
	\end{bmatrix}
\end{align}
\begin{definition}{行列式}{determinant}
	
	$A$的行列式定义为:
	\begin{compactenum}
		\item $n=1$时,$\det (A)=a_{11};$
		\item $n>1$时,行展开或列展开(Laplace展开)
		\begin{align}
			\det (A) = \sum_{i=1}^na_{ij}C_{ij} = \sum_{\ell=1}^na_{k\ell}C_{k\ell}.
		\end{align}
	\end{compactenum}
\end{definition}
易得,单位矩阵行列式$\det I=1$,对角矩阵行列式
\begin{align}
	\det(\diag(a_1,\ldots,a_n))=a_1\cdots a_n.
\end{align}
\begin{theorem}{三角矩阵的行列式}{determinant of triangular matrix}
	三角矩阵的行列式等于对角元的乘积。
\end{theorem}
按行列式的定义进行计算即证。
\section{行列式的性质}
行列式可看做$n$个$n$维向量到数域$\FF$的映射:
\[
	\det (A)=T(a_1,\ldots,a_n),
\]
\begin{theorem}{行列式的性质}{}
	由定义,行列式是线性的:
	\begin{compactenum}
		\item $T(a_1,\ldots,ka_i,\ldots,a_n)=kT(a_1,\ldots,a_i,\ldots,a_n);$
		\item $T(\ldots,a_i+b_i,\ldots)=T(\ldots,a_i,\ldots)+T(\ldots,b_i,\ldots).$
	\end{compactenum}
	除了线性,行列式还满足初等变换相关的性质:
	\begin{compactenum}
		\item[3.] 交换$A$任意两行或两列得到$B$,则$\det(B)=-\det(A);$
		\begin{corollary}
			若$A$中有任意两行或两列相同,则$\det(A)=0.$
		\end{corollary}
		\item[4.] 将$A$的第$i$行乘一个常数加到第$j$行得到$B$,则$\det(A)=\det(B).$
		\begin{corollary}
			若$A$的行/列之间线性相关,或者说秩小于阶,则$\det (A)=0.$
		\end{corollary}
	\end{compactenum}
\end{theorem}
因此$A$可逆$\iff\det(A)\neq 0.$
\begin{definition}{全反对称张量}{Levi-Civita symbol}
	定义全反对称张量(Levi-Civita symbol)~$\epsilon_{i_1\cdots i_n}$,其中$i_1,\ldots,i_n$取值范围$1,\ldots,n$,满足
	\begin{compactitem}
		\item $\epsilon_{1\cdots n}=1;$
		\item $\epsilon_{\cdots i_p\cdots i_q\cdots}=-\epsilon_{\cdots i_q\cdots i_p\cdots};$
		\item 任两个指标相同则$\epsilon_{\cdots i_p\cdots i_p\cdots}=0.$
	\end{compactitem}
	如果$i_1,\ldots,i_n$是$1,\ldots,n$的一个全排列,其从$1,\ldots,n$变换而来需要两两交换的次数为$p$,则 
	\[
		\epsilon_{i_1\cdots i_n}=(-1)^p.
	\]
\end{definition}
利用全反对称张量,行列式也可以定义为
\begin{align}
	\det(A)=\sum_{i_1,\ldots,i_n=1}^n\epsilon_{i_1\cdots i_n}a_{1i_i}\cdots a_{ni_n}.
\end{align}
或者进行一个抽象的定义:行列式$\det(A)=T(a_1,\ldots,a_n)$是$n$个$n$维向量到数域$\FF$的映射,且满足以下三个性质:
\begin{compactenum}
	\item $\det I=1;$
	\item 任意交换两列,行列式反号;
	\item 线性$\ldots$
\end{compactenum}
\section{行列式的运算}
\paragraph{行列式与矩阵运算}
行列式在矩阵转置下不变
\[
	\det(A\tp)=\det(A),
\]
\begin{theorem}{行列式与矩阵乘法}{determinant of multiplication}
	\begin{equation}
		\det(AB)=\det(A)\det(B).
	\end{equation}
\end{theorem}
\begin{proof}
	若$A$不可逆,则$AB$不可逆,等式成立:
	\[
		\det(AB)=\det(A)\det(B)=0.
	\]
	若$A$可逆,则可表示为一系列初等矩阵的乘积$A=E_k\cdots E_1$,从而
	\begin{align*}
		&\det(AB)=\det(E_k\cdots E_1B)=\det(E_k)\det(E_{k-1}\cdots E_1B)\\
		&=\det(E_k)\cdots\det(E_1)\det(B)=\det(E_k\cdots E_1)\det(B)=\det(A)\det(B).
		\qedhere
	\end{align*}
\end{proof}
\begin{corollary}
	\begin{align}
		\det(A\iv)=\det(A)\iv.
	\end{align}
\end{corollary}

但是行列式与矩阵加法之间并无必然联系:
\begin{align}
	\det(A+B)\neq\det(A)+\det(B).
\end{align}
但是当$A,B$满足特殊的条件时,$\det(A+B)$可以化简,见后文的\thmref{thm:matrix determinant lemma}。
\begin{theorem}{分块矩阵的行列式·一}{determinant of bolck matrix I}
	若$A$是$m$阶方阵,$D$是$n$阶方阵,$B$是$m\times n$矩阵,则
	\begin{align}
		\begin{vmatrix}
			A&B\\0&D
		\end{vmatrix}=\abs A\abs D.
	\end{align}
\end{theorem}
\begin{proof}
	对$A,D$进行$LU$分解,$A=L_AU_A,D=L_DU_D$,则
	\[
		\begin{bmatrix}
			A&B\\0&D
		\end{bmatrix}=
		\begin{bmatrix}
			L_A\\ &L_D
		\end{bmatrix}
		\begin{bmatrix}
			U_A&L_A\iv B\\ &U_D
		\end{bmatrix}
	\]
	前者为下三角矩阵,后者为上三角矩阵,故
	\[
		\begin{vmatrix}
			A&B\\0&D
		\end{vmatrix}=\abs{L_A}\abs{L_D}\abs{U_A}\abs{U_D}=\abs A\abs D.
		\qedhere
	\]
\end{proof}
\begin{theorem}{分块矩阵的行列式·二}{determinant of bolck matrix II}
	若$A$是$m$阶方阵,$D$是$n$阶方阵,且$A,D$至少一个可逆,$B$是$m\times n$矩阵,$C$是$n\times m$矩阵,则
	\begin{align}
		\begin{vmatrix}
			A&B\\C&D
		\end{vmatrix}=
		\abs A\abs{D-CA\iv B}=
		\abs D\abs{A-BD\iv C}
	\end{align}
\end{theorem}
\begin{proof}
	注意到
	\begin{align*}
		\begin{bmatrix}
			I&0\\-CA\iv&I
		\end{bmatrix}
		\begin{bmatrix}
			A&B\\C&D
		\end{bmatrix}&=
		\begin{bmatrix}
			A&B\\0&-CA\iv B+D
		\end{bmatrix};\\
		\begin{bmatrix}
			I&-BD\iv\\0&I
		\end{bmatrix}
		\begin{bmatrix}
			A&B\\C&D
		\end{bmatrix}&=
		\begin{bmatrix}
			A-BD\iv D&0\\C&D
		\end{bmatrix}.
	\end{align*}
	对上式取行列式即证。
\end{proof}
\begin{theorem}{\href{https://en.wikipedia.org/wiki/Matrix_determinant_lemma}{矩阵行列式引理}}{matrix determinant lemma}
	若$A$可逆且$u,v$均为$n$维列向量,则
	\begin{align}
		\det(A+uv\tp)=(1+v\tp A\iv u)\det(A).
	\end{align}
\end{theorem}
\begin{proof}
	先证明命题对于$A=I$成立,事实上
	\begin{align*}
		\begin{bmatrix}
			I\\v\tp&1
		\end{bmatrix}
		\begin{bmatrix}
			I+uv\tp&u\\ &1
		\end{bmatrix}
		\begin{bmatrix}
			I\\-v\tp&1
		\end{bmatrix}=
		\begin{bmatrix}
			I&u\\ &1+v\tp u
		\end{bmatrix}.
	\end{align*}
	故
	\begin{align}
		\det(I+uv\tp)=1+v\tp u
	\end{align}
	进而 
	\[
		\det(A+uv\tp)=\det(A)\det(I+A\iv uv\tp)=(1+v\tp A\iv u)\det(A).
		\qedhere
	\]
\end{proof}
行列式的运算技巧不宜写得过多,因为这远非线性代数的精髓。
\section{Cramer法则、伴随矩阵}
\begin{theorem}{Cramer法则}{Cramer's rule}
	考虑线性方程组
	\[
		Ax=b,
	\]
	$Ae_i$是$A$的第$i$列,定义矩阵$B_i$为把$A$中第$i$列换为$b$的矩阵 
	\[
		B_i:=A[e_1,\ldots,e_{i-1},x,e_{i+1},\ldots,e_n],
	\]
	左右取行列式,则
	\begin{align}
		x_i=\frac{\det(B_i)}{\det(A)}.
	\end{align}
\end{theorem}
Cramer法则的应用之一是矩阵求逆,利用$AA\iv=I$,把$A\iv$的元素看做未知数,解线性方程组
\[
	(A\iv)_{ij}=\frac{C_{ji}}{\det(A)}.
\]
\begin{definition}{伴随矩阵}{adjoint matrix}
	$n$阶方阵$A$的伴随矩阵(adjoint matrix)~$\adj(A)$
	\begin{align}
		\adj(A)_{ij}:=C_{ji}.
	\end{align}
\end{definition}
根据Cramer法则,
\begin{align}
	A\iv=\frac{\adj(A)}{\det(A)}.
\end{align}
\begin{remark}
	即使$A$不可逆,$\adj(A)$依然存在。
\end{remark}
\begin{theorem}{伴随矩阵的性质}{characterist of adjoint}
	伴随矩阵与原矩阵可交换,其乘积为
	\begin{align}
		A\adj(A)=\adj(A)A=\det(A)I.
	\end{align}
\end{theorem}
\begin{proof}
	根据Laplace展开,
	\begin{align*}
		\bigfkh{A\adj(A)}_{ij}&=\sum_{k}A_{ik}\adj(A)_{kj}=\sum_{k}(-)^{j+k}A_{ik}M_{jk}=\vd_{ij}\det(A);\\
		\bigfkh{\adj(A)A}_{ij}&=\sum_{k}\adj(A)_{ik}A_{kj}=\sum_{k}(-)^{i+k}M_{ki}A_{kj}=\vd_{ij}\det(A).%
	\end{align*}
	对应元素相等,即证。
\end{proof}
\begin{corollary}
	进而有
	\[
		\det(A)\det\bigkh{\adj(A)}=\det\bigkh{\det(A)I}=\det(A)^n.
	\]
	故伴随矩阵的行列式
	\begin{align}
		\det\bigkh{\adj(A)}=\det(A)^{n-1}.
	\end{align}
	上面的性质并不要求$A$可逆。特别的,若$A$可逆,则伴随矩阵也可逆
	\begin{align}
		\adj(A)\iv=\bigkh{\det(A)A\iv}\iv=\frac{A}{\det(A)},
	\end{align}
	伴随矩阵的伴随矩阵
	\begin{align}
		\adj\bigkh{\adj(A)}=\adj(A)\iv\det\bigkh{\adj(A)}=\det(A)^{n-2}A.
	\end{align}
\end{corollary}

\begin{remark}
	计算机运用Cramer法则解$n$元线性方程组的时间复杂度是$\bigo(n\cdot n!)$,且在数值上不稳定\footnote{Cramer, Gabriel (1750). "Introduction à l'Analyse des lignes Courbes algébriques" (in French). Geneva: Europeana. pp. 656-659. Retrieved 2012-05-18.},这是不可接受的。与其在计算方面的作用相比,其理论价值更为重大,即:研究了方程组的系数与方程组解的存在性与唯一性关系。
\end{remark}
