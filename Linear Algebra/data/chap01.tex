\chapter{向量和矩阵}

\section{向量}

给定数域(number field)~$\FF$。直到\chapref{chap:complex linear space}复线性空间之前,均只考虑实数域$\FF=\RR$。

\begin{definition}{向量}{vector}
	一个$n$维向量(vector)~$v$由$n$个标量(scalar)~$v_1,\ldots,v_n$组成:
	\[
		v=\begin{bmatrix}
			v_1\\\vdots\\v_n
		\end{bmatrix},\quad v_i\in\FF.
	\]
	组成向量的标量$v_1,\ldots,v_n$称为向量的分量(component)。
	%向量的维度$\dim(v)=n$即分量的个数。
\end{definition}
如无特别说明,向量均默认为列向量,即$n$行1列。
\begin{definition}{零向量、反向量}{zero vector}
	零向量(zero vector)是所有分量均为0的向量,记作0;
	
	$v$的反向量(reverse vector)对应每个分量取相反数,记作$-v$。
\end{definition}
\begin{definition}{向量的加法}{vector addition}
	向量的加法(addition)即对应分量相加:%,只有分量数相同的向量之间才可以相加。
	\[
		v+u=\begin{bmatrix}
			v_1\\\vdots\\v_n
		\end{bmatrix}+\begin{bmatrix}
			u_1\\\vdots\\u_n
		\end{bmatrix}=\begin{bmatrix}
			v_1+u_1\\\vdots\\v_n+u_n
		\end{bmatrix}.
	\]
\end{definition}
\begin{theorem}{向量加法的性质}{property of vector addition}
    \begin{itemize}
    	\item 交换律:$v+u=u+v;$
    	\item 结合律:$v+(u+w)=(v+u)+w.$
    	\item 零向量:$0+v=v+0=v.$
    	\item 反向量:$v+(-v)=0.$
    \end{itemize}
\end{theorem}
\begin{definition}{向量的数乘}{vector scalar product}
	向量与标量的数乘(scalar product)即每个分量乘标量
	\[
		cv=c\begin{bmatrix}
			v_1\\\vdots\\v_n
		\end{bmatrix}=\begin{bmatrix}
			cv_1\\\vdots\\cv_n
		\end{bmatrix}.
	\]
\end{definition}
\begin{theorem}{向量数乘的性质}{property of vector scalar product}
    \begin{itemize}
    	\item $1v=v,\;(-1)v=-v,\;0v=0.$
    	\item 结合律:$c(dv)=(cd)v\equiv cdv;$
    	\item 对标量的分配律:$(c+d)v=cv+dv;$
    	\item 对向量的分配律:$c(v+u)=cv+cu.$
    \end{itemize}
\end{theorem}
\begin{definition}{线性组合}{linear combination}
	向量$v$和$u$的线性组合(linear combination)定义为
	\[
		cv+du,\quad\forall c,d\in\FF.
	\]
	以此类推,$n$个向量$v_1,v_2,\ldots,v_n$的线性组合形如
	\[
		c_1v_1+c_2v_2+\cdots+c_nv_n,\quad c_i\in\FF.
	\]
\end{definition}
线性组合是线性代数中最重要的概念之一。
\begin{definition}{向量的内积}{vector inner product}
	向量$v$和$u$的内积(inner product)结果是一个标量,其值为
	\begin{equation}
		v\cdot u:=\sum_{i=1}^nv_iu_i=v_1u_1+v_2u_2+\cdots+v_nu_n.
	\end{equation}
	特别的,$v^2:=v\cdot v.$
\end{definition}
\begin{theorem}{向量内积的性质}{property of vector inner product}
    \begin{itemize}
    	\item 交换律:$v\cdot u=u\cdot v;$
    	\item 与数乘的结合律:$(cv)\cdot u=c(v\cdot u)\equiv cv\cdot u;$
    	\item 分配律:$(v+u)\cdot w=v\cdot w+u\cdot w;$
    	\item 正定性:$v^2\geqslant 0$,取等号当且仅当$v=0.$
    \end{itemize}
\end{theorem}
\begin{definition}{向量的长度}{vector norm}
	通过内积我们可以定义向量的长度(norm)
	\begin{equation}
		\norm v:=\sqrt{v^2}=\kh{v_1^2+v_2^2+\cdots+v_n^2}^{1/2}.
	\end{equation}
	单位向量(unit vector)是长度为1的向量。和$v$同向的单位向量是
    \begin{equation}
        \hat v:=\frac{v}{\norm v},
    \end{equation}
\end{definition}

% \begin{definition}{两向量间的夹角}{angle between two vectors}
% 	定义向量的夹角$\theta\in[0,\pi]$满足
% 	\[
% 		\cos\theta=\frac{v\cdot u}{\norm v\norm u}.
% 	\]
% \end{definition}
% %故$v\perp u\ifnf v\cdot u=0$。
% \begin{theorem}{Cauchy不等式}{Cauchy Inequality}
% 	因为$\abs{\cos\theta}\leqslant 1$,故 
% 	\begin{equation}
% 		\abs{v\cdot u}\leqslant\norm v\norm u.
% 	\end{equation}
% \end{theorem}

\section{矩阵}
\begin{definition}{矩阵}{matrix}
    $m\times n$的矩阵(matrix)形如
    \[
        A=\begin{bmatrix}
            A_{11}&\cdots&A_{1n}\\
            \vdots&\ddots&\vdots\\
            A_{m1}&\cdots&A_{mn}
        \end{bmatrix},\quad A_{ij}\in\FF.
    \]
    $A_{ij}$是矩阵第$i$行第$j$列的元素。
    
    矩阵的行数和列数分别为$\row(A)=m$,$\col(A)=n$。特别地,当$m=n$时,称矩阵是$n$阶方阵(square matrix)。
\end{definition}

矩阵的加法和数乘与向量的运算规律相同,是平凡的。
\begin{definition}{矩阵和向量的乘法}{matrix vector product}
	$m\times n$矩阵$A$左乘$n$维向量$x$,结果$b=Ax$是一个$m$维向量,其分量为:
	\begin{equation}
		b_i=\sum_{j=1}^nA_{ij}x_j,\quad i=1,\ldots,m.
	\end{equation}
\end{definition}
因此$Ax$是看成$A$所有列的线性组合,或者说$A$的各行与$x$分别内积。这样线性方程组就可以等价地写成$Ax=b$的形式,即:
\[
	\begin{cases}
        A_{11}x_1+\cdots+A_{1n}x_n=b_1,\\
        \qquad\cdots\\
        A_{m1}x_1+\cdots+A_{mn}x_n=b_m,
    \end{cases}\iff
    \begin{bmatrix}
        A_{11}&\cdots&A_{1n}\\
        \vdots&\ddots&\vdots\\
        A_{m1}&\cdots&A_{mn}
    \end{bmatrix}
    \begin{bmatrix}
        x_1\\\vdots\\x_n
    \end{bmatrix}=
    \begin{bmatrix}
        b_1\\\vdots\\b_m
    \end{bmatrix},
\]
这种思想有助于我们掌握线性代数的理念。
\begin{definition}{矩阵的乘法}{matrix product}
	$m\times n$矩阵$A$左乘$n\times p$矩阵$B$,结果是一个$m\times p$的矩阵$C=AB$,其分量为
	\begin{align}
		C_{ij}=\sum_{k=1}^nA_{ik}B_{kj}.
	\end{align}
	因此若$A,B$可以相乘,要求$\col(A)=\row(B).$
\end{definition}
\begin{theorem}{矩阵乘法的性质}{property of matrix product}
    \begin{itemize}
    	\item 一般不满足交换律,即$AB\neq BA$,比如
    	\[
    		\begin{bmatrix}
    			0&1\\1&0
    		\end{bmatrix}\begin{bmatrix}
    			1&0\\0&-1
    		\end{bmatrix}=\begin{bmatrix}
    			0&-1\\1&0
    		\end{bmatrix},\quad\begin{bmatrix}
    			1&0\\0&-1
    		\end{bmatrix}\begin{bmatrix}
    			0&1\\1&0
    		\end{bmatrix}=\begin{bmatrix}
    			0&1\\-1&0
    		\end{bmatrix}.
    	\]
        \item 结合律:$(AB)C=A(BC)\equiv ABC;$
    	\item 左分配律:$(A+B)C=AC+BC;$
    	\item 右分配律:$A(B+C)=AB+AC.$
    \end{itemize}
\end{theorem}
下面给出矩阵乘法结合律的证明。
\begin{proof}
	对于$A_{m\times n},B_{n\times p},C_{p\times q}$
	\begin{align*}
		\bigfkh{A(BC)}_{ij}&=\sum_{k=1}^nA_{ik}(BC)_{kj}=\sum_{k=1}^nA_{ik}\sum_{\ell=1}^pB_{k\ell}C_{\ell j}\\
		&=\sum_{\ell=1}^p\sum_{k=1}^nA_{ik}B_{k\ell}C_{\ell j}=\sum_{\ell=1}^p(AB)_{i\ell}C_{\ell j}=\bigfkh{(AB)C}_{ij}.
	\end{align*}
	对应元素均相等,即证。
\end{proof}
\begin{definition}{单位矩阵}{unit matrix}
	$n$阶单位矩阵(unit matrix)是$n$阶方阵,对角项均为1,其余项均为0:
	\[
		I_n=\begin{bmatrix}
		1&\\ &\ddots\\&&1
		\end{bmatrix}.
	\]
\end{definition}
易证,$\forall m\times n$的矩阵$A$,均有
\[
	I_mA=AI_n=A.
\]
\begin{theorem}{交换性}{commutable}
	若方阵$A$与任意方阵可交换(commutable),则$A=cI,\;c\in\FF$
\end{theorem}
\begin{proof}
	取$B=e_{ij}$,表示仅$i$行$j$列为1,其余项均为0
	\begin{align*}
		(Ae_{ij})_{k\ell}=\sum_{p=1}^nA_{kp}(e_{ij})_{p\ell}=A_{ki}\delta_{j\ell};\\
		(e_{ij}A)_{k\ell}=\sum_{p=1}^n(e_{ij})_{kp}A_{p\ell}=\delta_{ki}A_{j\ell},
	\end{align*}
	当$k\neq i=j=\ell$时,$A_{ki}=0$;当$k=i\neq j=\ell$时,$A_{ii}=A_{jj}$。
\end{proof}
\begin{definition}{分块矩阵}{block matrix}
	可以将矩阵分块,每一块(block)是一个小矩阵,比如
	\begin{align*}
		\fkh{\begin{array}{cc|cc}
			1&0&1&0\\
			0&1&0&1\\
			\hline
			0&0&1&0\\
			0&0&0&1	
		\end{array}}\equiv\begin{bmatrix}
			I&I\\0&I
		\end{bmatrix}.
	\end{align*}
\end{definition}
\begin{remark}
	分块矩阵乘法:每个块当作矩阵的元素,块之间使用矩阵乘法。
\end{remark}

\section{矩阵的逆和转置}

\begin{definition}{矩阵的逆}{inverse matrix}
	方阵$A$的逆矩阵(inverse matrix)~$A\iv$满足
	\[
		AA\iv=A\iv A=I.
	\]
	奇异矩阵(singular matrix)是不可逆的矩阵。
\end{definition}
若$A,B$均可逆,则$(AB)\iv=B\iv A\iv$。但$A+B$不一定可逆。
\begin{example}{二阶方阵的逆}{inverse of 2x2 matrix}
	\begin{equation}
		\begin{bmatrix}
			a&b\\c&d
		\end{bmatrix}\iv=\frac1{ad-bc}
		\begin{bmatrix}
			d&-b\\
			-c&a
		\end{bmatrix}.
	\end{equation}
\end{example}
\begin{example}{对角矩阵的逆}{inverse of diagonal matrix}
	非对角项均为0的方阵称为对角矩阵(diagonal matrix),若对角项为$a_1,\ldots,a_n$,可记作
	\[
		\diag(a_1,\ldots,a_n).
	\]
	对角矩阵有很多简单的性质,比如对角矩阵的乘法很简单
	\begin{align}
		\diag(a_1,\ldots,a_n)\diag(b_1,\ldots,b_n)=\diag(a_1b_1,\ldots,a_nb_n).
	\end{align}
	%故
	显然,对角矩阵的逆为
	\begin{align}
		\diag(a_1,\ldots,a_n)\iv=\diag(a_1^{-1},\ldots,a_n^{-1}).
	\end{align}
\end{example}
% 若$A$可逆,则方程组$Ax=b$的解为 \[x=A\iv b.\]
\begin{theorem}{左逆和右逆}{linverse and rinverse}
	若左逆(left inverse)存在,则其也是右逆(right inverse),反之也成立。
\end{theorem}
\begin{proof}
	设$M_n$为全体$n$阶方阵的集合,$\forall A\in M_n$且存在左逆$B\in M_n$,即$BA=I$。构造映射$f,g: M_n\to M_n$,其中
	\[
		f:X\mapsto AXB,\quad g:X\mapsto BXA,
	\]
	而$g(f(X))=BAXBA=X$,故$g$是可逆的,又
	\[
		g(AB)=BABA=I=g(I),
	\]
	故$AB=I$。
\end{proof}
\begin{example}{幂零矩阵}{nilpotent matrix}
	若$A$是幂零矩阵(nilpotent matrix),即$\exists n\in\mathbb N$使$A^n=O$,
	
	则$I+A$可逆,且
	\[
		(I+A)\iv=I-A+A^2-\cdots+(-A)^{n-1}.
	\]
	因为
	\[
		I=I^n+A^n=(I+A)\bigfkh{I^{n-1}-I^{n-2}A+\cdots+(-A)^{n-1}}.
	\]
\end{example}
%\section{矩阵的转置}
\begin{definition}{矩阵的转置}{transpose matrix}
	$m\times n$矩阵$A$的转置(transpose)~$A\tp$是$n\times m$矩阵,且 
	\[
		(A\tp)_{ij}=A_{ji}.
	\]
	特别的,若$S\tp=S$,则$S$是对称矩阵(sysmmetric matrix)。
\end{definition}
\begin{theorem}{矩阵转置的性质}{property of transpose}
    \begin{itemize}
    	\item $(A+B)\tp=A\tp+B\tp,\enspace (cA)\tp=cA\tp;$
    	\item $(AB)\tp=B\tp A\tp;$
    	\item $(A\iv)\tp=(A\tp)\iv.$
    \end{itemize}
\end{theorem}
\section{矩阵的迹}
\begin{definition}{矩阵的迹}{trace}
	$n$阶方阵的迹(trace)是对角元的和
	\begin{equation}
		\tr(A):=\sum_{i=1}^nA_{ii}.
	\end{equation}
\end{definition}
\begin{theorem}{矩阵迹的性质}{property of trace}
    \begin{itemize}
    	\item $\tr(A+B)=\tr(A)+\tr(B),\enspace \tr(cA)=c\tr(A);$
    	\item $\tr(A\tp)=\tr(A);$
    	\item 两个$n$维列向量$u,v$,有$\tr(uv\tp)=v\tp u$
    	\[
    		\tr(uv\tp)=\sum_{i=1}^nu_iv_i=u\tp v=v\tp u.
    	\]
    	\item $\tr(AB)\neq\tr(A)\tr(B).$
    \end{itemize}
\end{theorem}
\begin{theorem}{交换矩阵乘法的迹}{trace of commute multiplication}
	$\forall m\times n$的矩阵$A$和$n\times m$的矩阵$B$,都有
	\begin{equation}
		\tr(AB)=\tr(BA).
	\end{equation}
\end{theorem}
\begin{proof}
	直接展开计算:
	\begin{align*}
		\tr(AB)&=\sum_{i=1}^m(AB)_{ii}=\sum_{i=1}^m\sum_{j=1}^nA_{ij}B_{ji}\\
		&=\sum_{j=1}^n\sum_{i=1}^mB_{ji}A_{ij}=\sum_{j=1}^n(BA)_{jj}=\tr(BA).
		\qedhere
	\end{align*}
\end{proof}
除此之外,迹还有一些重要性质,将在后面讲。