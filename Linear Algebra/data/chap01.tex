\chapter{向量和矩阵}

\section{向量}

\begin{definition}
	{数域}{number field}
	给定复数集的子集$\FF\subset\CC$,若满足:
	\begin{itemize}
		\item 非平凡:$0,1\in\FF$;
		\item 封闭性:$\forall a,b\in\FF$,有$a\pm b\in\FF$,$ab\in\FF$,且$a/b\in\FF$ (当$b\neq 0$时)。
	\end{itemize}
	则称$\FF$是一个数域(number field)。
\end{definition}

\begin{example}
	{数域的例子}{number field}
	\begin{itemize}
		\item 最小的数域:有理数域$\QQ$;
	
		\item 二次数域(quadratic field):如$\QQ(\sqrt 2)=\nset{a+b\sqrt2}{a,b\in\QQ}$;
	
		\item 实数域$\RR$、复数域$\CC$。
	\end{itemize}
\end{example}

\begin{remark}
	直到\chapref{chap:complex linear space}复线性空间之前,均只考虑实数域$\FF=\RR$。
\end{remark}

\begin{definition}{向量}{vector}
	一个$n$维向量(vector)~$v$由$n$个标量(scalar) $v_1,\ldots,v_n\in\FF$组成,记作:
	\[
		v=\begin{bmatrix}
			v_1\\\vdots\\v_n
		\end{bmatrix},
	\]
	组成向量$v$的标量$v_1,\ldots,v_n$称为$v$的分量(component)。
	%向量的维度$\dim(v)=n$即分量的个数。
	\tcblower
	所有$n$维向量构成的集合记作$\FF^n$。
\end{definition}

\begin{remark}
	如无特别说明,向量均默认为列向量,即$n$行1列。
\end{remark}

\begin{definition}{零向量、反向量}{zero vector}
	零向量(zero vector)是所有分量均为0的向量,记作0;
	
	一个向量$v$的反向量(opposite vector)对应每个分量取相反数,记作$-v$。
\end{definition}

% \paragraph{向量的加法和数乘}

\begin{definition}{向量的加法}{vector addition}
	向量的加法(addition)即对应分量相加。
	向量的减法定义为与其反向量相加。
	%,只有分量数相同的向量之间才可以相加。
	% \[
	% 	v+u=\begin{bmatrix}
	% 		v_1\\\vdots\\v_n
	% 	\end{bmatrix}+\begin{bmatrix}
	% 		u_1\\\vdots\\u_n
	% 	\end{bmatrix}=\begin{bmatrix}
	% 		v_1+u_1\\\vdots\\v_n+u_n
	% 	\end{bmatrix}.
	% \]
\end{definition}

\begin{corollary}
	向量加法的性质:%property of vector addition
	$\forall v,u,w\in\FF^n$,
    \begin{itemize}
    	\item 交换律:$v+u=u+v$;
    	\item 结合律:$v+(u+w)=(v+u)+w\equiv v+u+w$;
    	\item 零向量:$0+v=v+0=v$;
    	\item 反向量:$v+(-v)=0$。
    \end{itemize}
\end{corollary}

\begin{definition}{向量的数乘}{vector scalar product}
	向量与标量的数乘(scalar product)即每个分量乘标量。
	% \[
	% 	cv=c\begin{bmatrix}
	% 		v_1\\\vdots\\v_n
	% 	\end{bmatrix}=\begin{bmatrix}
	% 		cv_1\\\vdots\\cv_n
	% 	\end{bmatrix}.
	% \]
\end{definition}

\begin{corollary}
	向量数乘的性质:%property of vector scalar product
	$\forall v,u\in\FF^n,c,d\in\FF$,
    \begin{itemize}
    	\item $1v=v,\;(-1)v=-v,\;0v=0.$
    	\item 结合律:$c(dv)=(cd)v\equiv cdv;$
    	\item 对标量的分配律:$(c+d)v=cv+dv;$
    	\item 对向量的分配律:$c(v+u)=cv+cu.$
    \end{itemize}
\end{corollary}

\begin{definition}{线性组合}{linear combination}
	% 向量$v$和$u$的线性组合形如
	% \[
	% 	cv+du,\quad\forall c,d\in\FF.
	% \]
	一般地,$n$个向量$v_1,v_2,\ldots,v_n$的线性组合(linear combination)形如
	\[
		c_1v_1+c_2v_2+\cdots+c_nv_n,\quad\forall c_i\in\FF.
	\]
\end{definition}

\begin{remark}
	线性组合是线性代数中最重要的概念之一。
\end{remark}

\begin{definition}{(实)向量的内积}{vector inner product}
	两个向量的内积(inner product)结果是一个实数$\inp\cdot\cdot:\FF^n\times\FF^n\to\RR$。对于实向量
	\begin{equation}
		\inp vu:=\sum_{i=1}^nv_iu_i\equiv v_1u_1+v_2u_2+\cdots+v_nu_n.
	\end{equation}
	有时可以将内积写成点乘的形式:$\inp vu=v\cdot u$。
	% 特别地,简记$v^2\equiv\inp vv.$
\end{definition}

\begin{corollary}
	实向量内积的性质:%property of vector inner product
	$\forall v,u\in\FF^n,c\in\FF$,
    \begin{itemize}
    	\item 交换律:$\inp vu=\inp uv$;
    	\item 与数乘的结合律:$\inp{cv}u=\inp v{cu}=c\inp vu$;
    	\item 分配律:$\inp{v+u}w=\inp vw+\inp uw$;
    	\item 正定性:$\inp vv\geqslant 0$,且$\inp vv=0\iff v=0.$
    \end{itemize}
\end{corollary}

\begin{definition}{向量的长度}{vector norm}
	向量的长度(或范数,norm)可通过内积定义:
	\begin{equation}
		\norm v:=\sqrt{\inp vv}=\kh{v_1^2+v_2^2+\cdots+v_n^2}^{1/2}.
	\end{equation}
	长度为1的向量是单位向量(unit vector)。
\end{definition}
\begin{corollary}
	与向量$v$同向的单位向量是
	\begin{equation}
		\hat v:=\frac{v}{\norm v},
	\end{equation}
\end{corollary}
\begin{remark}
	仅第$i$个分量为1,其余均为0的单位向量记为$e_i$。
\end{remark}

% \begin{definition}{两向量间的夹角}{angle between two vectors}
% 	定义向量的夹角$\theta\in[0,\pi]$满足
% 	\[
% 		\cos\theta=\frac{v\cdot u}{\norm v\norm u}.
% 	\]
% \end{definition}
% %故$v\perp u\ifnf v\cdot u=0$。
% \begin{theorem}{Cauchy不等式}{Cauchy Inequality}
% 	因为$\abs{\cos\theta}\leqslant 1$,故 
% 	\begin{equation}
% 		\abs{v\cdot u}\leqslant\norm v\norm u.
% 	\end{equation}
% \end{theorem}

\section{矩阵}
\begin{definition}{矩阵}{matrix}
    一个$m$行$n$列的矩阵(matrix) $A$形如
    \[
        A=\begin{bmatrix}
            A_{11}&\cdots&A_{1n}\\
            \vdots&\ddots&\vdots\\
            A_{m1}&\cdots&A_{mn}
        \end{bmatrix},\quad A_{ij}\in\FF.
    \]
    其中$A_{ij}$是矩阵第$i$行第$j$列的元素(entry)。
    % 矩阵$A$的行数和列数分别记作$\row(A)=m,\enspace\col(A)=n$。
	特别地,当行数$m=$列数$n$时,称矩阵是$n$阶方阵(square matrix)。
    \tcblower
	所有$m$行$n$列矩阵构成的集合记作$\FF^{m\times n}$。
\end{definition}

矩阵加法和数乘的运算规律与向量相同,是平凡的(trivial)。

\begin{definition}
	{矩阵和向量的乘法(左乘)}{matrix vector product}
	$m\times n$矩阵$A$左乘$n$维向量$x$,结果$b=Ax$是一个$m$维向量,其各分量为:
	\begin{equation}
		b_i=\sum_{j=1}^nA_{ij}x_j,\quad i=1,\ldots,m.
	\end{equation}
\end{definition}

\begin{remark}
	$Ax$可以看成$A$所有列的线性组合,或者说$A$的各行与$x$分别内积。这样线性方程组就可以等价地写成$Ax=b$的形式,即:
	\[
		\lhkh{\begin{aligned}
	        A_{11}x_1+\cdots+A_{1n}x_n&=b_1,\\
	        \cdots&\\
	        A_{m1}x_1+\cdots+A_{mn}x_n&=b_m,
	    \end{aligned}}\iff
	    \begin{bmatrix}
	        A_{11}&\cdots&A_{1n}\\
	        \vdots&\ddots&\vdots\\
	        A_{m1}&\cdots&A_{mn}
	    \end{bmatrix}
	    \begin{bmatrix}
	        x_1\\\vdots\\x_n
	    \end{bmatrix}=
	    \begin{bmatrix}
	        b_1\\\vdots\\b_m
	    \end{bmatrix},
	\]
	这种思想有助于我们掌握线性代数的理念。
\end{remark}

\begin{definition}{矩阵的乘法}{matrix product}
	$m\times n$矩阵$A$左乘$n\times p$矩阵$B$,结果$AB$是一个$m\times p$的矩阵,其分量为
	\begin{align}
		(AB)_{ij}=\sum_{k=1}^nA_{ik}B_{kj}.
	\end{align}
	特别地,当$A$是方阵时,将$n$个$A$相乘简记为$A^n$。
	% \[
	% 	A^n\equiv\underbrace{A\cdots A}_n.
	% \]
\end{definition}

\begin{remark}
	若$A$可以左乘$B$,要求$A$的列数$=B$的行数。
\end{remark}

\begin{corollary}
	矩阵乘法的性质:%property of matrix product
	$\forall A,B,C$可运算的,
    \begin{itemize}
    	\item 左分配律:$(A+B)C=AC+BC;$
    	\item 右分配律:$A(B+C)=AB+AC.$
        \item 结合律:$(AB)C=A(BC)\equiv ABC;$
    \end{itemize}
\end{corollary}

\begin{proof}
	仅给出矩阵乘法结合律的证明:
	$\forall A\in\FF^{m\times n},B\in\FF^{n\times p},C\in\FF^{p\times q}$
	\begin{align*}
		\bigfkh{A(BC)}_{ij}&=\sum_{k=1}^nA_{ik}(BC)_{kj}=\sum_{k=1}^nA_{ik}\sum_{\ell=1}^pB_{k\ell}C_{\ell j}\\
		&=\sum_{\ell=1}^p\sum_{k=1}^nA_{ik}B_{k\ell}C_{\ell j}=\sum_{\ell=1}^p(AB)_{i\ell}C_{\ell j}=\bigfkh{(AB)C}_{ij}.
	\end{align*}
	对应元素均相等,即证。
\end{proof}

\begin{remark}
	矩阵乘法一般不满足交换律,即一般地,$AB\neq BA$,
	比如
	\[
		\begin{bmatrix}
			0&1\\1&0
		\end{bmatrix}\begin{bmatrix}
			1&0\\0&-1
		\end{bmatrix}=\begin{bmatrix}
			0&-1\\1&0
		\end{bmatrix}\neq\begin{bmatrix}
			0&1\\-1&0
		\end{bmatrix}=\begin{bmatrix}
			1&0\\0&-1
		\end{bmatrix}\begin{bmatrix}
			0&1\\1&0
		\end{bmatrix}.
	\]
	若$AB=BA$,则称$A,B$是可交换的(commutable)。
\end{remark}

\begin{corollary}
	若$A,B$可交换,则$A,B$必须首先为同阶方阵。
\end{corollary}

\begin{definition}
	{对角矩阵}{diagonal matrix}
	当方阵$A$的非对角项均为0时,称$A$是对角的(diagonal),可记作
	\begin{equation}
		A=\begin{bmatrix}
			A_{11}\\ &\ddots\\ &&A_{nn}
		\end{bmatrix}\equiv\diag(A_{11},\ldots,A_{nn}).
	\end{equation}
\end{definition}

\begin{remark}
	对角矩阵有很多简单的性质,比如对角矩阵的乘法很简单
	\begin{align}
		\diag(a_1,\ldots,a_n)\diag(b_1,\ldots,b_n)=\diag(a_1b_1,\ldots,a_nb_n).
	\end{align}
	因此任意同阶对角矩阵可交换。
\end{remark}

\begin{definition}{单位矩阵}{identity matrix}
	$n$阶单位矩阵(identity matrix)是$n$阶对角矩阵,对角项均为1:
	\[
		I_n=\begin{bmatrix}
		1&\\ &\ddots\\&&1
		\end{bmatrix}.
	\]
\end{definition}

\begin{corollary}
	$\forall A\in\FF^{m\times n}$,均有
	\[
		I_mA=AI_n=A.
	\]
\end{corollary}

\begin{theorem}{}{}
	若方阵$A$与任意方阵可交换,则$A=cI,\;c\in\FF$称作纯量矩阵(scalar matrix)。
\end{theorem}
\begin{proof}
	定义$e_{ij}\in\FF^{n\times n}$表示仅$i$行$j$列为1,其余项均为0的矩阵。则
	\begin{align*}
		(Ae_{ij})_{k\ell}=\sum_{p=1}^nA_{kp}(e_{ij})_{p\ell}=A_{ki}\delta_{j\ell};\\
		(e_{ij}A)_{k\ell}=\sum_{p=1}^n(e_{ij})_{kp}A_{p\ell}=\delta_{ki}A_{j\ell},
	\end{align*}
	当$k\neq i=j=\ell$时,$A_{ki}=0$;当$k=i\neq j=\ell$时,$A_{ii}=A_{jj}$。
\end{proof}
\begin{definition}{分块矩阵}{block matrix}
	可以将矩阵分块,每一块(block)是一个小矩阵,比如
	\begin{align*}
		\fkh{\begin{array}{cc|cc}
			1&0&1&0\\
			0&1&0&1\\
			\hline
			0&0&1&0\\
			0&0&0&1	
		\end{array}}\equiv\begin{bmatrix}
			I&I\\O&I
		\end{bmatrix}.
	\end{align*}
\end{definition}

\begin{remark}
	分块矩阵乘法:每个块当作矩阵的元素,块之间使用矩阵乘法。
\end{remark}



% \subsection{矩阵转置}

\begin{definition}{矩阵的转置}{transpose matrix}
	矩阵$A\in\FF^{m\times n}$的转置(transpose)记作$A\tp\in\FF^{n\times m}$,其元素由$A$给出:
	\[
		(A\tp)_{ij}=A_{ji}.
	\]
\end{definition}
\begin{corollary}
	矩阵转置的性质:%property of transpose
	$\forall A,B$可运算的,
	\begin{itemize}
		\item $(A\tp)\tp=A$;
		\item $(A+B)\tp=A\tp+B\tp$,$(cA)\tp=cA\tp$;
		\item $(AB)\tp=B\tp A\tp$。
	\end{itemize}
\end{corollary}

% \subsection{矩阵的迹}

\begin{definition}{矩阵的迹}{trace}
	$n$阶方阵的迹(trace)是对角元的和
	\begin{equation}
		\tr(A):=\sum_{i=1}^nA_{ii}.
	\end{equation}
\end{definition}
\begin{corollary}
	矩阵迹的性质:%property of trace
	$\forall A,B\in\FF^{n\times n},\enspace c\in\FF$
    \begin{itemize}
    	\item 线性:$\tr(A+B)=\tr(A)+\tr(B),\enspace \tr(cA)=c\tr(A)$;
    	\item 转置:$\tr(A\tp)=\tr(A)$;
    	% \item 两个$n$维列向量$u,v$,有$\tr(uv\tp)=v\tp u$
    	% \[
    	% 	\tr(uv\tp)=\sum_{i=1}^nu_iv_i=u\tp v=v\tp u.
    	% \]
    \end{itemize}
\end{corollary}
\begin{remark}
	一般地,$\tr(AB)\neq\tr(A)\tr(B)$。
\end{remark}
\begin{theorem}{交换矩阵乘法的迹}{trace of commute multiplication}
	$\forall A\in\FF^{m\times n},B\in\FF^{n\times m}$,都有
	\begin{equation}
		\tr(AB)=\tr(BA).
	\end{equation}
\end{theorem}
\begin{proof}
	直接展开计算:
	\begin{align*}
		\tr(AB)&=\sum_{i=1}^m(AB)_{ii}=\sum_{i=1}^m\sum_{j=1}^nA_{ij}B_{ji}\\
		&=\sum_{j=1}^n\sum_{i=1}^mB_{ji}A_{ij}=\sum_{j=1}^n(BA)_{jj}=\tr(BA).
		\qedhere
	\end{align*}
\end{proof}

\begin{remark}
	除此之外,迹还有一些重要性质,将在后面讲。
\end{remark}

\section{矩阵的逆}

\begin{definition}{矩阵的逆}{inverse matrix}
	方阵$A\in\FF^{n\times n}$的逆矩阵(inverse)记作$A\iv\in\FF^{n\times n}$,满足
	\[
		AA\iv=A\iv A=I_n.
	\]
	若存在$A\iv$,则称矩阵$A$是可逆的;否则称为不可逆的,也称奇异的(singular)。
\end{definition}

\begin{corollary}
	矩阵逆的性质:
	$\forall A,B\in\FF^{n\times n}$可逆,
	\begin{itemize}
		\item 逆矩阵的逆:$(A\iv)\iv=A$;
		\item 数乘:$(cA)\iv=c\iv A\iv$;
		\item 矩阵乘法:$(AB)\iv=B\iv A\iv$;
		\item 转置:$(A\iv)\tp=(A\tp)\iv$。
	\end{itemize}
\end{corollary}

\begin{remark}
	一般地,$A,B$均可逆$\notimplies(A+B)$可逆。
	% 除非满足一些额外性质,见\thmref{thm:Woodbury matrix identity}。
\end{remark}

\begin{example}{二阶方阵的逆}{inverse of 2x2 matrix}
	\begin{equation}
		\begin{bmatrix}
			a&b\\c&d
		\end{bmatrix}\iv=\frac1{ad-bc}
		\begin{bmatrix}
			d&-b\\
			-c&a
		\end{bmatrix}.
	\end{equation}
\end{example}
\begin{example}{对角矩阵的逆}{inverse of diagonal matrix}
	显然,对角矩阵可逆$\iff$所有对角元均不为0,且逆为
	\begin{align}
		\diag(a_1,\ldots,a_n)\iv=\diag(a_1^{-1},\ldots,a_n^{-1}).
	\end{align}
	这对于分块对角矩阵也成立,即式中$a_i$可以为可逆矩阵。
\end{example}
% 若$A$可逆,则方程组$Ax=b$的解为 \[x=A\iv b.\]
\begin{theorem}{左逆和右逆}{linverse and rinverse}
	若$A,B,C$为方阵,且$BA=AC=I$,则$B=C$。
\end{theorem}
\begin{proof}
	考察恒等式$B(AC)=(BA)C$即得$B=C$。
\end{proof}
\begin{remark}
	事实上,可以有更强的命题:若$A,B$为方阵,则$BA=I\iff AB=I$。
\end{remark}
% \begin{proof}
% 	$\forall A\in\FF^{n\times n}$且存在左逆$B\in\FF^{n\times n}$,即$BA=I$。构造映射$f,g:\FF^{n\times n}\to\FF^{n\times n}$,其中
% 	\[
% 		f:X\mapsto AXB,\quad g:X\mapsto BXA,
% 	\]
% 	而$g(f(X))=BAXBA=X$,故$g$是满射,又
% 	\[
% 		g(AB)=BABA=I=g(I),
% 	\]
% 	故$AB=I$。
% \end{proof}
\begin{example}{幂零矩阵}{nilpotent matrix}
	若$A$是幂零矩阵(nilpotent matrix),即$\exists n\in\mathbb N$使$A^n=O$,
	则$I+A$可逆,且
	\[
		(I+A)\iv=I-A+A^2-\cdots+(-A)^{n-1}.
	\]
	因为
	\[
		(I+A)\bigkh{I-A+\cdots+(-A)^{n-1}}=I+A^n=I.
	\]
\end{example}

\subsectionstar{Woodbury矩阵恒等式}

\begin{theorem}
	{Woodbury矩阵恒等式}{Woodbury matrix identity}
	给定$A\in\FF^{n\times n},U\in\FF^{n\times k},C\in\FF^{k\times k},V\in\FF^{k\times n}$且$A,C,(C\iv+VA\iv U)$可逆,则$(A+UCV)$可逆,且 
	\begin{equation}
		\label{eqn:Woodbury}
		(A+UCV)\iv=A\iv-A\iv U(C\iv+VA\iv U)\iv VA\iv.
	\end{equation}
\end{theorem}
\begin{proof}
	直接代入验证即可:
	\begin{align*}
		&(A+UCV)(A\iv-A\iv U(C\iv+VA\iv U)\iv VA\iv)\\
		% ={}&(A+UCV)A\iv(I-U(C\iv+VA\iv U)\iv VA\iv)\\
		% ={}&(I+UCVA\iv)(I-U(C\iv+VA\iv U)\iv VA\iv)\\
		% ={}&I-UCVA\iv-(I+UCVA\iv)U(C\iv+VA\iv U)\iv VA\iv\\
		% ={}&I-UCVA\iv-U(I+CVA\iv U)(C\iv+VA\iv U)\iv VA\iv\\
		={}&I-UCVA\iv-UC(C\iv+VA\iv U)(C\iv+VA\iv U)\iv VA\iv=I.
		\qedhere
		% ={}&I-UCVA\iv-UCVA\iv
	\end{align*}
\end{proof}

\begin{theorem}
	{Sherman-Morrison公式}{Sherman-Morrison formula}
	给定可逆矩阵$A\in\FF^{n\times n}$和向量$u,v\in\FF^n$,则$(A+uv\tp)$可逆$\iff 1+v\tp A\iv u\neq 0$,此时
	\begin{equation}
		\label{eqn:Sherman-Morrison}
		(A+uv\tp)\iv=A\iv-\frac{A\iv uv\tp A\iv}{1+v\tp A\iv u}.
	\end{equation}
	特别地,当$A=I$时,
	\begin{equation}
		\label{eqn:(I+uvT)-1}
		(I+uv\tp)\iv=I-\frac{uv\tp}{1+v\tp u}.
	\end{equation}
\end{theorem}
\begin{proof}
	对Woodbury恒等式\eqref{eqn:Woodbury}取$A=I_n,\enspace C=I_k$,得到
	\begin{equation}
		\label{eqn:(I-uvT)-1}
		(I_n+UV)\iv=I_n-U(I_k+VU)\iv V.
	\end{equation}
	特别地,当$k=1$时,$U,V$都是向量,
	令$U=u,\enspace V=v\tp$即得式\eqref{eqn:(I+uvT)-1},将$u$替换为$A\iv u$即得式\eqref{eqn:Sherman-Morrison}。
\end{proof}

\begin{example}
	{Sherman-Morrison公式的应用}{}
	求逆矩阵:
	\[
		A=\begin{bmatrix}
			0&1&\cdots&1\\
			1&0&\cdots&1\\
			\vdots&\vdots&\ddots&\vdots\\
			1&1&\cdots&0
		\end{bmatrix}
	\]
	设$v=[1\enspace\cdots\enspace 1]\tp$,不难看出$A=-I+vv\tp$,故
	\[
		A\iv=-I-\frac{vv\tp}{1-n}=\frac1{n-1}\begin{bmatrix}
			2-n&1&\cdots&1\\
			1&2-n&\cdots&1\\
			\vdots&\vdots&\ddots&\vdots\\
			1&1&\cdots&2-n
		\end{bmatrix}.
	\]
\end{example}

\begin{theorem}
	{华罗庚恒等式}{Hua's identity}
	对Woodbury恒等式\eqref{eqn:Woodbury}取$n=k$且$U=V=I_n$,有 
		\begin{subequations}
			\begin{align}
				(A+B)\iv&=A\iv-A\iv(A\iv+B\iv)\iv A\iv\\
				\label{eqn:Hua}
				&=A\iv-(A+AB\iv A)\iv.
			\end{align}
		\end{subequations}
		% 式\eqref{eqn:Hua}称为华罗庚恒等式()。
\end{theorem}

\subsectionstar{分块矩阵的逆}

给定分块矩阵
\[
	M=\begin{bmatrix}
		A&B\\C&D
	\end{bmatrix},
\]
当$M$满足某些条件时,其逆可以被表达出来。比如$M$为分块对角时,
\begin{equation}
	\begin{bmatrix}
		A\\ &D
	\end{bmatrix}\iv=\begin{bmatrix}
		A\iv\\ &D\iv
	\end{bmatrix}.
\end{equation}
下面介绍比较一般的情况,首先引入Schur补。

\begin{definition}
	{Schur补}{Schur complement}
	当$A$可逆时,记$M/A:=D-CA\iv B$是$A$在$M$中的Schur补(Schur complement);
	当$D$可逆时,记$M/D:=A-BD\iv C$是$D$在$M$中的Schur补。
\end{definition}

\begin{theorem}
	{分块矩阵的逆}{inverse of block matrix}
	当$A$可逆且$M/A=D-CA\iv B$可逆时,$M$可逆,且
	\begin{subequations}
		\begin{align}
			\begin{bmatrix}
				A&B\\C&D
			\end{bmatrix}\iv&=\begin{bmatrix}
				I&-A\iv B\\ &I
			\end{bmatrix}\begin{bmatrix}
				A\iv\\ &(D-CA\iv B)\iv
			\end{bmatrix}\begin{bmatrix}
				I\\-CA\iv&I
			\end{bmatrix}\\
			\label{eqn:[AB;CD]-1=M/A}
			&=\begin{bmatrix}
				A\iv+A\iv B(M/A)\iv CA\iv&-A\iv B(M/A)\iv\\
				-(M/A)\iv CA\iv&(M/A)\iv
			\end{bmatrix};
		\end{align}
	\end{subequations}
	当$D$可逆且$M/D=A-BD\iv C$可逆时,$M$可逆,且
	\begin{subequations}
		\begin{align}
			\begin{bmatrix}
				A&B\\C&D
			\end{bmatrix}\iv&=\begin{bmatrix}
				I\\-D\iv C&I
			\end{bmatrix}\begin{bmatrix}
				(A-BD\iv C)\iv\\ &D\iv
			\end{bmatrix}\begin{bmatrix}
				I&-BD\iv\\ &I
			\end{bmatrix}\\
			\label{eqn:[AB;CD]-1=M/D}
			&=\begin{bmatrix}
				(M/D)\iv&-(M/D)\iv BD\iv\\
				-D\iv C(M/D)\iv&D\iv+D\iv C(M/D)\iv BD\iv
			\end{bmatrix}.
		\end{align}
	\end{subequations}
	\tcblower 
	当$A,D,M/A,M/D$均可逆时,$M$的逆可以写成简单的分解:
	\begin{eqnarray}
		\begin{bmatrix}
			A&B\\C&D
		\end{bmatrix}\iv=\begin{bmatrix}
			(A-BD\iv C)\iv\\ &(D-CA\iv B)\iv
		\end{bmatrix}\begin{bmatrix}
			I&-BD\iv\\-CA\iv&I
		\end{bmatrix}.
	\end{eqnarray}
\end{theorem}

\begin{proof}
	可以直接代入验证。下面介绍得到这一形式的思路:
	由于分块对角矩阵的逆是易求的。
	当$A$可逆时,可以利用Gauss-Jordan消元法(见\secref{sec:elimination})将$M$化为分块对角的,即LDU分解(见\thmref{thm:LDU decomposition})
	\[
		\begin{bmatrix}
			A&B\\C&D
		\end{bmatrix}=\begin{bmatrix}
			I\\CA\iv&I
		\end{bmatrix}\begin{bmatrix}
			A\\ &D-CA\iv B
		\end{bmatrix}\begin{bmatrix}
			I&A\iv B\\ &I
		\end{bmatrix},
	\]
	上式的分块对角矩阵中出现了$A$的Schur补$M/A$;当$D$可逆时,同理有
	\[
		\begin{bmatrix}
			A&B\\C&D
		\end{bmatrix}=\begin{bmatrix}
			I&BD\iv\\ &I
		\end{bmatrix}\begin{bmatrix}
			A-BD\iv C\\ &D
		\end{bmatrix}\begin{bmatrix}
			I\\D\iv C&I
		\end{bmatrix},
	\]
	对上两式求逆即得。
\end{proof}

\begin{corollary}
	对比\eqref{eqn:[AB;CD]-1=M/A}和\eqref{eqn:[AB;CD]-1=M/D},可得
	\begin{subequations}
		\begin{align}
			A\iv+A\iv B(M/A)\iv CA\iv&=(M/D)\iv\\
			-A\iv B(M/A)\iv&=-(M/D)\iv BD\iv\\
			-(M/A)\iv CA\iv&=-D\iv C(M/D)\iv\\
			(M/A)\iv&=D\iv+D\iv C(M/D)\iv BD\iv
		\end{align}
	\end{subequations}
	可得Woodbury矩阵恒等式(\thmref{thm:Woodbury matrix identity})。
\end{corollary}

\begin{corollary}
	当$C=O$时,$M$可逆$\iff A,D$均可逆,且
	\begin{equation}
		\begin{bmatrix}
			A&B\\ &D
		\end{bmatrix}\iv=\begin{bmatrix}
			A\iv&-A\iv BD\iv\\
			& D\iv
		\end{bmatrix};
	\end{equation}
\end{corollary}

\begin{theorem}
	{分块反对角矩阵的逆}{}
	当$M$是分块反对角的,即$A,D=O$时,$M$可逆$\iff B,C$均可逆,且
	\begin{equation}
		\begin{bmatrix}
			&B\\C
		\end{bmatrix}\iv=\begin{bmatrix}
			&C\iv\\B\iv
		\end{bmatrix}.
	\end{equation}
\end{theorem}


