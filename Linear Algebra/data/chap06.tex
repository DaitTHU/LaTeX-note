
\chapter{特征值和特征向量}
\section{特征值和特征向量}
\begin{definition}{特征值和特征向量}{eigenvalue and eigenvector}
	方阵$A$的特征向量(eigenvector)~$x\neq 0$满足:
	\[
		Ax=\lambda x,\quad\lambda\in\FF.
	\]
	其中$\lambda$称为$A$的特征值(eigenvalue)。
\end{definition}
\begin{theorem}{特征值的性质}{}
	\begin{compactenum}
		\item $A$的特征向量$x$也是$A^n$的特征向量,特征值是$\lambda^n;$
		\item 若$A$可逆,则$x$也是$A\iv$的特征向量,特征值是$\lambda^{-1};$
		\item 三角矩阵的特征值就是对角元;
		\item $A$可逆$\iff A$所有特征值非0。
		
		只需注意到$A$可逆时$Ax=0$只有零解即可。
	\end{compactenum}
\end{theorem}
\begin{definition}{特征子空间}{eigen-subspace}
	$A$的所有特征值为$\lambda$的特征向量再加上0构成$\RR^n$的一个线性子空间,这个子空间就是$\N(A-\lambda I).$
\end{definition}
\begin{theorem}{特征值}{eigenvalue}
	假设$n$阶矩阵$A$有特征向量$x_1,\ldots,x_r$,对应特征值为$\lambda_1,\ldots,\lambda_r$,且这些特征值两两不等,则$x_1,\ldots,x_r$线性无关。
\end{theorem}
\begin{proof}
	运用数学归纳法证明。$r=1$时,定理自动成立;
	
	假设$r=m-1$定理成立,若定理在$r=m$时不成立,则
	\begin{equation}
		\label{eqn:proof eigen 1}
		x_m=c_1x_1+\cdots+c_{m-1}x_{m-1},\tag{$\ast$}
	\end{equation}
	两边同时左乘$A$得
	\begin{equation}
		\label{eqn:proof eigen 2}
		\lambda_mx_m=c_1\lambda_1x_1+\cdots+c_{m-1}\lambda_{m-1}x_{m-1},\tag{$\ast\ast$}
	\end{equation}
	$\lambda_m$\eqref{eqn:proof eigen 1} $-$ \eqref{eqn:proof eigen 2}得,
	\[
		0=c_1(\lambda_m-\lambda_1)x_1+\cdots+c_{m-1}(\lambda_m-\lambda_{m-1})x_{m-1},
	\]
	由$x_1,\ldots,x_{m-1}$线性无关可得所有的$c_i=0$,矛盾!
\end{proof}
\section{特征多项式}
求特征值$\lambda$需要解如下的特征方程(eigenfunction)
\[
	\det(\lambda I-A)=\lambda^n+a_1\lambda^{n-1}+\cdots+a_n=0,
\]
特别地,特征多项式(eigen-polynomial)的系数有如下性质:
\begin{align}
	a_1=-\tr(A),\quad a_n=(-)^n\det(A)
\end{align}
由Vieta定理,在考虑重根的情况下,
\begin{align}
	\tr(A)=\sum_{i=1}^n\lambda_i,\quad \det(A)=\prod_{i=1}^n\lambda_i.
\end{align}
更一般地,\href{https://en.wikipedia.org/wiki/Cayley%E2%80%93Hamilton_theorem}{Cayley-Hamilton定理}给出了其他项的系数表达式,比如
\begin{subequations}
	\begin{align}
		a_2&=\frac12\bigkh{\tr(A)^2-\tr(A^2)},\\
		a_3&=\frac16\bigkh{\tr(A)^3-3\tr(A)\tr(A^2)+2\tr(A^3)},
	\end{align}
\end{subequations}
\begin{theorem}{相似变换的特征多项式}{}
	$A$的相似变换$B\iv AB$和$A$有相同的特征多项式。
\end{theorem}
\begin{proof}
	\[
		\lambda I-B\iv AB=B\iv(\lambda I-A)B.
	\]
	对两边取行列式即证。
\end{proof}
\begin{corollary}
	相似变换与原矩阵的迹相同
	\begin{equation}
		\tr(B\iv AB)=\tr(A).
	\end{equation}
\end{corollary}

\section{矩阵对角化}
\begin{theorem}{可对角化判定}{}
	$n$阶矩阵$A$可对角化$\iff A$有$n$个线性无关的特征向量$x_1,\ldots,x_n$,
	
	此时$A=X\Lambda X\iv$,且 
	\begin{align}
		X=(x_1,\ldots,x_n),\quad\Lambda=\diag(\lambda_1,\ldots,\lambda_n).
	\end{align}
\end{theorem}
\begin{proof}
	假设$A$有$n$个线性无关的特征向量$x_1,\ldots,x_n$,
	\[
		AX=(Ax_1,\ldots,Ax_n)=(\lambda_1x_1,\ldots,\lambda_nx_n)=X\Lambda,
	\]
	故$A$可对角化;反过来说,若$A$可对角化为$X\Lambda X\iv$,则$AX=X\Lambda$,即
	\[
		(Ax_1,\ldots,Ax_n)=(\lambda_1x_1,\ldots,\lambda_nx_n),
	\]
	故$x_1,\ldots,x_n$是$A$的特征向量,又$X$可逆,故$x_1,\ldots,x_n$线性无关。
\end{proof}
\begin{corollary}
	有$n$个互不相同特征值的$n$阶矩阵$A$可对角化。
\end{corollary}
当特征值重复时,引入两个概念
\begin{definition}{几何重数和代数重数}{}
	几何重数(geometric multiplicity, GM):特征值$\lambda$对应的最大线性无关的特征向量的个数,即$\dim\bigkh{\N(\lambda I-A)}$。

	代数重数(algebraic multiplicity, AM):特征值$\lambda$作为特征方程$\det(\lambda I-A)=0$的根的重复次数。
\end{definition}
特征方程可以写成
\[
	\prod_{i=1}^r(\lambda-\lambda_i)^{m_i}=0,
\]
其中$\lambda_i$是互不相同的根,$m_i$是$\lambda_i$的代数重数。
\begin{theorem}{}{GM <= AM}
	GM $\leqslant$ AM.
\end{theorem}
\begin{proof}
	考虑$n$阶矩阵$A$,假设特征值$\lambda_1$的GM $=\dim(\lambda_1I-A)=m$,
	
	取$\{x_1,\ldots,x_m\}$为$\C(\lambda_1I-A)$的一组正交归一基。
	
	取$\{b_1,\ldots,b_{n-m}\}$为$\N(\lambda_1I-A)^\perp$的一组正交归一基。
	
	设$n\times n$矩阵 
	\[
		P=(x_1,\ldots,x_m,b_1,\ldots,b_{n-m})=(X,B),
	\]
	$P$是可逆的且$P\iv=P\tp$,且$X\tp B=0$,则 
	\[
		P\iv AP=\begin{bmatrix}
			\lambda_1I_m&X\tp AB\\0&B\tp AB
		\end{bmatrix},
	\]
	分块三角矩阵
	\[
		\det(\lambda I-P\iv AP)=(\lambda-\lambda_1)^m\det(\lambda I-B\tp AB),
	\]
	$A$和$P\iv AP$有相同的特征方程,故$\lambda_1$必然是$A$的特征方程的根,且其AM $\geqslant$ GM。
\end{proof}
\begin{corollary}
	$n$阶矩阵$A$的全部特征值为$\{\lambda_1,\ldots,\lambda_r\}$,$A$可对角化当且仅当
	\[
		\sum_{i=1}^r\dim(\lambda_iI-A)=n,
	\]
	即所有特征值的AM = GM。
\end{corollary}
\begin{theorem}{同时对角化}{}
	若$A,B$可对角化,则他们可以同时对角化当且仅当$AB=BA$。
\end{theorem}
\begin{proof}
	若$A,B$可以同时对角化,故
	\[
		A=X\Lambda_AX\iv,\quad B=X\Lambda_BX\iv,
	\]
	故
	\[
		AB-BA=X(\Lambda_A\Lambda_B-\Lambda_B\Lambda_A)X\iv=0;
	\]
	
	若$AB=BA$,下证$A,B$可同时对角化。
	
	设$A$的特征值为$\{\lambda_1,\ldots,\lambda_s\}$,$\lambda_i$对应特征子空间为$V_i$,几何重数$m_i=\dim(V_i)$,%为方便,
	记$n_i:=m_1+\cdots+m_{i-1}$,
	取$\{v_{n_i+1},\ldots,v_{n_i+m_i}\}$表示$V_i$的一组基,记$X=(v_1,\ldots,v_n)$,则$X$可对角化$A$
	\[
		X\iv AX=\begin{bmatrix}
			\lambda_1I_{m_1}\\ &\ddots\\ &&\lambda_sI_{m_s}
		\end{bmatrix},
	\]
	$\forall x\in V_i$,
	\[
		(AB-BA)x=(A-\lambda_iI)Bx=0,
	\]
	故$Bx\in V_i$,从而$X\iv BX$和$X\iv AX$一样是分块对角的:
	\[
		X\iv BX=\begin{bmatrix}
			B_1\\ &\ddots\\ &&B_s
		\end{bmatrix},
	\]
	其中$B_i$是$m_i\times m_i$的。给定$B$的特征值$\xi_j$,其必是$B_1,\ldots,B_s$其中至少一个的特征值,不妨考虑$\xi_j$是$B_i$的特征值\footnote{无需考虑$\N(B_i)$},若$\xi_j$的AM $>B_i$的GM,则$\xi_j$的AM $>B$的GM,$B$便不能被对角化,与前提矛盾!故$B_i$均可被特定的$Y_i$对角化,即$Y_i\iv B_iY_i=\Lambda_i$,构造
	\[
		Y=\begin{bmatrix}
			Y_1\\ &\ddots\\ &&Y_s
		\end{bmatrix}
	\]
	则$Y\iv X\iv BXY$是对角化的,同时$Y\iv X\iv AXY$也是对角化的,取$Z=XY$,便可同时对角化$A,B$。
\end{proof}
\sectionstar{Jordan标准型}
不是所有方阵都可以对角化,如果$n$阶矩阵$A$有$r<n$个线性独立的特征向量,怎么把$A$变成最接近对角矩阵的形式?

\begin{theorem}{Jordan标准型}{Jordan normal form}
	$n$阶矩阵$A$有$r$个特征值,则存在$B$,使得
	\[
		B^{-1}AB=\begin{bmatrix}
			J_1             \\
			 & \ddots       \\
			 &        & J_r
		\end{bmatrix},\quad
		J_i=\begin{bmatrix}
			\lambda_i & 1                              \\
			          & \lambda_i & \ddots             \\
			          &           & \ddots & 1         \\
			          &           &        & \lambda_i
		\end{bmatrix}
	\]
	其中$J_i$称为Jordan块,$\lambda_i$是$A$的第$i$个特征值。
\end{theorem}
其证明是线性代数的核心。其中一些概念需要等到\chapref{chap:linear mapping}线性映射才会提及。先给出几个概念证明引理。
\begin{definition}{广义特征向量}{general eigenvector}
	线性映射$T:V\to V$的广义特征向量(general eigenvector) $v\in V$且$v\neq 0$,使得$(T-\lambda I)^kv=0$对某个正整数$k$成立。这里$I:V\to V$是恒等映射。

	使得$(T-\lambda I)^dv=0$成立的最小正整数$d$称为$v$的幂指数(exponent)。
\end{definition}
\begin{theorem}{}{}
	给定正整数$k$,广义特征方程$(T-\lambda I)^kv=0$有解当且仅当$\lambda$是$T$的特征值。
\end{theorem}
\begin{proof}
	若$\lambda$是$T$的特征值,则$(T-\lambda I)v=0$,左乘$(T-\lambda I)^{k-1}$即可。
	
	若$(T-\lambda I)^kv=0$有解,则$w=(T-\lambda I)^{k-1}v$满足$(T-\lambda I)w=0$,$\lambda$是$T$的特征值。
\end{proof}
\begin{theorem}{}{}
	令$u_i:=(T-\lambda I)^iv$,则$B=\{u_0,\ldots,u_{d-1}\}$是一组线性无关的向量。
\end{theorem}
\begin{proof}
	设
	\[
		\sum_{i=0}^{d-1}a_iu_i=\sum_{i=0}^{d-1}a_i(T-\lambda I)^iv=0,
	\]
	左乘$(T-\lambda I)^{d-1}$,左边只剩$a_0(T-\lambda I)^{d-1}v$,故$a_0=0$;
	
	递推地左乘$(T-\lambda I)^{d-2},(T-\lambda I)^{d-3},\ldots$可得到所有系数为0,从而$u_0,\ldots,u_{d-1}$线性无关。
\end{proof}
\begin{theorem}{}{}
	\[
		Tu_j=\begin{cases}
			\lambda u_j+u_{j+1},&1\leqslant j<d-1\\
			\lambda u_j,&j=d-1\\
			0,&j>d-1
		\end{cases}
	\]
\end{theorem}
\begin{proof}
	$1\leqslant j<d-1$时,$(T-\lambda I)u_j=u_{j+1}$,即$Tu_j=\lambda u_j+u_{j+1}$。
	
	$j=d-1$时,$(T-\lambda I)u_j=0$,即$Tu_j=\lambda u_j$;
	
	$j>d-1$时,$u_j=0$。
\end{proof}
\begin{theorem}{}{}
	$X=\spn(B)$是$T$的不变子空间,即$T(X)\subset X$
\end{theorem}
\begin{proof}
	由上式,$\forall u=a_0u_0+\cdots+a_{d-1}u_{d-1}\in X$
	\[
		Tu=\sum_{i=0}^{d-2}a_i(\lambda u_i+u_{i+1})+a_{d-1}\lambda u_{d-1}\in X.
	\]
	因为$X$是$T$的不变子空间,我们可以把$T$看成是$X\to X$的线性映射。
	取$B$作为$X$的一组基,则$T$在$B$下的表示矩阵为
	\[
		T=\begin{bmatrix}
			\lambda\\ 1&\lambda\\ &\ddots&\ddots\\ &&1&\lambda
		\end{bmatrix}
	\]
	这与Jordan块在\thmref{thm:Jordan normal form} 中的定义仅仅是转置的差别。
\end{proof}

接下来我们将证明$V$中存在一组基,$T$在这组基上的表示矩阵是分块对角的,而且每一块都是Jordan块的形式。
\begin{theorem}{}{}
	若$v_1,\ldots,v_r$是$T$的广义特征向量,且相应的幂指数是$d_i$,设 
	\[
		u_{ij}:=(T-\lambda_iI)^jv_i,\quad V_i:=\spn(u_{i0},\ldots,u_{id-1}).
	\]
	之前证明了$V_i$是$T$的不变子空间,且$T$在$V_i$上的表示矩阵是Jordan块。故
	$T$在$V_1\oplus\cdots\oplus V_r$上的表示矩阵是分块对角的,且每一块都是Jordan块的形式。
\end{theorem}
%平凡的。
所以我们只要证明存在这样一组广义特征向量$v_1,\ldots,v_r$使得$V=V_1\oplus\cdots\oplus V_r$就可以证明Jordan标准型的定理。

假设$\lambda$是$T$的某个特征值。如果$T-\lambda I$可以写成Jordan块的形式,则$T$也可以写成Jordan块的形式。所以以下我们用$T-\lambda I$代替$T$,或者说,考虑有一个特征值是0的线性映射$T$。
\begin{theorem}{}{}
	设$K_i=\ker(T^i),\,U_i=\Im(T^i)$,则
	\[
		K_1\subset K_2\subset\cdots,\quad U_1\supset U_2\supset\cdots
	\]
\end{theorem}
\begin{proof}
	待补
\end{proof} 

\section{对称矩阵}

\begin{definition}
	{对称矩阵}{symmetric matrix}
	若$S\tp=S$,则称$S$是对称矩阵(symmetric matrix);若$S\tp=-S$,则称$S$是反对称矩阵(anti-symmetric matrix)。
\end{definition}
\begin{theorem}{对称矩阵的性质·一}{characterist of symmetric matrix I}
	若$S$是一个$n$阶实对称矩阵,则$S$至少有一个实特征值$\lambda$
\end{theorem}
\begin{proof}
	由代数基本定理,对任何矩阵,$S$的特征方程至少会得到一个复特征值$\lambda$,其对应的特征向量为$z$ (一般也是复的),则$\bar z\tp z>0$。
	\[
		Sz=\lambda z,\quad S\bar z=\bar S\bar z=\overline{Sz}=\bar\lambda\bar z,
	\]
	由$S$的对称的性质,注意到
	\[
		\bar z\tp Sz=\lambda\bar z\tp z=\lambda(\bar z\tp z)\tp=\lambda z\tp\bar z=(Sz)\tp\bar z=z\tp S\bar z=\bar\lambda z\tp\bar z.
	\]
	故$\lambda=\bar\lambda$.
\end{proof}

\begin{corollary}
	由代数基本定理的递归性,可推知$S$的所有特征值都是实数。
\end{corollary}

\begin{theorem}{对称矩阵的性质·二}{characterist of symmetric matrix II}
	$v$是$S$的特征向量,若$w\perp v$,则$Sw\perp v$。
\end{theorem}
\begin{proof}
	
	\[
		(Sw)\tp v=w\tp S\tp v=w\tp Sv=\lambda w\tp v=0.
		\qedhere
	\]
\end{proof}
\begin{theorem}{对称矩阵的性质·三}{characterist of symmetric matrix III}
	若$W$是$\RR^n$的一个线性子空间,且在$S$的作用下稳定,即:
	\[
		\forall w\in W,\enspace Sw\in W,
	\]
	则$W^\perp$也在$S$的作用下稳定:
	\[
		\forall u\in W^\perp,\enspace Su\in W^\perp.
	\]
\end{theorem}
\begin{proof}
	$\forall w\in W,u\in W^\perp$
	\[
		(Su)\tp w=u\tp S\tp w=u\tp(Sw)=0.
		\qedhere
	\]
\end{proof}
\begin{theorem}{谱定理}{spectral theorem}
	对称矩阵$S$总可以被一个正交矩阵$Q$对角化。
\end{theorem}
\begin{proof}
	由\thmref{thm:characterist of symmetric matrix I} 和推论可知$S$至少有一个实特征值$\lambda_1$和实特征向量$q_1$且$q_1\tp q_1=1$,$S$在$q_1$张成的一维线性空间上是稳定的。
	
	由\thmref{thm:characterist of symmetric matrix III} 可知$S$作用在$\C(q_1)^\perp$上也是稳定的,假设$\C(q_1)^\perp$上有一组正交归一基为$\{a_1,\ldots,a_{n-1}\}$,构造矩阵$X_1=[q_1,a_1,\ldots,a_{n-1}]$,且$X_1$是正交的$X_1\tp X_1=I$,
	\[
		X_1\tp SX_1=X_1\tp[\lambda q_1,Sa_1,\ldots,Sa_{n-1}]=\begin{bmatrix}
			\lambda_1\\ &S_1
		\end{bmatrix}.
	\]
	$S_1$是一个$(n-1)$阶方阵,且$(S_1)_{ij}=a_i\tp Sa_j$的,显然它也是对称的。
	
	重复上述步骤,直到用$S$的特征向量构造出$\RR^n$的一组正交归一基:
	
	\noindent
	对$S_1$可构造$(n-1)$阶的正交矩阵$X_2$,使得
	\[
		X_2\tp S_1X_2=\begin{bmatrix}
			\lambda_2\\ &S_2
		\end{bmatrix},
	\]
	其中$S_2$是一个$(n-2)$阶对称方阵。从而
	\[
		\begin{bmatrix}
			1\\ &X_2\tp
		\end{bmatrix}X_1\tp SX_1\begin{bmatrix}
			1\\ &X_2
		\end{bmatrix}=\begin{bmatrix}
			\lambda_1\\ &\lambda_2\\ &&S_2
		\end{bmatrix}.
	\]
	$Q_2:=X_1\diag(1, X_2)$也是正交的……最终有
	\[
		Q_n\tp SQ_n=\begin{bmatrix}
			\lambda_1\\ &\ddots\\ &&\lambda_n
		\end{bmatrix}.
	\]
	$Q_n$也是正交的。
\end{proof}

对角化对称矩阵$S$的正交矩阵$Q$可被构造:
\begin{compactitem}
	\item 若$S$的特征值互不相同,对应的归一特征向量$q_i$两两正交,可选$Q=[q_1,\ldots,q_n]$;
	\item 若$S$的特征值有重复,取$\{q_i\}$为相应特征子空间的正交基。
\end{compactitem}
\section{正定矩阵}
\begin{definition}{二次型}{quadratic form}
	二次型(quadratic form)是形如$x\tp Sx$的二次多项式,其中$S$是实对称矩阵。
\end{definition}
\begin{definition}{正定矩阵}{positive definite matrix}
	给定对称矩阵$S$,如果$\forall x\neq 0$,二次型$x\tp Sx>0$,则称$S$是正定的(positive definite)。
\end{definition}
\begin{theorem}{正定矩阵的判定}{determination of positive definite matrix}
	对于对称矩阵$S$,下述命题是等价的:
	\begin{compactenum}
		\item $\forall x\neq 0$,二次型$x\tp Sx>0$;
		\item $S$的所有$n$个特征值都是正的;
		\item $S$可以只通过换行和倍加后得到$n$个正的主元;
		\item $S$的所有左上行列式(前$i$行$i$列子矩阵的行列式)均$>0$;
		\item 存在$A$列之间线性无关,使得$S=A\tp A$。
	\end{compactenum}
\end{theorem}
\begin{proof}
	$1\implies 2$:$S$对称,则$\Lambda=Q\tp SQ$
	\[
		\lambda_i=e_i\tp \Lambda e_i=e_i\tp Q\tp SQe_i=(Qe_i)\tp S(Qe_i)>0.
	\]
	$2\implies 1$:$S$的所有$n$个特征值都是正的,故
	\[
		x\tp Sx=x\tp Q\Lambda Q\tp x=\sum_{i=1}^n\lambda_i(Q\tp x)^2_i>0.
	\]
	$5\implies 1$:$A$列之间线性无关,故$\forall x\neq 0,\enspace Ax\neq 0$
	\[
		x\tp Sx=x\tp A\tp Ax=(Ax)\tp(Ax)>0.
	\]
	$1\implies 5$:$S$正定,故
	\[
		S=Q\Lambda Q\tp=Q\begin{bmatrix}
			\sqrt{\lambda_1}\\ &\ddots\\ &&\sqrt{\lambda_n}
		\end{bmatrix}\begin{bmatrix}
			\sqrt{\lambda_1}\\ &\ddots\\ &&\sqrt{\lambda_n}
		\end{bmatrix}Q\tp=:A\tp A.
	\]
	$3\implies 4$:行倍加不改变所有左上行列式,$S$做行倍加得到上三角矩阵$U$,其$i\times i$的左上行列式就是前$i$个主元的乘积,所以$>0$。\\
	$4\implies 3$:$U$的左上行列式都$>0$,所以前$i$个主元乘积都$>0$,所以主元全正。\\
	$3\implies 5$:$S=LDU$,由$S$对称且$LDU$分解唯一可知$L=U\tp$,又主元全正,故
	\[
		S=U\tp DU=U\tp\begin{bmatrix}
			\sqrt{a_1}\\ &\ddots\\ &&\sqrt{a_n}
		\end{bmatrix}\begin{bmatrix}
			\sqrt{a_1}\\ &\ddots\\ &&\sqrt{a_n}
		\end{bmatrix}U=:A\tp A.
	\]
	$5\implies 3$:$A$列之间线性无关,故$A=QR$
	\[
		A\tp A=R\tp Q\tp QR=R\tp R=LDU.
		\qedhere
	\]
\end{proof}
\begin{definition}{半正定矩阵}{positive semidefinite matrix}
	如果$\forall x\neq 0$,二次型$x\tp Sx\geqslant 0$,则称$S$是半正定的(positive semidefinite)。
\end{definition}
\begin{theorem}{半正定但非正定矩阵的判定}{determination of positive semi-but-not definite matrix}
	\begin{compactenum}
		\item $S$的最小特征值是0;
		\item 存在$A$列之间线性相关,使得$S=A\tp A$。
	\end{compactenum}
\end{theorem}
半正定但非正定矩阵的行列式为0.