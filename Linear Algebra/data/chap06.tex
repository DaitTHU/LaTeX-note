
\chapter{特征值和特征向量}
\section{特征值和特征向量}
\begin{definition}{特征值和特征向量}{eigenvalue and eigenvector}
	给定方阵$A$,若存在$x\neq 0$和$\lambda\in\FF$满足:
	\[
		Ax=\lambda x,
	\]
	则称$x$是$A$的特征向量(eigenvector),$\lambda$是对应的特征值(eigenvalue)。
	矩阵$A$所有特征值构成的集合称为$A$的谱(spectrum)。
\end{definition}

\begin{corollary}
	特征值的性质:
	\begin{compactenum}
		\item $A$的特征向量$x$也是$A^n$的特征向量,特征值是$\lambda^n;$
		\item 若$A$可逆,则$x$也是$A\iv$的特征向量,特征值是$\lambda^{-1};$
		\item 三角矩阵的特征值就是对角元;
		\item $A$可逆$\iff A$所有特征值非0。
		
		% 只需注意到$A$可逆时$Ax=0$只有零解即可。
	\end{compactenum}
\end{corollary}

\begin{theorem}{不同特征值对应特征向量线性无关}{eigenvalue}
	给定$A$的一组特征向量$x_1,\ldots,x_r$,对应特征值为$\lambda_1,\ldots,\lambda_r$。
	若$\lambda_1,\ldots,\lambda_r$两两不等,则$x_1,\ldots,x_r$线性无关。
\end{theorem}
\begin{proof}
	运用数学归纳法证明。$r=1$时,$x_1\neq 0$自然线性无关;
	
	假设$r=m-1$时,$x_1,\ldots,x_{m-1}$线性无关;当$r=m$时,考虑
	\begin{equation}
		\label{eqn:proof eigen 1}
		x_m=c_1x_1+\cdots+c_{m-1}x_{m-1},\tag{$\ast$}
	\end{equation}
	两边同时左乘$A$得
	\begin{equation}
		\label{eqn:proof eigen 2}
		\lambda_mx_m=c_1\lambda_1x_1+\cdots+c_{m-1}\lambda_{m-1}x_{m-1},\tag{$\ast\ast$}
	\end{equation}
	$\lambda_m$\eqref{eqn:proof eigen 1} $-$ \eqref{eqn:proof eigen 2}得,
	\[
		0=c_1(\lambda_m-\lambda_1)x_1+\cdots+c_{m-1}(\lambda_m-\lambda_{m-1})x_{m-1},
	\]
	由$x_1,\ldots,x_{m-1}$线性无关可得所有的$c_i=0$,故$x_1,\ldots,x_m$线性无关。

	综上,定理对所有可能的$r$均成立。
\end{proof}

\section{特征多项式}

\begin{definition}{特征子空间}{eigen-subspace}
	$A$的所有特征值为$\lambda$的特征向量张成$\RR^n$的一个线性子空间:$\N(A-\lambda I).$
\end{definition}

% \begin{corollary}
% 	如果$\lambda$是$A$的特征值,则$(A-\lambda I)$必然不可逆。
% \end{corollary}

\begin{definition}
	{特征方程和特征多项式}{eigenfunction and eigen-polynomial}
	求特征值$\lambda$需要解特征方程(eigenfunction):
	\[
		\det(\lambda I-A)=\lambda^n+a_1\lambda^{n-1}+\cdots+a_n=0,
	\]
	而$p_A(\lambda):=\det(\lambda I-A)$称为特征多项式(eigen-polynomial)。
\end{definition}

\begin{corollary}
	特别地,$p_A(0)=a_n=\det(-A)$,由Vieta定理:在考虑重根的情况下,
	\begin{equation}
		\label{eqn:det lambda}
		\det(A)=\prod_{i=1}^n\lambda_i.
	\end{equation}
	另一方面,通过对行列式进行Laplace展开,可得$a_1=-\tr(A)$,由Vieta定理:
	\begin{equation}
		\label{eqn:trace lambda}
		\tr(A)=\sum_{i=1}^n\lambda_i,
	\end{equation}
\end{corollary}

\subsectionstar{Cayley-Hamilton定理}

\begin{theorem}
	{Cayley-Hamilton定理}{Cayley-Hamilton theorem}
	给定特征多项式$p_A$的系数$1,a_1,\ldots,a_n$,可定义矩阵多项式
	\begin{equation}
		p_A^*(X)=X^n+a_1X^{n-1}+\cdots+a_{n-1}X+a_nI,
	\end{equation}
	则$p_A^*(A)=O$。
\end{theorem}

\noindent
\textit{完全错误的证明.}
将特征多项式$p_A(\lambda)=\det(\lambda I-A)$中的$\lambda$替换为$A$,自然得到$p_A(A)=\det(AI-A)=0$。

% \begin{proof}
% 	仅证明$A$的特征向量$x_1,\ldots,x_n$构成$\FF^n$的一组基的情形(即$A$可对角化),
% 	\begin{align*}
% 		p_A(A)x_i&=A^nx_i+c_{n-1}A^{n-1}x_i+\cdots+c_0x_i\\
% 		&=\lambda_i^2x_i+c_{n-1}\lambda_i^{n-1}x_i+\cdots+c_0x_i=p_A(\lambda)x_i=0,
% 	\end{align*}
% 	故$\forall x\in\FF^n$,都有$p_A(A)x=0$,取$x=e_1,\ldots,e_n$可得$p_A(A)=O$。
% \end{proof}

% \begin{remark}
% 	特征多项式$p_A:\FF\to\FF$与矩阵多项式$p_A^*$是完全不同的两个映射。
% \end{remark}

\begin{proof}
	考察伴随矩阵$\adj(\lambda I-A)$,可以写成如下形式:
	\[
		\adj(\lambda I-A)=B_1\lambda^{n-1}+\cdots+B_{n-1}\lambda+B_n,
	\]
	其中$B_1,\ldots,B_n$完全由$A$决定。
	由伴随矩阵的性质:
	\[
		(\lambda I-A)\adj(\lambda I-A)=\det(\lambda I-A)I=p_A(\lambda)I,
	\]
	两边展开可得
	\[
		B_1\lambda^n+(B_2-AB_1)\lambda^{n-1}+\cdots+(B_n-AB_{n-1})\lambda-AB_n=I\lambda^n+a_1I\lambda^{n-1}+\cdots+a_{n-1}\lambda+a_nI,
	\]
	作为系数的矩阵均由$A$确定,与$\lambda$无关。由于上式$\forall\lambda$均成立,故对应系数相同:
	\[
		B_1=I,\quad
		B_{i+1}-AB_i=a_iI,\quad
		-AB_n=a_nI.
	\]
	因而$p_A^*(A)$可以化为裂项和(telescoping sum)
	\[
		p_A^*(A)=A^nB_1+A^{n-1}(B_2-AB_1)+\cdots+A(B_n-AB_{n-1})-AB_n=O.
		\qedhere
	\]
\end{proof}

\begin{theorem}
	{Faddeev-LeVerrier算法}{Faddeev-LeVerrier algorithm}
	特征多项式$p_A(\lambda)$的第$k$个系数$a_k$可以递归地求出:
	\begin{equation}
		a_k=-\frac1k\bigfkh{\tr(A^k)+a_1\tr(A^{k-1})+\cdots+a_{k-1}\tr(A)}.
		% a_m=-\frac1m\sum_{k=1}^ma_{m-k}\tr(A^k).
	\end{equation}
\end{theorem}

\begin{proof}
	沿用\thmref{thm:Cayley-Hamilton theorem} 证明中的定义,
	% 并额外定义$B_0=O,\enspace c_0=1$。
	% 两边取迹得
	% \[
	% 	(n-i)a_i=\tr(AB_i)+na_i
	% \]
	由Jacobi公式\eqref{eqn:Jacobi det'},有 
	\[
		p_A'(\lambda)=\tr(\adj(\lambda I-A)I)=\tr(B_1\lambda^{n-1}+\cdots+B_n)
	\]
	展开
	\[
		n\lambda^{n-1}+(n-1)a_1\lambda^{n-2}+\cdots+a_{n-1}=\tr(B_1)\lambda^{n-1}+\tr(B_2)\lambda^{n-2}+\cdots+\tr(B_n).
	\]
	由于上式$\forall\lambda$均成立,故对应系数相同:
	\[
		(n-k)a_k=\tr(B_{k+1}),
	\]
	又$B_k$满足递推关系$B_{k+1}=AB_k+a_kI$,两边取迹可得
	\[
		(n-k)a_k=\tr(AB_k)+na_k,\implies a_k=-\frac1k\tr(AB_k),
	\]
	再展开$B_k$即证:
	\[
		B_k=AB_{k-1}+a_{k-1}I=\cdots=A^{k-1}+a_1A^{k-2}+\cdots+a_{k-1}I.
		\qedhere
	\]
\end{proof}

\begin{corollary}
	取$\lambda=0$可得伴随矩阵:
	\begin{equation}
		\adj(-A)=B_n=A^{n-1}+a_1A^{n-2}+\cdots+a_{n-1}I.
	\end{equation}
	易证这满足$\adj(-A)(-A)=(-A)\adj(-A)=\det(-A)I=c_0I$。
\end{corollary}

\begin{example}
	{特征多项式的前几项}{}
	\begin{subequations}
		\begin{align}
			a_2&=\frac12\bigkh{\tr(A)^2-\tr(A^2)},\\
			a_3&=\frac16\bigkh{\tr(A)^3-3\tr(A)\tr(A^2)+2\tr(A^3)},
		\end{align}
	\end{subequations}
	更一般地,$a_k$的显性表达式由一个$k$阶行列式给出:
	\begin{equation}
		a_k=\frac{(-)^k}{k!}\begin{vmatrix}
			\tr(A)&k-1\\
			\tr(A^2)&\tr(A)&k-2\\
			\vdots&\vdots&\ddots&\ddots\\
			\vdots&\vdots&\vdots&\ddots&1\\
			\tr(A^k)&\tr(A^{k-1})&\tr(A^{k-2})&\cdots&\tr(A)
		\end{vmatrix}.
	\end{equation}
\end{example}

\section{矩阵对角化}

由于对角矩阵有很多简单的性质,考虑相似变换$\Lambda=X\iv AX$,其中$\Lambda$为对角矩阵。

\begin{theorem}{相似变换与特征多项式}{}
	$A$的相似变换$B\iv AB$和$A$有相同的特征多项式。
\end{theorem}
\begin{proof} 
	对下式两边取行列式即证。
	\[
		\lambda I-B\iv AB=\lambda B\iv IB-B\iv AB=B\iv(\lambda I-A)B.
		\qedhere
	\]
\end{proof}

显然,不是所有方阵都可以对角化。

\begin{theorem}{可对角化判定}{}
	$n$阶矩阵$A$可对角化$\iff A$有$n$个线性无关的特征向量$x_1,\ldots,x_n$。
	\tcblower
	此时$A=X\Lambda X\iv$,$X$由特征向量给出,对角矩阵$\Lambda$由对应特征值$\lambda_1,\ldots,\lambda_n$ (可能相同)给出:
	\begin{equation}
		X=[x_1,\ldots,x_n]
		\quad
	\Lambda=\diag(\lambda_1,\ldots,\lambda_n).
	\end{equation}
\end{theorem}
\begin{proof}
	假设$A$有$n$个线性无关的特征向量$x_1,\ldots,x_n$,
	\[
		AX=[Ax_1,\ldots,Ax_n]=[\lambda_1x_1,\ldots,\lambda_nx_n]=X\Lambda,
	\]
	故$A$可对角化;反过来说,若$A$可对角化为$X\Lambda X\iv$,则$AX=X\Lambda$,即
	\[
		[Ax_1,\ldots,Ax_n]=[\lambda_1x_1,\ldots,\lambda_nx_n],
	\]
	故$x_1,\ldots,x_n$是$A$的特征向量,又$X$可逆,故$x_1,\ldots,x_n$线性无关。
\end{proof}

\begin{corollary}
	有$n$个互不相同特征值的$n$阶矩阵$A$可对角化。
\end{corollary}

\paragraph{特征值的重数}

当特征值重复时,引入两个概念

\newcommand*{\GM}{\mathrm{GM}}
\newcommand*{\AM}{\mathrm{AM}}

\begin{definition}{几何重数和代数重数}{}
	几何重数(geometric multiplicity, GM):特征值$\lambda_i$对应的线性无关特征向量的最大个数,即特征子空间的维数$\dim\N(\lambda_iI-A)$。

	代数重数(algebraic multiplicity, AM):特征值$\lambda_i$作为特征方程$\det(\lambda I-A)=0$的根$\lambda=\lambda_i$的重复次数。
	\[
		\det(\lambda I-A)=\prod_{i=1}^r(\lambda-\lambda_i)^{m_i}=0,
	\]
	其中$\lambda_i$是互不相同的根,$m_i$是$\lambda_i$的代数重数。
\end{definition}

\begin{corollary}
	代数重数的和$\textstyle\sum_{i=1}^rm_i=n$矩阵的阶数。
\end{corollary}

\begin{theorem}{}{GM <= AM}
	$\GM\leqslant\AM.$
\end{theorem}
\begin{proof}
	考虑$n$阶矩阵$A$,假设特征值$\lambda_1$的GM $=\dim\N(\lambda_1I-A)=m$,
	\begin{itemize}
		\item 取$\N(\lambda_1I-A)$的一组正交归一基$\{x_1,\ldots,x_m\}$;
		\item 取$\N(\lambda_1I-A)^\perp=\C(\lambda_1I-A)\tp$的一组正交归一基$\{b_1,\ldots,b_{n-m}\}$。
	\end{itemize}
	则$\{x_1,\ldots,x_m,b_1,\ldots,b_{n-m}\}$构成$\RR^n$的一组正交归一基,可以此定义$n$阶正交矩阵$P$ 
	\[
		P=[x_1,\ldots,x_m,b_1,\ldots,b_{n-m}]=:[X,B],
	\]
	因为$x_i\tp b_j=0$,有$X\tp B=0$,则$P\iv AP$是分块三角的:
	\[
		P\iv AP=\begin{bmatrix}
			X\tp\\B\tp
		\end{bmatrix}A\begin{bmatrix}
			X&B
		\end{bmatrix}=\begin{bmatrix}
			\lambda_1I_m&X\tp AB\\0&B\tp AB
		\end{bmatrix},
	\]
	其特征方程为
	\[
		\det(\lambda I-P\iv AP)=(\lambda-\lambda_1)^m\det(\lambda I-B\tp AB),
	\]
	显然,其$(\lambda-\lambda_1)$的次数$\geq m$。
	而$A$和$P\iv AP$有相同的特征方程,故$\lambda_1$的$\AM\geqslant\GM$。
\end{proof}

\begin{corollary}
	$n$阶矩阵$A$的全部特征值为$\{\lambda_1,\ldots,\lambda_r\}$,$A$可对角化当且仅当
	\[
		\sum_{i=1}^r\dim\N(\lambda_iI-A)=n,
	\]
	即所有特征值的$\AM_i=\GM_i$。
\end{corollary}

\begin{theorem}
	{分块对角矩阵的对角化}{}
	分块对角矩阵可对角化$\iff$所有块矩阵均可对角化。
\end{theorem}

\begin{proof}
	考虑只有2个分块对角矩阵$A=\diag(A_1,A_2)$,因为对于$k$块对角矩阵,可令$A_2$是$k-1$块的,通过数学归纳法证明即可。

	一方面,若$A_1,A_2$可对角化,即
	\[
		X_1\iv A_1X_1=\Lambda_1,\quad X_2\iv A_2X_2=\Lambda_2,
	\]
	则令$X=\diag(X_1,X_2),\,\Lambda=\diag(\Lambda_1,\Lambda_2)$可将$A$对角化:
	\[
		X\iv AX=\Lambda;
	\]
	另一方面,若$A$可对角化,任取$A$的一个特征值$\lambda_i$,
	考虑其关于$A,A_1,A_2$的代数重数$\AM,\AM_1,\AM_2$和几何重数$\GM,\GM_1,\GM_2$,由
	\[
		\det(\lambda I_n-A)=\det(\lambda I_{n_1}-A_1)\det(\lambda I_{n_2}-A_2)\implies\AM=\AM_1+\AM_2;
	\]
	又
	\[
		\rank(\lambda_iI_n-A)=\rank(\lambda_iI_{n_1}-A_1)+\rank(\lambda_iI_{n_2}-A_2)\implies\GM=\GM_1+\GM_2.
	\]
	由$A$可对角化知$\AM=\GM$,而$\AM_1\geq\GM_1,\,\AM_2\geq\GM_2$,
	故$\AM_1=\GM_1,\,\AM_2=\GM_2$,即$A_1,A_2$均可对角化。
\end{proof}

\begin{theorem}{同时对角化}{}
	若$A,B$可对角化,则$A,B$可以同时对角化$\iff A,B$可交换($AB=BA$)。
\end{theorem}

\begin{proof}
	若$A,B$可以同时对角化,即存在可逆矩阵$X$和对角矩阵$\Lambda_A,\Lambda_B$,使得
	\[
		A=X\Lambda_AX\iv,\quad B=X\Lambda_BX\iv,
	\]
	故
	\[
		AB-BA=X(\Lambda_A\Lambda_B-\Lambda_B\Lambda_A)X\iv=0;
	\]
	另一方面,若$AB=BA$,下证$A,B$可同时对角化。
	由$A$可对角化,可知存在$X$使得
	\[
		X\iv AX=\begin{bmatrix}
			\lambda_1I_{n_1}\\ &\ddots\\ &&\lambda_rI_{n_r}
		\end{bmatrix},
	\]
	又由$AB=BA$可得
	\[
		(X\iv AX)(X\iv BX)=(X\iv BX)(X\iv AX),
	\]
	即$X\iv BX$应是分块对角的:
	\[
		X\iv BX=\begin{bmatrix}
			B_1\\ &\ddots\\ &&B_r
		\end{bmatrix}.
	\]
	由$B$可对角化知$B_i$可对角化,$Y_i\iv B_iY_i=\Lambda_i$。
	令$Y=\diag(Y_1,\ldots,Y_r)$,$\Lambda_B=\diag(\Lambda_1,\ldots,\Lambda_r)$,则$Y\iv BY=\Lambda_B$。
	由此$A,B$可以被$XY$同时对角化。
\end{proof}

\sectionstar{Jordan标准型}

不是所有方阵都可以对角化,如果$n$阶矩阵$A$有$r<n$个线性独立的特征向量,怎么把$A$变成最接近对角矩阵的形式?
其实,任何矩阵都可以通过相似变换变成一种特殊的分块对角矩阵,并且其与对角矩阵只相差了一个幂零矩阵。

\begin{theorem}{Jordan标准型}{Jordan normal form}
	$n$阶矩阵$A$有$r$个特征值,则存在$B$,使得
	\[
		B^{-1}AB=\begin{bmatrix}
			J_1             \\
			 & \ddots       \\
			 &        & J_r
		\end{bmatrix},\quad
		J_i=\begin{bmatrix}
			\lambda_i & 1                              \\
			          & \lambda_i & \ddots             \\
			          &           & \ddots & 1         \\
			          &           &        & \lambda_i
		\end{bmatrix}
	\]
	其中$J_i$称为Jordan块,$\lambda_i$是$A$的第$i$个特征值。
\end{theorem}
其证明是线性代数的核心。其中一些概念需要等到\chapref{chap:linear mapping}线性映射才会提及。先给出几个概念证明引理。
\begin{definition}{广义特征向量}{general eigenvector}
	线性映射$T:V\to V$的广义特征向量(general eigenvector) $v\in V$且$v\neq 0$,使得$(T-\lambda I)^kv=0$对某个正整数$k$成立。这里$I:V\to V$是恒等映射。

	使得$(T-\lambda I)^dv=0$成立的最小正整数$d$称为$v$的幂指数(exponent)。
\end{definition}
\begin{theorem}{}{}
	给定正整数$k$,广义特征方程$(T-\lambda I)^kv=0$有解当且仅当$\lambda$是$T$的特征值。
\end{theorem}
\begin{proof}
	若$\lambda$是$T$的特征值,则$(T-\lambda I)v=0$,左乘$(T-\lambda I)^{k-1}$即可。
	
	若$(T-\lambda I)^kv=0$有解,则$w=(T-\lambda I)^{k-1}v$满足$(T-\lambda I)w=0$,$\lambda$是$T$的特征值。
\end{proof}
\begin{theorem}{}{}
	令$u_i:=(T-\lambda I)^iv$,则$B=\{u_0,\ldots,u_{d-1}\}$是一组线性无关的向量。
\end{theorem}
\begin{proof}
	设
	\[
		\sum_{i=0}^{d-1}a_iu_i=\sum_{i=0}^{d-1}a_i(T-\lambda I)^iv=0,
	\]
	左乘$(T-\lambda I)^{d-1}$,左边只剩$a_0(T-\lambda I)^{d-1}v$,故$a_0=0$;
	
	递推地左乘$(T-\lambda I)^{d-2},(T-\lambda I)^{d-3},\ldots$可得到所有系数为0,从而$u_0,\ldots,u_{d-1}$线性无关。
\end{proof}
\begin{theorem}{}{}
	\[
		Tu_j=\begin{cases}
			\lambda u_j+u_{j+1},&1\leqslant j<d-1\\
			\lambda u_j,&j=d-1\\
			0,&j>d-1
		\end{cases}
	\]
\end{theorem}
\begin{proof}
	$1\leqslant j<d-1$时,$(T-\lambda I)u_j=u_{j+1}$,即$Tu_j=\lambda u_j+u_{j+1}$。
	
	$j=d-1$时,$(T-\lambda I)u_j=0$,即$Tu_j=\lambda u_j$;
	
	$j>d-1$时,$u_j=0$。
\end{proof}
\begin{theorem}{}{}
	$X=\lspan(B)$是$T$的不变子空间,即$T(X)\subset X$
\end{theorem}
\begin{proof}
	由上式,$\forall u=a_0u_0+\cdots+a_{d-1}u_{d-1}\in X$
	\[
		Tu=\sum_{i=0}^{d-2}a_i(\lambda u_i+u_{i+1})+a_{d-1}\lambda u_{d-1}\in X.
	\]
	因为$X$是$T$的不变子空间,我们可以把$T$看成是$X\to X$的线性映射。
	取$B$作为$X$的一组基,则$T$在$B$下的表示矩阵为
	\[
		T=\begin{bmatrix}
			\lambda\\ 1&\lambda\\ &\ddots&\ddots\\ &&1&\lambda
		\end{bmatrix}
	\]
	这与Jordan块在\thmref{thm:Jordan normal form} 中的定义仅仅是转置的差别。
\end{proof}

接下来我们将证明$V$中存在一组基,$T$在这组基上的表示矩阵是分块对角的,而且每一块都是Jordan块的形式。
\begin{theorem}{}{}
	若$v_1,\ldots,v_r$是$T$的广义特征向量,且相应的幂指数是$d_i$,设 
	\[
		u_{ij}:=(T-\lambda_iI)^jv_i,\quad V_i:=\lspan(u_{i0},\ldots,u_{id-1}).
	\]
	之前证明了$V_i$是$T$的不变子空间,且$T$在$V_i$上的表示矩阵是Jordan块。故
	$T$在$V_1\oplus\cdots\oplus V_r$上的表示矩阵是分块对角的,且每一块都是Jordan块的形式。
\end{theorem}
%平凡的。
所以我们只要证明存在这样一组广义特征向量$v_1,\ldots,v_r$使得$V=V_1\oplus\cdots\oplus V_r$就可以证明Jordan标准型的定理。

假设$\lambda$是$T$的某个特征值。如果$T-\lambda I$可以写成Jordan块的形式,则$T$也可以写成Jordan块的形式。所以以下我们用$T-\lambda I$代替$T$,或者说,考虑有一个特征值是0的线性映射$T$。
\begin{theorem}{}{}
	设$K_i=\ker(T^i),\,U_i=\Im(T^i)$,则
	\[
		K_1\subset K_2\subset\cdots,\quad U_1\supset U_2\supset\cdots
	\]
\end{theorem}
\begin{proof}
	待补
\end{proof} 

\section{对称矩阵}

\begin{definition}
	{对称矩阵}{symmetric matrix}
	若$S\tp=S$,则称$S$是对称矩阵(symmetric matrix);
	
	若$A\tp=-A$,则称$A$是反对称矩阵(skew-symmetric/anti-symmetric)。
\end{definition}
\begin{theorem}{对称矩阵的性质·一}{characterist of symmetric matrix I}
	若$S$是一个$n$阶实对称矩阵,则$S$至少有一个实特征值$\lambda$
\end{theorem}
\begin{proof}
	由代数基本定理,对任何矩阵,$S$的特征方程至少会得到一个复特征值$\lambda$,其对应的特征向量为$z$ (一般也是复的),则$\bar z\tp z>0$。
	\[
		Sz=\lambda z,\quad S\bar z=\bar S\bar z=\overline{Sz}=\bar\lambda\bar z,
	\]
	由$S$的对称的性质,注意到
	\[
		\bar z\tp Sz=\lambda\bar z\tp z=\lambda(\bar z\tp z)\tp=\lambda z\tp\bar z=(Sz)\tp\bar z=z\tp S\bar z=\bar\lambda z\tp\bar z.
	\]
	故$\lambda=\bar\lambda$.
\end{proof}

\begin{corollary}
	由代数基本定理的递归性,可推知$S$的所有特征值都是实数。
\end{corollary}

\begin{theorem}{对称矩阵的性质·二}{characterist of symmetric matrix II}
	$v$是$S$的特征向量,若$w\perp v$,则$Sw\perp v$。
\end{theorem}
\begin{proof}
	
	\[
		(Sw)\tp v=w\tp S\tp v=w\tp Sv=\lambda w\tp v=0.
		\qedhere
	\]
\end{proof}
\begin{theorem}{对称矩阵的性质·三}{characterist of symmetric matrix III}
	若$W$是$\RR^n$的一个线性子空间,且在$S$的作用下稳定,即:
	\[
		\forall w\in W,\enspace Sw\in W,
	\]
	则$W^\perp$也在$S$的作用下稳定:
	\[
		\forall u\in W^\perp,\enspace Su\in W^\perp.
	\]
\end{theorem}
\begin{proof}
	$\forall w\in W,u\in W^\perp$
	\[
		(Su)\tp w=u\tp S\tp w=u\tp(Sw)=0.
		\qedhere
	\]
\end{proof}
\begin{theorem}{谱定理}{spectral theorem}
	对称矩阵$S$总可以被一个正交矩阵$Q$对角化。
\end{theorem}
\begin{proof}
	由\thmref{thm:characterist of symmetric matrix I} 和推论可知$S$至少有一个实特征值$\lambda_1$和实特征向量$q_1$且$q_1\tp q_1=1$,$S$在$q_1$张成的一维线性空间上是稳定的。
	
	由\thmref{thm:characterist of symmetric matrix III} 可知$S$作用在$\C(q_1)^\perp$上也是稳定的,假设$\C(q_1)^\perp$上有一组正交归一基为$\{a_1,\ldots,a_{n-1}\}$,构造矩阵$X_1=[q_1,a_1,\ldots,a_{n-1}]$,且$X_1$是正交的$X_1\tp X_1=I$,
	\[
		X_1\tp SX_1=X_1\tp[\lambda q_1,Sa_1,\ldots,Sa_{n-1}]=\begin{bmatrix}
			\lambda_1\\ &S_1
		\end{bmatrix}.
	\]
	$S_1$是一个$(n-1)$阶方阵,且$(S_1)_{ij}=a_i\tp Sa_j$的,显然它也是对称的。
	
	重复上述步骤,直到用$S$的特征向量构造出$\RR^n$的一组正交归一基:
	
	\noindent
	对$S_1$可构造$(n-1)$阶的正交矩阵$X_2$,使得
	\[
		X_2\tp S_1X_2=\begin{bmatrix}
			\lambda_2\\ &S_2
		\end{bmatrix},
	\]
	其中$S_2$是一个$(n-2)$阶对称方阵。从而
	\[
		\begin{bmatrix}
			1\\ &X_2\tp
		\end{bmatrix}X_1\tp SX_1\begin{bmatrix}
			1\\ &X_2
		\end{bmatrix}=\begin{bmatrix}
			\lambda_1\\ &\lambda_2\\ &&S_2
		\end{bmatrix}.
	\]
	$Q_2:=X_1\diag(1, X_2)$也是正交的……最终有
	\[
		Q_n\tp SQ_n=\begin{bmatrix}
			\lambda_1\\ &\ddots\\ &&\lambda_n
		\end{bmatrix}.
	\]
	$Q_n$也是正交的。
\end{proof}

\begin{remark}
	对角化对称矩阵$S$的正交矩阵$Q$可被构造:
	\begin{compactitem}
		\item 若$S$的特征值互不相同,对应的归一特征向量$q_i$两两正交,可选$Q=[q_1,\ldots,q_n]$;
		\item 若$S$的特征值有重复,取$\{q_i\}$为相应特征子空间的正交基。
	\end{compactitem}
\end{remark}

\section{正定矩阵}

\begin{definition}{二次型}{quadratic form}
	二次型(quadratic form)是形如$x\tp Sx$的二次多项式,其中$S$是实对称矩阵。
\end{definition}
\begin{definition}{正定矩阵}{positive definite matrix}
	给定对称矩阵$S$,如果$\forall x\neq 0$,二次型$x\tp Sx>0$,则称$S$是正定的(positive definite)。
\end{definition}
\begin{theorem}{正定矩阵的判定}{determination of positive definite matrix}
	对于对称矩阵$S$,下述命题是等价的:
	\begin{compactenum}
		\item $\forall x\neq 0$,二次型$x\tp Sx>0$;
		\item $S$的所有$n$个特征值都是正的;
		\item $S$可以只通过换行和倍加后得到$n$个正的主元;
		\item $S$的所有左上行列式(前$i$行$i$列子矩阵的行列式)均$>0$;
		\item 存在$A$列之间线性无关,使得$S=A\tp A$。
	\end{compactenum}
\end{theorem}
\begin{proof}
	$1\implies 2$:$S$对称,则$\Lambda=Q\tp SQ$
	\[
		\lambda_i=e_i\tp \Lambda e_i=e_i\tp Q\tp SQe_i=(Qe_i)\tp S(Qe_i)>0.
	\]
	$2\implies 1$:$S$的所有$n$个特征值都是正的,故
	\[
		x\tp Sx=x\tp Q\Lambda Q\tp x=\sum_{i=1}^n\lambda_i(Q\tp x)^2_i>0.
	\]
	$5\implies 1$:$A$列之间线性无关,故$\forall x\neq 0,\enspace Ax\neq 0$
	\[
		x\tp Sx=x\tp A\tp Ax=(Ax)\tp(Ax)>0.
	\]
	$1\implies 5$:$S$正定,故
	\[
		S=Q\Lambda Q\tp=Q\begin{bmatrix}
			\sqrt{\lambda_1}\\ &\ddots\\ &&\sqrt{\lambda_n}
		\end{bmatrix}\begin{bmatrix}
			\sqrt{\lambda_1}\\ &\ddots\\ &&\sqrt{\lambda_n}
		\end{bmatrix}Q\tp=:A\tp A.
	\]
	$3\implies 4$:行倍加不改变所有左上行列式,$S$做行倍加得到上三角矩阵$U$,其$i\times i$的左上行列式就是前$i$个主元的乘积,所以$>0$。\\
	$4\implies 3$:$U$的左上行列式都$>0$,所以前$i$个主元乘积都$>0$,所以主元全正。\\
	$3\implies 5$:$S=LDU$,由$S$对称且$LDU$分解唯一可知$L=U\tp$,又主元全正,故
	\[
		S=U\tp DU=U\tp\begin{bmatrix}
			\sqrt{a_1}\\ &\ddots\\ &&\sqrt{a_n}
		\end{bmatrix}\begin{bmatrix}
			\sqrt{a_1}\\ &\ddots\\ &&\sqrt{a_n}
		\end{bmatrix}U=:A\tp A.
	\]
	$5\implies 3$:$A$列之间线性无关,故$A=QR$
	\[
		A\tp A=R\tp Q\tp QR=R\tp R=LDU.
		\qedhere
	\]
\end{proof}
\begin{definition}{半正定矩阵}{positive semi-definite matrix}
	如果$\forall x\neq 0$,二次型$x\tp Sx\geqslant 0$,则称$S$是半正定的(positive semi-definite)。
\end{definition}
\begin{theorem}{半正定但非正定矩阵的判定}{determination of positive semi-but-not definite matrix}
	\begin{compactenum}
		\item $S$的最小特征值是0;
		\item 存在$A$列之间线性相关,使得$S=A\tp A$。
	\end{compactenum}
\end{theorem}
\begin{corollary}
	半正定但非正定矩阵的行列式为0。
\end{corollary}

\begin{theorem}
	{Cholesky分解}{Cholesky decomposition}
	正定矩阵$A$可以分解为
	\begin{equation}
		A=LL\tp,
	\end{equation}
	其中$L$是对角元为正的下三角矩阵,这称为Cholesky分解。
\end{theorem}


