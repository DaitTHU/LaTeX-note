\chapter{线性映射}
\label{chap:linear mapping}

\begin{definition}{映射}{mapping}
	给定两个集合$S,S'$,如果$\forall x\in S$,均有一个对应的$f(x)\in S'$,这种对应关系$f$便称为映射(mapping),记作
	\begin{equation}
		f:S\to S',\enspace x\mapsto f(x).
	\end{equation}
	其中$S$称为定义域(domain),$S'$称为陪域(codomain)。$f(x)$称为$x$的像(image),所有像的集合$f(S)=\nset{f(x)}{x\in S}$称为值域(image of $S$),有$f(S)\subset S'$。
	\tcblower
	给定$y\in S'$,所有满足$f(x)=y$的$x$的集合称为$y$的原像(preimage),记作
	\begin{equation}
		f\iv(y):=\nset{x\in S}{f(x)=y}.
	\end{equation}
\end{definition}

\begin{remark}
	符号$f(\cdot)$便代表映射$x\mapsto f(x)$。
\end{remark}

\begin{definition}{映射的复合}{mapping composition}
	映射 $f:U\to V,\enspace g:V\to W$的复合(composition)构成一个新的映射:
	\[
		g\circ f:U\to W,\enspace x\mapsto g(f(x)).
	\]
\end{definition}
\begin{corollary}
	映射的复合满足结合律
	\[
		(h\circ g)\circ f=h\circ (g\circ f)\equiv h\circ g\circ f.
	\]
\end{corollary}
\begin{definition}{单射、满射和双射}{injection, surjection and bijection}
	\begin{itemize}
		\item 单射(injection):$\forall x,y\in S$且$x\neq y$,均有$f(x)\neq f(y)$。
		\item 满射(surjection):$f(S)=S'$。
		\item 双射(bijection):既是单射又是满射。
	\end{itemize}
\end{definition}

\begin{definition}
	{恒等映射}{identity mapping}
	定义$S$上的恒等映射(identity)为
	\begin{equation}
		\id_S:S\to S,\enspace x\mapsto x.
	\end{equation}
\end{definition}

\begin{corollary}
	$\forall f:S\to S'$,有$f\circ\id_S=\id_{S'}\circ f=f$。
\end{corollary}

\begin{definition}
	{逆映射}{inverse mapping}
	给定映射$f:S\to S'$,若存在$g:S'\to S$使得
	\[
		g\circ f=\id_S,\quad f\circ g=\id_{S'},
	\]
	则称映射$f:S\to S'$可逆,$g=f\iv$为$f$的逆映射(inverse)。
\end{definition}
\begin{theorem}{有关映射的等价描述}{}
	\begin{enumerate}
		\item 映射$f$为单射$\iff$存在映射$g$使得$g\circ f=\id$;
		\item 映射$f$为满射$\iff$存在映射$g$使得$f\circ g=\id$;
		\item 映射$f$为双射$\iff f$可逆。
	\end{enumerate}
\end{theorem}
\begin{proof}
	显然$(1),(2)\implies(3)$,故只需证明$(1),(2)$。
	证明留作习题。
	\qedhere
	% 若$f$可逆,$g:S'\to S$为$f$的逆。若$x,y\in S$,满足$f(x)=f(y)$,则 
	% \[
	% 	x=g\bigkh{f(x)}=g\bigkh{f(y)}=y.
	% \]
	% 故$f$为单射;又$\forall z\in S'$,取$x=g(z)\in S$,可得$f(x)=f\bigkh{g(z)}=z$,故$f$为满射。
	
	% 若$f$为双射,因为$f$为满射,$\forall z\in S'$,有$x\in S$使得$f(x)=z$,又因$f$为单射,因此$x$是唯一的,我们可以定义$g(z)=x$,故$g$是$f$的逆映射。
\end{proof}
\section{线性映射和矩阵}
\begin{definition}{线性映射}{linear mapping}
	给定两个线性空间$V,W$,若映射$T:V\to W$满足:
	\begin{compactenum}
		\item $\forall u,v\in V,\enspace T(u+v)=T(u)+T(v);$
		\item $\forall c\in\FF,\enspace T(cu)=cT(u).$
	\end{compactenum}
	则称映射$T$是线性映射(或线性变换,linear mapping)。特别地,$V\to\FF$的称为线性函数。
\end{definition}
\begin{corollary}
	$T(0)=0.$
\end{corollary}
\begin{theorem}{线性映射与基}{linear mapping and base}
	给定两个线性空间$V,W$,$\{v_1,\ldots,v_n\}$是$V$中的一组基,$\{w_1,\ldots,w_n\}$是$W$中任意$n$个元素,则存在唯一的线性映射$T:V\to W$使得
	\[
		T(v_1)=w_1,\enspace\ldots,\enspace T(v_n)=w_n.
	\]
\end{theorem}
\begin{proof}
	(存在性) $\forall v\in V$均可唯一写成基的线性组合$v=c_1v_1+\cdots+c_nv_n$,定义映射$T:V\to W$
	\[
		T(v)=c_1w_1+\cdots+c_nw_n,
	\]
	下面证明$T$是线性映射,再任取$u=d_1v_1+\cdots+d_nv_n\in V$
	\begin{align*}
		T(v+u)&=T\bigkh{(c_1+d_1)v_1+\cdots+(c_n+d_n)v_n}\\
		&=(c_1+d_1)w_1+\cdots+(c_n+d_n)w_n=T(v)+T(u);\\
		T(cv)&=T(cc_1v_1+\cdots+cc_nv_n)=cc_1w_1+\cdots+cc_nw_n=cT(v).
	\end{align*}
	(唯一性) 假设存在另一个线性映射$F:V\to W$满足
	\[
		F(v_1)=w_1,\enspace\ldots,\enspace F(v_n)=w_n,
	\]
	则 
	\[
		F(v)=c_1F(v_1)+\cdots+c_nF(v_n)=c_1w_1+\cdots+c_nw_n=T(v).
	\]
	综上,存在唯一的线性映射。
\end{proof}

\begin{remark}
	只要知道线性映射在基上的值,就唯一决定了这个线性映射。
\end{remark}

\begin{example}{矩阵定义线性映射}{linear mapping defined by matrix}
	$m\times n$的矩阵$A$可定义一个线性映射:
	\begin{equation}
		L_A:\FF^n\to\FF^m,\enspace x\mapsto Ax.
	\end{equation}
\end{example}
\begin{theorem}{线性映射和矩阵}{linear mapping and matrix}
	设$L:\FF^n\to\FF^m$是线性映射,则存在唯一的矩阵$A$使得$L=L_A$。
\end{theorem}
\begin{proof}
	设$\{e_1,\ldots,e_n\}$是$\FF^n$的标准基,$\{f_1,\ldots,f_m\}$是$\FF^m$的标准基,$\forall x\in\FF^n$,有$x=x_1e_1+\cdots+x_ne_n$,则
	\[
		L(x)=x_1L(e_1)+\cdots+x_nL(e_n).
	\]
	$L(e_i)\in\FF^m$,故可写成基的线性组合
	\[
		L(e_i)=a_{1i}f_1+\cdots+a_{mi}f_m.
	\]
	故
	\begin{align*}
		L(x)&=x_1(a_{11}f_1+\cdots+a_{m1}f_m)+\cdots+x_n(a_{n1}f_1+\cdots+a_{nm}f_m)\\
		&=(a_{11}x_1+\cdots+a_{1n}x_n)f_1+\cdots+(a_{m1}x_1+\cdots+a_{mn}x_n)f_m\\
		&=\begin{bmatrix}
			a_{11}x_1+\cdots+a_{1n}x_n\\\vdots\\
			a_{m1}x_1+\cdots+a_{mn}x_n
		\end{bmatrix}=\begin{bmatrix}
			a_{11}&\cdots&a_{1n}\\
			\vdots&\ddots&\vdots\\
			a_{m1}&\cdots&a_{mn}
		\end{bmatrix}\begin{bmatrix}
			x_1\\\vdots\\x_n
		\end{bmatrix}=:Ax.
	\end{align*}
	便唯一确定了一个矩阵$A$。
\end{proof}

\begin{remark}
	线性映射给出了矩阵和向量乘法的自然定义。
\end{remark}

\section{线性映射的性质}
利用线性映射和矩阵的对应,线性映射的加法和数乘等价于矩阵的加法和数乘,零映射对应零矩阵,这些都是平凡的。

给出加法、数乘、零映射的定义后,所有$V\to W$的线性映射的集合$\{T\}$便构成一个线性空间,可验证满足8条公理。
\begin{definition}{线性映射的核}{kernel}
	线性映射$F:V\to W$的核(kernel)是所有满足$F(v)=0$的向量$v$的集合
	\[
		\ker(F)\equiv\set{v\in V}{F(v)=0}.
	\]
\end{definition}
\begin{corollary}
	$\ker(F)$是$V$的线性子空间。$\ker (L_A)=\N(A)$。
\end{corollary}
\begin{theorem}{核和单射}{kernel and injection}
	\begin{center}
		$\ker(F)=\{0\}\iff F$是单射。
	\end{center}
\end{theorem}
\begin{proof}
	(矩阵版本)对应矩阵零空间为$\{0\}$,$Av=b$若有解则解必唯一。
	
	(抽象版本)若$u,v\in V$满足$F(u)=F(v)$,则$F(u-v)=F(u)-F(v)=0$,从而$u-v=0.$
\end{proof}
\begin{theorem}{核的性质}{characterist of kernel}
	若线性映射$F:V\to W$的核$\ker(F)=\{0\}$,则对于线性无关的一组$v_1,\ldots,v_n\in V$,有$F(v_1),\ldots,F(v_n)$线性无关。
\end{theorem}
\begin{proof}
	(矩阵版本)对应矩阵零空间为$\{0\}$,则列满秩,列之间线性无关。
	
	(抽象版本)假设$x_1F(v_1)+\cdots+x_nF(v_n)=0$,则
	\[
		F(x_1v_1+\cdots+x_nv_n)=0,\implies x_1v_1+\cdots+x_nv_n=0.
	\]
	$v_1,\ldots,v_n$线性无关,故只有零解。
\end{proof}
\begin{definition}{线性映射的像}{image}
	线性映射$F:V\to W$的像(image)是所有$F(v)$的集合
	\[
		\im(F)\equiv\set{F(v)\in W}{\forall v\in V}.
	\]
\end{definition}
\begin{corollary}
	$\im(F)$是$W$的线性子空间。$\im(L_A)=\C(A)$。
\end{corollary}
\begin{theorem}{核和像的关系}{kernel and image}
	$V$是线性空间,$L:V\to W$是线性映射 
	\begin{equation}
		\dim(V)=\dim\bigkh{\ker(L)}+\dim\bigkh{\im(L)}.
	\end{equation}
\end{theorem}
\begin{proof}
	(矩阵版本)
	\[
		\dim(V)=\dim\bigkh{\N(A)}+\dim\bigkh{\C(A\tp)}=\dim\bigkh{\N(A)}+\dim\bigkh{\C(A)}.
	\]
	(抽象版本) 略
\end{proof}
\begin{theorem}{核、像和双射}{kernel, image and bijection}
	线性映射$L:V\to W$,且$\dim(V)=\dim(W)$,则 
	\[
		\ker(F)=\{0\}\iff\im(F)=W\iff L~\text{是双射.}
	\]
\end{theorem}
\begin{proof}
	略
\end{proof}
\section{基的变换}
\begin{definition}{坐标向量}{coordinate vector}
	设$B=\{v_1,\ldots,v_n\}$是线性空间$V$上的一组基,$V$中的向量$v\in V$可唯一写成$v=x_1v_1+\cdots+x_nv_n$,其在基$B$下的坐标向量(coordinate vector)为
	\[
		x_B(v)=\begin{bmatrix}
			x_1\\\vdots\\x_n
		\end{bmatrix}.
	\]
	显然$x_B:V\to\FF^n$是线性映射,且是一个双射。
\end{definition}
我们可以选取$V$上的另一组基$B'=\{u_1,\ldots,u_n\}$,基变换矩阵:
\[
	(u_1,\ldots,u_n)=(v_1,\ldots,v_n)M,\tag{$\ast$}
\]
$v$也可以写成$v=y_1u_1+\cdots+y_nu_n$,由于向量在基的变换下保持不变,故
\[
	v=(v_1,\ldots,v_n)(x_1,\ldots,x_n)\tp=(u_1,\ldots,u_n)(y_1,\ldots,y_n)\tp.
\]
可以推出
\[
	\begin{bmatrix}
		y_1\\\vdots\\y_n
	\end{bmatrix}=M\iv\begin{bmatrix}
		x_1\\\vdots\\x_n
	\end{bmatrix}.\tag{$\ast\ast$}
\]
\begin{theorem}{换基矩阵}{base transfomation matrix}
	$L:V\to W$是一个线性映射,$B=\{v_1,\ldots,v_n\}$是$V$上的一组基,$B'=\{w_1,\ldots,w_m\}$是$W$上的一组基。则存在唯一的$m\times n$矩阵$M_{B'}^B(L)$,使得$\forall v\in V$,
	\[
		x_{B'}\bigkh{L(v)}=M_{B'}^B(L)x_B(v).
	\]
\end{theorem}
\begin{proof}
	$\forall v\in V$,有
	\[
		v=x_1v_1+\cdots+x_nv_n,\quad L(v)=x_1L(v_1)+\cdots+x_nL(v_n).
	\]
	$L(v_i)\in W$,所以
	\[
		L(v_i)=m_{1i}w_1+\cdots+m_{mi}w_m.
	\]
	写成矩阵的形式即$\bigkh{L(v_1),\ldots,L(v_n)}=(w_1,\ldots,w_m)M$,从而
	\[
		L(v)=\bigkh{L(v_1),\ldots,L(v_n)}(x_1,\ldots,x_n)\tp=(w_1,\ldots,w_m)M(x_1,\ldots,x_n)\tp.
	\]
	故$L(v)$在$B'$上的坐标为$M(x_1,\ldots,x_n)\tp$。
\end{proof}

$M_{B'}^B(L)$是所有线性变换$L:V\to W$到$\dim(W)\times\dim(V)$矩阵的线性映射,并且是一个双射。

特别地,当$L\equiv\id:V\to V$时, 
\[
	x_{B'}(v)=M_{B'}^B(\id)x_B(v).
\]
\begin{theorem}{线性变换的复合与矩阵乘法}{linear transfomation composition with matrix multiplication}
	线性映射$L_1:U\to V,\enspace L_2:V\to W$,$B,B',B''$分别是$U,V,W$上的一组基,则
	\[
		M_{B''}^B(L_2\circ L_1)=M_{B''}^{B'}(L_2)M_{B'}^B(L_1).
	\]
\end{theorem}
\noindent 线性映射的复合等价于对应矩阵的乘法,由此可自然得到矩阵乘法的规则。
\begin{theorem}{$M_{B'}^B(\id)$可逆}{MB'Bid inversible}
	\[
		M_{B'}^B(\id)=M_B^{B'}(\id)\iv.
	\]
\end{theorem}
\begin{theorem}{}{}
	线性映射$L:V\to W$,$B,B'$是$V$上的两组基,$C,C'$是$W$上的两组基,则
	\[
		M_{C'}^{B'}(L)=M_{C'}^C(\id)M_C^B(L)M_B^{B'}(\id)=M_C^{C'}(\id)\iv M_C^B(L)M_B^{B'}(\id)
	\]
\end{theorem}
\begin{proof}
	利用$L=\id_W\circ L\circ\id_V$。
\end{proof}
\begin{corollary}
	$L:V\to V$,$B,B'$是$V$上的两组基,则
	\[
		M_{B'}^{B'}(L)=M_B^{B'}(\id)\iv M_B^B(L)M_B^{B'}(\id).
	\]
	因此相似变换就是换基,矩阵对角化就是找到描述线性变换的最好的基。
\end{corollary}

\section{对偶空间}

如何从已知的线性空间构造新的线性空间?
\begin{definition}{对偶空间}{dual space}
	线性空间$V$的对偶空间(dual space) $V^\ast$是所有线性映射$L:V\to\RR$构成的线性空间。
\end{definition}
\begin{theorem}{对偶空间的基}{dual space base}
	通过$V$的一组基$\{v_1,\ldots,v_n\}$可构造$V^\ast$的基$\{v^{\ast 1},\ldots,v^{\ast n}\}$,满足
	\begin{equation}
		v^{\ast i}(v_j)=\delta^i{}_j=
		\begin{cases}
			1,&i=j\\0,&i\neq j
		\end{cases}
	\end{equation}
\end{theorem}
\begin{proof}
	(完备性) $\forall L\in V^\ast$,由\thmref{thm:linear mapping and base},$L$可由其在基上的取值$\{L(v_1),\ldots,L(v_n)\}$唯一决定,又$L(v_1)v^{\ast 1}+\cdots+L(v_n)v^{\ast n}$和$L$在基$v_1,\ldots,v_n$上取到了相同的值,故二者相等:
	\[
		L=L(v_1)v^{\ast 1}+\cdots+L(v_n)v^{\ast n}.
	\]
	即$L$可被写成$\{v^{\ast 1},\ldots,v^{\ast n}\}$的线性组合。

	(线性无关)若基的线性组合是零映射
	$x_1v^{\ast 1}+\cdots+x_nv^{\ast n}=O$,
	则$O(v_i)=x_i=0$,即只有零解。
\end{proof}
\begin{example}{Fourier变换}{Fourier transfomation}
	Fourier变换
	\[
		\hat f(k)=\int f(x)\e{\i k\cdot x}\d^3x,
	\]
	就是将$\RR^3$的函数变成$(\RR^3)^\ast$的函数。
\end{example}
\begin{example}{对偶的对偶}{double dual space}
	依定义,对偶空间$V^\ast$的对偶空间$V^{\ast\ast}$是所有线性映射$F:V^\ast\to\RR$构成的线性空间。%我们可以构造一组基$\{v^{\ast\ast}_1,\ldots,v^{\ast\ast}_n\}$,满足
	%v^{\ast\ast}_i(v^{\ast j})=\delta_i{}^j.
	$\forall v\in V$,可定义映射$u^{\ast\ast}\in V^{\ast\ast}$,使得$\forall L\in V^\ast$
	\begin{equation}
		u^{\ast\ast}(L)=L(u).
	\end{equation}
	因此$V^{\ast\ast}$和$V$是自然同构的(natural isomorphism):$V^{\ast\ast}\cong V$。这是范畴论(category theory)的概念,粗糙地说就是这种同构关系不依赖于基的选取,而$V$和$V^\ast$的同构是依赖于基的。因此我们可以将$V^{\ast\ast}$和$V$视为同一个线性空间,
	从而$V^{\ast\ast},V^{\ast\ast\ast},\ldots$也就再没有研究价值了。
\end{example}
\begin{theorem}{对偶空间的基变换}{}
	给定线性空间$V$及其中的两组基$\{v_1,\ldots,v_n\}$和$\{u_1,\ldots,u_n\}$,我们可以给出对偶空间$V^\ast$的基$\{v^{\ast 1},\ldots,v^{\ast n}\}$和$\{u^{\ast 1},\ldots,u^{\ast n}\}$,满足
	\[
		v^{\ast i}(v_j)=\delta^i{}_j,\quad u^{\ast i}(u_j)=\delta^i{}_j,
	\]
	若$(v_1,\ldots,v_n)=(u_1,\ldots,u_n)A$,则$(v^{\ast 1},\ldots,v^{\ast n})\tp=A\iv(u^{\ast 1},\ldots,u^{\ast n})\tp$
\end{theorem}
\begin{proof}
	若$(v^{\ast 1},\ldots,v^{\ast n})\tp=B(u^{\ast 1},\ldots,u^{\ast n})\tp$,即
	\[
		v_i=\sum_{j=1}^nu_jA^j{}_i,\quad v^{\ast i}=\sum_{j=1}^nB^i{}_ju^{\ast j}.
	\]
	则
	\begin{align*}
		v^{\ast i}(v_j)&=\sum_{k=1}^nB^i{}_ku^{\ast k}\biggkh{\sum_{\ell=1}^n u_\ell A^\ell{}_j}=\sum_{k,\ell}B^i{}_kA^\ell{}_ju^{\ast k}(u_\ell)\\
		&=\sum_{k,\ell}B^i{}_kA^\ell{}_j\vd^k{}_\ell=\sum_{k=1}^nB^i{}_kA^k{}_j=\delta^i{}_j.
	\end{align*}
	故$B=A\iv$。
\end{proof}
\section{直和、直积}
\begin{definition}{线性空间的和}{sum of linear space}
	线性空间$U$的两个子空间$V,W$的和(sum) $V+W$定义为所有$v+w,v\in V,w\in W$的集合:
	\begin{equation}
		V+W\equiv\set{v+w}{v\in V,w\in W}.
	\end{equation}
\end{definition}
显然,$V+W$也是$U$的子空间。
\begin{definition}{线性空间的直和}{direct sum of linear space}
	线性空间$U$是$V$和$W$的直和(direct sum) $U=V\oplus W$,若$\forall u\in U$,存在唯一的$v\in V,w\in W$使得$u=v+w$。
\end{definition}
\begin{theorem}{和与直和}{sum and direct sum}
	若$U=V+W$且$V\cap W=\{0\}$,则$U=V\oplus W$。
\end{theorem}
\begin{proof}
	假设$u\in U$可以写成$u=v+w=v'+w'$,则$v-v'=w-w'$,
	
	又$v-v'\in V,\enspace w-w'\in W$且$V\cap W=\{0\}$,所以$v-v'=w-w'=0$。故分解是唯一的。
\end{proof}
\begin{theorem}{直和的存在}{exist of direct sum}
	$U$是一个有限维线性空间,$V$是$U$的子空间,则存在$U$的子空间$W$使得$U=V\oplus W.$
\end{theorem}
\begin{proof}
	取$V$的一组基$\{v_1,\ldots,v_r\}$,可将其扩张成$U$的一组基$\{v_1,\ldots,v_r,$\ $w_1,\ldots,w_m\}$,取$W=\spn(w_1,\ldots,w_m)$即可。
\end{proof}
\begin{corollary}
	\begin{equation}
		\dim(V\oplus W)=\dim(V)+\dim(W).
	\end{equation}
\end{corollary}
\begin{definition}{线性空间的直积}{direct product}
	两个线性空间$V,W$的直积(direct product) $V\times W$是所有形如$(v,w),v\in V,w\in W$的元素的集合:
	\begin{equation}
		V\times W\equiv\set{(v,w)}{v\in V,w\in W}.
	\end{equation}
\end{definition}
\begin{theorem}{直基的维度}{}
	$V\times W$是一个线性空间。且
	\begin{equation}
		\dim(V\times W)=\dim(V)+\dim(W).
	\end{equation}
\end{theorem}
\section{张量}
\begin{definition}{多重线性映射}{multiple linear mapping}
	映射$L:V_1\times\cdots\times V_r\to W$是一个多重线性映射(multiple linear mapping),若其对于每一个变量都是线性的:
	\[
		L(\ldots,au+bw,\ldots)=aL(\ldots,u,\ldots)+bL(\ldots,w,\ldots).
	\]
\end{definition}
\begin{definition}{张量空间$V^\ast\otimes V^\ast$}{tensor space V*oxV*}
	考虑所有多重线性函数$L:V\times V\to\RR$的集合,我们可以在这个集合上定义加法和数乘:
	\begin{compactitem}
		\item 加法:$(L_1+L_2)(u,v)=L_1(u,v)+L_2(u,v);$
		\item 数乘:$(cL)(u,v)=cL(u,v);$
		\item 零元:$O(u,v)\equiv 0.$
	\end{compactitem}
	因此这个集合构成一个线性空间,称作张量空间(tensor space) $V^\ast\otimes V^\ast$。
	其中的每一个元素$L$是二阶协变张量(covariant tensor),记作$(0,2)$张量。
\end{definition}
若$V$的一组基为$\{v_1,\ldots,v_n\}$,则$\forall L\in V^\ast\otimes V^\ast$ 
\[
	L(u,v)=L\biggkh{\sum_{i=1}^na_iv_i,\sum_{j=1}^nb_jv_j}=\sum_{i,j}a_ib_jL(v_i,v_j).
\]
$n^2$个函数值$L(v_i,v_j)$便可唯一确定函数$L$。

\begin{example}{$V^\ast\otimes V^\ast$的基}{}
	给定对偶空间$V^\ast$的一组基$\{v^{\ast 1},\ldots,v^{\ast n}\}$满足$v^{\ast i}(v_j)=\delta^i{}_j$。继而定义张量$v^{\ast i}\otimes v^{\ast j}\in V^\ast\otimes V^\ast$满足
	\[
		v^{\ast i}\otimes v^{\ast j}(u,v)=v^{\ast i}(u)v^{\ast j}(v).
	\]
	从而
	\[
		v^{\ast i}\otimes v^{\ast j}(v_k,v_\ell)=v^{\ast i}(v_k)v^{\ast j}(v_\ell)=\delta^i{}_k\vd^j{}_\ell.
	\]
	$n^2$个张量$v^{\ast i}\otimes v^{\ast j}$构成$V^\ast\otimes V^\ast$的一组基。
	\tcblower
	张量$\forall w\in V^\ast\otimes V^\ast$,
	\[
		w=\sum_{i,j}w_{ij}v^{\ast i}\otimes v^{\ast j},\quad w_{ij}=w(v_i,v_j).
	\]
	给出$V,V^\ast$的另一组基$\{u_1,\ldots,u_n\},\{u^{\ast 1},\ldots,u^{\ast n}\}$,有变换
	\begin{align*}
		(u_1,\ldots,u_n)&=(v_1,\ldots,v_n)A,\\
		(u^{\ast 1},\ldots,u^{\ast n})\tp&=(v^{\ast 1},\ldots,v^{\ast n})\tp A\iv.
	\end{align*}
	张量$w$在基$\{u^{\ast i}\otimes u^{\ast j}\}$下的分量
	\[
		w'_{ij}=w(u_i,u_j)=w\biggkh{\sum_{k=1}^nv_kA^k{}_i,\sum_{\ell=1}^nv_\ell A^\ell{}_j}=\sum_{k,\ell}w_{k\ell}A^k{}_iA^\ell{}_j.
	\]
	因此这也是协变(covariant)的含义:分量在坐标变换下同基的变换规律一致。
\end{example}
\begin{definition}{张量积}{tensor product}
	$U,V$是两个线性空间,定义$u\in U,v\in V$的张量积(tensor product)是一个新的元素$u\otimes v$,且满足以下性质:
	\begin{compactitem}
		\item 结合律:$(u\otimes v)\otimes w=u\otimes(v\otimes w)\equiv u\otimes v\otimes w;$
		\item 左分配律:$(u_1+u_2)\otimes v=u_1\otimes v+u_2\otimes v;$
		\item 右分配律:$u\otimes(v_1+v_2)=u\otimes v_1+u\otimes v_2;$
		\item 数乘:$(au)\otimes v=u\otimes(av)=a(u\otimes v).$
	\end{compactitem}
\end{definition}
张量积并不满足交换律,即$u\otimes v\neq v\otimes u$是两个不同的张量。
\begin{definition}{线性空间的张量积}{linear space tensor product}
	若$U,V$是两个线性空间,各自有一组基$\{u_1,\ldots,u_m\},\{v_1,\ldots,v_n\}$,定义基的张量积$\set{u_i\otimes v_j}{1\leqslant i\leqslant m,1\leqslant j\leqslant n}$张成的线性空间为$U,V$的张量积,记为$U\otimes V$。
\end{definition}
由定义 
\[
	\dim(U\otimes V)=\dim(U)\dim(V).
\]
\begin{example}{}{}
	$\forall u\in U,v\in V$,$u=x_1u_1+\cdots+x_mu_m,\enspace v=y_1v_1+\cdots+y_nv_n$,则 
	\[
		u\otimes v=\sum_{i=1}^m\sum_{j=1}^nx_iy_ju_i\otimes v_j\in U\otimes V.
	\]
	但并不是所有$U\otimes V$的元素都能写成$u\otimes v$的形式。
\end{example}
\begin{example}{张量空间$V\otimes V$}{}
	$V\otimes V$是所有$V^\ast\times V^\ast\to\RR$的双线性函数构成的线性空间,%但我们也可以用张量积定义。
	$\forall v\in V\otimes V$都可以写为
	\[
		v=\sum_{i,j}v^{ij}v_i\otimes v_j.
	\]
	称作二阶逆变张量(contravariant tensor),记作$(2,0)$张量。
	\tcblower
	换基时,
	\[
		v^{k\ell}=\sum_{i,j}A^k{}_iA^\ell{}_jv'^{ij},\quad v'^{ij}=\sum_{k,\ell}(A\iv)^i{}_k(A\iv)^j{}_\ell v^{k\ell}.
	\]
	逆变(contravariant)的含义:换基时分量每个指标对应的变换矩阵是基的变换矩阵的逆矩阵。
\end{example}
\begin{example}{混合张量$V\otimes V^\ast$}{}
	$(1,1)$张量$v\in V\otimes V^\ast$可以写成
	\[
		v=\sum_{i,j}v^i{}_jv_i\otimes v^{\ast j}.
	\]
\end{example}
\begin{example}{$V\otimes\cdots\otimes V\otimes V^\ast\otimes\cdots\otimes V^\ast=V^{\otimes k}\otimes V^{\ast\otimes\ell}$}{}
	$V\otimes\cdots\otimes V\otimes V^\ast\otimes\cdots\otimes V^\ast=V^{\otimes k}\otimes V^{\ast\otimes\ell}$的基 
	\[
		\set{v_{i_1}\otimes\cdots\otimes v_{i_k}\otimes v^{\ast j_1}\otimes\cdots\otimes v^{\ast j_\ell}}{1\leqslant i_1,\ldots,i_k,j_1,\ldots,j_\ell\leqslant n}
	\]
	$V^{\otimes k}\otimes V^{\ast\otimes\ell}$的元素 
	\[
		v=\sum_{\substack{i_1,\ldots,i_k\\j_1,\ldots,j_\ell}}(v^{i_1\cdots i_k}{}_{j_1\cdots j_\ell})v_{i_1}\otimes\cdots\otimes v_{i_k}\otimes v^{\ast j_1}\otimes\cdots\otimes v^{\ast j_\ell}.
	\]
	是$(k,\ell)$阶张量。基变换
	\begin{align*}
		&v'^{i_1\cdots i_k}{}_{j_1\cdots j_\ell}=\sum_{\substack{p_1,\ldots,p_k\\q_1,\ldots,q_\ell}}(A\iv)^{i_1}{}_{p_1}\cdots(A\iv)^{i_k}{}_{p_k}(v^{p_1\cdots p_k}{}_{q_1\cdots q_\ell})A^{q_1}{}_{j_1}\cdots A^{q_\ell}{}_{j_\ell}.
	\end{align*}
\end{example}