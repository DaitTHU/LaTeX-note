\chapter{群、环、域}

\section{二元运算}

\begin{definition}{二元运算}{binary operation}
	集合$S$上的一个二元运算(binary operation)形如映射$\circ:S\times S\to S$。

	其中$S\times S\equiv S^2$是笛卡尔积(Cartesian product),
	\[
		A\times B:=\set{(a,b)}{a\in A, b\in B}.
	\]
\end{definition}
二元运算在$S$上是封闭的(property of closure)。
\begin{definition}{恒等元}{identity element}
	$e\in S$是恒等元(identity element),若$\forall a\in S,\enspace e\circ a=a\circ e=a.$
\end{definition}
\begin{definition}{可逆}{inversible}
	$a\in S$是可逆的(inversible),若$\exists a\iv\in S,\enspace a\circ a\iv=a\iv\circ a=e.$
\end{definition}
特别地,简记
\[
	a^m\equiv a\circ\cdots\circ a,\quad a^{-m}\equiv a\iv\circ\cdots\circ a\iv.
\]

\section{群与子群}

\begin{definition}{群}{group}
	群(group)是有二元运算$\circ:G\times G\to G$和集合$G$并满足下列性质的组合$(G,\circ)$:
	\begin{compactitem}
		\item 结合律:$(a\circ b)\circ c=a\circ(b\circ c)\equiv a\circ b\circ c;$
		\item 单位元:$\exists e\in G$使得$e\circ a=a\circ e=a;$
		\item 逆:$\forall a\in G,\enspace\exists a\iv$使得$a\iv\circ a=a\circ a\iv=e.$
	\end{compactitem}
	若还满足交换律,则称为交换群或Abel群。

	%若仅满足结合律,则称为半群;有单位元的半群又叫含幺半群。
	\tcblower
	群的阶(order) $\ord(G)$ 表示其元素的个数。群可分为有限群和无限群。
\end{definition}
\begin{theorem}{单位元和逆元的唯一性}{uniqueness of identity element and inverse}
	在群中只能有一个单位元,而群中的每个元素都正好有一个逆元素。
\end{theorem}
\begin{proof}
	若一个群存在两个单位元$e,e'$,则 
	\[
		e=e\circ e'=e';
	\]
	若一个元素$a$存在两个逆$b,c$,则 
	\[
		b=b\circ e=b\circ(a\circ c)=(b\circ a)\circ c=e\circ c=c.
		\qedhere
	\]
\end{proof}
\begin{example}{群的例子}{example of group}
	\begin{compactitem}
		\item 整数加群$(\ZZ,+)$:单位元0;
		\item 非零实数乘法群$(\RR\backslash\{0\},\times)$:单位元1;
		\item 一般线性(general linear)群$\GL(n)$:所有$n$阶可逆矩阵集合,单位元$I_n$。
	\end{compactitem}
\end{example}
\begin{theorem}{消去律}{cancellation law}
	$\forall a,b,c\in G$,有
	\begin{subequations}
		\begin{gather}
			a\circ b=a\circ c\implies b=c;\\
			b\circ a=c\circ a\implies b=c;\\
			b\circ a=a~\text{或}~a\circ b=a\implies b=e.
		\end{gather}
	\end{subequations}
\end{theorem}
\begin{proof}
	左乘/右乘$a\iv$.
\end{proof}
\begin{remark}
	逆$a\iv$的存在很关键,如果$G$上的运算只是结合的,则$(G,\circ)$是一个半群(semigroup),有单位元的半群又叫幺半群(monoid)。
\end{remark}
\begin{definition}{置换群和对称群}{permutation group and symmetric group}
	给定有限集合$T$,所有可逆映射$f:T\to T$构成一个群$\sym(T)$,运算是映射的复合,称做置换群(permutation group)
	\tcblower
	当$T=\{1,2,\ldots,n\}$时,对应的置换群称为对称群(symmetric group) $S_n$。
\end{definition}
\begin{example}{$S_2$}{symmetric group S2}
	$S_2=\{1,p\}$,其中 
		\begin{align*}
			1=\id:\{1,2\}\to\{1,2\}&,\\
			p:\{1,2\}\to\{1,2\}&,\quad p(1)=2,\enspace p(2)=1.
		\end{align*}
		$S_2$是交换群。可列出Cayley表
		\begin{align*}
			\begin{array}{c|cc}
				&1&p\\
				\hline
				1&1&p\\
				p&p&1
			\end{array}
		\end{align*}
\end{example}
\begin{example}{$S_3$}{symmetric group S3}
	$S_3$:定义生成元$x,y$满足:
	\begin{align*}
		x(1)=2,\quad x(2)=3,\quad x(3)=1;\\
		y(1)=2,\quad y(2)=1,\quad y(3)=3.
	\end{align*}
	可以证明生成元之间的关系:$x^3=1,\enspace y^2=1,\enspace x^2y=yx$,故$S_3$中所有元素都能写成生成元的积:
	\[
		S_3=\{1,x,x^2,y,xy,x^2y\},
	\]
	易知$S_3$不交换。根据生成元,$S_3$还可写为$S_3=\set{x,y}{x^3=1,y^2=1,x^2y=yx}$。
	
	生成元及其关系称作一个群的表现(presentation),一个群的表现不唯一。
\end{example}
\begin{definition}{子群}{subgroup}
	$H\subset G$是$G$的子群(subgroup),若$H$满足
	\begin{itemize}
		\item 封闭性:$\forall a,b\in H$,$a\circ b\in H$;
		\item 单位元:$e\in H$;
		\item 逆元:$\forall a\in H,\enspace\exists a\iv\in H$。
	\end{itemize}
	$\{e\}$和$G$都是平凡的子群,其他子群称为真子群(proper subgroup)。
\end{definition}

\begin{example}{子群的例子}{}
	\begin{itemize}
		\item 圆群:$(\set{z\in\CC}{\abs z=1},\times)\subset(\CC\backslash\{0\},\times)$;
		\item 特殊线性群$\SL(n)\subset\GL(n)$:所有行列式为1的$n$阶方阵;
	\end{itemize}
\end{example}

\begin{definition}{循环群}{cyclic group}
	循环群(cyclic group)是
	\begin{equation}
		Z_n\equiv\set{1,x,\ldots,x^{n-1}}{x^n=1},
	\end{equation}
	其生成元为$x$。
\end{definition}
\begin{example}{}{}
	$S_3$有两个子群是循环群:$\set{x^k}{x^3=1}=Z_3$和$\set{y^k}{y^2=1}=Z_2$。
\end{example}

\section{群同态}

\begin{definition}{群同态}{group homomorphism}
	$(G,\circ),(G',\circ')$是群,映射$\phi:G\to G'$是群同态(group homomorphism)若$\forall a,b\in G$
	\begin{equation}
		\phi(a\circ b)=\phi(a)\circ'\phi(b).
	\end{equation}
	也称映射$\phi$和群上的乘法相容(compatible)。
\end{definition}
\begin{theorem}{群同态下的单位元和逆元}{}
	若$\phi:G\to G'$是群同态,$G,G'$的单位元分别为$1,1'$,$a\in G$,则
	\begin{equation}
		\phi(1)=1',\quad \phi(a\iv)=\phi(a)\iv.
	\end{equation}
\end{theorem}
\begin{proof}
	(1)由$\phi(1)=\phi(1\circ 1)=\phi(1)\circ'\phi(1)$,再运用消去律可得$1'=\phi(1)$;

	(2)由$\phi(a\iv)\circ'\phi(a)=\phi(a\iv\circ a)=\phi(1)=1'$可得$\phi(a\iv)=\phi(a)\iv$。
\end{proof}
\begin{example}
	线性空间和$+$构成一个群,线性映射都是群同态。
\end{example}
\begin{definition}{群同态的像}{image of group homomorphism}
	群同态$\phi:G\to G'$的像(image) $\im\phi$定义为
	\begin{equation}
		\im\phi\equiv\set{x\in G'}{\exists a\in G,\enspace\phi(a)=x}.
	\end{equation}
\end{definition}
\begin{definition}{群同态的核}{kernel of group homomorphism}
	群同态$\phi:G\to G'$的核(kernel) $\ker\phi$定义为
	\begin{equation}
		\ker\phi\equiv\set{a\in G}{\phi(a)=1'}.
	\end{equation}
\end{definition}
\begin{theorem}{}{}
	若$\phi:G\to G'$是群同态,则$\im\phi$是$G'$的子群,$\ker\phi$是$G$的子群。
\end{theorem}
\begin{proof}
	考虑线性空间在$+$下构成的群,此时线性映射作为群同态的像与核同之前线性映射的像与核相同。
\end{proof}
\begin{definition}{左陪集}{left coset}
	$H$是$G$的子群,$a\in G$,则
	\begin{equation}
		a\circ H\equiv\set{a\circ h}{h\in H}
	\end{equation}
	是$H$在$G$下的一个左陪集(left coset)。同理可定义右陪集。
\end{definition}
\begin{theorem}{}{}
	群同态$\phi:G\to G'$,$a,b\in G$,则以下命题等价:
	\begin{enumerate}
		\item $\phi(a)=\phi(b)$;
		\item $a\iv\circ b\in\ker\phi$;
		\item $b\in a\circ\ker\phi$;
		\item $b\circ\ker\phi=a\circ\ker\phi$。
	\end{enumerate}
\end{theorem}
\begin{proof}
	$(1)\Rightarrow(2)$:
	\[
		\phi(a)=\phi(b)\implies 1'=\phi(a)\iv\circ'\phi(b)=\phi(a\iv\circ b)\implies a\iv\circ b\in\ker\phi;
	\]

	$(2)\Rightarrow(3)$:
	\[
		a\iv\circ b=h\in\ker\phi\implies b=a\circ h\in a\circ\ker\phi.
	\]

	$(3)\Rightarrow(4)$:由$b\in a\circ\ker\phi$,$\exists h\in\ker\phi$使得$b=a\circ h$;
	$\forall b'\in b\circ\ker\phi$,$\exists h'\in \ker\phi$使得$b'=b\circ h'=(a\circ h)\circ h'=a\circ(h\circ h')\in a\circ\ker\phi$,故$b\circ\ker\phi\subset a\circ\ker\phi$;同理由$a=b\circ h\iv$可以证明$a\circ\ker\phi\subset b\circ\ker\phi$,故$a\circ\ker\phi=b\circ\ker\phi.$

	$(4)\Rightarrow(1)$:$\forall h\in\ker\phi,\enspace\exists h'\in\ker\phi$使得$a\circ h=b\circ h'$
	\begin{align*}
		&\implies \phi(a\circ h)=\phi(b\circ h')\implies\phi(a)\circ'\phi(h)=\phi(b)\circ'\phi(h')\\
		&\implies\phi(a)\circ'1'=\phi(b)\circ'1'\implies\phi(a)=\phi(b).
	\end{align*}
	故以上4个命题等价。
\end{proof}
\begin{remark}~
	\begin{itemize}
		\item 群同态的核不仅告诉我们$G$中的哪些元素映射到1,也告诉我们哪些元素的像相同;
		\item 上面的命题在线性方程组中的应用就是$Ax=b$的通解$=$特解$+\,\{Ax=0\}$的通解。
	\end{itemize}
\end{remark}
\begin{corollary}
	群同态$\phi:G\to G'$是单射$\iff\ker\phi=\{1\}$。
\end{corollary}
\begin{definition}{正规子群}{normal subgroup}
	$N\subset G$是$G$的正规子群(normal subgroup)若$\forall a\in N,\forall g\in G$,共轭$g\circ a\circ g\iv\in N$。
\end{definition}
\begin{theorem}{}{}
	$\phi:G\to G'$是群同态,$\ker\phi$是$G$的正规子群。
\end{theorem}
\begin{definition}{中心}{center of group}
	群$G$的中心(center)是
	\begin{equation}
		Z_G\equiv\set{z\in G}{z\circ x=x\circ z,\forall x\in G},
	\end{equation}
	中心$Z_G$总是$G$的正规子群。
\end{definition}
\begin{example}{}{}
	\begin{itemize}
		\item 行列式是GL的群同态:
		\[
			\det:\GL(n)\to(\RR\backslash\{0\},\times)
		\]
		核$\ker\det=\SL(n)$是$\GL(n)$的正规子群。
		\item $Z_{\SL(2)}=\{I,-I\}$;
		\item $Z_{S_n}=\{1\},\enspace n\geq 3$。
	\end{itemize}
\end{example}

\section{群同构}

\begin{definition}{群同构}{isomorphism}
	若群同态$\phi:G\to G'$是双射,则称$\phi$为群同构(isomorphism),称$G,G'$是同构的(isomorphic),记作$G\simeq G'$。
	
	群到自己的同构$\phi:G\to G$也叫自同构。(automorphism)
\end{definition}
恒等映射$\id:G\to G$是自同构。
\begin{example}{群同构的例子}{}
	\begin{itemize}
		\item 指数函数
		\[
			\exp:(\RR,+)\to(\RR_{>0},\times),\enspace x\mapsto\e x.
		\]
		\item $P$是投影矩阵:$P^2=P$
		\[
			S_2\to\{I,I-2P\}.
		\]
	\end{itemize}
\end{example}

\section{等价关系}

\begin{definition}{等价关系}{}
	集合$S$上的等价关系$\sim$是$S$中两个元素$a,b$之间的关系,记作$a\sim b$,满足:
	\begin{itemize}
		\item 传递性:$a\sim b,b\sim c\implies a\sim c;$
		\item 对称性:$a\sim b\implies b\sim a;$
		\item 自反性:$\forall a,\enspace a\sim a.$
	\end{itemize}
\end{definition}
\begin{remark}~
	\begin{itemize}
		\item 等价关系可以看成$=$的抽象;
		\item 等价关系可以理解为映射$f:S\times S\to\{0,1\}$满足 
		\[
			a\sim b\iff f(a,b)=1.
		\]
		\item 未竟
	\end{itemize}
\end{remark}


