\chapter{线性方程组}

线性方程(linear equation)是未知数最高次数为1的方程。考虑$m$个$n$元线性方程构成的线性方程组(linear equation set)
\[
	\sum_{j=1}^nA_{ij}x_j=b_i.\quad i=1,2,\ldots,m.
\]
可以把系数写成系数矩阵$A$,即$Ax=b$。

\section{消元法}
\label{sec:elimination}

\begin{definition}{Gauss消元法}{Gaussian elimination}
	消元法(elimination)就是通过对方程之间倍加消元,得到一个上三角方程组,比如
	\[
		\begin{equationset}
			x-2y&=1\\3x+2y&=11
		\end{equationset}\implies
		\begin{equationset}
			x-2y&=1\\8y&=8
		\end{equationset}
	\]
	而主元(pivot element)就是每个方程的第一个非0系数。
\end{definition}

\begin{remark}
	当主元数目$<$未知数时,消元法失效,若最终结果:
	\begin{compactitem}
		\item 得到$0\neq 0$,则方程组无解;
		\item 得到$0=0$,则方程组有无穷多解。
	\end{compactitem}
	因此消元法要求:方程个数与未知数个数相同。
\end{remark}

\begin{method}{消元法算法}{elimination algorithm}
	\begin{compactenum}
		\item 找到第1个$x_1$系数不为0的方程并移到最上面;%$x_1$的系数就是第一个主元。
		\item 通过将方程1倍加,从第2个到第$n$个方程中消去$x_1$ ;
		\item 得到的第2个到第$n$个方程构成$(n-1)$元的线性方程组,重复步骤1。
		\item 最后结果如果是一个上三角方程组,便可从最后一个方程开始解出全部未知数。否则消元法失效。
	\end{compactenum}
\end{method}



\section{矩阵的行变换}
方程中,置换、倍加、倍乘同时作用在系数矩阵$A$和$b$上,因此可以写成增广矩阵(augmented matrix) $[A\enspace b]$并对其消元得到$[I\enspace b']$。类似的,可以考虑对一般矩阵进行置换、倍加、倍乘的操作。
\begin{definition}{矩阵的初等行变换}{}
	\begin{compactitem}
		\item 倍加(row addition):一行乘系数加到另一行
		\item 对换(row switching):交换两行
		\item 倍乘(row multiplication):一行乘以一个非零系数
	\end{compactitem}
\end{definition}

\begin{remark}
	利用Gauss-Jordan消元法%Gauss-Jordan elimination
	对增广矩阵$[A\enspace I]$做消元操作:
	\[
		[A\enspace I]\enspace\to\enspace[I\enspace A\iv].
	\]
	就可以得到$A$的逆$A\iv$。
\end{remark}

\begin{definition}{行等价}{row-equivalent}
	如果一个矩阵可以经过行变换得到另一个矩阵,则它们行等价(row-equivalent)。
\end{definition}

\begin{definition}{行阶梯矩阵}{row echelon matrix}
	矩阵$M$若其满足以下性质:
	\begin{compactitem}
		\item 如果$M$的第$i$行是0行,则下面的所有行的都是0行;
		\item 如果$M$的第$i$行不全是0,则从左数第一个非0元素叫做主元。每个主元所在的列都在其上一行的主元的所在列的右边;
		\item 同一列中在主元下面的元素都是0。
	\end{compactitem}
	则称矩阵$M$为行阶梯型(row echelon form)。
	% 所有与$A$行等价的行阶梯型矩阵的集合记作$ref(A)$。
	\tcblower
	比如,形如(其中主元$\ast$非0,$\cdot$任意)
	\[
		\begin{bmatrix}
			0&\ast&\cdot&\cdot&\cdot&\cdot&\cdot\\
			0&0&0&\ast&\cdot&\cdot&\cdot\\
			0&0&0&0&\ast&\cdot&\cdot\\
			0&0&0&0&0&0&0
		\end{bmatrix},
	\]
	就是一个行阶梯矩阵,其中$\ast$是主元。
\end{definition}
\begin{remark}
	消元法就是把增广矩阵变成行阶梯矩阵的过程。
\end{remark}

显然,行阶梯矩阵并不唯一,还可以进一步化简。

\begin{definition}{约化行阶梯矩阵}{reduced row echelon matrix}
	约化行阶梯型(reduced row echelon form)矩阵还满足以下额外性质:
	\begin{compactitem}
		\item 每个主元都是1;
		\item 主元所在列只有主元非0,称为主列,其他列称为自由列。
	\end{compactitem}
	若$A$与约化行阶梯矩阵$U$行等价,记作$U=\rref(A)$。
	\tcblower
	按上面的例子,其约化行阶梯矩阵为
	\[
		\begin{bmatrix}
			0&1&\cdot&0&0&\cdot&\cdot\\
			0&0&0&1&0&\cdot&\cdot\\
			0&0&0&0&1&\cdot&\cdot\\
			0&0&0&0&0&0&0
		\end{bmatrix},
	\]
	第2,4,5列为主列,其余为自由列。
\end{definition}
\begin{theorem}
	{}{}
	一个矩阵的约化行阶梯矩阵是唯一的。
\end{theorem}

\begin{proof}
	唯一性证明见Lay书的附录A。
\end{proof}

\begin{remark}
	借助约化行阶梯矩阵的概念,我们可归纳出解方程组$Ax=b$的方法:
	\begin{compactitem}
		\item 将增广矩阵$[A\enspace b]$约化为$[\rref(A)\enspace b']$;
		\item 解的存在性:若$\rref(A)$有0行,且$b'$对应行元素非0,则无解;反之有解;
		\item 解的唯一性:若$\rref(A)$没有自由列,则解唯一。
	\end{compactitem}
\end{remark}

\begin{definition}{初等矩阵}{Elementary Matrix}
	对$m\times n$矩阵$A$行变换$\iff$用$m\times m$初等矩阵(elementary matrix)左乘$A$。
	
	初等矩阵有以下三种类型:
	\begin{compactitem}
		\item 倍加:$A$的第$i$行的$a$倍加到第$j$行($r_j'=ar_i+r_j$)
		\[
			E_{ij}(a):=
			\begin{bmatrix}
				\ddots&&a\\ &\ddots\\ &&\ddots
			\end{bmatrix}=I+ae_{ij},
		\]
		逆:$E_{ij}(a)\iv=E_{ij}(-a)$;
		\item 置换:置换$A$的第$i$行和第$j$行($r_i\rightleftarrows r_j$)
		\[
			E_{ij}:=
			\begin{bmatrix}
				\ddots\\ &0&&1\\ &&\ddots\\ &1&&0\\ &&&&\ddots
			\end{bmatrix}=I+e_{ij}+e_{ji}-e_{ii}-e_{jj},
		\]
		逆:$E_{ij}\iv=E_{ij}$;
		%置换矩阵(permutation matrix)与自己互逆;
		\item 倍乘:$A$的第$i$行乘一个非0常数$c$ ($r_i'=cr_i$)
		\[
			E_i(c):=
			\begin{bmatrix}
				\ddots\\ &c\\ &&\ddots
			\end{bmatrix}=I+(c-1)e_{ii}.
		\]
		逆:$E_i(c)\iv=E_i(c\iv).$
	\end{compactitem}
\end{definition}
消元法:用一系列初等矩阵$E_1,\ldots,E_k$左乘$A$,把$A$化简成行阶梯矩阵。
\section{LU分解}
\begin{definition}{上/下三角矩阵}{upper/lower triangular matrix}
	上三角矩阵(upper triangular matrix)~$U$是主对角线以下元素都是0的方阵
	\[
		U_{ij}=0,\quad\forall i>j,
	\]
	同理可定义下三角矩阵(lower ...)~$L$满足$L_{ij}=0,\enspace\forall i<j.$
\end{definition}
不难注意到,倍加矩阵和逆矩阵都同时是上/下三角矩阵。这是LU分解的基础。
\begin{theorem}{LU分解}{LU decomposition}
	对于方阵$A$来说,$A$的LU分解(lower-upper decomposition/factorization)是将$A$分解成一个下三角矩阵$L$和一个上三角矩阵$U$的乘积:
	\begin{equation}
		A=LU,
	\end{equation}
	有时需要再乘上一个置换矩阵$PA=LU$,称为LUP分解(LU factorization with partial pivoting)。
	% 如果$A$化成行阶梯矩阵$U$的过程中没有置换,则$A$有一个LU分解;反之,则存在一个置换矩阵$P$,使得$PA$有一个LU分解。
\end{theorem}
\begin{proof}
	$U$是和$A$等价的行阶梯矩阵,$U$是上三角的。
	如果$A$化成行阶梯矩阵$U$的过程中没有置换,则从$A$到$U$的过程中,我们只需消去主元下面的元素:
	\[
		E_k\cdots E_2E_1A=U.
	\]
	由于$E_i$及其逆$E_i\iv$都是下三角的,故
	\[
		L=E_1\iv E_2\iv\cdots E_k\iv,
	\]
	也是下三角的,%进而$A=LU$;
	$A=LU$;反之,则存在一个置换矩阵$P$,使得$PA=LU$。
\end{proof}

\begin{remark}
	LU分解可以被视为Gauss消元法的矩阵形式。在数值计算上,LU分解经常被用来解线性方程组,且在求逆矩阵和计算行列式中都是一个关键的步骤。
\end{remark}

\begin{theorem}
	{LDU分解}{LDU decomposition}
	LDU分解(lower-diagonal-upper decomposition)是将矩阵写成
	\[
		A=LDU,
	\]
	其中$D$是对角矩阵,$L,U$是单位下/上三角的(unitriangular),即其对角元均为1。
\end{theorem}




