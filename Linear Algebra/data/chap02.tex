\chapter{线性方程组}

线性方程(linear equation)是未知数最高次数为1的方程。考虑$m$个$n$元线性方程构成的线性方程组(linear equation set)
\[
	\lhkh{\begin{aligned}
		A_{11}x_1+\cdots+A_{1n}x_n&=b_1,\\
		\cdots&\\
		A_{m1}x_1+\cdots+A_{mn}x_n&=b_m,
	\end{aligned}}
\]
可以把系数写成系数矩阵$A$,未知数和常数写成向量$x,b$,即$Ax=b$。

\section{消元法}
\label{sec:elimination}

如何解线性方程组,得到$x$?如果$A$是可逆的,将$Ax=b$左乘逆$A\iv$,当然可以得到$x=A\iv b$。但一般我们不知道逆$A\iv$,甚至$A$本身可能都不可逆,此时如何求解?

\begin{theorem}{Gauss消元法}{Gaussian elimination}
	消元法(elimination)就是通过对方程之间倍加消元,得到一个等价的上三角方程组。
	\[
		\text{e.g. }\lhkh{\begin{aligned}
			x_1-2x_2&=1\\3x_1+2x_2&=11
		\end{aligned}}\iff
		\lhkh{\begin{aligned}
			x_1-2x_2&=1\\8x_2&=8
		\end{aligned}}\iff
		\lhkh{\begin{aligned}
			x_1\phantom{-2x_2}&=3\\x_2&=1
		\end{aligned}}
	\]
	具体算法为:
	\begin{compactenum}
		\item 找到第1个$x_1$系数不为0的方程作为方程(1);%$x_1$的系数就是第一个主元。
		\item 通过将方程(1)倍加,从方程(2)到方程$(m)$中消去$x_1$ ;
		\item 得到的方程(2)到方程$(m)$构成$(n-1)$元的线性方程组,重复步骤1。
		\item 最后结果如果是一个上三角方程组,便可从最后一个方程开始解出全部未知数。
		否则消元法失效,此时:
		\begin{itemize}
			\item 若得到$0\neq 0$,则方程组无解;
			\item 若得到$0=0$,则方程组有无穷多解。
		\end{itemize}
	\end{compactenum}
	上三角方程组中每个方程的第一个非0系数称为主元(pivot element)。
\end{theorem}

\begin{remark}
	当主元数目$<$未知数时,消元法失效。因此有唯一解要求:独立方程个数与未知数个数相同。
\end{remark}

\section{矩阵的行变换}

由于消元法对方程的倍加操作同时作用在系数矩阵$A$和常数项$b$上,因此可以写成分块矩阵的形式:$[A\enspace b]$,称为增广矩阵(augmented matrix),对其消元得到$[I\enspace x]$:
\[
	\fkh{\begin{array}{cc|c}
		1&-2&1\\3&2&11
	\end{array}}
	\enspace\sim\enspace
	\fkh{\begin{array}{cc|c}
		1&-2&1\\ &8&8
	\end{array}}
	\enspace\sim\enspace
	\fkh{\begin{array}{cc|c}
		1&&3\\ &1&1
	\end{array}}.
\]
类似的,可以考虑对一般矩阵进行这种操作。

\begin{definition}{矩阵的初等行变换}{}
	矩阵有三种非退化的初等行变换(elementary row operation):
	\begin{compactitem}
		\item 倍加(row addition):一行乘系数加到另一行;
		\item 对换(row switching):交换两行;
		\item 倍乘(row multiplication):一行乘以一个非零系数。
	\end{compactitem}
\end{definition}

\begin{definition}{行等价}{row-equivalent}
	如果一个矩阵可以经过行变换得到另一个矩阵,则它们行等价(row equivalent)。
\end{definition}

\begin{remark}
	行等价是一种等价关系,满足自反性、对称性和传递性。见\dfnref{dfn:equivalence relation}。
\end{remark}

\paragraph{行阶梯矩阵}

Gauss消元法所得到的阶梯型方程组,对应的矩阵可称为行阶梯型矩阵。

\begin{definition}{行阶梯矩阵}{row echelon matrix}
	矩阵$M$若其满足以下性质:
	\begin{compactitem}
		\item 如果$M$的第$i$行是0行,则下面的所有行的都是0行;
		\item 如果$M$的第$i$行不全是0,则从左数第一个非0元素是主元。
		\item 记第$i$行的主元所在列为第$\ell_i$列,若第$i,j$行都不是0行且$i<j$,则$\ell_i<\ell_j$。
		% 每个主元所在的列都在其上一行的主元的所在列的右边;
		% \item 主元下面的元素都是0。
	\end{compactitem}
	则称矩阵$M$为行阶梯型(row echelon form)。
	% 所有与$A$行等价的行阶梯型矩阵的集合记作$ref(A)$。
	\tcblower
	比如,形如($\ast$表示任意非0数,不同位置的$\ast$不一定相等,而$\cdot$可以是任意数)
	\[
		\begin{bmatrix}
			0&\ast&\cdot&\cdot&\cdot&\cdot&\cdot\\
			0&0&0&\ast&\cdot&\cdot&\cdot\\
			0&0&0&0&\ast&\cdot&\cdot\\
			0&0&0&0&0&0&0
		\end{bmatrix}
	\]
	就是一个行阶梯矩阵,其中$\ast$是主元。
\end{definition}
\begin{remark}
	消元法就是把增广矩阵变成行阶梯矩阵的过程。
\end{remark}

显然,行阶梯矩阵并不唯一,还可以进一步化简。

\begin{definition}{约化行阶梯矩阵}{reduced row echelon matrix}
	约化行阶梯型(reduced row echelon form)矩阵还满足以下额外性质:
	\begin{compactitem}
		\item 每个主元都是1;
		\item 主元所在列只有主元非0,称为主列,其他列称为自由列。
	\end{compactitem}
	若$A$与约化行阶梯矩阵$U$行等价,记作$U=\rref(A)$。
	\tcblower
	按上面的例子,其约化行阶梯矩阵为
	\[
		\begin{bmatrix}
			0&1&\cdot&0&0&\cdot&\cdot\\
			0&0&0&1&0&\cdot&\cdot\\
			0&0&0&0&1&\cdot&\cdot\\
			0&0&0&0&0&0&0
		\end{bmatrix},
	\]
	第2,4,5列为主列,其余为自由列。
\end{definition}
\begin{theorem}
	{}{}
	一个矩阵的约化行阶梯矩阵是唯一的。
\end{theorem}

\begin{proof}
	唯一性证明见Lay书的附录A。
\end{proof}

\begin{remark}
	借助约化行阶梯矩阵的概念,我们可归纳出解方程组$Ax=b$的方法:
	\begin{compactitem}
		\item 将增广矩阵$[A\enspace b]$约化为$[\rref(A)\enspace b']$;
		\item 解的存在性:若$\rref(A)$有0行,且$b'$对应行元素非0,则无解;反之有解;
		\item 解的唯一性:若$\rref(A)$没有自由列,则解唯一。
	\end{compactitem}
\end{remark}

\paragraph{初等矩阵}

我们可以将矩阵的行变换等价地用矩阵乘法表示,即就结果而说:
\begin{center}
	对$m\times n$矩阵$A$进行初等行变换$\iff$用一个特定的$m\times m$矩阵左乘$A$
\end{center}
称为初等矩阵(elementary matrix)。

\begin{definition}
	{初等矩阵}{}
	与初等行变换对应,初等矩阵有以下三种类型:
	\begin{itemize}
		\item 倍加:$A$的第$i$行的$a$倍加到第$j$行($r_j'=ar_i+r_j$)
		\[
			E_{i\to j}(a):=
			\begin{bmatrix}
				\ddots\\ &1&&a\\ &&\ddots\\ &&&1\\ &&&&\ddots
			\end{bmatrix}=I+ae_{ij},
		\]
		\item 置换:置换$A$的第$i$行和第$j$行($r_i\rightleftarrows r_j$)
		\[
			E_{i\leftrightarrow j}:=
			\begin{bmatrix}
				\ddots\\ &0&&1\\ &&\ddots\\ &1&&0\\ &&&&\ddots
			\end{bmatrix}=I+e_{ij}+e_{ji}-e_{ii}-e_{jj},
		\]
		\item 倍乘:$A$的第$i$行乘一个非0常数$c$ ($r_i'=cr_i$)
		\[
			E_i(c):=
			\begin{bmatrix}
				\ddots\\ &1\\ &&c\\ &&&1\\ &&&&\ddots
			\end{bmatrix}=I+(c-1)e_{ii}.
		\]
	\end{itemize}
	其中$\ddots$所省略的对角项均为1,但阶数不一定相同。
\end{definition}

\begin{corollary}
	初等矩阵的逆还是初等矩阵:
	\begin{subequations}
		\begin{align}
			E_{i\to j}(a)\iv&=E_{i\to j}(-a),\\
			E_{i\leftrightarrow j}\iv&=E_{i\leftrightarrow j},\\
			E_i(c)\iv&=E_i(c\iv)
		\end{align}
	\end{subequations}
\end{corollary}

\begin{remark}
	消元法:用一系列初等矩阵$E_1,\ldots,E_k$左乘$A$,把$A$化简成行阶梯矩阵。
\end{remark}

\begin{definition}
	{初等矩阵的广义定义}{}
	$\forall u,v\in\FF^m$,$\sigma\in\FF$,定义$I+\sigma uv\tp$为(广义的)初等矩阵。
\end{definition}
\begin{remark}
	三种行变换对应的初等矩阵也可以写成$I+\sigma uv\tp$的形式:
	\begin{subequations}
		\begin{align}
			E_{i\to j}(a)&=I+ae_ie_j\tp,\\
			E_{i\leftrightarrow j}&=I-(e_i-e_j)(e_i-e_j)\tp,\\
			E_i(c)&=I+(c-1)e_ie_i\tp.
		\end{align}
	\end{subequations}
	由式\eqref{eqn:(I+uvT)-1},初等矩阵都是可逆的,并且逆也是初等矩阵:
	\begin{equation}
		(I+\sigma uv\tp)\iv=I-\frac{\sigma}{1+\sigma vu\tp}uv\tp,
	\end{equation}
\end{remark}

\begin{theorem}
	{可逆的等价命题}{}
	若$A$是$n$阶方阵,则下列叙述等价:
	\begin{enumerate}
		\item $A$可逆;
		\item $\forall b\in\FF^n$,$Ax=b$的解唯一;
		\item 齐次线性方程组$Ax=0$只有零解;
		\item $A$的行阶梯矩阵有$n$个主元;
		\item $\rref(A)=I$;
		\item $A$可以写成有限个初等矩阵的乘积。
	\end{enumerate}
\end{theorem}

\begin{proof}
	采用轮转证法:

	$(1)\implies(2)$:左乘$A\iv$可得$x=A\iv b$;

	$(2)\implies(3)$:取$b=0$,又注意到$A0=0$,由解的唯一性得证;

	$(3)\implies(4)$:方程组有唯一解$\iff$主元数$=$未知数个数$=n$;

	$(4)\implies(5)$:显然;

	$(5)\implies(6)$:在$A$化为$\rref(A)=I$的初等变换中,与每步初等变换过程等价的初等矩阵为$E_1,\ldots,E_k$,即$E_k\cdots E_1A=I$,因此$A=E_1\iv\cdots E_k\iv$,且$E_1\iv,\ldots,E_k\iv$也是初等矩阵;

	$(6)\implies(1)$:初等矩阵均可逆,故其乘积$A$也可逆。
\end{proof}

\begin{remark}
	可以证明:$n$阶可逆矩阵可以表示成不超过$n^2$个初等矩阵的乘积。
\end{remark}

\begin{corollary}
	利用Gauss-Jordan消元法
	对增广矩阵$[A\enspace I]$做消元操作,就可以得到$A$的逆$A\iv$:
	\[
		[A\enspace I]\enspace\sim\enspace[I\enspace A\iv].
	\]
\end{corollary}

\begin{definition}
	{对角占优矩阵}{diagonally dominant matrix}
	给定方阵$A\in\FF^{n\times n}$,若每个对角元均在此行占优,即$\forall i=1,\ldots,n$
	\begin{equation}
		\abs{A_{ii}}>\sum_{j\neq i}\abs{A_{ij}},
	\end{equation}
	则称$A$是(严格)对角占优的(strict diagonally dominant);若上式不等号仅为$\geq$,则称$A$是弱对角占优的(weak \textasciitilde)。
\end{definition}

\begin{theorem}
	{对角占优矩阵可逆}{}
	若方阵$A$严格对角占优,则$A$可逆。
\end{theorem}

\begin{proof}
	若$A$不可逆,则$\exists x\neq 0$使得$Ax=0$。令$k=\arg\max_i\abs{x_i}$,即$\abs{x_k}=\max_i\abs{x_i}$。考虑$Ax$的第$k$行:
	\[
		\sum_{i=1}^nA_{ki}x_i=0,\iff A_{kk}x_k=-\sum_{i\neq k}A_{ki}x_i,
	\]
	由三角不等式
	\[
		\abs{A_{kk}}\abs{x_k}=\abs{\sum_{i\neq k}A_{ki}x_i}\leq\sum_{i\neq k}\abs{A_{ki}}\abs{x_i}\leq\sum_{i\neq k}\abs{A_{ki}}\abs{x_k},
	\]
	与$A$对角占优矛盾!故$A$可逆。
\end{proof}

\section{LU分解}

\begin{definition}{上/下三角矩阵}{upper/lower triangular matrix}
	上三角矩阵(upper triangular matrix)~$U$是主对角线以下元素都是0的方阵
	\[
		U_{ij}=0,\quad\forall i>j,
	\]
	同理可定义下三角矩阵(lower ...)~$L$满足$L_{ij}=0,\enspace\forall i<j.$
\end{definition}
不难注意到,倍加矩阵和逆矩阵都同时是上/下三角矩阵。这是LU分解的基础。
\begin{theorem}{LU分解}{LU decomposition}
	对于方阵$A$来说,$A$的LU分解(lower-upper decomposition/factorization)是将$A$分解成一个下三角矩阵$L$和一个上三角矩阵$U$的乘积:
	\begin{equation}
		A=LU,
	\end{equation}
	有时需要再乘上一个置换矩阵$PA=LU$,称为LUP分解(LU factorization with partial pivoting)。
	% 如果$A$化成行阶梯矩阵$U$的过程中没有置换,则$A$有一个LU分解;反之,则存在一个置换矩阵$P$,使得$PA$有一个LU分解。
\end{theorem}
\begin{proof}
	$U$是和$A$等价的行阶梯矩阵,$U$是上三角的。
	如果$A$化成行阶梯矩阵$U$的过程中没有置换,则从$A$到$U$的过程中,我们只需消去主元下面的元素:
	\[
		E_k\cdots E_2E_1A=U.
	\]
	由于$E_i$及其逆$E_i\iv$都是下三角的,故
	\[
		L=E_1\iv E_2\iv\cdots E_k\iv,
	\]
	也是下三角的,%进而$A=LU$;
	$A=LU$;反之,则存在一个置换矩阵$P$,使得$PA=LU$。
\end{proof}

\begin{remark}
	LU分解可以被视为Gauss消元法的矩阵形式。在数值计算上,LU分解经常被用来解线性方程组,且在求逆矩阵和计算行列式中都是一个关键的步骤。
\end{remark}

\begin{theorem}
	{LDU分解}{LDU decomposition}
	LDU分解(lower-diagonal-upper decomposition)是将矩阵写成
	\[
		A=LDU,
	\]
	其中$D$是对角矩阵,$L,U$是单位下/上三角的(unitriangular),即其对角元均为1。
\end{theorem}

