\newcommand*{\tv}{\mathrm t}  % transverse
\newcommand*{\lt}{\mathrm l}  % longitudinal
\newcommand*{\Db}{\mathrm D}
\newcommand*{\Fm}{\mathrm F}

\chapter{Bose统计和Fermi统计}

不同于理想气体,实际气体不满足非简并条件,$\e\alpha\not\gg 1$,服从Bose分布和Fermi分布:
\begin{equation}
	a_i=\frac{\omega_i}{\e{\alpha+\beta\varepsilon_i}\pm 1},
\end{equation}
约定
$\pm$号的($+$)对应Fermi分布,($-$)对应Bose分布。
\begin{definition}{巨配分函数}{Grand Partition Function}
	定义巨配分函数的对数
	\begin{align}
		\ln\Xi(\alpha,\beta,y):=\pm\sum\omega_i\ln\kh{1\pm\e{-\alpha-\beta\varepsilon_i}},
	\end{align}
\end{definition}
可得
\begin{gather}
	N=-\pv{\ln\Xi}{\alpha};\\ % =\sum\frac{\omega_i}{\e{\alpha+\beta\varepsilon_i}\pm 1}
	E=-\pv{\ln\Xi}{\beta}.
\end{gather}
物态方程
\begin{align}
	Y_k=-\frac1\beta\pv{\ln\Xi}{y_k}.
\end{align}

再来确定熵,由
\[
	\d E=T\d S+\sum Y_k\d y_k+\mu\d N,
\]
于是
\[
	T\d S=-\d\kh{\pv{\ln\Xi}\beta}+\frac1\beta\sum\pv{\ln\Xi}{y_k}\d y_k-\mu\d N.
\]
利用
\[
	\d\ln\Xi=\pv{\ln\Xi}\alpha\d\alpha+\pv{\ln\Xi}\beta\d\beta+\sum\pv{\ln\Xi}{y_k}\d y_k
\]
消去求和项,可得
\[
	T\d S=\frac1\beta\d\kh{\ln\Xi-\alpha\pv{\ln\Xi}\alpha-\beta\pv{\ln\Xi}\beta}-\kh{\mu+\frac\alpha\beta}\d N.
\]
对封闭系统,$\d N\equiv 0$
\[
	\d S=\frac1{\beta T}\d\kh{\ln\Xi-\alpha\pv{\ln\Xi}\alpha-\beta\pv{\ln\Xi}\beta},
\]
系数对应Boltzmann常数$\kB$,
因而对开放系统
\begin{align}\label{Bose-mu}
	\alpha=-\beta\mu=-\frac\mu{\kB T}.
\end{align}
因此熵
\begin{align}
	S=\kB\kh{\ln\Xi-\alpha\pv{\ln\Xi}\alpha-\beta\pv{\ln\Xi}\beta}{\color[gray]{.8}-\,S'},
\end{align}

另一方面,由Boltzmann关系
\begin{align*}
	S & =\kB\ln\Om[F;B]\simeq\kB\sum\fkh{a_i\ln\kh{\frac{\omega_i}{a_i}\mp 1}\mp\omega_i\ln\kh{1\mp\frac{a_i}{\omega_i}}} \\
	  & =\kB\sum\fkh{a_i\kh{\alpha_\beta\varepsilon_i}\pm\omega_i\ln\kh{1\pm\e{-\alpha-\beta\varepsilon_i}}}                             \\
	  & =\kB\kh{\alpha N+\beta E+\ln\Xi}=\kB\kh{\ln\Xi-\alpha\pv{\ln\Xi}\alpha-\beta\pv{\ln\Xi}\beta}.
\end{align*}
因此$S'=0$。
\section{弱简并理想Bose气体和Fermi气体}
弱简并条件$n\lambda_T^3<1$,宏观量可对$n\lambda_T^3\equiv\e{-\alpha}$展开。\exmref{exm:Monatomic Molecule} 已给出单原子气体平动:
\begin{align}
	g(\varepsilon)\d\varepsilon=g_s\frac{2\pi V}{h^3}(2m)^{3/2}\sqrt\varepsilon\d\varepsilon,
\end{align}
因此弱简并单原子Bose气体和Fermi气体的巨配分函数的对数
\begin{align*}
	\ln\Xi(\alpha,\beta,V) & =\pm\int\zti g(\varepsilon)\ln\kh{1\pm\e{-\alpha-\beta\varepsilon}}\d\varepsilon\\
	&=\pm 2\pi g_sV\kh{\frac{2m}{h^2}}^{3/2}\int\zti\sqrt\varepsilon\ln\kh{1\pm\e{-\alpha-\beta\varepsilon}}\d\varepsilon,
\end{align*}

由于$\e{-\alpha}<1$,用展开式
\[
	\ln(1+ x)=\sum_{n=1}^\infty\frac{x^n}n,\quad\abs x<1,
\]
展开
\begin{align*}
	\int\zti\sqrt\varepsilon\ln\kh{1\pm\e{-\alpha-\beta\varepsilon}}\d\varepsilon=\pm\sum_{n=1}^\infty\frac{(\mp)^{n-1}}n\int\zti\sqrt\varepsilon\e{-n(\alpha+\beta\varepsilon)}\d\varepsilon \\
	=\pm\sum_{n=1}^\infty\frac{(\mp )^{n-1}}n\e{-n\alpha}\frac{\sqrt\pi}{2(n\beta)^{3/2}}=\pm\frac{\sqrt\pi}{2\beta^{3/2}}f(\alpha).
\end{align*}
其中
\[
	f(\alpha)=\sum_{n=1}^\infty(\mp)^{n-1}n^{-5/2}\e{-n\alpha}=\e{-\alpha}\mp 2^{-5/2}\e{-2\alpha}+\cdots.
\]
故
\begin{align}
	\ln\Xi=\pm g_sV\kh{\frac{2\pi m}{h^2\beta}}^{3/2}f(\alpha)=\frac{g_sV}{\lambda_T^3}f(\alpha).
\end{align}

反解出$\alpha$
\[
	N=-\pv{\ln\Xi}\alpha=-\frac{g_sV}{\lambda_T^3}f'(\alpha),
\]
因此
\[
	\xi:=\frac{n\lambda_T^3}{g_s}=-f'(\alpha)=\e{-\alpha}\mp 2^{-3/2}\e{-2\alpha}+\cdots
\]
进而
\(\e{-\alpha}=\xi\pm{2^{-3/2}}\xi^2+\cdots,\)
%\[\alpha=-\ln\frac{n\lambda_T^3}{g_s}\mp\e{-3/2}\frac{n\lambda_T^3}{g_s}+\cdots\]
\[
	f(\alpha)=\xi\pm 2^{-5/2}\xi^2+\cdots.
\]
宏观量
\begin{align}\notag
	E & =-\pv{\ln\Xi}\beta=-\ln\Xi\pv{\ln\ln\Xi}\beta=\frac32\frac{\ln\Xi}\beta \\
	  & =\frac32N\kB T\kh{1\pm 2^{-5/2}\xi+\cdots}.
\end{align}
比热
\begin{align}
	C_V=\kh{\pv ET}_V=\frac32N\kB\kh{1\mp 2^{-7/2}\xi+\cdots}.
\end{align}
物态方程
\begin{align}
	p=\frac1\beta\pv{\ln\Xi}V=\frac{\ln\Xi}{\beta V}=\frac23\frac EV=\frac{N\kB T}V\kh{1\pm 2^{-5/2}\xi+\cdots}.
\end{align}
熵
\begin{align}\notag
	S&=\kB\kh{\ln\Xi-\alpha\pv{\ln\Xi}\alpha-\beta\pv{\ln\Xi}\beta}=\kB\kh{\frac53\beta E+N\alpha}\\
	&=N\kB\fkh{\kh{\frac52-\ln\xi}\pm 2^{-7/2}\xi+\cdots}.
\end{align}
\paragraph{讨论}弱简并条件($n\lambda_T^3<1$)下,$E,p,S$:Fermi $>$半经典$>$ Bose;$C_V$反之。而强简并条件下,Bose气体和Fermi气体性质完全不同。
\section{Bose-Einstein凝聚}
Bose气体的化学势满足
\[
	a_i=\frac{\omega_i}{\e{\beta(\varepsilon_i-\mu)}-1}.
\]
$a_i\geqslant 0$,故$\varepsilon_i\geqslant \mu$。
取$\varepsilon_0=0$,则$\mu\leqslant 0$
\[
	N=\sum_i\frac{\omega_i}{\e{\beta(\varepsilon_i-\mu)}-1}.
\]
随着温度的降低,化学势增加。直到相变点$T_\crt$,$\mu=0$.

计算$T_\crt$,单原子分子能量准连续
\begin{align*}
	N&=\int\zti\frac{g(\varepsilon)\d\varepsilon}{\e{\beta_\crt\varepsilon}-1}=2\pi g_sV\kh{\frac{2m}{h^2}}^{3/2}\int\zti\frac{\sqrt\varepsilon\d\varepsilon}{\e{\beta_\crt\varepsilon}-1}\\
	&=2\pi g_sV\kh{\frac{2m}{h^2\beta_\crt}}^{3/2}\cdot\Gamma\kh{\frac32}\zeta\kh{\frac32}.
\end{align*}
由
% \[\int\zti\frac{x^{n-1}}{\e x-1}\d x=\Gamma(n)\zeta(n).\]
$\Gamma(3/2)=\sqrt\pi/2,\;\zeta(3/2)=2.612$可得
\begin{align}\label{Bose-TC}
	T_\crt=\frac{h^2}{2\pi m\kB}\kh{\frac n{2.612g_s}}^{2/3}.
\end{align}
$T\to T_\crt$时,$\mu\to0$,基态上的粒子数显著增加;另一方面,准连续近似时$g(\varepsilon)\propto\sqrt\varepsilon$忽略了$\varepsilon=0$态。

故激发态中应将$N$分为基态$N_0$和激发态$N_+$两部分,$N_+$部分推导与之前相同
\[
	\frac{N_+}N=\kh{\frac{\beta_\crt}\beta}^{3/2}=\kh{\frac T{T_\crt}}^{3/2}.
\]
因此基态
\begin{align}
	N_0=N\fkh{1-\kh{\frac T{T_\crt}}^{3/2}}.
\end{align}
当$T<T_\crt$降低时,$N_0$不断增多;$T\to 0$时$N_0\to N$,越来越多的粒子处于基态,称为\textbf{Bose-Einstein凝聚}。这个凝聚可看做动量空间的凝聚。

\paragraph{凝聚后的宏观现象}$\varepsilon=0$粒子
\[
	E=0,\;p=0,\;G=N\mu=E+pV-TS=0.
\]
对$E$等无贡献,起粒子源作用;宏观量子态。

$\varepsilon>0$粒子的贡献,注意$\alpha=0$
\begin{align}\notag
	\ln\Xi&=-\int\zti g(\varepsilon)\ln\kh{1-\e{-\beta\varepsilon}}\d\varepsilon\\\notag
	%=\frac23CV\beta^{-3/2}\int\zti\frac{x^{3/2}\d x}{\e{-x}-1}
	&=2\pi g_sV\kh{\frac{2m}{h^2\beta}}^{3/2}\cdot\frac23\int\zti\frac{x^{3/2}\d x}{\e x-1}\\
	&=\zeta\kh{\frac52}\cdot g_sV\kh{\frac{2\pi m}{h^2\beta}}^{3/2}.% \cdot\frac{3\sqrt\pi}4\cdot 1.341.
\end{align}
故
\begin{gather}
	E=-\pv{\ln\Xi}\beta=0.770N\kB T\kh{\frac T{T_\crt}}^{3/2},\\
	p=\frac1\beta\pv{\ln\Xi}V\propto m^{3/2}g_sT^{5/2},\\
	S=\kB\kh{\ln\Xi+N\alpha+\beta E}\propto m^{3/2}g_sVT^{3/2},\\
	C_V=\kh{\pv ET}_V=1.926N\kB\kh{\frac T{T_\crt}}^{3/2}.
\end{gather}
\paragraph{讨论:}
\begin{compactenum}
	\item $T\to 0$时,$E,p,S\to0$
	\item $C_V$在相变点前后的变化
	\begin{center}
		\begin{tikzpicture}
			\coor 5{4.5}{T/T_\crt}{C_V/N\kB};
			\draw[thick,dashed](0,2*1.926)node[left]{1.926}--(1,2*1.926)--(1,0)node[below]{1};
			\draw[thick,dashed](0,3)node[left]{1.5}--(5,3);
			\draw[thick,domain=0:1]plot(\x,{2*1.926*\x^1.5});
			\draw[thick,domain=1:5]plot(\x,{3+(2*1.926-3)/(\x^1.5)});
		\end{tikzpicture}
		\captionof{figure}{热容$C_V$随温度的变化}
	\end{center}
	\item $p\vs  V$
		\subitem 半经典极限$pV=N\kB T$;
		\subitem 凝聚时$p\propto T^{3/5}$与$V$无关。
	\item 凝聚体积$V_\crt$,由式\eqref{Bose-TC}知
	\[
	T=\frac{h^2}{2\pi m\kB}\kh{\frac{N/V_\crt}{2.612g_s}}^{2/3}.
\]
	因此 
	\begin{align}
		V_\crt=\frac{N}{2.612g_s}\kh{\frac{h^2}{2\pi m\kB T}}^{3/2}=\frac{N\lambda_T^3}{2.612g_s}.
	\end{align}
\end{compactenum}
由于历史条件,当时还不知道全同多粒子系存在(量子起源的)统计关联:对Bose子是有效吸
引;而Fermi子是有效排斥。因此,即使没有动力学相互作用,仍可在一定条件下由于有效相互
作用而发生凝聚现象。这是一种纯粹量子起源的相变。

实现Bose-Einstein凝聚极其困难,原则上要使气体冷却至
$\lambda_T\geqslant\avg d$,
但大多数情况下,在远高于BEC的$T_\crt$到达以前,已发生液化甚至固化的相变。为了实现原子气体的BEC,必须用极稀薄的气体,且要求
\begin{center}
	二体弹性碰撞的弛豫时间$\ll$形成分子集团的非弹性碰撞的弛豫时间
\end{center}
对于碱金属原子气体,前者$\sim\SI{10}\ms$,而后者有几秒至几分钟。%$\tau_\mathrm{elas}\ll$$\tau_\mathrm{inelas}$

BEC-BCS Crosssover Fermionic condensation.
\paragraph{液He}$T_\crt=\SI{2.17}\K$,$T<T_\crt$时的液He II具有超流性。

$T=T_\crt$时,比热趋于无穷,$C_T\vs T$曲线形似$\lambda$,故称$\lambda$相变。
\section{光子气体}
光子是一种特殊的Bose子,严格来说,光子没有Bose-Einstein凝聚\footnote{广义上来说,赋予光子以质量是可以发生BEC的。}。讨论黑体辐射,$T,V$给定,满足相对论关系
\begin{align}
	\varepsilon=h\nu=cp.
\end{align}
光子间无相互作用,符合理想气体。%光子自旋$s=1$,简并度$g_s=2$;
光子质量为0,因此$\lambda_T\to\infty$,且光子数不守恒,没有$\alpha$
\[
	a_i=\frac{\omega_i}{\e{\beta\varepsilon_i}-1}.
\]
\paragraph{黑体辐射公式}能完全吸收照射到它上面的各种波长的电磁波的物体,称为黑体。当$V$很大时,能量准连续,$(\nu,\nu+\d\nu)$内状态数
\[
	g(\nu)\d\nu=\frac{g_sV}{h^3}4\pi\kh{\frac{h\nu}c}^2\frac{h\d\nu}{c}=\frac{4\pi g_sV}{c^3}\nu^2\d\nu;
\]
光子数 
\[
	n(\nu)\d\nu=\frac{g(\nu)\d\nu}{\e{\beta h\nu}-1},
\]
光子$g_s=2$,能量 
\begin{align}
	u(\nu)\d\nu=\frac{n(\nu)}Vh\nu\d\nu=\frac{8\pi\nu^2}{c^3}\frac{h\nu}{\e{\beta h\nu}-1}\d\nu.
\end{align}
上式即Planck定律。

低频高温下,$h\nu\ll\kB T$,变为经典的\Rayl-Jeans定律
\[
	u(\nu)\d\nu\simeq\frac{8\pi\nu^2}{c^3}\kB T\d\nu.
\]
高频低温极限,变成Wein定律
\[
	u(\nu)\d\nu\simeq\frac{8\pi h\nu^3}{c^3}\e{-\beta h\nu}\d\nu.
\]

辐射场总能量
\[
	u=\int\zti u(\nu)\d\nu=\frac{8\pi\kB^4}{h^3c^3}\int\zti\frac{x^3\d x}{\e x-1}=\frac{8\pi^5\kB^4}{15h^3c^3}T^4.
\]
辐射通量密度
\begin{align}
	J=\frac c4u=\sigma T^4.
\end{align}
其中$\sigma=\SI{5.6704e-8}{\W\per\m\squared\per\K\squared}.$及Stefan-Boltzmann定律。

若将能量密度按波长分布
\[
	u(\lambda)\d\lambda=\frac{8\pi h}{\lambda^3}\frac{1}{\e{\beta hc/\lambda}-1}\frac c{\lambda^2}\d\lambda.
\]
其极大值满足Wein位移定律
\begin{align}
	\lambda_\mathrm mT=\frac{hc}{4.96\kB}=\SI{2.89777}{\mm\K}.
\end{align}
\paragraph{热力学}
\[
	g(\varepsilon)\d\varepsilon=2\cdot\frac{4\pi V}{h^3}\frac{\varepsilon^2\d\varepsilon}{c^3}.
\]
配分函数
\begin{align*}
	\ln\Xi(\beta,V)&=-\int\zti g(\varepsilon)\ln\kh{1-\e{-\beta\varepsilon}}\d\varepsilon\\
	&=-\frac{8\pi V}{h^3c^3\beta^3}\int\zti x^2\ln(1-\e x)\d x=\frac{8\pi^5V}{45h^3c^3\beta^3}.
\end{align*}

能量
\[
	E=-\pv{\ln\Xi}\beta=\frac{8\pi^5V}{15h^3c^3\beta^4}=:bVT^4.
\]
与前面一致。而比热
\[
	C_V=\kh{\pv ET}_V=4bVT^3,
\]
随着温度上升而增加,因为光子数不守恒。

压强
\[
	p=\frac1\beta\pv{\ln\Xi}V=\frac13\frac EV=\frac13bT^4.
\]
熵等热力学量
\begin{gather*}
	S=\kB\kh{\ln\Xi-\beta E}=4\kB\ln\Xi=\frac43bVT^3;\\
	F=U-TS=-\frac13U;\\
	G=F+pV=0,\implies\mu=0.
\end{gather*}
与光子数不守恒对应。
\section{声子气体}
在Einstein模型中,我们将固体晶格振动简谐近似为独立的简谐振子,频率$\nu$,量子数为$n$的振子激发态相当于产生了$n$个能量为$h\nu$的粒子,称为声子。

声子气体不可分,符合Bose分布,且声子数不守恒
\[
	a_i=\frac{\omega_i}{\e{\beta h\nu_i}-1}.
\]

Einstein模型定量不符,因为忽略了低频振动,而低温下的热激发主要在低频(长波)部分,当波长$\gg$原子间距时,可看做$0-\omega_\Db$的连续谱。


声波分为横波(transverse)和纵波(longitudinal),速度分别为$v_\tv$和$v_\lt$;横波有两种振动方式,纵波只有一种。
\[
	\varepsilon=\hbar\omega,\quad p=\hbar k;\quad \omega=kv.
\]
纵波声子状态数
\[
	\frac V{h^3}\cdot 4\pi p_\lt^2\d p_\lt=\frac V{2\pi^2v_\lt^3}\omega^2\d\omega.
\]
横波同理,故总状态数
\[
	g(\omega)\d\omega=\frac V{2\pi^2}\kh{2v_\tv^{-3}+v_\lt^{-3}}\omega^2\d\omega=:B\omega^2\d\omega.
\]
由
\[
	3	N=\int_0^{\omega_\Db}g(\omega)\d\omega=\frac B3\omega_\Db^3,\implies B=\frac{9N}{\omega_\Db^3}.
\]
可得
\begin{align*}
	g(\omega)=\begin{cases}
		9N\omega^2/\omega_\Db^3,&0\leqslant\omega\leqslant\omega_\Db\\
		0,&\omega>\omega_\Db
	\end{cases}
\end{align*}
能量 
\begin{align*}
	E&=E_0+\int\zti\frac{ g(\omega)\hbar\omega}{\e{\beta\hbar\omega}-1}\d\omega=E_0+\frac{9N\hbar}{\omega_\Db^3}\int_0^{\omega_\Db}\frac{\omega^3\d\omega}{\e{\beta\hbar\omega}-1}
\end{align*}
取Debye温度
\[
	\theta_\Db:=\frac{\hbar\omega_\Db}\kB\sim\SI{200}\K
\]
并取$y=\theta_\Db/T=\beta\hbar\omega_\Db$
\begin{gather}
	E=E_0+3N\kB T\Debye(y).\\
	C_V=3N\kB\fkh{4\Debye(y)-\frac{3y}{\e{y}-1}}.
\end{gather}
其中Debye函数
\[
	\Debye(y)=\frac 3{y^3}\int_0^{y}\frac{x^3\d x}{\e x-1}.
\]

高温极限$y\ll 1$
\begin{align*}
	\Debye(y)=\frac3{y^3}\int_0^yx^2-\frac{x^3}2+\bigo(x^4)\d x=1-\frac38y+\bigo(y^2).
\end{align*}
\[
	E\simeq E_0+3N\kB T,\quad C_V\simeq 3N\kB.
\]
低温极限$y\gg 1$,可认为
\begin{gather}
	\notag
	\Debye(y)=\frac3{y^3}\int\zti\frac{x^3\d x}{\e x-1}=\frac{\pi^4}{5y^3}.\\
	\label{eqn:CV-Debye}
	C_V=3N\kB\frac{4\pi^4}5\kh{\frac T{\theta_\Db}}^3\propto T^3.
\end{gather}
与试验符合。
\begin{compactenum}
	\item 固体中原子作用强,不能直接用近独立粒子统计。$T$较低时,简谐近似成立——原子集体振动的简正模式。
	相互独立:近独立的理想声子气体。
	\item 声子是准粒子,与振动激发态等效的粒子,有能量、动量等,
	%但不同于电子等,
	只存在于固体中,$\varepsilon$与$p$的关系(色散关系)可不同于普通粒子。
	\item 实际固体比热:金属、自由电子气贡献。

	化合物的分子间振动为声频,适用Debye模型;分子内振动为光频,适用Einstein模型。
\end{compactenum}
\section{Fermi气体}
讨论简并费米气体的低温性质,$n\lambda_T^3\geqslant 1$,相互作用弱。
\[
	a_i=\frac{\omega_i}{\e{\alpha+\beta\varepsilon_i}+1}.
\]
能级$\varepsilon_i$的每个量子态上的平均粒子数
\[
	f_i:=\frac{a_i}{\omega_i}=\frac1{\e{\alpha+\beta\varepsilon_i}+1}.
\]
\paragraph{完全Fermi气}由Pauli原理,粒子不能都处于$\varepsilon=0$态,但尽可能低,即存在$\varepsilon_\Fm$:当$\varepsilon<\varepsilon_\Fm$时,各量子态各有一个粒子;而$\varepsilon>\varepsilon_\Fm$时,态无粒子
\[
	\lim_{T\to0}f_i=\lim_{T\to0}\frac1{\e{(\varepsilon-\mu)/\kB T}+1}=\begin{cases}
	1,&\varepsilon<\mu(T=0)\equiv\varepsilon_\Fm\\0,&\varepsilon>\varepsilon_\Fm
\end{cases}
\]

单原子为例,能量准连续
\[
	g(\varepsilon)\d\varepsilon=2\pi g_s\kh{\frac{2m}{h^2}}^{3/2}V\sqrt\varepsilon\d\varepsilon=:CV\sqrt\varepsilon\d\varepsilon.
\]
有
\[
	N=\int_0^{\varepsilon_\Fm}g(\varepsilon)\d\varepsilon=\frac23CV\varepsilon_\Fm^{3/2}.
\]
故
\begin{align}
	\varepsilon_\Fm=\frac{h^2}{2m}\kh{\frac{3N}{4\pi g_sV}}^{2/3}.
\end{align}
零点能
\begin{align}
	U_0=\int_0^{\varepsilon_\Fm}g(\varepsilon)\varepsilon\d\varepsilon=\frac35N\varepsilon_\Fm.
\end{align}
零点压强
\[
	p_0=-\kh{\pv FV}_T=-\pv{U_0}V=-\dv{\varepsilon_\Fm}V\dv{U_0}{\varepsilon_\Fm}=\frac23\frac{U_0}V.
\]
熵
\[
	S=\kB\ln\Om[F]=0.
\]
\begin{example}{金属中的电子气}{Electron Gas in Metals}
	电子$m_\elc\sim\SI{e-30}\kg$,数密度$\sim\SI{e28}{\per\m\cubed}$, % \tothe{3}
	自旋$g_s=2$,故$\varepsilon_\Fm\sim\SI1\eV$,
	\[
	v_\Fm\sim\sqrt{\frac{2\varepsilon_\Fm}m}\sim\SI[per-mode=symbol]{e6}{\m\per\s}
\]
	压强$p_0\sim\SI{e4}\atm$,这是纯粹的量子效应。
\end{example}
\paragraph{强简并Fermi气}Fermi温度
\[
	T_\Fm:=\frac{\varepsilon_\Fm}\kB.
\]
对于金属电子气,$T_\Fm\sim\SI{e4}\K$。

低温情形$T\ll T_\Fm$,热运动能量小,粒子分布基本不变,只有$\varepsilon_\Fm$附近的粒子可能是跳到高能级态上:
\begin{center}
	\begin{tikzpicture}
		\coor 43{\varepsilon}{f_i};
		\draw[thick,dashed](2,1.25)--(2,0); % node[below]{$\mu$};
		\draw[thick,dashed](1,2.5)--(1,0)node[below]{$\mu-\kB T$};
		\node[left]at(0,2.5){1};
		\draw[thick,domain=0:3]plot(\x,{2.5/(e^(6*\x-12)+1)})node[below]{$\mu+\kB T$};
	\end{tikzpicture}
	\captionof{figure}{强简并Fermi气粒子分布}
\end{center}
定性估计比热$C_V$:相对$T=0$时,能量增量
\[
	\D E\simeq N\frac{\kB T}{\varepsilon_\Fm}\D\varepsilon,\quad\D\varepsilon=\kB T.
\]
比热
\[
	C_V\simeq 2\kB N\frac{\kB T}{\varepsilon_\Fm}\sim T.
\]

单原子,能量准连续,需计算积分
\[
	Q_\ell:=\int\zti f(\varepsilon)\varepsilon^\ell\d\varepsilon.
\]
注意到$f$的特点,可在$\varepsilon=\mu$展开
\begin{align}\notag
	Q_\ell&=\cancel{\edg{\frac{\varepsilon^\ell}{\ell+1}f(\varepsilon)}\zti}-\frac1{\ell+1}\int\zti f'(\varepsilon)\varepsilon^{\ell+1}\d\varepsilon % =:\int\zti f'(\varepsilon)\upsilon(\varepsilon)
	\d\varepsilon\\
	&=\sum_{n=0}^\infty\frac{\upsilon^{(n)}(\mu)}{n!}\int\zti f'(\varepsilon)(\varepsilon-\mu)^n\d\varepsilon,\quad\upsilon(\varepsilon):=-\frac{\varepsilon^{\ell+1}}{\ell+1}.
\end{align}
令$\eta:=\beta(\varepsilon-\mu)$,则
\[
	f(\varepsilon)=\frac1{\e\eta+1},\quad f'(\varepsilon)=-\frac{\beta\e\eta}{(\e\eta+1)^2}
\]
故
\[
	Q_\ell=-\sum_{n=0}^\infty\frac{\upsilon^{(n)}(\mu)}{n!\beta^n}\int_{-\beta\mu}^{+\infty}\frac{\eta^n\e\eta}{(\e\eta+1)^2}\d\eta.
\]

低温下,积分下限$-\beta\mu\to-\infty$
\begin{align*}
	Q_\ell&\simeq-\sum_{n=0}^\infty\frac{\upsilon^{(n)}(\mu)}{n!\beta^n}\int\iti\frac{\eta^n\e\eta}{(\e\eta+1)^2}\d\eta\\
	&=-\fkh{\upsilon(\mu)+\frac{\upsilon''(\mu)}{2\beta^2}\frac{\pi^2}3+\cdots}.
\end{align*}
故
\begin{align}
	N=CVQ_{1/2}&=\frac23CV\mu^{3/2}\fkh{1+\frac{\pi^2}8\alpha^{-2}+\bigo\!\kh{\alpha^{-4}}};\\
	U=CVQ_{3/2}&=\frac25CV\mu^{5/2}\fkh{1+\frac{5\pi^2}8\alpha^{-2}+\bigo\!\kh{\alpha^{-4}}}.
\end{align}
其中$\alpha^{-1}(T)=-\frac1{\beta\mu}=\frac{\kB T}{\mu}$。

巨配分函数$\ln\Xi=\frac23\beta U$,压强$p=\frac1\beta\pv{\ln\Xi}V$,
熵
\begin{align}\notag
	S&=\kB\kh{\ln\Xi-\alpha\pv{\ln\Xi}\alpha-\beta\pv{\ln\Xi}\beta}\\\notag
	&=\kB\frac4{15}CV\beta^{-3/2}(-\alpha)^{5/2}\fkh{0+\frac{5\pi^2}4\alpha^{-2}+\bigo\!\kh{\alpha^{-4}}}\\
	&=\frac{\pi^2}3CV\mu^{1/2}\kB^2T\fkh{1+\bigo\!\kh{\alpha^{-2}}}.
\end{align}

利用
\[
	N=\frac23CV\varepsilon_\Fm^{3/2}.
\]
结合$\varepsilon_\Fm=\mu_0$反解出$\mu$
\[
	\mu=\mu_0\fkh{1-\frac{\pi^2}{12}\alpha^{-2}+\bigo\!\kh{\alpha^{-4}}},
\]
不同于Bose气体,$\mu$可正可负。

宏观量用可观测量表示
\begin{gather}
	U=U_0\fkh{1+\frac{5\pi^2}{12}\kh{\frac T{T_\Fm}}^2+\bigo(T^4)},\\
	\label{eqn:CV-Fermi}
	C_V=N\kB\cdot\frac{\pi^2}{2}\frac T{T_\Fm}\fkh{1+\bigo(T^2)}.
\end{gather}
电子气对金属热容量的贡献首先由Sommerfeld解决。

因此低温下金属比热的实验值是电子气和晶格振动(Debye模型)共同贡献
\[
	C_V\sim \underset{\text{Fermi}}{c_\elc T}+\underset{\text{Debye}}{c_\vb T^3}.
\]
与实验符合得很好。
\begin{example}{电子比热vs.晶格比热}{}
	低温下,式\eqref{eqn:CV-Debye}给出晶格比热和式\eqref{eqn:CV-Fermi}给出电子气比热分别为
	\[
	C_V^\vb=N\kB\frac{12\pi^4}5\kh{\frac T{\theta_\Db}}^3,\quad C_V^\elc=N\kB\frac{\pi^2}2\frac T{T_\Fm}.
	\]
	对铜,$\theta_\Db\sim\SI{300}\K,\;T_\Fm\sim\SI{8e4}\K$,二者比值
	\[
	\frac{C_V^\elc}{C_V^\vb}=\frac5{24\pi^2}\frac T{T_\Fm}\kh{\frac{\theta_\Db}T}^3\sim\frac8{T^2}.
	\]
\end{example}