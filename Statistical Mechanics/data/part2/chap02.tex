\chapter{Boltzmann分布}
\section{理想气体}
理想气体在常温常压下满足非简并条件:
\[
	\e\alpha\gg 1,
\]
符合半经典分布。%分子运动可分为质心平动和内部运动,两种运动独立
%\varepsilon\tot=\varepsilon_\tl+\varepsilon_\i,\quad\omega\tot=\omega_\tl\omega_\i.
分子运动可分为质心的平动(translation)、分子内部的振动(vibration)和转动(rotation),三者是相互独立的。
\paragraph{质心平动}
$\varepsilon_\tl$准连续
\[
	\varepsilon_\tl=\frac1{2m}\kh{p_x^2+p_y^2+p_z^2}.
\]
由\exmref{exm:Monatomic Molecule},配分函数
\begin{align}\notag
	Z_\tl(\beta,V) & =\int\zti g(\varepsilon)\e{-\beta\varepsilon}\d\varepsilon                                                                      \\
	               & =\frac{2\pi V}{h^3}\kh{2m}^{3/2}\int\zti\sqrt\varepsilon\e{-\beta\varepsilon}\d\varepsilon=V\kh{\frac{2\pi m}{h^2\beta}}^{3/2}.
\end{align}
单原子分子只有质心平动
\[
	N=\e{-\alpha}Z_\tl,\implies\e{\alpha}=\frac VN\kh{\frac{2\pi m\kB T}{h^2}}^{3/2}\gg 1.
\]
说明稀薄、高温、质量大的情况下,满足非简并条件,
\begin{gather}
	E_\tl=-N\pv{\ln Z_\tl}\beta=\frac32N\kB T.\\
	C_V^\tl=\kh{\pv ET}_V=\frac32N\kB.
\end{gather}
物态方程
\begin{align}
	p=\frac N\beta\pv{\ln Z_\tl}V=\frac{N\kB T}V.
\end{align}
熵
\begin{align}\notag
	S_\tl & =N\kB\kh{\ln Z_\tl-\beta\pv{\ln Z_\tl}\beta-\ln N+1}                 \\
	      & =\frac52N\kB+N\kB\ln\fkh{\frac VN\kh{\frac{2\pi m\kB T}{h^2}}^{3/2}}
\end{align}
\paragraph{振动}
考虑一维振动,%取约化质量$\mu$,
有
\[
	\varepsilon_\vb=h\nu\kh{n+\frac12},\quad\omega_\vb=1.
\]
能量间距
\[
	\D\varepsilon_\vb=h\nu\sim\SI{0.1}\electronvolt\gg\SI{0.025}\electronvolt.
\]
因此必须考虑能级的分立性。
定义振动的特征温度
\[
	\theta_\vb:=\frac{\D\varepsilon_\vb}\kB=\frac{h\nu}\kB\sim\SI{e3}\K.
\]
则配分函数
% \[Z_\nu=\sum\omega_\nu(n)\e{-\beta\varepsilon_\nu(n)}.\]
\begin{align}
	Z_\vb=\sum_{n=0}^\infty\e{-(n+1/2)\beta h\nu}=\frac{\e{-\theta_\vb/2T}}{1-\e{-\theta_\vb/T}}=\frac12\csch\kh{\frac{\theta_\vb}{2T}}.
\end{align}

振动内能
\begin{align}
	E_\vb=-N\pv{\ln Z_\vb}\beta=Nh\nu\fkh{\frac12+\underset{\text{激发能}}{\frac{\e{-\beta h\nu}}{1-\e{-\beta h\nu}}}}=:N\avg\varepsilon.
\end{align}
其中$\avg\varepsilon$为单个原子的振动能。

比热
\begin{align}
	C_V^\vb=\kh{\pv{E_\vb}T}_V=N\kB\Einst\!\kh{\frac{\theta_\vb}T},
\end{align}
其中Einstein函数
\[
	\Einst(x):=\frac{x^2\e x}{\kh{\e x-1}^2}.
\]
\begin{compactitem}
	\item 
	低温极限,由
	\[
		x\gg 1,\quad\Einst(x)\simeq x^2\e{-x}\ll 1
	\]
	因此比热
	\begin{align}
		C_V^\vb\simeq N\kB\kh{\frac{\theta_\vb}T}^2\e{-\theta_\vb/T}.
	\end{align}
	相对振动而言,常温属于低温范围,故振动自由度对$C_V$贡献很小。
	\item
	高温极限,能量准连续
	\begin{align*}
		Z_\vb & =\int\zti g(\varepsilon)\e{-\beta\varepsilon}\d\varepsilon=\frac{2\pi}{h\omega}\int\zti\e{-\beta\varepsilon}\d\varepsilon=\frac{2\pi}{\beta h\omega}=\frac T{\theta_\vb}.
	\end{align*}
	内能
	\[
		E_\vb=N\kB T.
	\]
	
	或由低温算得的结果,使$\beta\to0$
	\[
		E_\vb\simeq Nh\nu\kh{\frac12+\frac1{\beta h\nu}}\simeq N\kB T.
	\]
\end{compactitem}
比较有三个自由度$p_x,p_y,p_z$的平动能量$E_\tl=3\kB T/2$和有两个自由度的振动能量$E_\vb=\kB T$,可以总结这样一个规律:
\begin{theorem}{能量均分定理}{Equi-Partition Theorem}
	高温下,能量准连续,能量的每个自由度对能量贡献为$\kB T/2$。
\end{theorem}
\paragraph{转动}
以异核双原子分子\footnote{同核需考虑全同性原理。比如正氢(两个氢核自旋平行) $\ell$为奇数;仲氢(两个氢核自旋反平行) $\ell$为偶数。}为例,
\[
	\varepsilon_\rt=\frac{\hbar^2}{2I}\ell(\ell+1),\quad\omega_\rt=2\ell+1.
\]
转动特征温度
\[
	\theta_\rt=\frac{\hbar^2}{2I\kB}\sim\SI{10}\K.
\]
配分函数并没有显式表达式:
\[
	Z_\rt=\sum_{\ell=0}^\infty(2\ell+1)\e{-\ell(\ell+1)\,\theta_\rt/T}.
\]
\begin{compactitem}
	\item
	高温极限下,能量准连续
	\[
		Z_\rt=\int\zti g(\varepsilon)\e{-\beta\varepsilon}\d\varepsilon=\frac{8\pi^2I}{h^2}\cdot\frac1\beta=\frac T{\theta_\rt}.
	\]
	或由$\theta_\rt/T\ll 1$
	\[
		Z_\rt\simeq\int_0^{+\infty}\e{-\ell(\ell+1)\,\theta_\rt/T}\cdot(2\ell+1)\d\ell=\frac T{\theta_\rt}.
	\]
	能量和比热
	\begin{align}
		E_\rt=N\kB T,\quad C_V^\rt=N\kB.
	\end{align}
	\item 
	低温极限,$\theta_\rt/T\gg 1$,$Z_\rt$的项很快趋于0
	\[
		Z_\rt=1+3\e{-2\theta_\rt/T},
	\]
	能量和比热
	\begin{align}
		E_\rt&=6N\kB\theta_\rt\e{-2\theta_\rt/T},\\
		C_V^\rt&=12N\kB\kh{\frac{\theta_\rt}T}^2\e{-2\theta_\rt/T}.
	\end{align}
\end{compactitem}
原子内部结构(电子、原子核等)的自由度对宏观量(特别是$C_V$)无贡献。因为在稳定性、结合能、能级间距上,从分子、原子到原子核越来越稳定,特征温度越来越高,越来越难以被激发。即,结合的紧密的自由能被冻结了,通常不激发。
\sectionstar{Maxwell速度分布律}
考虑气体分子质心平动
\[
	\varepsilon=\frac1{2m}\kh{p_x^2+p_y^2+p_z^2},\quad \e{-\alpha}=\frac NV\kh{\frac{h^2}{2\pi m\kB T}}^{3/2}.
\]
在体积$V$内,动量在$\d p_x\nd p_y\nd p_z$范围内的分子数为
\[
	\frac V{h^3}\e{-\alpha-\beta\varepsilon}\d p_x\nd p_y\nd p_z=N\kh{\frac1{2\pi m\kB T}}^{3/2}\e{-\beta\kh{p_x^2+p_y^2+p_z^2}/2m}\d p_x\nd p_y\nd p_z.
\]
从而单位体积内,速度在$\d v_x\nd v_y\nd v_z$范围内的分子数为
\begin{align}
	\begin{aligned}
		&f(v_x,v_y,x_z)\d v_x\nd v_y\nd v_z=n\kh{\frac m{2\pi\kB T}}^{3/2}\e{-m\kh{v_x^2+v_y^2+v_z^2}/2\kB T}\d v_x\nd v_y\nd v_z.
	\end{aligned}
\end{align}
上式即\textbf{Maxwell速度分布率},满足
\[
	\int\iti\int\iti\int\iti f(v_x,v_y,x_z)\d v_x\nd v_y\nd v_z=n.
\]
将坐标$(v_x,v_y,v_z)$换为球坐标$(v,\theta,\phi)$,并对$\theta,\phi$直接积分,便得到速率的分布:
\begin{align}
	f(v)\d v=4\pi n\kh{\frac m{2\pi\kB T}}^{3/2}\e{-mv^2/2\kB T}v^2\d v.
\end{align}
最概然速率$v_\mathrm m$
\begin{gather}
	\dd v\kh{\e{-mv^2/2\kB T}v^2}=0,\implies
	v_\mathrm m=\sqrt{\frac{2\kB T}m}.
\end{gather}
平均速率$\avg v$
\begin{align}
	\avg v=\pi n\kh{\frac m{2\pi\kB T}}^{3/2}\int\zti v\e{-mv^2/2\kB T}v^2\d v=\sqrt{\frac{8\kB T}{\pi m}}.
\end{align}
方均根速率$v_\mathrm s$
\begin{align}\notag
	\avg{v^2}&=\pi n\kh{\frac m{2\pi\kB T}}^{3/2}\int\zti v^2\e{-mv^2/2\kB T}v^2\d v=\frac{3\kB T}m;\\
	v_\mathrm s&=\sqrt{\avg{v^2}}=\sqrt{\frac{3\kB T}m}.
\end{align}

单位时间碰撞单位面积器壁上的分子数
\begin{align}\notag
	\Gamma&=\int\iti\int\iti\int\zti f(v_x,v_y,x_z) v_x\d v_x\nd v_y\nd v_z\\
	&=n\kh{\frac m{2\pi\kB T}}^{1/2}\int\zti\e{-mv_x^2/2\kB T}v_x\d v_x=n\kh{\frac{\kB T}{2\pi m}}^{1/2}=\frac14n\avg v.
\end{align}
\section{Einstein模型}
讨论固体晶格振动对$C_V$的贡献。
\paragraph{简谐近似}
晶格间强耦合,当$T$不太高时,振幅小,在平衡位置展开
\iffalse
	\[
	\varPhi(x_1,x_2,\ldots,x_{3N})=\edg\phi_{x_i=0}+\sum\edg{\pv\phi{x_i}}_{x_i=0}x_i+\frac12\sum\edg{\pw\phi{x_i}{x_j}}_{x_j=0}x_ix_j+\cdots
\]
	保留至二次项
	\[
	H=\sum\frac12\dot x_i^2+\sum\frac12C_{ij}x_ix_j+\phi_0,\quad C_{ij}:=\edg{\pw\phi{x_i}{x_j}}_{x_j=0}.
\]
	使用正交变换使其对角化
	\begin{align*}
		xCx\tp=q\Lambda q\tp.
	\end{align*}
\fi
\[
	H=\sum_{i=1}^{3N}\kh{\frac{p_i^2}{2m}+2\pi^2m\nu_i^2q_i^2}+\phi_0,\quad p_i=m\dot q_i.
\]
晶格振动约化为$3N$个独立、可区别的简谐振动。
\paragraph{Einstein模型}$\nu_i=\nu$且
\[
	\varepsilon_n=h\nu\kh{n+\frac12},\quad n=0,1,2,\ldots
\]
和前面理想气体的振动类似,配分函数
\[
	z(\beta)=\sum_{n=0}^\infty\e{-\beta\varepsilon_n}=\frac{\e{-\beta h\nu/2}}{1-\e{-\beta h\nu}}.
\]
能量,注意此处是$3N$
\begin{align*}
	E & =-3N\pv{\ln z(\beta)}\beta                          \\
	  & =3Nh\nu\kh{\frac12+\frac1{\e{\beta h\nu}-1}}+\phi_0
	=\frac{3Nh\nu}{\e{\beta h\nu}-1}+E_0.
\end{align*}
热容
\[
	C_V=3N\kB\Einst\!\kh{\frac{\theta_{\Einst}}T},
\]
其中Einstein温度$\theta_{\Einst}\sim\SIrange{100}{300}\K$。
高温极限
\[
	C_V\simeq 3N\kB;
\]
低温极限
\[
	C_V\simeq 3N\kB\kh{\frac{\theta_E}T}^2\e{-\theta_E/T}\to 0.
\]
与实验定性相符,定量不符。因为有低频模,低温时仍能被激发,从而对$C_V$有贡献(Debye)
\section{顺磁物质的磁性}
磁化强度$M$和磁场强度$H$有唯象的关系
\[
	M=\chi H,
\]
其中$\chi$为磁化率。

材料中磁感应强度
\[
	B=\mu_0(H+M)=\mu H,
\]
其中$\mu_0$为真空中磁导率,$\mu=1+\chi$为磁导率。
\begin{theorem}{Curie定律}{Curie's Law}
	顺磁体$\chi>0$,且
	\begin{align}
		\chi\propto\frac CT.
	\end{align}
\end{theorem}
顺磁物质的分子具有恒定的磁矩
\[
	\vec\mu=\muB g\vec J,\qquad\text{Bohr\;磁子\;}\muB:=\frac{e\hbar}{2m_\elc}
\]
其中\Lande 因子\footnote{表达式证明可见本人量子力学笔记。}
\[
	g=1+\frac{j(j+1)+s(s+1)-\ell(\ell+1)}{2j(j+1)}.
\]

单个磁性粒子能级
\[
	\varepsilon_{m_i}=-\vec\mu\cdot\vec B=-\muB g\vec J\cdot\vec B=-\mu_0\muB gHm_i,
\]
其中$J$在$H$方向的投影$m_i=-j,-j+1,\ldots,j$。

定义$a:=\beta\mu_0\muB gH$,配分函数
\begin{align}\notag
	Z(\beta,H) & =\sum_{m_i=-j}^j\e{-\beta\varepsilon_{m_i}}=\frac{\e{aj}-\e{-a(j+1)}}{1-\e{-a}} \\
	           & =\division{\sinh\fkh{a\kh{j+\frac12}}}{\sinh\frac a2}
\end{align}

处于$m_i$态的粒子数
\[
	N(m_i)=\e{-\alpha-\beta\varepsilon_{m_i}}=\frac NZ\e{-\beta\varepsilon_{m_i}}.
\]
总粒子数
\[
	N=\sum N(m_i)=\e{-\alpha}Z(\beta,H).
\]
磁化强度
\[
	M=\sum N(m_i)\muB gm_i=\frac N\beta\pv{\ln Z}{\mu_0H}=N\muB gj\Brill_j(a).
\]
其中Brillouin函数
\[
	\Brill_j(a)=\frac1j\hkh{\kh{j+\frac12}\coth\fkh{a\kh{j+\frac12}}-\frac12\coth\frac a2}.
\]

磁化强度密度
\begin{align}
	m=\frac MV=n\muB gj\Brill_j(a). % \equiv\frac n\beta\pv{\ln Z}{\mu_0H}.
\end{align}
高温弱场极限$a\ll 1$,由
\[
	\coth x=\frac1x+\frac x3+\cdots,\quad x\ll 1
\]
故$\Brill_j(a)\simeq\frac a3(j+1)$
因此
\[
	m=\frac13n\mu_0\kh{\muB g}^2j(j+1)\beta H\propto\frac HT.
\]
即Curie定律~\ref{thm:Curie's Law}。

低温强场极限$a\gg 1$,$\Brill_j(a)\simeq 1$
\[
	m\simeq n\muB gj.
\]
能量和比热
\begin{gather}
	E=-N\pv{\ln Z}\beta=-N\mu_0\muB gj\Brill_j(a)
	C_B=\kh{\pv ET}_B.
\end{gather}

\paragraph{绝热退磁}考虑Helmholtz自由能
\begin{align}
	F=-\frac N\beta\ln Z=:\frac1\beta\phi(\beta H).
\end{align}
熵
\begin{align}
	S=-\pv FT=\kB\Bigl[-\phi(\beta H)+\beta H\phi'(\beta H)\Bigr]
\end{align}
绝热情况,固定$S$,$\beta H=\const$,当$H$下降时,$T$下降
\[
	T_\mathrm f=T_\mathrm i\frac{H_\mathrm f}{H_\mathrm i}.
\]
$S(H=0)$随温度变化小的物质Ce$_2$Mg$_3($NO$_3)_{12}\cdot($H$_2$O$)_{24}$。
\section{负绝对温度}
1951年,Pursell和Pound在很纯的LiF晶体的核自旋系统中实现了负绝对温度的状态。1956年,Ramsey提出了有关负绝对温度的热力学与统计理论。根据
\[
	\frac1T=\pv SE,
\]
一般情况下,内能越高,可能的微观状态数愈多。但也有例外,这使负绝对温度的实现成为可能。
\paragraph{核自旋系统}孤立系统,以粒子数$N$能量$E$和外磁场$B$为参量。在外场中
\[
	\D E=\pm\frac{e\hbar}{2M}B=:\pm\varepsilon.
\]
记能量为$\pm\varepsilon$的核磁矩数为$M_\pm$,则系统的粒子数和能量有
\begin{align*}
	\begin{cases}
		N=N_++N_- \\
		E=(N_+-N_-)\varepsilon
	\end{cases}\implies
	N_\pm=\frac N2\kh{1\pm\frac E{N\varepsilon}}.
\end{align*}
熵
\begin{align*}
	S & =\kB\ln\frac{N!}{N_+!N_-!}\simeq\kB\kh{N\ln N-N_+\ln N_+-N_-\ln N_-}.
	  %\\& =N\kB\fkh{\ln 2-\frac12\kh{1+\frac E{N\varepsilon}}\ln\kh{1+\frac E{N\varepsilon}}-\frac12\kh{1-\frac E{N\varepsilon}}\ln\kh{1-\frac E{N\varepsilon}}}
\end{align*}
故温度的倒数
\begin{align}
	\frac 1T=\pv SE=\frac{\kB}{2\varepsilon}\ln\frac{N\varepsilon-E}{N\varepsilon+E}.
\end{align}
因此当$E<0$时,$T>0$;$E>0$时$T<0$。当能量$E$从负转正的过程中,绝对温度的变化为
\[
	+T_0\enspace\longrightarrow\enspace+\infty\enspace\to\enspace -\infty\enspace\longrightarrow\enspace-T_0
\]
实现负绝对温度的条件相当苛刻:
\begin{compactenum}
	\item 系统的能量$E$有上界;
	\item 系统内部实现平衡(系统能与环境隔绝一段时间)。即

	      系统本身达到平衡的弛予时间$\ll$系统与环境达到平衡的弛予时间\footnote{对于LiF晶体,前者为$\SI{e-5}\s$,后者为$\SI{5}\min$。}
\end{compactenum}
只有满足以上条件,划分才有意义。
\begin{center}
	\begin{tikzpicture}
		\draw[thick,-latex](0,0)node[left]{$O$}--(4.5,0)node[right]{$N_+$};
		\draw[thick,-latex](0,-2.5)--(0,2.5)node[left]{$T$};
		% \node[shift={(-135:9pt)}]at(0,0){$O$};
		\draw[thick,domain=.001:1.77]plot(\x,{.5/ln((4-\x)/\x)});
		\draw[thick,domain=2.23:3.999]plot(\x,{.5/ln((4-\x)/\x)});
		\draw[thick,dashed](2,-2.5)--(2,2.5);
		\draw[thick](4,0)node[above]{$N$}--(4,-.06);
	\end{tikzpicture}
	\captionof{figure}{$T\vs N_+$图像}
\end{center}