\chapter{复相系统的热力学性质}
复相系统:由几个物理性质均匀的部分构成,每一个均匀部分称为一相。特别地,化学成分相同,但相不同构成单元复项系统。
\section{粒子数可变系统的热力学方程}
开放系统粒子数可变,设均匀系有$k$个组元,粒子数分别为$N_1,\ldots,N_k$;描述系统时,除几何、力学参量外,需加上化学参量。

以$T,p,\hkh{N_i}$为自变量,适用Gibbs自由能,$G$是广延量
\[
	G(T,p,\lambda\hkh{N_i})=\lambda G(T,p,\hkh{N_i}).
\]
由
\begin{theorem}{Euler定理}{Euler's Theorem}
	函数$f(x_1,\ldots,x_k)$是$m$阶齐次函数,若$\forall\lambda\leqslant 0$
	\[
		f(\lambda x_1,\ldots,\lambda x_k)=\lambda^mf(x_1,\ldots,x_k).
	\]
	在两边对$\lambda$求导,得到
	\[
		\sum_{i=1}^kx_i\pv f{x_i}=mf.
	\]
\end{theorem}
继而
\[
	G=\sum_{i=1}^kN_i\kh{\pv G{N_i}}_{\neq N_i}=:\sum_{i=1}^kN_i\mu_i.
\]
化学势$\mu_i$代表仅增加一个$i$组元粒子引起的$G$变化。摩尔Gibbs函数就是化学势$\mu$;孤立单元复相系,两相$\alpha$与$\beta$平衡的条件为化学势相同$\mu_\alpha=\mu_\beta$。
\paragraph{化学反应}
考虑化学反应,设各化学计量数为$\nu_i$
\[
	0\rightleftharpoons\sum_{i=1}^r\nu_i{\mathrm A}_i.
\]
在恒温恒压的条件下,平衡时Gibbs自由能%\footnote{在混合系统中,总势一般不是各组分简单相加。}
最小
\begin{align}
	\d G=\sum_{i=1}^r\mu_i\d N_i=\d n\cdot\sum_{i=1}^r\nu_i\mu_i=0.
\end{align}
应当注意:$\mu_i$均来自于混合系统的Gibbs自由能。

非平衡时,由$\vd G<0$,可得$\textstyle\sum\nu_i\mu_i>0$时,$\d n<0$,平衡逆向进行。
\iffalse
考虑一个由理想气体参与的化学反应,理想气体的化学势
\[
	\mu_i=RT\ln p_i+\const(T).
\]
带入上式可得,有
\[
	\sum_{i=1}^r\nu_iRT\ln p_i=\const(T).
\]
%\[\prod p_i^{\nu_i}=-\exp\kh{\sum\nu_i\phi_i}.\]
又$p=RT\zkh{{\rm A}_i}$,%当$T$恒定时
\begin{align}
	\prod_{i=1}^r\zkh{{\rm A}_i}^{\nu_i}=:K(T).
\end{align}
%=(RT)^{-\sum\nu_i}\exp\kh{\sum\nu_i\phi_i}
$K$称为化学平衡常数,是$T$的函数。
\fi
\section{相变热力学}
前面已经提到,在有I和II两相共存的相变过程中,相变平衡的条件为$\mu_\mathrm I=\mu_\mathrm{II}$。

\paragraph{Gibbs相律}有$n$个组分的系统在$r$个相共存时,系统$T,p,\mu_1,\ldots,\mu_n$共$(n+2)$个强度量,每个相有一个Gibbs-Duhem关系
\[
	S\d T-V\d p+\sum_{i=1}^nN_i\d\mu_i=0.
\]
因此自由度
\begin{align}
	f=n+2-r.
\end{align}
\paragraph{Clapeyron方程}
同种物质两相共存时化学势相同
\begin{center}
	\begin{tikzpicture}
		\coor44Tp;
		\draw[thick,domain=1:3]plot(\x,{e^\x/7+.5});
		\node at(1.5,2.5){II};
		\node at(3,1){I};
	\end{tikzpicture}
	\captionof{figure}{I为高温相,II为低温相}
\end{center}
共存曲线下,
\[
	\mu_\mathrm I(T,p)=\mu_\mathrm{II}(T,p).
\]
由$\d\mu=-s\d T+v\d p$
\iffalse
	故
	\[
		\pv{\mu_\mathrm I}T\d T+\pv{\mu_\mathrm I}p\d p=\pv{\mu_\mathrm{II}}T\d T+\pv{\mu_\mathrm{II}}p\d p.
	\]
	由Gibbs自由能给出Maxwell关系
	\begin{align*}
		\pw GTN=\pw GNT & \implies\kh{\pv\mu{T}}_N=-\kh{\pv SN}_T=-s, \\
		\pw GpN=\pw GNp & \implies\kh{\pv\mu{p}}_N=\kh{\pv VN}_p=v.
	\end{align*}
\fi
可得
\[
	\kh{s_\mathrm I-s_\mathrm{II}}\d T=\kh{v_\mathrm I-v_\mathrm{II}}\d p.
\]
由I相转变II相中需要吸收的相变潜热
\[
	\ell:=T(s_\mathrm I-s_\mathrm{II}).
\]
得到\begin{theorem}{Clapeyron方程}{Clapeyron equation}
	凝固点(或沸点)随压强的变化:
	\begin{equation}
		\dv Tp=\frac{T(v_\mathrm I-v_\mathrm{II})}\ell.
	\end{equation}
\end{theorem}
\begin{center}
	\begin{tikzpicture}
		\fill[gclr](0,0)--(2.06,1.28)--(4.31,2.4)--(4.31,3.36)--(4.9,3.36)--(4.9,0);
		\fill[sclr](0,0)--(2.06,1.28)--(1.08,3.36)--(0,3.36);
		\fill[lclr](2.06,1.28)--(1.08,3.36)--(4.31,3.36)--(4.31,2.4);
		%\shade[left color=lclr,right color=gclr](1.08,3.36) .. controls (1.14,2.83) and (1.44,1.95) .. (2.06,1.28)--(2.06,1.28) .. controls (3.03,1.44) and (3.71,1.84) .. (4.31,2.4)--(4.31,3.36);
		\draw[thick,fill=sclr](0,0) .. controls (0.76,0.09) and (1.56,0.58) .. (2.06,1.28) ;
		\draw[thick,fill=lclr](2.06,1.28) .. controls (3.03,1.44) and (3.71,1.84) .. (4.31,2.4) ;
		\draw[thick,fill=lclr](2.06,1.28) .. controls (1.44,1.95) and (1.14,2.83) .. (1.08,3.36) ;
		%\draw[dashed](2.06,1.28) .. controls (2.34,1.75) and (2.57,2.47) .. (2.64,3.35) ;
		\draw[dashed,thick](4.31,2.4)--(4.31,3.36);
		\draw[fill](2.06,1.28)circle(.05);
		\draw[fill](4.31,2.4)circle(.05);
		\coor{5.1}{3.6}Tp;
		\draw(2.06,0)node[below]{$T_\mathrm{tr}$}--(2.06,.1);
		\draw(4.31,0)node[below]{$T_\crt$}--(4.31,.1);
		\node at(.8,1.4){$\sfas$};
		\node at(3.1,.7){$\gfas$};
		\node at(2.6,2.6){$\lfas$};
	\end{tikzpicture}
	%\tikzchap 水的$T\vs p$图\\
	\begin{tikzpicture}
		\fill[gclr](0,0)--(2.06,1.28)--(4.31,2.4)--(4.31,3.36)--(4.9,3.36)--(4.9,0);
		\fill[sclr](0,0)--(2.06,1.28)--(2.64,3.35)--(0,3.36);
		\fill[lclr](2.06,1.28)--(2.64,3.36)--(4.31,3.36)--(4.31,2.4);
		\draw[thick,fill=sclr](0,0) .. controls (0.76,0.09) and (1.56,0.58) .. (2.06,1.28) ;
		\draw[thick,fill=lclr](2.06,1.28) .. controls (3.03,1.44) and (3.71,1.84) .. (4.31,2.4) ;
		%\draw[thick](2.06,1.28) .. controls (1.44,1.95) and (1.14,2.83) .. (1.08,3.36) ;
		\draw[thick,fill=sclr](2.06,1.28) .. controls (2.34,1.75) and (2.57,2.47) .. (2.64,3.36) ;
		\draw[dashed,thick](4.31,2.4)--(4.31,3.36);
		\draw[fill](2.06,1.28)circle(.05);
		\draw[fill](4.31,2.4)circle(.05);
		\coor{5.1}{3.6}Tp;
		\draw(2.06,0)node[below]{$T_\mathrm{tr}$}--(2.06,.1);
		\draw(4.31,0)node[below]{$T_\crt$}--(4.31,.1);
		\node at(1,1.8){$\sfas$};
		\node at(3.1,.7){$\gfas$};
		\node at(3.3,2.6){$\lfas$};
	\end{tikzpicture}
	\captionof{figure}{水(左)和一般纯净物(右)的$p\vs T$相图}
\end{center}
对一般的汽-液相变,$v_\gfas\gg v_\lfas$,若气体符合理想气体,则有
\[
	\dv Tp\doteq\frac{Tv_\gfas}\ell=\frac{RT^2}{\ell p}\implies p=p_0\exp\fkh{\frac\ell{R}\kh{\frac1{T_0}-\frac1T}}.
\]
水凝固存在反常膨胀,$\d T/\d p<0.$
\paragraph{气液两相的转变和临界点}
考虑等温线
\begin{center}
	\begin{tikzpicture}
		\fill[lclr](0,0)--(0,5.13)--(1.42,5.13)--(2.09,3.98)--(0.91,0);
		\fill[gclr](1.42,5.13)--(2.09,3.98)--(3,.35)--(5.79,.35)--(5.79,5.13);
		\fill[glclr](0.91,0)--(0.91,.49)--(5.35,.49)--(5.79,0.35)--(5.79,0);
		\draw[dashed,fill=glclr](0.91,0.49) .. controls (0.93,1.72) and (1.27,3.98) .. (2.09,3.98) .. controls (2.91,3.98) and (4.11,1.01) .. (5.35,0.49) ;
		\draw[thick](0.65,5.13) -- (1.13,2.47) -- (3.54,2.47) ;
		\draw[thick](3.54,2.47) .. controls (4.24,1.72) and (4.89,1.4) .. (5.79,1.25) ;
		\draw[thick](1.03,5.13) -- (1.37,3.24) -- (2.99,3.24) ;
		\draw[thick](2.99,3.24) .. controls (3.9,2.4) and (4.82,2.11) .. (5.79,2.08)node[right]{$T<T_\crt$};
		\draw[thick,fill=gclr](1.42,5.13) .. controls (1.49,4.73) and (1.68,3.99) .. (2.09,3.98) .. controls (2.5,3.97) and (4.64,2.88) .. (5.79,2.91)node[right]{$T=T_\crt$};
		\draw[thick](1.8,5.13) .. controls (2.11,4.24) and (3.43,3.97) .. (5.79,3.74)node[right]{$T>T_\crt$};
		%\node[right]at(5.79,4.57){$T\gg T_\crt$};
		\coor6{5.5}Vp;
		\draw[fill](2.09,3.98)circle(.05)node[above]{C};
		\node at(.5,2.5){$\lfas$};
		\node at(2.5,1.2){$\lfas+\gfas$};
		\node at(4,4.5){$\gfas$};
	\end{tikzpicture}
	\captionof{figure}{$p\vs V$相图,等温线}
\end{center}
当$T<T_\crt$时,会存在气液共存区,在共存区中化学势相同
\[
	\mu_\lfas(T,p)=\mu_\gfas(T,p).
\]
由于图中是等温线,所以共存区中等温线垂直于$p$轴;在临界点C有
\begin{align}
	\pv pV=0,\qquad\pv[2]pV=0.
\end{align}
理想气体不存在液化,下面考虑Van der Waals气体。

~
\begin{example}{Van der Waals气体的约化变量}{Reduced Variables of Van der Waals Gas}
	对于临界温度$T_\crt$
	\begin{align}
		\begin{cases}
			\pv pv=-\frac{RT}{(v-b)^2}+\frac{2a}{v^3}=0, \\
			\pv[2]pv=\frac{2RT}{(v-b)^3}-\frac{6a}{v^4}=0.
		\end{cases}\implies\begin{cases}
			v_\crt=3b,              \\
			T_\crt=\frac{8a}{27Rb}.
		\end{cases}.
	\end{align}
	此时
	\begin{align}
		p_\crt=\frac{RT_\crt}{v_\crt-b}-\frac a{v_\crt^2}=\frac a{27b^2}.
	\end{align}
	可以定义约化变量$\tilde T:=T/T_\crt$等,可得
	% \[\tilde p=\frac{8\tilde T}{3\tilde v-1}-\frac3{\tilde v^2}.\]
	\begin{align}
		\kh{\tilde p+\frac3{\tilde v^2}}\kh{\tilde v-\frac13}=\frac83\tilde T.
	\end{align}
\end{example}
气液共存线上气液摩尔比$x$,则
\[
	v=v_\gfas x+v_\lfas(1-x),\implies x=\frac{v-v_\lfas}{v_\gfas-v_\lfas}.
\]
\begin{center}
	\begin{tikzpicture}[scale=.8]
		\draw[dashed,glclr](0.91,0.49) .. controls (0.93,1.72) and (1.27,3.98) .. (2.09,3.98) .. controls (2.91,3.98) and (4.11,1.01) .. (5.35,0.49) ;
		\draw[thick](0.65,5.13) -- (1.13,2.47)node[midway,left]{$\lfas$} -- (3.54,2.47)node[midway,above]{$\lfas+\gfas$};
		\draw[thick](3.54,2.47) .. controls (4.24,1.72) and (4.89,1.4) .. (5.79,1.25)node[midway,above]{$\gfas$};
		\coor6{5.5}vp;
		\draw[dashed](1.13,2.47)--(1.13,0)node[below]{$v_\lfas$};
		\draw[dashed](2,2.47)--(2,0)node[below]{$v$};
		\draw[dashed](3.54,2.47)--(3.54,0)node[below]{$v_\gfas$};
	\end{tikzpicture}
	\captionof{figure}{杠杆原理}
\end{center}

投影到$T\vs V$图上与$p\vs V$图类似,只有$p<p_\crt$时才会有气液共存区,共存区的等压线垂直于$T$轴(对应沸点)。
\section{Landau相变理论}
在超导、磁性等一大类相变中,有一区别不同相的热力学量,称为序参量$\eta$。

无序相序参量$\eta=0$;有序相$\eta\neq 0$,对应对称破缺;$\eta$可以是复数(超导、超流);序参量有一维标量,也可以是二维和三维的。可以与
空间的维数不同。

相变中,Gibbs自由能
\[
	\d G=-S\d T-y\d Y+\sum\mu_i\d N_i=0.
\]
对于固定$Y,T$下实现的过程有$\mu_i$必须相等,但对导数$S,y$无限制。若$S,y$在相变点不连续,则称相变为一级相变;若二阶导数不连续,则称为二级相变。

临界点:连续相变的相变点,临界温度$T_\crt$;临界现象:物质在连续相变临界点邻域的统计热力学行为。

唯象的,Landau自由能在临界点
\begin{align}
	F(T,\eta)=F_0(T)+\frac12a(T)\eta^2+\frac14b(T)\eta^4+\cdots,
\end{align}
由于系统对$\pm\eta$是对称的,展开中不含$\eta$的奇次幂。

求极值
\[
	\pv F\eta=\eta(a+b\eta^2)=0,\quad\pv[2]F\eta=a+3b\eta^2>0.
\]
有三个解
\[
	\eta=0,\pm\sqrt{-\frac ab}.
\]
$\eta=0$对应无序态,$T>T_\crt$;$\eta_\pm$对应有序态,$T<T_\crt$。当$T\to T_\crt$时,序参量$\eta$在$T_\crt$连续地由零转变到非零,即$a(T_\crt)=0$。

$T_\crt$附近泰勒展开
\[
	a(T)=a_0\kh{\frac T{T_\crt}-1},\quad b(T)=b_0.
\]
$T<T_\crt$时,$a(T)<0$,故$b_0>0$。
\[
	\eta=
	\begin{cases}
		0,&T\geqslant T_\crt\\
		\pm\sqrt{\frac{a_0}{b_0}\kh{1-\frac T{T_\crt}}},&T<T_\crt
	\end{cases}
\]
临界指数$\beta=1/2$。

熵是连续的
\begin{align*}
	S=-\pv FT=S_0-\frac{a_0}{2T_\crt}\eta^2=
	\begin{cases}
		S_0,                             & T\geqslant T_\crt \\
		S_0+\frac{a_0^2}{2b_0T_\crt}\kh{\frac T{T_\crt}-1}, & T<T_\crt
	\end{cases}
\end{align*}
比热却不连续了,
\begin{align*}
	C=T\pv ST=
	\begin{cases}
		C_0,                 & T\geqslant T_\crt \\
		C_0+\frac{a_0^2}{2b_0T_\crt^2}T, & T<T_\crt
	\end{cases}
\end{align*}
因此是\textbf{二级相变}。有序相的比热大于无序相的比热,且$T=T_\crt$处比热的突变是有限的,$\alpha=\alpha'=1$。
\subparagraph*{外加场$B$}在弱场$B$下,序参量$m$
\[
	G(m,B)=F_0+\frac12am^2+\frac14bm^2-Bm.
\]
平衡时
\[
	\pv Gm=a\eta+bm^3-B=0.
\]
$T=T_\crt$时,$a=0,\;B=bm^3$故$\delta=3$。

磁化率
\[
	\chi=\mu_0\kh{\pv mB}_T=\frac{\mu_0}{a+3bm^3}=
	\begin{cases}
		\frac{\mu_0}{a_0}\kh{\frac T{T_\crt}-1}^{-1},&T\geqslant T_\crt \\
		\frac{\mu_0}{2a_0}\kh{\frac T{T_\crt}-1}^{-1}, & T<T_\crt
	\end{cases}
\]
$\gamma'=\gamma=1$。

临界指数$\alpha=1,\,\beta=1/2,\,\gamma=1,\,\delta=3$与实验结果有差异,原因是没考虑涨落。