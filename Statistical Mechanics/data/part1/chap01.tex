% \setcounter{section}{-1}

\chapter{热力学基本定律}

%\section*{宏观}

宏观物质可以用很少的量表征。这种特性源于:
\textit{宏观测量与原子时间尺度相比极其缓慢,与原子空间尺度相比十分粗糙。}
宏观体系忽略了系统内部每个粒子的具体运动,正如Anderson所说:\textbf{\textit{More is different.}}
而热力学便是\textit{唯象}地描述多粒子行为的宏观理论。

\section{热力学第零定律}

\begin{definition}{热力学系统}{thermal system}
	热力学系统(thermal system)是大量微观粒子组成的有限宏观体系。
\end{definition}

平衡态指宏观性质不随时间改变的状态。

\begin{theorem}{热力学第零定律:热平衡定律}{thermal equilibrium}
	若系统A和系统B热平衡,且系统A和系统C也热平衡,则B和C热平衡。
\end{theorem}

\begin{corollary}
	互为热平衡的体系有一共同的物理性质,称为温度$T$。
\end{corollary}

\begin{definition}{物态方程}{state equation}
	物态方程(state equation)是温度$T$与其它状态参量间的关系。
\end{definition}

\begin{example}
	{理想气体物态方程}{}
	理想气体(ideal gas)的压强$p$、体积$V$、温度$T$和物质的量$n$之间的关系为:
	\begin{equation}
		pV=nRT.
	\end{equation}
	其中$R=\SI{8.314}{J/K.mol}$是理想气体常数。
\end{example}

\begin{example}
	{Van der Waals气体物态方程}{}
	Van der Waals气体考虑了分子间的相互作用和分子体积:
	\begin{equation}
		\biggkh{p+a\frac{n^2}{V^2}}(V-nb)=nRT.
	\end{equation}
	其中$a$与分子间的相互作用有关,$b$与分子体积有关。
\end{example}

\section{热力学第一定律}

热力学系统在外界影响下,会从一个平衡态过渡到另一个平衡态,在这个过程中的任一时刻,系统的状态都不是平衡态。
但是如果这个过程中的变化速度足够慢,每一瞬时都可以无限接近平衡态,我们就可以当做平衡态去处理这个过程。

\begin{definition}{准静态过程}{quasistatic process}
	准静态过程(quasistatic process)指每一瞬时,系统状态都无限接近平衡态的过程。
\end{definition}

系统的能量包括内能$U$和整体运动能量。对于封闭系统,能量交换有功$W$和热量$Q$两种方式。准静态过程中,
\begin{align}
	\vd W=\sum_iY_i\d y_i,
\end{align}
其中$(Y_i,y_i)$分别是广义力和广义坐标,如$(-p,V),(\mu_0H,M)$等。

\begin{theorem}{热力学第一定律:能量守恒定律}{Energy Conservation Law}
	一个热力学系统的内能增量$\d U$等于外界对它所做的功$\vd W$与外界向它传递的热量$\vd Q$的和:
	\begin{align}
		\d U=\vd W+\vd Q.
	\end{align}
\end{theorem}

\begin{remark}
	如果系统是\textbf{绝热}($\vd Q\equiv 0$)的,我们便可以用机械功$\vd W$测量内能的变化$\D U$,通过指定基准态的内能$U_0$就可以得出任意状态的内能$U$。进而我们可以测量导热系统的传热$\vd Q$。
\end{remark}

\begin{definition}{热容}{heat capacity}
	定义热容(heat capacity)是物质在单位温度变化下所吸收或放出的热量:
	\begin{equation}
		C:=\lim_{\D T\to0}\frac{\D Q}{\D T}.
	\end{equation}
	比热容(specific heat capacity)是单位质量的热容。
\end{definition}

\begin{remark}
	显然,热容与过程相关,可定义等容热容$C_V$和等压热容$C_p$。
\end{remark}

内能标准全微分式:将$U$全微分式中各变量微分前的系数用可测量表达出来。
\begin{example}{静流体系统}{static fluid system}
	以$T,V$为变量
	\[
		\d U=\underset{C_V}{\underline{\kh{\pv UT}_V}}\d T+\kh{\pv UV}_T\d V.
	\]
	已知 
	\begin{align*}
		C_p&=\kh{\frac{\vd Q}{\d T}}_p=\kh{\frac{\d U+p\d V}{\d T}}_p=\kh{\pv UT}_p+p\kh{\pv VT}_p\\
		&=\underset{C_V}{\underline{\kh{\pv UT}_V}}+\underset{\text{target}}{\underline{\kh{\pv UV}_T}}\kh{\pv VT}_p+p\kh{\pv VT}_p.
	\end{align*}
	因此
	\begin{align}
		\d U=C_V\d T+\fkh{(C_p-C_V)\kh{\pv TV}_p-p}\d V.
	\end{align}
\end{example}
\section{热力学第二定律}
\begin{theorem}{热力学第二定律}{Second Law of Thermodynamics}
	宏观的自发过程是不过逆的。
	\begin{compactitem}
		\item Clausius表述:不可能把热量从低温物体传到高温物体,而不引起其它变化。
		\item Kelvin表述:不可能从单一热源吸热,使之完全变成有用功,而不引起其它变化。
	\end{compactitem}
\end{theorem}
\begin{theorem}{Carnot定理}{Carnot's Theorem}
	在相同高、低温热源之间工作的热机中,可逆机的效率最高:
	\begin{align}
		\eta=1-\frac{Q_2}{Q_1}=1-\frac{T_2}{T_1}.
	\end{align}
	可逆机效率只与热源温度有关,与工作物质无关。
\end{theorem}
\paragraph{热力学温标}借助Carnot机可实现绝对温标。
\begin{theorem}{Clausius不等式}{Clausius inequality}
	在热力学循环中,系统热的变化及温度之间的关系:
	\begin{equation}
		\oint\frac{\vd Q}T\leqslant 0.
	\end{equation}
	当且仅当为可逆热机时取等号,此过程定义为可逆过程。
\end{theorem}
进而定义可逆过程中的熵
\begin{align}
	\d S:=\frac{\vd Q}T.
\end{align}
热力学第二定律的熵表述:孤立系统的熵不减,熵是热运动混乱程度的量度。
\begin{example}{熵的计算}{Calculating Entropy}
	将质量相同而温度分别为$T_1$和$T_2$的两杯水在等压下绝热的混合,求熵变。

	\textbf{解:}终态温度$T=(T_1+T_2)/2$,第一杯水的熵变为
	\[
		\D S_1=\int_{T_1}^T\frac{C_p\d T}T=C_p\ln\frac{T_1+T_2}{2T_1},
	\]
	第二杯水的熵变$\D S_2$同理可求,
	总熵增
	\[
		\D S_1+\D S_2=C_p\ln\frac{(T_1+T_2)^2}{4T_1T_2}\geqslant 0.
	\]
	取等号当且仅当$T_1=T_2$。
\end{example}
\section{热力学第三定律}
\begin{theorem}{热力学第三定律}{Third Law of Thermodynamics}
	% Nernst定理:
	$T\to 0$时,等温过程的熵变$\D_TS\to 0$

	Nernst原理:不可能使一个物体冷却到绝对温度的零度。
\end{theorem}