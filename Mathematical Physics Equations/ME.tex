\def\coursename{数理方程}
\def\coursefullname{数理方程与特殊函数}
\def\courseEnglishname{Mathematical Equations and Special Functions}
\def\teachername{姚国武}
\def\beginday{2022/2/24}
\def\endday{2022/6/16}

\documentclass[a4paper, 11pt]{article}

\usepackage[UTF8]{ctex}

\usepackage[T1]{fontenc}								% 字体
\catcode`\。=\active
\newcommand{。}{.} % {\ifmmode\text{.}\else .\fi}
\catcode`\(=\active
\catcode`\)=\active
\newcommand{(}{(}
\newcommand{)}{)}

% \usepackage{zhlineskip}

\usepackage{nicematrix}
% \usepackage{setspace}
% \linespread{1}						% 一倍行距
\setlength{\headheight}{14pt}			% 页眉高度
% \setlength{\lineskip}{0ex}			% 行距
\renewcommand\arraystretch{.82}		% 表格

\usepackage{amssymb, amsmath, amsfonts, amsthm}			% 数学符号,公式,字体,定理环境
\everymath{\displaystyle}			% \textstyle \scriptstyle \scriptscriptstyle
\allowdisplaybreaks[4]      		% 使用行间公式格式
% \makeatletter
% \renewcommand{\maketag@@@}[1]{\hbox{\m@th\normalsize\normalfont#1}}
% \makeatother
\newif\ifcontent\contenttrue		% if 显示目录
\newif\ifparskip\parskipfalse		% if 增加目录后的行距
\newif\ifshowemail\showemailfalse	% if 显示 email
\def\firstandforemost{
	\maketitle
	%\thispagestyle{empty}\clearpage
	\ifcontent
		\renewcommand{\contentsname}{目录}
		\tableofcontents
		\thispagestyle{empty}
		\clearpage
	\fi
	\ifparskip
		\setlength{\parskip}{.8ex}	% 设置额外的段距,目录后
	\fi								% 在 \firstandforemost 前设置 \parskiptrue
	\makenomenclature
	\printnomenclature
	\setcounter{page}{1}
}

\usepackage{mathtools}									% \rcase 环境等

% \usepackage{physics}

\usepackage[]{siunitx}									% 国际制单位
\sisetup{
	inter-unit-product = \ensuremath{{}\cdot{}},
	per-mode = symbol,
	per-mode = reciprocal-positive-first,
	range-units = single,
	separate-uncertainty = true,
	range-phrase = \ifmmode\text{\;-\;}\else\;-\;\fi
}
\DeclareSIUnit\angstrom{\text{Å}}
\DeclareSIUnit\atm{\text{atm}}
% SIunits 额外定义了一个 \square 表示平方,
% 还会把 \cdot 空格加大,真有够无语的 😅

\usepackage{authblk}									% 作者介绍
\ifx \coursefullname\undefined
	\ifx \coursename\undefined
		\def\coursename{笔记}
	\fi
	\def\coursefullname{\coursename}
\fi
\ifx \authorname\undefined
	\def\authorname{Dait}
\fi
\ifx \departmentname\undefined
	\def\departmentname{DEP 00, THU}
\fi
\ifx \emailaddress\undefined
	\def\emailaddress{daiyj20@mails.tsinghua.edu.cn}
\fi
\ifx \beginday\undefined
	\def\beginday{2021}
\fi
\ifx \endday\undefined
	\def\endday{\number\year/\number\month/\number\day}
\fi
\ifx \titleannotation\undefined
	\ifx \teachername\undefined
		\title{\textbf{\coursefullname}}
	\else
		\title{\textbf{\coursefullname}\\\small\textit{主要整理自\teachername 老师讲义}}
	\fi
\else
	\title{\textbf{\coursefullname}\\\small\textit{(\titleannotation)}}
\fi
\newif\ifdefaultauthor\defaultauthortrue
\ifdefaultauthor
	\author{by~\authorname~at~\departmentname}
	\ifshowemail
		\affil{\emailaddress}
	\fi
\fi
\ifx \endday\beginday
	\date{\beginday}
\else
	\date{\beginday~-~\endday}
\fi

\usepackage{hyperref}									% 链接
\ifx \courseEnglishname\undefined
	\def\courseEnglishname{Note}
\fi
\ifx \authorEnglishname\undefined
	\def\authorEnglishname{Dait}
\fi
\hypersetup{
	% dvipdfm								% 表示用 dvipdfm 生成 pdf
	pdftitle={\coursename},
	pdfauthor={\authorname},
	colorlinks=true, breaklinks=true,		% 超链接设置
	linkcolor=black, citecolor=black, urlcolor=blue
}

\usepackage[british]{babel}								% 长单词自动连字符换行
\hyphenation{long-sen-ten-ce}				% 自定义拆分方式

\usepackage{tikz}
\usetikzlibrary{quotes, angles}
\usepackage{pgfplots}
\pgfplotsset{compat=1.17}								% TikZ
\newcommand{\coor}[5][0]{
	\draw[thick,latex-latex](#1,#3)node[left]{$#5$}--(#1,0)node[shift={(-135:7pt)}]{$O$}--(#1+#2,0)node[right]{$#4$}
}			% 坐标轴

\usepackage{enumerate}									% 编号
\usepackage{paralist}
\setlength{\pltopsep}{1ex}
\setlength{\plitemsep}{1ex}
\ifx \eqnrange\undefined
	\numberwithin{equation}{section}
\else
	\numberwithin{equation}{\eqnrange}
\fi

\renewcommand{\thempfootnote}{\Roman{mpfootnote}}
\renewcommand{\thefootnote}{\Roman{footnote}}		% 注释上标 I, II,...
\newcommand{\sectionstar}[1]{
	\section[\hspace{-.8em}*\hspace{.3em}#1]{\hspace{-1em}*\hspace{.5em}#1}
}
\newcommand{\subsectionstar}[1]{					% 带星号的 section 和 subsection
	\subsection{\hspace{-1em}*\hspace{.5em}#1}
}
\newcommand{\subsubsectionstar}[1]{					% 带星号的 section 和 subsection
	\subsubsection{\hspace{-1em}*\hspace{.5em}#1}
}
\newcommand*{\appendiks}{
	\appendix
	\part*{附录}
	\addcontentsline{toc}{part}{附录}
}
\iffalse			% 不清楚
	\newcommand{\varsection}[1]{
		\refstepcounter{section}
		\section*{\thesection\quad #1}
		\addcontentsline{toc}{section}{\makebox[0pt][r]{*}\thesection\quad #1}
	}
\fi

\usepackage{fancyhdr}									% 页眉页脚
\ifx \coursename\undefined
	\def\coursename{笔记}
\fi
\fancyhf{}\pagestyle{fancy}
\fancyhead[L]{\coursename\rightmark}
\fancyhead[R]{by~\authorname}
\fancyfoot[C]{-~\thepage~-} 			%页码

\usepackage{colortbl, booktabs}							% 表

\usepackage{graphicx}
\usepackage{float}
\usepackage{caption}									% 图
\captionsetup{
	margin=20pt, format=hang,
	justification=justified
}
\newcounter{tikzpic}
\def\tikzchap{
	\stepcounter{tikzpic}\\
	\small 图~\thetikzpic\quad
}
\newcounter{linetable}
\newcommand{\tablechap}[1]{
	\stepcounter{linetable}
	{\small 表~\thelinetable\quad #1}\\[1em]
}

\usepackage{tcolorbox}									% 盒子
\tcbuselibrary{theorems, skins, breakable}
\definecolor{MatchaGreen}{HTML}{73C088}		% 抹茶绿B7C6B3
\newtcbtheorem[number within = subsection]{example}{例}{
	enhanced, breakable, sharp corners,
	attach boxed title to top left = {yshifttext = -1mm},
	before skip = 2ex,
	colback = MatchaGreen!5,				% 文本框内的底色
	colframe = MatchaGreen,					% 文本框框沿的颜色
	fonttitle = \bfseries,					% 标题字体用粗体	coltitle 默认 white,
	boxed title style = {
			sharp corners, size = small, colback = MatchaGreen,
		}
}{exm}
\definecolor{MelancholyBlue}{HTML}{9EAABA}	% melancholy: 沮丧
\newcounter{pslt}
\setcounter{pslt}{-1}
\newtcbtheorem[use counter = pslt]{posulate}{假设}{
	enhanced, breakable, sharp corners,
	attach boxed title to top left = {yshifttext = -1mm}, before skip = 2ex,
	colback = MelancholyBlue!5, colframe = MelancholyBlue, fonttitle = \bfseries,
	boxed title style = {
			sharp corners, size = small, colback = MelancholyBlue,
		}
}{psl}
\definecolor{PureBlue}{HTML}{80A3D0}
\newtcbtheorem[number within = subsection]{definition}{定义}{
	enhanced, breakable, sharp corners,
	attach boxed title to top left = {yshifttext = -1mm}, before skip = 2ex,
	colback = PureBlue!5, colframe = PureBlue, fonttitle = \bfseries,
	boxed title style = {
			sharp corners, size = small, colback = PureBlue,
		}
}{dfn}
\definecolor{PeachRed}{HTML}{EA868F}
\newtcbtheorem[number within = subsection]{theorem}{定理}{
	enhanced, breakable, sharp corners,
	attach boxed title to top left = {yshifttext = -1mm}, before skip = 2ex,
	colback = PeachRed!5, colframe = PeachRed, fonttitle = \bfseries,
	boxed title style = {
			sharp corners, size = small, colback = PeachRed,
		}
}{thm}
\definecolor{SchembriumYellow}{HTML}{fbd26a}	% 申博太阳城黄
\newtcbtheorem[number within = section]{method}{方法}{
	enhanced, breakable, sharp corners,
	attach boxed title to top left = {yshifttext = -1mm}, before skip = 2ex,
	colback = SchembriumYellow!5, colframe = SchembriumYellow, fonttitle = \bfseries,
	boxed title style = {
			sharp corners, size = small, colback = SchembriumYellow,
		}
}{mtd}
% 保留颜色
\definecolor{fadedgold}{HTML}{D9CBB0}		% 褪色金
\definecolor{saturatedgold}{HTML}{F0E0C2}	% staurated: 饱和
\definecolor{elegantblue}{HTML}{C4CCD7}		% elegant: 优雅
\definecolor{ivory}{HTML}{F1ECE6}			% 象牙
\definecolor{gloomypruple}{HTML}{CCC1D2}	% 阴沉紫
% \textcolor[HTML]{FFC23A}					% 石板灰

\definecolor{Green}{rgb}{0,.8,0}

\usepackage{imakeidx}								% 索引

\usepackage{nomencl}								% 关键词
%\setlength{\nomitemsep}{0.2cm}							% 设置术语之间的间距
\renewcommand{\nomentryend}{.}							% 设置打印出术语的结尾的字符
\renewcommand{\eqdeclaration}[1]{见公式:(#1)}			% 设置打印见公式的样式
\renewcommand{\pagedeclaration}[1]{见第 (#1) 页}		% 设置打印页的样式
\renewcommand{\nomname}{术语表} 						% 修改术语表标题的名称。

\usepackage{array}
\usepackage{booktabs} % 三线表
\usepackage{multirow}
% 手动排版,尽量杜绝使用

\newcommand{\bs}[1]{\hspace{-#1 pt}}		% 手动减间距	backspace
\newcommand{\bv}[1]{\vspace{-#1 pt}}		% 手动缩行距	backvspace
\def\directlisteqn{\vspace{-1ex}}
\newcommand*{\qqquad}{\qquad\quad}
\newcommand*{\qqqquad}{\qquad\qquad}
\iffalse									% 尽量避免孤行
	\widowpenalty=4000
	\clubpenalty=4000
\fi

% 杂项符号
\let\geq\geqslant
\def\avg{\overline}
\let\ifaoif\iff
\let\iff\relax
\newcommand*{\rqed}{\tag*{$\square$}}								% 靠右 QED
\newcommand*{\halfqed}{\tag*{$\boxdot$}}
\newcommand*{\thus}{\quad\Rightarrow\quad}							% =>
\newcommand*{\iff}{\enspace\Leftrightarrow\enspace}						% <=>	if and only if
\newcommand*{\ifnf}{\quad\Leftrightarrow\quad}						% <=>	if and only if
\newcommand*{\turnto}{\quad\to\quad}
\newcommand*{\normalize}{\quad\overset{\mathrm{normalize}}{-\!\!\!-\!\!\!-\!-\!\!\!\longrightarrow}\quad}
\newcommand*{\vthus}{\\$\Downarrow$\\}
\newcommand*{\viff}{\\$\Updownarrow$\\}
\newcommand*{\vs}{~\text{-}~}
\newcommand{\eg}[1][]{\subparagraph*{例#1:}}
\newcommand*{\prf}{\noindent\textbf{证明:}\quad}
\newcommand{\dpfr}[2]{\displaystyle\frac{#1}{#2}}					% 大分数
\newcommand{\frdp}[2]{\frac{\displaystyle #1}{\displaystyle #2}}
\newcommand{\spark}[1]{\;\textcolor{red}{#1}}
\newcommand*{\verylongrightarrow}{ -\bs4-\bs4-\bs4-\bs4-\bs5\longrightarrow}
\newcommand*{\semilongrightarrow}{-\bs4-\bs5\longrightarrow}

% 简化更常用的希腊字母
\newcommand*{\vf}{\varphi}
\newcommand*{\vF}{\varPhi}
\newcommand*{\vp}{\varPsi}
\newcommand*{\ve}{\varepsilon}
\newcommand*{\vC}{\varTheta}
\newcommand*{\ct}{\theta}			% 还是建议用 @ + Tab 快捷键

% 正体符号
\newcommand*{\cns}{\mathrm{const}}
\newcommand*{\plusc}{{\color{lightgray}\,+\,\cns}}
\newcommand*{\e}{\mathop{}\!\mathrm{e}^}	% e
\let\accenti\i
\renewcommand*{\i}{\mathrm{i}}
\newcommand*{\D}{\Delta}
\newcommand*{\p}{\partial}

\usepackage{bm}											% 粗体 \bm
\newcommand{\hbm}[1][r]{\hat{\bm #1}}	% 应该不会有两个字母的
\newcommand{\ibm}[1]{\,\bm #1}
\newcommand{\uvec}[1]{\mathop{}\!\hat{\bm #1}}

% Using EnglischeSchT script font style
%\newfontfamily{\calti}{EnglischeSchT}
%\newcommand{\mathcalti}[1]{\mbox{\calti{#1}}}
%\newcommand{\mathcaltibf}[1]{\mbox{\bf\calti{#1}}}

\usepackage{mathrsfs}									% 花体 \mathscr
% \usepackage{boondox-cal}								% 小写花体 \mathcal
\newcommand*{\RR}{\mathbb R}
\newcommand*{\CC}{\mathbb C}
\newcommand*{\ZZ}{\mathbb Z}
\newcommand*{\NN}{\mathbb N}
\newcommand*{\sC}{\mathscr C}			% n 阶连续可导函数
\newcommand*{\sR}{\mathscr R}			% 黎曼可积
% 算符用 \mathcal
\newcommand*{\cL}{\mathcal L}			% 表示一般算子
\newcommand{\cl}[1]{\mathcal L\fkh{#1}}
\newcommand{\cli}[1]{\mathcal L^{-1}\!\fkh{#1}}
\newcommand{\cf}[2][\!\,]{\mathcal F_\mathrm{#1}\fkh{#2}}
\newcommand{\cfi}[2][\!\,]{\mathcal F_\mathrm{#1}^{-1}\!\fkh{#2}}
% \newcommand{\cl}[2][0]{\mathcal L\ikh[#1]{#2}}
% \newcommand{\cli}[2][0]{\mathcal L^{-1}\ikh[#1]{#2}}
% \newcommand{\cf}[2][0]{\mathcal F\ikh[#1]{#2}}
% \newcommand{\cfi}[2][0]{\mathcal F^{-1}\ikh[#1]{#2}}

\usepackage{cancel}										% 删除线

\usepackage{xfrac}

% \usepackage{emoji}	需要 LuaTeX

% 导数等
\let\divides\div
\renewcommand*{\div}{\nabla\cdot}
\newcommand*{\curl}{\nabla\times}
\newcommand*{\lap}{\Delta}
\let\accentd\d
\renewcommand*{\d}{\mathop{}\!\mathrm{d}}
\newcommand*{\nd}{\mathrm{d}}
\newcommand*{\vd}{\mathop{}\!\delta}											% δ
\newcommand{\dd}[2][\;\!\!]{\frac{\nd^{#1}}{\nd #2^{#1}}}						% d/dx			我知道 \,\! 很愚蠢,但是 {} 无法在 Math Preview 上预览
\newcommand{\dn}[2]{\frac{\nd^{#1}}{\nd #2^{#1}}}								% d^n/dx^n		\dn2x≡\dd[2]x
\newcommand{\dv}[3][\;\!\!]{\frac{\nd^{#1}#2}{\nd #3^{#1}}}						% df/dx
\newcommand{\du}[3]{\frac{\nd^{#1}#2}{\nd #3^{#1}}}								% d^nf/dx^n		\du2fx≡\dv[2]fx
\newcommand{\pp}[2][\;\!\!]{\frac{\p^{#1}}{\p #2^{#1}}}							% ∂/∂x
\newcommand{\pn}[2]{\frac{\p^{#1}}{\p #2^{#1}}}									% ∂^n/∂x^n		\pn2x≡\pp[2]x
\newcommand{\pv}[3][\;\!\!]{\frac{\p^{#1}#2}{\p #3^{#1}}}						% ∂f/∂x
\newcommand{\pu}[3]{\frac{\p^{#1}#2}{\p #3^{#1}}}								% ∂^nf/∂x^n		\pu2x≡\pv[2]x
\newcommand{\pw}[3]{\frac{\p^2 #1}{\p #2\p #3}}									% ∂^2f/∂x∂y
\newcommand{\pvv}[6]{															% ∂^(m+n)f/∂x^m∂y^n
	\ifnum#4=1
		\ifnum#6=1
			\frac{\p^{#1}#2}{\p #3\p #5}
		\else
			\frac{\p^{#1}#2}{\p #3\p #5^{#6}}
		\fi
	\else
		\ifnum#6=1
			\frac{\p^{#1}#2}{\p #3^{#4}\p #5}
		\else
			\frac{\p^{#1}#2}{\p #3^{#4}\p #5^{#6}}
		\fi
	\fi}
\newcommand{\dvd}[2]{\left.#1\middle\slash #2\right.}							% 斜除

% 积分
\newcommand*{\intt}{\bs2\int\bs8\int}											% ∫∫
\newcommand*{\inttt}{\int\bs8\int\bs8\int}										% ∫∫∫
\newcommand*{\intdt}{\int\bs3\cdot\bs2\cdot\bs2\cdot\bs4\int}					% ∫...∫
\newcommand*{\zti}{_0^{+\infty}}												% _0^+∞
\newcommand*{\iti}{_{-\infty}^{+\infty}}										% _-∞^+∞
\newcommand{\fmto}[3][\infty]{_{#2=#3}^{#1}}

% 括号
\newcommand{\abs}[1]{\left\lvert#1\right\rvert}									% |x| 绝对值
\newcommand{\norm}[1]{\left\lVert#1\right\rVert}								% ||x|| 模
\newcommand{\edg}[1]{\left.#1\right\rvert}										% f|  竖线
\newcommand{\kh}[1]{\left(#1\right)}											% (x) 括号
\newcommand{\bigkh}[1]{\bigl(#1\bigr)}
\newcommand{\Bigkh}[1]{\Bigl(#1\Bigr)}
\newcommand{\biggkh}[1]{\biggl(#1\biggr)}
\newcommand{\fkh}[1]{\left[#1\right]}											% [x] 方括号
\newcommand{\bigfkh}[1]{\bigl[#1\bigr]}
\newcommand{\Bigfkh}[1]{\Bigl[#1\Bigr]}
\newcommand{\biggfkh}[1]{\biggl[#1\biggr]}
\newcommand{\hkh}[1]{\left\{#1\right\}}											% {x} 花括号
\newcommand{\zkh}[1]{\lfloor\bs{4.7}\lceil #1\rceil\bs{4.7}\rfloor}				% [x] 中括号
\newcommand{\floor}[1]{\left\lfloor#1\right\rfloor}
\newcommand{\ceil}[1]{\left\lceil#1\right\rceil}
\newcommand{\set}[2]{\left\{#1\,\middle\vert\,#2\right\}}						% {x|x1,x2,...} 集合
\newcommand{\ave}[1]{\left\langle #1\right\rangle}								% <x> 平均值
\newcommand{\bra}[1]{\left\langle #1\right\vert}								% <ψ| 左矢
\newcommand{\ket}[1]{\left\vert #1\right\rangle}								% |ψ> 右矢
\newcommand{\brkt}[2]{\left\langle #1\middle\vert #2\right\rangle}				% <φ|ψ> 内积
\newcommand{\ktbr}[2]{\left\vert#1\right\rangle \bs3\left\langle #2\right\vert}	% |ψ><φ|
\newcommand{\inp}[2]{\left\langle #1,#2\right\rangle}							% <f,g> 内积

% 数学运算符
\let\Real\Re
\let\Imagine\Im
\let\Re\relax
\let\Im\relax
\DeclareMathOperator{\Re}{Re}					% 
\DeclareMathOperator{\Im}{Im}					% 
\DeclareMathOperator{\sech}{sech}				% 
\DeclareMathOperator{\csch}{csch}				% 
\DeclareMathOperator{\arcsec}{arcsec}			% 
\DeclareMathOperator{\arccot}{arccot}			% 
\DeclareMathOperator{\arccsc}{arccsc}			% 
\DeclareMathOperator{\arsinh}{arsinh}			% 
\DeclareMathOperator{\arcosh}{arcosh}			% 
\DeclareMathOperator{\artanh}{artanh}			% 
\DeclareMathOperator{\sgn}{sgn}					% 符号函数
\DeclareMathOperator{\Li}{Li}					% 
\DeclareMathOperator{\Si}{Si}
\DeclareMathOperator{\Ci}{Ci}
\DeclareMathOperator{\sinc}{sinc}
\DeclareMathOperator{\Heaviside}{H}
\DeclareMathOperator{\arr}{A}					% 排列数
\DeclareMathOperator{\com}{C}					% 组合数
\DeclareMathOperator{\Res}{Res}					% 留数
\DeclareMathOperator{\supp}{supp}				% 支撑集
\DeclareMathOperator{\Int}{Int}					% 内部
\DeclareMathOperator{\Ext}{Ext}					% 外部
\newcommand*{\bigo}{\mathcal O}
\newcommand{\degree}{^\circ}

% 线性代数
% \newif\ifLinearAlgebra\LinearAlgebratrue
% \ifLinearAlgebra
\DeclareMathOperator{\rank}{rank}
\DeclareMathOperator{\id}{id}
\newcommand*{\tp}{^\top}					% AT 转置
\newcommand*{\cj}{^\ast}					% A* 共轭
\newcommand*{\dg}{^\dagger}					% A† 共轭转置
\newcommand*{\iv}{^{-1}}					% A-1
% \fi

% 物理学家
\newcommand*{\Schr}{Schrödinger}
\newcommand*{\Legd}{Legendre}
\newcommand*{\deB}{de Broglie}
\newcommand*{\Rayl}{Rayleigh}
\newcommand*{\Lande}{Landé}

% 粒子
\newcommand*{\elc}{\mathrm e}
\newcommand*{\pton}{\mathrm p}
\newcommand*{\nton}{\mathrm n}
\newcommand*{\mol}{\mathrm m}

% 物理常数
\newcommand*{\NA}{N_{\bs1\mathrm A}}						% Avogadro 常数
\newcommand*{\kB}{k_{\mathrm B}}							% Boltzmann 常数
\newcommand*{\muB}{\mu_\mathrm B}							% Bohr 磁矩

% 
\newcommand*{\Ek}{E_{\mathrm k}}							% 动能
\newcommand*{\eff}{_\mathrm{eff}}							% 有效下标
\newcommand*{\tot}{_\mathrm{tot}}
\newcommand*{\maxi}{_\mathrm{max}}
\newcommand*{\mini}{_\mathrm{min}}
\newcommand*{\lSI}{\tag{SI}}
\newcommand*{\CGS}{\tag{CGS}}								% cm, g, s 制
\newcommand*{\FWHM}{\mathrm{FWHM}}


\newenvironment{equationset}{\left\{\begin{aligned}}{\end{aligned}\right.}

\newcommand*{\Ct}{\varTheta}
\newcommand*{\Dif}{\mathcal D}
\newcommand*{\Par}{\mathcal P}
\newcommand*{\even}{\mathrm e}
\newcommand*{\odd}{\mathrm o}
\newcommand*{\sine}{\mathrm s}
\newcommand*{\cosi}{\mathrm c}
\newcommand*{\Cooo}{\mathscr D}
\newcommand*{\Schwsp}{\mathscr S}
\DeclareMathOperator{\FS}{FS}

\begin{document}
\firstandforemost

\section{偏微分方程的定解问题}
我们先从$\bcancel{\text{三}}$两个具体问题出发推导典型方程.
\paragraph*{弦振动}
弦在自身张力作用下平衡位置在$x$轴,横向位移$u=u(x,t)$,取$\zkh{x_1,x_2}$段进行受力分析
\begin{align*}
	x~\text{轴:}\quad & T_2\cos\theta_2-T_1\cos\theta_2=0,                   \\
	y~\text{轴:}\quad & T_2\sin\theta_2-T_1\sin\theta_2=\rho\vd x\pu 2ut.,
\end{align*}
其中$T$为弦内张力,$\rho$为弦的线密度.

对于$\theta\ll 1$的情况,有$\cos\theta\doteq 1,\sin\theta\doteq\theta,$
\[
	T_1=T_2=:T,
\]
又$\theta=\p u/\p x,$
\[
	T(\theta_2-\theta_1)=T\vd\!\kh{\pv ux}=\rho\vd x\pu 2ut.
\]
设$a^2:=T/\rho$,得到波动方程\nomenclature{$u_t$}{The number of angels per unit area\nomrefeq}
%$\rho$ - 线密度,$T$ - 张力,$g$ - 外力\\
\begin{align}
	\pv[2]ut=a^2\pv[2]ux.
\end{align}
\iffalse
	\begin{example}{Maxwell方程组}{}
		电动力学中
		\begin{align*}
			\nabla\cdot\bm D  & =\rho,              \\
			\nabla\cdot\bm B  & =0,                 \\
			\nabla\times\bm E & =-\pv{\bm B}t,      \\
			\nabla\times\bm H & =\bm J+\pv{\bm D}t.
		\end{align*}
	\end{example}
\fi
\paragraph*{热传导问题}
由Fourier热传导定律,一段时间内流入物体$\Omega$的热量为
\[
	Q=\int_{t_1}^{t_2}\bs5\oint_{\p\Omega}k\pv u{\bm n}\d S\d t=\int_{t_1}^{t_2}\bs5\int_\Omega k\lap u\d V\d t,
\]
其中$k$为介质的热传导系数,$u=u(x,y,z,t)$为各点温度.

另一方面,从比热容$c$的角度看,
\[
	Q=\int_{t_1}^{t_2}\bs5\int_\Omega c\rho\pv ut\d V\d t.
\]
其中$\rho$为物体的体密度.

因为上式对任意$\Omega\times\zkh{t_1,t_2}$均成立,设$a^2=k/c\rho$,得到热传导方程
%$c$ - 比热,$\rho$ - 密度,$k$ - 热传导系数\\
\begin{align}
	\pv ut=a^2\pv[2]ux.
\end{align}
\subsection{定解问题及适定性}
$V\subseteq\mathbb R^n$内$m$阶PDE
\[
	F\kh{x,u,\pv u{x_1},\pv u{x_2},\ldots,\pv u{x_n},\ldots,\frac{\p^mu}{\p x_1^{m_1}\cdots\p x_n^{m_n}}}=0,
\]
其中$m=m_1+m_2+\cdots+m_n,$有各阶偏导数连续的解$u=u(x_1,x_2,\ldots,x_n)$称为\textbf{古典解}.

\textbf{通解}是$m$阶PDE有$m$个任意函数的解;\textbf{特解}是不包含任何任意函数或任意常数的解.

\begin{definition}{定解条件}{Definite Condition}
	确定解中函数的条件称为定解条件
	\begin{compactenum}[I.]
		\item $\edg u_{x=0}=f,\qquad\qquad\edg u_{\p V}=f$ (Dirichlet);
		\item $\edg{u_t}_{x=0}=f,\qquad\qquad\edg{u_t}_{\p V}=f$ (Neumann);
		\item $\kh{u_t+\sigma u}_{x=0}=f,\qquad\kh{\pv un+\sigma u}_{\p V}=f$ (Robin).
	\end{compactenum}
\end{definition}
\textbf{适定解}:存在、唯一且稳定.
\subsection{一阶线性方程的通解法}
\subsubsection*{1.常数变易法}
在微积分中已经学过
\[
	y'+P'y=Q.
\]
有解
\begin{align}\label{Variation of Constant}
	y=\e{-P}\fkh{\int Q\e{P}\d x+\cns}.
\end{align}

对于$u(x,y)$的方程
\[
	u_x+Ax=B.
	\]
\begin{align*}
	\begin{cases}
		\varphi=\exp\kh{-\int A\d x},  \\
		\psi=\int B\varphi^{-1}\d x
	\end{cases}
	\thus u=\varphi\zkh{\psi+g(y)}.
\end{align*}
其中$g\in\sC$.
\subsubsection*{2.变量代换}
对$u=u(x,y)$的方程
\[
	au_x+bu_y+cu=f,
	\]
作变量代换
\begin{align*}
	\begin{cases}
		\xi=\xi(x,y)  \\
		\eta=\eta(x,y)
	\end{cases}
	\thus
	\begin{cases}
		u_x=u_\xi\xi_x+u_\eta\eta_x \\
		u_y=u_\xi\xi_y+u_\eta\eta_y
	\end{cases}
\end{align*}
变成$u=u(\xi,\eta)$的方程
\[
	\kh{a\xi_x+b\xi_y}u_\xi+\kh{a\eta_x+b\eta_y}u_\eta+cu=f.
	\]
使$a\xi_x+b\xi_y=0$,得到\textbf{特征方程}和\textbf{特征曲线}
\begin{align}
	\frac{\d x}a=\frac{\d y}b,\thus\xi(x,y)=\cns.
\end{align}
剩下
\[
	\kh{a\eta_x+b\eta_y}u_\eta+cu=f.
	\]
对$\eta$积分便可求出通解.注意:要$a\eta_x+b\eta_y\neq 0$,应有Jacobi行列式
\begin{align*}
	\det J(\xi,\eta)=\left\lvert\frac{\p(\xi,\eta)}{\p(x,y)}\right\rvert=
	\begin{vmatrix}
		\xi_x  & \xi_x  \\
		\eta_x & \eta_y
	\end{vmatrix}\neq 0.
\end{align*}
\subsection{波动方程的行波解和d'Alembert公式}
一维波动方程
\[
	\pv[2]ut=a^2\pv[2]ux,\quad a>0.
	\]
可分解为
\[
	\kh{\pp t+a\pp x}\kh{\pp t-a\pp x}u=0.
	\]
等价于
\begin{align*}
	\begin{cases}
		\pv vt+a\pv vx=0, \\[1ex]
		\pv ut-a\pv ux=v.
	\end{cases}
	\thus
	\begin{cases}
		\xi=x-at, \\
		\eta=x+at.
	\end{cases}
\end{align*}
方程化为
\[\pw u\eta\xi=0,\thus u=f(\xi)+g(\eta).\]
即\textbf{行波解}
\begin{align}
	u(x,t)=f(x-at)+g(x+at),\quad f,g\in\sC^2.
\end{align}
\subsubsection*{d'Alembert公式}
无限长弦自由振动
\begin{align*}
	\begin{cases}
		u_{tt}=a^2u_{xx},&t>0,\;x\in\mathbb R. \\
		u|_{t=0}=\varphi(x),\quad u_t|_{t=0}=\psi(x).
	\end{cases}
\end{align*}
将初值问题带入通解
\begin{align*}
	u|_{t=0}   & =f(x)+g(x)=\varphi(x),   \\
	u_t|_{t=0} & =-af'(x)+ag'(x)=\psi(x).
\end{align*}
因此
\begin{align}
	u(x,t)=\frac12\zkh{\varphi(x-at)+\varphi(x+at)}+\frac1{2a}\int_{x-at}^{x+at}\psi(\xi)\d\xi.
\end{align}
对于$\varphi\in\sC^2,\psi\in\sC,$解是适定的.

非无限情况,可以延拓至无限.比如端点固定的半无限边界条件:
\begin{align*}
	 & u(0,t)=0,\quad x>0,\;t>0.                   \\
	 & u(x,0)=\varphi(x),\quad u_t(x,0)=\psi(x).
\end{align*}
进行奇延拓
\begin{align*}
	\varphi_\odd(x)=
	\begin{cases}
		\varphi(x),   & x\geqslant 0, \\
		-\varphi(-x), & x<0.
	\end{cases}
	\quad
	\psi_\odd(x)=
	\begin{cases}
		\psi(x),   & x\geqslant 0, \\
		-\psi(-x), & x<0.
	\end{cases}
\end{align*}
问题便适用d'Alembert公式,解
\begin{align*}
	u(x,t) & =\frac12\zkh{\varphi_\odd(x-at)+\varphi_\odd(x+at)}+\frac1{2a}\int_{x-at}^{x+at}\psi_\odd(\xi)\d\xi \\
	       & = 
	\begin{cases}
		\frac12\zkh{\varphi(x-at)+\varphi(x+at)}+\frac1{2a}\int_{x-at}^{x+at}\psi(\xi)\d\xi, & x\geqslant at, \\
		\frac12\zkh{\varphi(x+at)-\varphi(at-x)}+\frac1{2a}\int_{at-x}^{x+at}\psi(\xi)\d\xi, & x<at.
	\end{cases}
\end{align*}
端点自由半无限弦$u_t(0,t)=0$则采用偶延拓;对有界弦,则两端均延拓.

中心对称的球面波$u=u(r)$,采用球坐标
\[\lap=\frac1{r^2}\pp r\kh{r^2\pp r}+\frac1{r^2\sin\theta}\pp \theta\kh{\sin\theta\pp \theta}+\frac1{r^2\sin^2\theta}\pu 2{}\phi.\]
波动方程变为
\[u_{tt}=a^2\kh{u_{rr}+\frac2ru_r}=\frac{a^2}r(ru)_{rr}.\]
设$v=ru,$则可解得
\[u(r)=\frac1r\zkh{f(r-at)+g(r+at)}.\]
掺入边界条件$\varphi(r),\psi(r)$后,作奇延拓即可.

\eg 一般的三维波动方程
\begin{align*}
	\begin{cases}
		u_{tt}=a^2(u_{xx}+u_{yy}+u_{zz}),                     \\
		u|_{t=0}=\varphi(x,y,z),\quad u_t|_{t=0}=\psi(x,y,z).
	\end{cases}
\end{align*}
采用球面平均法,定义球面平均
\[\overline u(r,t;M)=\frac1{4\pi r}\oint_{S_r(M)}u(x,y,z,t)\d S,\]
其中$S_r(M)$表示以$M$为球心,$r$为半径的球面。

略去证明过程,问题转化为中心对称球面波问题
\begin{align*}
	\begin{cases}
		\overline u_{tt}=\frac1r\kh{r\overline u}_{rr},                                \\
		(r\overline u)_{r=0}=0,                                                        \\
		\overline u(r,0)=\overline\varphi(r),\quad\overline u_t(r,0)=\overline\psi(r).
	\end{cases}
\end{align*}
奇延拓解,注意:$(r\varphi)_\odd=r\varphi_\even,$
\begin{align*}
	r\overline u(r,t)= & \frac12\zkh{(r-at)\overline\varphi_\even(r-at)+(r+at)\overline\varphi_\even(r+at)}+\frac1{2a}\int_{r-at}^{r+at}\rho\overline\psi_\even(\rho)\d\rho.
\end{align*}
为得到$u(x,y,z,t),$取极限
\begin{align*}
	u(M,t)=\lim_{r\to 0^+}\overline u(r,t)
	=\dd t\kh{t\overline\varphi(at)}+t\overline\psi(at),
\end{align*}
上式称为Possion公式.

求解三维波动方程的关键在于计算球面平均
\begin{align}\label{3D-Wave}
	u(M,t) & =\pp t\kh{\frac1{4\pi a^2t}\oint_{S_{at}}\varphi\d S}+\frac1{4\pi a^2t}\oint_{S_{at}}\psi\d S\\\notag
	& =\pp t\kh{\frac1{4\pi}\int_0^{2\pi}\bs5\int_0^\pi t\varphi\sin\theta\d\theta\nd\phi}+\frac1{4\pi}\int_0^{2\pi}\bs5\int_0^\pi t\psi\sin\theta\d\theta\nd\phi,
\end{align}
其中采用球坐标。
\iffalse
\begin{align*}
	\varphi\begin{pmatrix}
		\xi \\\eta\\\zeta
	\end{pmatrix}=\varphi\begin{pmatrix}
		x+at\sin\theta\cos\phi \\
		y+at\sin\theta\sin\phi \\
		z+at\cos\theta
	\end{pmatrix}.
\end{align*}
\fi
\eg 二维波动方程
\begin{align*}
	\begin{cases}
		u_{tt}=a^2(u_{xx}+u_{yy}), \\
		u|_{t=0}=\varphi(x,y),\quad u_t|_{t=0}=\psi(x,y).
	\end{cases}
\end{align*}
采用升维法
\begin{gather*}
	U(x,y,z,t)=u(x,y,t),\quad\varPhi(x,y,z)=\varphi(x,y),\\
	U(x,y,z,t)=\pp t\kh{\frac1{4\pi a^2t}\oint_{S_{at}}\varPhi\d S}+\frac1{4\pi a^2t}\oint_{S_{at}}\varPsi\d S
\end{gather*}
由于$U,\varPhi,\varPsi$取值与$z$无关,故积分区域可投影到$xy$平面上。
记$D_r$是以$(x,y)$为圆心,$r$为半径的圆内区域,在$D_r$上
\[\d S=\sqrt{1+\kh{\pv\zeta\xi}^2+\kh{\pv\zeta\eta}^2}\d A=\frac{r\d A}{\sqrt{r^2-(\xi-x)^2-(\eta-y)^2}}.\]
故
\begin{align}\label{2D-Wave}
	\begin{aligned}
		u(x,y,t)&= \frac1{2\pi a}\pp t\int_{D_{at}}\frac{\varphi(\xi,\eta)\d\xi\nd\eta}{\sqrt{a^2t^2-(\xi-x)^2-(\eta-y)^2}} \\
		&\qquad +\frac1{2\pi a}\int_{D_{at}}\frac{\psi(\xi,\eta)\d\xi\nd\eta}{\sqrt{a^2t^2-(\xi-x)^2-(\eta-y)^2}}.
	\end{aligned}
\end{align}
(\ref{3D-Wave})是在球面上积分,而(\ref{2D-Wave})是在圆域上积分。\footnote{这个差别在物理上产生了截然不同的效果:对三维情况,波的传播既有清晰的前阵面,也有清晰的后阵面,可用于传播信号,这称为Huygens原理;对二维情况,波的传播有清晰的前阵面,但没有后阵面,这称为波的弥漫,或说这种波具有后效现象,不适合于传播信号。}

对一维弦振动方程,也可升二维,再将积分投影到$\zkh{x-at,x+at}$上。
记$C_r$是以$(x,y)$为圆心,$r$为半径的圆,在$C_r$上
\begin{gather*}
	\d\ell=\sqrt{1+\kh{\dv\eta\xi}^2}\d\xi=\frac{r\d\xi}{\sqrt{r^2-(\xi-x)^2}}.
\end{gather*}
故
\begin{align*}
	&\quad\int_{D_{at}}\frac{\varPhi(\xi,\eta)\d\xi\nd\eta}{\sqrt{a^2t^2-(\xi-x)^2-(\eta-y)^2}}=\int_0^{at}\oint_{C_r}\frac{\varphi(\xi)}{\sqrt{a^2t^2-r^2}}\d\ell\d r\\
	&=2\int_0^{at}\int_{x-r}^{x+r}\frac{\varphi(\xi)}{\sqrt{a^2t^2-r^2}}\frac{r\d\xi}{\sqrt{r^2-(\xi-x)^2}}\d r\tag{交换次序}\\
	&=\kh{\int_x^{x+at}\bs5\int_{\xi-x}^{at}+\int_{x-at}^x\int_{x-\xi}^{at}}\frac{\varphi(\xi)}{\sqrt{a^2t^2-r^2}}\frac{2r\d r}{\sqrt{r^2-(\xi-x)^2}}\d\xi
\end{align*}
推出d'Alembert公式:
\[u(x,t)=\pp t\kh{\frac1{2a}\int_{x-at}^{x+at}\varphi(\xi)\d\xi}+\frac1{2a}\int_{x-at}^{x+at}\psi(\xi)\d\xi.\]
\subsection{二阶线性偏微分方程及标准型}
两个自变量的二阶其次线性偏微分方程
\[a_{11}u_{xx}+2a_{12}u_{xy}+a_{22}u_{yy}+b_1u_x+b_2u_y+cu=0.\]
进行变量代换
\begin{align*}
	\begin{cases}
		\xi=\xi(x,y) \\
		\eta=\eta(x,y)
	\end{cases}
	\quad\text{其中}~
	\det J(\xi,\eta)\neq 0.
\end{align*}
变成
\[A_{11}u_{\xi\xi}+2A_{12}u_{\xi\eta}+A_{22}u_{\eta\eta}+B_1u_\xi+B_2u_\eta+cu=0.\]
其中
\begin{align*}
	A_{11} & =a_{11}\xi_x^2+2a_{12}\xi_x\xi_y+a_{22}\xi_y^2,                          \\
	A_{12} & =a_{11}\xi_x\eta_x+a_{12}\kh{\xi_x\eta_y+\xi_y\eta_x}+a_{22}\xi_y\eta_y, \\
	A_{22} & =a_{11}\eta_x^2+2 a_{12}\eta_x\eta_y+a_{22}\eta_y^2;                     \\
	B_1    & =a_{11}\xi_{xx}+2a_{12}\xi_{xy}+a_{22}\xi_{yy}+b_1\xi_x+b_2\xi_y,        \\
	B_2    & =a_{11}\eta_{xx}+2a_{12}\eta_{xy}+a_{22}\eta_{yy}+b_1\eta_x+b_2\eta_y,
\end{align*}
用矩阵表达即
\begin{align*}
	\begin{bmatrix}
		A_{11} & A_{12} \\
		A_{12} & A_{22}
	\end{bmatrix}=\begin{bmatrix}
		\xi_x  & \xi_x  \\
		\eta_x & \eta_y
	\end{bmatrix}\begin{bmatrix}
		a_{11} & a_{12} \\
		a_{12} & a_{22}
	\end{bmatrix}\begin{bmatrix}
		\xi_x  & \xi_x  \\
		\eta_x & \eta_y
	\end{bmatrix}\tp.
\end{align*}

考虑使$A_{11}=0$或$A_{22}=0$则有
\[a_{11}\kh{\dv yx}^2-2a_{12}\dv yx+a_{22}=0.\]
其判别式
\[\Delta=a_{12}^2-a_{11}a_{22}.\]
\paragraph*{$\Delta>0$,双曲型}不妨设$a_{11}\neq 0$.
\begin{align}
	\dv yx=\frac{a_{12}\pm\sqrt\Delta}{a_{11}}\thus
	\begin{cases}
		\xi(x,y)=\cns \\
		\eta(x,y)=\cns'
	\end{cases}
\end{align}
继而$A_{11}=A_{22}=0,$
\begin{gather}\notag
	A_{12}=-\frac{2\Delta}{a_{11}}\xi_y\eta_y\neq 0,\\
	u_{\xi\eta}+\frac1{2A_{12}}\kh{B_1u_\xi+B_2u_\eta+cu}=0.
\end{gather}
上式就是\textbf{双曲型方程的标准型}。

若再作变量替换$p=(\xi+\eta)/2,\;q=(\xi-\eta)/2,$方程可化为
\begin{align}
	u_{pp}-u_{qq}+\frac1{A_{12}}\zkh{\kh{B_1+B_2}u_q+\kh{B_1-B_2}u_p+cu}=0.
\end{align}
\paragraph*{$\Delta=0$,抛物型}只有一个线性ODE,不妨设$a_{11}\neq 0$.
\begin{align}
	\dv yx=\frac{a_{12}}{a_{11}}\thus\xi(x,y)=\cns.
\end{align}
任取$\eta(x,y)$,可得$A_{11}=A_{12}=0$
\begin{align}
	u_{\eta\eta}+\frac1{A_{22}}\kh{B_1u_\xi+B_2u_\eta+cu}=0.
\end{align}
\paragraph*{$\Delta<0$,椭圆型}ODE解为复函数,不妨设$a_{11}\neq 0,$
\begin{align}
	\dv yx=\frac{a_{12}\pm\i\sqrt{-\Delta}}{a_{11}}\thus \xi(x,y)\pm\i\eta(x,y)=\cns.
\end{align}
其中$\xi,\eta$为实函数,$A_{12}=0,A_{11}=A_{22}\neq 0,$
\begin{align}
	u_{\xi\xi}+u_{\eta\eta}+\frac1{A_{11}}\kh{B_1u_\xi+B_2u_\eta+cu}=0.
\end{align}
\subsection{叠加原理和齐次化原理}
定义算子为从函数类到函数类的映射$\mathcal T$。
%一般的,二阶线性PDE中
%\[\Par\equiv\sum_{i,j=1}^na_{ij}(X)\pw{}{x_i}{x_j}+\sum_{i=1}^nb_i(X)\pp{x_i}+c(X).\]
%是$\sC^2\to\sC$的线性算子.
\begin{theorem}{叠加原理}{Superposition Principle}
	%所谓叠加原理,即
	线性算子$\cL$可与$\textstyle\lim,\sum,\int$等运算符交换.% 在课程范围内,这是适当的.
\end{theorem}

齐次化原理也称冲量原理,源于求解有外力的弦振动方程
\begin{align*}
	\begin{cases}
		u_{tt}=a^2u_{xx}+f(x,t),\\
		u(x,0)=\varphi(x),\quad u_t(x,0)=\psi(x).
	\end{cases}
\end{align*}
利用叠加原理,解$u=u_0+w$;其中$u_0$表示$f\equiv 0$的齐次化解;$w$表示$\varphi,\psi\equiv 0$的解,即纯受迫振动.

考虑时间段$\zkh{\tau,\tau+\vd\tau}$内,位移分布$w(x,t)=:v(x,t;\tau)\vd\tau$,外力冲量$f(x,\tau)\vd\tau$,有
\begin{align*}
	%w(x,t)=\int_0^tv(x,t;\tau)\d\tau,\quad
	\begin{cases}
			v_{tt}=a^2v_{xx},\\
			v|_{t=\tau}=0\quad v_t|_{t=\tau}=f(x,\tau).
	\end{cases}\Rightarrow\quad v(x,t;\tau)=\frac1{2a}\int_{x-a(t-\tau)}^{x+a(t-\tau)}f(\xi,\tau)\d\xi.
\end{align*}
%因此
%\[=\frac1{2a}\int_0^t\bs3\int_{x-a(t-\tau)}^{t+a(t-\tau)}f(\xi,\tau)\d\xi\nd\tau.\]
叠加得
\begin{align}
	\begin{aligned}
		u(x,t)= &\;\frac12\zkh{\varphi(x-at)+\varphi(x+at)}+\frac1{2a}\int_{x-at}^{x+at}\psi(\xi)\d\xi \\
		        & +\frac1{2a}\int_0^t\bs3\int_{x-a(t-\tau)}^{x+a(t-\tau)}f(\xi,\tau)\d\xi\nd\tau.
	\end{aligned}
\end{align}
\begin{theorem}{齐次化原理}{Homogeneity Principle}
	定解问题
	\begin{align*}
		\begin{cases}
			\frac{\p^mu}{\p t^m}=\Par u+f(\bm x,t),&\bm x\in\mathbb R^n,\\
			\edg u_{t=0}=\edg{\pv ut}_{t=0}=\cdots=\edg{\frac{\p^{m-1}u}{\p t^{m-1}}}_{t=0}=0.
		\end{cases}
	\end{align*}
	其中$\Par$为常系数线性偏微分算子,有解
	\[u(\bm x,t)=\int_0^tv(\bm x,t;\tau)\d\tau,\]
	$v$满足
	\begin{align*}
		\begin{cases}
			\pu mvt=\Par v,&\bm x\in\mathbb R^n,\\
			\edg v_{t=\tau}=\edg{\pv v{t}}_{t=\tau}=\cdots=\edg{\pu{m-2}vt}_{t=\tau}=0, \\
			\edg{\pu{m-1}vt}_{t=\tau}=f(\bm x,\tau).
		\end{cases}
	\end{align*}

	当$\bm x\in V\subsetneq\mathbb R^n$时,还应加上其次边界条件
	\[\edg{\Par u}_{\p V}=0,\quad\edg{\Par v}_{\p V}=0.\]
\end{theorem}
\clearpage
\section{分离变量法}
回忆一些典型ODE方程的基本解法
\paragraph{二阶常系数其次方程}
\begin{align}
	y''+ay'+by=0.
\end{align}
其特征方程$\lambda^2+a\lambda+b=0$的解$\lambda_{1,2}$
\begin{compactenum}[I.]
	\item $\Delta>0,$\quad$y=A\e{\lambda_1x}+B\e{\lambda_2x};$
	\item $\Delta=0,$\quad$\lambda_{1,2}=\lambda,$\quad$y=(A+Bx)\e{\lambda x};$
	\item $\Delta<0,$\quad$\lambda_{1,2}=\alpha\pm\i\beta$,\quad$y=(A\cos\beta x+B\sin\beta x)\e{\alpha x}$.
\end{compactenum}
\paragraph{二阶非其次方程}
\begin{align}
	y''+a(x)y'+b(x)y=f(x).
\end{align}
根据叠加定理,其解$y=y_0+y_\mathrm s;$其中$y_0$为$f(x)\equiv 0$时的其次通解,$y_\mathrm s$为特解.

若$y_1,y_2$为对应的其次方程线性无关解,则可求出一特解
\begin{align}\label{General to Particular Solution}
	y_\mathrm s(x)=\int_{x_0}^x\frac{y_1(\xi)y_2(x)-y_1(x)y_2(\xi)}{y_1(\xi)y_2'(\xi)-y_1'(\xi)y_2(\xi)}f(\xi)\d \xi,\quad\forall x_0.
\end{align}
特解满足其次边界条件
\[y_\mathrm s(x_0)=0,\quad y'_\mathrm s(x_0)=0.\]
\paragraph{Euler方程}
\begin{align}
	x^2y''+axy'+by=f(x).
\end{align}
令$x=\e t,\;u(t)=y(\e t)=y(x)$方程变为
\begin{align}
	u''+(a-1)u'+bu=f(\e t).
\end{align}
\clearpage
\subsection{特征值和特征函数}
在讨论分离变量前,先引入Fourier级数的概念。
\begin{definition}{Fourier级数}{Fourier Series}
	$f(x)$周期为$2\ell$,则$f(x)$的Fourier级数
	\begin{align}
		\FS f(x)=\frac{a_0}2+\sum_{n=1}^\infty\kh{a_n\cos\frac{n\pi x}\ell+b_n\sin\frac{n\pi x}\ell}.
	\end{align}
	其中Fourier系数
	\begin{align*}
		a_n & =\frac1\ell\int_{-\ell}^\ell f(x)\cos\frac{n\pi x}\ell\d x; \\
		b_n & =\frac1\ell\int_{-\ell}^\ell f(x)\sin\frac{n\pi x}\ell\d x.
	\end{align*}
\end{definition}
\begin{theorem}{Parserval等式}{Parserval Equality}
	若$f\in\mathscr L^2\zkh{-\ell,\ell}$即平方可积,则
	\[\frac1\ell\int_{-\ell}^\ell f^2(x)\d x=\frac{a_0^2}2+\sum_{n=1}^\infty\kh{a_n^2+b_n^2}.\]
\end{theorem}
\begin{theorem}{Dirichlet收敛定理}{Dirichlet Convergence Theorem}
	若
	\begin{compactenum}
		\item $f,f'$连续或分段连续,至多有有限个第一类间断点;%(可去或跳跃)
		\item $f$至多有有限个极值点;
	\end{compactenum}
	则
	\begin{align*}
		\FS f(x)
		=\begin{cases}
			\kh{f(x^+)+f(x^-)}/2,        & x\in(-\ell,\ell) \\
			\kh{f(-\ell^+)+f(\ell^-)}/2, & x=\pm\ell
		\end{cases}
	\end{align*}
	在$f$连续点处,$\FS f=f$.
\end{theorem}
\eg[1] 有界弦问题
\begin{align*}
	\begin{cases}
		u_{tt}=a^2u_{xx},&x\in\kh{0,L},\;t>0 \\
		u(0,t)=u(L,t)=0,                 \\
		u(x,0)=\varphi(x)\quad u_t(x,0)=\psi(x).
	\end{cases}
\end{align*}
假设变量是可以分离的,即$u(x,t)=X(x)T(t),$则引入\textbf{特征值}$\lambda$
\[\frac{X''}X=\frac{T''}{a^2T}=-\lambda.\]
要使$X''+\lambda X=0$且$X(0)=X(L)=0$仅有
\[\lambda_n=\kh{\frac{n\pi}L}^2,\quad X_n(x)=\sin\frac{n\pi x}L,\quad n=1,2,\ldots.\]
进而$T''+a^2\lambda T=0$
\[T_n(t)=C_n\cos\frac{n\pi at}L+D_n\sin\frac{n\pi at}L.\]
因此原方程应有\textit{形式解}
\begin{align}
	u(x,t)=\sum_{n=1}^\infty\fkh{C_n\cos\frac{n\pi at}L+D_n\sin\frac{n\pi at}L}\sin\frac{n\pi x}L.
\end{align}
$C_n,D_n$由初始条件确定
\begin{align*}
	u(x,0)&=\sum_{n=1}^\infty C_n\sin\frac{n\pi x}L=\varphi(x),            \\
	u_t(x,0)&=\sum_{n=1}^\infty D_n\frac{n\pi a}L\sin\frac{n\pi x}L=\psi(x).
\end{align*}
恰好对应Fourier正弦系数$\varphi^{\sine}_n,\psi^{\sine}_n$
\begin{align*}
	C_n & =\frac2L\int_0^L\varphi(x)\sin\frac{n\pi x}L\d x=:\varphi_n^{\sine},                  \\
	D_n & =\frac2{n\pi a}\int_0^L\psi(x)\sin\frac{n\pi x}L\d x=:\frac{L}{n\pi a}\psi_n^{\sine}.
\end{align*}

至于形式解是否符合条件,不是考试涉及的范围.
\eg[2] 圆域上的Laplace方程,采用极坐标
\begin{align*}
	\begin{cases}
		\lap_2u=u_{rr}+\frac1ru_r+\frac1{r^2}u_{\theta\theta}=0,&r<r_0\\
		u(r_0,\theta)=\varphi(\theta)
	\end{cases}
\end{align*}
还应包括自然条件,即
\[\limsup_{r\to0^+}\abs{u(r,\theta)}<\infty.\]
和周期边界条件
\[u(r,0)=u(r,2\pi),\quad u_\theta(r,0)=u_\theta(r,2\pi).\]

分离变量$u(r,\theta)=R(r)\Ct(\theta)$
\begin{align*}
	\begin{cases}
		r^2R''+rR'+\lambda R=0\\
		\Ct''-\lambda\Ct=0.
	\end{cases}
\end{align*}
由周期条件,特征值$\lambda=n^2,\;n=0,1,2,\ldots$
\[\Ct_n(\theta)=a_n\cos n\theta+b_n\sin n\theta=:\begin{Bmatrix}
	\cos n\theta\\
	\sin n\theta
\end{Bmatrix}\]
再解$R$的Euler方程,可得
\[R_n(r)=\begin{cases}
	c_0+d_0\ln r,&n=0\\
	c_nr^n+d_nr^{-n},&n\geqslant 1
\end{cases}\]
再由自然条件$\abs{R(0)}<\infty$,可知$d_i\equiv 0$,进而
\[u(r,\theta)=\frac{a_0}2+\sum_{n=1}^\infty r^n(a_n\cos n\theta+b_b\sin n\theta),\]
结合边界条件$u(r_0,\theta)=\varphi(\theta)$,求得系数
\[a_n=\frac1{\pi r_0^n}\int_0^{2\pi}\varphi(\vartheta)\cos n\vartheta\d\vartheta,\quad b_n=\frac1{\pi r_0^n}\int_0^{2\pi}\varphi(\vartheta)\sin n\vartheta\d\vartheta.\]
因此
\begin{align*}
	u(r,\theta)&=\frac1{2\pi}\int_0^{2\pi}\varphi(\vartheta)\d\vartheta+\\
	&\qquad\frac1\pi\sum_{n=1}^\infty\kh{\frac r{r_0}}^n\int_0^{2\pi}\varphi(\vartheta)(\cos n\vartheta\cos n\theta+\sin n\vartheta\sin n\theta)\d\vartheta\\
	&=\frac1{2\pi}\int_0^{2\pi}\varphi(\vartheta)\fkh{1+2\sum_{n=1}^\infty\kh{\frac r{r_0}}^n\cos n(\theta-\vartheta)}\d\vartheta
\end{align*}
又由Euler公式
\begin{align*}
\sum_{n=0}^\infty k^n\cos n\theta=\sum_{n=0}^\infty\Re k^n\e{\i n\theta}=\Re\frac1{1-k\e{\i\theta}}=\frac{1-k\cos\theta}{1+k^2-2k\cos\theta}.
\end{align*}
故得到圆域内的Poisson公式
\begin{align}
	u(r,\theta)&=\frac1{2\pi}\int_0^{2\pi}\frac{(r_0^2-r^2)\varphi(\vartheta)}{r_0^2+r^2-2r_0r\cos(\theta-\vartheta)}\d\vartheta
\end{align}
积分内式子称为Poisson核.

\eg[3] 圆域上的Neumann问题
\begin{align*}
	\begin{cases}
		\lap_2u=0,&r<R\\
		\edg{\pv un}_{r=R}=\varphi(\theta)
	\end{cases}
\end{align*}
由~\textcolor[rgb]{0,.8,0}{Green公式}
	\begin{align*}
		{\color[rgb]{0,.8,0}{\int_{D_R}\lap u\d x\nd y=\oint_{r=R}\pv un\d\ell}}=R\int_0^{2\pi}\varphi(\theta)\d\theta=0.
	\end{align*}

	剩下的证明留给读者
\clearpage
\subsection{Sturm-Liouville定理}
\eg 有限杆热传导问题,设杆温度$u=u(x,t)$,则定解问题
\begin{align*}
	\begin{cases}
		u_t=a^2u_{xx},&x\in\kh{0,L},\;t>0,\\
		u(0,t)=0,\quad u_t(L,t)+hu(L,t)=0,\\
		u(x,0)=\varphi(x).
	\end{cases}
\end{align*}
其中热交换常数$h>0$.

分离常数$u(x,t)=X(x)T(t),$
\begin{align*}
	\begin{cases}
		X''+\beta^2X=0, \\
		X(0)=0,\quad X'(L)+hX(L)=0.
	\end{cases}\thus\beta+h\tan\beta L=0.
\end{align*}
$\Rightarrow\quad X_n(x)=\sin\beta_nx,\quad T_n(t)=C_n\e{-\beta_n^2a^2t}$.可以验证$X_n$是正交的,即
\[\ave{X_m,X_n}=\int_0^LX_m(x)X_n(x)\d x=0,\quad m\neq n.\]
形式解
\[u_n(x,t)=\sum_{n=1}^\infty C_n\even^{-\beta_n^2a^2t}\sin\beta_nx.\]

再根据初始条件
\[u(x,0)=\sum_{n=1}^\infty C_n\sin\beta_nx=\varphi(x).\]
在后面可以看到,这是广义Fourier级数,其系数
\[C_n=\frac{\ave{\varphi,\sin\beta_nx}}{\ave{\sin\beta_nx,\sin\beta_nx}}=\frac{\int_0^L\varphi(x)\sin\beta_nx\d x}{\int_0^L\sin^2\beta_nx\d x}.\]
\begin{definition}{加权内积}{Weight Inner Product}
	定义在$\zkh{a,b}$上的实函数$f,g$的$\rho$\;-\;加权内积
	\[\inp fg_\rho:=\int_a^bf(x)g(x)\rho(x)\d x.\]
	其中权函数$\rho\geqslant 0$分段连续且零点孤立.
	% 若$f,g$正交,则$\inp fg_\rho=0$.
	%定义范数$\norm f_\rho:=\inp ff_\rho^{1/2}$.及

	加权平方可积函数空间
	\[\mathscr L_\rho^2\zkh{a,b}:=\set f{\norm f_\rho<\infty}.\]
\end{definition}
\begin{theorem}{广义Fourier级数}{Generalized Fourier Series}
	若$f_1,f_2,\ldots$在$\mathscr L_\rho^2$中完备且加权正交,则$\forall f\in\mathscr L_\rho^2$有广义Fourier级数展开
	\[f(x)\sim\sum_{i=1}^\infty a_if_i(x),\quad a_i=\frac{\inp f{f_i}_\rho}{\norm{f_i}^2_\rho}.\]
	且有Parserval等式
	\[\norm f_\rho^2=\sum_{i=1}^n\inp f{\frac{f_i}{\norm{f_i}_\rho}}_\rho^2.\]
\end{theorem}
\begin{definition}{Sturm-Liouville方程}{Sturm-Liouville Equation}
	定义域$\zkh{a,b}$,常见一维特征值问题都可以化为
	\[\kh{k(x)f'(x)}'-q(x)f(x)+\lambda\rho(x)f(x)=0.\]
	其中$\lambda$为参数,$k,q,\rho$为实函数.

	记
	\[\cl f:=-\frac{\kh{kf'}'-qf}\rho.\]
	则
	\[\inp{\cl f}g_\rho-\inp f{\cl g}_\rho=\zkh{k(fg'-f'g)}_a^b.\]
	好的边界条件可以使上式等于0.
\end{definition}
\paragraph*{正则Sturm-Liouville问题}
\[\cl f(x)=\lambda f(x).\]

正则条件:
\[
	k\in\sC^2\zkh{a,b},\;q,\rho\in\sC\zkh{a,b};\;k,\rho>0.
\]
保证了Sturm-Liouville方程在$\zkh{a,b}$上没有奇点.

边界条件:
\begin{center}
	可分:$\begin{cases}
			c_1f(a)+c_2f'(a)=0, \\
			d_1f(b)+d_2f'(b)=0.
		\end{cases}$\quad 或 \quad 周期:
	$\begin{cases}
			f(a)=f(b) \\
			f'(a)=f'(b)
		\end{cases}$
\end{center}
这些边界条件是好的,可使$\cL$是对称算子,即
\[\inp{\cl f}g_\rho=\inp f{\cl g}_\rho.\]
\begin{theorem}{Sturm-Liouville定理1}{Sturm-Liouville Theorem I}
	正则Sturm-Liouville问题
	\begin{compactenum}
		\item 有可数多个实特征值
		      \[\lambda_1<\lambda_2<\cdots<\lambda_n<\cdots,\quad\lim_{n\to\infty}\lambda_n=+\infty.\]
		      % 对应特征函数
		      % \[X_1(x),X_2(x),\ldots,X_n(x),\ldots\]
		\item 特征函数加权$\rho$\;-\;正交;
		\item 特征值的特征子空间至多2维,

		      可分边界条件下,特征子空间为1维;
		\item 特征函数构成$\mathscr L_\rho^2$上完备的正交基底.
	\end{compactenum}
\end{theorem}
奇异Sturm-Liouville问题:不满足正则条件或区间无界;但若
\begin{center}
	$a$或$b$是$k(x)$的一级零点,是$q(x)$的至多一级极点, % 时,称$a$或$b$是方程的正则奇点.
\end{center}
定理结论依旧成立!
\begin{theorem}{Sturm-Liouville定理2}{Sturm-Liouville Theorem II}
	若正则Sturm-Liouville问题还满足$q\geqslant 0$;可分边界条件下还需满足$c_1c_2\leqslant 0,\;d_1d_2\geqslant 0$.
	\begin{compactenum}
		\item 所有特征值非负;

		      特别的,存在0特征值(对应特征函数1)的充要条件是$q\equiv 0$且两端为第二类边界条件;
		\item 若正则条件为周期条件,那么其对应于每一个非最小特征值$\lambda_0$
		的特征值有两个相互正交的特征函数.
	\end{compactenum}
\end{theorem}
\eg 扇形域上的Dirichlet问题
\begin{align*}
	\begin{cases}
		\lap_2u=0,&r\in\kh{1,\mathrm e},\;\theta\in\kh{0,\frac\pi{2}}.\\
		u(1,\theta)=u(\mathrm e,\theta)=0,\\
		u(r,0)=0,\quad u\kh{r,\frac\pi{2}}=g(r).
	\end{cases}
\end{align*}
化为Sturm-Liouville型
\[\kh{rR'}'+\frac{\lambda}rR=0,\]
对应\(k=r,\;q=0,\;\rho=\frac1r\),因此\(\lambda>0\),进而
\begin{gather*}
	\lambda_n=(n\pi)^2,\quad R_n(r)=\sin\kh{n\pi\ln r},\quad n=1,2,\ldots
	\\
	\Ct_n(\theta)=A_n\e{n\pi\theta}+B_n\e{-n\pi\theta}.
\end{gather*}
再结合$u$的边界条件即可解出.
\subsection{非其次方程}
对非齐次方程,有时可靠直觉找出特解$v$,但对一般的非齐次方程,需要借助齐次化原理转化为其次方程:
\paragraph*{齐次化原理法}
设$\Dif_t$中关于2阶导数的系数为1
\begin{align*}
	\begin{cases}
		\Dif_xu+\Dif_tu=f(x,t),& x\in(a,b),\;t>0\\
		\kh{\alpha_1u-\beta_1u_x}_{x=a}=0,\quad\kh{\alpha_2u+\beta_2u_x}_{x=b}=0,\\
		\edg u_{t=0}=\edg{u_t}_{t=0}=0.
	\end{cases}
\end{align*}
由 \pageref{thm:Homogeneity Principle} 页的定理 \ref{thm:Homogeneity Principle} 知,有解
\[u(x,t)=\int_0^tv(x,t-\tau;\tau)\d\tau\]
$v$满足
\begin{align*}
	\begin{cases}
		\Dif_xv+\Dif_tv=0,&x\in(a,b),\;t>0\\
		\kh{\alpha_1v-\beta_1v_x}_{x=a}=0,\quad\kh{\alpha_2v+\beta_2v_x}_{x=b}=0,\\
		\edg v_{t=0}=0,\quad\edg{v_t}_{t=0}=f(x,\tau).
	\end{cases}
\end{align*}
\paragraph*{广义Fourier展开法}对于 
\begin{align*}
	\begin{cases}
		\Dif_xu+\Dif_tu=f(x,t),& x\in(a,b),\;t>0\\
		\kh{\alpha_1u-\beta_1u_x}_{x=a}=0,\quad\kh{\alpha_2u+\beta_2u_x}_{x=b}=0,\\
		\edg u_{t=0}=\varphi(x),\quad\edg{u_t}_{t=0}=\psi(x).
	\end{cases}
\end{align*}
\begin{compactenum}
	\item 首先分离变量,求出$f\equiv 0$对应的特征值问题
	\[\Dif_xX=-\lambda X.\]
	解出特征值$\{\lambda_n\}$和本征函数$\{X_n(x)\}$,注意$n$的取值是否含0。
	\item 根据Sturm-Liouville定理判断$\{X_n\}$的完备性,展开
	\begin{gather*}
		u(x,t)=\sum T_n(t)X_n(x),\quad f(x)=\sum f_n(t)X_n(x),\\
		\varphi(x)=\sum\varphi_nX_n(x),\quad \psi(x)=\sum\psi_nX_n(x).
	\end{gather*}
	求解 
	\begin{align*}
		\begin{cases}
			\Dif_tT_n(t)-\lambda_nT_n(t)=f_n(t),\\
			T_n(t)=\varphi_n,\quad T_n'(t)=\psi_n.
		\end{cases}
	\end{align*}
	\item 解出$T_n(t)$,得$u(x,t)$
\end{compactenum}
\paragraph*{一般的非齐次问题}
对于一般的非齐次问题
\begin{align*}
	\begin{cases}
		\Dif_xu+\Dif_tu=f(x,t),&x\in(a,b),\;t>0\\
		\kh{\alpha_1u-\beta_1u_x}_{x=a}=g_1(t),\quad\kh{\alpha_2u+\beta_2u_x}_{x=b}=g_2(t),\\
		\edg u_{t=0}=\varphi(x),\quad\edg{u_t}_{t=0}=\psi(x).
	\end{cases}
\end{align*}
首先将非齐次边界条件齐次化,寻找$v(x,t)$满足 
\[
	\kh{\alpha_1v-\beta_1v_x}_{x=a}=g_1(t),\quad\kh{\alpha_2v+\beta_2v_x}_{x=b}=g_2(t).
\]

显然$v$不唯一,若无法直觉看出,可设为$x$的线性函数
\[
	v(x,t)=A(t)x+B(t),
\]
整理得
\[
	\begin{cases}
		(\alpha_1a-\beta_1)A(t)+\alpha_1B(t)=g_1(t),\\
		(\alpha_2b+\beta_2)A(t)+\alpha_2B(t)=g_2(t).
	\end{cases}
\]
便可解出$A,B$若无解还可设为$x$的二次函数。

这样便可将$u$分解为$u=v+w$,而$w$的边界条件是齐次化的:
\begin{align*}
	\begin{cases}
		\Dif_xw+\Dif_tw=f(x,t)-\Dif_xv-\Dif_tv,&x\in(a,b),\;t>0\\
		\kh{\alpha_1w-\beta_1w_x}_{x=a}=0,\quad\kh{\alpha_2w+\beta_2w_x}_{x=b}=0,\\
		\edg w_{t=0}=\varphi(x)-\edg v_{t=0},\quad\edg{w_t}_{t=0}=\psi(x)-\edg{v_t}_{t=0}.
	\end{cases}
\end{align*}
这在前面已解决。
\clearpage
\section{积分变换法}
\begin{definition}{积分变换}{Integral Transformation}
	积分变换
	\[\mathcal T\zkh{f}(\xi)=\int_a^bf(x)K(x,\xi)\d x.\]
	其中$K$是核函数(kernel).
\end{definition}
\subsection{Fourier变换}
\setcounter{subsubsection}{1}
\begin{definition}{Fourier变换}{Fourier Transformation}
	定义Fourier变换
	\[\hat f(\xi)\equiv\cf f(\xi):=\int\iti f(x)\e{-\i\xi x}\d x,\quad \xi\in\RR,\]
	及其逆变换
	\[f(x)=\mathcal F^{-1}\big[\hat f\big](x)=\frac1{2\pi}\int\iti\hat f(\xi)\e{\i\xi x}\d\xi.\]
\end{definition}
Fourier变换的相关性质见第 \pageref{The Property of Fourier Transformation} 页的附录内容。
\eg 无界长杆热传导
\begin{align*}
	\begin{cases}
		u_t=a^2u_{xx}+f(x,t),\\
		u(x,0)=\varphi(x).
	\end{cases}\longrightarrow\quad\begin{cases}
		\hat u_t=-a^2\xi^2\hat u+\hat f,\\
		\hat u(\xi,0)=\hat\varphi(\varphi).
	\end{cases}
\end{align*}
因此
\begin{align*}
	\hat u(\xi,t)&=\e{-a^2\xi^2t}\fkh{\int_0^t\hat f(\xi,\tau)\e{a^2\xi^2\tau}\d\tau+\hat\varphi(\xi)}.
	% \\
	% &=\int_0^t\hat f(\xi,\tau)\e{-a^2\xi^2(t-\tau)}\d\tau+\hat\varphi(\xi)\e{-a^2\xi^2t}.
\end{align*}
再用Fourier逆变换
\begin{align*}
	u(x,t)&=\cfi{\int_0^t\hat f(\xi,\tau)\e{-a^2\xi^2(t-\tau)}\d\tau}+\cfi{\hat\varphi(\xi)\e{-a^2\xi^2t}}\\
	&=\int_0^t\cfi{\hat f(\xi,\tau)}\!\ast\cfi{\e{-a^2\xi^2(t-\tau)}}\d\tau+\cfi{\hat\varphi(\xi)}\ast\cfi{\e{-a^2\xi^2t}}\\
	&=\int_0^tf(x,t)\ast\frac{\e{-x^2/4a^2(t-\tau)}}{2a\sqrt{\pi(t-\tau)}}\d\tau+\frac1{2a\sqrt{\pi t}}\int\iti\varphi(\eta)\e{-(x-\eta)^2/4a^2t}\d\eta\\
	&=\int_0^t\frac1{2a\sqrt{\pi(t-\tau)}}\int\iti f(\eta,\tau)\e{-(x-\eta)^2/4a^2(t-\tau)}\d\eta\d\tau+\\
	&\qquad\qquad\frac1{2a\sqrt{\pi t}}\int\iti\varphi(\eta)\e{-(x-\eta)^2/4a^2t}\d\eta.
\end{align*}
以上只是形式解.
\subsubsection{Fourier正余弦变换}
对于定义在$\RR_{\geqslant  0}$上的函数,可将其奇偶延拓至$\RR$后再作4ier变换
\begin{definition}{Fourier正余弦变换}{Fourier Sine and Cosine Transformation}
	正余弦变换分别为
	\begin{align*}
		\hat f_\sine(\xi)\equiv\cf[s]f(\xi):=\int\zti f(x)\sin\xi x\d x;\\
		\hat f_\cosi(\xi)\equiv\cf[c]f(\xi):=\int\zti f(x)\cos\xi x\d x.
	\end{align*}
	其反变换
	\begin{align*}
		f(x)&=\frac2\pi\int\zti\hat f_\sine(\xi)\sin\xi x\d\xi=:\cfi[s]{\hat f_\sine(\xi)}\\
		&=\frac2\pi\int\zti\hat f_\cosi(\xi)\cos\xi x\d\xi=:\cfi[c]{\hat f_\cosi(\xi)}
	\end{align*}
\end{definition}
变换的性质见第 \pageref{The Property of Sine and Cosine Transformation} 页的附录内容。
\eg 半无界杆热传导方程问题
\begin{align*}
	\begin{cases}
		u_t=a^2u_{xx},&x>0,\;t>0\\
		u(x,0)=0,\\
		u_x(0,t)=\varphi(x)\\
	\end{cases}
\end{align*}
自然边界条件
\[\lim_{x\to\infty}u(x,t)=\lim_{x\to\infty}u_t(x,t)=0\]

若用Fourier正弦变换,则会出现$u(0,t)$,此边界条件没有给出,因此只能用余弦变换,记$U(\xi,t)=\cf[c]{u(x,t)}$
\begin{align*}
	U_t(\xi,t)&=a^2\zkh{u_x(x,t)\cos\xi x-\xi u(x,t)\sin\xi x}\zti-a^2\xi^2U(\xi,t)\\
	&=-a^2\varphi(t)-a^2\xi^2U(\xi,t).
\end{align*}
故
\[U(\xi,t)=-a^2\int_0^t\varphi(\tau)\e{-a^2\xi^2(t-\tau)}\d\tau\]
逆变换
\begin{align*}
	u(x,t)&=-\frac{2a^2}\pi\int\zti\fkh{\int_0^t\varphi(\tau)\e{-a^2\xi^2(t-\tau)}\d\tau}\cos\xi x\d\xi\\
	&=-\frac{2a^2}\pi\int_0^t\varphi(\tau)\fkh{\int\zti\e{-a^2\xi^2(t-\tau)}\cos\xi x\d\xi}\d\tau\\
	&=-a\int_0^t\varphi(\tau)\frac{\e{-x^2/4a^2(t-\tau)}}{\sqrt{\pi(t-\tau)}}\d\tau.
\end{align*}

\eg 半无界弦振动问题
\begin{align*}
	\begin{cases}
		u_{tt}=a^2u_{xx},& x>0,\;t>0\\
		u_x(0,t)=f(t),\\ % \quad u(0,t)=f(t),
		u(x,0)=u_t(x,0)=0
	\end{cases}
\end{align*}
自然边界条件$\lim_{x\to\infty}u(x,t)=\lim_{x\to\infty}u_t(x,t)=0$。
\[U_{tt}(\xi,t)=-a^2f(t)-a^2\xi^2U(\xi,t).\]
其次方程通解$\cos a\xi t,\;\sin a\xi t$,运用(\ref{General to Particular Solution})和边界条件$U(\xi,0)=U_t(\xi,0)=0$
\begin{align*}
	U(\xi,t)&=\int_0^t-a^2f(\tau)\frac{\cos a\xi\tau\sin a\xi t-\cos a\xi t\sin a\xi\tau}{\cos a\xi\tau\kh{\sin a\xi\tau}'-\kh{\cos a\xi\tau}'\sin a\xi\tau}\d\tau\\
	&=-\frac a\xi\int_0^tf(\tau)\sin a\xi(t-\tau)\d\tau.
\end{align*}
逆变换
\begin{align*}
	u(x,t)&=-\frac{2}\pi\int\zti\fkh{\frac a\xi\int_0^tf(\tau)\sin a\xi(t-\tau)\d\tau}\cos \xi x\d\xi\\
	&=-2a\int_0^tf(\tau)\cdot\frac1{2\pi}\int\iti\frac1\xi\sin a\xi(t-\tau)\e{\i\xi x}\d\xi\d\tau
\end{align*}
由
\[\cfi{\frac{\sin a\xi}\xi}=\frac{\Heaviside(x+a)-\Heaviside(x+a)}2.\]
其中$\Heaviside(x)$为Heaviside跳跃函数,故
\begin{align*}
	u(x,t) % &=-a\int_0^tf(\tau)\fkh{H(x+a(t-\tau))-H(x-a(t-\tau))}\d\tau\\
	% =-a\int_0^{t-x/a}f(\tau)\Heaviside(x+a(t-\tau))\d\tau
	=-a\int_0^{t-x/a}f(\tau)\d\tau.\quad x<at
\end{align*}
\subsubsection{Fourier变换与分离变量法}
无界区域是否有相应的分离变量法?
\eg 无界长杆热传导
\begin{align*}
	\begin{cases}
		u_t=a^2u_{xx},& x\in\RR,\;t>0\\
		u(x,0)=\varphi(x)
	\end{cases}
\end{align*}
分离变量$u(x,t)=X(x)T(t)$特征值
\[X''+\lambda X=0,\;T'+\lambda a^2T=0\]
讨论知$\lambda=\xi^2\geqslant 0$
\[X=\e{\i\xi x},\quad T=\e{-a^2\xi^2t}.\]
故
\[u(x,t)=\frac1{2\pi}\int\iti A(\xi)X(x,\xi)T(t,\xi)\d\xi=\frac1{2\pi}\int\iti A(\xi)\e{\i\xi x}\e{-a^2\xi^2t}\d\xi.\]
边界条件
\[u(x,0)=\frac1{2\pi}\int\iti A(\xi)\e{\i\xi x}\d\xi\equiv\cfi A=\varphi(x),\]
故$\varphi(x)$是$A(\xi)$的Fourier变换。

可见对无界区间上的问题,分离变量法依然适用,只是特征值是连续分布的。
\subsection{Laplace变换}
\begin{definition}{Laplace变换}{Laplace Transformation}
	定义Laplace变换
	\[\bar f(\xi)=\cl f(\xi):=\int\zti f(x)\e{-\xi x}\d x,\quad\Re\xi>\sigma_0.\]
	
	其逆变换
	\[f(x)=\cli{\bar f(\xi)}=\frac1{2\pi\i}\int_{\sigma-\i\infty}^{\sigma+\i\infty}\bar f(\xi)\e{\xi x}\d\xi,\quad x\geqslant 0.\]
\end{definition}
注意Laplace变换存在条件需要$\exists M,\sigma_0$使得
	\[\abs{f(x)}<M\e{\sigma_0x},\quad\forall x>0.\]
	% $x$为$f$连续点处
% 再看Laplace逆变换$\cli{\bar f(\xi)}\to f(x)$。
% \begin{theorem}{Riemann-Mellin反演公式}{Riemann-Mellin Inversion Formula}
其逆变换中$\sigma>\sigma_0$;注意,Laplace变换的积分中本身不包含$\RR_{<0}$的部分,一般认为$f(<0)\equiv 0$。
\begin{theorem}{第一展开定理}{}
	设$F(\xi)$在$\infty$邻域内有Laurent展开式
	\[F(\xi)=\sum\fmto n1\frac{c_n}{\xi^n},\]
	则
	\[f(x)=\cli{F(\xi)}=\sum\fmto n1\frac{c_n}{(n-1)!}x^{n-1},\quad x\geqslant 0\]
\end{theorem}
\begin{theorem}{第二展开定理}{}
	设$F(\xi)=A(\xi)/B(\xi)$是有理函数,$\deg A<\deg B$,$B(\xi)$只有单零点$\xi_1,\xi_2,\ldots,\xi_n$且是$F(\xi)$的单极点,则
	\[f(x)=\cli{F(\xi)}=\sum\fmto[n]k1\frac{A(\xi_k)}{B'(\xi_k)}\e{\xi_kx}=\sum\fmto[n]k1\Res\fkh{F(\xi)\e{\xi x},\xi_k}.\]
\tcblower
	若$F(\xi)$满足
	\begin{compactenum}
		\item 在$\CC$上除了有限个奇点$\xi_1,\xi_2,\ldots,\xi_n$外解析;
		\item 在半平面$\Re\xi>\sigma_0$上解析;
		\item $\exists M>0,\;R>0$使得当$\abs\xi>R$时
		\[\abs{F(\xi)}\leqslant\frac M{\abs\xi},\]
	\end{compactenum}
	则对于$x\geqslant 0$有
	\[f(x)=\cli{F(\xi)}=\sum\fmto[n]k1\Res\fkh{F(\xi)\e{\xi x},\xi_k}.\]
\end{theorem}
\subsubsection{Laplace定理的应用}
\eg[1] 有限杆热传导
\begin{align*}
	\begin{cases}
		u_t=u_{xx},&x\in(0,L),\;t>0,     \\
		u_x(0,t)=0,\quad u(L,t)=A, \\
		u(x,0)=B.
	\end{cases}
\end{align*}
关于$t$作Laplace变换,记$U(x,s)=\cl{u(x,t)}(s)$
\[s U(x,s)-u(x,0)=U_{xx}(x,s),\]
由$u(x,0)=B$,
\[U(x,s)=c_1\cosh\sqrt s x+c_2\sinh\sqrt s x+\frac Bs.\]
再由$U_x(0,s)=0,\;U(L,s)=A/s$
\[U(x,s)=\frac{(A-B)\cosh\sqrt s x}{s\cosh\sqrt s L}+\frac Bs.\]
逆变换
\begin{align*}
	u(x,t)&=\cli{\frac{(A-B)\cosh\sqrt s x}{s\cosh\sqrt s L}}(t)+\cli{\frac Bs}(t)\\
	&=(A-B)\cli{\frac{\cosh\sqrt s x}{s\cosh\sqrt s L}}(t)+B.
\end{align*}
括号内函数的孤立奇点\footnote{注意,0不是此函数的孤立奇点,不能应用留数算出。逆变换结果中的1是围道积出来的。详情可见附录}应使得$\cosh\sqrt s L=0$
\[s_n=-\fkh{\frac{(2n-1)\pi}{2L}}^2,\quad n=1,2,\ldots\]
由
\begin{align*}
	u(x,t)&=B+(A-B)\hkh{1+\sum\fmto n1\Res\fkh{\frac{\cosh\sqrt s x}{s\cosh\sqrt s L}\e{s t},s_n}}\\
	&=A+\frac{4(A-B)}\pi\sum\fmto n1\frac{(-1)^n}{2n-1}\cos\fkh{\frac{(2n-1)\pi}{2L}x}\exp\fkh{-\frac{(2n-1)^2\pi^2}{4L^2}t}.
\end{align*}
\eg[2] 有限长弦受迫振动
\begin{align*}
	\begin{cases}
		u_{tt}=a^2u_{xx},&x\in(0,L),\;t>0\\
		u(0,t)=0,\quad u_t(L,t)=A\sin\omega t\\
		u(x,0)=u_t(x,0)=0
	\end{cases}
\end{align*}
关于$t$作Laplace变换,记$U(x,s)=\cl{u(x,t)}(s)$
\[s^2U(x,s)-s u(x,0)-u_t(x,0)=a^2U_{xx}(x,s).\]
结合$U_x(L,s)=A\omega/(s^2+\omega^2)$
\[U(x,s)=\frac{Aa\omega}{s(s^2+\omega^2)}\sinh\frac{xs}a\sech\frac{Ls}a.\]
其所有的孤立奇点:可去奇点0;一级奇点$\pm\i\omega_k$,
\[\omega_k:=\begin{cases}
	\omega,&k=0\\
	\frac{2k-1}{2L}\pi a,&k\geqslant 1
\end{cases}\]
故
\begin{align*}
	u(x,t)&=\sum\fmto k0\Res[U(x,s)\e{st},\i\omega_k]+\Res[U(x,s)\e{st},-\i\omega_k]\\
	&=2\Re\sum\fmto k0\Res[U(x,s)\e{st},\i\omega_k]\\
	&=2\Re\fkh{\frac{Aa\omega}{s(s+\i\omega)}\sinh\frac{xs}a\sech\frac{Ls}a\e{st}}_{\i\omega}+\\
	&\qquad 2\Re\sum\fmto k1\fkh{\frac{Aa\omega}{s(s^2+\omega^2)}\sinh\frac{xs}a\cdot\frac aL\csch\frac{Ls}a\e{st}}_{\i\omega_k}\\
	&=\frac{Aa}\omega\sinh\frac{\omega x}a\sec\frac{\omega L}a\sin\omega t+\\
	&\qquad 16Aa\omega^2L\sum\fmto k1\frac{(-1)^{k-1}}{(2k-1)[4\omega^2L^2-(2k-1)^2\pi^2a^2]}\sin\frac{\omega_kx}a\sin\omega_kt.
\end{align*}
\clearpage
\section{基本解方法}
\subsection{广义函数}
\paragraph*{$\delta$函数}
源于物理中对集中分布物理量的数学描述。满足
\[\int_a^bf(x)\delta(x)\d x=\begin{cases}
    f(x),&0\in(a,b)\\
    0,&0\notin(a,b)
\end{cases}\]
因此$f\ast\delta(\xi)=f(\xi)$。

\begin{theorem}{$\delta$复合函数}{}
	设$u(x)\in\sC(\RR)$且在实轴上只有单零点$x_1,x_2,\ldots,x_n$,则 
	\begin{align}
		\delta(u(x))=\sum\fmto[n]k1\frac{\vd(x-x_k)}{\abs{u'(x_k)}}.
	\end{align}
\end{theorem}
比如$\delta(ax)=\delta(x)/a$。
\paragraph*{广义函数}\hspace{4ex}
\begin{definition}{线性泛函}{Linear Functional}
	函数空间$\mathscr V$到数域$\mathbb F$的映射$\mathcal T$,若
	\[\mathcal T(au+bv)=a\mathcal Tu+b\mathcal Tv\]
	$\forall a,b\in\mathbb F,\;u,v\in\mathscr V$均成立,则$\mathcal T$是$\mathscr V$上的线性泛函,即广义函数。$\mathscr V$上所有线性泛函构成一个线性空间,称作$\mathscr V$的对偶空间$\mathscr V^*$。
\end{definition}
\eg 记实函数$f$的支撑集
\begin{align}
	\supp f:=\set x{f(x)\neq 0}.
\end{align}
并记
\[\mathscr L_0(\RR):=\set f{\int\iti\abs{f(x)}\d x<\infty,\;\supp f\,\text{有界}}\]
则$\forall f\in\mathscr L_0(\RR)$,$f$确定了$\sC(\RR)$上一个线性泛函:
\begin{align}
	\mathcal T_f\zkh{\varphi(x)}\equiv\inp f\varphi:=\int\iti f(x)\varphi(x)\d x\in\RR.
\end{align}
\paragraph*{广义函数的导数}定义函数空间
\begin{align}
	\Cooo(\RR)\equiv\sC_0^\infty(\RR):=\set f{f\in\sC^\infty(\RR),\;\supp f\,\text{有界}}
\end{align}
广义函数$f\in\Cooo^\ast(\RR)$,定义其$n$阶导数 
\begin{align}
	\inp{f^{(n)}}\varphi:=(-1)^n\inp f{\varphi^{(n)}},\quad\forall\varphi\in\Cooo(\RR).
\end{align}
特别的 
\[\int\iti\delta^{(n)}(x)\varphi(x)\d x=(-1)^n\varphi^{(n)}(0).\]

泛函意义下,任意一个广义函数都是无穷阶可导的。
\paragraph*{广义函数的卷积}
给定$f,g\in\Cooo^*(\RR)$的卷积$f\ast g(x)$也是广义函数
\begin{align}
	\inp {f\ast g}\varphi:=\Bigl\langle f(x),\bigl\langle g(y),\varphi(x+y)\bigr\rangle\Bigr\rangle.
\end{align}
一般的,对常系数微分算子$\Dif$
\[\Dif(f\ast g)=\Dif f\ast g=f\ast\Dif g.\]
\paragraph*{广义函数的Fourier变换}
速降函数空间$\Schwsp(\RR)\subset\sC^\infty(\RR)$,%$\forall\varphi\in\Schwsp(\RR)$,有
\begin{align}
	\Schwsp(\RR):=\set\varphi{\lim_{x\to\infty}x^m\varphi^{(n)}(x)=0,\quad\forall m,n\in\ZZ^+}.
\end{align}
比如$\varphi(x)=p(x)\e{-ax^2}$,$p(x)$是多项式,$a>0$,则$\varphi(x)\in\Schwsp(\RR)$。

给定广义函数$f\in\Schwsp^*$,$f$的Fourier变换及其逆变换也是广义函数,定义为
\begin{align}
	\begin{aligned}
		\inp{\cf f}\varphi&=\inp f{\cf\varphi}\\
		\inp{\cfi f}\varphi&=\inp f{\cfi\varphi}
	\end{aligned},\quad\forall\varphi\in\Schwsp.
\end{align}
Fourier变换作用转移到基本函数上,它们保持着经典意义下的基本性质。
\paragraph*{广义函数序列的收敛性}
设广义函数列$\{f_i\}$及广义函数$f$,若对基本函数空间的$\varphi$都有
\[\lim_{n\to\infty}\inp{f_n}\varphi=\inp f\varphi.\]
则称$f_n$弱收敛到$f$,本笔记记作$\lim_{n\to\infty}f_n(x)\circeq f(x)$。

\begin{theorem}{}{}
	若基本函数空间为$\Schwsp$,则
	\[\lim_{n\to\infty}f_n(x)\circeq f(x)\ifnf\lim_{n\to\infty}f_n'(x)\circeq f'(x).\]
\end{theorem}
\paragraph*{高维广义函数}与一维类似,函数空间
\begin{gather*}
	\Cooo(\RR^n)\equiv\sC_0^\infty(\RR^n):=\set f{f\in\sC^\infty(\RR^n),\;\supp f\,\text{有界}},\\
	\Schwsp(\RR^n):=\set\varphi{\lim_{\abs X\to\infty}\abs X^m\frac{\p^k\varphi(X)}{\p x_1^{k_1}\cdots\p x_n^{k_n}}=0,\quad\forall m,k\in\ZZ^+}
\end{gather*}
对函数空间$\mathscr V$上的线性泛函$f$
\[\inp f\varphi=\int_{\RR^n}f(X)\varphi(X)\d X,\quad\forall\varphi\in\mathscr V\]
构成$\mathscr V^\ast$。

广义函数$f\in\Cooo^\ast(\RR^n)$的偏导数定义为
\[\inp{\frac{\p^kf}{\p x_1^{k_1}\cdots\p x_n^{k_n}}}\varphi:=(-1)^k\inp f{\frac{\p^k\varphi}{\p x_1^{k_1}\cdots\p x_n^{k_n}}}\]
特别的,$f=\delta$时
\[\inp{\frac{\p^k\delta}{\p x_1^{k_1}\cdots\p x_n^{k_n}}}\varphi=(-1)^k\frac{\p^k\varphi(\mathbf 0)}{\p x_1^{k_1}\cdots\p x_n^{k_n}}.\]
\begin{example}{}{}
	可以证明
	\begin{align}
		\delta(x_1,\ldots,x_n)=\begin{cases}
			\frac1{2\pi}\lap_2\ln r,&n=2\\
			-\frac1{(n-2)S_{n}}\lap_n\frac1{r^{n-2}},&n\geqslant 3
		\end{cases}
	\end{align}
	其中$r=\abs{\bm x}$,$S_n$为$n$维单位球的表面积。特别的,$S_3=4\pi$。
\end{example}

广义函数的卷积
\[\inp {f\ast g}\varphi:=\Bigl\langle f(X),\bigl\langle g(Y),\varphi(X+Y)\bigr\rangle\Bigr\rangle.\]

\subsection{\textit{Pu} = 0型方程的基本解} % {$\$}
讨论用基本解方法求解方程
\[\Par u(M)=f(M),\quad M\in\RR^n,\]
其中,$\Par$是常系数线性偏微分算子。

视$f,u$为广义函数,它们在广义函数空间里可以自由地进行各种运算和交换。通过这种方式得到的解叫广义函数解,简称作\textbf{广义解}。如果解是一个正则广义函数,甚至还有足够的光滑性,那么这种解是经典解。
\begin{definition}{$\Par u=0$型方程的基本解}{Basic Solution of Pu=0}
	方程
	\begin{align}
		\Par U(M)=\vd(M)
	\end{align}
	的解$U(M)$称作$\Par u=0$型方程的基本解。
\end{definition}
对一般的函数$f$,其对应的解是一般的源所产生的物理场。故基本解也叫点源函数。

若$\Par u=0$型方程有基本解$U(M)$,令
\begin{align}
	u(M):=U\ast f(M)=\int U(M-N)f(N)\d N.
\end{align}
由积分叠加原理
\[\Par u(M)=\int\Par U(M-N)f(N)\d N=\int\delta(M-N)f(N)\d N=f(M).\]
即$u$满足方程$\Par u=f$。
\eg[1] 求方程
\[y'+ay=f(x),\quad a>0\]
的基本解。

基本解$U$满足$U'+aU=\delta(x)$,即
\[\dd x(U\e{ax})=\vd(x)\e{ax}=\delta(x),\thus U(x)=\Heaviside(x)\e{-ax}.\]
因此原方程的解
\[y(x)=U\ast f(x)=\int\iti\Heaviside(\xi)\e{-a\xi}f(x-\xi)\d\xi=\int_{-\infty}^xf(\xi)\e{-a(x-\xi)}\d\xi.\]
\eg[2] 求3维Helmholtz方程
\[\lap_3 u+cu=0\]
的基本解。

由于只有一个点源,方程具有对称性,可设方程有球对称基本解$U(r)$,
当$r>0$时,$\delta(r)=0$,
\[\frac1{r^2}\dd r\kh{r^2\dv Ur}+cU=\frac1r\fkh{\dd[2]r(rU)+crU}=0,\]

当$c=0$时,化为Laplace方程
\[(rU)''=0,\thus U(r)=\frac Ar+\cancel{B}.\]
记$B_r$为半径为$r$的球内部分,则
\[\int_{B_r}\lap U\d V=\int_{B_r}\delta(x,y,z)\d V=1.\]
由Green公式
\[\int_{B_r}\lap U\d V=\oint_{\p B_r}\pv Un\d S=-\frac A{r^2}\cdot 4\pi r^2=1,\thus U(r)=-\frac1{4\pi r}.\]

当$c<0$时,记$k:=\sqrt{-c}$
\[(rU)''-k^2rU=0,\thus U=A\frac{\e{-kr}}r+B\frac{\e{kr}}r.\]
由积分条件
\[\int_{B_r}\lap U-k^2U\d V=\oint_{\p B_r}\pv Un\d S-k^2\int_{B_r}U\d V=1,\]
且
\begin{align*}
	&\quad\oint_{\p B_r}\kh{\pp\rho\frac{\e{k\rho}}\rho}_rr^2\sin\theta\d\theta\nd\phi-k^2\int_{B_r}\frac{\e{k\rho}}\rho\rho^2\sin\theta\d\rho\nd\theta\nd\phi\\
	&=4\pi(kr-1)\e{kr}-4\pi\fkh{(k\rho-1)\e{k\rho}}_0^r=-4\pi.
\end{align*}
故$-4\pi(A+B)=1$,即
\[U(r)=-\frac{\e{-kr}}{4\pi r},\;-\frac{\e{kr}}{4\pi r}.\]

当$c>0$时,记$k:=\sqrt c$
\[U=A\frac{\cos kr}r+B\cancel{\frac{\sin kr}r}.\]
$\sin kr/r$在$\CC$上是整函数,无奇异性,不能作为基本解。
积分$-4\pi A=1$
\[U(r)=-\frac{\cos kr}{4\pi r}.\]
\subsection{Possion方程Green函数法}
首先引入Green公式。
\begin{theorem}{Green公式}{Green Formula}
	设非空有界开集$\Omega\subset\RR^n(n\geqslant 2)$满足边界$\p\Omega$光滑,则
	
	$\forall u,v\in\sC^2(\Omega)\cap\sC(\bar\Omega)$,有
	\begin{gather}\label{Green's 1st formula}
		\int_\Omega v\lap u\d V=\oint_{\p\Omega}v\pv u{\bm n}\d S-\int_\Omega\nabla v\cdot\nabla u\d V;\\\label{Green's 2nd formula}
		\int_\Omega u\lap v-v\lap u\d V=\oint_{\p\Omega}u\pv v{\bm n}-v\pv u{\bm n}\d S.
	\end{gather}
	%其中$\p u/\p\bm n=\nabla u\cdot\bm n$。
\end{theorem}
\prf 由散度定理
\begin{align}
	\int_\Omega\nabla\cdot\bm A\d V=\oint_{\p\Omega}\bm A\cdot\bm n\d S,
\end{align}
取$\bm A=v\nabla u$,
可得式(\ref{Green's 1st formula}):
\[\oint_{\p\Omega}v\pv u{\bm n}\d S\equiv\oint_{\p\Omega}v\nabla u\cdot\bm n\d S=\int_\Omega\div(v\nabla u)\d V=\int_\Omega\nabla v\cdot\nabla u+v\nabla^2u\d V.\]
将$u,v$互换位置并与原式相减,即得式(\ref{Green's 2nd formula})。\qed
\paragraph*{Poisson方程第I边值问题}
\begin{align}
	\begin{cases}
		\lap u=-f(M),&M\in V\subseteq\RR^3\\
		\edg u_{\p V}=\varphi(M),
	\end{cases}
\end{align}
物理上看,这是静电场的基本问题:空间区域$V$内有电荷体密度$\rho=-\varepsilon f$,边界上电位已知为$\varphi$,求$V$内电位$u$。

由叠加原理,$u=v+w$,$v,w$分别满足
\begin{align*}
	\begin{cases}
		\lap v=-f(M),\\
		\edg v_{\p V}=0,
	\end{cases}\quad
	\begin{cases}
		\lap w=0,\\
		\edg w_{\p V}=\varphi(M),
	\end{cases}
\end{align*}
其中$v$表示在边界接地条件下体内电荷产生的电场,$w$表示由边界约束引起的电场。
\begin{definition}{Poisson第I边值问题的Green函数}{}
	定解问题
	\begin{align}
		\begin{cases}
			\lap G(M;N)=-\vd(M-N),\\
			\edg G_{\p V}=0.
		\end{cases}
	\end{align}
	的解$G(M;N)$称为Poisson第I边值问题的Green函数。
\end{definition}
物理上看,Green函数就是边界接地条件下,置于$V$内$N$点电荷为$+\varepsilon$的点源在$V$内$M$点产生的电场,仍然是一个基本解。
\begin{align}
	u(M)=\int_Vf(N)G(M;N)\d N-\oint_{\p V}\varphi(N)\pv Gn\d S.
\end{align}
\prf 
\begin{align*}
	u(M)&=\int_Vu(N)\vd(M-N)\d N=-\int_Vu(N)\,\D G(M;N)\d N\\
	&=-\int_V u\,\D G\d N+\int_VG\,\D u\d N-\int_VG\,\underline{\D u}\d N\tag{Green-II}\\
	&=\oint_{\p V}\cancel{G}\pv un-\underline{u}\pv Gn\d S+\int_VG f\d N\\
	&=-\oint_{\p V}\psi\pv Gn\d S+\int_VGf\d N.\rqed
\end{align*}
Fourier方法是求Green函数的基本方法,但对于一些特殊的区域,可以采用一些特殊方法,如镜像法。

\subparagraph*{镜像法}分解$G=U+g$,$U$满足$\lap U=-\vd(M-N)$,可取
\begin{align*}
	U=\begin{cases}
		-\frac1{2\pi}\ln r,&n=2\\[1ex]
		\frac1{4\pi r},&n= 3
	\end{cases}
\end{align*}
%其中$r$是$M,N$间距离。
而$g$满足 
\[\begin{cases}
	\lap g=0,\\
	\edg g_{\p V}=-U,
\end{cases}\]
是$N$的点电荷在边界上的感应电荷产生的电场。

区域外的点源在$V$内产生的电场满足Laplace方程,可以将边界感应电荷产生的电场$g$看作区域\accentd{外}某些虚设电荷产生的等效电场,这种来源于物理效应的方法叫镜像法。% 它的关键困难在于如何在区域外合适地虚设电荷,对应某些特殊的区域如半空间、球域等等,可以用较直观的方法找到。

\eg[1] 求上半空间Poisson方程第I边值问题的Green函数

$V$内$N(\xi,\eta,\zeta)$点的正电荷$\varepsilon$在空间$(x,y,z)$产生的电场为
\[U_0=\frac1{4\pi r}=\frac1{4\pi\sqrt{(x-\xi)^2+(y-\eta)^2+(z-\zeta)^2}},\]
可虚设电荷$-\varepsilon$于$N$关于$z=0$平面对称的点$M_1(\xi,\eta,-\zeta)$,产生的电场
\[U_1=\frac{-1}{4\pi\sqrt{(x-\xi)^2+(y-\eta)^2+(z+\zeta)^2}},\]
在边界上$\edg{U_0}_{z=0}=-\edg{U_1}_{z=0}$
{\small\[G=\frac1{4\pi}\fkh{\frac1{\sqrt{(x-\xi)^2+(y-\eta)^2+(z-\zeta)^2}}-\frac1{\sqrt{(x-\xi)^2+(y-\eta)^2+(z+\zeta)^2}}}\]}
边界方向导数
\[\edg{\pv Gn}_{\zeta=0}=\edg{-\pv G\zeta}_{\zeta=0}=\frac{-z}{2\pi\fkh{(x-\xi)^2+(y-\eta)^2+z^2}^{3/2}}\]
故对于上半空间Dirichlet问题
\[\begin{cases}
	\lap u=0,&z>0\\
	\edg u_{z=0}=\varphi(x,y)
\end{cases}\]
解的Poisson公式
\[u(x,y,z)=\frac z{2\pi}\int_{\RR^2}\frac{\varphi(\xi,\eta)\d\xi\nd\eta}{\fkh{(x-\xi)^2+(y-\eta)^2+z^2}^{3/2}}\]

二维情况
\[G=\frac1{4\pi}\ln\frac{(x-\xi)^2+(y+\eta)^2}{(x-\xi)^2+(y-\eta)^2},\quad u(x,y)=\frac y\pi\int\iti\frac{\varphi(\xi)\d\xi}{(x-\xi)^2+y^2}\]
\eg[2] 求半径为$R$的球域内Poisson方程第I边值问题的Green函数
\begin{gather*}
	\begin{cases}
		\lap_3G(M;N)=-\vd(M-N),&0\leqslant r<R\\
		\edg G_{r=R}=0
	\end{cases}\\
	\thus \edg{\pv Gn}_{\rho=R}=\frac{R^2-r^2}{4\pi R(R^2+r^2-2Rr\cos\psi)^{3/2}}.
\end{gather*}
故球内Dirichlet问题,解的Poisson形式
\begin{gather*}
	\begin{cases}
		\lap_3u=0,&0\leqslant r<R\\
		\edg u_{r=R}=\varphi(\theta,\phi)
	\end{cases}\\
	\thus u(r,\theta,\phi)=\frac1{4\pi R}\oint_{S_R}\frac{(R^2-r^2)\varphi(\theta_0,\phi_0)\d S_0}{(R^2+r^2-2Rr\cos\psi)^{3/2}}.
\end{gather*}
\subparagraph*{Fourier法}
Fourier法是求Green函数的基本方法,主要思想是按照特征函数作广义Fourier展开,包括分离变量与积分变换。
\eg[1] 求矩形区域$\Omega=[0,L]\times[0,M]$上第I类边值Poisson方程的Green函数
\begin{align*}
	\begin{cases}
		\lap_2G(x,y;\xi,\eta)=-\vd(x-\xi,y-\eta),&x,\xi\in[0,L];\;y,\eta\in[0,M]\\
		G(0,y)=G(L,y)=G(x,0)=G(x,M)=0,
	\end{cases}
\end{align*}
有特征值 
\[G(x,y)=\sum_{m,n}C_{mn}\sin\frac{m\pi x}L\sin\frac{n\pi y}M.\]
系数 
{\small\begin{align*}
	\fkh{\kh{\frac{m\pi}L}^2+\kh{\frac{n\pi}M}^2}C_{mn}&=\frac4{LM}\int_0^M\bs3\int_0^L\vd(x-\xi,y-\eta)\sin\frac{m\pi x}L\sin\frac{n\pi y}M\d x\nd y\\
	&=\frac4{LM}\sin\frac{m\pi\xi}L\sin\frac{n\pi\eta}M.
\end{align*}}
后略。
\eg[2] 求解 
\begin{align*}
	\begin{cases}
		\lap_2u=0,& x>0,y\in[0,a]\\
		u(x,0)=\varphi(x),\enspace u(x,a)=\psi(x),\\
		u(0,y)=0,
	\end{cases}
\end{align*}
有
\begin{align*}
	u(x,y)&=-\fkh{\int\zti\varphi(\xi)\edg{\pv Gn}_{\eta=0}\d\xi+\int\zti\psi(\xi)\edg{\pv Gn}_{\eta=a}\d\xi}\\
	&=\int\zti\varphi(\xi)\edg{\pv G\eta}_{\eta=0}\d\xi-\int\zti\psi(\xi)\edg{\pv G\eta}_{\eta=a}\d\xi
\end{align*}
进而
\begin{align*}
	\pv G\eta&=\sum_{n=1}^\infty4n\int\zti\frac{\sin\omega\xi\sin\omega x}{n^2\pi^2+a^2\omega^2}\d\omega\cdot\cos\frac{n\pi\eta}a\cos\frac{n\pi y}a;\\
	u(x,y)&=\sum_{n=1}^\infty4n\int\zti\bs5\int\zti\frac{\sin\omega\xi\sin\omega x}{n^2\pi^2+a^2\omega^2}\fkh{\varphi(\xi)-(-1)^n\psi(\xi)}\d\omega\nd\xi\cdot\cos\frac{n\pi y}a.
\end{align*}
\paragraph*{$^\ast$Poisson方程第II, III边值问题}
\begin{align}
	\begin{cases}
		\lap u=-f(M),&M\in V\subseteq\RR^n\\
		\kh{\alpha u+\beta\pv un}_{\p V}=\varphi(M),&\alpha,\beta\neq 0.
	\end{cases}
\end{align}
相应的Green函数
\begin{align*}
	\begin{cases}
		\lap G=-\vd(M-N),&M,N\in V\subseteq\RR^n\\
		\kh{\alpha G+\beta\pv Gn}_{\p V}=0,&\alpha,\beta\neq 0.
	\end{cases}
\end{align*}
原问题的解
\begin{align}
	u(M)&=\int_Vf(N)G(M;N)\d N+\frac1\beta\oint_{\p V}\varphi(N)G(M;N)\d S\\
	&=\int_Vf(N)G(M;N)\d N-\frac1\alpha\oint_{\p V}\varphi(N)\pv Gn\d S
\end{align}
$\beta=0$即第I边值问题。$\alpha=0$即第II边值问题,但此时表示内部有热源而边界绝热的稳恒温度场,这是不可能的,因此需要修正为
\begin{align*}
	\begin{cases}
		\lap G=-\vd(M-N)+\frac1v,\\
		\edg{\pv Gn}_{\p V}=0,
	\end{cases}\enspace\text{或}\quad
	\begin{cases}
		\lap G=-\vd(M-N),\\
		\edg{\pv Gn}_{\p V}=-\frac1s,
	\end{cases}
\end{align*}
其中$v,s$为$V$的体积和表面积。有解
\begin{align}
	u(M)=\int_Vf(N)G(M;N)\d N+\oint_{\p V}\varphi(N)G(M;N)\d S+\cns.
\end{align}
相容性条件
\begin{align}
	\int_Vf(M)\d M+\oint_{\p V}\varphi(M)\d S=0.
\end{align}
\subsection{初值问题的基本解方法}
本节主要用基本解方法来求解发展方程,如$u_t=\Par u$型方程
\begin{align}
	\begin{cases}
		u_t=\Par u+f(M,t),&M\in\RR^n,t>0\\
		u(M,0)=\varphi(M).
	\end{cases}
\end{align}
这里所涉及的$\Par$是关于空间变量$M$的常系数线性偏微分算子。

\begin{definition}{$u_t=\Par u$型方程初值问题的基本解}{Basic Solution of ut=Pu}
	基本解$U(M,t)$满足
	\begin{align}
		\begin{cases}
			U_t=\Par U,&M\in\RR^n,t>0\\
			U(M,0)=\vd(M).
		\end{cases}
	\end{align}
\end{definition}
有
\begin{align}
	u(M,t)=U(M,t)\ast\varphi(M)+\int_0^tU(M,t-\tau)\ast f(M,\tau)\d\tau.
\end{align}
证明过程略。

以及$u_{tt}=\Par u$型方程
\begin{align}
	\begin{cases}
		u_{tt}=\Par u+f(M,t),&M\in\RR^n,t>0\\
		u(M,0)=\varphi(M),\enspace u_t(M,0)=\psi(M).
	\end{cases}
\end{align}
\begin{definition}{$u_{tt}=\Par u$型方程初值问题的基本解}{Basic Solution of utt=Pu}
	基本解$U(M,t)$满足
	\begin{align}
		\begin{cases}
			U_{tt}=\Par U,&M\in\RR^n,t>0\\
			U(M,0)=0,\enspace U_t(M,0)=\vd(M).
		\end{cases}
	\end{align}
\end{definition}
有
\begin{align}
	\begin{aligned}
		u(M,t)=&\;U\ast\psi+\pp t(U\ast\varphi)+\int_0^tU(M,t-\tau)\ast f(M,\tau)\d\tau.
	\end{aligned}
\end{align}
\paragraph*{*混合问题的Green函数}对于发展方程的混合问题通常用分离变量法或积分变换法求解,当然也可以用Green函数(点源函数)法。
\eg 一维波动方程的混合问题
\begin{align*}
	\begin{cases}
		u_{tt}=a^2u_{xx}+f(x,t),&x\in(0,L),\;t>0\\
		u(0,t)=u(L,t)=0\\
		u(x,0)=\varphi(x),\quad u_t(x,0)=\psi(x)
	\end{cases}
\end{align*}
Green函数$G(x,t;\xi)$满足 
\begin{align*}
	\begin{cases}
		G_{tt}=a^2G_{xx},&x,\xi\in(0,L),\;t>0\\
		G(0,t)=G(L,t)=0\\
		G(x,0)=0,\quad G_t(x,0)=\vd(x-\xi)
	\end{cases}
\end{align*}
利用Fourier方法可得
\[
	G(x,t;\xi)=\sum\fmto n1\frac2{n\pi a}\sin\frac{n\pi\xi}L\sin\frac{n\pi x}L\sin\frac{n\pi at}L,
\]
进而解
\begin{align}
	u(x,t)=\int_0^L\psi(\xi)G\d\xi+\pp t\int_0^L\varphi(\xi)G\d\xi+\int_0^t\bs4\int_0^L f(\xi,\tau)G(x,t-\tau;\xi)\d\xi\nd\tau
\end{align}

\input{Append.tex}
\end{document}