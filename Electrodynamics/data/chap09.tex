\chapter{相对论电动力学}
\label{chap:relativistic electromagnetic}
%三维一般波动方程
%\lapla u-\frac1{a^2}\pv[2]ut=-f(\bm x,t),
%中的$a$是波速,依惯性系而不同。
\paragraph{Michelson-Morley实验}
人们最开始假设光和声音一样,都需要某种介质来支持它的传播。最初假想光是在一种叫作以太(ether)的媒介中传播的,光在相对于以太以不同状态运动时会有不同的速度。为了验证以太是否存在以及光速是否是常数,1887年Michelson和Morley设计了如下所示的光学干涉装置:
\begin{center}
	\begin{tikzpicture}[scale=1.2]
		\draw[->-](0, 0)--(2, 0)node[shift={(.2, -.3)}]{$O$};
		\draw[thick](2-.5, -.5)--(2.5, .5)node[above right]{BS};
		\draw[->-](2, 0)--(5, 0)node[midway, below]{$c+u$};
		\draw[-<-](2.1, .1)--(5, .1)node[midway, above]{$c-u$};
		\fill[gray!50](5, .5)rectangle(5.2, -.5);
		\draw[thick](5, -.5)--(5, .5)node[above]{M$_1$};
		\draw[->-](2, 0)--(2, 3)node[midway, left]{$\sqrt{c^2-u^2}$};
		\draw[-<-](2.1, .1)--(2.1, 3);
		\fill[gray!50](1.5, 3)rectangle(2.5, 3.2);
		\draw[thick](1.5, 3)--(2.5, 3)node[right]{M$_2$};
		\draw[<-](3.5, 1.5)--(4.5, 1.5)node[midway, below]{$u$}node[midway, above]{ether wind};
	\end{tikzpicture}
	\captionof{figure}{Michelson-Morley光学干涉仪的实验装置图}
	\label{fig:Michelson-Morley}
\end{center}
图中最左侧相干光源发出的光在经过分束器(beam splitter, BS)中心$O$后,其中一半的光被透射后抵达镜子M$_1$然后被反射回$O$点,另一半的光被反射后抵达镜子M$_2$然后被反射回$O$点,并与从M$_1$反射回来的光发生干涉。假设整个装置在以太中以速度$u$向右运动,那么这等价于装置不动,而以太风以速度$u$向左吹来。%我们可以把“以太风”想象成是一条水流速度为$u$向左均匀流动的河流,把光想象成是一艘固有速度是$c$的船。那么
容易得出光在水平方向从$O$到M$_1$再返回$O$的路径上所花费的时间是:
\[
	T_1=\frac L{c-u}+\frac L{c+u}=\frac{2L/c}{1-u^2/c^2}.
\]
同理,容易得出光在竖直方向上从$O$到M$_2$再返回$O$的路径上所花费的时间是:
\[
	T_2=\frac{2L}{\sqrt{c^2-u^2}}=\frac{2L/c}{\sqrt{1-u^2/c^2}}.
\]
容易发现$T_1\neq T_2$。由于装置在以太中的运动速度$u\ll c$,%对小量做Taylor展开得出光沿着两条路径上传播的时间差是:
\[
	T_1-T_2=\frac{2L}c\biggkh{1+\frac{u^2}{c^2}}-\frac{2L}c\biggkh{1+\frac{u^2}{2c^2}}+\bigo(u^4)=\frac{Lu^2}{c^3}+\bigo(u^4).
\]
由于存在光程差,$O$点上应当观测到干涉条纹,且随着$u$的改变,干涉条纹也应当发生偏移。
然而,Michelson-Morley的实验结果发现光学干涉条纹并未发生任何改变,%也就是说上述的时间差结果是0!
% 而且把整个实验装置旋转一定角度后发现干涉条纹依然没有发生任何改变!
这也就从实验上证明了以太并不存在,光速恒为常数$c$!
\section{狭义相对论}
此节仅介绍必要内容,而略去狭义相对论的大部分推导内容。%如尺缩效应、钟慢效应等。
\subsection{Lorentz变换}
给定惯性系$S$和$S'$,$S'$系相对$S$系有$x$方向的相对速度$v$。
同一事件(event)在$S,S'$系中的坐标分别为$(ct,x,y,z)$和$(ct',x',y',z')$,则Lorentz变换给出:
%\footnote{为了之后计算方便,也为了更好地凸显出理论的对称性,我们把$c$这个常数定义成1,也就是把光速$c$定义成一个标准单位.这样处理后时间和空间的量纲也统一起来了}
\begin{equation}
	\begin{bmatrix}
		ct'\\x'\\y'\\z'
	\end{bmatrix}=
	\begin{bmatrix}
		\gamma&\gamma\beta\\-\gamma\beta&\gamma\\ &&1\\ &&&1
	\end{bmatrix}
	\begin{bmatrix}
		ct\\x\\y\\z
	\end{bmatrix}
\end{equation}
其中相对速度$\beta\equiv v/c$,Lorentz因子
\[
    \gamma\equiv\frac1{\sqrt{1-\beta^2}}.
\]

如果质点在$S$系中有$x$方向的速度$u=\d x/\d t$,则其在$S'$中的速度为
\begin{equation}
    u'=\dv{x'}{t'}=\frac{\gamma(\d x-v\d t)}{\gamma(\d t-v\d t/c^2)}=\frac{u-v}{1-vu/c^2}.
\end{equation}

\paragraph{四矢量}

为讨论形式的统一,我们定义四矢量(4-vector)。

\begin{definition}{四矢量}{4-vector}
    定义逆变量(contravariant)形如
    \begin{equation}
        a^\mu\equiv(a^0,a^1,a^2,a^3)\tp,
    \end{equation}
    % 称$a^0$是时间(temporal)分量。

    与之对应的是协变量(covariant),形如
    \begin{equation}
        a_\mu\equiv(a_0,a_1,a_2,a_3).
    \end{equation}
\end{definition}
时空连续统(continuum)由位置四矢量表示:
\begin{equation}
    x^\mu\equiv(ct,x,y,z)\tp.
\end{equation}
\begin{definition}{Einstein求和约定}{Einstein summation convetion}
    表达式上下标中出现重复指标时,表示对其遍历求和,即
    \begin{equation}
        a_\mu b^\mu\equiv\sum_\mu a_\mu b^\mu,
    \end{equation}
    因此Einstein约定:这个求和符号是可以省略的。
\end{definition}
% 使用Einstein求和约定,定义四维标量积
% \[
%     a_\mu b^\mu:=a_0b^0+a_1b^1+a_2b^2+a_3b^3\equiv a^\mu b_\mu.
% \]
则Lorentz变换可以表示为
\[
    \bar x^\mu=\Lambda^\mu{}_\nu x^\nu,\quad\Lambda^\mu{}_\nu\equiv
    \begin{bmatrix}
		\gamma&\gamma\beta\\-\gamma\beta&\gamma\\ &&1\\ &&&1
	\end{bmatrix}
\]

\paragraph{时空结构}

狭义相对论时空的具体几何结构由间隔(interval)确定
\[
    s^2=-c^2(t_A-t_B)^2+(x_A-x_B)^2+(y_A-y_B)^2+(z_A-z_B)^2.
\]
可以验证:间隔在Lorentz变换下是不变的:
\footnote{
不变量(invariant)表示在所有惯性系中均保持相同的值;守恒量(conserved quantity)表示在某个过程前后保持相同的值。比如,质量是不变量,却不守恒;能量是守恒的,但不是不变量;电荷量既是不变量,又是守恒量;而速度既不是不变量,也不守恒。

在每个封闭系统中,总的相对论能量和动量都是守恒的。
}
\[
    (s^2)'=-c^2(t_A'-t_B')^2+(x_A'-x_B')^2+(y_A'-y_B')^2+(z_A'-z_B')^2=s^2.
\]
\begin{itemize}
    \item 若$s^2<0$,则称间隔是类时的(timelike),因为这是同地不同时事件间隔的符号;
    \item 若$s^2>0$,则是类空的(spacelike),因为这是同时不同地事件间隔的符号;
    \item 若$s^2=0$,则是类光的(lightlike),因为这是当两个事件由一个以光速传播的信号连接时所保持的关系。
\end{itemize}
在微分形式上,无穷小间隔是
\begin{equation}
    \d s^2=g_{\mu\nu}\d x^\mu\nd x^\nu.
\end{equation}
式中$g_{\mu\nu}$是度规张量(metric tensor)。对于狭义相对论的平坦时空(广义相对论中为弯曲时空),采取Minkowski度规:
\footnote{Minkowski度规有两种:
\begin{itemize}
    \item 东海岸度规$\eta_{\mu\nu}=\diag(-1,1,1,1)$,常用于弦论和宇宙学,其好处是空间分量上的度规分量和我们平时习惯的Euclide空间的度规相同,Griffiths采用此;
    \item 西海岸度规$\eta_{\mu\nu}=\diag(1,-1,-1,-1)$,常用于粒子物理学,Jackson采用此;
    \item 其实还有几乎没人喜欢的$\i ct$。
\end{itemize}
}
\begin{equation}
    g_{\mu\nu}\equiv
    \begin{bmatrix}
        -1\\ &1\\ &&1\\ &&&1
    \end{bmatrix}.
\end{equation}
逆变度规张量$g^{\mu\nu}$应满足
\begin{equation}
    g_{\mu\nu}g^{\nu\rho}=\delta_\mu{}^\rho.
\end{equation}
% 为$g_{\mu\nu}$的归一化余子式。
对平坦时空来说,二者相同$g_{\mu\nu}=g^{\mu\nu}$。

逆变坐标$x_\mu$可由协变坐标$x^\mu$和度规张量得到
\begin{equation}
    x_\mu=g_{\mu\nu}x^\nu.
\end{equation}
同样有$x^\mu=g^{\mu\nu}x_\nu$,写成分量形式就是
\[
    x_0=-x^0,\enspace x_1=x^1,\enspace x_2=x^2,\enspace x_3=x^3.
\]

\subsection{相对论动力学}

\paragraph{相对论运动学}

由于钟慢效应(time dilation),在静止系中的时间流速是最慢的,可定义
固有时间(proper time)
\[
    \d\tau=\frac{\d t}{\sqrt{1-u^2/c^2}}\equiv\gamma_u\d t.
\]
进而定义四矢量速度
\begin{equation}
    \eta^\mu\equiv\dv{x^\mu}\tau,
\end{equation}
和四矢量动量
\begin{equation}
    p^\mu\equiv m\eta^\mu.
\end{equation}
其中$m$为(静止)质量。

$p^\mu$的时间分量是$p^0=\gamma_umc$,对应相对论能量:
\[
    E=\gamma_umc^2,
\]
静止时,$\gamma_u=1,\enspace E=mc^2$,二者之差定义为动能
\[
    \Ek\equiv E-mc^2=\frac12mu^2+\frac38\frac{mu^4}{c^2}+\frac5{16}\frac{mu^6}{c^4}+\bigo(u^8),
\]
相对论动量
\[
    \bm p=m\bm\eta=m\dv{\bm x}\tau=\gamma_um\dv{\bm x}t=\gamma_um\bm u.
\]
有关系
\begin{equation}
    E^2=p^2c^2+m^2c^4.
\end{equation}
\begin{example}{Compton散射}{Compton scattering}
    频率为$\nu$的光子与一个静止的电子发生Compton散射,散射后光子的频率变为$\nu'$,散射角为$\theta$,反冲电子动量大小为$p_\elc$,反冲角为$\phi$。
    \begin{center}
        \usetikzlibrary{decorations.pathmorphing}
        \begin{tikzpicture}
            \coordinate (O) at (0, 0);
            \coordinate (x) at (3, 0);
            \coordinate (v) at (3, 2);
            \coordinate (e) at (2, -2);
            \draw[->, thick, gray, decorate, decoration={
                snake, amplitude=.4mm, segment length=2mm, post length=1mm}]
                (-3, 0)node[above]{$h\nu$}--(O);
            \draw[->, thick, decorate, decoration={
                snake, amplitude=.4mm, segment length=4mm, post length=1mm}]
                (O)--(v)node[right]{$h\nu'$};
            \draw[->, thick](O)--(e)node[right]{e$^-$};
            \draw[dashed](O)--(x);
            \pic[draw, "$\theta$", angle radius=8mm, angle eccentricity=1.2]{angle=x--O--v};
            \pic[draw, "$\phi$", angle radius=10mm, angle eccentricity=1.2]{angle=e--O--x};
        \end{tikzpicture}
        \captionof{figure}{Compton散射示意图}
    \end{center}
    由横向、纵向的动量守恒以及能量守恒:
    \begin{subequations}
        \begin{align}
            \frac{h\nu}c&=\frac{h\nu'}c\cos\theta+p_\elc\cos\phi,\\
            0&=\frac{h\nu'}c\sin\theta-p_\elc\sin\phi,\\
            h\nu+m_\elc c^2&=h\nu'+\sqrt{(m_\elc c)^2+(p_\elc c)^2}.
        \end{align}
    \end{subequations}
    解得
    \begin{equation}
        h\nu'=\frac{h\nu}{1+\alpha(1-\cos\theta)},\quad\alpha\equiv\frac{h\nu}{m_\elc c^2}.
    \end{equation}
    将上式改写成波长的形式,得到Compton移动
    \begin{equation}
        \D\lambda=\frac h{m_\elc c}(1-\cos\theta).
    \end{equation}
\end{example}

\paragraph{相对论动力学}

相对论中的力依然遵循Newton第二定律中的定义:
\[
    \bm F\equiv\dv{\bm p}t,
\]
满足
\begin{equation}
    \dv{\bm p}t\cdot\bm u=\dd t\frac{m\bm u}{\sqrt{1-u^2/c^2}}\cdot\bm u=\dv Et.
\end{equation}
因此功能关系:
\begin{equation}
    W\equiv\int\bm F\cdot\d\bm\ell=\int\dv{\bm p}t\cdot\bm u\d t=\D E.
\end{equation}
而Newton第三定律在不同地的时候不再成立,因为在相对论中,同时性是相对的。

可得力在Lorentz变换中的关系
\begin{subequations}
    \begin{align}
        F_x'&=\frac{F_x-v\bm u\cdot\bm F/c^2}{1-vu_x/c^2},\\
        F_y'&=\frac{F_y}{\gamma(1-vu_x/c^2)},\\
        F_z'&=\frac{F_z}{\gamma(1-vu_x/c^2)}.
    \end{align}
\end{subequations}
如果质点在$S$系中保持静止($u=0$),则其受到的力的变换为:
\begin{equation}
    \bm F_\perp'=\frac1\gamma\bm F_\perp,\qquad F_\parallel'=F_\parallel.
\end{equation}
\section{相对论电动力学}
为什么会有磁场?利用静电学和相对论,我们可以计算出载电流线和运动电荷之间的磁力,而不需要援引磁力定律。
\begin{example}{导线的磁场力}{}
    $S$系中有一导线,其中正负电荷静止时的线密度为$\pm\lambda_0$,通电正负电荷后在导线中反向运动,速度为$v_\pm=\pm v$,则正负电荷的线密度为$\lambda_\pm=\pm\gamma\lambda_0$,电流为$I=2\lambda v$,导线外距离$r$处有一电荷$q$沿平行于导线的方向以速度$u<v$运动。

    %$S$系中导线没有净电荷,因此电荷$q$不受电场力。
    \begin{center}
        \begin{tikzpicture}[xscale=.8]
            \draw[->, thick](-3, 2.3)node[left]{$\lambda$}--(-2, 2.3)node[right]{$\bm v$};
            \draw[->, thick](3, 1.7)node[right]{$-\lambda$}--(2, 1.7)node[left]{$-\bm v$};
            \draw[thick](-3, 2)--(3, 2);
            \draw[dashed](0, 0)--(0, 2)node[midway, right]{$r$};
            \draw[->, thick](0, 0)--(.5, 0)node[above]{$\bm u$};
            \fill[black](0, 0)circle(0.1)node[left]{$q$};
        \end{tikzpicture}
        \begin{tikzpicture}[xscale=.8]
            \draw[->, thick](-3, 2.3)node[left]{$\lambda$}--(-2, 2.3)node[right]{$\bm v$};
            \draw[->, thick](3, 1.7)node[right]{$-\lambda$}--(2, 1.7)node[left]{$-\bm v$};
            \draw[thick](-3, 2)--(3, 2);
            \draw[dashed](0, 0)--(0, 2)node[midway, right]{$r$};
            \fill[black](0, 0)circle(0.1)node[left]{$q$};
        \end{tikzpicture}
        \captionof{figure}{$S$系和$S'$系中导线与电荷示意图}
    \end{center}
    转换到粒子静止的参考系$S'$,则电荷$q$不受磁场力,仅受电场力。
    $S'$系中正负电荷的速度为
    \[
        v_\pm=\frac{v\mp u}{1\mp vu/c^2},
    \]
    正负电荷线密度变为$\lambda_\pm=\pm\gamma_{v_\pm}\lambda_0$,总电荷线密度和电场为
    \[
        \lambda\tot=-\frac{2vu}{c^2}\gamma_u\lambda,\quad E=\frac{\lambda\tot}{2\pi\varepsilon_0r}.
    \]
    因此$S'$系中电荷所受的电场力为
    \[
        F'=qE=-\frac{vu\gamma_u\lambda}{c^2}\frac q{\pi\varepsilon_0r}.
    \]
    变回到$S$系中,
    \[
        F=\frac1{\gamma_u}F'=-\frac{vu\lambda}{c^2}\frac q{\pi\varepsilon_0r}=-qu\frac{\mu_0I}{2\pi}.
    \]
    而$S$系中导线没有净电荷,$q$不受电场力,故此力只能是磁场力!
\end{example}
\paragraph{场的变换}
假设:
\begin{compactenum}
	\item 电荷是不变的;
	\item 无论场是如何产生的,转换规则都是相同的。
\end{compactenum}
得到
\begin{align}
    E_x'&=E_x,&E_y'&=\gamma(E_y-vB_z),&E_z'&=\gamma(E_z+vB_y);\\
    B_x'&=B_x,&B_y'&=\gamma\Bigkh{B_y+\frac v{c^2}E_z},&B_z'&=\gamma\Bigkh{B_z-\frac v{c^2}E_y}.
\end{align}
\begin{example}{匀速运动电荷的场}{E, B of q with v}
    考虑$S'$系中,电荷是静止的,没有磁场只有电场:
    \[
        E'=\frac1{4\pi\varepsilon_0}\frac q{r'^2}\uvec r',
    \]
    变换到原系$S$:
    \[
        E_x=E_x',\quad E_y=\gamma E_y',\quad E_z=\gamma E_z',
    \]
    得到电场为$\uvec r$方向,但并不各向同性:
    \begin{equation}
        \bm E=\frac1{4\pi\varepsilon_0}\frac{1-\beta^2}{(1-\beta^2\sin^2\theta)^{3/2}}\frac q{r^2}\uvec r.
    \end{equation}
    磁场为$\uvec\phi$方向:
    \begin{equation}
        \bm B=\frac{\bm v}{c^2}\times\bm E=\frac{\mu_0}{4\pi}\frac{(1-\beta^2)\sin\theta}{(1-\beta^2\sin^2\theta)^{3/2}}\frac{qv}{r^2}\uvec\phi.
    \end{equation}
\end{example}
\subsection{电磁场张量}
电磁场的变换是由一个反对称的二阶张量连接起来的
\[
    t^{\mu\nu}=
    \begin{bmatrix}
        0&t^{01}&t^{02}&t^{03}\\
        -t^{01}&0&t^{12}&t^{13}\\
        -t^{02}&-t^{12}&0&t^{23}\\
        -t^{03}&-t^{13}&-t^{23}&0
    \end{bmatrix}.
\]
变换为 
\[
    \bar t^{\mu\nu}=\Lambda^\mu{}_\rho\Lambda^\nu{}_\sigma t^{\rho\sigma}.
\]
对应分量的变换规则为
\begin{align*}
    \bar t^{01}&=t^{01},&\bar t^{02}&=\gamma(t^{02}-\beta t^{12}),&\bar t^{03}&=\gamma(t^{03}+\beta t^{31}),\\
    \bar t^{23}&=t^{23},&\bar t^{31}&=\gamma(t^{31}+\beta t^{03}),&\bar t^{12}&=\gamma(t^{12}-\beta t^{02}).
\end{align*}
与场变换比较,可得电磁张量
\begin{equation}
    F^{\mu\nu}=
    \begin{bmatrix}
        0&E_x/c&E_y/c&E_z/c\\
        -E_x/c&0&B_z&-B_y\\
        -E_y/c&-B_z&0&B_x\\
        -E_z/c&B_y&-B_x&0
    \end{bmatrix}.
\end{equation}
或对偶(dual)张量
\begin{equation}
    G^{\mu\nu}=
    \begin{bmatrix}
        0&B_x&B_y&B_z\\
        -B_x&0&-E_z/c&E_y/c\\
        -B_y&E_z/c&0&-E_x/c\\
        -B_z&-E_y/c&E_x/c&0
    \end{bmatrix}.
\end{equation}
\paragraph{张量符号中的电动力学}
用张量的语言重新表述电动力学定律(Maxwell方程和Lorentz力)。统一电荷源$\rho$和电流源$\bm J$,定义四矢量电流源
\begin{equation}
    J^\mu=(c\rho,J_x,J_y,J_z)\tp,
\end{equation}
\begin{definition}{四维导数}{}
    定义 
    \begin{equation}
        \p_\mu\equiv\pp{x^\mu},\quad\p^\mu\equiv\pp{x_\mu}.
    \end{equation}
    有
    \begin{equation}
        \p_\mu a^\mu=\p_0a^0+\p_1a^1+\p_2a^2+\p_3a^3=\pv{a^0}{x^0}+\pv{a^1}{x^1}+\pv{a^2}{x^2}+\pv{a^3}{x^3}.
    \end{equation}
\end{definition}
连续性方程\eqref{eqn:continuity}变为 
\[
    \p_\mu J^\mu=\pv\rho t+\div\bm J=0,
\]
Maxwell方程组变为 
\begin{subequations}
    \begin{align}
        \p_\nu F^{\mu\nu}&=\mu_0J^\mu,\\
        \p_\nu G^{\mu\nu}&=0.
    \end{align}
\end{subequations}
$\mu=0$给出 
\begin{subequations}
    \begin{align}
        \div\bm E&=\rho/\varepsilon_0,\\
        \div\bm B&=0;
    \end{align}
\end{subequations}
$\mu=1,2,3$给出$x,y,z$分量 
\begin{subequations}
    \begin{align}
        \curl\bm B-\mu_0\varepsilon_0\pv{\bm E}t&=\mu_0\bm J,\\
        \curl\bm E+\pv{\bm B}t&=\bm0.
    \end{align}
\end{subequations}

定义四矢量Minkowski力 
\begin{equation}
    K^\mu:=q\eta_\nu F^{\mu\nu},
\end{equation}
得到Lorentz力的表达式
\[
    \bm F=q(\bm E+\bm u\times\bm B),
\]

\subsection{相对论势}

将标量势和矢量势的定义:
\begin{align*}
    \bm E&=-\nabla\Phi-\pv{\bm A}t,\\
    \bm B&=\curl\bm A,
\end{align*}
展开,可见对称性:
\begin{alignat*}{2}
    E_x&=-\pv\Phi x-\pv{A_x}t,&\qquad B_x&=\pv{A_y}z-\pv{A_z}y,\\
    E_y&=-\pv\Phi y-\pv{A_y}t,&B_y&=\pv{A_z}x-\pv{A_x}z,\\
    E_z&=-\pv\Phi z-\pv{A_z}t,&B_z&=\pv{A_x}y-\pv{A_y}x,
\end{alignat*}

四矢量势
\begin{equation}
    A^\mu:=(\Phi/c,A_x,A_y,A_z)\tp,
\end{equation}
则电磁张量可以写成
\begin{equation}
    F^{\mu\nu}=\p^\mu A^\nu-\p^\nu A^\mu.
\end{equation}
% 注意这里是协变量$x_\mu$,$x_0=-ct$有一个负号。
\begin{example}{匀速运动电荷的场·续}{E, B of q with v, II}
    续\exmref{exm:E, B of q with v},我们用势的方法再做一遍。$S'$系中电荷静止 
    \[
        \Phi'=\frac1{4\pi\varepsilon_0}\frac q{r'},\quad\bm A'=\bm 0,
    \]
    变回$S$系
    \[
        \Phi=\gamma\Phi',\quad A_x=\gamma\frac v{c^2}\Phi',\quad A_y=A_z=0,
    \]
    其中 
    \[
        r'=\sqrt{x'^2+y'^2+z'^2}=\sqrt{\gamma^2x^2+y^2+z^2}.
    \]
    则
    \[
        \bm E=-\nabla\Phi-\pv{\bm A}t=\frac1{4\pi\varepsilon_0}\frac{\gamma q\bm r}{(\gamma^2x^2+y^2+z^2)^{3/2}}.
    \]
\end{example}
利用四矢量势,Maxwell方程组变为
\[
    \p^\mu\p_\nu A^\nu-\p^\nu\p_\nu A^\mu=\mu_0J^\mu,
\]
使用Lorenz规范
\[
    \p_\mu A^\mu=\frac1{c^2}\pv\Phi t+\div\bm A=0.
\]
由此我们得到了Maxwell方程组最简单的形式:
\begin{equation}
    \p^\nu\p_\nu A^\mu=-\mu_0J^\mu.
\end{equation}
