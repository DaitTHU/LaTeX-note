\chapter{Maxwell方程组}
\label{chap:Maxwell}
\section{Maxwell位移电流}
在前面的讨论中,我们已经得到了关于电场和磁场的基本定律,分别是:
\begin{align*}
    \text{电场Gauss定律}&\,\eqref{eqn:divD}&\div\bm D&=\rho,\\
    \text{磁场Gauss定律}&\,\eqref{eqn:divB-macro}&\div\bm B&=0,\\
    \text{Ampère定律}&\,\eqref{eqn:curlH}&\curl\bm H&=\bm J,\\
    \text{Faraday电磁感应定律}&\,\eqref{eqn:curlE-B}&\curl\bm E&=-\pv{\bm B}t.
\end{align*}
%除了Faraday定律外,所有定律都是在稳恒态条件下得到的。
但在\chapref{chap:magnetostatics}推导Ampère定律时,我们设定了稳恒电流条件$\div\bm J=0$,这在非稳恒态问题中便显然不适用了,而应该修正为连续性方程\eqref{eqn:continuity}:%与连续性方程不一致了。
%Maxwell受Faraday观察结果的启发,发现了方程组中的矛盾。
\[
    \pv\rho t+\div\bm J=\div\biggkh{\pv{\bm D}t+\bm J}=0.
\]
上式利用了电场的Gauss定律,从而Ampère定律也被修正为
\begin{equation}
    \label{eqn:curlH-JD}
    \curl\bm H=\bm J+\pv{\bm D}t.
\end{equation}
Maxwell将附加项$\p\bm D/\p t$称为位移电流(displacement current)。
\section{Maxwell方程组}
修正Ampère定律后,我们便得到了Maxwell方程组:\footnote{事实上Maxwell原稿中的方程组是分量形式,并不简洁,其微分形式是由Heaviside给出的。}
\begin{subequations}
    \begin{align}
        \label{eqn:Maxwell divD}
        \div\bm D&=\rho,\\
        \label{eqn:Maxwell divB}
        \div\bm B&=0,\\
        \label{eqn:Maxwell curlE}
        \curl\bm E&=-\pv{\bm B}t,\\
        \label{eqn:Maxwell curlH}
        \curl\bm H&=\bm J+\pv{\bm D}t.
    \end{align}
\end{subequations}
将Maxwell方程和Lorentz力
\begin{equation}
    \label{eqn:Lorentz}
    \bm F=q(\bm E+\bm v\times\bm B),
\end{equation}
以及Newton第二定律结合起来,就可以完全描写相互作用 带电粒子和电磁场的经典动力学。
%Maxwell方程组是非常自洽、有非常深刻的物理背景。
\begin{example}{真空中电磁波}{electromagnetic wave}
    在无源的真空中
    \[
        \rho=0,\enspace\bm J=\bm 0;\enspace\bm D=\varepsilon_0\bm E,\enspace\bm H=\bm B/\mu_0;
    \]
    利用式\eqref{eqn:curlcurl},可得
    \begin{align*}
        \lapla\bm E&%=\nabla(\div\bm E)-\curl(\curl\bm E)=0+\pp t(\curl\bm B)
        =\mu_0\varepsilon_0\pv[2]{\bm E}t,\enspace
        \lapla\bm B%&=\nabla(\div\bm B)-\curl(\curl\bm B)=0-\mu_0\varepsilon_0\pp t(\curl\bm B)
        =\mu_0\varepsilon_0\pv[2]{\bm B}t.
    \end{align*}
    因此真空中电场和磁场服从波动方程(wave equation),有平面波解,且波速为
    \[
        \frac1{\sqrt{\mu_0\varepsilon_0}}=\SI{299 792 458}{m/s},
    \]
    与真空中光速$c$相同!这意味着光就是电磁波(electromagnetic wave)。Maxwell方程组将电、磁和光作为同一现象的表现形式。
\end{example}
\subsection{标量势和矢量势}
Maxwell方程组是4个联立的一阶偏微分方程组,它们将电场和磁场的各个分量联系起来。在简单情况下这一方程组可以解出来,但是引入势往往更方便。

利用没有磁荷这一事实\eqref{eqn:Maxwell divB},我们引入矢量势$\bm A$满足 
\begin{equation}
    \label{eqn:curlA}
    \bm B=\curl\bm A,
\end{equation}
继而式\eqref{eqn:Maxwell curlE}变为 
\[
    \curl\bm E+\pv{\bm B}t=\curl\biggkh{\bm E+\pv{\bm A}t}=\bm 0,
\]
因此可以引入矢量势$\Phi$使得
\begin{equation}
    \label{eqn:-nablaPhi-pApt}
    \bm E=-\nabla\Phi-\pv{\bm A}t.
\end{equation}
剩下的式\eqref{eqn:Maxwell divD}和\eqref{eqn:Maxwell curlH}可变形为(考虑微观情形而非宏观平均)
\begin{subequations}
    \begin{align}
        \lapla\Phi+\pp t(\div\bm A)&=-\frac\rho{\varepsilon_0},\\
        \lapla\bm A-\frac1{c^2}\pv[2]{\bm A}t-\nabla\biggkh{\div\bm A+\frac1{c^2}\pv\Phi t}&=-\mu_0\bm J.
    \end{align}
\end{subequations}

\paragraph{规范变换}

为了解耦合(uncouple)标量势$\Phi$和矢量势$\bm A$,我们可以利用势的定义里隐含的任意性,即电场和磁场在下列变换中保持不变:
\begin{align}
    \bm A&\mapsto\bm A+\nabla\varLambda,\\
    \Phi&\mapsto\Phi-\pv\varLambda t.
\end{align}
上式称为规范变换,场在规范变换下的不变性称为规范不变性(gauge invariance)。因此,我们总能将不满足某规范的$\Phi$和$\bm A$通过规范变换使得新的$\Phi'$和$\bm A'$满足此规范。
\paragraph{Lorenz规范}
一种直接的想法是令梯度内的项为0
\begin{equation}
    \div\bm A+\frac1{c^2}\pv\Phi t=0,
\end{equation}
这称为Lorenz规范\footnote{注意区分Lorentz和Lorenz。},从而Maxwell方程组可以写成
\begin{subequations}
    \begin{align}
        \lapla\Phi-\frac1{c^2}\pv[2]\Phi t&=-\frac\rho{\varepsilon_0};\\
        \lapla\bm A-\frac1{c^2}\pv[2]{\bm A}t&=-\mu_0\bm J.
    \end{align}
\end{subequations}
上面两式其实形式是一样的,可定义D'Alembertian%\footnote{其实就是四维赝Rieman}
\begin{equation}
    \square:=\frac1{c^2}\pp[2]t-\lapla.
\end{equation}
若规范变换前后均满足Lorenz规范,则
\begin{equation}
    \lapla\varLambda-\frac1{c^2}\pv[2]\varLambda t=0.
\end{equation}
\paragraph{Coulomb规范}
另一种规范是\secref{sec:vector potential}提到的Coulomb规范
\[
    \div\bm A=0,
\]
得到的方程为
\begin{subequations}
    \begin{align}
        \label{eqn:Coulomb gauge Phi}
        \lapla\Phi&=-\frac\rho{\varepsilon_0};\\
        \label{eqn:Coulomb gauge A}
        \lapla\bm A-\frac1{c^2}\pv[2]{\bm A}t&=-\mu_0\bm J+\frac1{c^2}\pp t\nabla\Phi.
    \end{align}
\end{subequations}

式\eqref{eqn:Coulomb gauge Phi}是Poisson方程,可直接得到
\[
    \Phi(\bm x,t)=\frac1{4\pi\varepsilon_0}\int_V\frac{\rho(\bm x',t)}{|\bm x-\bm x'|}\d v'
\]
这说明:标量势$\Phi(\bm x,t)$是由电荷密度$\rho(\bm x',t)$产生的瞬时Coulomb势\footnote{这是其被称为Coulomb规范的原因。},%二者在不同地方同时改变,
在空间中的传播没有任何延迟。

对于式\eqref{eqn:Coulomb gauge A},将电流密度$\bm J$写成纵向(longitudinal,或无旋irrotational)电流$\bm J_\text l$和横向(transverse,或螺旋solenoidal)电流$\bm J_\text t$之和,满足
\[
    \curl\bm J_\text l=\bm 0,\quad\div\bm J_\text t=0.
\]
从而将式\eqref{eqn:Coulomb gauge A}解耦合为(略去推导过程)
\begin{align*}
    \frac1{c^2}\pp t\nabla\Phi&=\mu_0\bm J_\text l;\\
    \lapla\bm A-\frac1{c^2}\pv[2]{\bm A}t&=-\mu_0\bm J_\text t.
\end{align*}
这说明:标量势方程的源可以用纵向电流$\bm J_\text l$表示;而矢量势满足波动方程,在空间中以光速$c$传播,其源可以用横向电流$\bm J_\text t$表示。在Coulomb规范下,横向辐射场仅由矢量势给出,而瞬时Coulomb势只对近场有贡献。

\begin{example}{Coulomb规范下电磁波的表达}{em wave in Coulomb gauge}
    无源情况下,第一个方程$\Phi=0$,第二个方程
    \[
        \lapla\bm A-\frac1{c^2}\pv[2]{\bm A}t=\bm 0\implies\bm A=\bm A_0\e{\i(\bm k\cdot\bm x-\omega t)}.
    \]
    从而
    \begin{align*}
        \bm B&=\curl\bm A=\i\bm k\times\bm A;\\
        \bm E&=0-\pv{\bm A}t=\i\omega\bm A.
    \end{align*}
\end{example}
\paragraph{磁荷存在吗?}
如果自由磁荷真的存在,Maxwell方程组就要改写
\begin{align*}
    \div\bm D&=\rho_\elc,\\
    \div\bm B&=\rho_\text m,\\
    \curl\bm E&=-\bm J_\text m-\pv{\bm B}t,\\
    \curl\bm H&=\bm J_\elc+\pv{\bm D}t.
\end{align*}
我们需要新的理论,标量势和矢量势的形式也需要改写。
\footnote{事实上,无论磁荷存在与否,理论物理学家都可以提出许多漂亮的理论,但是决定磁荷存在性仍需要实验物理学家的实验验证。举个例子:Higgs粒子是粒子物理学标准模型(standard model)预言的粒子,也有一些Higgsless的理论
%\footnote{唐老师在课上不点名的提到了一位已经离开工物系的大佬,推测可能是高原宁院士。}
大受欢迎,直到2012年Higgs粒子被发现。}
\subsection{波动方程的Green函数}
Lorenz规范下,$\Phi$和$\bm A$均满足有源波动方程
\begin{equation}
    \label{eqn:DAlembertianPsi=4pif}
    \lapla\Psi-\frac1{c^2}\pv[2]\Psi t=-4\pi f(\bm x,t).
\end{equation}
%其Green函数
%\biggkh{\lapla-\frac1{c^2}\pp[2]t}G(\bm x,t;\bm x',t')=-4\pi\vd(\bm x-\bm x')\vd(t-t'),
%表示在$\bm x'$处的点源,且仅在$t'$时刻激发。
对$t$进行Fourier变换,%频域里
\[
    (\nabla^2+k^2)\hat\Psi(\bm x,\omega)=-4\pi\hat f(\bm x,\omega),\quad k:=\frac\omega c.
\]
则频域中的Green函数$\hat G_k(\bm x,\bm x')$满足
\[
    (\nabla^2+k^2)\hat G_k(\bm x,\bm x')=-4\pi\vd(\bm x-\bm x').
\]
%不同于时域中的Green函数$G(\bm x,t;\bm x',t')$
$\hat G_k$是关于源球对称的,因此
\[
    \frac1R\dd[2]R(R\hat G_k)+k^2\hat G_k=-4\pi\vd(\bm R),\quad\bm R:=\bm x-\bm x'.
\]
当$R\neq 0$时, 
\[
    \pp[2]R(R\hat G_k)+k^2(R\hat G_k)=0,%\implies R\hat G_k(R)=A\e{\i kR}+B\e{-\i kR}.
\]
%即Green函数可以写成
有两个线性无关的球面波(spherical wave)解:
\[
    \hat G_k^\pm(R)=\frac{\e{\pm\i kR}}R.
\]
分别为发散(diverging)球面波$\hat G_k^+(R)$和会聚(converging)球面波$\hat G_k^-(R)$,后者的例子是透镜聚焦。则频域下Green函数是通解的线性组合
\[
    \hat G_k(R)=A\hat G_k^+(R)+B\hat G_k^-(R),
\]
Green函数要求在点源处
\[
    \lim_{kR\to0}\hat G_k(R)=\frac1R\implies A+B=1.
\]
回到时域下的Green函数,满足
\[
    \biggkh{\lapla-\frac1{c^2}\pp[2]t}G^\pm(\bm x,t;\bm x',t')=-4\pi\vd(\bm x-\bm x')\vd(t-t'),
\]
对$t$进行Fourier变换, 
\[
    (\lapla+k^2)\hat G(\bm x,\omega;\bm x',t')=-4\pi\vd(\bm x-\bm x')\e{\i\omega t'}.
\]
可得 
\[
    \hat G^\pm(\bm x,\omega;\bm x',t')=\hat G_k^\pm(R)\e{\i\omega t'}=\frac{\e{\i(\pm kR+\omega t')}}R,
\]
进行Fourier反变换,
\[
    G^\pm(R,\tau)=\frac1{2\pi}\int\iti\frac{\e{\i(\pm kR-\omega\tau)}}R\d\omega,\quad\tau:=t-t',
\]
对于非色散(nondispersive)介质,$c$与频率无关,Green函数有通解
\[
    G^\pm(R,\tau)=\frac1R\vd\biggkh{\tau\mp\frac Rc}
\]
即
\begin{equation}
    G^\pm(\bm x,t;\bm x',t')=\frac1{|\bm x-\bm x'|}\vd\biggkh{t'-\Bigkh{t\mp\frac{|\bm x-\bm x'|}c}}.
\end{equation}
分别为推迟(retarded)和超前(advanced) Green函数。$\delta$函数的自变量表明:在$\bm x$点于时间$t$观测的效应是由距离为$\bm R$的源在%推迟(或超前)
时间为$t'=t\mp R/c$时的作用引起的。\footnote{课上的理解是:无穷远处产生的源经过$\bm x$后仍需要$R/c$的时间去经过$\bm x'$。}
从而有源波动方程\eqref{eqn:DAlembertianPsi=4pif}的两个解为 
\[
    \Psi^\pm(\bm x,t)=\int G^\pm(\bm x,t;\bm x',t')f(\bm x',t')\d v'\nd t',
\]
我们考虑一个在时间和空间中作局域分布的源$f(\bm x',t')$,这个源分布只在$t'=0$前后一段有限的时间间隔内不为零。则在无源的波动方程仅仅满足其次波动方程
\[
    \lapla\Psi-\frac1{c^2}\pv[2]\Psi t=0,
\]
想像两种极限情形:
\begin{compactenum}
	\item $t\to-\infty$,存在一个特定的“背景”波$\Psi_\text{in}$,
	\[
        \Psi(\bm x,t)=\Psi_\text{in}(\bm x,t)+\int G^+(\bm x,t;\bm x',t')f(\bm x',t')\d v'\nd t';
    \]
	\item $t\to+\infty$,存在一个特定的“稳态”波$\Psi_\text{out}$,
	\[
        \Psi(\bm x,t)=\Psi_\text{out}(\bm x,t)+\int G^-(\bm x,t;\bm x',t')f(\bm x',t')\d v'\nd t',
    \]
\end{compactenum}
最常见的物理情景是$\Psi_\text{in}=0$%“超前”这个概念总是太科幻了,
\begin{equation}
    \Psi(\bm x,t)=\int\frac{f(\bm x',t')}{|\bm x-\bm x'|}\d v',\quad t'=t-\frac1c|\bm x-\bm x'|.
\end{equation}
物理学家已经就一维、二维和三维情形下的有限时间的初值和终值问题进行过广泛研究。\footnote{Morse and Feshbach, Methods of Theoretical Physics, pp843-847.}
\subsection{推迟势和推迟场}
如果电荷密度和电流密度分布随时间变化,观测点的标量势和矢量势将会延迟:
\begin{align}
    \Phi(\bm x,t)&=\frac1{4\pi\varepsilon_0}\int\frac{\rho(\bm x',t')}{|\bm x-\bm x'|}\d v',\\
    \bm A(\bm x,t)&=\frac{\mu_0}{4\pi}\int\frac{\bm J(\bm x',t')}{|\bm x-\bm x'|}\d v'.
\end{align}
由
%但电场和磁场并不能简单的变换\eqref{eqn:E(x)}和\eqref{eqn:B(x)}中的$\rho,\bm J$,而需要从
\[
    \bm E=-\nabla\Phi-\pv{\bm A}t,\quad\bm B=\curl\bm A
\]
得到
\begin{align}
    \bm E(\bm x,t)&=\frac1{4\pi\varepsilon_0}\int\biggfkh{-\nabla'\rho-\frac1{c^2}\pv{\bm J}{t'}}\frac{\d v'}{|\bm x-\bm x'|},\\
    \bm B(\bm x,t)&=\frac{\mu_0}{4\pi}\int\frac{\nabla'\times\bm J}{|\bm x-\bm x'|}\d v',
\end{align}
注意上式中$\nabla'\rho$和$\nabla'\times\bm J$是先求梯度和散度再令$t'=t-R/c$。

Jackson中用$[\cdot]_\text{ret}$表示变量$t'=t-R/c$,这样的好处是可以有效区分$\nabla[f]_\text{ret}\neq[\nabla f]_\text{ret}$,因为前者的$t$与$\bm x'$有关,进一步的:
\begin{align*}
    [\nabla'\rho]_\text{ret}&=\nabla'[\rho]_\text{ret}-\Bigfkh{\pv\rho{t'}}_\text{ret}\nabla'(t-R/c);\\
    [\nabla'\times\bm J]_\text{ret}&=\nabla'\times[\bm J]_\text{ret}+\Bigfkh{\pv{\bm J}{t'}}_\text{ret}\times\nabla'(t-R/c).
\end{align*}
而$\nabla'R=-\uvec R$,故
\begin{align}
    \bm E(\bm x,t)&=\frac1{4\pi\varepsilon_0}\int\biggfkh{\biggkh{\frac{\rho(\bm x',t')}R+\frac1c\pv\rho{t'}}\uvec R-\frac1{c^2}\pv{\bm J}{t'}}\frac{\d v'}R,\\
    \bm B(\bm x,t)&=\frac{\mu_0}{4\pi}\int\biggfkh{\frac{\bm J(\bm x',t')}R+\frac1c\pv{\bm J}{t'}}\times\frac{\nvec R}R\d v'.
\end{align}
上式称为Jefimenko对Coulomb定律和Biot-Savart定律的推广。
若$\rho$和$\bm J$是与时间无关的,则可重新得到静电场和静磁场的表达式\eqref{eqn:E(x)}和\eqref{eqn:B(x)}。
\begin{example}{电荷量变化的点电荷}{time-varying point charge}
    空间中原点处一点电荷,其电荷量随时间变化
    \[
        \rho(\bm x',t')=Q(t')\vd(\bm x'),
    \]
    则
    \[
        \Phi(r,t)=\frac1{4\pi\varepsilon_0r}Q\Bigkh{t-\frac rc}.
    \]
\end{example}
\begin{example}{阶跃电流}{current}
    无限长直导线通阶跃电流
    \[
        I(t)=\begin{cases}
            0,&t\leqslant 0\\
            I_0,&t>0
        \end{cases}
    \]
    电流密度为
    \[
        \bm J(\bm x',t')=I(t')\vd(x')\vd(y')\uvec z.
    \]
    矢量势
    \[
        \bm A(\bm x,t)=\frac{\mu_0}{4\pi}\uvec z\int\iti I\Bigkh{t-\frac Rc}\frac{\d z'}R,\quad R=\sqrt{\rho^2+(z-z')^2},
    \]
    $t\leqslant\rho/c$时,$\bm A=\bm 0$;
    
    $t>\rho/c$时,仅$z'^2\leqslant(ct)^2-\rho^2$部分有贡献,则
    \[
        \bm A(\rho,t)=\frac{\mu_0I_0}{2\pi}\ln\biggfkh{\frac{ct+\sqrt{(ct)^2-\rho^2}}{\rho}}\uvec z.
    \]
    由式\eqref{eqn:-nablaPhi-pApt}和\eqref{eqn:curlA},电场和磁场分别为
    \begin{align*}
        \bm E(\rho,t)&=-\frac{\mu_0I_0}{2\pi}\frac c{\sqrt{(ct)^2-\rho^2}}\uvec z,\\
        \bm B(\rho,t)&=\frac{\mu_0I_0}{2\pi}\frac{ct}{\rho\sqrt{(ct)^2-\rho^2}}\uvec\phi.
    \end{align*}
    $t\to+\infty$时,电磁场趋于稳定,与恒稳电流情形相同:
    \[
        \bm E=\bm 0,\quad\bm B=\frac{\mu_0I_0}{2\pi\rho}\uvec\phi.
    \]
\end{example}
\section{宏观电磁场}
尽管我们已经得到了Maxwell方程组、电磁场与标量势矢量势的关系,但我们仍没有处理$\bm B\&\bm H,\bm D\&\bm E$之间的关系。在真空中
\[
    \bm D=\varepsilon_0\bm E,\enspace\bm B=\mu_0\bm H,\quad c^2=\frac1{\varepsilon_0\mu_0}.
\]
有质介质中,
\[
    \bm D=\varepsilon_0\bm E+\bm P,\quad \bm H=\bm B/\mu_0-\bm M.
\]
对于线性介质,$\bm P=\varepsilon_0\chi_\elc\bm E$,
\[
    \bm D=\varepsilon\bm E,\quad\bm H=\bm B/\mu;
\]
在非线性介质中,关系变得复杂。
\paragraph{微观场}
考虑一个仅由电子和原子核构成的微观世界,由于原子之内是如此的空旷,以至于可以认为是真空条件,因此微观场的Maxwell方程
\begin{align*}
    \div\bm e&=\frac\varrho{\varepsilon_0},\\
    \div\bm b&=0,\\
    \curl\bm e+\pv{\bm b}t&=\bm 0,\\
    \curl\bm b-\frac1{c^2}\pv{\bm e}t&=\mu_0\bm j.
\end{align*}
$\varrho,\bm j$是微观电荷密度和电流密度,$\bm e,\bm b$是微观电场和磁场,而$\bm d,\bm h$并无定义。
\paragraph{宏观场}
宏观场被定义为微观场的空间平均:
\[
    \bm E(\bm x,t)=\ave{\bm e(\bm x,t)},\quad \bm B(\bm x,t)=\ave{\bm b(\bm x,t)},
\]
其中空间平均值是相对于测试函数$f(\bm x)$而言的,其定义为:
\[
    \ave{F(\bm x,t)}:=\int F(\bm x-\bm x',t)f(\bm x')\d v',
\]
因此,空间和时间的微分运算与平均运算可交换:
\[
    \pp{x_\alpha}\ave{F}=\ave{\pv F{x_\alpha}},\quad\pp t\ave{F}=\ave{\pv Ft}.
\]
从而宏观场的Maxwell方程为: 
\begin{align}
    \label{eqn:average rho}
    \varepsilon_0\div\bm E&=\ave{\varrho},\\
    \notag
    \div\bm B&=0,\\
    \notag
    \curl\bm E+\pv{\bm B}t&=\bm0,\\
    \label{eqn:average j}
    \frac1{\mu_0}\curl\bm B-\varepsilon_0\pv{\bm E}t&=\ave{\bm j}.
\end{align}
因此导出场$\bm D$和$\bm H$是
通过从$\ave\varrho$和$\ave{\bm j}$提出某些可以看作介质的宏观性质的贡献而引进的。
下面便研究$\ave\varrho$和$\ave{\bm j}$。
\paragraph{电荷密度}
我们将电荷$\varrho$分为自由电荷$\varrho_\text{free}$和束缚电荷$\varrho_\text{bound}$,后者被束缚在分子中,因此
\begin{align*}
    \varrho_\text{free}(\bm x,t)&=\sum_jq_j\vd(\bm x-\bm x_j);\\
    \varrho_\text{bound}(\bm x,t)&=\sum_n\varrho_n(\bm x,t).
\end{align*}
$\varrho_n$是第$n$个分子的电荷密度,分子质心坐标为$\bm x_n$,分子电荷相对质心坐标为$\bm x_{nj}$,则其平均值为
\[
    \ave{\varrho_n(\bm x,t)}=\sum_{j(n)}q_jf(\bm x-\bm x_n-\bm x_{nj}),
\]
因为$\bm x_{nj}$具有原子线度的数量级,适合在$(\bm x-\bm x_n)$处Taylor展开: 
\begin{align*}
    \ave{\varrho_n(\bm x,t)}={}&\sum_{j(n)}q_j\Big[f(\bm x-\bm x_n)-\bm x_{nj}\cdot\nabla f(\bm x-\bm x_n)\,+\\
    &\qqquad\frac12\sum_{\alpha,\beta}(\bm x_{nj})_\alpha(\bm x_{nj})_\beta\pw f{x_\alpha}{x_\beta}(\bm x-\bm x_n)+\cdots\Big]
\end{align*}
式中各项分别对应分子电荷、偶极矩、四极矩……
\begin{align*}
    \ave{\varrho_n(\bm x,t)}={}&q_nf(\bm x-\bm x_n)-\bm p_n\cdot\nabla f(\bm x-\bm x_n)+\frac16\sum_{\alpha,\beta}(\mathcal Q_n)_{\alpha\beta}\pw f{x_\alpha}{x_\beta}(\bm x-\bm x_n)+\cdots
\end{align*}
将上式各项表示成空间平均的形式:
\begin{align*}
    \ave{\varrho_n(\bm x,t)}={}&\ave{q_n\vd(\bm x-\bm x_n)}-\div\ave{\bm p_n\vd(\bm x-\bm x_n)}+\frac16\sum_{\alpha,\beta}\pw{}{x_\alpha}{x_\beta}\ave{(\mathcal Q_n)_{\alpha\beta}\vd(\bm x-\bm x_n)}+\cdots
\end{align*}
就平均而言,可以把分子看作位于分子质心(也可选取其他固定点)上的点多极子的集合。对所有分子求和,得到
\[
    \ave{\varrho(\bm x,t)}=\rho(\bm x,t)-\div\bm P(\bm x,t)+\sum_{\alpha,\beta}\pw{}{x_\alpha}{x_\beta}\mathcal Q_{\alpha\beta}(\bm x,t)+\cdots
\]
$\rho,\bm P,\mathcal Q_{\alpha\beta}$分别是宏观电荷密度、宏观电极化强度、宏观电四极矩密度……
因此式\eqref{eqn:average rho}可以写成 
\[
    \div\bm D=\rho,
\]
宏观电位移矢量定义为
\[
    D_\alpha:=\varepsilon_0E_\alpha+P_\alpha-\sum_\beta\pv{\mathcal Q_{\alpha\beta}}{x_\beta}+\cdots
\]
前两项是熟知的结果,更高阶项原则上是存在的,但几乎可以忽略不计。
\paragraph{电流密度}
$\ave{\bm j}$的推导比$\ave{\varrho}$的处理复杂的多,在这里仅给出结果。定义分子磁矩和宏观磁化强度分别为
\[
    \bm m_n=\frac12\sum_{j(n)}q_j\bm x_{jn}\times\bm v_{jn},\quad\bm M(\bm x,t)=\sum_n\bm m_n\vd(\bm x-\bm x_n),
\]
宏观电流密度
\[
    \bm J(\bm x,t)=\ave{\bm j_\text{free}+\bm j_\text{bound}},\quad\bm j_\text{free}(\bm x,t):=\sum_jq_j\bm v_j\vd(\bm x-\bm x_j).
\]
进而定义磁场强度\footnote{本式其实是运动介质的%Minkowski电动力学的
非相对论性极限。}
\[
    \bm H:=\frac{\bm B}{\mu_0}-\bm M-(\bm D-\varepsilon_0\bm E)\times\bm v.
\]
因此式\eqref{eqn:average j}可以写成
\[
    \curl\bm H-\pv{\bm D}t=\bm J.
\] 

\section{电磁场的能量、动量和角动量}
对于单个电荷$q$,外部电磁场$E$和$B$对其做功的功率为$q\bm v\cdot\bm E$。则对于电荷和电流的连续分布,在有限体积$V$中,各场做功的总功率为
\[
    \int_V\bm J\cdot\bm E\d v,
\]
利用Maxwell方程组和式\eqref{eqn:divAxB}变形
\begin{align*}
    \bm J\cdot\bm E&=\Bigkh{\curl\bm H-\pv{\bm D}t}\cdot\bm E\\
    %&=\bm H\cdot(\curl\bm E)-\div(\bm E\times\bm H)-\bm E\cdot\pv{\bm D}t\\
    &=-\Bigfkh{\div(\bm E\times\bm H)+\bm H\cdot\pv{\bm B}t+\bm E\cdot\pv{\bm D}t}.
\end{align*}
假设:
\begin{compactenum}
    \item 总电磁能量密度为(即使是在时变场中):
    \[
        u=w_\elc+w_\text m=\frac12(\bm E\cdot\bm D+\bm H\cdot\bm B);%\frac12\int_V\bm E\cdot\bm D\d v,\quad w_B=\frac12\int_V\bm H\cdot\bm B\d v.
    \]
	\item 宏观介质的电磁性质是线性的,
    \[
        \bm D=\varepsilon\bm E,\quad\bm H=\bm B/\mu;
    \]
    \item 介质无色散或损失%可以忽略不计
    ,即$\varepsilon,\mu$不随时间(或频率)变化故
    \[
        \pv ut=\bm E\cdot\pv{\bm D}t+\bm H\cdot\pv{\bm B}t,
    \]
\end{compactenum}
因此 
%-\int_V\bm J\cdot\bm E\d v=\int_V\Bigfkh{\pv ut+\div(\bm E\times\bm H)}\d v
%因为上式$V$是任意的,故
\begin{align}
    \pv ut+\div\bm S=-\bm J\cdot\bm E,
\end{align}
$\bm S$定义为Poynting矢量
\begin{equation}
    \bm S:=\bm E\times\bm H.
\end{equation}
%注意上式仅对线性和非色散介质成立。
\begin{theorem}{Poynting定律(电磁场的能量守恒定律)}{Poynting theorem}
    电磁能量在一定体积内的变化率,加上单位时间内流出的能量,等于该体积内各源的场所做的总功的负值。
\end{theorem}
我们的重点一直是放在电磁场的能量上。场在单位时间和单位体积内所做的功$\bm J\cdot\bm E$代表电磁能转换为机械能或热能。因为物质归根到底是由带电粒子(电子和原子核)组成的,所以可把这能量转换率看作每单位体积带电粒子的能量增加率。这样一来,我们就可以把微观场($\bm E,\bm B$)的Poynting定理,解释为粒子和场的组合系统的能量守恒定律。%如果用Em表示体积V内诸粒子的总能量,就得到
\paragraph{能量守恒}
将空间$V$中的总能量$E$分成粒子的机械能(包含热能) $E_\text{m}$和场能$E_\text{f}$,并假定没有粒子移出体积外,总的能量随时间的变化
\begin{align*}
    \dv{E_\text m}t+\dv{E_\text f}t&=\int_V\bm J\cdot\bm E\d v+\dd t\int_Vu\d v=-\oint_{\p V}\bm S\cdot\uvec n\d a.
\end{align*}
若空间中没有电荷(源)
\[
    E=E_\text f=\int_Vu\d v=\frac{\varepsilon_0}2\int_V(E^2+c^2B^2)\d v.
\]
\begin{example}{导线的Joule热}{Joule heating of a wire}
    圆柱形导线长度为$L$,半径$a$,两端电压为$V$,则电场强度
    \[
        E=\frac VL,
    \]
    导体表面磁场
    \[
        B=\frac{\mu_0}{2\pi}\frac Ia.
    \]
    Poynting矢量向里,大小为
    \[
        S=\frac VL\cdot\frac1{\mu_0}\frac{\mu_0}{2\pi}\frac Ia=\frac{VI}{2\pi aL},
    \]
    通过导线表面的单位时间能量为
    \[
        \int \bm S\cdot\d\bm a=S\cdot 2\pi aL=VI.
    \]
\end{example}
\paragraph{动量守恒}
电磁场对带电粒子的力是Lorentz力
\[
    \bm F=q(\bm E+\bm v\times\bm B).
\]
则粒子动量的变化率为
\[
    \dv{\bm P_\text m}t=\int_V(\rho\bm E+\bm J\times\bm B)\d v
\]
消去源,得到
\begin{align*}
    &\rho\bm E+\bm J\times\bm B=\varepsilon_0(\div\bm E)\bm E+\biggkh{\frac1{\mu_0}\curl\bm B-\varepsilon_0\pv{\bm E}t}\times\bm B\\
    ={}&\varepsilon_0\Bigfkh{\bm E(\div\bm E)-c^2\bm B\times(\curl\bm B)-\bm E\times(\curl\bm E)-\pp t(\bm E\times\bm B)},
\end{align*}
右边出现了%Poynting矢量对
时间的导数,我们依此定义电磁场的动量
\[
    \bm P_\text f:=\varepsilon_0\int_V\bm E\times\bm B\d v=\frac1{c^2}\int_V\bm E\times\bm H\d v.
\]
电磁场的动量密度
\[
    \bm g:=\frac1{c^2}\bm E\times\bm H=\frac1{c^2}\bm S.
\]
则
\[
    \dv{\bm P_\text m}t+\dv{\bm P_\text f}t=\varepsilon_0\int_V\Bigfkh{\bm E(\div\bm E)-\bm E\times(\curl\bm E)-c^2\bm B\times(\curl\bm B)}\d v.
\]
右边的体积分完全可以加上$c^2\bm B(\div\bm B)\equiv 0$以更对称。并且注意到
\[
    [\bm E(\div\bm E)-\bm E\times(\curl\bm E)]_\alpha=\sum_\beta\pp{x_\beta}\Bigkh{E_\alpha E_\beta-\frac12E^2\vd_{\alpha\beta}};
\]
从而右边的体积分具有散度的形式,定义Maxwell应力张量(Maxwell stress tensor)
\[
    \mathcal T_{\alpha\beta}:=\varepsilon_0\Bigfkh{E_\alpha E_\beta+c^2B_\alpha B_\beta-\frac12(E^2+c^2B^2)\vd_{\alpha\beta}}.
\]
则
\[
    \dd t({\bm P_\text m}+{\bm P_\text f})_\alpha=\oint_{\p V}\sum_\beta\mathcal  T_{\alpha\beta}\hat n_\beta\d a.
\]
\paragraph{角动量守恒}
角动量与动量有关:
\[
    \bm L=\bm x\times\bm P,
\]
故结论与动量守恒相似,仅仅是所有表达式左乘$\bm x$罢了。而且也涉及并矢这种不是很优雅的表达,特略。
\begin{example}{Feynman盘详谬}{paradox of Feynman disc}
    塑料圆盘可以自由旋转,一个线圈(一个短的螺线管)与旋转轴同心。该电磁阀携带由一个小电池提供的稳定电流。此外,一些相互绝缘的小金属球围绕圆盘的圆周均匀间隔,每个球体上都有相同的$Q$电荷。一切静止。假设螺线管中的电流突然中断。圆盘是否会旋转?

    详谬是:角动量守恒似乎表明,在场消散前后,圆盘保持静止;但由于消散的磁场会产生一个与圆盘周边相切的强环形电场,静电荷将被该场推动,圆盘必然会开始旋转。

    但事实是:由于电磁场具有角动量,因此圆盘转动也不违反角动量守恒;另外,磁场也不能立刻消失,而必须将存储的磁能消散到线圈的电阻中,这会将电流的角动量传递到线圈中,使圆盘旋转。
\end{example}
\subsection{色散介质中的Poynting定律}
\label{ssec:Poynting in dissipative media}
在有色散介质中,$\varepsilon,\mu$与频率$\omega$有关,前面的假设不再成立,但仍有
\[
    \int_V\bm J\cdot\bm E\d v=-\int_V\Bigfkh{\div(\bm E\times\bm H)+\bm H\cdot\pv{\bm B}t+\bm E\cdot\pv{\bm D}t}.
\]
为了讨论色散,有必要利用Fourier变换将时域变到频域中:
\begin{align*}
    \bm E(\bm x,t)&=\int\iti\hatbm E(\bm x,\omega)\e{-\i\omega t}\d\omega,\\
    \bm D(\bm x,t)&=\int\iti\hatbm D(\bm x,\omega)\e{-\i\omega t}\d\omega;
\end{align*}
在此我们仍假设介质是各向同性的,从而
\[
    \hatbm D(\bm x,\omega)=\varepsilon(\omega)\hatbm E(\bm x,\omega),
\]
其中$\varepsilon$是复数,而场$\bm E(\bm x,t)$必然仍是实数,因此%简单的将积分变量$\omega$变为$-\omega$可得
\[
    \hatbm E(\bm x,-\omega)=\hatbm E\cj(\bm x,\omega),\quad\hatbm D(\bm x,-\omega)=\hatbm D\cj(\bm x,\omega),
\]
从而$\varepsilon(-\omega)=\varepsilon\cj(\omega)$,色散的性质表面:$\bm D(\bm x,t)$和$\bm E(\bm x,t)$在时间上的联系是非局域性的。因此$\bm E\cdot(\p\bm D/\p t)\neq\p(\bm E\cdot\bm D/2)/\p t$,我们先不考虑空间相关性,利用Fourier变换
\begin{align*}
    \bm E\cdot\pv{\bm D}t&=\int\iti\int\iti\hatbm E\cj(\omega')\cdot\bigfkh{-\i\omega\varepsilon(\omega)\hatbm E(\omega)}\e{-\i(\omega-\omega')t}\d\omega\nd\omega'\\
    &=\frac12\int\iti\int\iti\bigfkh{-\i\omega\varepsilon(\omega)+\i\omega'\varepsilon\cj(\omega')}\hatbm E\cj(\omega')\cdot\hatbm E(\omega)\e{-\i(\omega-\omega')t}\d\omega\nd\omega'.
\end{align*}
我们现在假设电场在一个相对狭窄的范围内由频率分量主导,而在这个范围内$\varepsilon$有可观测的变化。然后我们可以%展开因子iω'e*(ω')在ω' = ω周围的方括号中得到
将$\i\omega'\varepsilon\cj(\omega')$在$\omega$附近Taylor展开:
\[
    -\i\omega\varepsilon(\omega)+\i\omega'\varepsilon\cj(\omega')=2\omega\Im\bigfkh{\varepsilon(\omega)}-\i(\omega-\omega')\dd\omega\bigfkh{\omega\varepsilon\cj(\omega)}+\cdots
\]
因此 
\begin{align*}
    \bm E\cdot\pv{\bm D}t={}&\int\iti\int\iti\hatbm E\cj(\omega')\cdot\hatbm E(\omega)\,\omega\Im\bigfkh{\varepsilon(\omega)}\e{-\i(\omega-\omega')t}\d\omega\nd\omega'\,+\\
    &\frac12\pp t\int\iti\int\iti\hatbm E\cj(\omega')\cdot\hatbm E(\omega)\,\dd\omega\bigfkh{\omega\varepsilon\cj(\omega)}\e{-\i(\omega-\omega')t}\d\omega\nd\omega'
\end{align*}
第一项表示电场能量转化为热能(或更一般地转化为各种形式的辐射);第二项表示有效能量密度。

$\bm H\cdot(\p\bm B/\p t)$的表达式是相似的,只需$\bm E,\varepsilon$换为$\bm H,\mu$即可。
\paragraph{谐振场}
假设场有谐振的形式: 
\[
    \bm E=\tilde{\bm E}(t)\cos(\omega_0t+\varphi_\elc),\quad\bm H=\tilde{\bm H}(t)\cos(\omega_0t+\varphi_\text m),
\]
其中$\tilde{\bm E}(t),\tilde{\bm H}(t)$相对$1/\omega$缓慢变化。则
\begin{align*}
    \ave{\bm E\cdot\pv{\bm D}t+\bm H\cdot\pv{\bm B}t}={}&\omega_0\Im\bigfkh{\varepsilon(\omega_0)}\ave{E^2}+
    \omega_0\Im\bigfkh{\mu(\omega_0)}\ave{H^2}+\pv{u\eff}t.
\end{align*}
其中 
\[
    u\eff=\frac12\Re\biggfkh{\dv{(\omega\varepsilon)}\omega(\omega_0)}\ave{E^2}+\frac12\Re\biggfkh{\dv{(\omega\mu)}\omega(\omega_0)}\ave{H^2}.
\]
Poynting定律形式变为
\begin{align}
    \label{eqn:Poynting with loss}
    \pv{u\eff}t+\div\bm S={}&-\bm J\cdot\bm E-\omega_0\Im\bigfkh{\varepsilon(\omega_0)}\ave{E^2}-\omega_0\Im\bigfkh{\mu(\omega_0)}\ave{H^2}
\end{align}
右边第一项是Ohm热损耗,后面两项是在介质中的吸收耗散,不包括传导损耗。
上述方程给出了现实情况下的电磁能局部守恒,在这种情况下,除了能量从局部流出外($\div\bm S\neq 0$),介质的加热可能会造成损失($\Im\varepsilon\neq 0,\Im\mu\neq 0$),从而导致(先前假设中提到的)场能量的缓慢衰减。
\subsectionstar{谐振场的Poynting定律}
我们假定所有的场和源都是随时间简谐的,即 
\[
    \bm E(\bm x,t)=\Re\bigfkh{\bm E(\bm x)\e{-\i\omega t}},
\]
其中$\bm E(\bm x)$一般是复的,有相位的信息。
%因此
%\bm J\cdot\bm E=\frac12\Re\bigfkh{\bm J\cj(\bm x)\cdot\bm E(\bm x)+\bm J(\bm x)\cdot\bm E(\bm x)\e{-\i 2\omega t}},

谐振场的Maxwell方程组可以写成
\begin{alignat*}{2}
    \div\bm D&=\rho,&\qquad\curl\bm E-\i\omega\bm B&=\bm 0,\\
    \div\bm B&=0,&\curl\bm H+\i\omega\bm D&=\bm J.  
\end{alignat*}
场在体积$V$上所做的时间平均工作量
\begin{align*}
    \frac12\int_V\bm J\cj\cdot\bm E\d v&=\frac12\int_V(\curl\bm H\cj-\i\omega\bm D\cj)\cdot\bm E\d v\\
    &=\frac12\int_V\Bigfkh{-\div(\bm E\times\bm H\cj)-\i\omega(\bm E\cdot\bm D\cj-\bm B\cdot\bm H\cj)}\d v.
\end{align*}
复Poynting矢量
\[
    \bm S=\frac12\bm E\times\bm H\cj,
\]
简谐电磁场能量密度
\[
    w_\elc=\frac14\bm E\cdot\bm D\cj,\quad w_\text m=\frac14\bm B\cdot\bm H\cj.
\]
故简谐场的Poynting定律为
\[
    \frac12\int_V\bm J\cj\cdot\bm E\d v+2\i\omega\int_V(w_\elc-w_\text m)\d v+\oint_{\p V}\bm S\cdot\uvec n\d a=0.
\]
实部给出了时间平均量的能量守恒;虚部与反应性(或储存的)能量及其交替流动有关。如果能量密度$w_\elc$和$w_\text m$体积分为实数,就像具有无损耗介质和完美导体的系统一样,实部将是:
\[
    \frac12\int_V\Re(\bm J\cj\cdot\bm E)\d v+\oint_{\p V}\Re(\bm S\cdot\uvec n)\d a=0.
\]
场对$V$中源的稳态时间平均功率等于通过边界面$\p V$进入$V$的平均功率流,由$\Re(\bm S)$的法向分量计算。由于系统各组成部分的损失,上述方程的第二项有一个实部,这可以解释这种耗散。
\paragraph{阻抗和导纳}
阻抗$Z$ (impedance)、电阻$R$ (resistance)、电抗$X$ (reactance)
\[
    Z=R+\i X,
\]
$X>0$:感抗(inductance);$X<0$:容抗(capacitance);\footnote{电气工程师的$i$已经被电流占用,故虚数单位采用j,阻抗写作$Z=R+\mathrm j X.$}

导纳$Y$ (admittance)、电导$G$ (conductance)、电纳$B$ (susceptance)
\[
    Y=G+\i B,
\]
$B>0$:容纳(capacitive susceptance);$B<0$:感纳(inductive susceptance)。

阻抗$Z$和导纳$Y$的关系为$Y=Z\iv$。
\section{电磁场的变换性质、空间反演和时间反演}
\begin{definition}{空间反演}{}
    空间反演即
    \[
        \bm x_i'=-\bm x_i,
    \]
    在此反演变换下,(真)矢量将同步反转(奇变换):
    \[
        \bm V'=-\bm V;
    \]
    而赝矢量(pseudovector)将保持不变(偶变换):
    \[
        \bm A'=\bm A.
    \]
\end{definition}
我们可以对熟知的一些矢量进行归类:
\begin{compactitem}
	\item 真矢量:坐标$\bm x$、速度$\bm v$、动量$\bm p$、电场$\bm E$;
	\item 赝矢量:角动量$\bm L$、力矩$\bm N$、磁场$\bm B$。
\end{compactitem}
注意Poynting矢量是真矢量与赝矢量叉乘($\bm S=\bm E\times\bm H$),但其仍是真矢量。
\begin{definition}{时间反演}{}
    时间反演即
    \[
        t'=-t,
    \]
    同样有奇偶变换之分,但这与真赝矢量并无联系。
\end{definition}

