% \setcounter{equation}{0}
% \renewcommand{\theequation}{A.\arabic{equation}}

\chapter{矢量分析}
\label{chap:vector analysis}

\section{矢量代数}
\label{sec:vector algebra}

\begin{definition}
    {标量和矢量}{scalar & vector}
    标量(scalar)是只有大小(magnitude)没有方向(direction)的量(quantity);
    矢量(vector)是既有方向又有大小的量。
\end{definition}

\begin{compactitem}
    \item 矢量加法:$\bm A+\bm B$;
    \item 矢量数乘:$a\bm A$;
    \item 标量积:$\bm A\cdot\bm B$;
    \item 矢量积:$\bm A\times\bm B$。
	\item 矢量的分量表示:直角坐标(Cartesian coordinate)中
	\begin{equation}
        \bm A=A_x\uvec x+A_y\uvec y+A_z\uvec z.
    \end{equation}
    位置向量(position vector)
    \begin{equation}
        \bm r=x\uvec x+y\uvec y+z\uvec z.
    \end{equation}
	\item 标量三重积
    \begin{equation}
        \bm A\cdot(\bm B\times\bm C)=(\bm A\times\bm B)\cdot\bm C
        =\begin{vmatrix}
            A_x&A_y&A_z\\
            B_x&B_y&B_z\\
            C_x&C_y&C_z
        \end{vmatrix}.
    \end{equation}
    \item 矢量三重积:back cab规则
    \begin{subequations}
        \begin{align}
            \label{eqn:bac-cab}
            \bm A\times(\bm B\times\bm C)&=\bm B(\bm A\cdot\bm C)-\bm C(\bm A\cdot\bm B);\\
            (\bm A\times\bm B)\times\bm C&=\bm B(\bm A\cdot\bm C)-\bm A(\bm B\cdot\bm C).
        \end{align}
    \end{subequations}
    可以将多个矢量积简化,比如
    \begin{align*}
        (\bm A\times\bm B)\cdot(\bm C\times\bm D)&=(\bm A\cdot\bm C)(\bm B\cdot\bm D)-(\bm A\cdot\bm D)(\bm B\cdot\bm C).
    \end{align*}
\end{compactitem}

\section{微积分}
\label{sec:calculus}

\subsection{微分}
\label{ssec:differential calculus}

\begin{compactitem}
	\item 函数$(x)$的微分
    \[
        \d y=\dv yx\d x.
    \]
	\item 标量场$f(x,y,z)$的微分
    \begin{align}
        \d f=\pv fx\d x+\pv fy\d y+\pv fz\d z=:\nabla f\cdot\d\bm\ell.
    \end{align}
    梯度(gradient)
    \begin{align}
        \nabla f:=\pv fx\uvec x+\pv fy\uvec y+\pv fz\uvec z.
    \end{align}
	\item Hamiltonian
	\begin{align}
        \nabla:=\pp x\uvec x+\pp y\uvec y+\pp z\uvec z.
    \end{align}
    \item 散度(divergence)
    \begin{align}
        \div\bm A=\pv{A_x}x+\pv{A_y}y+\pv{A_z}z.
    \end{align}
    \item 旋度(curl)
    \begin{equation}
        \curl\bm A=\begin{vmatrix}
            \uvec x&\uvec y&\uvec z\\
            \pp x&\pp y&\pp z\\
            A_x&A_y&A_z
        \end{vmatrix}=\biggkh{\pv{A_z}y-\pv{A_y}z}\uvec x+\biggkh{\pv{A_x}z-\pv{A_z}x}\uvec y+\biggkh{\pv{A_y}x-\pv{A_x}y}\uvec z.
    \end{equation}
\end{compactitem}
\paragraph{乘法法则}
\begin{compactitem}
	\item 常微分乘法法则
    \[
        \d(fg)=(\d f)g+f\d g.
    \]
    \item 梯度、旋度、散度的简单乘法法则
    \begin{align}
        \nabla(fg)&=(\nabla f)g+f\nabla g;\\
        \div(f\bm A)&=(\nabla f)\cdot\bm A+f(\div\bm A);\\
        \curl(f\bm A)&=(\nabla f)\times\bm A+f(\curl\bm A).
    \end{align}
    \item 梯度、旋度、散度的复杂乘法法则
    \begin{align}
        \label{eqn:gradA.B}
        \nabla(\bm A\cdot\bm B)&=\bm A\times(\curl\bm B)+\bm B\times(\curl\bm A)+(\bm A\cdot\nabla)\bm B+(\bm B\cdot\nabla)\bm A;\\
        \label{eqn:divAxB}
        \div(\bm A\times\bm B)&=(\curl\bm A)\cdot\bm B-(\curl\bm B)\cdot\bm A;\\
        \label{eqn:curlAxB}
        \curl(\bm A\times\bm B)&=(\bm B\cdot\nabla)\bm A-(\bm A\cdot\nabla)\bm B+\bm A(\div\bm B)-\bm B(\div\bm A).
    \end{align}
\end{compactitem}
\paragraph{二重微分}
\begin{compactitem}
	\item 标量Laplacian
    \begin{align}
        \lap f\equiv\div(\nabla f)=\pv[2]fx+\pv[2]fy+\pv[2]fz.
    \end{align}
	\item 矢量Laplacian
	\begin{align}
        \lap\bm A\equiv(\nabla^2A_x)\uvec x+(\nabla^2A_y)\uvec y+(\nabla^2A_z)\uvec z.
    \end{align}
    \item 梯度的旋度恒为0,旋度和散度恒为0
    \begin{align}
        \label{eqn:curlgrad}
        \curl(\nabla f)&\equiv \bm 0;\\
        \label{eqn:divcurl}
        \div(\curl\bm A)&\equiv 0.
    \end{align}
    \item 二重旋度
    \begin{align}
        \label{eqn:curlcurl}
        \curl(\curl\bm A)=\nabla(\div\bm A)-\lap\bm A.
    \end{align}
\end{compactitem}

\subsection{积分}
\label{ssec:integral calculus}

\begin{compactitem}
	\item 线积分、面积分和体积分
	\item 微积分基本定理
	\[
        \int_a^b f'(x)\d x=f(b)-f(a).
    \]
    \item 梯度定理
    \begin{align}
        \int_A^B\nabla f\cdot\d\bm\ell=f(B)-f(A).
    \end{align}
    \item 散度定理(Gauss定理)
    \begin{align}\label{eqn:div}
        \int_V\div\bm A\d v=\oint_{\p V}\bm A\cdot\d\bm a.
    \end{align}
    \item 旋度定理(Stokes定理)
    \begin{align}\label{eqn:curl}
        \int_S(\curl\bm A)\cdot\d\bm a=\oint_{\p S}\bm A\cdot\d\bm\ell.
    \end{align}
    \item 分部积分
    \begin{align}\notag
        \int_a^bfg'\d x&=fg|_a^b-\int_a^bf'g\d x\\
        \int_Vf(\div\bm A)\d v&=\oint_{\p V}f\bm A\cdot\d\bm a-\int_V\bm A\cdot\nabla f\d v.
    \end{align}
\end{compactitem}

\section{曲线坐标系}
\label{sec:curvilinear coordinates}

\subsection{Lam\'e系数}

给定曲线坐标系(curvilinear coordinate)的正交基$\nvec e_1,\nvec e_2,\nvec e_3$,点的坐标为$(q_1,q_2,q_3)$,则
\[
    \d\bm r=\pv{\bm r}{q_1}\d q_1+\pv{\bm r}{q_2}\d q_2+\pv{\bm r}{q_3}\d q_3,
\]
由正交性,$\p\bm r/\p q_i$应沿着$\nvec e_i$方向,其模值定义为Lam\'e系数
\begin{equation}
    H_i:=\abs{\pv{\bm r}{q_i}}=\sqrt{\biggkh{\pv x{q_i}}^2+\biggkh{\pv y{q_i}}^2+\biggkh{\pv z{q_i}}^2}.
\end{equation}
因此 
\begin{equation}
    \d\bm r=H_1\d q_1\uvec e_1+H_2\d q_2\uvec e_2+H_3\d q_3\uvec e_3.
\end{equation}
面积微元和体积微元为
\begin{align}
    \d a_i&=H_jH_k\d q_j\nd q_k,\enspace (ijk)=(123);\\
    \d v&=H_1H_2H_3\d q_1\nd q_2\nd q_3.
\end{align}

\paragraph{梯度}
在曲线坐标系下,梯度可以通过Lam\'e系数表示:
\begin{equation}
    \label{eqn:grad Lame}
    \nabla f=\frac1{H_1}\pv f{q_1}\uvec e_1+\frac1{H_2}\pv f{q_2}\uvec e_2+\frac1{H_3}\pv f{q_3}\uvec e_3.
\end{equation}

\paragraph{散度}
矢量$\bm A$在曲线坐标系下的分量:
\[
    \bm A=A_1\uvec e_1+A_2\uvec e_2+A_3\uvec e_3.
\]
根据面积微元和体积微元的表达式,散度为:
\begin{equation}
    \label{eqn:div Lame}
    \div\bm A=\frac1{H_1H_2H_3}\biggfkh{\pv{(H_2H_3A_1)}{q_1}+\pv{(H_3H_1A_2)}{q_2}+\pv{(H_1H_2A_3)}{q_3}}.
\end{equation}

\paragraph{旋度}
省去推导过程
\begin{equation}
    \label{eqn:curl Lame}
    \curl\bm A=\frac1{H_1H_2H_3}\begin{vmatrix}
        H_1\uvec e_1&H_2\uvec e_2&H_3\uvec e_3\\
        \pp{q_1}&\pp{q_2}&\pp{q_3}\\
        H_1A_1&H_2A_2&H_3A_3
    \end{vmatrix}.
\end{equation}

\paragraph{Laplacian}
结合式\eqref{eqn:grad Lame}和式\eqref{eqn:div Lame},得到
\begin{equation}
    \label{eqn:Laplacian Lame}
    \lap f=\frac1{H_1H_2H_3}\biggfkh{\pp{q_1}\biggkh{\frac{H_2H_3}{H_1}\pv f{q_1}}+\pp{q_2}\biggkh{\frac{H_3H_1}{H_2}\pv f{q_2}}+\pp{q_3}\biggkh{\frac{H_1H_2}{H_3}\pv f{q_3}}}.
\end{equation}

\subsection{球坐标系}

在球坐标系中的坐标为$(r,\theta,\phi)$,$r$称为径向距离(radial distance),$\theta$称为极角(polar angle),$\phi$称为方位角(azimuthal angle)。如下图所示。

\begin{center}
    \begin{tikzpicture}
        \coordinate[label=left:$O$] (O) at (0, 0);
        \coordinate[label=left:$x$] (x) at (-1.5, -1.5);
        \coordinate[label=right:$y$] (y) at (3, 0);
        \coordinate[label=left:$z$] (z) at (0, 3);
        \coordinate (P) at (1.8, 2.5);
        \coordinate (Q) at (1.8, -1);
		\draw[->](O)--(y);
		\draw[->](O)--(z);
		\draw[->](O)--(x);
		\draw(O)--(P)node[midway, above]{$r$};
		\draw[dashed](O)--(Q)--(P);
		\pic[draw, angle radius=3mm]{right angle=P--Q--O};
        \pic[draw, angle radius=5mm, angle eccentricity=1.5, <-, "$\theta$"]{angle=P--O--z};
        \pic[draw, angle radius=3mm, angle eccentricity=1.8, ->, "$\phi$"]{angle=x--O--Q};
	\end{tikzpicture}
    \captionof{figure}{球坐标系}
\end{center}
% 基变换
% \begin{equation}
%     \begin{bmatrix}
%         \uvec r\\\uvec\theta\\\uvec\phi
%     \end{bmatrix}=
%     \begin{bmatrix}
%         \sin\theta\cos\phi&\sin\theta\sin\phi&\cos\theta\\
%         \cos\theta\cos\phi&\cos\theta\sin\phi&-\sin\theta\\
%         -\sin\phi&\cos\phi&0
%     \end{bmatrix}
%     \begin{bmatrix}
%         \uvec x\\\uvec y\\\uvec z
%     \end{bmatrix}
% \end{equation}
坐标变换
\begin{align}
    \begin{cases}
        x=r\sin\theta\cos\phi\\
        y=r\sin\theta\sin\phi\\
        z=r\cos\theta
    \end{cases}
\end{align}
Lam\'e系数:
\begin{subequations}
    \begin{align}
        H_r&=\sqrt{\biggkh{\pv xr}^2+\biggkh{\pv yr}^2+\biggkh{\pv zr}^2}=1,\\
        H_\theta&=\sqrt{\biggkh{\pv x\theta}^2+\biggkh{\pv y\theta}^2+\biggkh{\pv z\theta}^2}=r,\\
        H_\phi&=\sqrt{\biggkh{\pv x\phi}^2+\biggkh{\pv y\phi}^2+\biggkh{\pv z\phi}^2}=r\sin\theta,
    \end{align}
\end{subequations}
% 微元
% \begin{align}
%     \d\bm\ell&=\d r\uvec r+r\d\theta\uvec\theta+r\sin\theta\d\phi\uvec\phi\\
%     \d v&=r^2\sin\theta\d r\nd\theta\nd\phi
% \end{align}
矢量微分
\begin{align}
    %\nabla&=\pp r\uvec r+\frac1r\pp\theta\uvec\theta+\frac1{r\sin\theta}\pp\phi\uvec\phi.\\
    \nabla f&=\pv fr\uvec r+\frac1r\pv f\theta\uvec\theta+\frac1{r\sin\theta}\pv f\phi\uvec\phi,\\
    \div\bm A&=\frac1{r^2}\pp r(r^2A_r)+\frac1{r\sin\theta}\pp\theta(\sin\theta\,A_\theta)+\frac1{r\sin\theta}\pp\phi A_\phi,\\
    \curl\bm A&=\frac1{r^2\sin\theta}
    \begin{vmatrix}
        \uvec r&r\uvec\theta&r\sin\theta\uvec\phi\\
        \pp r&\pp\theta&\pp\phi\\
        A_r&rA_\theta&r\sin\theta A_\phi
    \end{vmatrix},\\
    \lap f&=\frac1{r^2}\pp r\biggkh{r^2\pv fr}+\frac1{r^2\sin\theta}\pp\theta\biggkh{\sin\theta\pv f\theta}+\frac1{r^2\sin^2\theta}\pv[2]f\phi.
\end{align}

\subsection{柱坐标系}

坐标变换
\begin{align}
    \begin{cases}
        x=\rho\cos\phi\\
        y=\rho\sin\phi\\
        z=z
    \end{cases}
\end{align}
Lam\'e系数:
\begin{subequations}
    \begin{align}
        H_\rho&=\sqrt{\biggkh{\pv x\rho}^2+\biggkh{\pv y\rho}^2+\biggkh{\pv z\rho}^2}=1,\\
        H_\phi&=\sqrt{\biggkh{\pv x\phi}^2+\biggkh{\pv y\phi}^2+\biggkh{\pv z\phi}^2}=\rho,\\
        H_z&=\sqrt{\biggkh{\pv xz}^2+\biggkh{\pv yz}^2+\biggkh{\pv zz}^2}=1,
    \end{align}
\end{subequations}
矢量微分
\begin{align}
    \nabla f&=\pv f\rho\uvec\rho+\frac1\rho\pv f\phi\uvec\theta+\pv fz\uvec z,\\
    \div\bm A&=\frac1\rho\pp\rho(\rho A_\rho)+r\pv{A_\phi}\phi+\pv{A_z}z,\\
    \curl\bm A&=\frac1\rho
    \begin{vmatrix}
        \uvec\rho&\rho\uvec\phi&\uvec z\\
        \pp\rho&\pp\phi&\pp z\\
        A_\rho&\rho A_\phi&A_z
    \end{vmatrix},\\
    \lap f&=\frac1\rho\pp\rho\biggkh{\rho\pv f\rho}+\frac1{\rho^2}\pv[2]f\theta+\pv[2]fz.
\end{align}

\section{Dirac函数}

\begin{example}{引入Dirac $\delta$函数}{introduce delta function}
    计算$\bm F(\bm r)$的散度:
    \[
        \bm F(\bm r)=\frac{\uvec r}{r^2},\implies\div\bm F=\frac1{r^2}\pp r\Bigkh{r^2\cdot\frac1{r^2}}=0,
    \]
    但考虑其在半径为$r$球面上的面积分:
    \[
        \oint_{\p V}\bm F\cdot\d\bm a=\oint_{\p V}\frac1{r^2}r^2\sin\theta\d\theta\nd\phi=\int_0^{2\pi}\int_0^\pi\sin\theta\d\theta\nd\phi=4\pi.
    \]
    这与散度定理\eqref{eqn:div}矛盾!
    原因在于:$\div\bm F$在除了原点以外的任意一点均为0,但其体积分为$4\pi$。
\end{example}
\begin{definition}{Dirac $\delta$函数}{Dirac delta function}
    $\delta$函数是一个将任意函数$f$映射到$f(0)$的线性泛函(linear functional),即
    \[
        \int\iti f(x)\vd(x)\d x=f(0).
    \]
    其特点为:$\forall x\neq 0,\,\delta(x)=0$,但
    \[
        \int\iti\delta(x)\d x=1,
    \]
    因此$\delta$函数并不是一个严格意义上的函数,而是一个广义函数(general function)或分布(distribution)。
    %其只有出现在积分中才有实质意义。
    在物理上,$\delta$函数可以代表质点(或点电荷)的密度。
\end{definition}

因此\exmref{exm:introduce delta function} 中的散度其实应为$4\pi\vd(\bm r)$,进而
\begin{equation}
    \lap\biggkh{\frac1r}=-\div\biggkh{\frac{\bm r}{r^3}}=-4\pi\vd(\bm r).
\end{equation}
\section{向量场理论}
给定$\div\bm F$和$\curl\bm F$,能否确定$\bm F$?
\begin{theorem}{Helmholtz定理}{Helmholtz theorem}
    给定
    \[
        \begin{cases}
            \div\bm F=D(\bm r)\\
            \curl\bm F=\bm C(\bm r)
        \end{cases}
    \]
    若当$r\to\infty$时,$D(\bm r)$和$\bm C(\bm r)$均比$1/r^2$收敛得更快,且$\bm F(\bm r)$收敛到0,则$\bm F$可被唯一确定为
    \begin{align}
        \bm F=-\nabla V+\curl\bm A,
    \end{align}
    此处
    \begin{align}
        \begin{cases}
            V(\bm r)=\frac1{4\pi}\int\frac{D(\bm r')}{|\bm r-\bm r'|}\d v'\\
            \bm A(\bm r)=\frac1{4\pi}\int\frac{\bm C(\bm r')}{|\bm r-\bm r'|}\d v'
        \end{cases}
    \end{align}
\end{theorem}
\begin{theorem}{无旋场与标量势}{irrotational field and scalar potential}
    对于无旋场$\bm F$,下列命题是等价的:
    \begin{compactitem}
        \item $\curl\bm F=\bm 0$;
        \item $\bm F$可以被写为标量势(scalar potential)的梯度:$\bm F=-\nabla V$;
        \item 给定起点终点,线积分与路径无关\[
            \int_A^B\bm F\cdot\d\bm\ell=V(A)-V(B);
        \]
        \item 闭合线积分恒为0.
    \end{compactitem}
\end{theorem}
与式\eqref{eqn:curlgrad}联系,注意$V$并不唯一。
\begin{theorem}{无源场与矢量势}{solenoidal field and vector potential}
    对于无源场$\bm F$,下列命题是等价的:
    \begin{compactitem}
        \item $\div\bm F=0$;
        \item $\bm F$可以被写为矢量势(vector potential)的旋度:$\bm F=\curl\bm A$;
        \item 给定边界,面积分与表面形状无关
        \[
            \int_S\bm F\cdot\d\bm a=\oint_{\p S}\bm A\cdot\d\bm\ell;
        \]
        \item 闭合面积分恒为0.
    \end{compactitem}
\end{theorem}
与式\eqref{eqn:divcurl}联系,注意$\bm A$并不唯一 。

% \sectionstar{并矢}

\chapter{特殊函数}

\section{Legendre多项式}

证明递推式\eqref{eqn:Legendre recurrence 1}
\[
    P'_{\ell+1}-P'_{\ell-1}-(2\ell+1)P_\ell=0,
\]
\begin{proof}
    由Rodrigues公式
    \begin{equation}
        P_\ell(x)=\frac1{2^\ell\ell!}\dd[\ell]x(x^2-1)^\ell.
    \end{equation}
    可得
    \begin{align*}
        P'_\ell&=\frac1{2^\ell\ell!}\dd[\ell+1]x(x^2-1)^\ell=\frac1{2^{\ell-1}(\ell-1)!}\dd[\ell]x\bigfkh{x(x^2-1)^{\ell-1}}\\
        &=\frac1{2^{\ell-1}(\ell-1)!}\dd[\ell-1]x\bigfkh{\bigkh{(2\ell-1)x^2-1}(x^2-1)^{\ell-2}}.
    \end{align*}
    从而 
    \begin{align*}
        P_{\ell+1}'-P_{\ell-1}'&=\frac1{2^\ell\ell!}\dd[\ell]x\bigfkh{\bigkh{(2\ell+1)x^2-1}(x^2-1)^{\ell-1}}-\frac1{2^{\ell-1}(\ell-1)!}\dd[\ell]x(x^2-1)^{\ell-1}\\
        &=\frac1{2^\ell\ell^2}\dd[\ell]x\bigfkh{(2\ell+1)(x^2-1)^\ell}=(2\ell+1)P_\ell.\qedhere
    \end{align*}
\end{proof}
