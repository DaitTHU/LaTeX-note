% \setcounter{equation}{0}
% \renewcommand{\theequation}{A.\arabic{equation}}

\chapter{矢量分析}
\label{chap:vector analysis}

\section{矢量代数}

\begin{compactitem}
	\item 定义:标量和矢量。
	\item 矢量的运算:加法、数乘、标量积、矢量积(平凡的)。
	\item 标量三重积
    \begin{align}
        \bm A\cdot(\bm B\times\bm C)&=\bm C\cdot(\bm A\times\bm B)=(\bm A\times\bm B)\cdot\bm C\\\notag
        &=\begin{vmatrix}
            A_x&A_y&A_z\\
            B_x&B_y&B_z\\
            C_x&C_y&C_z
        \end{vmatrix}.
    \end{align}
    \item 矢量三重积:back cab规则
    \begin{align}
        \label{eqn:bac-cab}
        \bm A\times(\bm B\times\bm C)&=\bm B(\bm A\cdot\bm C)-\bm C(\bm A\cdot\bm B);\\
        (\bm A\times\bm B)\times\bm C&=\bm B(\bm A\cdot\bm C)-\bm A(\bm B\cdot\bm C).
    \end{align}
    可以将多个矢量积简化,比如
    \begin{align*}
        (\bm A\times\bm B)\cdot(\bm C\times\bm D)&=(\bm A\cdot\bm C)(\bm B\cdot\bm D)-(\bm A\cdot\bm D)(\bm B\cdot\bm C).
    \end{align*}
    \item 位置向量和分离向量
    \item 向量转换
\end{compactitem}
\section{微积分}
\begin{compactitem}
	\item 常微分
    \[
        \d f=\dv fx\d x.
    \]
	\item 标量场$f(x,y,z)$的微分
    \begin{align}
        \d f=\pv fx\d x+\pv fy\d y+\pv fz\d z=:\nabla f\cdot\d\bm\ell.
    \end{align}
    梯度(gradient)
    \begin{align}
        \nabla f:=\pv fx\uvec x+\pv fy\uvec y+\pv fz\uvec z.
    \end{align}
	\item Hamiltonian
	\begin{align}
        \nabla:=\pp x\uvec x+\pp y\uvec y+\pp z\uvec z.
    \end{align}
    \item 散度(divergence)
    \begin{align}
        \div\bm A=\pv{A_x}x+\pv{A_y}y+\pv{A_z}z.
    \end{align}
    \item 旋度(curl)
    \begin{align}\notag
        &\curl\bm A=\begin{vmatrix}
            \uvec x&\uvec y&\uvec z\\
            \pp x&\pp y&\pp z\\
            A_x&A_y&A_z
        \end{vmatrix}\\
        &=\biggkh{\pv{A_z}y-\pv{A_y}z}\uvec x+\biggkh{\pv{A_x}z-\pv{A_z}x}\uvec y+\biggkh{\pv{A_y}x-\pv{A_x}y}\uvec z.
    \end{align}
\end{compactitem}
\paragraph{乘法法则}
\begin{compactitem}
	\item 常微分乘法法则
    \[
        \d(fg)=(\d f)g+f\d g.
    \]
    \item 梯度、旋度、散度的简单乘法法则
    \begin{align}
        \nabla(fg)&=(\nabla f)g+f\nabla g;\\
        \div(f\bm A)&=(\nabla f)\cdot\bm A+f(\div\bm A);\\
        \curl(f\bm A)&=(\nabla f)\times\bm A+f(\curl\bm A).
    \end{align}
    \item 梯度、旋度、散度的复杂乘法法则
    \begin{align}
        \label{eqn:gradA.B}
        \nabla(\bm A\cdot\bm B)&=\bm A\times(\curl\bm B)+\bm B\times(\curl\bm A)+(\bm A\cdot\nabla)\bm B+(\bm B\cdot\nabla)\bm A;\\
        \label{eqn:divAxB}
        \div(\bm A\times\bm B)&=(\curl\bm A)\cdot\bm B-(\curl\bm B)\cdot\bm A;\\
        \label{eqn:curlAxB}
        \curl(\bm A\times\bm B)&=(\bm B\cdot\nabla)\bm A-(\bm A\cdot\nabla)\bm B+\bm A(\div\bm B)-\bm B(\div\bm A).
    \end{align}
\end{compactitem}
\paragraph{二重微分}
\begin{compactitem}
	\item 标量Laplacian
    \begin{align}
        \lapla f\equiv\div(\nabla f)=\pv[2]fx+\pv[2]fy+\pv[2]fz.
    \end{align}
	\item 矢量Laplacian
	\begin{align}
        \lapla\bm A\equiv(\nabla^2A_x)\uvec x+(\nabla^2A_y)\uvec y+(\nabla^2A_z)\uvec z.
    \end{align}
    \item 梯度的旋度恒为0,旋度和散度恒为0
    \begin{align}
        \label{eqn:curlgrad}
        \curl(\nabla f)&\equiv \bm 0;\\
        \label{eqn:divcurl}
        \div(\curl\bm A)&\equiv 0.
    \end{align}
    \item 二重旋度
    \begin{align}
        \label{eqn:curlcurl}
        \curl(\curl\bm A)=\nabla(\div\bm A)-\lapla\bm A.
    \end{align}
\end{compactitem}
\paragraph{积分}
\begin{compactitem}
	\item 线积分、面积分和体积分
	\item 微积分基本定理
	\[
        \int_a^b f'(x)\d x=f(b)-f(a).
    \]
    \item 梯度定理
    \begin{align}
        \int_A^B\nabla f\cdot\d\bm\ell=f(B)-f(A).
    \end{align}
    \item 散度(Gauss)定理
    \begin{align}\label{eqn:div}
        \int_V\div\bm A\d v=\oint_{\p V}\bm A\cdot\d\bm a.
    \end{align}
    \item 旋度(Stokes)定理
    \begin{align}\label{eqn:curl}
        \int_S(\curl\bm A)\cdot\d\bm a=\oint_{\p S}\bm A\cdot\d\bm\ell.
    \end{align}
    \item 分部积分
    \begin{align}\notag
        \int_a^bfg'\d x&=fg|_a^b-\int_a^bf'g\d x\\
        \int_Vf(\div\bm A)\d v&=\oint_{\p V}f\bm A\cdot\d\bm a-\int_V\bm A\cdot\nabla f\d v.
    \end{align}
\end{compactitem}
\section{曲线坐标}
\paragraph{球坐标系}坐标变换
\begin{align}
    \begin{cases}
        x=r\sin\theta\cos\phi\\
        y=r\sin\theta\sin\phi\\
        z=r\cos\theta
    \end{cases}
\end{align}
基变换
\begin{equation}
    \begin{bmatrix}
        \uvec r\\\uvec\theta\\\uvec\phi
    \end{bmatrix}=
    \begin{bmatrix}
        \sin\theta\cos\phi&\sin\theta\sin\phi&\cos\theta\\
        \cos\theta\cos\phi&\cos\theta\sin\phi&-\sin\theta\\
        -\sin\phi&\cos\phi&0
    \end{bmatrix}
    \begin{bmatrix}
        \uvec x\\\uvec y\\\uvec z
    \end{bmatrix}
\end{equation}
微元
\begin{align}
    \d\bm\ell&=\d r\uvec r+r\d\theta\uvec\theta+r\sin\theta\d\phi\uvec\phi\\
    \d v&=r^2\sin\theta\d r\nd\theta\nd\phi
\end{align}
矢量微分
\begin{align}
    %\nabla&=\pp r\uvec r+\frac1r\pp\theta\uvec\theta+\frac1{r\sin\theta}\pp\phi\uvec\phi.\\
    \nabla f&=\pv fr\uvec r+\frac1r\pv f\theta\uvec\theta+\frac1{r\sin\theta}\pv f\phi\uvec\phi,\\
    \div\bm A&=\frac1{r^2}\pp r(r^2A_r)+\frac1{r\sin\theta}\pp\theta(\sin\theta\,A_\theta)+\frac1{r\sin\theta}\pp\phi A_\phi,\\
    \curl\bm A&=\frac1{r^2\sin\theta}
    \begin{vmatrix}
        \uvec r&r\uvec\theta&r\sin\theta\uvec\phi\\
        \pp r&\pp\theta&\pp\phi\\
        A_r&rA_\theta&r\sin\theta A_\phi
    \end{vmatrix}.
\end{align}
\paragraph{柱坐标系}略
\section{Dirac函数}
\begin{example}{引入Dirac $\delta$函数}{introduce delta function}
    计算
    \[
        \bm F(\bm r)=\frac{\uvec r}{r^2}.
    \]
    的散度
    \[
        \div\bm F=\frac1{r^2}\pp r\Bigkh{r^2\cdot\frac1{r^2}}=0,
    \]
    但
    \[
        \oint_{\p V}\bm F\cdot\d\bm a=\oint_{\p V}\frac1{r^2}r^2\sin\theta\d\theta\nd\phi=\int_0^{2\pi}\int_0^\pi\sin\theta\d\theta\nd\phi=4\pi.
    \]
    这与散度定理\eqref{eqn:div}矛盾!
    原因在于:$\div\bm F$在除了原点以外的任意一点均为0,但其体积分为$4\pi$。
\end{example}
\begin{definition}{Dirac $\delta$函数}{Dirac delta function}
    $\delta$函数是一个将任意函数$f$映射到$f(0)$的线性泛函(linear functional),即
    \[
        \int\iti f(x)\vd(x)\d x=f(0).
    \]
    其特点为:$\forall x\neq 0,\,\delta(x)=0$,但
    \[
        \int\iti\delta(x)\d x=1,
    \]
    因此$\delta$函数并不是一个严格意义上的函数,而是一个广义函数(general function)或分布(distribution)。
    %其只有出现在积分中才有实质意义。
    在物理上,$\delta$函数可以代表质点(或点电荷)的密度。
\end{definition}

因此\exmref{exm:introduce delta function} 中的散度其实应为$4\pi\vd(\bm r)$,进而
\begin{equation}
    \lapla\biggkh{\frac1r}=-\div\biggkh{\frac{\bm r}{r^3}}=-4\pi\vd(\bm r).
\end{equation}
\section{向量场理论}
给定$\div\bm F$和$\curl\bm F$,能否确定$\bm F$?
\begin{theorem}{Helmholtz定理}{Helmholtz theorem}
    给定
    \[
        \begin{cases}
            \div\bm F=D(\bm r)\\
            \curl\bm F=\bm C(\bm r)
        \end{cases}
    \]
    若当$r\to\infty$时,$D(\bm r)$和$\bm C(\bm r)$均比$1/r^2$收敛得更快,且$\bm F(\bm r)$收敛到0,则$\bm F$可被唯一确定为
    \begin{align}
        \bm F=-\nabla V+\curl\bm A,
    \end{align}
    此处
    \begin{align}
        \begin{cases}
            V(\bm r)=\frac1{4\pi}\int\frac{D(\bm r')}{|\bm r-\bm r'|}\d v'\\
            \bm A(\bm r)=\frac1{4\pi}\int\frac{\bm C(\bm r')}{|\bm r-\bm r'|}\d v'
        \end{cases}
    \end{align}
\end{theorem}
\begin{theorem}{无旋场与标量势}{irrotational field and scalar potential}
    对于无旋场$\bm F$,下列命题是等价的:
    \begin{compactitem}
        \item $\curl\bm F=\bm 0$;
        \item $\bm F$可以被写为标量势(scalar potential)的梯度:$\bm F=-\nabla V$;
        \item 给定起点终点,线积分与路径无关\[
            \int_A^B\bm F\cdot\d\bm\ell=V(A)-V(B);
        \]
        \item 闭合线积分恒为0.
    \end{compactitem}
\end{theorem}
与式\eqref{eqn:curlgrad}联系,注意$V$并不唯一。
\begin{theorem}{无源场与矢量势}{solenoidal field and vector potential}
    对于无源场$\bm F$,下列命题是等价的:
    \begin{compactitem}
        \item $\div\bm F=0$;
        \item $\bm F$可以被写为矢量势(vector potential)的旋度:$\bm F=\curl\bm A$;
        \item 给定边界,面积分与表面形状无关
        \[
            \int_S\bm F\cdot\d\bm a=\oint_{\p S}\bm A\cdot\d\bm\ell;
        \]
        \item 闭合面积分恒为0.
    \end{compactitem}
\end{theorem}
与式\eqref{eqn:divcurl}联系,注意$\bm A$并不唯一 。

% \sectionstar{并矢}
