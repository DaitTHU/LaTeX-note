% \setcounter{equation}{0}
% \renewcommand{\theequation}{A.\arabic{equation}}

\chapter{矢量分析}
\label{chap:vector analysis}

\section{矢量代数}
\label{sec:vector algebra}

\begin{compactitem}
    \item 矢量加法:$\bm A+\bm B$;
    \item 矢量数乘:$a\bm A$;
    \item 标量积:$\bm A\cdot\bm B$;
    \item 矢量积:$\bm A\times\bm B$。
	\item 矢量的分量表示:直角坐标(Cartesian coordinate)中
	\begin{equation}
        \bm A=A_x\uvec x+A_y\uvec y+A_z\uvec z.
    \end{equation}
    位置向量(position vector)
    \begin{equation}
        \bm r=x\uvec x+y\uvec y+z\uvec z.
    \end{equation}
	\item 标量三重积
    \begin{equation}
        \bm A\cdot(\bm B\times\bm C)=(\bm A\times\bm B)\cdot\bm C
        =\begin{vmatrix}
            A_x&A_y&A_z\\
            B_x&B_y&B_z\\
            C_x&C_y&C_z
        \end{vmatrix}.
    \end{equation}
    \item 矢量三重积:back cab规则
    \begin{subequations}
        \begin{align}
            \label{eqn:bac-cab}
            \bm A\times(\bm B\times\bm C)&=\bm B(\bm A\cdot\bm C)-\bm C(\bm A\cdot\bm B);\\
            (\bm A\times\bm B)\times\bm C&=\bm B(\bm A\cdot\bm C)-\bm A(\bm B\cdot\bm C).
        \end{align}
    \end{subequations}
    可以将多个矢量积简化,比如
    \begin{align*}
        (\bm A\times\bm B)\cdot(\bm C\times\bm D)&=(\bm A\cdot\bm C)(\bm B\cdot\bm D)-(\bm A\cdot\bm D)(\bm B\cdot\bm C).
    \end{align*}
\end{compactitem}

\section{微积分}
\label{sec:calculus}

\subsection{微分}
\label{ssec:differential calculus}

\begin{compactitem}
	\item 函数$y=y(x)$的微分
    \[
        \d y=\dv yx\d x.
    \]
	\item 标量场$\psi(x,y,z)$的微分
    \begin{equation}
        \d\psi=\pv\psi x\d x+\pv\psi y\d y+\pv\psi z\d z.
    \end{equation}
\end{compactitem}

\begin{definition}
    {nabla算符}{nabla}
    定义nabla算符(Hamilton算符)
    \begin{equation}
        \nabla:=\pp x\uvec x+\pp y\uvec y+\pp z\uvec z.
    \end{equation}
    标量场$\psi$的梯度(gradient)为
    \begin{equation}
        \nabla\psi=\pv\psi x\uvec x+\pv\psi y\uvec y+\pv\psi z\uvec z.
    \end{equation}
    矢量场$\bm A$的散度(divergence)为
    \begin{equation}
        \div\bm A=\pv{A_x}x+\pv{A_y}y+\pv{A_z}z.
    \end{equation}
    矢量场$\bm A$的旋度(curl)为
    \begin{equation}
        \curl\bm A=\begin{vmatrix}
            \uvec x&\uvec y&\uvec z\\
            \pp x&\pp y&\pp z\\
            A_x&A_y&A_z
        \end{vmatrix}
        =\biggkh{\pv{A_z}y-\pv{A_y}z}\uvec x+\biggkh{\pv{A_x}z-\pv{A_z}x}\uvec y+\biggkh{\pv{A_y}x-\pv{A_x}y}\uvec z.
    \end{equation}
    标量Laplacian
    \begin{equation}
        \lapla\psi \equiv \div(\nabla\psi) = \pv[2]\psi x + \pv[2]\psi y + \pv[2]\psi z.
    \end{equation}
    矢量Laplacian
    \begin{equation}
        \lapla\bm A\equiv\lapla A_x\uvec x+\lapla A_y\uvec y+\lapla A_z\uvec z.
    \end{equation}
\end{definition}

\begin{theorem}
    {简单乘法法则}{simple product rules}
    由Leibniz法则易得梯度、散度和旋度的简单乘法法则:
    \begin{align}
        \nabla(\psi\phi)&=(\nabla\psi)\phi+\psi\nabla\phi;\\
        \div(\psi\bm A)&=(\nabla\psi)\cdot\bm A+\psi(\div\bm A);\\
        \curl(\psi\bm A)&=(\nabla\psi)\times\bm A+\psi(\curl\bm A).
    \end{align}
\end{theorem}

\begin{theorem}
    {复杂乘法法则}{complex product rules}
    梯度、散度和旋度的复杂乘法法则:
    \begin{align}
        \label{eqn:gradA.B}
        \nabla(\bm A\cdot\bm B)&=\bm A\times(\curl\bm B)+\bm B\times(\curl\bm A)+(\bm A\cdot\nabla)\bm B+(\bm B\cdot\nabla)\bm A;\\
        \label{eqn:divAxB}
        \div(\bm A\times\bm B)&=(\curl\bm A)\cdot\bm B-(\curl\bm B)\cdot\bm A;\\
        \label{eqn:curlAxB}
        \curl(\bm A\times\bm B)&=(\bm B\cdot\nabla)\bm A-(\bm A\cdot\nabla)\bm B+\bm A(\div\bm B)-\bm B(\div\bm A).
    \end{align}
    其中
    \begin{equation}
        (\bm A\cdot\nabla)\bm B:=A_x\pv{B_x}x+A_y\pv{B_x}y+A_z\pv{B_x}z.
    \end{equation}
\end{theorem}

\begin{corollary}
    \begin{equation}
        (\bm A\cdot\nabla)\bm B=\frac12[\nabla(\bm A\cdot\bm B)-\curl(\bm A\times\bm B)-\bm B\times(\curl\bm A)+\bm A(\div\bm B)+\bm B(\div\bm A)].
    \end{equation}
    特别地,$\bm B=\bm A$时,
    \begin{equation}
        (\bm A\cdot\nabla)\bm A=\frac12\nabla(A^2)-\bm A\times(\curl\bm A).
    \end{equation}
\end{corollary}

\begin{theorem}
    {梯度场无旋、旋度场无源}{curlgrad=divcurl=0}
    梯度的旋度恒为$\bm0$,旋度和散度恒为0
    \begin{align}
        \label{eqn:curlgrad}
        \curl(\nabla\psi)&\equiv \bm 0;\\
        \label{eqn:divcurl}
        \div(\curl\bm A)&\equiv 0.
    \end{align}
\end{theorem}

\begin{theorem}
    {二重微分}{second order differential}
    旋度的旋度
    \begin{equation}
        \label{eqn:curlcurl}
        \curl(\curl\bm A)=\nabla(\div\bm A)-\lapla\bm A.
    \end{equation}
    Laplacian的乘法法则
    \begin{align}
        \lapla(\psi\phi)&=\psi\lapla\phi+2\nabla\psi\cdot\nabla\phi+(\lapla\psi)\phi,\\
        \lapla(\psi\bm A)&=\psi\lapla\bm A+2(\nabla\psi\cdot\nabla)\bm A+(\lapla\psi)\bm A,\\
        \lapla(\bm A\cdot\bm B)&=\bm A\cdot\lapla\bm B-\bm B\cdot\lapla\bm A+2\div[(\bm B\cdot\nabla)\bm A+\bm B\times(\curl\bm A)].
    \end{align}
\end{theorem}

\begin{theorem}
    {三重微分}{third order differential}
    散度、旋度、散度与Laplacian可交换:
    \begin{align}
        \lapla(\nabla\psi)&=\nabla(\lapla\psi),\\
        \lapla(\div\bm A)&=\div(\lapla\bm A),\\
        \lapla(\curl\bm A)&=\curl(\lapla\bm A).
    \end{align}
\end{theorem}

\iffalse
\begin{example}
    {}{}
    \begin{align}
        \nabla r&=\uvec r,\\
        \div\bm r&=3,\\
        \curl\bm r&=\bm 0,\\
        \div(f(r)\uvec r)&=\frac2rf(r)+\pv fr,\\
        \curl(f(r)\uvec r)&=\bm 0,\\
        (\bm a\cdot\nabla)f(r)\uvec r&=[\bm a-(\bm a\cdot\uvec r)\uvec r]\frac{f(r)}r+(\bm a\cdot\uvec r)\uvec r\,\pv fr,\\
        \nabla(\bm a\cdot\bm r)&=\bm a+(\div\bm a)\bm r+\bm r\times(\curl\bm a).
    \end{align}
\end{example}
\fi

\subsection{积分}
\label{ssec:integral calculus}

\begin{compactitem}
	\item 微积分基本定理
	\begin{equation}
        \int_a^b\psi'(x)\d x=\edg{\psi(x)}_a^b=\psi(b)-\psi(a).
    \end{equation}
    \item 分部积分
    \begin{equation}
        \int_a^b\psi(x)\phi'(x)\d x=[\psi(x)\phi(x)]_a^b-\int_a^b\psi'(x)\phi(x)\d x.
    \end{equation}
    其实就是Leibniz法则$+$微积分基本定理$+$移项
    % \begin{equation}
    %     \int_V\psi\div\bm F\d V=\oint_{\p V}\psi\bm F\cdot\d\bm A-\int_V\bm F\cdot\nabla\psi\d V.
    % \end{equation}
\end{compactitem}

\begin{theorem}
    {体积分}{volume integral}
    区域$V$的体积分
    \begin{align}
        \int_V\nabla\psi\d V&=\oint_{\p V}\psi\d\bm A,\\
        \label{eqn:div}
        \int_V\div\bm F\d V&=\oint_{\p V}\bm F\cdot\d\bm A,\\
        \int_V\curl\bm F\d V&=-\oint_{\p V}\bm F\times\d\bm A.
    \end{align}
    其中\eqref{eqn:div}称为散度定理,也称Gauss定理。
\end{theorem}

\begin{theorem}
    {面积分}{surface integral}
    曲面$S$的面积分
    \begin{align}
        \int_S\nabla\psi\times\d\bm A&=-\oint_{\p S}\psi\d\bm\ell,\\
        \label{eqn:curl}
        \int_S(\curl\bm F)\cdot\d\bm A&=\oint_{\p S}\bm F\cdot\d\bm\ell.
    \end{align}
    其中\eqref{eqn:curl}称为旋度定理,也称Stokes定理。
\end{theorem}



\section{曲线坐标系}
\label{sec:curvilinear coordinates}

\subsection{Lam\'e系数}

给定曲线坐标系(curvilinear coordinate)的正交基$\nvec e_1,\nvec e_2,\nvec e_3$,位置矢量
\[
    \bm r=q_1\uvec e_1+q_2\uvec e_2+q_3\uvec e_3,
\]
位置矢量$\bm r=\bm r(q_1,q_2,q_3)$的全微分
\[
    \d\bm r=\pv{\bm r}{q_1}\d q_1+\pv{\bm r}{q_2}\d q_2+\pv{\bm r}{q_3}\d q_3,
\]
由正交性,$\p\bm r/\p q_i$应沿着$\nvec e_i$方向,其模值定义为Lam\'e系数
\begin{equation}
    H_i:=\abs{\pv{\bm r}{q_i}}=\sqrt{\biggkh{\pv x{q_i}}^2+\biggkh{\pv y{q_i}}^2+\biggkh{\pv z{q_i}}^2}.
\end{equation}
因此$\p\bm r/\p q_i=H_i\uvec e_i$,从而
\begin{equation}
    \d\bm r=H_1\d q_1\uvec e_1+H_2\d q_2\uvec e_2+H_3\d q_3\uvec e_3.
\end{equation}
面积微元和体积微元为
\begin{align}
    \d A_i&=H_jH_k\d q_j\nd q_k,\enspace (ijk)=(123);\\
    \d V&=H_1H_2H_3\d q_1\nd q_2\nd q_3.
\end{align}

\begin{theorem}
    {曲线坐标系下的nabla算符}{nabla in curvilinear coordinates}
    在曲线坐标系下,梯度、散度、旋度和Laplacian都可以用Lam\'e系数表示:
    \begin{align}
        \label{eqn:grad Lame}
        \nabla\psi&=\frac1{H_1}\pv\psi{q_1}\uvec e_1+\frac1{H_2}\pv\psi{q_2}\uvec e_2+\frac1{H_3}\pv\psi{q_3}\uvec e_3;\\
        \label{eqn:div Lame}
        \div\bm F&=\frac1{H_1H_2H_3}\biggfkh{\pv{(H_2H_3F_1)}{q_1}+\pv{(H_3H_1F_2)}{q_2}+\pv{(H_1H_2F_3)}{q_3}};\\
        \label{eqn:curl Lame}
        \curl\bm F&=\frac1{H_1H_2H_3}\begin{vmatrix}
            H_1\uvec e_1&H_2\uvec e_2&H_3\uvec e_3\\
            \pp{q_1}&\pp{q_2}&\pp{q_3}\\
            H_1F_1&H_2F_2&H_3F_3
        \end{vmatrix};\\
        \label{eqn:Laplacian Lame}
        \lapla\psi&=\frac1{H_1H_2H_3}\biggfkh{\pp{q_1}\biggkh{\frac{H_2H_3}{H_1}\pv\psi{q_1}}+\pp{q_2}\biggkh{\frac{H_3H_1}{H_2}\pv\psi{q_2}}+\pp{q_3}\biggkh{\frac{H_1H_2}{H_3}\pv\psi{q_3}}}.
    \end{align}
\end{theorem}

\subsection{球坐标系}
\label{ssec:spherical coordinate}

球坐标(spherical coordinate)为$(r,\theta,\phi)$,其中$r$称为径向距离(radial distance),$\theta$称为极角(polar angle),$\phi$称为方位角(azimuthal angle)。如下图所示。
\begin{center}
    \includegraphics[page=\pagenumref{tikz:spherical coordinate}]{figures/tikz/layouts.pdf}
    \captionof{figure}{球坐标系}
    \label{fig:spherical coordinate}
\end{center}
坐标变换
\begin{subequations}
    \begin{align}
        x&=r\sin\theta\cos\phi,\\
        y&=r\sin\theta\sin\phi,\\
        z&=r\cos\theta.
    \end{align}
\end{subequations}
基变换
\begin{equation}
    \begin{bmatrix}
        \uvec r\\\uvec\theta\\\uvec\phi
    \end{bmatrix}=
    \begin{bmatrix}
        \sin\theta\cos\phi&\sin\theta\sin\phi&\cos\theta\\
        \cos\theta\cos\phi&\cos\theta\sin\phi&-\sin\theta\\
        -\sin\phi&\cos\phi&0
    \end{bmatrix}
    \begin{bmatrix}
        \uvec x\\\uvec y\\\uvec z
    \end{bmatrix}
\end{equation}
Lam\'e系数:
\begin{subequations}
    \begin{align}
        H_r&=\sqrt{\biggkh{\pv xr}^2+\biggkh{\pv yr}^2+\biggkh{\pv zr}^2}=1,\\
        H_\theta&=\sqrt{\biggkh{\pv x\theta}^2+\biggkh{\pv y\theta}^2+\biggkh{\pv z\theta}^2}=r,\\
        H_\phi&=\sqrt{\biggkh{\pv x\phi}^2+\biggkh{\pv y\phi}^2+\biggkh{\pv z\phi}^2}=r\sin\theta,
    \end{align}
\end{subequations}
矢量微分
\begin{align}
    \nabla\psi&=\pv\psi r\uvec r+\frac1r\pv\psi \theta\uvec\theta+\frac1{r\sin\theta}\pv\psi \phi\uvec\phi,\\
    \div\bm F&=\frac1{r^2}\pp r(r^2F_r)+\frac1{r\sin\theta}\pp\theta(\sin\theta\,F_\theta)+\frac1{r\sin\theta}\pp\phi F_\phi,\\
    \curl\bm F&=\frac1{r^2\sin\theta}
    \begin{vmatrix}
        \uvec r&r\uvec\theta&r\sin\theta\uvec\phi\\
        \pp r&\pp\theta&\pp\phi\\
        F_r&rF_\theta&r\sin\theta F_\phi
    \end{vmatrix},\\
    \lapla\psi&=\frac1{r^2}\pp r\biggkh{r^2\pv\psi r}+\frac1{r^2\sin\theta}\pp\theta\biggkh{\sin\theta\pv\psi\theta}+\frac1{r^2\sin^2\theta}\pv[2]\psi\phi.
\end{align}

\subsection{柱坐标系}
\label{ssec:cylindrical coordinate}

柱坐标(cylindrical coordinate)为$(\rho,\phi,z)$,对应二维极坐标(polar coordinate) $(\rho,\phi)$,其中$\rho$称为极径(radial distance),$\phi$称为极角(azimuth),$z$为高度(height)。如下图所示。
\begin{center}
    \includegraphics[page=\pagenumref{tikz:cylindrical coordinate}]{figures/tikz/layouts.pdf}
    \captionof{figure}{柱坐标系}
    \label{fig:cylindrical coordinate}
\end{center}
坐标变换
\begin{subequations}
    \begin{align}
        x&=\rho\cos\phi,\\
        y&=\rho\sin\phi,\\
        z&=z.
    \end{align}
\end{subequations}
Lam\'e系数:
\begin{subequations}
    \begin{align}
        H_\rho&=\sqrt{\biggkh{\pv x\rho}^2+\biggkh{\pv y\rho}^2+\biggkh{\pv z\rho}^2}=1,\\
        H_\phi&=\sqrt{\biggkh{\pv x\phi}^2+\biggkh{\pv y\phi}^2+\biggkh{\pv z\phi}^2}=\rho,\\
        H_z&=\sqrt{\biggkh{\pv xz}^2+\biggkh{\pv yz}^2+\biggkh{\pv zz}^2}=1,
    \end{align}
\end{subequations}
矢量微分
\begin{align}
    \nabla\psi&=\pv\psi\rho\uvec\rho+\frac1\rho\pv\psi\phi\uvec\phi+\pv\psi z\uvec z,\\
    \div\bm F&=\frac1\rho\pp\rho(\rho F_\rho)+\frac1\rho\pv{F_\phi}\phi+\pv{F_z}z,\\
    \curl\bm F&=\frac1\rho
    \begin{vmatrix}
        \uvec\rho&\rho\uvec\phi&\uvec z\\
        \pp\rho&\pp\phi&\pp z\\
        F_\rho&\rho F_\phi&F_z
    \end{vmatrix},\\
    \lapla\psi&=\frac1\rho\pp\rho\biggkh{\rho\pv\psi\rho}+\frac1{\rho^2}\pv[2]\psi\phi+\pv[2]\psi z.
\end{align}

\section{Dirac函数}

\begin{example}{引入Dirac $\delta$函数}{introduce delta function}
    计算$\bm F(\bm r)=\uvec r/r^2$的散度(不包括原点):
    \[
        \div\bm F=\frac1{r^2}\pp r\Bigkh{r^2\cdot\frac1{r^2}}=0,
    \]
    但考虑其在半径为$r$球面上的面积分:
    \[
        \oint_{\p V}\bm F\cdot\d\bm A=\oint_{\p V}\frac1{r^2}r^2\sin\theta\d\theta\nd\phi=\int_0^{2\pi}\int_0^\pi\sin\theta\d\theta\nd\phi=4\pi.
    \]
    这与散度定理\eqref{eqn:div}矛盾!
    原因在于原点处的散度是奇点(singularity)。
    % 原因在于:$\div\bm F$在除了原点以外的任意一点均为0,但其体积分为$4\pi$。
\end{example}
\begin{definition}{Dirac $\delta$函数}{Dirac delta function}
    $\delta$函数是一个将任意函数$f$映射到$f(0)$的线性泛函(linear functional),即
    \[
        \int\iti f(x)\vd(x)\d x=f(0).
    \]
    其特点为:$\forall x\neq 0,\,\delta(x)=0$,但
    \[
        \int\iti\delta(x)\d x=1,
    \]
    因此$\delta$函数并不是一个严格意义上的函数,而是一个广义函数(general function)或分布(distribution)。
    %其只有出现在积分中才有实质意义。
    在物理上,$\delta$函数可以代表质点(或点电荷)的密度。
\end{definition}

\begin{corollary}
    \exmref{exm:introduce delta function} 中的散度其实应为$4\pi\vd(\bm r)$,进而
    \begin{equation}
        \lapla\biggkh{\frac1r}=-\div\biggkh{\frac{\uvec r}{r^2}}=-4\pi\vd(\bm r).
    \end{equation}
\end{corollary}

\section{向量场理论}

我们只讨论性质足够好的(well-behaved)向量场$\bm F$,而不考虑分析学中的那些反例。具体来说,$\bm F$在区域$V$中二次连续可微。

\begin{theorem}{无旋场与标量势}{irrotational field and scalar potential}
    对于无旋场$\bm F$,下列命题是等价的:
    \begin{enumerate}
        \item $\curl\bm F=\bm 0$;
        \item $\bm F$可以写成标量势(scalar potential)的梯度:$\bm F=-\nabla\Phi$;
        \item 给定路径$C$的起点终点$A,B$,线积分与路径无关
        \begin{equation}
            \int_A^B\bm F\cdot\d\bm\ell=\Phi(A)-\Phi(B);
        \end{equation}
        \item 闭合线积分恒为0.
    \end{enumerate}
\end{theorem}
\begin{proof}
    由式\eqref{eqn:curlgrad}可知$(2)\implies(1)$;易知$(3)\iff(4)$;由Stokes定理\eqref{eqn:curl}可知$(4)\iff(1)$;最后确定参考点$\Phi=0$后,可由$(3)\implies(2)$得到$\Phi$。
\end{proof}
\begin{remark}
    $\Phi$并不唯一,$\Phi'=\Phi+\const$可以给出同一个$\bm F$,需要确定$\Phi=0$的参考点,最简单的参考点是无穷远处$\Phi(\infty)=0$。
    \[
        \Phi(\bm r)=\int_\infty^{\bm r}\bm F\cdot\d\bm\ell.
    \]
\end{remark}

\begin{theorem}{无源场与矢量势}{solenoidal field and vector potential}
    对于无源场$\bm F$,下列命题是等价的:
    \begin{enumerate}
        \item $\div\bm F=0$;
        \item $\bm F$可以写成矢量势(vector potential)的旋度:$\bm F=\curl\bm A$;
        \item 给定面$S$的边界$\p S$,面积分与表面形状无关
        \begin{equation}
            \int_S\bm F\cdot\d\bm A=\oint_{\p S}\bm A\cdot\d\bm\ell;
        \end{equation}
        \item 闭合面积分恒为0.
    \end{enumerate}
\end{theorem}
\begin{proof}
    由式\eqref{eqn:divcurl}可知$(2)\implies(1)$;易知$(3)\iff(4)$;由Gauss定理\eqref{eqn:div}可知$(4)\iff(1)$;最后可通过$(3)\implies(2)$构造$\bm A$。
\end{proof}

\begin{remark}
    $\bm A$并不唯一,$\bm A'=\bm A+\nabla\varLambda$可以给出同一个$\bm F$。
    可以通过确定一个规范(gauge)来唯一确定$\bm A$,比如最简单的Coulomb规范$\div\bm A=0$。
\end{remark}

\begin{theorem}{Helmholtz分解}{Helmholtz decomposition}
    向量场$\bm F$可以被唯一地分解为无旋场和无源场的叠加:
    \begin{equation}
        \bm F=-\nabla\Phi+\curl\bm A,
    \end{equation}
    其中
    \begin{subequations}
        \begin{align}
            \Phi(\bm r)&=\frac1{4\pi}\int_V\frac{\div[']\bm F(\bm r')}{\abs{\bm r-\bm r'}}\d V'-\frac1{4\pi}\oint_{\p V}\frac{\bm F(\bm r')}{\abs{\bm r-\bm r'}}\cdot\d\bm A,\\
            \bm A(\bm r)&=\frac1{4\pi}\int_V\frac{\curl[']\bm F(\bm r')}{\abs{\bm r-\bm r'}}\d V'+\frac1{4\pi}\oint_{\p V}\frac{\bm F(\bm r')}{\abs{\bm r-\bm r'}}\times\d\bm A.
        \end{align}
    \end{subequations}
\end{theorem}

\begin{proof}
    利用$\delta$函数
    \[
        \bm F(\bm r)=\int_V\bm F(\bm r')\vd(\bm r-\bm r')\d V'=-\frac1{4\pi}\int_V\bm F(\bm r')\lapla\biggkh{\frac1{\abs{\bm r-\bm r'}}}\d V'.
    \]
    剩下的展开部分在推导$\Phi$和$\bm A$时有体现,此处不再赘述。    
\end{proof}

\begin{corollary}
    若$V$无界且$\bm F=o(1/r)$,则
    \begin{subequations}
        \begin{align}
            \Phi(\bm r)&=\frac1{4\pi}\int_V\frac{\div[']\bm F(\bm r')}{\abs{\bm r-\bm r'}}\d V',\\
            \bm A(\bm r)&=\frac1{4\pi}\int_V\frac{\curl[']\bm F(\bm r')}{\abs{\bm r-\bm r'}}\d V'.
        \end{align}
    \end{subequations}
\end{corollary}

\sectionstar{并矢}

对于形如$\bm A(\bm B\cdot\bm C)$的表达式,我们可以将其视为一个线性算符:
\[
    \bm C\mapsto\bm A(\bm B\cdot\bm C),
\]

\begin{definition}
    {并矢}{dyadic}
    矢量$\bm A,\bm B$的并矢(dyadic)是一个张量积(tensor product),记为$\bm A\bm B$,满足
    \begin{subequations}
        \begin{align}
            \bm A\bm B\cdot\bm C&=\bm A(\bm B\cdot\bm C),\\
            \bm C\cdot\bm A\bm B&=(\bm C\cdot\bm A)\bm B.
        \end{align}
    \end{subequations}
\end{definition}

\begin{corollary}
    在确定一组基$\uvec e_1,\uvec e_2,\uvec e_3$后,
    \begin{align*}
        \bm A&=A_1\uvec e_1+A_2\uvec e_2+A_3\uvec e_3,\\
        \bm B&=B_1\uvec e_1+B_2\uvec e_2+B_3\uvec e_3,
    \end{align*}
    并矢
    \[
        \bm A\bm B=\sum_{i,j=1}^3 A_iB_j\uvec e_i\uvec e_j.
    \]
    其中$\uvec e_i\uvec e_j$是并矢的基,满足
    \begin{subequations}
        \begin{align}
            \uvec e_i\uvec e_j\cdot\uvec e_k&=\uvec e_i(\uvec e_j\cdot\uvec e_k)=\delta_{jk}\uvec e_i,\\
            \uvec e_i\cdot\uvec e_j\uvec e_k&=(\uvec e_i\cdot\uvec e_j)\uvec e_k=\delta_{ij}\uvec e_k.
        \end{align}
    \end{subequations}
\end{corollary}

\begin{definition}
    {双重点积}{double dot product}
    两个并矢$\bm A\bm B$和$\bm C\bm D$的双重点积(double dot product)定义为
    \begin{equation}
        \bm A\bm B:\bm C\bm D=(\bm A\cdot\bm D)(\bm B\cdot\bm C).
    \end{equation}
\end{definition}

\begin{corollary}
    对于基来说
    \begin{equation}
        (\uvec e_i\uvec e_j):(\uvec e_k\uvec e_\ell)=\delta_{i\ell}\delta_{jk}.
    \end{equation}
\end{corollary}

\iffalse
\begin{corollary}
    引入Einstein求和约定,并矢可表示为
    \begin{subequations}
        \begin{align}
            (\bm A\bm B)_{ij}&=A_iB_j,\\
            (\bm A\bm B\cdot\bm C)_i&=A_iB_jC^j,\\
            (\bm C\cdot\bm A\bm B)_j&=C^iA_iB_j,\\
            (\bm A\bm B:\bm C\bm D)&=A_iB_jC^jD^i.
        \end{align}
    \end{subequations}
\end{corollary}
\fi

\chapter{特殊函数}

\section{Legendre多项式}
\label{sec:Legendre polynomial}

广义(generalized) Legendre方程:
\begin{equation}
    \dd x\biggfkh{(1-x^2)\dv Px}+\biggfkh{\ell(\ell+1)-\frac{m^2}{1-x^2}}P=0.
\end{equation}
考虑$m=0$的情形,得到Legendre方程:
\begin{equation}
    \dd x\biggfkh{(1-x)^2\dv Px}+\ell(\ell+1)P=0.
\end{equation}
采用级数解,设
\[
    P(x)=x^\alpha\sum_{j=0}^\infty a_jx^j.
\]
$\alpha$是待定系数,从而 
\[
    \sum_{j=0}^\infty(\alpha+j)(\alpha+j-1)a_j x^{\alpha+j-2}-\sum_{j=0}^\infty\bigfkh{(\alpha+j)(\alpha+j+1)-\ell(\ell+1)}a_jx_{\alpha+j}=0.
\]
解得
\[
    \begin{cases}
        a_0\alpha(\alpha-1)=0,&j=0\\
        a_1\alpha(\alpha+1)=0,&j=1\\
        a_{j+2}=\frac{(\alpha+j)(\alpha+j+1)-\ell(\ell+1)}{(\alpha+j+1)(\alpha+j+2)}a_j,&j\geqslant 0
    \end{cases}
\]
对初始条件分情况讨论:
\begin{compactitem}
	\item $a_0=a_1=0$,解$P\equiv 0$是平凡的;
	\item $a_0=0,\enspace a_1\neq 0$,展开式事实上与$a_0\neq 0,\enspace a_1=0$等价:
	\[
        P(x)=x^\alpha(a_1x+a_3x^3+\cdots)=x^{\alpha+1}(a_1+a_3x^2+\cdots);
    \]
	\item $a_0\neq 0,\enspace a_1\neq 0$,由递推关系知,级数会在$x=\pm 1$发散。
\end{compactitem}
综上,我们最终选取$a_0\neq 0,\enspace a_1=0$,从而$\alpha(\alpha-1)=0$,
\[
    P(x)=\begin{cases}
        a_0+a_2x^2+a_4x^4+\cdots,&\alpha=0\\
        a_0x+a_2x^3+a_4x^5+\cdots,&\alpha=1
    \end{cases}
\]
这个级数在$\abs{x}<1$时当然收敛,但在$x=\pm 1$时会发散,除非级数在某一项截断(即此项之后的系数均为0)。考察递推关系:
\[
    a_{j+2}=\begin{cases}
        \frac{j(j+1)-\ell(\ell+1)}{(j+1)(j+2)}a_j,&\alpha=0\\[2ex]
        \frac{(j+1)(j+2)-\ell(\ell+1)}{(j+2)(j+3)}a_j,&\alpha=1
    \end{cases}
\]
因此,收敛性要求$\ell=0,1,2,\ldots$
\begin{theorem}{Legendre多项式的递推关系}{recurrence relation among Legendre polynomial}
    可直接从Rodrigues公式和Legendre方程得到:
    \begin{subequations}
        \begin{align}
            \label{eqn:Legendre recurrence 1}
            (2\ell+1)P_\ell&=P'_{\ell+1}-P'_{\ell-1},\\
            (\ell+1)P_{\ell+1}&=(2\ell+1)xP_\ell+\ell P_{\ell-1},\\
            P'_{\ell+1}&=xP'_\ell+\ell(\ell+1)P_\ell,\\
            (x^2-1)P'_\ell&=\ell xP_\ell-\ell P_{\ell-1}.
        \end{align}
    \end{subequations}
\end{theorem}
下面证明递推式\eqref{eqn:Legendre recurrence 1}。
\begin{proof}
    由Rodrigues公式
    \begin{equation}
        P_\ell(x)=\frac1{2^\ell\ell!}\dd[\ell]x(x^2-1)^\ell.
    \end{equation}
    可得
    \begin{align*}
        P'_\ell&=\frac1{2^\ell\ell!}\dd[\ell+1]x(x^2-1)^\ell=\frac1{2^{\ell-1}(\ell-1)!}\dd[\ell]x\bigfkh{x(x^2-1)^{\ell-1}}\\
        &=\frac1{2^{\ell-1}(\ell-1)!}\dd[\ell-1]x\bigfkh{\bigkh{(2\ell-1)x^2-1}(x^2-1)^{\ell-2}}.
    \end{align*}
    从而 
    \begin{align*}
        P_{\ell+1}'-P_{\ell-1}'&=\frac1{2^\ell\ell!}\dd[\ell]x\bigfkh{\bigkh{(2\ell+1)x^2-1}(x^2-1)^{\ell-1}}-\frac1{2^{\ell-1}(\ell-1)!}\dd[\ell]x(x^2-1)^{\ell-1}\\
        &=\frac1{2^\ell\ell^2}\dd[\ell]x\bigfkh{(2\ell+1)(x^2-1)^\ell}=(2\ell+1)P_\ell.\qedhere
    \end{align*}
\end{proof}


\section{Bessel函数}
\label{sec:Bessel function}

Bessel方程
\begin{equation}
    \dv[2]Rx+\frac1x\dv Rx+\biggkh{1-\frac{\nu^2}{x^2}}R=0.
\end{equation}
采用级数解,设 
\[
    R(x)=x^\alpha\sum_{j=0}^\infty a_jx^j.
\]
解得
\[
    \begin{cases}
        \alpha^2=\nu^2,&j=0\\
        a_1\bigfkh{(\alpha+1)^2-\nu^2}=0,&j=1\\
        a_{j-2}=\bigfkh{\nu^2-(j+\alpha)^2}a_j,&j\geqslant 2
    \end{cases}
\]
故$\alpha=\pm\nu,\enspace a_1=0$,从而
\[
    a_{2j}=-\frac1{4j(j+\alpha)}a_{2j-2},\quad j=1,2,\ldots
\]
选取\footnote{$\nu$可以不是整数,此时阶乘$\nu!$推广为$\Gamma(\nu+1)$。}
\[
    \alpha=\nu,\enspace a_0=\frac1{2^\nu\nu!}.
\]
解得(第一类) Bessel函数
\begin{equation}
    J_\nu(x)=\sum_{j=0}^\infty\frac{(-)^j}{j!(j+\nu)!}\Bigkh{\frac x2}^{2j+\nu},
\end{equation}
当$\nu$不为整数时,$J_{-\nu}$和$J_\nu$是线性无关的,可作为Bessel方程的解系;但当$\nu$为整数时,二者线性相关:
\[
    J_{-\nu}(x)=(-)^\nu J_\nu(x),
\]
为了得到线性无关的另一解,定义Neumann函数(第二类Bessel函数)
\begin{equation}
    Y_\nu(x):=\frac{J_\nu(x)\cos(\nu\pi)-J_{-\nu}(x)}{\sin(\nu\pi)}.
\end{equation}
当$\nu$为整数时,$Y_\nu(x)$的值定义为上式$\nu$趋近于该整数的极限(即可去间断点)。可以证明,$J_\nu,Y_\nu$线性无关。

Bessel方程的另一对重要线性无关解是Hankel函数(第三类Bessel函数)
\begin{equation}
    H_\nu^\pm(x)=J_\nu(x)\pm\i Y_\nu(x).
\end{equation}
三类Bessel函数均满足形如下面的递推公式
\begin{align*}
    \Omega_{\nu-1}+\Omega_{\nu+1}&=\frac{2\nu}x\Omega_\nu,\\
    \Omega_{\nu-1}-\Omega_{\nu+1}&=2\Omega'_\nu.
\end{align*}


\begin{example}{Bessel函数的渐进性}{limit of Bessel function}
    $x\to 0$时, 
    \begin{align*}
        J_\nu(x)&\simeq\frac1{\nu!}\Bigkh{\frac x2}^\nu,\quad\\
        Y_\nu(x)&\simeq
        \begin{cases}
            \frac2\pi\Bigfkh{\ln\Bigkh{\frac x2}+\gamma},&\nu=0\\[1ex]
            -\frac{(\nu-1)!}\pi\Bigkh{\frac x2}^{-\nu}+\frac{\cot(\nu\pi)}{\nu!}\Bigkh{\frac x2}^\nu,&\nu\neq 0
        \end{cases}
    \end{align*}
    其中$\gamma$是Euler-Mascheroni常数
    \[
        \gamma:=\lim_{n\to\infty}\biggkh{-\ln n+\sum_{k=1}^n\frac1k}=\num{0.5772156649}\ldots.
    \]
    \tcblower
    $x\to\infty$时, 
    \begin{align*}
        J_\nu(x)&\simeq\sqrt{\frac2{\pi x}}\cos\biggkh{x-\frac{\nu\pi}2-\frac\pi{4}},\\
        Y_\nu(x)&\simeq\sqrt{\frac2{\pi x}}\sin\biggkh{x-\frac{\nu\pi}2-\frac\pi{4}}.
    \end{align*}
\end{example}
若将$k$替换为$\i k$,便会得到修正(modified) Bessel方程
\[
    \dv[2]Rx+\frac1x\dv Rx-\biggkh{1+\frac{\nu^2}{x^2}}R=0.
\]
修正Bessel函数只是纯虚数参数的Bessel函数
\[
    I_\nu(x)=\i^{-\nu}J_\nu(\i x),\quad K_\nu(x)=\i^{\nu+1}\frac\pi 2H_\nu^+(\i x).
\]

