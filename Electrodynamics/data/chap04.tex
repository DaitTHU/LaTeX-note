\renewcommand{\mag}{\mathrm m}

\chapter{静磁学、时变场}

\label{chap:magnetostatics}

本章主要介绍静磁学(magnetostatics)的内容,与前几章介绍的静电学在多处均有呼应。%当然,本文为节省篇幅省略了一些科学史的介绍及磁学的发展研究历程,感兴趣的读者可以自行查找相关资料了解。同时,对于一些过于基础性的概念也仅仅只做大致的描述。
类似于静电学,我们建立了静磁学的基本方程组,通过几个边值问题阐述了静磁学问题的处理方法。%(如均匀带电圆环在全空间产生的磁场……)。
%,……
%第三节介绍似稳场(quasi-static field)理论,……

第二节介绍Faraday定律,将静电学和静磁学中的恒稳态问题变为与时间有关的时变场(time-varying field)问题。

\section{磁感应强度、电流密度}
\label{sec:introduction to magnetostatics}

%值得注意的是,
磁场(magnetic field)的基本规律最早并不是直接根据人类对磁性材料的接触中得出的,原因在于静磁学和静电学存在一个根本上的差别:人们没有发现自由磁荷(free magnetic charge)的存在。
在磁学中,基本单元是磁偶极子(magnetic dipole),用于测量磁场的磁偶极子一般是小磁针,对应电学中的试探电荷。
\begin{definition}
    {磁感应强度}{magnetic induction intensity}
    磁场通常用磁感应强度(magnetic induction intensity) $\bm B$描述,其方向定义为磁偶极子平衡时磁矩$\bm\mu$的方向,大小可以由磁偶极子偏转时所受力矩$\bm N$来衡量:
    \begin{equation}
        \bm N=\bm\mu\times\bm B.
    \end{equation}
\end{definition}
Ørsted等人的实验发现:通电导线可使小磁针偏转,也就是说磁场可以由电流产生。
\begin{definition}
    {电流密度}{current density}
    电流(current)相当于运动电荷,其大小为单位时间通过某界面的电荷量:
    \begin{equation}
        I:=\dv qt,
    \end{equation}
    电流也可以用电流密度(current density)矢量$\bm J$描述:
    \begin{equation}
        \bm J:=\rho\bm v,
    \end{equation}
    其大小为单位时间流过单位面积的电荷量,其方向为电荷运动的方向。
\end{definition}

\begin{corollary}
    对于离散情形,
    \begin{equation}
        \bm J=\sum_i q_i\bm v_i\vd(\bm r-\bm r_i).
    \end{equation}
\end{corollary}

\begin{theorem}
    {连续性方程}{continuity equation}
    由于电荷守恒(conservation of charge),电荷密度和电流密度满足连续性方程(continuity equation)
    \begin{equation}
        \label{eqn:continuity}
        \pv\rho t+\div\bm J=0,
    \end{equation}
\end{theorem}
\begin{proof}
    考虑空间一体积微元$\vd V$,其内电荷量不随时间改变:
    \begin{equation*}
        \label{eqn:d(rhoV)/dt=0}
        \dv{(\rho\vd V)}t=\dv\rho t\vd V+\rho\dv{(\delta V)}t=0,\tag{$\ast$}
    \end{equation*}
    由于$\rho=\rho(x,y,z,t)$是时空的函数,其全微分为
    \[
        \d\rho=\pv\rho t\d t+\pv\rho x\d x+\pv\rho y\d y+\pv\rho z\d z,
    \]
    其迁移导数(convective derivative)为
    \[
        \dv\rho t=\pv\rho t+\pv\rho xv_x+\pv\rho yv_y+\pv\rho zv_z=\pv\rho t+\bm v\cdot\nabla\rho;
    \]
    而体积元的变化率
    \[
        \dv{(\delta V)}t=\oint_{\p(\delta V)}\bm v\cdot\d\bm A=\int_{\delta V}\div\bm v\d V=\vd V\div\bm v.
    \]
    因此\eqref{eqn:d(rhoV)/dt=0}变为
    \[
        \dv\rho t+\rho\div\bm v=\pv\rho t+\bm v\cdot\nabla\rho+\rho\div\bm v=\pv\rho t+\div(\rho\bm v)=0.
        \qedhere
    \]
\end{proof}

\begin{definition}
    {稳恒电流}{steady current}
    稳恒(steady-state)磁现象满足空间各处的$\rho$不变,于是连线性方程\eqref{eqn:continuity}变为
    \begin{equation}
        \div\bm J=0.
    \end{equation}
\end{definition}
%我们从电流与磁感应强度之间的关系出发,建立静磁学基本定律。

\section{Biot-Savart定律}
\label{sec:Biot-Savart law}

%尽管磁现象的发现有很悠久的历史,但直到建立了电流和磁场之间的联系,磁现象才得到定量描述。
为了描述电流在激发磁场时的作用,我们引入电流元(current element)的概念:$\d\bm\ell$是通电流$I$的导线的长度元,指向电流方向,则电流元为$I\d\bm\ell$。
\begin{theorem}{Biot-Savart定律}{Biot and Savart Law}
    原点处电流元$I\d\bm\ell$在$\bm r$处激发的磁感应强度$\d\bm B$为
    \begin{equation}
        \label{eqn:B-S law}
        \d\bm B=k_\mag\frac{I\d\bm\ell\times\uvec r}{r^2},
    \end{equation}
    其中$k_\mag$是比例系数,在SI单位制中,
    \[
        k_\mag=\frac{\mu_0}{4\pi}\approx\SI{1e-7}{N/A^2},
    \]
    其中真空磁导率$\mu_0\approx 4\pi\times{10^{-7}}\,\si{N/A^2}.$
\end{theorem}
\begin{remark}
    尽管电流元磁感应强度\eqref{eqn:B-S law}与点电荷电场强度\eqref{eqn:E of q}有相似的形式,但电流元只有在积分中才有意义,因为电流元本身并不满足连续性方程\eqref{eqn:continuity}。
    \footnote{将电流元$I\d\bm\ell$换成运动电荷$q\bm v$可以满足连续性方程,但得到的表达式
    \[
        \bm B=k_\mag\frac{q\bm v\times\uvec r}{r^2}.
    \]
    与时间相关,并且只有当电荷速度$v\ll c$且加速度可忽略不计时才成立。
    而式\eqref{eqn:B-S law}是精确的:
    考虑相对论电磁场理论(包括加速度效应),
    在电荷单元$e\to0$极限下,稳恒电流系统会激发静磁场,并且形式上等于对式\eqref{eqn:B-S law}遍历电流积分所得的场。}
\end{remark}
\begin{example}{无限长直导线的磁场}{B of straight line}
    $z$轴上的长直导线电流方向为$+z$,距离其$\rho$处的磁感应强度方向服从右手定则,即$\uvec\phi$方向,
    大小为
    \[
        B_\phi=\frac{\mu_0}{4\pi}\int\iti\frac{I\rho\d z}{(z^2+\rho^2)^{3/2}}=\frac{\mu_0}{2\pi}\frac I\rho.
    \]
\end{example}

\begin{theorem}
    {Ampère力}{Ampère force}
    电流元$I\d\bm\ell$在磁感应强度$\bm B$中受Ampère力
    \begin{equation}
        \d\bm F=I\d\bm\ell\times\bm B;
    \end{equation}
\end{theorem}

\begin{example}{闭合回路之间的力}{F of closed lines}
    由闭合电流回路2产生的磁场$\bm B$对闭合电流回路1的总力:
    \begin{align*}
        \bm F_{12}&=\frac{\mu_0}{4\pi}I_1I_2\oint\oint\frac{\d\bm\ell_1\times(\d\bm\ell_2\times\uvec r_{12})}{\bm r_{12}^2}\\
        &=\frac{\mu_0}{4\pi}I_1I_2\oint\oint\frac{(\uvec r_{12}\cdot\d\bm\ell_1)\d\bm\ell_2-\uvec r_{12}(\d\bm\ell_1\cdot\d\bm\ell_2)}{\bm r_{12}^2}\\
        % &=\frac{\mu_0}{4\pi}I_1I_2\biggfkh{\oint\cancel{\oint\frac{\uvec r_{12}\cdot\d\bm\ell_1}{\bm r_{12}^2}}\d\bm\ell_2-\oint\oint\frac{\bm r_{12}}{\abs{\bm r_{12}}^3}(\d\bm\ell_1\cdot\d\bm\ell_2)}\\
        &=-\frac{\mu_0}{4\pi}\oint\oint\frac{I_1I_2}{r_{12}^2}\uvec r_{12}(\d\bm\ell_1\cdot\d\bm\ell_2).
    \end{align*}
    利用\eqref{eqn:bac-cab}将双叉乘化为更对称的形式,第一项的环路积分为0:
    \[
        \oint\frac{\uvec r_{12}\cdot\d\bm\ell_1}{\bm r_{12}^2}=\oint\frac{\d r_{12}}{r_{12}^2}=0.
    \]
    第二项是对称的,因此满足Newton第三定律。
\end{example}
\begin{example}{两条长平行直导线之间的力}{F of 2 parallel lines}
    由\exmref{exm:B of straight line},两条长平行直导线之间的力为
    \begin{equation}
        \dv F\ell=\frac{\mu_0}{2\pi}\frac{I_1I_2}d.
    \end{equation}
    旧的SI单位制便由此式及$\mu_0$定义电流单位A。
\end{example}

\paragraph{电流密度表示}

电流元可以写成电流密度$\bm J$的形式
\[
    I\d\bm\ell=\d q\bm v=(\rho\d V)\bm v=\bm J\d V,
\]
因此磁场中电流受到的力可以写成:
\[
    \bm F=\int_V\d q(\bm v\times\bm B)=\int_V\rho(\bm v\times\bm B)\d V=\int_V\bm J\times\bm B\d V.
\]
\begin{definition}
    {Lorentz力}{Lorentz force}
    单个电荷$q$在磁场中受到的Lorentz力为
    \begin{equation}
        \bm F=q\bm v\times\bm B.
    \end{equation}
\end{definition}

\begin{corollary}
    Lorentz力不做功。
\end{corollary}

% 总力矩:
% \[
%     \bm N=\bm r\times\bm F=\int_V\bm r\times(\bm J\times\bm B)\d V.
% \]

\begin{theorem}
    {磁场的Gauss定律}{Gauss's law for magnetism}
    静磁场无源
    \begin{equation}
        \label{eqn:divB}
        \div\bm B=0,
    \end{equation}
\end{theorem}

\begin{proof}
    利用电流密度,Biot-Savart定律\eqref{eqn:B-S law}可以写成旋度的形式
    \begin{equation}
        \label{eqn:B(x)}
        \bm B(\bm r)=\frac{\mu_0}{4\pi}\int_V\bm J(\bm r')\times\frac{\bm r-\bm r'}{\abs{\bm r-\bm r'}^3}\d V'
        =\frac{\mu_0}{4\pi}\curl\int_V\frac{\bm J(\bm r')}{\abs{\bm r-\bm r'}}\d V'.
    \end{equation}
    由\eqref{eqn:divcurl}即得。
\end{proof}

\begin{theorem}
    {Ampère定律的微观形式}{Ampère's law in microscopic form}
    磁场的旋度
    \begin{equation}
        \label{eqn:curlB}
        \curl\bm B=\mu_0\bm J.
    \end{equation}
\end{theorem}

\begin{proof}
    利用\eqref{eqn:curlcurl}
    \begin{align*}
        \curl\biggfkh{\curl\frac{\bm J(\bm r')}{\abs{\bm r-\bm r'}}}&=\nabla\biggkh{\div\frac{\bm J(\bm r')}{\abs{\bm r-\bm r'}}}-\lapla\biggkh{\frac{\bm J(\bm r')}{\abs{\bm r-\bm r'}}}\\
        &=\nabla\biggfkh{\frac{\cancel{\div\bm J(\bm r')}}{\abs{\bm r-\bm r'}}+\bm J(\bm r')\cdot\nabla\biggkh{\frac1{\abs{\bm r-\bm r'}}}}+4\pi\bm J(\bm r')\vd(\bm r-\bm r').
    \end{align*}
    则
    \[
        \curl\bm B=\frac{\mu_0}{4\pi}\biggfkh{-\nabla\int_V\bm J(\bm r')\cdot\nabla'\biggkh{\frac1{\abs{\bm r-\bm r'}}}\d V'+4\pi\bm J(\bm r)},
    \]
    对积分项进行分部积分:
    \[
        \int_V\bm J(\bm r')\cdot\nabla'\biggkh{\frac1{\abs{\bm r-\bm r'}}}\d V'=\cancel{\oint_{\p V}\frac{\bm J(\bm r')}{\abs{\bm r-\bm r'}}\cdot\d\bm A'}-\int_V\frac{\cancel{\nabla'\cdot\bm J(\bm r')}}{\abs{\bm r-\bm r'}}\d V'=0.
    \]
    注意积分区域$V$必须考虑所有的电流分布,因此边界(一般是无穷远处)项为0。
\end{proof}
\iffalse
\begin{align*}
    \curl\bm B&=\frac{\mu_0}{4\pi}\curl\biggfkh{\curl\int_V\frac{\bm J(\bm r')}{\abs{\bm r-\bm r'}}\d V'}\\
    &\qquad\textcolor{gray}{\curl(\curl\bm A)=\nabla(\div\bm A)-\lapla\bm A}\\
    &=\frac{\mu_0}{4\pi}\biggfkh{\nabla\int_V\div\frac{\bm J(\bm r')}{\abs{\bm r-\bm r'}}\d V'-\int_V\lapla\frac{\bm J(\bm r')}{\abs{\bm r-\bm r'}}\d V'}\\
    &\qquad\textcolor{gray}{\div(\varphi\bm A)=A\cdot\nabla\varphi+\varphi\div\bm A}\\
    &=\frac{\mu_0}{4\pi}\biggfkh{\nabla\int_V\biggkh{\bm J(\bm r')\cdot\nabla\frac1{\abs{\bm r-\bm r'}}+0}\d V'-\int_V\bm J(\bm r')\lapla\frac1{\abs{\bm r-\bm r'}}\d V'}\\
    %=\frac{\mu_0}{4\pi}\biggfkh{-\nabla\int_V\bm J(\bm r')\cdot\nabla'\frac1{\abs{\bm r-\bm r'}}\d V'+4\pi\bm J(\bm r)}\\
    &\qquad\textcolor{gray}{\int_V\bm A\cdot\nabla f\d V=-\int_Vf\div\bm A\d V+\oint_{\p V}f\bm A\cdot\d\bm A}\\
    &=\mu_0\bm J(\bm r)+\frac{\mu_0}{4\pi}\nabla\biggfkh{\int_V\frac{\cancel{\nabla'\cdot\bm J(\bm r')}}{\abs{\bm r-\bm r'}}\d V'-\cancel{\oint_{\p V}\frac{\bm J(\bm r')\cdot\uvec n'}{\abs{\bm r-\bm r'}}\cdot\d A'}};
\end{align*}
\fi

\begin{theorem}{Ampère定律}{Ampere's law}
    磁感应强度沿闭合曲线$\p S$的环量与穿过$S$的电流$I_i$的关系为:
    \begin{equation}
        \label{eqn:Ampere}
        \oint_{\p S}\bm B\cdot\d\bm\ell=\mu_0\sum_S I_i.
    \end{equation}
\end{theorem}

\begin{proof}
    对\eqref{eqn:curlB}两边取$S$的面积分
    \[
        \int_S(\curl\bm B)\cdot\d\bm A=\mu_0\int_S\bm J\cdot\d\bm A.
    \]
    右边积分项即穿过$S$的电流$I$。
\end{proof}

\begin{remark}
    在高度对称的条件下,可利用Ampère定律得到磁感应强度的大小。
    比如\exmref{exm:B of straight line} 可以直接得到$B\cdot 2\pi\rho=\mu_0I$。
\end{remark}

\section{矢量势}
\label{sec:vector potential}

\begin{definition}{矢量势}{vector potential}
    由于$\bm B$可以写成旋度的形式,可
    定义矢量势(vector potential)满足
    \begin{equation}
        \label{eqn:B=curlA}
        \bm B=\curl\bm A.
    \end{equation}
\end{definition}

\begin{theorem}
    {规范变换}{gauge transformation}
    对$\bm A$进行规范(gauge)变换:
    \[
        \bm A\mapsto\bm A'=\bm A+\nabla\varLambda,
    \]
    其中$\varLambda$为任意标量场。磁场在规范变换下是不变的。
\end{theorem}

\begin{proof}
    \[
        \curl\bm A'=\curl(\bm A+\nabla\varLambda)=\curl\bm A+\bm 0=\bm B.
        \qedhere    
    \]
\end{proof}

\begin{corollary}
    因此$\bm A$并不唯一,有必要选定一个额外的限制条件,称为规范条件(gauge condition)。
\end{corollary}

\begin{definition}
    {Coulomb规范}{Coulomb gauge}
    一个方便的选择是Coulomb规范:
    \begin{equation}
        \div\bm A=0.
    \end{equation}
    从而$\lapla\varLambda=0$,在无边界条件时,$\varLambda=\const$,$\bm A$形式即最简洁的:
    \begin{equation}
        \label{eqn:A-J}
        \bm A(\bm r):=\frac{\mu_0}{4\pi}\int\frac{\bm J(\bm r')}{\abs{\bm r-\bm r'}}\d V',
    \end{equation}
\end{definition}

\begin{corollary}
    在Coulomb规范下,
    我们也可以得到$\bm A$三个分量的Poisson方程:
    \[
        \curl\bm B=\curl(\curl\bm A)=\nabla(\cancel{\div\bm A})-\lapla\bm A.
    \]
    即
    \begin{equation}
        \lapla\bm A=-\mu_0\bm J.
    \end{equation}
\end{corollary}

\section{磁矩}
\label{sec:magnetic moment}

\begin{example}{环形电流回路的磁场}{B of loop current}
    在球坐标中,环形电流的电流密度$\bm J$仅有$\phi$分量
    \[
        J_\phi=I\sin\theta'\vd(\cos\theta')\frac{\delta(r'-a)}a.
    \]
    在Coulomb规范下,矢量势$\bm A$也仅有$\phi$分量:
    \begin{align*}
        A_\phi&=\frac{\mu_0}{4\pi}\int\frac{J_\phi}{\abs{\bm r-\bm r'}}\d V'\\
        &=\frac{\mu_0I}{4\pi a}\int\zti\oint\frac{\sin\theta'\vd(\cos\theta')\vd(r'-a)r^2\d\Omega'\d r'}{\sqrt{r^2+r'^2-2rr'(\cos\theta\cos\theta'+\sin\theta\sin\theta'\cos\phi')}}\\
        &=\frac{\mu_0Ia}{4\pi}\int_0^{2\pi}\frac{\cos\phi'\d\phi'}{\sqrt{a^2+r^2-2ar\sin\theta\cos\phi'}}\\
        &=\frac{\mu_0I}\pi\frac{a}{\sqrt{a^2+r^2+2ar\sin\theta}}\biggfkh{\biggkh{\frac2{k^2}-1}K(k)-\frac2{k^2}E(k)}.%\frac{(2-k^2)K(k)-2E(k)}{k^2}.
    \end{align*}
    其中 
    \[
        k^2:=\frac{4ar\sin\theta}{a^2+r^2+2ar\sin\theta}.
    \]
    $K(k),E(k)$为第一、二类完全椭圆积分
    \begin{alignat*}{2}
        K(k)&=\int_0^1\frac{\d x}{\sqrt{(1-x^2)(1-k^2x^2)}}&,\enspace\abs{k}&<1,\\
        E(k)&=\int_0^1\sqrt{\frac{1-k^2x^2}{1-x^2}}\d x,&\abs{k}&<1.
    \end{alignat*}
    \begin{center}
        \includegraphics[width=0.8\linewidth]{figures/ellipseKE.pdf}
        \captionof{figure}{第一、二类完全椭圆积分}
    \end{center}
    当$r\gg a$或$r\ll a$或$\theta\to 0$时,
    \[
        A_\phi(r,\theta)=\frac{\mu_0I}4\frac{a^2r\sin\theta}{(a^2+r^2)^{3/2}}\biggfkh{1+\frac{15}8\biggkh{\frac{ar\sin\theta}{a^2+r^2}}^2+\cdots},
    \]
    因此对于$r\gg a$处的远场
    \begin{align*}
        B_r&=\frac1{r\sin\theta}\pp\theta(\sin\theta A_\phi)=\frac{\mu_0Ia^2}2\frac{\cos\theta}{r^3},\\
        B_\theta&=-\frac1r\pp r(rA_\phi)=\frac{\mu_0Ia^2}4\frac{\sin\theta}{r^3},\\
        B_\phi&=0.
    \end{align*}
    与电偶极矩的电场\eqref{eqn:E dipole}比较,可定义环形电流的磁偶极矩(magnetic dipole moment)大小
    \[
        m=I\cdot\pi a^2,
    \]
    沿$+z$方向,从而 
    \begin{subequations}
        \begin{align}
            \label{eqn:A dipole}
            \bm A&=\frac{\mu_0m\sin\theta}{4\pi r^2}\uvec\phi;\\
            \label{eqn:B dipole }
            \bm B&=\frac{\mu_0m}{4\pi r^3}(2\cos\theta\uvec r+\sin\theta\uvec\theta).
        \end{align}
    \end{subequations}
    同样,上式磁场的散度$\div\bm B$在原点处的积分不为0,应进行修正:
    \begin{equation}
        \label{eqn:B dipole with delta}
        \bm B(\bm r)=\frac{\mu_0}{4\pi}\biggfkh{\frac{3(\bm m\cdot\uvec r)\uvec r-\bm m}{r^3}+\frac{8\pi}3\bm m\vd(\bm r)}.
    \end{equation}
    可得磁场的旋度$\curl\bm B=\mu_0\bm m\times\nabla\delta(\bm r)$,即磁偶极子的电流密度为
    \begin{equation}
        \bm J(\bm r)=\bm m\times\nabla\delta(\bm r).
    \end{equation}
\end{example}

\paragraph{多极矩展开}

与\secref{sec:multipole expansion}静电多极展开类似,考虑一个局域分布的电流,我们也可以对矢量势做多极矩展开:
\footnote{这个问题可以用矢量球谐函数来做一个完整的处理,将在\chapref{chap:multipole radiation}多极辐射中介绍,这里只满足于最低的近似阶。}
\begin{align}
    \notag
    \bm A(\bm r)&=\frac{\mu_0}{4\pi}\int_V\frac{\bm J(\bm r')}{\abs{\bm r-\bm r'}}\d V'\\
    \notag
    &=\frac{\mu_0}{4\pi}\int_V\bm J(\bm r')\biggkh{\frac1r+\frac{\bm r'\cdot\uvec r}{r^2}+\cdots}\d V'\\
    &=\bm 0+\frac{\mu_0}{4\pi}\frac{\bm m\times\uvec r}{r^2}+\cdots.
\end{align}
由于$\bm J$的体积分为$\bm 0$,磁单极矩(magnetic monopole moment)不存在,因此最低阶非零项为磁偶极矩(magnetic dipole moment) $\bm m$,简称磁矩。经过推导,可以得到磁矩的表达式:
\begin{equation}
    \bm m:=\frac12\int_V\bm r'\times\bm J(\bm r')\d V'.
\end{equation}

\iffalse
下面我们用一些数学上的小tricks:
\[
    \oint_{\p V}fg\bm J\cdot\d\bm A'=\int_V(g\bm J\cdot\nabla'f+f\bm J\cdot\nabla'g+fg\cancel{\nabla'\cdot\bm J})\d V'=0.
\]
将$f,g$赋予特定的形式,可以得到
\begin{alignat*}{2}
    \int_V J_\alpha \d V'&=0,&\qquad (f=1,\enspace g=x_\alpha');\\
    \int_V(x_\alpha'J_\beta+x_\beta'J_\alpha)\d V'&=0,&(f=x_\alpha',\enspace g=x_\beta')\\
    \int_V x_\alpha'J_\alpha\d V'&=0,&(f=g=x_\alpha').
\end{alignat*}
\begin{definition}{全反对称张量}{Levi-Civita symbol}
    引入全反对称张量(Levi-Civita symbol)
    \begin{equation}
        \epsilon_{\alpha\beta\gamma}=\begin{cases}
            1,&(\alpha\beta\gamma)=(123)\\
            -1,&(\alpha\beta\gamma)=(321)\\
            0,&\text{otherwise}
        \end{cases}
    \end{equation}
    一般地定义:$\epsilon_{1\cdots n}=1$,置换任意两个下标则反号,下标相同则为0。
\end{definition}
利用全反对称张量,我们可以将叉乘的结果记作
\begin{equation}
    (\bm a\times\bm b)_\alpha=\sum_{\beta,\gamma}\epsilon_{\alpha\beta\gamma}a_\beta b_\gamma.
\end{equation}
由此可得\eqref{eqn:A approx}中的积分项
\[
    \bm r'\cdot\int_V\bm r'J_\alpha\d V'=-\frac12\biggfkh{\bm r\times\int_V(\bm r'\times\bm J)\d V'}_\alpha.
\]
\fi

\begin{example}
    {电流回路的磁矩}{m of I}
    平面$S$边缘上的电流回路的磁矩
    \begin{align*}
        \bm m=\frac12\oint_{\p S}\bm r\times I\d\bm\ell=IS\uvec n.
    \end{align*}
    与\exmref{exm:B of loop current} 中的定义一致。
\end{example}

\begin{example}
    {带电粒子轨道角动量的磁矩}{m of L}
    带电粒子的磁矩与其轨道角动量$\bm L=m\bm r\times\bm v$有关:
    \[
        \bm m=\frac12q\bm r\times\bm v=\frac q{2m}\bm L.
    \]
\end{example}

\begin{example}
    {磁场的体积分}{volume integral of B}
    与\exmref{exm:volume integral of E} 类似,磁场在球内的体积分
    \[
        \int_{r<R}\bm B(\bm r)\d V=\frac{\mu_0R^2}3\int_V\frac{r_<}{r_>^2}\uvec r'\times\bm J(\bm r')\d V',
    \]
    其中$r_<:=\min(r',R),\,r_>:=\max(r',R)$。
    \tcblower
    考虑两种极端情形:电荷分布全在球内($r'<R$)和全在球外($r'>R$),有
    \begin{equation}
        \label{eqn:int B dV}
        \int_{r<R}\bm B(\bm r)\d V=\begin{cases}
            \frac{2\mu_0}{3}\bm m,&\text{电流全在球内};\\[1ex]
            -\frac{4\pi}{3}R^3\bm B(\bm 0),&\text{电流全在球外}.
        \end{cases}
    \end{equation}
    这也对应于磁偶极子磁场\eqref{eqn:B dipole with delta}中的$2\mu_0\bm m\vd(\bm r)/3$项。
\end{example}

\subsection{磁矩的力、力矩和能量}
\label{ssec:force, torque and energy of magnetic moment}

外部磁感应强度$\bm B(\bm r)$中电流密度$\bm J(\bm r)$受到的总力
\[
    \bm F=\int_V\bm J(\bm r)\times\bm B(\bm r)\d V\simeq(\bm m\times\nabla)\times\bm B=\nabla(\bm m\cdot\bm B)-\bm m\cancel{(\div\bm B)},
\]
即
\footnote{此式包含了隐藏动量(hidden momentum) $\d(\bm E\times\bm m)/c^2\d t$的效应。}
\begin{equation}
    \label{eqn:F=nabla(m dot B)}
    \bm F=\nabla(\bm m\cdot\bm B),
\end{equation}
% 注意其与$\bm F=(\bm p\cdot\nabla)\bm E$的区别。特别地,对于稳衡磁场,也可以写成$(\bm m\cdot\nabla)\bm B$,
% \[
%     \nabla(\bm m\cdot\bm B)=\bm m\times\cancel{(\curl\bm B)}+\bm B\times\cancel{(\curl\bm m)}+(\bm m\cdot\nabla)\bm B+(\bm B\cdot\cancel{\nabla)\bm m}.
% \]
此时,可由$\bm F=-\nabla U$定义永久磁矩$\bm m$的势能为
\begin{equation}
    U=-\bm m\cdot\bm B,
\end{equation}
\begin{remark}
    $U$并不是外磁场$\bm B$中磁矩$\bm m$的总能量,因为它不包含产生磁矩$\bm m$并维持$\bm m$恒定所需的能量。
\end{remark}
可见,当磁矩$\bm m$和$\bm B$方向相同时,$U$最低,因此磁矩倾向于偏转自己平行于磁场。
力矩与作用在磁偶极子上的力矩表达式相同:
\begin{equation}
    \bm N=\int_V\bm r'\times(\bm J\times\bm B)\d V'\simeq\bm m\times\bm B,
\end{equation}
上式也可作为定义磁感应强度的一种方法。

\section{宏观介质静磁学}
\label{sec:macroscopic magnetostatics}

在物质中,由原子电子贡献的有效(effective)原子电流和本征(intrinsic)磁矩产生的偶极场在原子尺度上变化非常明显。因此在宏观观测时,我们也需要对微观取空间平均\footnote{不需要对时间平均。}:
\begin{equation}
    \label{eqn:divB-macro}
    \div\bm B_\text{micro}=0\implies\div\bm B=0.
\end{equation}

\begin{definition}
    {磁化强度}{magnetization}
    定义磁矩密度(magnetic moment density)或磁化强度(magnetization)为单位体积的磁矩:
    \begin{equation}
        \bm M:=\sum_iN_i\ave{\bm m_i}.
    \end{equation}
\end{definition}
可得矢量势
\begin{equation}
    \bm A(\bm r)=\frac{\mu_0}{4\pi}\int_V\frac{\bm J(\bm r')+\nabla'\times\bm M(\bm r')}{\abs{\bm r-\bm r'}}\d V'.
\end{equation}
因此 
\[
    \curl\bm B=\mu_0(\bm J+\curl\bm M),
\]
其中$\bm J_\mag:=\curl\bm M$是有效电流密度。
定义磁场强度(magnetic field)
\begin{equation}
    \bm H:=\frac{\bm B}{\mu_0}-\bm M.
\end{equation}
从而 
\begin{equation}
    \label{eqn:curlH}
    \curl\bm H=\bm J.
\end{equation}
式\eqref{eqn:divB-macro}和\eqref{eqn:curlH}考虑了有效原子电流的贡献,是\eqref{eqn:divB}和\eqref{eqn:curlB}的宏观对应。
\begin{definition}
    {磁导率}{permeability}
    对于各向同性抗磁性(diamagnetic)和顺磁性(paramagnetic)物质,$\bm H$和$\bm B$有简单的线性关系:
    \begin{equation}
        \label{eqn:B=muH}
        \bm B=\mu\bm H,
    \end{equation}
    其比例系数$\mu$称为磁导率(permeability)。
    
    根据磁导率可以将物质分为以下磁性:
    \begin{itemize}
        \item 顺磁性(paramagnetism):$\mu\gtrapprox\mu_0$,如碱金属(Li, Na, Cs)、碱土金属(Mg, Ca);
        \item 抗磁性(diamagnetism):$\mu\lessapprox\mu_0$,如H$_2$O、Cu、C (石墨)、Pb等;
        \item 铁磁性(ferromagnetism):$\mu\gg\mu_0$,且$\mu$与$\bm H$的关系较为复杂,如\figref{fig:hysteresis},称为磁滞现象(hysteresis)。如Fe, Co, Ni。%由于其,有时可对边界条件作出简化假设。
    \end{itemize}
    \begin{center}
        \includegraphics[width=0.7\linewidth]{figures/hysteresis.pdf}
        \captionof{figure}{磁滞回线}
        \label{fig:hysteresis}
    \end{center}
\end{definition}

\paragraph{表面分布}

考虑介质界面上$\bm B$和$\bm H$的分布,$\uvec n$是边界上从区域(1)指向区域(2)的法向量,我们可以得到:
\begin{subequations}
    \begin{align}
        \label{eqn:Bn2-Bn1}
        (\bm B_2-\bm B_1)\cdot\uvec n_{21}&=0,\\
        \label{eqn:Ht2-Ht1}
        (\bm H_2-\bm H_1)\times\uvec n_{21}&=-\bm K,
    \end{align}
\end{subequations}
其中$\bm K$是理想化的表面电流密度(surface current density)。
% 对于满足线性关系\eqref{eqn:B=muH}的各向同性介质,可以写成
% \begin{subequations}
%     \begin{align}
%         \bm B_1\cdot\uvec n_{21}&=\bm B_2\cdot\uvec n_{21},\\
%         \frac1{\mu_1}(\bm B_2\times\uvec n_{21})&=\frac1{\mu_2}(\bm B_1\times\uvec n_{21})+\bm K.
%     \end{align}
% \end{subequations}

当$\mu_1\gg\mu_2$时,$H_{2,\perp}\gg H_{1,\perp}$,磁场$\bm H_2$与边界面垂直。所以,极高磁导率材料表面上$\bm H$的边界条件与导体表面上$\bm E$的边界条件相同:
\begin{center}
	\includegraphics[page=\pagenumref{tikz:H between magnetic surfaces}]{figures/tikz/layouts.pdf}
	\captionof{figure}{极高磁导率材料表面的$\bm H$表现}
    \label{fig:H between magnetic surfaces}
\end{center}
因此,我们可以把静电势理论用于磁场上,高磁导率材料的表面是近似的等势面。这种类比在许多磁铁设计问题中用到:先决定场的形式,然后把极面设计成等势面的形状。
\begin{example}{}{}
    对于真空中的极高磁导率材料,若其表面无电流,则
    \[
        H_\parallel=H_{0,\parallel},\quad B_\perp=B_{0,\perp},
    \]
    因此材料内磁场
    \[
        B^2=B_{0,\perp}^2+\frac{\mu^2}{\mu_0^2}B_{0,\parallel}^2.
    \]
    两边同除$\mu$便得到了能量密度的形式,由于材料中储存的能量是有限的,故$B_{0,\parallel}=0$
\end{example}

\subsection{静磁学中的边界条件问题}
\label{ssec:boundary-value problems in magnetostatics}

\paragraph{磁势}
%对于磁化材料外部的磁场问题,
若材料是磁导率$\mu\gg\mu_0$的硬铁磁体(其磁矩密度$\bm M$与外场无关),则材料外表面上的$\bm B$垂直于界面,内表面上的$\bm B$平行于界面。如果其内没有电流($\bm J=\bm0$),则可引入磁势$\Phi_\mag$并解其Laplace方程。
\begin{equation}
    \bm H=-\nabla\bm\Phi_\mag,\quad\lapla\Phi_\mag=-\rho_\mag=\div\bm M.
\end{equation}
若无边界条件,则有解
\begin{equation}
    \Phi_\mag(\bm r)=-\frac1{4\pi}\int_V\frac{\nabla'\cdot\bm M(\bm r')}{\abs{\bm r-\bm r'}}\d V',
\end{equation}
若$\bm M$是局域的,可通过分部积分得到
\begin{equation}
    \Phi_\mag(\bm r)=-\frac1{4\pi}\div\int_V\frac{\bm M(\bm r')}{\abs{\bm r-\bm r'}}\d V',
\end{equation}
在$r\gg r'$的地方
\[
    \Phi_\mag(\bm r)\simeq\frac{\bm m\cdot\uvec r}{4\pi r^2},\quad\bm m=\int_V\bm M\d V.
\]
对于有边界条件的情形,在边界$\p V$处$\bm M$瞬间变为0,运用静电学中的结论:
\begin{equation}
    \label{eqn:PhiM with boundary}
    \Phi_\mag(\bm r)=-\frac1{4\pi}\int_V\frac{\nabla'\cdot\bm M(\bm r')}{\abs{\bm r-\bm r'}}\d V'+\frac1{4\pi}\oint_{\p V}\frac{\bm M(\bm r')\cdot\uvec n'}{\abs{\bm r-\bm r'}}\d A',
\end{equation}
若铁磁体是均匀磁化的,则$\div\bm M=0$,第一项为0;
第二项为有效面磁荷密度$\sigma_\mag=\bm M\cdot\uvec n$的贡献。

\paragraph{矢量势}
我们接着求矢量势,由于$\bm J=\bm 0$,仅剩有效电流密度项
\begin{equation}
    \lapla\bm A=-\mu_0(\bm 0+\curl\bm M),
\end{equation}
可得
\begin{equation}
    \label{eqn:A with boundary}
    \bm A(\bm r)=\frac1{4\pi}\int_V\frac{\nabla'\times\bm M(\bm r')}{\abs{\bm r-\bm r'}}\d V'+\frac1{4\pi}\oint_{\p V}\frac{\bm M(\bm r')\times\uvec n'}{\abs{\bm r-\bm r'}}\d A',
\end{equation}
上式两项分别对应有效电流密度$\bm J_\mag$和有效面电流密度$\bm K$的贡献
\begin{subequations}
    \begin{align}
        \bm J_\mag&=\curl\bm M;\\
        \bm K&=\bm M\times\uvec n.
    \end{align}
\end{subequations}
\begin{example}{均匀磁化球}{uniform magnetized sphere}
    真空中半径为$R$的永磁体球,其磁化强度$\bm M=M_0\uvec z$。应用式\eqref{eqn:PhiM with boundary}
    \[
        \Phi_\mag(r,\theta)=0+\frac1{4\pi}\oint_{r'=R}\frac{M_0\cos\theta'}{\abs{\bm r-\bm r'}}R^2\d\Omega',
    \]
    由于
    \[
        \cos\theta'=\sqrt{\frac{4\pi}3}Y_{10}(\theta',\phi'),
    \]
    以及球谐函数的正交性,$1/\abs{\bm r-\bm r'}$的展开式\eqref{eqn:1/|r-r'|=YY}仅有$Y_{10}$项保留,故
    \[
        \Phi_\mag(r,\theta)=\frac13M_0R^2\cdot\frac{r_<}{r_>}\cos\theta.
    \]
    %$r_<,r_>$是$r,R$中的较小/大者。

    球内,$r_<=r,\enspace r_>=R$,表现为匀强磁场:
    \[
        \Phi_\mag=\frac13M_0r\cos\theta,\quad \bm H_\internal=-\frac13\bm M,\enspace\bm B_\internal=\frac{2\mu_0}3\bm M;
    \]
    球外,$r_<=R,\enspace r_>=r$,表现为磁偶极子(没有更高阶的项):
    \[
        \Phi_\mag=\frac{M_0R^3}3\frac{\cos\theta}{r^2}=\frac{\bm m\cdot\uvec r}{4\pi r^2},\quad\bm m=\frac{4\pi}3R^3\bm M.
    \]
    $\bm B$线是连续的闭合曲线,而$\bm H$线不是。
    \tcblower
    也可通过矢量势的方法:
    \[
        A_\phi(r,\theta)=\frac{\mu_0}3M_0R^2\cdot\frac{r_<}{r_>^2}\sin\theta.
    \]
\end{example}
\begin{example}{外场下的磁化球}{magnetized sphere in external field}
    续\exmref{exm:uniform magnetized sphere},在全空间叠加一匀强场$\bm B_0=\mu_0\bm H_0$,便成为外场下的磁化球问题,球内场强为
    \[
        \bm B_\internal=\bm B_0+\frac{2\mu_0}3\bm M,\quad\bm H_\internal=\bm H_0-\frac13\bm M,
    \]
    如果球是顺磁或抗磁的,那么$\bm M$便是由外场引起的,且$\bm B_\internal=\mu\bm H_\internal$,从而得到与电介质\eqref{eqn:P-E}很相似的结果:
    \begin{equation}
        \label{eqn:M-B}
        \bm M=\frac3{\mu_0}\biggkh{\frac{\mu_\r-1}{\mu_\r+2}}\bm B_0.
    \end{equation}
    如果球是铁磁的,那么便可以得到$\bm B_\internal$和$\bm H_\internal$的限制关系:
    \[
        \bm B_\internal+2\mu_0\bm H_\internal=3\bm B_0,
    \]
    上式可以在磁滞图上画一条直线,并会产生交点。
\end{example}
\begin{example}{外场下的磁化球壳}{magnetized sphere shell in external field}
    内外半径为$a,b$的球壳磁导率为$\mu$,放置在匀强磁场$\bm B_0$中,由于没有电流可应用磁势求解
    \[
        \lapla\Phi_\mag=0
    \]
    由式\eqref{eqn:Phi(r, theta)},再加上$r\to\infty$时$\bm H=\bm H_0$的条件,可写出磁势的形式解,再通过边界条件
    \[
        \cdots
    \]
    \iffalse
    \begin{align*}
        \begin{cases}
            \edg{\pv{\Phi_\text{M,I}}\theta}_{r=a}=\edg{\pv{\Phi_\text{M,II}}\theta}_{r=a}\\[2ex]
            \edg{\pv{\Phi_\text{M,II}}\theta}_{r=b}=\edg{\pv{\Phi_\text{M,III}}\theta}_{r=b}
        \end{cases}
        \begin{cases}
            \mu_0\edg{\pv{\Phi_\text{M,I}}r}_{r=a}=\mu\edg{\pv{\Phi_\text{M,II}}r}_{r=a}\\[2ex]
            \mu_0\edg{\pv{\Phi_\text{M,II}}r}_{r=b}=\mu\edg{\pv{\Phi_\text{M,III}}r}_{r=b}
        \end{cases}
    \end{align*}
    \fi
    可得系数仅剩$\ell=1$项,壳内是一个匀强场:
    \[
        \Phi_\mag(r,\theta)=
        \begin{cases}
            a_1r\cos\theta,&r<a\\
            \ldots,&a<r<b\\
            -H_0r\cos\theta+\frac{d_1}{r^2}\cos\theta,&r>b
        \end{cases}
    \]
    当$\mu\gg\mu_0$时,壳内场强趋于0,这便是磁屏蔽效应。
\end{example}
最后探究了在一侧具有渐近均匀切向磁场的完全导电平面上的圆孔的影响,略。

\section{Faraday电磁感应定律}
\label{sec:Faraday}

我们在前几章里论述了静电学和静磁学中的稳恒态问题。我们虽然用了相似的数学方法,但把电现象和磁现象当作相互独立的现象来处理。两者之间唯一的联系在于如下事实:\textit{磁场是由电流产生的,而电流就是运动的电荷。}
但当我们考虑与时间有关的问题时,电现象和磁现象的几乎独立的性质便会消失:随时间变化的磁场会产生电场,反之亦然。%这时我们必须说电磁场,而不能说电场或磁场.只有在狭

Faraday在探究时变磁场中电流表现的实验表明:
\begin{compactenum}
    \item 接通/断开临近电路的稳恒电流;
    \item 临近电路的稳恒电流相对于本电路移动;
    \item 将一根永磁体插入/抽出电路
\end{compactenum}
都会在电路中产生瞬态(transient)电流。Faraday把瞬态电流的产生解释为电路环绕磁通量(magnetic flux)%\footnote{本笔记中,$\Phi$已经被标量势占用了,$\phi$被方位角占用了,$\psi$和$\varphi$似乎还没人占用。}
\begin{equation}
    \varPhi:=\int_S\bm B\cdot\d\bm A
\end{equation}
的变化所引起的。变化的磁通量会再电路周围感感生(induce)一个电场驱动电路电荷运动形成电流,其电动势(electromotive force, emf)为
\begin{equation}
    \emf:=\oint_{\p S}\bm E'\cdot\d\bm \ell.
\end{equation}
$\bm E'$是在运动坐标系或介质中的电场。则Faraday的观测结果可总结为:
\begin{equation}
    \label{eqn:emf propto varPhi'}
    \emf\propto-\dv\varPhi t.
    \tag{$\ast$}
\end{equation}
其符号由Lenz定律决定:感生电流磁通量总是对抗磁通量的变化。

\paragraph{Galileo变换}

下面我们希望得到式\eqref{eqn:emf propto varPhi'}中的比例系数$k$,%事实上此系数已经可以从已有的单位制推出,而不像$\varepsilon_0$等需要引入新的单位。
我们将从已有单位制推导(而非定义)该比例系数。

在狭义相对论(special relativity)发现之前(或处理低速情况时),所有物理学家都公认(虽然不经常明确说出)物理定律在Galileo变换(Galilean transformation):
\begin{align*}
    (t,\bm r)&\mapsto(t,\bm r+\bm vt),
\end{align*}
下应该是不变的。$\bm v$是两个参考系之间的相对速度。也就是说,两个以恒定速度$\bm v$作相对运动的观察者看到的物理现象是一样的。%,只要这两组空间坐标和时间坐标的关系遵从Galileo变换:

Faraday从实验上证实电磁感应现象在Galileo变换下不变:无论是载电流的初级电路静止、次级电路运动;或者是次级电路静止、而载电流的初级电路作同样的相对运动,在次级电路中感生的电流是相同的。

现在讨论一个运动电路的法拉第定律,如果电路以恒定速度$\bm v$移动,则磁场的迁移导数为
\begin{align*}
    \dv{\bm B}t&=\pv{\bm B}t+(\bm v\cdot\nabla)\bm B\\
    &=\pv{\bm B}t+\curl(\bm B\times\bm v)+(\bm B\cdot\cancel{\nabla)\bm v}-\bm B(\cancel{\div\bm v})+\bm v(\cancel{\div\bm B}),
\end{align*}
即,
\begin{align*}
    \dd t\int_S\bm B\cdot\d\bm A=\int_S\pv{\bm B}t\cdot\d\bm A+\oint_{\p S}(\bm B\times\bm v)\cdot\d\bm\ell,
\end{align*}
因此 
\[
    \oint_{\p V}\bigkh{\bm E'-k\bm v\times\bm B}\d\bm\ell=-k\int_S\pv{\bm B}t\cdot\d\bm A.
\]
由Galileo变换下电场力和Lorentz力的形式,应有:
\[
    \bm E'=\bm E+\bm v\times\bm B,
\]
即$k=1$,因此Faraday定律可以写成等式:
\begin{equation}
    \label{eqn:Faraday}
    \emf=-\dv\varPhi t,
\end{equation}
并可由Stokes公式转化为微分形式
\begin{equation}
    \label{eqn:curlE-B}
    \curl\bm E=-\pv{\bm B}t.
\end{equation}
而%我们知道
静电场的旋度\eqref{eqn:curlE}为0!由此可见,时变场将电场和磁场联系了起来。

\section{磁场的能量}
\label{sec:energy in the magnetic field}

我们在前面讨论静磁场时,避开了场能和能量密度的问题。
因为
要建立电流的稳恒组态及有关的磁场,必定经过一段初始的瞬变期,在这期间,电流和磁场从零增到终值。存在这种随时间变化的场,就存在使电流源做功的感生电动势。因为按照定义,场的能量等于建立这场时所做的总功,所以我们必须考虑这些贡献。

保持电流恒定不变,
\[
    \vd W=\int_V\vd\bm A\cdot\bm J\d V=\int_V\bm H\cdot\vd\bm B\d V.
\]
对于顺磁性和抗磁性物质,$2\bm H\cdot\vd\bm B=\vd(\bm H\cdot\bm B)$
\[
    W=\frac12\int_V\bm A\cdot\bm J\d V=\frac12\int_V\bm H\cdot\bm B\d V.
\]

将一个磁导率为$\mu$的物体放进电流源固定不动的磁场$\bm B$中,前后能量发生变化
\[
    \D W=\frac12\int_V\bm M\cdot\bm B_0\d V.
\]

\subsectionstar{自感和互感}
\label{ssec:self- and mutual inductance}

对于$N$个不导磁的载流电路系统,总能量可以描述为:
\[
    W=\frac12\sum_{i=1}^NL_iI_i^2+\frac12\sum_{i\neq j}M_{ij}I_iI_j.
\]
$L_i$是线圈$i$的自感(self-inductance),$M_{ij}$是线圈$i,j$之间的互感(mutual inductance),可以证明 
\[
    M_{ij}=M_{ji}.
\]

自感和互感可由
\[
    W=\frac12\int_V\bm J\cdot\bm A\d V.
\]
求出,用磁通量的形式即
\[
    L_i=\frac{\varPhi_{ii}}{I_i},\quad M_{ij}=\frac{\varPhi_{ij}}{I_j}.
\]
\begin{example}
    {同轴线缆的电感}{inductance of coaxial cable}
    内外半径为$r,R$的同轴线缆单位长度的电感为
    \begin{equation}
        L=\frac{\mu_0}{2\pi}\ln\biggkh{\frac Rr}.
    \end{equation}
\end{example}
若电路是导磁的,则$W$的表达式不适用了,需要换为
\[
    W=\frac12\int_V\bm H\cdot\bm B\d V.
\]
后面还讨论了圆导线的自电感。

\sectionstar{似稳场}
\label{sec:quasi-static field}

似稳场(quasi-static field)、涡流(eddy current)、磁扩散(magnetic diffusion)

