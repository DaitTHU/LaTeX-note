\chapter{宏观介质静电学}
\label{chap:electrostatics of macroscopic media}

本章内容分为两节:
\begin{compactenum}
    \item \secref{sec:multipole expansion}介绍给定电荷分布系统的电势及其多极展开(multipole expansion),%及处在外场中的电荷系统能量的多极展开。
    多极展开本质是利用逐项逼近的方法来计算电势或能量的,故在系统电荷的空间分布尤为复杂的情况下,使用多极展开去近似计算是一种行之有效的方法。
    
    和某些教材上使用并矢张量的标记法不同,本文使用的是球谐函数表示(通过球谐函数导出也更为自然和简洁)。

    \item \secref{sec:dielectric}介绍了电介质(dielectrics)及含电介质的边值问题。并扼要介绍了分子极化率(atomic polarizability)和宏观电极化率 (susceptibility)的主要特征及彼此之间的关系,导出了 Clausius-Mossotti 方程。最后介绍了含电介质时系统的能量表示。
\end{compactenum}

\section{多极矩展开}
\label{sec:multipole expansion}

考虑一个局域(localized)分布的电荷系统$\rho(\bm r')$在空间某点$\bm r$处激发的电势$\Phi(\bm r)$,所谓局域,是指电荷仅分布在有限体积$V$内,而我们感兴趣的点$\bm r$位于该体积之外(往往$r\gg r'$)。由$\Phi(\bm r)$的表达式\eqref{eqn:Phi-rho}和$1/\abs{\bm r-\bm r'}$的展开式\eqref{eqn:1/|r-r'|=YY},代入$r_<=r',\,r_>=r$,有
\begin{align}
    \notag
    \Phi(\bm r)&=\frac1{4\pi\varepsilon_0}\int_V\frac{\rho(\bm r')}{\abs{\bm r'-\bm r}}\d V'\\
    \notag
    &=\frac1{\varepsilon_0}\sum_{\ell=0}^\infty\sum_{m=-\ell}^\ell\frac1{2\ell+1}\biggfkh{\int_V\rho(\bm r'){r'^\ell}Y^\ast_{\ell m}(\theta',\phi')\d V'}\frac{Y_{\ell m}(\theta,\phi)}{r^{\ell+1}}\\
    \label{eqn:Phi multipole}
    &=:\frac1{\varepsilon_0}\sum_{\ell=0}^\infty\sum_{m=-\ell}^\ell\frac{q_{\ell m}}{2\ell+1}\frac{Y_{\ell m}(\theta,\phi)}{r^{\ell+1}}.
\end{align}
其中系数$q_{\ell m}$称为多极矩(multipole moment)
\begin{equation}
    q_{\ell m}:=\int_V\rho(\bm r'){r'^\ell}Y^\ast_{\ell m}(\theta',\phi')\d V'.
\end{equation}
另一方面,也可应用式\eqref{eqn:1/|r-r'|}展开$1/\abs{\bm r-\bm r'}$:
\begin{align}
    \notag
    \Phi(\bm r)&=\frac1{4\pi\varepsilon_0}\int_V\rho(\bm r')\sum_{\ell=0}^\infty\frac{r'^\ell}{r^{\ell+1}}P_\ell(\cos\gamma)\d V'\\
    \notag
    &=\frac1{4\pi\varepsilon_0}\biggfkh{\frac1r\int_V\rho\d V'+\frac1{r^2}\int_V\rho r'\cos\gamma\d V'+\frac1{r^3}\int_V\rho r'^2P_2(\cos\gamma)\d V'+\cdots}\\
    \label{eqn:legendre multipole}
    &=\frac1{4\pi\varepsilon_0}\biggfkh{\frac qr+\frac{\bm p\cdot\uvec r}{r^2}+\frac1{2r^3}\sum_{\alpha,\beta}Q_{\alpha\beta}\hat r_\alpha\hat r_\beta+\cdots}.
\end{align}
前三个积分项分别对应总电荷即电单极矩(electric monopole moment) $q$、电偶极矩(dipole) $\bm p$、电四极矩(quadrupole) $Q_{\alpha\beta}$:
\begin{subequations}
    \begin{align}
        q={}&\int_V\rho(\bm r')\d V',\\
        \bm p:={}&\int_V\bm r'\rho(\bm r')\d V',\\
        Q_{\alpha\beta}:={}&\int_V(3r_\alpha'r_\beta'-r'^2\vd_{\alpha\beta})\rho(\bm r')\d V'.
    \end{align}
\end{subequations}

% 上式也可通过对$1/\abs{\bm r-\bm r'}$在$\bm r'=0$处Taylor展开推导得到。

\begin{example}{前几个多极矩}{first few multipole moment}
    易得
    \begin{subequations}
        \begin{align}
            q_{00}&=\frac1{\sqrt{4\pi}}\int_V\rho(\bm r')\d V'=\frac1{\sqrt{4\pi}}q;\\
            q_{10}&=-\sqrt{\frac3{8\pi}}\int_V(x'-\i y')\rho(\bm r')\d V'=-\sqrt{\frac3{8\pi}}(p_x-\i p_y),\\
            q_{11}&=-\sqrt{\frac3{4\pi}}\int_V z'\rho(\bm r')\d V'=-\sqrt{\frac3{4\pi}}p_z;\\
            q_{20}&=\sqrt{\frac5{16\pi}}\int_V(3z'^2-r'^2)\rho(\bm r')\d V'=\sqrt{\frac5{16\pi}}Q_{zz},\\
            q_{21}&=-\sqrt{\frac{15}{8\pi}}\int_V(x'-\i y')z'\rho(\bm r')\d V'=-\sqrt{\frac{15}{8\pi}}(Q_{xz}-\i Q_{yz}),\\
            q_{22}&=\sqrt{\frac{15}{32\pi}}\int_V(x'-\i y')^2\rho(\bm r')\d V'=\sqrt{\frac{15}{32\pi}}(Q_{xx}-Q_{yy}-2\i Q_{xy}).
        \end{align}
    \end{subequations}
    负$m$项可由$q_{\ell,-m}=(-)^mq_{\ell m}^\ast$推出,
    \tcblower
    对于任何电荷分布,其最低阶的不为0的多极矩$q_{\ell m}$与原点的选择无关,但剩下所有较高阶的多极矩一般都与原点的位置有关。
\end{example}
\iffalse
由式\eqref{eqn:Phi multipole},电场分量可被表示为
\begin{align*}
    E_r&=-\pv\Phi r=\frac1{\varepsilon_0}\frac{\ell+1}{2\ell+1}q_{\ell m}\frac{Y_{\ell m}(\theta,\phi)}{r^{\ell+2}},\\
    E_\theta&=-\frac1r\pv\Phi\theta=-\frac1{\varepsilon_0}\frac{1}{2\ell+1}q_{\ell m}\frac1{r^{\ell+2}}\pv{Y_{\ell m}(\theta,\phi)}\theta,\\
    E_\phi&=-\frac1{r\sin\theta}\pv\Phi\phi=-\i\frac1{\varepsilon_0}\frac{1}{2\ell+1}q_{\ell m}\frac1{r^{\ell+2}}\frac{m}{\sin\theta}{Y_{\ell m}(\theta,\phi)},
\end{align*}
\fi
\begin{example}{等量异号电荷对的电场}{E of pm q}
    考虑位于$z$轴坐标分别为$\pm a$的等量异号电荷对$\pm q$,其在空间激发的电势为
    \begin{align*}
        \Phi&=\frac q{4\pi\varepsilon_0}\biggfkh{\frac1{\sqrt{a^2+r^2-2ar\cos\theta}}-\frac1{\sqrt{a^2+r^2+2ar\cos\theta}}}\\
        &=\frac q{4\pi\varepsilon_0r}\sum_{\ell=0}^\infty\Bigkh{\frac ar}^\ell\bigfkh{P_{\ell}(\cos\theta)-P_\ell(-\cos\theta)}\\
        &=\frac q{4\pi\varepsilon_0r}\biggfkh{\frac{2a}r\cos\theta+\Bigkh{\frac ar}^3(5\cos^3\theta-3\cos\theta)+\cdots}
    \end{align*}
    对于$r\gg a$的远场,仅保留第一项,与理想电偶极矩$p:=q\cdot 2a$的电势相同:
    \begin{subequations}
        \begin{align}
            \label{eqn:Phi dipole}
            \Phi&=\frac{p\cos\theta}{4\pi\varepsilon_0r^2},\\
            \label{eqn:E dipole}
            \bm E&=\frac p{4\pi\varepsilon_0r^3}(2\cos\theta\uvec r+\sin\theta\uvec\theta).
        \end{align}
    \end{subequations}
    但上式电场的旋度$\curl\bm E$在原点处的积分不为$\bm 0$,应进行修正:
    \begin{equation}
        \label{eqn:E dipole with delta}
        \bm E(\bm r)=\frac1{4\pi\varepsilon_0}\biggfkh{\frac{3(\bm p\cdot\uvec r)\uvec r-\bm p}{r^3}-\frac{4\pi}3\bm p\vd(\bm r)}.
    \end{equation}
    此时$\div\bm E=\bm p\cdot\nabla\delta(\bm r)/\varepsilon_0$,即电偶极子的电荷密度为
    \begin{equation}
        \rho(\bm r)=-\bm p\cdot\nabla\delta(\bm r).
    \end{equation}
\end{example}

\begin{example}
    {电场的体积分}{volume integral of E}
    对于局域分布的电荷系统,其电场在以原点为球心、半径为$R$的球内的体积分为
    \begin{align*}
        \int_{r<R}\bm E(\bm r)\d V&=-\int_{r<R}\nabla\Phi(\bm r)\d V=-\int_{r=R}\Phi(\bm r)\d\bm A=-\int_{r=R}\Phi(\bm r)\uvec rR^2\d\Omega\\
        &=-\frac{R^2}{4\pi\varepsilon_0}\int_{r=R}\int_V\frac{\rho(\bm r')}{\abs{\bm r-\bm r'}}\d V'\uvec r\d\Omega,
    \end{align*}
    其中球面上的单位矢量可以写成$Y_{1m}(\theta,\phi)$的线性组合:
    \[
        \uvec r=\sin\theta\cos\phi\uvec x+\sin\theta\sin\phi\uvec y+\cos\theta\uvec z,
    \]
    而$1/\abs{\bm r-\bm r'}$可由式\eqref{eqn:1/|r-r'|=YY}展开,由球谐函数的正交性,级数将只剩$\ell=1$阶项,即
    \[
        \int_{r<R}\bm E(\bm r)\d V=-\frac{R^2}{3\varepsilon_0}\int_V\frac{r_<}{r_>^2}\rho(\bm r')\uvec r'\d V'.
    \]
    其中$r_<:=\min(r',R),\,r_>:=\max(r',R)$。
    \tcblower
    考虑两种极端情形:电荷分布全在球内($r'<R$)和全在球外($r'>R$),有
    \begin{equation}
        \label{eqn:int E dV}
        \int_{r<R}\bm E(\bm r)\d V=\begin{cases}
            -\frac1{3\varepsilon_0}\bm p,&\text{电荷全在球内};\\[2ex]
            -\frac{4\pi}{3}R^3\bm E(\bm 0),&\text{电荷全在球外}.
        \end{cases}
    \end{equation}
    这也对应于电偶极子电场\eqref{eqn:E dipole with delta}中的$-\bm p\vd(\bm r)/3\varepsilon_0$项。
\end{example}

\subsection{能量的多极矩展开}
\label{ssec:multipole expansion of energy}

对$\Phi(\bm r)$在原点处Taylor展开
\[
    \Phi(\bm r)=\Phi(\bm 0)-\bm r\cdot\bm E(\bm 0)-\frac12\sum_{\alpha,\beta}x_\alpha x_\beta\pv{E_\beta}{x_\alpha}(\bm 0)+\cdots
\]
由于$\div\bm E=0$,上式可减去$r^2\div\bm E(\bm 0)/6$凑成$Q_{\alpha\beta}$的形式,
进而由\eqref{eqn:W-rhoPhi}得到静电能
\begin{equation}
    W=q\Phi(\bm 0)-\bm p\cdot\bm E(\bm 0)-\frac16\sum_{\alpha,\beta}Q_{\alpha\beta}\pv{E_\beta}{x_\alpha}(\bm 0)+\cdots
\end{equation}
上式体现了多极子和外场的相互作用:电荷与电势、电偶极子与电场、电四极子与电场梯度等。

\begin{example}
    {两个电偶极子之间的互能}{}
    \begin{equation}
        W=\frac{\bm p_1\cdot\bm p_2-3(\uvec n\cdot\bm p_1)(\uvec n\cdot\bm p_2)}{4\pi\varepsilon_0\abs{\bm r_1-\bm r_2}^3},\quad\uvec n:=\frac{\bm r_1-\bm r_2}{\abs{\bm r_1-\bm r_2}}.
    \end{equation}
\end{example}

\paragraph{电偶极子的能量、力矩和力}

对$\bm E$展开至一阶项,得到电偶极子在电场中受到的力为
\begin{equation}
    \label{eqn:F=(p dot nabla)E}
    \bm F=\int_V\rho\bm E\d V\simeq(\bm p\cdot\nabla)\bm E,
\end{equation}
一般地,$\bm p$不随坐标变化,上式也可写成$\nabla(\bm p\cdot\bm E)$: 
\[
    \nabla(\bm p\cdot\bm E)=\bm p\times\cancel{(\curl\bm E)}+\bm E\times\cancel{(\curl\bm p)}+(\bm p\cdot\nabla)\bm E+(\bm E\cdot\cancel{\nabla)\bm p}.
\]
由$\bm F=-\nabla U$,可定义电偶极子在电场中的势能为
\begin{equation}
    U=-\bm p\cdot\bm E.
\end{equation}
利用\exmref{exm:E of pm q} 中的模型,可以很直观地推出电偶极子在电场中受到的力矩为
\begin{equation}
    \bm N=\bm p\times\bm E,
\end{equation}

\section{电介质}
\label{sec:dielectric}

电介质(dielectric)是能够被电极化的绝缘体。与导体的自由电荷不同,电介质的带电粒子被原子、分子的内力或分子间的力紧密束缚着,称为束缚电荷(bound charge)。在外电场作用下,这些电荷也只能在微观范围内移动,产生极化(polarization)。
极化主要有三种机制:
\begin{compactenum}
    \item 畸变极化:原子核外的电子云分布产生畸变,从而产生不等于零的电偶极矩;
    \item 位移极化:分子原来正负电中心重合,在外电场作用下正负电中心分离产生电偶极矩;
    \item 转向极化:分子有固有电偶极矩,但取向混乱,宏观上电偶极矩总和等于零。在外电场作用下,各个电偶极子趋向于一致的排列,从而宏观电偶极矩不等于零。
\end{compactenum}
在静电场中,电介质内部可以存在电场,这是电介质与导体的基本区别。
\paragraph{宏观电场}
在前一章中,我们没有考虑有质介质(ponderable media)的存在,所以我们没有区分微观场(microscopic)和宏观场(macroscopic)。
但是静电学主要涉及有质介质中的电荷和场,因此%它们各自的电响应必须考虑在内:(1)分子电荷密度会被扭曲。(2)每个分子的多极矩都是不同的。
我们必须在微观层面上考虑有质介质的极化效应(polarization effect)。为了得到适用于宏观现象的Maxwell方程,我们对宏观小、微观大的区域求平均:% 在第6章中,我们只会小心翼翼地做到这一点
\[
    \bm E:=\ave{\bm E_\text{micro}},
\]
则旋量方程\eqref{eqn:curlE}对求平均后的宏观电场$\bm E$适用:
\begin{equation}
    \label{eqn:curlE-macro}
    \curl\bm E_\text{micro}=\bm 0\implies\curl\bm E=\bm 0.
\end{equation}
这意味着宏观电场$\bm E$同样可以表达成电势的梯度$-\nabla\Phi$。

\paragraph{电极化}

\newcommand{\pol}{\mathrm{p}}

%如果电场作用于由大量原子(或分子)组成的介质,束缚在每个分子中的电荷将执行摄动运动(perturbed motions),由此在介质中产生了电极化。
在电介质内的多极子被外电场极化的过程中,偶极子占主导地位。
\begin{definition}
    {电极化强度}{electric polarization}
    定义电极化强度(electric polarization)为单位体积的电偶极矩
    \begin{equation}
        \bm P:=\sum_i N_i\ave{\bm p_i}.
    \end{equation}
\end{definition}
假设更高阶的宏观多极矩密度可忽略,则有质介质的电势为
\begin{align}
    \notag
    \Phi(\bm r,\bm r')&=\frac1{4\pi\varepsilon_0}\int_V\biggfkh{\frac{\rho(\bm r')}{\abs{\bm r-\bm r'}}+\frac{\bm P(\bm r')\cdot(\bm r-\bm r')}{\abs{\bm r-\bm r'}^3}}\d V\\
    &=\frac1{4\pi\varepsilon_0}\int_V\frac{\rho(\bm r')-\div\bm P(\bm r')}{\abs{\bm r-\bm r'}}\d V'.
\end{align}
其中$\rho$是自由电荷(或过剩电荷)密度,
$\rho_\pol:=-\div\bm P$是极化电荷密度。%上式是由电荷分布$(\rho-\div\bm P)$激发电势的习惯表达式。
从而
%由$\bm E=-\nabla\Phi$,第一个Maxwell方程为 
\begin{equation}
    \label{eqn:divE-P}
    \div\bm E=-\lapla\Phi=\frac{\rho-\div\bm P}{\varepsilon_0}.
\end{equation}
若极化不均匀,则任何小体积内的电荷都会有净增加或净减少,这对应$\div\bm P$这一项。
\begin{definition}
    {电位移矢量}{electric displacement}
    将式\eqref{eqn:divE-P}移项,可定义电位移矢量(electric displacement)
    \begin{equation}
        \bm D:=\varepsilon_0\bm E+\bm P.
    \end{equation}
\end{definition}
从而式\eqref{eqn:divE-P}变为 
\begin{equation}
    \label{eqn:divD}
    \div\bm D=\rho.
\end{equation}
式\eqref{eqn:curlE-macro}和\eqref{eqn:divD}是式\eqref{eqn:curlE}和\eqref{eqn:divE}的宏观对应。

\begin{definition}
    {本构关系}{constitutive relation}
    % 为了解Maxwell方程组,我们需要找到。
    为了简化,我们假设介质是各向同性的(isotropic),且系统对外加场的响应是线性的(linear)。因此$\bm D$和$\bm E$之间的本构关系(constitutive relation)为
    \begin{equation}
        \bm P=\varepsilon_0\chi_\elc\bm E.
    \end{equation}
    其中无量纲常数$\chi_\elc$称为介质的电极化率(electric susceptibility)。由此得到 
    \begin{equation}
        \bm D=\varepsilon\bm E,
    \end{equation}
    其中$\varepsilon:=\varepsilon_\r\varepsilon_0$称为电容率或介电常数(permittivity),
    无量纲常数$\varepsilon_\r:=1+\chi_\elc$称为相对介电常数(relative permittivity)。
\end{definition}

% \begin{table}[!htp]
%     \centering
%     \caption{常见物质的相对介电常数}
%     \begin{tabular}{cc}
%         \toprule
%         物质&$\varepsilon_\r$\\
%         \midrule
%         真空&1\\
%         空气&1.0006\\
%         Si&11.68\\
%         \bottomrule
%     \end{tabular}
% \end{table}
如果电介质不仅是各向同性的,而且是均匀的,则$\varepsilon$与位置无关。式\eqref{eqn:divE-P}可被写为 
\begin{equation}
    \div\bm E=\frac\rho\varepsilon.
\end{equation}
%各向同性
均匀电介质中的所有问题都可归结为前几章的问题,仅需将所有的$\varepsilon_0$替换为$\varepsilon$。

\paragraph{表面分布}

考虑介质界面上$\bm D$和$\bm E$的分布,$\uvec n$是边界上从区域(1)指向区域(2)的法向量,我们可以得到:
\begin{subequations}
    \begin{align}
        \label{eqn:Pn2-Pn1}
        (\bm P_2-\bm P_1)\cdot\uvec n&=-\sigma_\pol,\\
        \label{eqn:Dn2-Dn1}
        (\bm D_2-\bm D_1)\cdot\uvec n&=\sigma,\\
        \label{eqn:Et2-Et1}
        (\bm E_2-\bm E_1)\times\uvec n&=\bm 0.
    \end{align}
\end{subequations}
其中$\sigma_\pol$是边界上表面极化电荷密度;
$\sigma$是宏观表面电荷密度(不包含极化电荷$\sigma_\pol$)。

对于非均匀介质,极化电荷在介质内部随处可见;对于均匀介质,极化电荷仅位于自由电荷附近和两种不同介质之间的界面处;极化电荷所在的界面不是理想的几何界面,实际上,它们的厚度包含了几层分子。

\subsection{含电介质的边值问题}
\label{ssec:boundary-value problem with dielectrics}

我们可以推广之前介绍的方法来解决电介质的边界问题。
\begin{example}{电介质中的镜像电荷}{method of images, dielectrics}
    半无限电介质$\varepsilon_1$与半无限电介质$\varepsilon_2$被隔开,
    考虑嵌入在$\varepsilon_1$中的点电荷$q$距离边界$d$,如图所示。
    \begin{center}
        \includegraphics[page=\pagenumref{tikz:method of images, dielectrics}]{figures/tikz/layouts.pdf}
        \captionof{figure}{电介质中的电荷}
        \label{fig:method of images, dielectrics}
    \end{center}
    为了计算$z>0$部分的电位,在$z<0$部分与$q$对称的位置放置镜像电荷$q'$:
    \[
        \Phi_1=\frac1{4\pi\varepsilon_1}\biggkh{\frac q{R_1}+\frac{q'}{R_2}},\quad z>0;
    \]
    为了计算$z<0$部分的电位,在$q$的位置替换为镜像电荷$q''$:
    \[
        \Phi_2=\frac1{4\pi\varepsilon_2}\frac{q''}{R_1},\quad z<0.
    \]
    由边界条件: 
    \begin{align*}
        \begin{cases}
            \varepsilon_1E_{1z}=\varepsilon_2E_{2z}\\
            E_{1\rho}=E_{2\rho}
        \end{cases}\implies
        \begin{cases}
            q-q'=q''\\
            \frac{q+q'}{\varepsilon_1}=\frac{q''}{\varepsilon_2}
        \end{cases}\implies
        \begin{cases}
            q'=-\frac{\varepsilon_2-\varepsilon_1}{\varepsilon_2+\varepsilon_1}q\\[1ex]
            q''=\frac{2\varepsilon_2}{\varepsilon_2+\varepsilon_1}q
        \end{cases}
    \end{align*}
    $\varepsilon_2>\varepsilon_1$时,$q,q'$符号相反,$\varepsilon_1$内的电场线外凸;
    
    $\varepsilon_2<\varepsilon_1$时,$q,q'$同号,$\varepsilon_1$内电场线内凸。
    \tcblower 
    由式\eqref{eqn:Pn2-Pn1},$z=0$上极化电荷面密度为
    %\bm P_i=(\varepsilon_i-\varepsilon_0)\bm E_i=-(\varepsilon_i-\varepsilon_0)\nabla\Phi(0^\pm).
    \[
        \sigma_\pol=P_{2z}-P_{1z}=-\frac q{2\pi}\frac{\varepsilon_0(\varepsilon_2-\varepsilon_1)}{\varepsilon_1(\varepsilon_2+\varepsilon_1)}\frac d{(\rho^2+d^2)^{3/2}}.
    \]
    当$\varepsilon_2\gg\varepsilon_1$时,介质$\varepsilon_2$表现很像导体($q'=-q$)。
\end{example}
\begin{example}{匀强电场中的介质球}{dielectric sphere in uniform electric field}
    半径为$a$的介质球$\varepsilon$在匀强电场$E_0$中。其电势
    \begin{alignat*}{2}
        \Phi_1&=\sum_{\ell=0}^\infty A_\ell r^\ell P_\ell(\cos\theta),&\quad&r<a\\
        \Phi_2&=\sum_{\ell=0}^\infty\bigfkh{B_\ell r^\ell+C_\ell r^{-(\ell+1)}}P_\ell(\cos\theta),&&r>a
    \end{alignat*}
    又$r\to\infty$时,$\Phi_2\simeq -E_0r\cos\theta$,故 
    \[
        \Phi_2=-E_0r\cos\theta+\sum_{\ell=0}^\infty C_\ell r^{-(\ell+1)}P_\ell(\cos\theta),
    \]
    由边界条件……
    \[
        \Phi=\begin{cases}
            -\frac{3}{\varepsilon_\r+2}E_0r\cos\theta,&r<a\\
            -E_0r\cos\theta+\biggkh{\frac{\varepsilon_\r-1}{\varepsilon_\r+2}}E_0\frac{a^3}{r^2}\cos\theta,&r>a
        \end{cases}
    \]
    因此介质球内是匀强电场
    \[
        E_1=\frac{3}{\varepsilon_\r+2}E_0;
    \]
    介质球外相当于外加场$E_0$加上原点的电偶极场,偶极矩朝向电偶极场的方向
    \[
        \bm p=4\pi\varepsilon_0\biggkh{\frac{\varepsilon_\r-1}{\varepsilon_\r+2}}a^3\bm E_0,
    \]
    球体内的电极化强度
    \begin{equation}
        \label{eqn:P-E}
        \bm P=(\varepsilon-\varepsilon_0)\bm E_1=3\varepsilon_0\biggkh{\frac{\varepsilon_\r-1}{\varepsilon_\r+2}}\bm E_0,
    \end{equation}
    因此表面极化电荷面密度:
    \[
        \sigma_\pol=\bm P\cdot\uvec r=3\varepsilon_0\biggkh{\frac{\varepsilon_\r-1}{\varepsilon_\r+2}}E_0\cos\theta.
    \]
\end{example}

\subsection{分子极化率和电极化率}
\label{ssec:molecular polarizability and electric susceptibility}

\newcommand{\molecule}{\mathrm{mol}}
\newcommand{\internal}{\mathrm{in}}
\newcommand{\near}{\mathrm{near}}

本节中讨论分子极化率(molecular polarizability) $\chi_\molecule$和电极化率(electric susceptibility) $\chi_\elc$之间的关系。%我们仅从给它列个大纲开始。

在稀薄(rarefied)介质中,宏观场与作用在任何分子(或分子群)上的场几乎没有区别;但在致密(dense)介质中,相邻分子的极化会产生一个内部场$\bm E_\internal$,所以分子上的总电场是$\bm E+\bm E_\internal$。
内部场$\bm E_\internal$除了考虑相邻分子的贡献$\bm E_\near$外,还要减去可以用极化强度$\bm P$描述的平均贡献$\bm E_\pol$:
\[
    \bm E_\internal=\bm E_\near-\bm E_\pol,
\]
由\eqref{eqn:int E dV},极化强度$\bm P$贡献的平均场为$\bm E_\pol=-\bm P/3\varepsilon_0$;
而对于大多数材料,可有效近似$\bm E_\near=\bm 0$;
故$\bm E_\internal=\bm P/3\varepsilon_0$。

\begin{definition}
    {分子极化率}{molecular polarizability}
    分子极化率描述单个分子在电场下的极化能力
    \begin{equation}
        \ave{\bm p_\molecule}=\varepsilon_0\gamma_\molecule(\bm E+\bm E_\internal).
    \end{equation}
\end{definition}

\begin{theorem}
    {Clausius-Mossotti方程}{Clausius-Mossotti equation}
    分子极化率$\gamma_\molecule$与电极化率$\chi_\elc$之间的关系为
    \begin{equation}
        \gamma_\molecule=\frac3N\frac{\chi_\elc}{\chi_\elc+3}=\frac3N\frac{\varepsilon_\r-1}{\varepsilon_\r+2}.
    \end{equation}
\end{theorem}

\begin{proof}
    由电极化强度的定义
    \[
        \bm P=N\ave{\bm p_\molecule}=N\varepsilon_0\gamma_\molecule(\bm E+\bm E_\internal)=N\gamma_\molecule\biggkh{\varepsilon_0\bm E+\frac{\bm P}3}.
    \]
    代入$\bm P=\varepsilon_0\chi_\elc\bm E,\enspace\varepsilon_\r=1+\chi_\elc$即得。
\end{proof}

\begin{remark}
    这种关系对气体等稀释物质最适用。
    对于液体和固体,尤其是$\varepsilon_\r$很大时,上式仅近似有效。
\end{remark}

\paragraph{分子极化率的模型}
由谐振子模型
\[
    m\omega_0^2\bm r=e\bm E\implies\bm p_\text{mol}=e\bm r=\frac{e^2\bm E}{m\omega_0^2}\implies\gamma_\text{mol}=\frac{e^2}{m\omega_0^2\varepsilon_0}.
\]
$\omega_0$为分子的固有频率。

\begin{example}
    {}{}
    常温常压(normal temperature and pressure, NTP)下,气体$\chi_\elc<10^{-3}$,固体和液体$\chi_\elc\sim 1$。
\end{example}

一般地,考虑不同能量的分子的Boltzmann分布,
\[
    \gamma_\text{mol}\simeq\frac{e^2}{m\omega_0^2\varepsilon_0}+\frac{p_0^2}{3\varepsilon_0\kB T}.
\]
$p_0$为分子固有偶极矩的大小。

\subsection{电介质中的静电能}
\label{ssec:electrostatic energy in dielectric media}

一般情况下,我们对电介质不作假设。可由微小功$\vd W$推出电介质中的静电能为
\[
    W=\iint_0^D\bm E\cdot\vd\bm D\d V.
\]
若介质是线性的,则
\begin{equation}
    W=\frac12\int_V\bm E\cdot\bm D\d V=\frac12\int_V\rho(\bm r)\Phi(\bm r)\d V.
\end{equation}
上式仅当介质是线性时才在宏观上成立。
\paragraph{电介质在极化过程中储存的能量}过程分电荷不变和电势不变;
\subparagraph{电荷分布不变过程}
若电介质放入前后源电荷分布$\rho(\bm r)$不变
\begin{align*}
    \D W&=\frac12\int_V(\bm E\cdot\bm D-\bm E_0\cdot\bm D_0)\d V\\
    &=\frac12\int_V(\bm E\cdot\bm D_0-\bm E_0\cdot\bm D)\d V+\frac12\int_V(\bm E+\bm E_0)\cdot\cancel{(\bm D-\bm D_0)}\d V\\
    &=-\frac12(\varepsilon-\varepsilon_0)\int_V\bm E\cdot\bm E_0\d V=-\frac12\int_V\bm P\cdot\bm E_0\d V,
\end{align*}
因此电介质的能量密度为
\begin{equation}
    w=-\frac12\bm P\cdot\bm E_0,
\end{equation}
%假设电介质移动了一个微小的位移$\vd x$,系统的总能量发生变化$\vd W$,则电介质受到的外力
%F=-\biggkh{\frac{\vd W}{\vd x}}_Q,
\subparagraph{电势不变过程}下面分析在电势分布不变情形下,电介质放入前后的能量变化(不能用$\vd W_V=-F\vd x$计算)。
\[
    \vd W=\int_V\Phi(\bm r)\vd\rho(\bm r)\d V
\]
又对于线性介质,
\[
    \vd W=\frac12\int_V\bigfkh{\rho(\bm r)\vd\Phi(\bm r)+\Phi(\bm r)\vd\rho(\bm r)}\d V,
\]
故$\rho(\bm r)\vd\Phi(\bm r)=\Phi(\bm r)\vd\rho(\bm r)$。

在电势分布不变情形下,电介质放入的过程可以分为两步:
\begin{compactenum}
    \item 断开电极板和电源的接触,移动电介质,这一过程中电荷分布不变,
    \[
        \vd W_1=\frac12\int_V\bigfkh{\rho(\bm r)\vd\Phi_1(\bm r)+0}\d V;
    \]
    \item 将极板与电源相连,电势恢复到初始值,$\vd\Phi_1+\vd\Phi_2=0$
    \[
        \vd W_2=\frac12\int_V\bigfkh{\rho(\bm r)\vd\Phi_2(\bm r)+\Phi(\bm r)\vd\rho_2(\bm r)}\d V=-2\vd W_1.
    \]
\end{compactenum}
因此$\vd W=\vd W_1+\vd W_2=-\vd W_1$,从而
\[
    \vd W_V=-\vd W_Q.
\]
力
\[
    F=-\biggkh{\frac{\vd W}{\vd x}}_Q=\biggkh{\frac{\vd W}{\vd x}}_V.
\]

