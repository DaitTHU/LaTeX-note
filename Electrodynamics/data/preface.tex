\preface

这门课属于是我本科期间学的最好的一门课之一了(百分制成绩99,等级制A$^+$),谨以此笔记纪念我在电动力学上付出的努力以及我今后一生与其的羁绊。

\paragraph{声明}
本课内容基于\textit{Classical Electrodynamics} by John David Jackson和\textit{Introduction to Electrodynamics} by David J. Griffiths。
其中,除\chapref{chap:vector analysis}矢量分析和\chapref{chap:relativistic electromagnetic}相对论电动力学是参考Griffiths书外,其余均参考Jackson书。

本笔记将Jackson书中的Chap. 2, 3: Boundary-Value Problems in Electrostatics: I, II合并为了一章:\chapref{chap:boundary-value problems}静电学中的边界条件问题。

本笔记一切公式均采取\textbf{国际单位制}(Système International d'Unités, SI),没有例外。

\paragraph{绪论}
在Jackson书的Introduction中,着重强调了以下关键词:
\begin{compactitem}
	\item force-field
	\item classical-quantum
	\item linear-nonlinear
	\item mathmetical idealization-truly physics
	\item space-time
	\item macroscopic-microscopic
\end{compactitem}