\preface

这门课属于是我本科期间学的最好的一门课之一了(百分制成绩99,等级制A$^+$)。

\paragraph{声明}

本课内容基于
\begin{itemize}
	\item Jackson, Classical Electrodynamics.
	\item Griffiths, Introduction to Electrodynamics.
\end{itemize}
其中,除附录 \ref{chap:vector analysis} 矢量分析和\chapref{chap:relativistic electromagnetic}相对论电动力学是参考Griffiths书外,其余均参考Jackson书。

本笔记将Jackson书中的Chap. 2, 3: Boundary-Value Problems in Electrostatics: I, II合并为了一章:\chapref{chap:boundary-value problems}静电学中的边界条件问题。

本笔记一切公式均采取\textbf{国际单位制}(Système International d'Unités, SI),没有例外。

\paragraph{绪论}
在Jackson书的Introduction中,着重强调了以下关键词:
\begin{compactitem}
	\item force-field
	\item classical-quantum
	\item linear-nonlinear
	\item mathmetical idealization-truly physics
	\item space-time
	\item macroscopic-microscopic
\end{compactitem}