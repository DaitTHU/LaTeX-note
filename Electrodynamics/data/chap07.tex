\chapter{波导}
\label{chap:waveguide}
%有金属边界存在时的电磁场,是一个具有实用价值的相当重要的题材。
在波长的数量级为数米或更短一些的高频情况下,产生和发送电磁波的唯一实用的方法,是利用线度可以与有关波长相比的金属结构。我们在本章里先考虑在一个导体附近的场,并讨论场对表面的穿透以及伴随着的电阻损失,然后用相当普遍的观点讨论中空金属管导引的电磁波和谐振腔等问题,而且在讨论的过程中引进特殊例子来说明这些问题。用两种不同观点讨论波导中的衰减和谐振腔的$Q$值。%其次把地球-电离层系统当作新奇的谐振腔来处理,接着简短地讨论电介质波导。我们还阐述波导中任意场的简正模展开,并应用到定域源产生的场。本章最后把简正模展开进一步应用到用变分法处理波导中障碍物问题。

\section{导体介质的场}
理想导体(perfect conductor)和超导体(superconductor)的异同?
最大的不同是不存在理想导体,但存在超导体。

理想导体表面电荷密度为$\varsigma$、表面电流密度为$\bm K$,理想导体外的电位移$\bm D$和磁场强度$\bm H$满足
\[
    \uvec n\cdot\bm D=\varsigma,\quad\uvec n\times\bm H=\bm K.
\]
在理想导体表面,仅存在法向的$\bm E$和切向的$\bm H$,并且这两个场在理想导体内骤降为0。

对于非理想的良导体,导体内的场随趋肤深度
\begin{equation}
    \delta:=\sqrt{\frac2{\mu\c\omega\sigma}}.
\end{equation}
指数衰减,且
其电导率$\sigma$是一个有限量,而Ohm定律$\bm J=\sigma\bm E$说明表面没有电流$\bm K$。边界条件应改为
\[
    \uvec n\times(\bm H-\bm H\c)=\bm 0,
\]
由于场在垂直于表面的方向上的空间变化比平行方向快得多,因此和法向导数相比,切向导数可以忽略。若$\xi$表示导体向内的法向坐标,则
\[
    \nabla\simeq-\uvec n\pp\xi.
\]
忽略导体内的位移电流,谐振场的Maxwell方程变为
\begin{align*}
    \bm E\c&\simeq\frac1\sigma\curl\bm H\c\simeq-\frac1\sigma\uvec n\times\pv{\bm H\c}\xi,\\
    \bm H\c&=-\i\frac1{\mu\c\omega}\curl\bm E\c\simeq\i\frac1{\mu\c\omega}\uvec n\times\pv{\bm E\c}\xi.
\end{align*}
变形得到
\begin{align*}
    \Bigkh{\pp[2]\xi+\i\frac2{\delta^2}}(\uvec n\times\bm H\c)&\simeq\bm 0,\\
    \uvec n\cdot\bm H\c&\simeq 0.
\end{align*}
第二个方程表面:导体内$\bm H$平行于表面,这与边界条件相符。解得
\begin{equation}
    \bm H\c=\bm H_\parallel\e{-\xi/\delta}\e{\i\xi/\delta}.
\end{equation}
$\bm H_\parallel$是导体外表面的切向磁场,则导体内电场
\begin{equation}
    \bm E\c\simeq\sqrt{\frac{\mu\c\omega}{2\sigma}}(1-\i)(\uvec n\times\bm H_\parallel)\e{-\xi/\delta}\e{\i\xi/\delta}.
\end{equation}
故导体内的$\bm H\c$和$\bm E\c$与表面平行,大小随指数迅速衰减,二者有相位差,且磁场比电场强得多。在表面外侧,除了法向的$\bm E$和切向的$\bm H$外,还有$\bm E$的一个微小的切向分量存在。这意味着有功率流入导体,每单位面积的吸收对时间平均的功率是
\begin{equation}
    \dv{P_\text{loss}}a=-\frac12\Re[\uvec n\cdot(\bm E\times\bm H\cj)]=\frac{\mu\c\omega\delta}4|\bm H_\parallel|^2.
\end{equation}
可把这个结果简单地解释为导体内的Ohm损失,由Ohm定律,导体表面附件有电流密度$\bm J=\sigma\bm E\c$,而
\[
    \dv{P_\text{loss}}V=\frac12\bm J\cdot\bm E\cj=\frac\sigma 2|\bm J|^2.
\]
定义有效面电流密度
\[
    \bm K\eff:=\int\zti\bm J\d\xi=\uvec n\times\bm H_\parallel.
\]
则
\[
    \dv{P_\text{loss}}a=\frac1{2\sigma\delta}|\bm K\eff|^2.
\]
\section{柱形空腔和波导}
电磁波在中空的金属柱体内的传播或激发,是一种极为重要的实用情形。如果柱体具有端面,就叫做空腔(cavity),否则就叫做波导(waveguide)。在我们讨论这问题时,都假定柱体的边界面是理想导体,在实际应用中发生的能量损失,可用上一节的方法来估计。

假定柱形曲面$S$的截面形状和大小沿柱体轴不变,且柱体内是介电常数$\varepsilon$、磁导率$\mu$的均匀非耗散介质。当柱体内的场具有谐振关系$\e{-\i\omega t}$时,Maxwell方程组可化为以下形式:
\begin{alignat*}{2}
    \div\bm E&=0,&\qquad\curl\bm E&=\i\omega\bm B,\\
    \div\bm B&=0,&\curl\bm B&=-\i\mu\varepsilon\omega\bm E.
\end{alignat*}
即
\[
    (\lapla+\mu\varepsilon\omega^2)\bm E=\bm 0,
\]
由边界限制,单独考虑$z$方向上的空间变化:
\[
    \bm E(x,y,z,t)=\bm E(x,y)\e{\i(\pm kz-\omega t)},
\]
由适当的线性组合可以给出沿$z$方向的行波或驻波,目前,波数$k$是一个未知参数,它可能是实数也可能是复数。假定了场对$z$有上述依赖关系后,波动方程就简化为二维形式:
\begin{equation}
    (\lapla_\t+\mu\varepsilon\omega^2-k^2)\bm E=0,
\end{equation}
其中$\lapla_\t$表示Laplace算符的横向部分:
\[
    \lapla=:\lapla_\t+\pp[2]z,
\]
再将场分解为平行和垂直于$z$轴两个分量$\bm E=\bm E_z+\bm E_\t$:
\[
    \bm E_z=E_z\uvec z,\quad\bm E_\t=(\uvec z\times\bm E)\times\uvec z.
\]
Maxwell方程组变为
\begin{alignat*}{2}
    \nabla_\t\cdot\bm E_\t&=-\pv{E_z}z,&\qquad\qquad\nabla_\t\cdot\bm B_\t&=-\pv{B_z}z,\\
    \pv{\bm E_\t}z+\i\omega\uvec z\times\bm B_\t&=\nabla_\t E_z,&\uvec z\cdot(\nabla_\t\times\bm E_\t)&=\i\omega\bm B_z,\\
    \pv{\bm B_\t}z-\i\mu\varepsilon\omega\uvec z\times\bm E_\t&=\nabla_\t B_z,&\uvec z\cdot(\nabla_\t\times\bm B_\t)&=\i\mu\varepsilon\omega\bm E_z.
\end{alignat*}
因此$\bm E_\t,\bm B_\t$可以被$E_z,B_z$确定
\begin{align}
    \bm E_\t&=\i\frac1{\mu\varepsilon\omega^2-k^2}(k\nabla_\t E_z-\omega\uvec z\times\nabla_\t B_z),\\
    \bm B_\t&=\i\frac1{\mu\varepsilon\omega^2-k^2}(k\nabla_\t B_z+\mu\varepsilon\omega\uvec z\times\nabla_\t E_z),
\end{align}
\paragraph{TEM模}
在讨论中空柱体内可以存在的各种场之前,我们先说一说简并型或特殊型解,即所谓横电磁(TEM)波。这种解只有垂直于传播方向的横场分量,$E_z=B_z=0,\enspace\bm E_\t=\bm E_\text{TEM}$且
\begin{equation}
    \nabla_\t\times\bm E_\text{TEM}=\bm 0,\quad\nabla_\t\cdot\bm E_\text{TEM}=0.
\end{equation}
因此$\bm E_\text{TEM}$是二维静电问题的一个解。由此可得轴向波数为
\begin{equation}
    \label{eqn:TEM k0}
    k_0=\omega\sqrt{\mu\varepsilon},
\end{equation}
磁场 
\[
    \bm B_\text{TEM}=\pm\sqrt{\mu\varepsilon}\uvec z\times\bm E_\text{TEM},
\]
电导率为无穷大的单个中空柱形导体内不可能存在TEM模。为了运载TEM模,必须有两个及以上的等势面(比如同轴线)。TEM模的一个重要性质是不存在截止频率,即任意$\omega$均可以传播($k$均是实数)。
\paragraph{TM波和TE波}
在中空柱体内(以及高频时的传输线上)出现两类场的构型。考虑纵向分量$E_z$和$B_z$满足的波动方程以及边界条件,就可以看出这两类场的构型的存在。对于理想导电柱体,边界条件是
\[
    \uvec n\times\bm E=\bm0,\quad\uvec n\cdot\bm B=0,
\]
显然$E_z$的边界条件为
\begin{equation}
    E_z|_S=0,
\end{equation}
相应的$B_z$的边界条件是
\begin{equation}
    \edg{\pv{B_z}n}_S=0,
\end{equation}
对于给定频率$\omega$来说,波数$k$只可以出现某些值(波导情况);或对于给定波数$k$来说,只允许某些$\omega$值(谐振腔)。由于$E_z$和$B_z$边界条件不同,所以一般来说本征值是不同的,因此场自然分为两种类型:
\begin{compactitem}
	\item TM波:横向只有磁场,处处$B_z=0$,电场边界条件$E_z|_S=0$;
	\item TE波:横向只有电场,处处$E_z=0$,磁场边界条件$\p B_z/\p n|_S=0.$
\end{compactitem}
\section{波导}
当波在一个均匀截面的中空波导中传播时,可得TM波和TE波的横磁场和横电场的关系如下:
\[
    \bm H_\t=\pm\uvec z\times\frac{\bm E_\t}Z,
\]
$Z$为波阻抗:
\[
    Z:=\begin{cases}
        \frac k{\varepsilon\omega}=\frac k{k_0}\sqrt{\frac\mu\varepsilon},&\text{TM}\\
        \frac {\mu\omega}k=\frac {k_0}k\sqrt{\frac\mu\varepsilon},&\text{TE}
    \end{cases}
\]
特别的,自由空间阻抗为
\begin{equation}
    Z_0=\sqrt{\frac{\mu_0}{\varepsilon_0}}=\mu_0c\simeq 120\pi\,\si{\ohm}
\end{equation}
这里$k_0$由\eqref{eqn:TEM k0}给出。
横向场可由纵向场决定
\begin{alignat*}{2}
    \text{TM}&:\quad&\bm E_\t&=\pm\i\frac k{\gamma^2}\nabla_\t\psi,\quad E_z=\psi\e{\pm\i kz};\\
    \text{TE}&:\quad&\bm H_\t&=\pm\i\frac k{\gamma^2}\nabla_\t\psi,\quad H_z=\psi\e{\pm\i kz}.
\end{alignat*}
式中$\gamma^2=\mu\varepsilon\omega^2-k^2$,标量函数$\psi$满足二维波动方程
\[
    (\lapla_\t+\gamma^2)\psi=0,
\]
边界条件:
\[
    \text{TM}:\enspace\psi|_S=0;\qquad\text{TE}:\enspace\edg{\pv\psi n}_S=0.
\]
可以确定一系列本征值$\gamma_\lambda$和本征函数$\psi_\lambda$,这些不同的解称为波导的波模。对于给定的频率$\omega$,每个$\lambda$对应的波数
\[
    k_\lambda^2=\mu\varepsilon\omega^2-\gamma_\lambda^2.
\]
由$k_\lambda^2=0$可定义截止频率(cutoff frequency):
\begin{equation}
    \label{eqn:cutoff omega}
    \omega_\lambda:=\frac{\gamma_\lambda}{\sqrt{\mu\varepsilon}},
\end{equation}
则上式可以写成
\[
    k_\lambda=\sqrt{\mu\varepsilon}\sqrt{\omega^2-\omega_\lambda^2}.
\]
截止频率的意义为:
\begin{compactitem}
    \item 当$\omega>\omega_\lambda$时,$k_\lambda^2>0$,$\lambda$模的波可以在波导中传播;
    \item 当$\omega<\omega_\lambda$时,$k_\lambda^2<0$,$\lambda$模的波不能传播。
\end{compactitem}
因此给定频率上,只有有限个波模可以传播。选择波导的尺寸,使得在使用频率上只能出现最低波模的波,这往往是最方便的。

波导中的波长总是大于自由空间中的波长,因此相速度就大于无限空间中的值
\[
    v_\varphi=\frac\omega{k_\lambda}=\frac c{\sqrt{\mu\varepsilon}}\frac1{\sqrt{1-(\omega_\lambda/\omega)^2}}>\frac c{\sqrt{\mu\varepsilon}}.
\]
\subsection{矩形波导中的波模}
\begin{example}{矩形波导中的波模}{modes in a rectangular waveguide}
    在内尺寸为$a,b$的矩形波导中(不失一般性取$a>b$),
    \[
        \Bigkh{\pp[2]x+\pp[2]y+\gamma^2}\psi=0,
    \]
    考虑TE波:$E_z=0,\,H_z=\psi$,边界上$\p\psi/\p n=0$可得
    \[
        \psi_{mn}=H_0\cos\Bigkh{\frac{m\pi x}a}\cos\Bigkh{\frac{n\pi y}b},\quad\gamma_{mn}=\pi\sqrt{\Bigkh{\frac ma}^2+\Bigkh{\frac nb}^2}.
    \]
    则TE模的最低截止频率为$m=1,\,n=0$
    \[
        \omega_{10}=\frac\pi{\sqrt{\mu\varepsilon}a}.
    \]
    %对应波长$\lambda_{10}=2a\sqrt{\mu\varepsilon}/c$,
    在自由空间(即波导中没有介质)中正好$\lambda_{10}=2a$。将TE$_{10}$的场写出:
    \begin{align*}
        H_z&=H_0\cos\Bigkh{\frac{\pi x}a}\e{\i(kz-\omega t)},\\
        H_x&=-\i\frac{ka}\pi H_0\sin\Bigkh{\frac{\pi x}a}\e{\i(kz-\omega t)},\\
        E_y&=\i\frac{\omega a\mu}\pi H_0\sin\Bigkh{\frac{\pi x}a}\e{\i(kz-\omega t)}.
    \end{align*}
    $\i$说明传播方向上$H_x,E_y$与$H_z$有$90\degree$的相位差。
    \tcblower
    TM波:$H_z=0,E_z=\psi,$
    \[
        E_z=E_0\sin\Bigkh{\frac{m\pi x}a}\sin\Bigkh{\frac{n\pi y}a},
    \]
    $\gamma_{mn}$同TM模,但$m,n$均不能为0,最低波模为$m=n=1$。
\end{example}
\subsection{波导中的能流和衰减}
我们扩大前面对任意截面形状的柱形波导所作的一般讨论,使它包括沿波导的能流以及波的衰减,后者是当电导率有限时由波导管壁中能量损耗所引起的。我们只限于讨论波导中只存在一种波模的情况;不过将简短提一下简并波模。能流用复Poynting矢量描写;
\begin{equation}
    \bm S=\frac12\bm E\times\bm H\cj=\frac{\omega k}{2\gamma^4}
    \begin{cases}
        \varepsilon\Bigkh{\uvec z|\nabla_\t\psi|^2+\i\frac{\gamma^2}k\psi\nabla_\t\psi\cj},&\text{TM}\\[1ex]
        \mu\Bigkh{\uvec z|\nabla_\t\psi|^2-\i\frac{\gamma^2}k\psi\cj\nabla_\t\psi},&\text{TE}
    \end{cases}
\end{equation}
因为$\psi$一般是实数,\footnote{有可能这样激发一个波导,使得某给定波模或一些波模的线性组合具有复数$\psi$。这时候横向能流对时间的平均值将不为零,但因为它是一个环流,所以实际上只代表贮藏的能量,在实用上不大重要。}
因此$\bm S$的第二项代表无功能流,并且对能流的时间平均没有贡献。总功率流
\[
    P=\int_A\bm S\cdot\uvec z\d a=\frac{\omega k}{2\gamma^4}
    \begin{Bmatrix}
        \varepsilon\\\mu
    \end{Bmatrix}
    \biggkh{\oint_{\p A}\cancel{\psi\cj\pv\psi n}\d\ell-\int_A\psi\cj\nabla_\t^2\psi\d a}.
\]
由特征方程$(\nabla_\t^2+\gamma_\lambda^2)\psi=0$,
\[
    P=\frac1{2\sqrt{\mu\varepsilon}}\biggkh{\frac\omega{\omega_\lambda}}^2\biggkh{1-\frac{\omega_\lambda^2}{\omega^2}}^{1/2}
    \begin{Bmatrix}
        \varepsilon\\\mu
    \end{Bmatrix}
    \int_A\psi\cj\psi\d a.
\]
时间平均能量密度
\[
    u=\frac14\Bigkh{\varepsilon\bm E\cdot\bm E\cj+\frac1\mu\bm B\cdot\bm B\cj},
\]
单位长度的场能量
\[
    U=\frac12\biggkh{\frac\omega{\omega_\lambda}}^2
    \begin{Bmatrix}
        \varepsilon\\\mu
    \end{Bmatrix}
    \int_A\psi\cj\psi\d a,
\]
能流的速度正是群速度
\[
    \frac PU=\frac k\omega\frac1{\mu\varepsilon}=\frac1{\sqrt{\mu\varepsilon}}\sqrt{1-\frac{\omega_\lambda^2}{\omega^2}}=v_\text g.
\]
结合前面说的相速度
\begin{equation}
    v_\text gv_\varphi=\frac1{\mu\varepsilon},
\end{equation}
在无限介质中,$v_\text g$总是小于$v_\varphi$,并且在截止频率时降为零。
\paragraph{良导体的波导}
对于理想导体,$k_\lambda$是实数或纯虚数;而对于电导率有限的良导体来说,$k_\lambda$是一个实数与一个小复数的和。

后面的略。
\paragraph{最小衰减频率}
略。
\section{边界条件的扰动}
略
\section{谐振腔}
一个谐振腔(resonant cavity)可以是任何形状的,具有一个封闭的导体表面。但通常我们将端面放置在一定长度的圆柱形波导上,以产生一个空腔。端面是垂直于圆柱体轴线的平面。

由$z=0,d$的边界条件,应有驻波形式:
\begin{align*}
    \text{TM}&:&E_z=\psi(x,y)\cos\Bigkh{\frac{p\pi z}d},&&p&=0,1,2,\ldots,\\
    \text{TE}&:&H_z=\psi(x,y)\sin\Bigkh{\frac{p\pi z}d},&&p&=1,2,3,\ldots,
\end{align*}
特征值频率
\[
    \omega_{\lambda_p}^2=\frac1{\mu\varepsilon}\Bigfkh{\gamma_\lambda^2+\Bigkh{\frac{p\pi}d}^2},
\]

对于TM模,$\psi=E_z$,
\[
    \psi(\rho,\phi)=E_0J_m(\gamma_{mn}\rho)\e{\pm\i m\phi},\quad\gamma_{mn}=\frac{x_{mn}}R,
\]
其中$x_{mn}$表示$J_m(x)$的第$n$个零点:
\begin{equation}
    \begin{matrix}
        x_{mn}&n=1&n=2&n=3\\
        m=0&2.405&5.520&8.654\\
        m=1&3.832&7.016&10.173\\
        m=2&5.136&8.417&11.620\\
        m=3&6.380&9.761&13.015
    \end{matrix}
\end{equation}
谐振频率为
\begin{equation}
    \omega_{mnp}=\frac1{\sqrt{\mu\varepsilon}}\sqrt{\Bigkh{\frac{x_{mn}}R}^2+\Bigkh{\frac{p\pi}d}^2}.
\end{equation}

对于TE模,$\psi=H_z$,
\[
    \gamma_{mn}=\frac{x'_{mn}}R,
\]
其中$x'_{mn}$表示$J'_m(x)$的第$n$个零点:
\begin{equation}
    \begin{matrix}
        x'_{mn}&n=1&n=2&n=3\\
        m=0&3.832&7.016&10.176\\
        m=1&1.841&5.331& 8.536\\
        m=2&3.054&6.706& 9.970\\
        m=3&4.201&8.015&11.346\\
    \end{matrix}
\end{equation}
谐振频率为
\begin{equation}
    \omega_{mnp}=\frac1{\sqrt{\mu\varepsilon}}\sqrt{\Bigkh{\frac{x'_{mn}}R}^2+\Bigkh{\frac{p\pi}d}^2}.
\end{equation}
\section{腔内的功率损失}
对于理想导体,$|E(\omega)|$在各个$\omega_i$上是一个个$\delta$函数;而对于良导体,则会有一定的半高宽。

定义品质因子(quality factor)
\begin{equation}
    Q:=\omega_0\frac{\text{stored energy}}{\text{power loss}}.
\end{equation}

