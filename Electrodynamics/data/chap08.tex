\chapter{多极辐射}
\label{chap:multipole radiation}
在第七章和第八章里,讨论了电磁波的性质,以及电磁波在有界的和无界的区域内的传播情况。但是,很少谈到如何产生这些电磁波的问题。本章就转入这个问题,并讨论定域振荡电荷和电流密度系统辐射的电磁波,我们的论述是直截了当的,而不去精心推敲数学形式。%当然,这样做只限于比较简单的辐射系统,我们把求系统辐射的渐近法(用任意$l$的矢量多极场)推迟到第十六章里论述。本章只讨论电偶极子、磁偶极子和电器极子,以及导体上电流的一些简单位形。也论述波导中一个源的简单多极子展开和孔的有效多极矩。

%本章后半部,用较大篇幅讨论散射和衍射。首先解释长波长情形下的散射,包括瑞利对蓝天的解释和有关的论题,然后讨论标量衍射理论和矢量衍射理论,并举了一些例子。最后讨论短波长情形下的散射和重要的光学定理。
\section{局域振荡源}
对于随时间变化的电荷和电流系统,我们都可以将随时间变化的量作Fourier分析,并分别处理各个Fourier分量。因此,不失一般性,我们只考虑随时间作正弦变化的局域电荷和电流系统所产生的势、场和辐射,设
\begin{align*}
    \rho(\bm r,t)&=\rho(\bm r)\e{-\i\omega t},\\
    \bm J(\bm r,t)&=\bm J(\bm r)\e{-\i\omega t},
\end{align*}
矢量势
\begin{align}
    \notag
    \bm A(\bm r,t)&=\frac{\mu_0}{4\pi}\int\frac{\bm J(\bm r',t')}{\abs{\bm r-\bm r'}}\vd\Bigkh{t'-t+\frac{\abs{\bm r-\bm r'}}c}\d t'\nd v'\\
    \label{eqn:A-Jeiwt}
    &=\frac{\mu_0}{4\pi}\int\frac{\bm J(\bm r')}{\abs{\bm r-\bm r'}}\e{\i k\abs{\bm r-\bm r'}}\d V'.
\end{align}
磁场和电场为
\[
    \bm H=\frac1{\mu_0}\curl\bm A,\quad\bm E=\i\frac Zk\curl\bm H.
\]
给定电流分布$\bm J(\bm r')$以后,至少在原则上可以通过计算\eqref{eqn:A-Jeiwt}的积分来定出场的分布。%9.4节将讨论一两个直接计算源积分的实例.
但是,现在我们先确立在电流源被局限于小区域内(与波长相比,确实很小)的极限情形下,场所具有的某些简单而普遍的性质。如果源的线度为$d$,波长为$\lambda=2\pi c/\omega$,并且$d<\lambda$,则有三个令人感兴趣的空间区域:近区、中间区、远区。

近区:$d\ll r\ll\lambda$,\eqref{eqn:A-Jeiwt}中的指数可以用1代替,并用球谐函数\eqref{eqn:1/|r-r'|=YY}展开,
\[
    \lim_{kr\to0}\bm A(\bm r)=\frac{\mu_0}{4\pi}\sum_{\ell=0}^\infty\sum_{m=-\ell}^\ell\frac{4\pi}{2\ell+1}\frac{Y_{\ell m}(\theta,\phi)}{r^{\ell+1}}\int\bm J(\bm r')r'^\ell Y\cj_{\ell m}(\theta',\phi')\d V'.
\]
上式表明,近场是准静态的,除了按$\e{-\i\omega t}$方式作简谐振荡外,其它性质都是静态的。

远区,$d\ll\lambda\ll r$,
\[
    \bm A(\bm r,t)=\frac{\mu_0}{4\pi}\frac{\e{\i kr}}r\int\bm J(\bm r')\e{-\i\bm k\cdot\bm r'}\d V'.
\]
可以对$\e{-\i\bm k\cdot\bm r'}$展开,略。

中间区,$d\ll r\sim\lambda$,近似失效,引入
球Bessel函数和Hankel函数
\begin{align}
    j_\ell(x)&:=\sqrt{\frac\pi{2x}}J_{\ell+1/2}(x),\\
    y_\ell(x)&:=\sqrt{\frac\pi{2x}}Y_{\ell+1/2}(x),\\
    h^\pm_\ell(x)&:=j_\ell(x)\pm\i y_\ell(x).
\end{align}
则
\[
    \bm A(\bm r)=\i\mu_0k\sum_{\ell,m}h_\ell^+(kr)Y_{\ell m}(\theta,\phi)\int\bm J(\bm r')j_\ell(kr')Y\cj_{\ell m}(\theta',\phi')\d V',
\]
推导和后面的略。

标量势与矢量势形式相同,我们考虑电单极的情况
\[
    \Phi(\bm r,t)=\frac1{4\pi\varepsilon_0r}q\Bigkh{t-\frac rc},
\]
式中$q(t)$是源的总电荷。因为电荷是守恒的,且根据定义,局域源是一个没有电荷流入或流出的源,所以总电荷$q$与时间无关。于是,一个定域源的势(和场)的电单极部分必然是静态的。%具有谐和时间依赖关系e-'(w≠0)的场没有单极子项。

%现在讨论w≠0时最低阶多极场。因为这些场可以由矢势通过(9.4)和(9.5)算出,所以我们在下面的论述中不再明显提及标势。
\section{电偶极辐射}
下面我们求电偶极子
\[
    \bm p:=\int\bm r'\rho(\bm r')\d V',
\]
激发的电场和磁场。
通过远区近似,
\[
    \bm A=-\i\frac{\mu_0\omega}{4\pi}\bm p\frac{\e{\i kr}}r.
\]
场
\begin{align*}
    \bm H&=\frac{ck^2}{4\pi}(\uvec n\times\bm p)\frac{\e{\i kr}}r\Bigkh{1+\i\frac1{kr}},\\
    \bm E&=\frac1{4\pi\varepsilon_0}\biggfkh{k^2(\uvec n\times\bm p)\times\uvec n\frac{\e{\i kr}}r+\bigkh{3\uvec n(\uvec n\cdot\bm p)-\bm p}\Bigkh{\frac1{r^3}-\i\frac{k}{r^2}}\e{\i kr}}.
\end{align*}
$r\to 0$时,
\begin{align*}
    \bm H&=\i\frac{\omega}{4\pi}\frac{\uvec n\times\bm p}{r^2},\\
    \bm E&=\frac1{4\pi\varepsilon_0}\frac{3\uvec n(\uvec n\cdot\bm p)-\bm p}{r^3};
\end{align*}
$r\to\infty$时,
\begin{align*}
    \bm H&=\frac{ck^2}{4\pi}(\uvec n\times\bm p)\frac{\e{\i kr}}r,\\
    \bm E&=Z_0\bm H\times\uvec n,
\end{align*}
% $Z_0$是自由空间阻抗。
由振荡偶极子辐射的时间平均功率
\[
    \dv P\Omega=\frac12\Re\bigfkh{r^2(\bm E\times\bm H\cj)\cdot\uvec n}=\frac{c^2Z_0}{32\pi^2}k^4p^2\sin^2\theta,
\]
故
\begin{equation}
    P=\frac{c^2Z_0^2k^4}{12\pi}p^2.
\end{equation}
为什么天空是蓝色的,特别是在垂直于太阳光线的天弧处?因为$\theta=90\degree$时太阳光线所激发的偶极辐射功率为0。
\begin{example}{天线}{center-fed, linear antenna}
    对于一个中间馈送(center-fed)的线性天线(antenna),其电流
    \[
        I_0\Bigkh{1-\frac{2\abs{z}}d}\e{-\i\omega t},
    \]
    则其电偶极矩为
    \[
        p=\i\frac{I_0d}{2\omega},
    \]
    辐射功率
    \[
        P=\frac{Z_0I_0^2}{48\pi}(kd)^2.
    \]
    由于天线向外辐射功率,因此它需要消耗能量,尽管天线是一个理想导体,但仍有辐射阻抗$R_\text{rad}$
    \[
        R_\text{rad}=\frac{Z_0}{24\pi}(kd)^2\simeq 5(kd)^2.
    \]
\end{example}
\section{磁偶极子和电四极子}
\paragraph{磁偶极子}
磁偶极矩对矢量势的贡献
\[
    \bm A=\frac{k\mu_0}{4\pi}(\uvec n\times\bm m)\frac{\e{\i kr}}r\Bigkh{\i-\frac1{kr}},
\]
得到磁场和电场
\begin{align*}
    \bm H&=\frac1{4\pi}\biggfkh{k^2(\uvec n\times\bm m)\times\uvec n\frac{\e{\i kr}}r+\bigkh{3\uvec n(\uvec n\cdot\bm m)-\bm m}\Bigkh{\frac1{r^3}-\i\frac{k}{r^2}}\e{\i kr}},\\
    \bm E&=-\frac{Z_0}{4\pi}k^2(\uvec n\times\bm m)\frac{\e{\i kr}}r\Bigkh{1+\i\frac1{ kr}}.
\end{align*}
与电偶极子形式高度相似。
\paragraph{电四极子}
电四极子对矢量势的贡献
\[
    \bm A=-\frac{\mu_0ck^2}{8\pi}\frac{\e{\i kr}}r\Bigkh{1+\i\frac1{kr}}\int\bm r'(\uvec n\cdot\bm r)\rho(\bm r')\d V'.
\]
磁场
\[
    \bm H=-\i\frac{ck^3}{24\pi}\frac{\e{\i kr}}r\uvec n\times(Q\cdot\uvec n).
\]
辐射功率 
\[
    \dv P\Omega=\frac{c^2Z_0}{1152\pi^2}k^6\abs{\bigfkh{\uvec n\times(Q\cdot\uvec n)}\times\uvec n}^2
\]
从而 
\[
    P=\frac{c^2Z_0k^6}{1440\pi}\sum_{\alpha,\beta}\abs{Q_{\alpha\beta}}^2\sim\omega^6.
\]


