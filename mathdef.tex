\theoremstyle{definition}
\newtheorem*{corollary}{推论}
\newtheorem*{remark}{注}
\newtheorem*{lemma}{引理}

%% text
\newcommand*{\vs}{~\text{-}~}
\newcommand*{\const}{\text{const}}
\newcommand*{\otherwise}{\text{otherwise}}
\newcommand*{\plusc}{{\color{lightgray}\,+\,\const}}

%% roman
\newcommand*{\e}{\mathop{}\!\mathrm{e}^}  % exp
\let\accenti\i
\renewcommand*{\i}{\mathrm{i}}

\usepackage{bm}
\newcommand{\hatbm}[1]{\hat{\bm{#1}}}  % bm with hat, maybe Fourier Transform
\newcommand{\nvec}[1]{\hat{\bm{#1}}}  % normalized vector, without space
\newcommand{\uvec}[1]{\mathop{}\!\nvec{#1}}  % unit vector, with possible space
\newcommand{\dotbm}[1]{\dot{\bm{#1}}}

\usepackage{mathrsfs}  % \mathscr

%% blackboard bold: number sets
\newcommand*{\NN}{\mathbb N}    % natural
\newcommand*{\ZZ}{\mathbb Z}    % integer
\newcommand*{\QQ}{\mathbb Q}    % rational
\newcommand*{\RR}{\mathbb R}    % real
\newcommand*{\CC}{\mathbb C}    % complex
\newcommand*{\FF}{\mathbb F}    % any number field
\newcommand*{\PP}{\mathbb P}    % prime

\usepackage{cancel}

%% vector operator
\let\divides\div
\newcommand*{\grad}{\nabla}         % gradient
\renewcommand*{\div}{\nabla\cdot}   % divergence
\newcommand*{\curl}{\nabla\times}   % curl
\newcommand*{\lap}{\nabla^2}        % Laplacian

%% differential
\let\accentd\d
\renewcommand*{\d}{\mathop{}\!\mathrm{d}}   % differential
\newcommand*{\nd}{\mathrm{d}}               % differential without space
\newcommand*{\vd}{\mathop{}\!\delta}    % delta: δ
\newcommand*{\D}{\Delta}                % Delta: Δ
\newcommand*{\p}{\partial}              % partial: ∂
\newcommand{\dd}[2][{}]{\frac{\nd^{#1}}{\nd{#2}^{#1}}}      % d/dx
\newcommand{\dv}[3][{}]{\frac{\nd^{#1}#2}{\nd{#3}^{#1}}}    % df/dx
\newcommand{\pp}[2][{}]{\frac{\p^{#1}}{\p{#2}^{#1}}}    % ∂/∂x
\newcommand{\pv}[3][{}]{\frac{\p^{#1}#2}{\p{#3}^{#1}}}  % ∂f/∂x
\newcommand{\pw}[3]{\frac{\p^2{#1}}{\p{#2}\p{#3}}}      % ∂^2f/∂x∂y

%% integral limits
\newcommand*{\zti}{_0^{+\infty}}            % zero to infinity
\newcommand*{\iti}{_{-\infty}^{+\infty}}    % -infinity to +infinity

% brackets with auto size
\newcommand{\abs}[1]{\left\lvert#1\right\rvert}     % absolute value: |x|
\newcommand{\norm}[1]{\left\lVert#1\right\rVert}    % norm: ||x||
\newcommand{\edg}[1]{\left.#1\right\rvert}          % edge line: f|
\newcommand{\kh}[1]{\left(#1\right)}                % parentheses: (x)
\newcommand{\fkh}[1]{\left[#1\right]}               % square brackets: [x]
\newcommand{\hkh}[1]{\left\{#1\right\}}             % braces: {x}
% \newcommand{\ang}[1]{\left\langle #1\right\rangle}  % angle brackets: <x>
\newcommand{\floor}[1]{\left\lfloor#1\right\rfloor} % floor
\newcommand{\ceil}[1]{\left\lceil#1\right\rceil}    % ceil
\newcommand{\ave}[1]{\left\langle #1\right\rangle}  % average: <x>
\newcommand{\set}[2]{\left\{#1\,\middle\vert\,#2\right\}}   % set: {x|x1,x2,...}
\newcommand{\bra}[1]{\left\langle #1\right\vert}    % bra: <ψ|
\newcommand{\ket}[1]{\left\vert #1\right\rangle}    % ket: |ψ>
\newcommand{\brkt}[2]{\left\langle #1\middle\vert #2\right\rangle}  % inner product of bra-ket: <φ|ψ>
\newcommand{\ktbr}[2]{\left\vert#1\right\rangle\hspace{-3pt}\left\langle #2\right\vert} % ket-bra: |ψ><φ|
\newcommand{\division}[2]{\left.{#1}\middle/{#2}\right.}    % division: A/B
\newcommand{\inp}[2]{\left\langle #1,#2\right\rangle}       % inner product: <u,v>

% brackets with fixed size
\newcommand{\nnorm}[1]{\lVert#1\rVert}
\newcommand{\nset}[2]{\{#1\,|\,#2\}}
\newcommand{\bigkh}[1]{\bigl(#1\bigr)}
\newcommand{\Bigkh}[1]{\Bigl(#1\Bigr)}
\newcommand{\biggkh}[1]{\biggl(#1\biggr)}
\newcommand{\bigfkh}[1]{\bigl[#1\bigr]}
\newcommand{\Bigfkh}[1]{\Bigl[#1\Bigr]}
\newcommand{\biggfkh}[1]{\biggl[#1\biggr]}

%% math operator
\let\Real\Re
\let\Imaginary\Im
\let\Re\relax
\let\Im\relax
\DeclareMathOperator{\Re}{Re}  % real part
\DeclareMathOperator{\Im}{Im}  % imaginary part
\DeclareMathOperator{\sech}{sech}
\DeclareMathOperator{\csch}{csch} 
\DeclareMathOperator{\arcsec}{arcsec}
\DeclareMathOperator{\arccot}{arccot} 
\DeclareMathOperator{\arccsc}{arccsc} 
\DeclareMathOperator{\arsinh}{arsinh} 
\DeclareMathOperator{\arcosh}{arcosh} 
\DeclareMathOperator{\artanh}{artanh} 
\DeclareMathOperator{\sinc}{sinc}
\DeclareMathOperator{\sgn}{sgn}     % sign function
\DeclareMathOperator{\id}{id}       % identity mapping
\DeclareMathOperator{\Res}{Res}     % residue
\DeclareMathOperator{\supp}{supp}   % support set

%% linear algebra
\DeclareMathOperator{\rank}{rank}   % rank
\DeclareMathOperator{\diag}{diag}   % diagonal
\DeclareMathOperator{\tr}{tr}       % trace
\newcommand*{\tp}{^\top}    % transpose: A^T
\newcommand*{\cj}{^\ast}    % conjugate: A*
\newcommand*{\dg}{^\dagger} % conjugate transpose/Hermite: A†
\newcommand*{\iv}{^{-1}}    % inverse: A^-1

%% physicists
\newcommand*{\Schr}{Schrödinger}
\newcommand*{\Legd}{Legendre}
\newcommand*{\deB}{de Broglie}
\newcommand*{\Rayl}{Rayleigh}
\newcommand*{\Lande}{Landé}

%% particles
\newcommand*{\elc}{\mathrm e}
\newcommand*{\pton}{\mathrm p}
\newcommand*{\nton}{\mathrm n}
\newcommand*{\mol}{\mathrm m}

%% physical constants/notation
\newcommand*{\NA}{N_{\mathrm A}}    % Avogadro constant
\newcommand*{\kB}{k_{\mathrm B}}    % Boltzmann constant
\newcommand*{\muB}{\mu_\mathrm B}   % Bohr magne
\newcommand*{\Ek}{E_{\mathrm k}}    % kinetic energy
\newcommand*{\FWHM}{\mathrm{FWHM}}  % full width at half maximum

%% subscript/superscript 
\newcommand*{\eff}{_\mathrm{eff}}   % effective
\newcommand*{\tot}{_\mathrm{tot}}   % total
\newcommand*{\maxi}{_\mathrm{max}}  % maximum
\newcommand*{\mini}{_\mathrm{min}}  % minimum

%% unit tag
\newcommand*{\lSI}{\tag{SI}}    % le SI
\newcommand*{\CGS}{\tag{CGS}}   % cm, g, s system

%% other
\newcommand*{\qqquad}{\qquad\quad}
\newcommand*{\qqqquad}{\qquad\qquad}

\newcommand{\notimplies}{\hspace{1ex}\not\hspace{-1ex}\implies}

\let\geq\geqslant
\let\leq\leqslant

\newcommand*{\avg}[1]{\overline{#1}}

\newcommand*{\bigo}{\mathcal O}     % big O notation
\newcommand*{\degree}{^\circ}       % degree

\newcommand{\fracdisp}[2]{\frac{\displaystyle #1}{\displaystyle #2}}

\newcommand{\lhkh}[1]{\left\{#1\right.} % left brace: {x
