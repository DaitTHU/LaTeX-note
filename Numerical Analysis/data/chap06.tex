\chapter{线性方程组的迭代解法}

\begin{definition}
    {线性方程组的迭代解法}{iterative method}
    构造一个向量序列,使其尽快收敛到线性方程组$Ax=b$的解。
\end{definition}

\section{迭代法基本概念}

\subsection{向量序列和矩阵序列的极限}

\begin{definition}
    {向量序列的极限}{limit of vector array}
    给定赋范空间$(\FF^n,\norm\cdot)$以及一组向量序列$x^{(1)},x^{(2)},\ldots$,若$\exists x\in\FF^n$使得
    \begin{equation}
        \lim_{i\to\infty}\norm{x^{(i)}-x}=0,\iff\lim_{i\to\infty}x^{(i)}=x.
    \end{equation}
    则称$x^{(i)}$收敛于$x$。
\end{definition}

\begin{remark}
    由于有限维线性空间中所有范数都是等价的,向量序列的收敛性不依赖于范数的选择。
\end{remark}

\begin{theorem}
    {向量序列的收敛性判定}{}
    向量序列收敛等价于其各分量序列收敛:
    \begin{equation}
        \lim_{i\to\infty}x^{(i)}=x,\iff\lim_{i\to\infty}x^{(i)}_j=x_j,\enspace \forall j=1,\ldots,n.
    \end{equation}
\end{theorem}

\begin{proof}
    选取无穷范数:
    \[
        \lim_{i\to\infty}\norm{x^{(i)}-x}_\infty=0,\iff\lim_{i\to\infty}\max_j\abs{x^{(i)}_j-x_j}=0,\iff\lim_{i\to\infty}\abs{x^{(i)}_j-x_j}=0,\enspace\forall j.
        \qedhere
    \]
\end{proof}

\begin{remark}
    类似地,我们可以定义矩阵序列的收敛性。
\end{remark}

\begin{theorem}
    {收敛到零矩阵}{limit=O}
    $\lim_{k\to\infty}A^{(k)}=O\iff\forall x\in\FF^n$均有$\lim_{k\to\infty}A^{(k)}x=0$。
\end{theorem}

\begin{proof}
    必要性:对于任一矩阵的从属范数,
    \[
        \norm{A^{(k)}x}\leq\norm{A^{(k)}}\norm x\to0;
    \]
    充分性:其矩阵各分量
    \[
        \abs{a_{ij}^{(k)}}=\abs{e_i\tp A^{(k)}e_j}\leq\norm{A^{(k)}e_j}_\infty\to 0.
        \qedhere
    \]
\end{proof}

\begin{theorem}
    {矩阵幂序列的收敛性}{matrix exponent convergence}
    给定$B\in\FF^{n\times n}$,下面三个命题等价:
    \begin{enumerate}
        \item $\lim_{k\to\infty}B^k=O$;
        \item $\rho(B)<1$;
        \item 存在一个矩阵范数$\norm\cdot$使得$\norm B<1$。
    \end{enumerate}
\end{theorem}

\begin{proof}
    $(1)\implies(2)$:
    采取反证法,若$\lambda$为$B$的特征值且$\abs\lambda\geq 1$其对应特征向量为$x$,则 
    \[
        \norm{B^kx}=\abs\lambda^k\norm x\not\to 0;
    \]

    $(2)\implies(3)$:
    由$\rho(B)<1$,给定$\epsilon=\frac{1-\rho(B)}2>0$,
    由\thmref{thm:norm<=rho+eps},
    存在一个矩阵范数使得$\norm B\leq\rho(B)+\epsilon<1$;

    $(3)\implies(1)$:
    由$\norm{B^k}\leq\norm B^k\to0$即得。
\end{proof}

\begin{theorem}
    {矩阵幂的矩阵范数与谱半径}{k-root norm B^k=rho}
    给定$B\in\FF^{n\times n}$和任一矩阵范数$\norm\cdot$,均有
    \begin{equation}
        \lim_{k\to\infty}\norm{B^k}^{1/k}=\rho(B).
    \end{equation}
\end{theorem}

\begin{proof}
    采用夹逼定理。
    一方面,由\thmref{thm:rho<=norm},
    \[
        \rho(B)=\rho(B^k)^{1/k}\leq\norm{B^k}^{1/k};
    \]
    另一方面,$\forall\epsilon>0$,记 
    \[
        B_\epsilon=\frac{B}{\rho(B)+\epsilon},
    \]
    则$\rho(B_\epsilon)<1$,由\thmref{thm:matrix exponent convergence},$\lim_{k\to\infty}B_\epsilon^k=O$,即$\exists N=N(\epsilon)>0$使得$\forall k>N$,
    \[
        \norm{B_\epsilon^k}=\frac{\norm{B^k}}{(\rho(B)+\epsilon)^k}<1,\iff\norm{B^k}^{1/k}<\rho(B)+\epsilon,\iff\lim_{k\to\infty}\norm{B^k}^{1/k}\leq\rho(B).
    \]
    综上,极限相等。    
\end{proof}

\subsection{迭代公式的构造}

\begin{theorem}
    {线性定常迭代}{linear stationary iterative method}
    若线性方程$Ax=b$与方程$x=Bx+f$同解,则可以构造一个定常(stationary)的迭代公式:
    \begin{equation}
        x^{(k+1)}=Bx^{(k)}+f,
    \end{equation}
    其中$B$称为迭代矩阵。若$x^{(k)}$收敛,则收敛值$x^*$就是方程的解。
\end{theorem}

\begin{theorem}
    {迭代法的收敛性分析}{convergence of iterative method}
    迭代法$x^{(k+1)}=Bx^{(k)}+f$收敛与以下命题等价:
    \begin{enumerate}
        \item $\lim_{k\to\infty}B^k=O$;
        \item $\rho(B)<1$;
        \item 存在一种矩阵范数$\norm\cdot$使得$\norm B<1$。
    \end{enumerate}
\end{theorem}

\begin{proof}
    记误差向量$e^{(k)}:=x^{(k)}-x^*$,则$e^{(k+1)}=Be^{(k)}$。迭代法收敛意味着$\forall e^{(0)}\in\FF^n$均有
    \[
        \lim_{k\to\infty}e^{(k)}=\lim_{k\to\infty}B^ke^{(0)}=0,
    \]
    由\thmref{thm:limit=O},$\lim_{k\to\infty}B^k=O$,由\thmref{thm:matrix exponent convergence} 知三个命题等价。
\end{proof}

\begin{theorem}
    {迭代法的误差分析}{error of iterative method}
    若迭代法收敛,第$k$次迭代的误差满足
    \begin{equation}
        \norm{e^{(k)}}\leq\frac{\norm B}{1-\norm B}\norm{x^{(k)}-x^{(k-1)}}\leq\frac{\norm B^k}{1-\norm B}\norm{x^{(1)}-x^{(0)}}.
    \end{equation}
\end{theorem}

\begin{proof}
    由$e^{(k)}=Be^{(k-1)}=B(e^{(k-1)}-e^{(k)})+Be^{(k)}$,故
    \[
        \norm{e^{(k)}}\leq\norm{B(e^{(k-1)}-e^{(k)})}+\norm{Be^{(k)}}\leq\norm B\Bigkh{\norm{e^{(k-1)}-e^{(k)}}+\norm{e^{(k)}}},
    \]
    移项即得第一个不等号;又$e^{(k)}-e^{(k-1)}=B(e^{(k-1)}-e^{(k-2)})$可得 
    \[
        \norm{e^{(k)}-e^{(k-1)}}\leq\norm B\norm{e^{(k-1)}-e^{(k-2)}},
    \]
    反复迭代即得第二个不等号。
\end{proof}

\begin{theorem}
    {迭代法的收敛速度}{converging speed of iterative method}
    若给定$\epsilon>0$,希望
    \[
        \frac{\norm{e^{(k)}}}{\norm{e^{(0)}}}\leq\epsilon,
    \]
    则要求迭代次数
    \begin{equation}
        \label{eqn:k<-lneps/R}
        k\geq\frac{-\ln\epsilon}{-\ln\norm{B^k}^{1/k}}.
    \end{equation}
\end{theorem}

\begin{proof}
    按矩阵从属范数的定义
    \[
        \norm{B^k}=\max_{e^{(0)}\neq 0}\frac{\norm{B^ke^{(0)}}}{\norm{e^{(0)}}}=\max_{e^{(0)}\neq 0}\frac{\norm{e^{(k)}}}{\norm{e^{(0)}}}.
    \]
    因此只需$\norm{B^k}\leq\epsilon$即可。
\end{proof}

\begin{remark}
    式\eqref{eqn:k<-lneps/R}并不好处理,可以用近似:
    \[
        k\approx\frac{-\ln\epsilon}{-\ln\rho(B)}.
    \]
\end{remark}

\begin{definition}
    {平均收敛率}{average converging speed}
    迭代法的平均收敛率为
    \begin{equation}
        R_k(B):=-\ln\norm{B^k}^{1/k}.
    \end{equation}
    其渐进收敛率$R(B):=R_\infty(B)=-\ln\rho(B)$。
\end{definition}

\section{(单步)定常线性迭代}

\subsection{Richardson迭代法}

\begin{example}
    {Richardson迭代法}{Richardson iterative method}
    最简单的情形$x=(I-A)x+b$,称为Richardson迭代:
    \begin{equation}
        x^{(k+1)}=(I-A)x^{(k)}+b,
    \end{equation}
    收敛条件为$\rho(I-A)<1$。
    定义残差$r^{(k)}:=b-Ax^{(k)}$,则有递推关系:
    \begin{align*}
        r^{(k+1)}&=(I-A)r^{(k)},\\
        x^{(k+1)}&=x^{(k)}+r^{(k)},
    \end{align*}
    继而
    \[
        x^{(k)}=x^{(0)}+r^{(0)}+\cdots+r^{(k-1)}=x_0+\sum_{i=0}^{k-1}(I-A)^ir^{(0)}.
    \]
    
\end{example}

\begin{definition}
    {Krylov子空间}{Krylov subspace}
    给定非零向量$v$和非零方阵$A$,定义
    \[
        \mathcal K_n(v):=\lspan(v,Av,\ldots,A^{n-1}v),
    \]
    称为Krylov子空间。
\end{definition}

\begin{corollary}
    显然
    \[
        \mathcal K_1(v)\subset\mathcal K_2(v)\subset\cdots\subset\mathcal K_n(v)\subset\cdots
    \]
    最大维度$\sup_{n\to\infty}\dim\mathcal K_n(v)$称为$v$关于$A$的度。
\end{corollary}

\begin{definition}
    {预处理矩阵}{preconditioning matrix}
    若$M$可逆且$M\iv$是$A\iv$的一个较好的近似,对$M\iv Ax=M\iv b$使用Richardson迭代:
    \begin{equation}
        x^{(k+1)}=(I-M\iv A)x^{(k)}+M\iv b=x^{(k)}+M\iv r^{(k)},
    \end{equation}
    矩阵$M$称为预处理矩阵(preconditioning matrix)。
\end{definition}

\begin{remark}
    预处理矩阵的选择依据:
    方便计算、
    好近似。
\end{remark}

\subsection{Jacobi迭代法和Gauss-Seidel迭代法}

将$A$拆解为$A=D+L+U$的形式,其中$D$是对角部分,$L$是严格下三角部分,$U$是严格上三角部分。
\footnote{本节对$L,U$的定义与教材上的定义相差一个负号。}

\begin{example}
    {Jacobi迭代法}{Jacobi iterative method}
    预处理矩阵$M=D$,迭代矩阵$B_\text J=-D\iv(L+U)$,迭代的分量形式为:
    \begin{equation}
        x_{i+1}^{(k)}=\frac1{a_{kk}}\biggkh{b_k-\sum_{j\neq k}a_{kj}x_i^{(j)}}.
    \end{equation}
\end{example}

\begin{example}
    {Gauss-Seidel迭代法}{GS iterative method}
    预处理矩阵$M=D+L$,迭代矩阵$B_\text G=-(D+L)\iv U$,迭代的分量形式为:
    \begin{equation}
        x_{i+1}^{(k)}=\frac1{a_{kk}}\biggkh{b_k-\sum_{j=1}^{k-1}a_{kj}x_{i+1}^{(j)}-\sum_{j=k+1}^na_{kj}x_i^{(j)}}.
    \end{equation}
\end{example}

\begin{remark}
    Jacobi法和G-S法收敛$\iff\rho(B_\text J)<1$和$\rho(B_\text G)<1$。
    另外,有时容易验证矩阵$A$满足收敛的充分条件。
\end{remark}

\begin{theorem}
    {Jacobi法和G-S法收敛的充分条件:对角占优}{}
    若$A$严格对角占优或不可约弱对角占优,则Jacobi法和G-S法收敛。
\end{theorem}

\begin{proof}
    对于Jacobi法,若$B_\text J$有特征值$\lambda$且$\abs\lambda\geq 1$,则 
    \[
        \det(\lambda I-B_\text J)=0,\implies\det(D+\lambda\iv(L+U))=0,
    \]
    若$A$严格对角占优(或不可约弱对角占优),则$D+\lambda\iv(L+U)$也为严格对角占优(或不可约弱对角占优),矛盾!故$\rho(B_\text J)<1$,Jacobi法收敛。

    对于G-S法,设$B_\text G$有特征值$\lambda$且$\abs\lambda\geq 1$,则 
    \[
        \det(\lambda I-B_\text G)=0,\implies\det(D+L+\lambda\iv U)=0,
    \]
    若$A$严格对角占优(或不可约弱对角占优),则$D+L+\lambda\iv U$也为严格对角占优(或不可约弱对角占优),矛盾!故$\rho(B_\text G)<1$,G-S法收敛。
\end{proof}

\begin{theorem}
    {对称矩阵的Jacobi法和G-S法收敛的充要条件}{}
    若$A$对称且对角元均为正,则Jacobi法和G-S法收敛$\iff D\pm(L+U)$均正定。
\end{theorem}

\subsection{超松弛迭代法}

\begin{example}
    {逐次超松弛迭代}{successive overrelaxation iterative method}
    预处理矩阵$M=D/\omega+L$,迭代矩阵$L_\omega=(D+\omega L)\iv[(1-\omega)D-\omega U]$,
    \begin{equation}
        x_{i+1}^{(k)}=(1-\omega)x_i^{(k)}+\frac\omega{a_{kk}}\biggkh{b_k-\sum_{j=1}^{k-1}a_{kj}x_{i+1}^{(j)}-\sum_{j=k+1}^na_{kj}x_i^{(j)}}.
    \end{equation}
    称为逐次超松弛迭代法(successive overrelaxation, SOR),其中$\omega$称为松弛因子。
\end{example}

\begin{theorem}
    {$\rho(L_\omega)$与$\omega$的关系}{}
    若$A$可逆且对角元非0,则
    \begin{equation}
        \rho(L_\omega)\geq\abs{1-\omega}.
    \end{equation}
\end{theorem}

\begin{proof}
    设$L_\omega$的特征值是$\lambda_1,\ldots,\lambda_n$,由于$L_\omega=(D+\omega L)\iv[(1-\omega)D-\omega U]$可以写成下三角矩阵和上三角矩阵的乘积:
    \[
        \lambda_1\cdots\lambda_n=\det(L_\omega)=\det(D\iv)\det((1-\omega)D)=(1-\omega)^n,
    \]
    故
    \[
        \rho(L_\omega)=\max_i\abs{\lambda_i}\geq\abs{\lambda_1\cdots\lambda_n}^{1/n}=\abs{1-\omega}.
        \qedhere
    \]
\end{proof}

\begin{corollary}
    SOR收敛的必要条件是$0<\omega<2$。
\end{corollary}

\begin{theorem}
    {对称正定矩阵的SOR收敛的充要条件}{}
    若$A$对称正定,则$0<\omega<2\iff$SOR收敛。
\end{theorem}

\begin{proof}
    只需证明充分性。
    设$\lambda$和$x$是$L_\omega$的任意一个特征值和对应的特征向量,则
    \[
        [(1-\omega)D-\omega U]x=\lambda(D+\omega L)x,
    \]
    上式两边与$x$内积得到
    \[
        % (1-\omega)\inp{Dx}x-\omega\inp{Ux}x=\lambda(\inp{Dx}x+\omega\inp{Lx}x),\implies
        \lambda=\frac{(1-\omega)\inp{Dx}x-\omega\inp{Ux}x}{\inp{Dx}x+\omega\inp{Lx}x},
    \]
    由$A$正定,$D$也正定,记$\inp{Dx}x=:p>0$,$\inp{Lx}x=a+\i b$,由$A$对称,$L\tp=U$,
    \[
        \inp{Ux}x=\inp{L\tp x}x=\inp x{Lx}=\overline{\inp{Lx}x}=a-\i b,
    \]
    可将$\lambda$用$\omega,p,a,b$表示出来,继而
    \[
        \abs\lambda^2-1=\frac{[(1-\omega)p-\omega a]^2+\omega^2b^2}{(p+\omega a)^2+\omega^2b^2}-1=-\frac{p\omega(2-\omega)(p+2a)}{(p+\omega a)^2+\omega^2b^2}
    \]
    由$A$正定,
    \[
        \inp{Ax}x=\inp{Dx}x+\inp{Lx}x+\inp{Ux}x=p+2a>0,
    \]
    又$2-\omega>0$,故$\abs\lambda^2<1$,从而$\rho(L_\omega)<1$,SOR收敛。
\end{proof}

\paragraph{最佳松弛因子}

希望选取最优的$\omega$使得迭代矩阵谱半径最小:
\begin{equation}
    \omega^*=\arg\min_{\omega}\rho(L_\omega).
\end{equation}

\begin{theorem}
    {最佳松弛因子}{optimal overrelaxation}
    若$A$为对称正定的(分块)三对角矩阵,则$\rho(B_\text G)=\rho(B_\text J)^2<1$且SOR的最佳松弛因子为
    \begin{equation}
        \omega^*=\frac2{1+\sqrt{1-\rho(B_\text G)}}.
    \end{equation}
    且$\rho(L_{\omega^*})=\omega^*-1$。
\end{theorem}

\begin{remark}
    对于对称正定三对角矩阵,G-S法收敛速度是Jacobi法的两倍,SOR的收敛速度最快。
\end{remark}

\begin{example}
    {Poisson方程的五点差分格式}{}
    讨论二元函数$u=u(x,y)$满足 
    \[
        \pv[2]ux+\pv[2]uy=f(x,y),
    \]
    边界条件$u|_{\p\Omega}=0$。采用差分法,在均匀网格点$(x_i,y_i)$附近作Taylor展开:
    \[
        u_{i\pm 1,j}=u_{ij}\pm h\edg{\pv ux}_{ij}+\frac{h^2}2\edg{\pv[2]ux}_{ij}\pm\frac{h^3}6\edg{\pv[3]ux}_{ij}+\bigo(h^4).
    \]
    可得
    \[
        \edg{\pv[2]ux}_{ij}=\frac{u_{i+1,j}+u_{i-1,j}-2u_{ij}}{h^2}+\bigo(h^2)
    \]
    同理可得$\p^2u/\p^2y$。代入原Poisson方程得到
    \begin{equation}
        u_{i+1,j}+u_{i-1,j}+u_{i,j+1}+u_{i,j-1}-4u_{ij}=h^2f_{ij}.
    \end{equation}
    若网格是$m\times n$的,则可以写成线性方程组$Ax=b$的形式,其中        
    \begin{alignat*}{3}
        x&=\begin{bmatrix}
            u_1\\\vdots\\ u_n
        \end{bmatrix},\enspace u_i=\begin{bmatrix}
            u_{1i}\\\vdots\\u_{mi}
        \end{bmatrix};&
        b&=-h^2\begin{bmatrix}
            f_1\\\vdots\\f_n
        \end{bmatrix},\enspace f_i=\begin{bmatrix}
            f_{1i}\\\vdots\\f_{mi}
        \end{bmatrix};\\
        A&=\begin{bmatrix}
            A_1&-I\\-I&A_2&\ddots\\ &\ddots&\ddots&-I\\ &&-I&A_n
        \end{bmatrix},&\quad A_i&=\begin{bmatrix}
            4&-1\\-1&4&\ddots\\ &\ddots&\ddots&-1\\ &&-1&4
        \end{bmatrix}.
    \end{alignat*}
    其中$A$是$mn$阶方阵,$A_i$是$m$阶方阵。
    故$A$是分块三对角正定矩阵,$B_\text J$的$mn$个特征值为
    \[
        \lambda_{ij}=\frac{\cos(i\pi h)+\cos(j\pi h)}2,\quad i=1,\ldots,m,\enspace j=1,\ldots,n.
    \]
    谱半径$\rho(B_\text J)=\cos(\pi h)$,则$\rho(B_\text G)=\cos^2(\pi h)$
    \[
        \omega^*=\frac2{1+\sin(\pi h)},
    \]
    因此,Jacobi法和SOR的收敛速度分别为
    \begin{alignat*}{3}
        R(B_\text J)&=-\ln(\cos(\pi h))=\frac{\pi^2}2h^2+\frac{\pi^4}{12}h^4+\bigo(h^6),\\
        R(L_{\omega^*})&=-2\ln(\cos(\pi h))+2\ln(1+\sin(\pi h))=2\pi h+\frac{\pi^3}3h^3+\bigo(h^5).
    \end{alignat*}
    可见Jacobi法和SOR的收敛速度差了一个数量级。
\end{example}

\section{非线性迭代方法}


\begin{definition}
    {算子}{operator}
    Hilbert空间$V$中的可逆对称算子$A:V\to V$满足:$\forall x,y\in V$,
    \begin{itemize}
        \item $\inp{Ax}y=\inp x{Ay}$;
        \item $\inp{Ax}x\geq 0$且$\inp{Ax}x=0\iff x=0$。
    \end{itemize}
    可定义范数$\norm x_A:=\sqrt{\inp{Ax}x}$。
\end{definition}

\begin{example}
    {}{}
    $\RR^n$中的内积和$n$阶对称正定矩阵$M$一一对应:
    \[
        \inp xy_M=\inp{Mx}y.
    \]
\end{example}

\begin{theorem}
    {Ritz变分问题}{}
    考虑算子方程
    \begin{equation}
        Ax=b,
    \end{equation}
    考虑$\varphi:V\to\RR$
    \begin{equation}
        \varphi(x):=\frac12\norm x_A^2-\inp bx.
    \end{equation}
    若算子$A$对称正定,则
    \begin{equation}
        Ax^*=b\iff x^*=\arg\min_{x\in V}\varphi(x).
    \end{equation}
\end{theorem}

\begin{proof}
    由$b=Ax^*$,$\varphi(x^*)=-\norm{x^*}_A^2/2$。
    $\forall x\in V$
    \[
        \varphi(x)-\varphi(x^*)=\frac12(\norm x_A^2-2\inp{Ax^*}x+\norm{x^*}_A^2)=\frac12\norm{x-x^*}^2_A.
    \]
    故$\varphi(x)\geq\varphi(x^*)$且$\varphi(x)=\varphi(x^*)\iff x=x^*$。
\end{proof}

因此,迭代法的选择策略为:给定$x^{(k)}$,确定$x^{(k+1)}$使得
\[
    \varphi(x^{(k+1)})<\varphi(x^{(k)}).
\]

\subsection{子空间方法}
\label{ssec:subspace method}

\begin{theorem}
    {子空间搜索}{subspace search}
    设$x+\Phi$是一个仿射(affine)平面,则
    \[
        y^*=\arg\min_{y\in\Phi}\varphi(x+y)\iff 
        % \forall y\in\Phi,\enspace\inp{b-A(x+y^*)}y=0\iff
        b-A(x+y^*)\perp\Phi.
    \]
\end{theorem}

\begin{proof}
    由于
    \[
        \varphi(x+y)-\varphi(x^*)=\frac12\norm{x+y-x^*}_A^2
    \]
    故对于$y^*$应有
    \[
        \forall y\in\Phi,\enspace\inp{x^*-(x+y^*)}y_A=0\iff
        x^*-(x+y^*)\perp_A\Phi.
        \qedhere
    \]
\end{proof}

\begin{theorem}
    {一维极小搜索方法}{1-D minimal search}
    给定$x^{(k)}$和搜索方向$p^{(k)}\neq 0$,考虑一维极小值问题
    \begin{equation}
        \label{eqn:alphak}
        \alpha_k=\arg\min_{\alpha\in\RR}\varphi(x^{(k)}+\alpha p^{(k)})=\frac{\inp{r^{(k)}}{p^{(k)}}}{\norm{p^{(k)}}_A^2}.
    \end{equation}
    并使
    \begin{equation}
        x^{(k+1)}=x^{(k)}+\alpha_kp^{(k)},
    \end{equation}
    若$\alpha_k\neq 0$,则$\varphi(x^{(k+1)})<\varphi(x^{(k)})$,且$r^{(k+1)}\perp p^{(k)}$。
\end{theorem}

\begin{proof}
    残差$r^{(k)}=b-Ax^{(k)}$,则
    \[
        \dd\alpha\varphi(x^{(k)}+\alpha p^{(k)})=-\inp{r^{(k)}}{p^{(k)}}+\alpha\norm{p^{(k)}}_A^2=0.
        \qedhere
    \]
\end{proof}

\subsection{最速下降法}
\label{ssec:steepest descent method}

\begin{theorem}
    {最速下降法}{steepest descent method}
    搜索方向选择$\varphi$在$x^{(k)}$处下降最快的方向:$p^{(k)}=r^{(k)}$,称为最速下降法(steepest descent method):
    \begin{subequations}
        \begin{align}
            r^{(k)}&=b-Ax^{(k)},\\
            \alpha_k&=\frac{\norm{r^{(k)}}^2}{\norm{r^{(k)}}_A^2},\\
            x^{(k+1)}&=x^{(k)}+\alpha_kr^{(k)}.
        \end{align}
    \end{subequations}
    则$\varphi(x^{(k)})$单调下降且有界$\varphi(x^*)$,极限$\lim_{k\to\infty}x^{(k)}=x^*$。
\end{theorem}

\begin{corollary}
    相邻两次搜索方向是正交的:$\inp{r^{(k+1)}}{r^{(k)}}=0$,且
    \[
        r^{(k+1)}=(I-\alpha_k A)r^{(k)}.
    \]
\end{corollary}

\begin{theorem}
    {最速下降法的误差收敛性}{SDM error}
    记误差$e^{(k)}:=x^{(k)}-x^*$,则误差的$A$ - 范数满足
    \begin{equation}
        \label{eqn:SDM e^k}
        \norm{e^{(k)}}_A\leq\biggkh{\frac{\cond_2(A)-1}{\cond_2(A)+1}}^k\norm{e^{(0)}}_A.
    \end{equation}
\end{theorem}

\begin{lemma}
    $\forall x\neq 0$,
    \begin{equation}
        \frac{\norm x_A^2\norm x_{A\iv}^2}{\norm x^4}\leq\frac{(\cond_2(A)+1)^2}{4\cond_2(A)}.
    \end{equation}
\end{lemma}

\begin{proof}
    设$\lambda\maxi,\lambda\mini$分别是对称正定算子$A$的最大最小特征值。
    则$\forall\lambda_i$为$A$的特征值,满足
    \[
        \frac1{\lambda_i}\leq\frac1{\lambda\mini}+\frac1{\lambda\maxi}-\frac{\lambda_i}{\lambda\mini\lambda\maxi}.
    \]
    则$\forall x$且$\norm x=1$,
    \begin{align*}
        \frac{\norm x_A^2\norm x_{A\iv}^2}{\norm x^4}&=\frac{\inp{Ax}x\inp{A\iv x}x}{\inp xx^2}=\sum_i\lambda_ix_i^2\sum_i\frac1{\lambda_i}x_i^2\\
        &\leq\sum_i\lambda_ix_i^2\biggkh{\frac1{\lambda\mini}+\frac1{\lambda\maxi}-\frac1{\lambda\mini\lambda\maxi}\sum_i\lambda_ix_i^2}\leq\frac{(\lambda\maxi+\lambda\mini)^2}{4\lambda\maxi\lambda\mini}.
    \end{align*}
    由$A$对称,$\cond_2(A)=\abs{\lambda\maxi}/\abs{\lambda\mini}$,由$A$正定,$\lambda\mini>0$。
\end{proof}

\begin{proof}
    由$Ae^{(k)}=-r^{(k)}$,有$\inp{e^{(k)}}{r^{(k)}}_A=-\norm{r^{(k)}}^2$,$\norm{e^{(k)}}_A=\norm{r^{(k)}}_{A\iv}$,故
    \begin{align*}
        \norm{e^{(k+1)}}_A^2&=\norm{e^{(k)}+\alpha_kr^{(k)}}_A^2
        =\norm{e^{(k)}}_A^2+\alpha_k^2\norm{r^{(k)}}_A^2+2\alpha_k\inp{e^{(k)}}{r^{(k)}}_A\\
        &=\norm{e^{(k)}}_A^2+\alpha_k^2\norm{r^{(k)}}_A^2-2\alpha_k\norm{r^{(k)}}^2=
        \norm{e^{(k)}}_A^2-\frac{\norm{r^{(k)}}^4}{\norm{r^{(k)}}_A^2}\\
        &=\norm{e^{(k)}}_A^2\biggkh{1-\frac{\norm{r^{(k)}}^4}{\norm{r^{(k)}}_A^2\norm{r^{(k)}}_{A\iv}^2}}\leq\norm{e^{(k)}}_A^2\biggkh{1-\frac{4\cond_2(A)}{(\cond_2(A)+1)^2}}\\
        &=\norm{e^{(k)}}_A^2\biggkh{\frac{\cond_2(A)-1}{\cond_2(A)+1}}^2.
    \end{align*}
    迭代即可得到。
\end{proof}

\begin{remark}
    虽然最速下降法总是收敛的,但当$A$病态时,$\cond_2(A)\gg 1$,收敛速度非常慢;当$\norm{r^{(k)}}$很小时,计算不稳定。
\end{remark}

\subsection{共轭梯度法}
\label{ssec:conjugate gradient method}

仍然考虑对称正定矩阵$A$和一维极小搜索,但不再沿着正交的$r^{(0)},r^{(1)},\ldots$,寻找另一组$p^{(0)},p^{(1)},\ldots$
对于第$k$次搜索:
\[
    \alpha_k=\arg\min_\alpha\varphi(x^{(k)}+\alpha p^{(k)}),\implies
    x^{(k+1)}=x^{(k)}+\alpha_kp^{(k)},
\]
由于
\[
    x^{(k+1)}=x^{(0)}+\alpha_0p^{(0)}+\cdots+\alpha_kp^{(k)}.
\]
我们希望$p^{(k)}$的选取能使得确定$\alpha_k$后,$x^{(k+1)}$是$x^{(0)}+\lspan(p^{(0)},\ldots,p^{(k)})$中最优的,
即一维极小搜索的结果与$(k+1)$维搜索结果相同:
\[
    (\alpha_0,\ldots,\alpha_k)=\mathop{\arg\min}\limits_{(\alpha_0,\ldots,\alpha_k)}\varphi(x^{(0)}+\alpha_0p^{(0)}+\cdots+\alpha_kp^{(k)}).
\]

\begin{definition}
    {$A$ - 正交向量组}{A-orthogonal}
    若向量组$p_1,\ldots,p_n,\ldots$满足$\forall i\neq j$均有
    \begin{equation}
        \inp{p_i}{p_j}_A=0,\iff p_i\perp_A p_j.
    \end{equation}
    则称为$A$ - 正交向量组,也称$A$ - 共轭向量组。
\end{definition}

\begin{corollary}
    给定一个线性无关的向量组,可按照Gram-Schmidt正交化得到对应的$A$ - 正交向量组。
\end{corollary}

\begin{theorem}
    {多维子空间搜索问题}{}
    记$y\in\lspan(p^{(0)},\ldots,p^{(k-1)})$,$x=x^{(0)}+y+\alpha p^{(k)}$,
    \[
        \varphi(x)=\varphi(x^{(0)}+y)+\alpha\inp{y}{p^{(k)}}_A-\alpha\inp{r^{(0)}}{p^{(k)}}+\frac{\alpha^2}2\norm{p^{(k)}}^2_A.
    \]
    上式出现了交叉项$\inp{y}{p^{(k)}}_A$,为了简化问题,自然可考虑使交叉项$\inp{y}{p^{(k)}}_A=0$,这要求$p^{(k)}$与$\lspan(p^{(0)},\ldots,p^{(k-1)})$是$A$ - 正交的。%(见\dfnref{dfn:A-orthogonal})

    由此便分解成了两个极小问题:
    \[
        \min_x\varphi(x)=\min_y\varphi(x^{(0)}+y)+\min_\alpha\biggfkh{-\alpha\inp{r^{(0)}}{p^{(k)}}+\frac{\alpha^2}2\norm{p^{(k)}}^2_A}
    \]
    前者满足$x^{(0)}+y^*=x^{(k)}$,后者得到
    \begin{equation}
        \alpha_k=\frac{\inp{r^{(0)}}{p^{(k)}}}{\norm{p^{(k)}}_A^2}=\frac{\inp{r^{(k)}}{p^{(k)}}}{\norm{p^{(k)}}_A^2}.
    \end{equation}
    由于$\inp{r^{(0)}}{p^{(k)}}=\inp{r^{(k)}}{p^{(k)}}$,故形式其实与\eqref{eqn:alphak}相同。
    并且有
    \begin{equation}
        \label{eqn:r^k+1=r^k-alpha_kAp^k}
        r^{(k+1)}=r^{(k)}-\alpha_kAp^{(k)}.
    \end{equation}
\end{theorem}

\begin{corollary}
    由子空间搜索\thmref{thm:subspace search},$r^{(k)}\perp\lspan(p^{(0)},\ldots,p^{(k-1)})$。
\end{corollary}

\begin{remark}
    与$p^{(0)},\ldots,p^{(k-1)}$ $A$ - 正交的$p^{(k)}$取法并不唯一。
\end{remark}

\paragraph{共轭梯度法}

当$r^{(k)}\neq 0$时,$p^{(0)},\ldots,p^{(k-1)},r^{(k)}$线性无关,可通过Gram-Schmidt正交化得到$p^{(k)}$:
\begin{equation}
    \label{eqn:Gram-Schmidt p^k}
    p^{(k)}=r^{(k)}-\sum_{i=0}^{k-1}\frac{\inp{r^{(k)}}{p^{(i)}}_A}{\norm{p^{(i)}}_A^2}p^{(i)},
    \tag{*}
\end{equation}

\begin{theorem}
    {}{}
    $r^{(k)}\perp_A\lspan(p^{(0)},\ldots,p^{(k-2)})$。
\end{theorem}

\begin{proof}
    $r^{(k)}\perp\lspan(p^{(0)},\ldots,p^{(k-1)})=\lspan(r^{(0)},\ldots,r^{(k-1)})$,由式\eqref{eqn:r^k+1=r^k-alpha_kAp^k}
    \[
        Ap^{(i)}=\frac{r^{(i+1)}-r^{(i)}}{\alpha_i}\perp r^{(k)},\enspace i=0,\ldots,k-2.
        \qedhere
    \]
\end{proof}

\begin{corollary}
    $p^{(k)}$为$r^{(k)}$和$p^{(k-1)}$的线性组合,式\eqref{eqn:Gram-Schmidt p^k}变为
    \[
        p^{(k)}=r^{(k)}-\frac{\inp{r^{(k)}}{p^{(k-1)}}_A}{\norm{p^{(k-1)}}_A^2}p^{(k-1)}=:r^{(k)}+\beta_{k-1}p^{(k-1)}.
    \]
    可以简化系数$\beta_k$的计算:
    \[
        \beta_k=-\frac{\inp{r^{(k+1)}}{p^{(k)}}_A}{\norm{p^{(k)}}_A^2}=\frac{\norm{r^{(k+1)}}^2}{\alpha_k\norm{p^{(k)}}_A^2}=\frac{\norm{r^{(k+1)}}^2}{\norm{r^{(k)}}^2}.
    \]
\end{corollary}

\begin{theorem}
    {共轭梯度法}{conjugate gradient method}
    共轭梯度法(conjugate gradient method):$p^{(0)}=r^{(0)}=b-Ax^{(0)}$,
    \begin{subequations}
        \begin{align}
            % r^{(k)}&=b-Ax^{(k)},\\
            \alpha_k&=\frac{\norm{r^{(k)}}^2}{\norm{p^{(k)}}_A^2},\\
            x^{(k+1)}&=x^{(k)}+\alpha_kp^{(k)},\\
            r^{(k+1)}&=r^{(k)}-\alpha_kAp^{(k)};\\
            \beta_k&=\frac{\norm{r^{(k+1)}}^2}{\norm{r^{(k)}}^2},\\
            p^{(k+1)}&=r^{(k+1)}+\beta_kp^{(k)}.
        \end{align}
    \end{subequations}
    若$r^{(k)}=0$,则$x^{(k)}=x^*$,计算终止。
\end{theorem}

\begin{corollary}
    由于$r^{(0)},\ldots,r^{(k)}$两两正交,理论上最多$n=\dim(V)$步便可以得到精确解。
\end{corollary}

\begin{theorem}
    {梯度下降法的误差收敛性}{}
    误差$e^{(k)}:=x^{(k)}-x^*$的$A$ - 范数满足
    \begin{equation}
        \label{eqn:CG e^k}
        \norm{e^{(k)}}_A\leq 2\biggkh{\frac{\sqrt{\cond_2(A)}-1}{\sqrt{\cond_2(A)}+1}}^k\norm{e^{(0)}}_A.
    \end{equation}
    其中$\lambda\maxi,\lambda\mini$分别是对称正定算子$A$的最大最小特征值。
\end{theorem}

\begin{proof}
    记$\alpha=\lambda\mini/\lambda\maxi$,取
    \[
        p_k^*(x)=\division{T_k\biggkh{\frac{2x-(1+\alpha)}{1-\alpha}}}{T_k\biggkh{-\frac{1+\alpha}{1-\alpha}}},
    \]
    其中$T_k$是Chebyshev多项式,由于$\abs t\geq 1$时,
    \[
        T_k(t)=\frac12\biggfkh{\Bigkh{t+\sqrt{t^2-1}}^k+\Bigkh{t-\sqrt{t^2-1}}^k}\gtrapprox \frac12\Bigkh{\abs t+\sqrt{t^2-1}}^k.
    \]
    则$\forall x\in(\alpha,1)$,
    \[
        \norm{p_k^*}_\infty=\division{1}{T_k\biggkh{-\frac{1+\alpha}{1-\alpha}}}\leq 2\biggkh{\frac{1+\alpha}{1-\alpha}-\sqrt{\Bigkh{\frac{1+\alpha}{1-\alpha}}^2-1}}^k=2\biggkh{\frac{1-\sqrt\alpha}{1+\sqrt\alpha}}^k.
        \qedhere
    \]
\end{proof}

\subsection{共轭梯度法的预处理}
\label{ssec:preconditioning}

从误差估计\eqref{eqn:CG e^k}可以看出,共轭梯度法的收敛速度强烈依赖于$A$的条件数,当$A$病态时收敛得很慢。为了改善收敛速度,可以先进行预处理(preconditioning),降低矩阵的条件数。

\begin{theorem}
    {预处理的共轭梯度算法}{preconditioning conjugate gradient}
    给定预处理矩阵$M$且对称正定,
    \[
        Ax=b\iff M\iv Ax=M\iv b,
    \]
    对预处理后的方程应用共轭梯度算法:$p^{(0)}=M\iv r^{(0)}$,
    \begin{subequations}
        \begin{align}
            % r^{(k)}&=b-Ax^{(k)},\\
            \alpha_k&=\frac{\norm{r^{(k)}}_{M\iv}^2}{\norm{p^{(k)}}_A^2},\\
            x^{(k+1)}&=x^{(k)}+\alpha_kp^{(k)},\\
            r^{(k+1)}&=r^{(k)}-\alpha_kAp^{(k)};\\
            \beta_k&=\frac{\norm{r^{(k+1)}}_{M\iv}^2}{\norm{r^{(k)}}_{M\iv}^2},\\
            p^{(k+1)}&=M\iv r^{(k+1)}+\beta_kp^{(k)}.
        \end{align}
    \end{subequations}
    当$r^{(k)}=0$时计算终止。
\end{theorem}

\begin{corollary}
    其误差仍满足式\eqref{eqn:CG e^k},但条件数改善为$\cond_2(M\iv A)$。
\end{corollary}

\begin{remark}
    预处理矩阵的选择依据:
    方便计算、
    好近似。
    \begin{itemize}
        \item 代数方法:分解$A=M+N$,其中$M=LL\tp$对称正定、$L$尽可能稀疏,而$N$尽可能小,称为$A$的不完全Cholesky分解。
        
        比如取为对角部分$M=D$,在$A$的对角元素相差较大时收敛速度会大大提高。
        \item 分析方法:谱等价,$\forall x$
        \[
            c_1\norm x_A^2\leq\norm x_M^2\leq c_2\norm x_A^2,
        \]
        则$\rho(M\iv A)\in[c_2\iv,c_1\iv]$。
    \end{itemize}
\end{remark}