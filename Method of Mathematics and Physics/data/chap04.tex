\chapter{基本解方法}
\section{广义函数}
\begin{definition}{$\delta$函数}{delta function}
	源于物理中对集中分布物理量的数学描述。满足
	\[
		\int_a^bf(x)\delta(x)\d x=
		\begin{cases}
			f(x),&0\in(a,b)\\
			0,&0\notin(a,b)
		\end{cases}
	\]
	因此$f\ast\delta(\xi)=f(\xi)$。
\end{definition}
\begin{theorem}{$\delta$复合函数}{}
	设$u(x)\in\sC(\RR)$且在实轴上只有单零点$x_1,x_2,\ldots,x_n$,则 
	\begin{align}
		\delta(u(x))=\sum_{k=1}^n\frac{\vd(x-x_k)}{\abs{u'(x_k)}}.
	\end{align}
\end{theorem}
比如$\delta(ax)=\delta(x)/a$。
\paragraph{广义函数}\hspace{4ex}
\begin{definition}{线性泛函}{Linear Functional}
	函数空间$\mathscr V$到数域$\mathbb F$的映射$\mathcal T$,若
	\[
		\mathcal T(au+bv)=a\mathcal Tu+b\mathcal Tv
	\]
	$\forall a,b\in\mathbb F,\;u,v\in\mathscr V$均成立,则$\mathcal T$是$\mathscr V$上的线性泛函,即广义函数。$\mathscr V$上所有线性泛函构成一个线性空间,称作$\mathscr V$的对偶空间$\mathscr V^*$。
\end{definition}
\begin{example}{支撑集}{support set}
	记实函数$f$的支撑集
	\begin{align}
		\supp f:=\set x{f(x)\neq 0}.
	\end{align}
	并记
	\[
		\mathscr L_0(\RR):=\set f{\int\iti\abs{f(x)}\d x<\infty,\;\supp f\,\text{有界}}
	\]
	则$\forall f\in\mathscr L_0(\RR)$,$f$确定了$\sC(\RR)$上一个线性泛函:
	\begin{align}
		\mathcal T_f\bigfkh{\varphi(x)}\equiv\inp f\varphi:=\int\iti f(x)\varphi(x)\d x\in\RR.
	\end{align}
\end{example}
\begin{theorem}{广义函数的导数}{}
	定义函数空间
	\begin{align}
		\Cooo(\RR)\equiv\sC_0^\infty(\RR):=\set f{f\in\sC^\infty(\RR),\;\supp f\,\text{有界}}
	\end{align}
	广义函数$f\in\Cooo^\ast(\RR)$,定义其$n$阶导数 
	\begin{align}
		\inp{f^{(n)}}\varphi:=(-1)^n\inp f{\varphi^{(n)}},\quad\forall\varphi\in\Cooo(\RR).
	\end{align}
	特别地, 
	\[
		\int\iti\delta^{(n)}(x)\varphi(x)\d x=(-1)^n\varphi^{(n)}(0).
	\]
	
	泛函意义下,任意一个广义函数都是无穷阶可导的。
\end{theorem}
\begin{theorem}{广义函数的卷积}{}
	给定$f,g\in\Cooo^*(\RR)$的卷积$f\ast g(x)$也是广义函数
	\begin{align}
		\inp {f\ast g}\varphi:=\Bigl\langle f(x),\bigl\langle g(y),\varphi(x+y)\bigr\rangle\Bigr\rangle.
	\end{align}
	一般地,对常系数微分算子$\Dif$
	\[
		\Dif(f\ast g)=\Dif f\ast g=f\ast\Dif g.
	\]
\end{theorem}
\begin{theorem}{广义函数的Fourier变换}{}
	速降函数空间$\Schwsp(\RR)\subset\sC^\infty(\RR)$,%$\forall\varphi\in\Schwsp(\RR)$,有
	\begin{align}
		\Schwsp(\RR):=\set\varphi{\lim_{x\to\infty}x^m\varphi^{(n)}(x)=0,\quad\forall m,n\in\ZZ^+}.
	\end{align}
	比如$\varphi(x)=p(x)\e{-ax^2}$,$p(x)$是多项式,$a>0$,则$\varphi(x)\in\Schwsp(\RR)$。
	
	给定广义函数$f\in\Schwsp^*$,$f$的Fourier变换及其逆变换也是广义函数,定义为
	\begin{align}
		\begin{aligned}
			\inp{\cf f}\varphi&=\inp f{\cf\varphi}\\
			\inp{\cfi f}\varphi&=\inp f{\cfi\varphi}
		\end{aligned},\quad\forall\varphi\in\Schwsp.
	\end{align}
	Fourier变换作用转移到基本函数上,它们保持着经典意义下的基本性质。
\end{theorem}
\begin{theorem}{广义函数序列的收敛性}{}
	设广义函数列$\{f_i\}$及广义函数$f$,若对基本函数空间的$\varphi$都有
	\[
		\lim_{n\to\infty}\inp{f_n}\varphi=\inp f\varphi.
	\]
	则称$f_n$弱收敛到$f$,本笔记记作$\lim_{n\to\infty}f_n(x)\circeq f(x)$。
\end{theorem}
\begin{theorem}{}{}
	若基本函数空间为$\Schwsp$,则
	\[
		\lim_{n\to\infty}f_n(x)\circeq f(x)\iff\lim_{n\to\infty}f_n'(x)\circeq f'(x).
	\]
\end{theorem}
\begin{theorem}{高维广义函数}{}
	与一维类似,函数空间
	\begin{gather*}
		\Cooo(\RR^n)\equiv\sC_0^\infty(\RR^n):=\set f{f\in\sC^\infty(\RR^n),\;\supp f\,\text{有界}},\\
		\Schwsp(\RR^n):=\set\varphi{\lim_{\abs X\to\infty}\abs X^m\frac{\p^k\varphi(X)}{\p x_1^{k_1}\cdots\p x_n^{k_n}}=0,\quad\forall m,k\in\ZZ^+}
	\end{gather*}
	对函数空间$\mathscr V$上的线性泛函$f$
	\[
		\inp f\varphi=\int_{\RR^n}f(X)\varphi(X)\d X,\quad\forall\varphi\in\mathscr V
	\]
	构成$\mathscr V^\ast$。
	
	广义函数$f\in\Cooo^\ast(\RR^n)$的偏导数定义为
	\[
		\inp{\frac{\p^kf}{\p x_1^{k_1}\cdots\p x_n^{k_n}}}\varphi:=(-1)^k\inp f{\frac{\p^k\varphi}{\p x_1^{k_1}\cdots\p x_n^{k_n}}}
	\]
	特别地,$f=\delta$时
	\[
		\inp{\frac{\p^k\delta}{\p x_1^{k_1}\cdots\p x_n^{k_n}}}\varphi=(-1)^k\frac{\p^k\varphi(\mathbf 0)}{\p x_1^{k_1}\cdots\p x_n^{k_n}}.
	\]
\end{theorem}
\begin{example}{$\delta$函数与Laplace算子}{}
	\begin{equation}
		\delta(x_1,\ldots,x_n)=\begin{cases}
			\frac1{2\pi}\lap_2\ln r,&n=2\\
			-\frac1{(n-2)S_{n}}\lap_n\frac1{r^{n-2}},&n\geqslant 3
		\end{cases}
	\end{equation}
	其中$r=\abs{\bm x}$,$S_n$为$n$维单位球的表面积。特别地,$S_3=4\pi$。
\end{example}
广义函数的卷积
\[
	\inp {f\ast g}\varphi:=\Bigl\langle f(X),\bigl\langle g(Y),\varphi(X+Y)\bigr\rangle\Bigr\rangle.
\]

\section{\textit{Pu} = 0型方程的基本解} % {$\$}
讨论用基本解方法求解方程
\[
	\Par u(M)=f(M),\quad M\in\RR^n,
\]
其中,$\Par$是常系数线性偏微分算子。

视$f,u$为广义函数,它们在广义函数空间里可以自由地进行各种运算和交换。通过这种方式得到的解叫广义函数解,简称作\textbf{广义解}。如果解是一个正则广义函数,甚至还有足够的光滑性,那么这种解是经典解。
\begin{definition}{$\Par u=0$型方程的基本解}{Basic Solution of Pu=0}
	方程
	\begin{align}
		\Par U(M)=\vd(M)
	\end{align}
	的解$U(M)$称作$\Par u=0$型方程的基本解。
\end{definition}
对一般的函数$f$,其对应的解是一般的源所产生的物理场。故基本解也叫点源函数。

若$\Par u=0$型方程有基本解$U(M)$,令
\begin{align}
	u(M):=U\ast f(M)=\int U(M-N)f(N)\d N.
\end{align}
由积分叠加原理
\[
	\Par u(M)=\int\Par U(M-N)f(N)\d N=\int\delta(M-N)f(N)\d N=f(M).
\]
即$u$满足方程$\Par u=f$。
\begin{example}{}{}
	求下面方程的基本解:
	\[
		y'+ay=f(x),\quad a>0
	\]
	
	基本解$U$满足
	\[
		U'+aU=\delta(x),
	\]
	即
	\[
		\dd x(U\e{ax})=\vd(x)\e{ax}=\delta(x),\implies U(x)=\Heaviside(x)\e{-ax}.
	\]
	因此原方程的解
	\[
		y(x)=U\ast f(x)=\int\iti\Heaviside(\xi)\e{-a\xi}f(x-\xi)\d\xi=\int_{-\infty}^xf(\xi)\e{-a(x-\xi)}\d\xi.
	\]
\end{example}
\begin{example}{3维Helmholtz方程}{}
	求3维Helmholtz方程的基本解:
	\[
		\lap_3 u+cu=0
	\]
	
	由于只有一个点源,方程具有对称性,可设方程有球对称基本解$U(r)$,
	当$r>0$时,$\delta(r)=0$,
	\[
		\frac1{r^2}\dd r\kh{r^2\dv Ur}+cU=\frac1r\fkh{\dd[2]r(rU)+crU}=0,
	\]
	
	当$c=0$时,化为Laplace方程
	\[
		(rU)''=0,\implies U(r)=\frac Ar+\cancel{B}.
	\]
	记$B_r$为半径为$r$的球内部分,则
	\[
		\int_{B_r}\lap U\d V=\int_{B_r}\delta(x,y,z)\d V=1.
	\]
	由Green公式
	\[
		\int_{B_r}\lap U\d V=\oint_{\p B_r}\pv Un\d S=-\frac A{r^2}\cdot 4\pi r^2=1,\implies U(r)=-\frac1{4\pi r}.
	\]
	
	当$c<0$时,记$k:=\sqrt{-c}$
	\[
		(rU)''-k^2rU=0,\implies U=A\frac{\e{-kr}}r+B\frac{\e{kr}}r.
	\]
	由积分条件
	\[
		\int_{B_r}\lap U-k^2U\d V=\oint_{\p B_r}\pv Un\d S-k^2\int_{B_r}U\d V=1,
	\]
	且
	\begin{align*}
		&\quad\oint_{\p B_r}\kh{\pp\rho\frac{\e{k\rho}}\rho}_rr^2\sin\theta\d\theta\nd\phi-k^2\int_{B_r}\frac{\e{k\rho}}\rho\rho^2\sin\theta\d\rho\nd\theta\nd\phi\\
		&=4\pi(kr-1)\e{kr}-4\pi\fkh{(k\rho-1)\e{k\rho}}_0^r=-4\pi.
	\end{align*}
	故$-4\pi(A+B)=1$,即
	\[
		U(r)=-\frac{\e{-kr}}{4\pi r},\;-\frac{\e{kr}}{4\pi r}.
	\]
	
	当$c>0$时,记$k:=\sqrt c$
	\[
		U=A\frac{\cos kr}r+B\cancel{\frac{\sin kr}r}.
	\]
	$\sin kr/r$在$\CC$上是整函数,无奇异性,不能作为基本解。
	积分$-4\pi A=1$
	\[
		U(r)=-\frac{\cos kr}{4\pi r}.
	\]
\end{example}
\section{Possion方程Green函数法}
首先引入Green公式。
\begin{theorem}{Green公式}{Green Formula}
	设非空有界开集$\Omega\subset\RR^n(n\geqslant 2)$满足边界$\p\Omega$光滑,则
	
	$\forall u,v\in\sC^2(\Omega)\cap\sC(\bar\Omega)$,有
	\begin{gather}
		\label{Green's 1st formula}
		\int_\Omega v\lap u\d V=\oint_{\p\Omega}v\pv u{\bm n}\d S-\int_\Omega\nabla v\cdot\nabla u\d V;\\
		\label{Green's 2nd formula}
		\int_\Omega(u\lap v-v\lap u)\d V=\oint_{\p\Omega}\biggkh{u\pv v{\bm n}-v\pv u{\bm n}}\d S.
	\end{gather}
	%其中$\p u/\p\bm n=\nabla u\cdot\bm n$。
\end{theorem}
\begin{proof}
	由散度定理
	\begin{align}
		\int_\Omega\nabla\cdot\bm A\d V=\oint_{\p\Omega}\bm A\cdot\bm n\d S,
	\end{align}
	取$\bm A=v\nabla u$,
	可得式\eqref{Green's 1st formula}:
	\[
		\oint_{\p\Omega}v\pv u{\bm n}\d S\equiv\oint_{\p\Omega}v\nabla u\cdot\bm n\d S=\int_\Omega\div(v\nabla u)\d V=\int_\Omega(\nabla v\cdot\nabla u+v\nabla^2u)\d V.
	\]
	将$u,v$互换位置并与原式相减,即得式\eqref{Green's 2nd formula}。
\end{proof}
\paragraph{Poisson方程第I边值问题}
\begin{align}
	\begin{cases}
		\lap u=-f(M),&M\in V\subseteq\RR^3\\
		\edg u_{\p V}=\varphi(M),
	\end{cases}
\end{align}
物理上看,这是静电场的基本问题:空间区域$V$内有电荷体密度$\rho=-\varepsilon f$,边界上电位已知为$\varphi$,求$V$内电位$u$。

由叠加原理,$u=v+w$,$v,w$分别满足
\begin{align*}
	\begin{cases}
		\lap v=-f(M),\\
		\edg v_{\p V}=0,
	\end{cases}\quad
	\begin{cases}
		\lap w=0,\\
		\edg w_{\p V}=\varphi(M),
	\end{cases}
\end{align*}
其中$v$表示在边界接地条件下体内电荷产生的电场,$w$表示由边界约束引起的电场。
\begin{definition}{Poisson第I边值问题的Green函数}{}
	定解问题
	\begin{align}
		\begin{cases}
			\lap G(M;N)=-\vd(M-N),\\
			\edg G_{\p V}=0.
		\end{cases}
	\end{align}
	的解$G(M;N)$称为Poisson第I边值问题的Green函数。
\end{definition}
物理上看,Green函数就是边界接地条件下,置于$V$内$N$点电荷为$+\varepsilon$的点源在$V$内$M$点产生的电场,仍然是一个基本解。
\begin{align}
	u(M)=\int_Vf(N)G(M;N)\d N-\oint_{\p V}\varphi(N)\pv Gn\d S.
\end{align}
\begin{proof}
	\begin{align*}
		u(M)&=\int_Vu(N)\vd(M-N)\d N=-\int_Vu(N)\,\D G(M;N)\d N\\
		&=-\int_V u\,\D G\d N+\int_VG\,\D u\d N-\int_VG\,\underline{\D u}\d N\tag{Green-II}\\
		&=\oint_{\p V}\cancel{G}\pv un-\underline{u}\pv Gn\d S+\int_VG f\d N\\
		&=-\oint_{\p V}\psi\pv Gn\d S+\int_VGf\d N.
	\end{align*}
\end{proof}
Fourier方法是求Green函数的基本方法,但对于一些特殊的区域,可以采用一些特殊方法,如镜像法。
\begin{method}{镜像法}{}
	分解$G=U+g$,$U$满足$\lap U=-\vd(M-N)$,可取
	\begin{align*}
		U=\begin{cases}
			-\frac1{2\pi}\ln r,&n=2\\[1ex]
			\frac1{4\pi r},&n= 3
		\end{cases}
	\end{align*}
	%其中$r$是$M,N$间距离。
	而$g$满足 
	\[
		\begin{cases}
			\lap g=0,\\
			\edg g_{\p V}=-U,
		\end{cases}
	\]
	是$N$的点电荷在边界上的感应电荷产生的电场。
	
	区域外的点源在$V$内产生的电场满足Laplace方程,可以将边界感应电荷产生的电场$g$看作区域\accentd{外}某些虚设电荷产生的等效电场,这种来源于物理效应的方法叫镜像法。% 它的关键困难在于如何在区域外合适地虚设电荷,对应某些特殊的区域如半空间、球域等等,可以用较直观的方法找到。
\end{method}
\begin{example}{镜像法应用·半空间}{}
	求上半空间Poisson方程第I边值问题的Green函数。
	
	$V$内$N(\xi,\eta,\zeta)$点的正电荷$\varepsilon$在空间$(x,y,z)$产生的电场为
	\[
		U_0=\frac1{4\pi r}=\frac1{4\pi\sqrt{(x-\xi)^2+(y-\eta)^2+(z-\zeta)^2}},
	\]
	可虚设电荷$-\varepsilon$于$N$关于$z=0$平面对称的点$M_1(\xi,\eta,-\zeta)$,产生的电场
	\[
		U_1=\frac{-1}{4\pi\sqrt{(x-\xi)^2+(y-\eta)^2+(z+\zeta)^2}},
	\]
	在边界上$\edg{U_0}_{z=0}=-\edg{U_1}_{z=0}$
    \begin{align*}
        G=\frac1{4\pi}\bigg[&\frac1{\sqrt{(x-\xi)^2+(y-\eta)^2+(z-\zeta)^2}}-\\
        &\qquad\frac1{\sqrt{(x-\xi)^2+(y-\eta)^2+(z+\zeta)^2}}\bigg],
    \end{align*}
	边界方向导数
	\[
		\edg{\pv Gn}_{\zeta=0}=\edg{-\pv G\zeta}_{\zeta=0}=\frac{-z}{2\pi\fkh{(x-\xi)^2+(y-\eta)^2+z^2}^{3/2}}
	\]
	故对于上半空间Dirichlet问题
	\[
		\begin{cases}
		\lap u=0,&z>0\\
		\edg u_{z=0}=\varphi(x,y)
	\end{cases}
	\]
	解的Poisson公式
	\[
		u(x,y,z)=\frac z{2\pi}\int_{\RR^2}\frac{\varphi(\xi,\eta)\d\xi\nd\eta}{\fkh{(x-\xi)^2+(y-\eta)^2+z^2}^{3/2}}
	\]
	\tcblower
	二维情况
	\[
		G=\frac1{4\pi}\ln\frac{(x-\xi)^2+(y+\eta)^2}{(x-\xi)^2+(y-\eta)^2},\quad u(x,y)=\frac y\pi\int\iti\frac{\varphi(\xi)\d\xi}{(x-\xi)^2+y^2}
	\]
\end{example}

\begin{example}{镜像法应用·球域}{}
	求半径为$R$的球域内Poisson方程第I边值问题的Green函数
	\begin{gather*}
		\begin{cases}
			\lap_3G(M;N)=-\vd(M-N),&0\leqslant r<R\\
			\edg G_{r=R}=0
		\end{cases}\\
		\implies \edg{\pv Gn}_{\rho=R}=\frac{R^2-r^2}{4\pi R(R^2+r^2-2Rr\cos\psi)^{3/2}}.
	\end{gather*}
	故球内Dirichlet问题,解的Poisson形式
	\begin{gather*}
		\begin{cases}
			\lap_3u=0,&0\leqslant r<R\\
			\edg u_{r=R}=\varphi(\theta,\phi)
		\end{cases}\\
		\implies u(r,\theta,\phi)=\frac1{4\pi R}\oint_{S_R}\frac{(R^2-r^2)\varphi(\theta_0,\phi_0)\d S_0}{(R^2+r^2-2Rr\cos\psi)^{3/2}}.
	\end{gather*}
\end{example}

\subparagraph*{Fourier法}
Fourier法是求Green函数的基本方法,主要思想是按照特征函数作广义Fourier展开,包括分离变量与积分变换。
\begin{example}{Fourier法应用}{}	
	求矩形区域$\Omega=[0,L]\times[0,M]$上第I类边值Poisson方程的Green函数
	\begin{align*}
		\begin{cases}
			\lap_2G(x,y;\xi,\eta)=-\vd(x-\xi,y-\eta),\quad x,\xi\in[0,L];\;y,\eta\in[0,M]\\
			G(0,y)=G(L,y)=G(x,0)=G(x,M)=0,
		\end{cases}
	\end{align*}
	有特征值 
	\[
		G(x,y)=\sum_{m,n}C_{mn}\sin\frac{m\pi x}L\sin\frac{n\pi y}M.
	\]
	系数 
	{\footnotesize
	\begin{align*}
		\fkh{\kh{\frac{m\pi}L}^2+\kh{\frac{n\pi}M}^2}C_{mn}&=\frac4{LM}\int_0^M\int_0^L\vd(x-\xi,y-\eta)\sin\frac{m\pi x}L\sin\frac{n\pi y}M\d x\nd y\\
		&=\frac4{LM}\sin\frac{m\pi\xi}L\sin\frac{n\pi\eta}M.
	\end{align*}}
	后略。
\end{example}
\begin{example}{Fourier法应用·有限高}{}
	\begin{equation*}
		\begin{cases}
			\lap_2u=0,\quad x>0,y\in[0,a]\\
			u(x,0)=\varphi(x),\enspace u(x,a)=\psi(x),\\
			u(0,y)=0,
		\end{cases}
	\end{equation*}
	有
	\begin{align*}
		u(x,y)&=-\fkh{\int\zti\varphi(\xi)\edg{\pv Gn}_{\eta=0}\d\xi+\int\zti\psi(\xi)\edg{\pv Gn}_{\eta=a}\d\xi}\\
		&=\int\zti\varphi(\xi)\edg{\pv G\eta}_{\eta=0}\d\xi-\int\zti\psi(\xi)\edg{\pv G\eta}_{\eta=a}\d\xi
	\end{align*}
	进而
	\begin{align*}
		\pv G\eta&=\sum_{n=1}^\infty4n\int\zti\frac{\sin\omega\xi\sin\omega x}{n^2\pi^2+a^2\omega^2}\d\omega\cdot\cos\frac{n\pi\eta}a\cos\frac{n\pi y}a;\\
		u(x,y)&=\sum_{n=1}^\infty4n\int\zti\int\zti\frac{\sin\omega\xi\sin\omega x}{n^2\pi^2+a^2\omega^2}\fkh{\varphi(\xi)-(-1)^n\psi(\xi)}\d\omega\nd\xi\cdot\cos\frac{n\pi y}a.
	\end{align*}
\end{example}
\paragraph{$^\ast$Poisson方程第II, III边值问题}
\begin{align}
	\begin{cases}
		\lap u=-f(M),&M\in V\subseteq\RR^n\\
		\kh{\alpha u+\beta\pv un}_{\p V}=\varphi(M),&\alpha,\beta\neq 0.
	\end{cases}
\end{align}
相应的Green函数
\begin{align*}
	\begin{cases}
		\lap G=-\vd(M-N),&M,N\in V\subseteq\RR^n\\
		\kh{\alpha G+\beta\pv Gn}_{\p V}=0,&\alpha,\beta\neq 0.
	\end{cases}
\end{align*}
原问题的解
\begin{align}
	u(M)&=\int_Vf(N)G(M;N)\d N+\frac1\beta\oint_{\p V}\varphi(N)G(M;N)\d S\\
	&=\int_Vf(N)G(M;N)\d N-\frac1\alpha\oint_{\p V}\varphi(N)\pv Gn\d S
\end{align}
$\beta=0$即第I边值问题。$\alpha=0$即第II边值问题,但此时表示内部有热源而边界绝热的稳恒温度场,这是不可能的,因此需要修正为
\begin{align*}
	\begin{cases}
		\lap G=-\vd(M-N)+\frac1v,\\
		\edg{\pv Gn}_{\p V}=0,
	\end{cases}\enspace\text{或}\quad
	\begin{cases}
		\lap G=-\vd(M-N),\\
		\edg{\pv Gn}_{\p V}=-\frac1s,
	\end{cases}
\end{align*}
其中$v,s$为$V$的体积和表面积。有解
\begin{align}
	u(M)=\int_Vf(N)G(M;N)\d N+\oint_{\p V}\varphi(N)G(M;N)\d S+\const.
\end{align}
相容性条件
\begin{align}
	\int_Vf(M)\d M+\oint_{\p V}\varphi(M)\d S=0.
\end{align}
\section{初值问题的基本解方法}
本节主要用基本解方法来求解发展方程,如$u_t=\Par u$型方程
\begin{align}
	\begin{cases}
		u_t=\Par u+f(M,t),&M\in\RR^n,t>0\\
		u(M,0)=\varphi(M).
	\end{cases}
\end{align}
这里所涉及的$\Par$是关于空间变量$M$的常系数线性偏微分算子。

\begin{definition}{$u_t=\Par u$型方程初值问题的基本解}{Basic Solution of ut=Pu}
	基本解$U(M,t)$满足
	\begin{align}
		\begin{cases}
			U_t=\Par U,&M\in\RR^n,t>0\\
			U(M,0)=\vd(M).
		\end{cases}
	\end{align}
\end{definition}
有
\begin{align}
	u(M,t)=U(M,t)\ast\varphi(M)+\int_0^tU(M,t-\tau)\ast f(M,\tau)\d\tau.
\end{align}
证明过程略。

以及$u_{tt}=\Par u$型方程
\begin{align}
	\begin{cases}
		u_{tt}=\Par u+f(M,t),&M\in\RR^n,t>0\\
		u(M,0)=\varphi(M),\enspace u_t(M,0)=\psi(M).
	\end{cases}
\end{align}
\begin{definition}{$u_{tt}=\Par u$型方程初值问题的基本解}{Basic Solution of utt=Pu}
	基本解$U(M,t)$满足
	\begin{align}
		\begin{cases}
			U_{tt}=\Par U,&M\in\RR^n,t>0\\
			U(M,0)=0,\enspace U_t(M,0)=\vd(M).
		\end{cases}
	\end{align}
\end{definition}
有
\begin{align}
	\begin{aligned}
		u(M,t)=&\;U\ast\psi+\pp t(U\ast\varphi)+\int_0^tU(M,t-\tau)\ast f(M,\tau)\d\tau.
	\end{aligned}
\end{align}
\paragraph{*混合问题的Green函数}对于发展方程的混合问题通常用分离变量法或积分变换法求解,当然也可以用Green函数(点源函数)法。
\begin{example}{一维波动方程的混合问题}{}
	一维波动方程的混合问题
	\begin{align*}
		\begin{cases}
			u_{tt}=a^2u_{xx}+f(x,t),\quad x\in(0,L),\;t>0\\
			u(0,t)=u(L,t)=0\\
			u(x,0)=\varphi(x),\quad u_t(x,0)=\psi(x)
		\end{cases}
	\end{align*}
	Green函数$G(x,t;\xi)$满足 
	\begin{align*}
		\begin{cases}
			G_{tt}=a^2G_{xx},\quad x,\xi\in(0,L),\;t>0\\
			G(0,t)=G(L,t)=0\\
			G(x,0)=0,\quad G_t(x,0)=\vd(x-\xi)
		\end{cases}
	\end{align*}
	利用Fourier方法可得
	\[
		G(x,t;\xi)=\sum_{n=1}^\infty\frac2{n\pi a}\sin\frac{n\pi\xi}L\sin\frac{n\pi x}L\sin\frac{n\pi at}L,
	\]
	进而解
	\begin{align}
		\begin{aligned}
			u(x,t)={}&\int_0^L\psi(\xi)G\d\xi+\pp t\int_0^L\varphi(\xi)G\d\xi+\\
			&\int_0^t\int_0^L f(\xi,\tau)G(x,t-\tau;\xi)\d\xi\nd\tau
		\end{aligned}
	\end{align}
\end{example}

% \input{Append.tex}
