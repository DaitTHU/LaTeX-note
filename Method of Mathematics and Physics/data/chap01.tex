\chapter{偏微分方程的定解问题}
% 我们先从$\bcancel{\text{三}}$两个具体问题出发推导典型方程。
\begin{example}{弦振动}{string vibration}
	线密度为$\rho$的弦在自身张力$T$的作用下平衡位置在$x$轴,横向位移$u=u(x,t)$,取$(x_1,x_2)$段进行受力分析
	\begin{align*}
		T_2\cos\theta_2-T_1\cos\theta_2&=0,\\
		T_2\sin\theta_2-T_1\sin\theta_2&=\rho\vd x\cdot\pv[2]ut.
	\end{align*}
	%其中$T$为弦内张力,为弦的线密度。
	限定$\theta\ll 1$,有$\cos\theta\doteq 1,\,\sin\theta\doteq\theta,$
	\begin{align*}
		T_1&=T_2=:T,\\
		T\vd\theta&=\rho\vd x\cdot\pv[2]ut.
	\end{align*}
	又$\theta=\p u/\p x$,得到波动方程
	% \nomenclature{$u_t$}{The number of angels per unit area\nomrefeq}
	%$\rho$ - 线密度,$T$ - 张力,$g$ - 外力\\
	\begin{align}
		\pv[2]ut=a^2\pv[2]ux,\quad a^2:=\frac T\rho.
	\end{align}
\end{example}
\begin{example}{热传导}{heat conduction}
	由Fourier热传导定律,一段时间内流入物体$\Omega$的热量为
	\[
		Q=\int_{t_1}^{t_2}\oint_{\p\Omega}k\pv u{\bm n}\d S\d t=\int_{t_1}^{t_2}\int_\Omega k\lap u\d V\d t,
	\]
	其中$k$为介质的热传导系数,$u=u(x,y,z,t)$为各点温度。
	
	另一方面,从比热容$c$的角度看,
	\[
		Q=\int_{t_1}^{t_2}\int_\Omega c\rho\pv ut\d V\d t.
	\]
	其中$\rho$为物体的体密度。
	
	因为上式对任意$\Omega\times(t_1,t_2)$均成立,得到热传导方程
	%$c$ - 比热,$\rho$ - 密度,$k$ - 热传导系数\\
	\begin{align}
		\pv ut=a^2\lap u,\quad a^2:=\frac k{c\rho}.
	\end{align}
\end{example}
\iffalse
	\begin{example}{Maxwell方程组}{}
		电动力学中
		\begin{align*}
			\nabla\cdot\bm D  & =\rho,              \\
			\nabla\cdot\bm B  & =0,                 \\
			\nabla\times\bm E & =-\pv{\bm B}t,      \\
			\nabla\times\bm H & =\bm J+\pv{\bm D}t.
		\end{align*}
	\end{example}
\fi
\section{定解问题及适定性}
$V\subseteq\mathbb R^n$内$m$阶PDE形如
\[
	F\kh{x,u,\pv u{x_1},\pv u{x_2},\ldots,\pv u{x_n},\ldots,\frac{\p^mu}{\p x_1^{m_1}\cdots\p x_n^{m_n}}}=0,
\]
其中$m=m_1+m_2+\cdots+m_n$。
\begin{definition}{定解条件}{Definite Condition}
	确定解中函数的条件称为定解条件,包括初始条件和边界条件
	\begin{compactenum}[I.]
		\item (Dirichlet)\quad$\edg u_{x=0}=f$,\qqqquad$\edg u_{\p V}=f$;
		\item (Neumann)\enspace$\edg{u_t}_{x=0}=f$,\qqqquad\!$\edg{u_t}_{\p V}=f$;
		\item (Robin)\qquad$\kh{u_t+\sigma u}_{x=0}=f$,\quad$\biggkh{\pv un+\sigma u}_{\p V}=f$.
	\end{compactenum}
\end{definition}
\begin{definition}{解}{}
	古典解:符合方程且各阶偏导数连续;

	通解:$m$阶PDE有$m$个任意函数的解;

	特解:不包含任何任意函数或任意常数的解;

	适定解:存在、唯一且稳定。
\end{definition}
\section{一阶线性方程的通解法}
\begin{method}{常数变易法}{constant variation method}
	对于$u(x,y)$的方程
	\[
		u_x+Ax=B.
	\]
	有解
	\begin{align*}
		\begin{cases}
			\varphi=\exp\biggkh{-\int A\d x},  \\
			\psi=\int B\varphi^{-1}\d x
		\end{cases}
		\implies u=\varphi\bigkh{\psi+g(y)}.
	\end{align*}
	其中$g\in\sC$。
\end{method}
\begin{method}{变量代换}
	对$u=u(x,y)$的方程
	\[
		au_x+bu_y+cu=f,
	\]
	作变量代换
	\begin{align*}
		\begin{cases}
			\xi=\xi(x,y)  \\
			\eta=\eta(x,y)
		\end{cases}
		\implies
		\begin{cases}
			u_x=u_\xi\xi_x+u_\eta\eta_x \\
			u_y=u_\xi\xi_y+u_\eta\eta_y
		\end{cases}
	\end{align*}
	变成$u=u(\xi,\eta)$的方程
	\[
		\kh{a\xi_x+b\xi_y}u_\xi+\kh{a\eta_x+b\eta_y}u_\eta+cu=f.
	\]
	使$a\xi_x+b\xi_y=0$,得到\textbf{特征方程}和\textbf{特征曲线}
	\begin{align}
		\frac{\d x}a=\frac{\d y}b,\implies\xi(x,y)=\const.
	\end{align}
	剩下
	\[
		\kh{a\eta_x+b\eta_y}u_\eta+cu=f.
	\]
	对$\eta$积分便可求出通解。注意:要$a\eta_x+b\eta_y\neq 0$,应有Jacobi行列式不为0
	\begin{align*}
		\det J(\xi,\eta)=\left\lvert\frac{\p(\xi,\eta)}{\p(x,y)}\right\rvert=
		\begin{vmatrix}
			\xi_x  & \xi_x  \\
			\eta_x & \eta_y
		\end{vmatrix}\neq 0.
	\end{align*}
\end{method}
\section{波动方程的行波解和d'Alembert公式}
一维波动方程
\[
	\pv[2]ut=a^2\pv[2]ux,\quad a>0.
\]
可分解为
\[
	\kh{\pp t+a\pp x}\kh{\pp t-a\pp x}u=0.
\]
等价于
\begin{align*}
	\begin{cases}
		\pv vt+a\pv vx=0, \\[1ex]
		\pv ut-a\pv ux=v.
	\end{cases}
	\implies
	\begin{cases}
		\xi=x-at, \\
		\eta=x+at.
	\end{cases}
\end{align*}
方程化为
\[
	\pw u\eta\xi=0,\implies u=f(\xi)+g(\eta).
\]
即\textbf{行波解}
\begin{align}
	u(x,t)=f(x-at)+g(x+at),\quad f,g\in\sC^2.
\end{align}
\begin{theorem}{d'Alembert公式}{d'Alembert formula}
	无限长弦自由振动
	\begin{align*}
		\begin{cases}
			u_{tt}=a^2u_{xx},&t>0,\;x\in\mathbb R. \\
			u|_{t=0}=\varphi(x),\quad u_t|_{t=0}=\psi(x).
		\end{cases}
	\end{align*}
	将初值问题带入通解
	\begin{align*}
		u|_{t=0}   & =f(x)+g(x)=\varphi(x),   \\
		u_t|_{t=0} & =-af'(x)+ag'(x)=\psi(x).
	\end{align*}
	因此
	\begin{align}
		u(x,t)=\frac12\bigfkh{\varphi(x-at)+\varphi(x+at)}+\frac1{2a}\int_{x-at}^{x+at}\psi(\xi)\d\xi.
	\end{align}
	对于$\varphi\in\sC^2,\psi\in\sC,$解是适定的。
\end{theorem}

非无限情况,可以延拓至无限。
\begin{example}{延拓}{}
	端点固定的半无限边界条件:
	\begin{align*}
		\begin{cases}
			u_{tt}=a^2u_{xx},\\
			u(0,t)=0,\\
			u(x,0)=\varphi(x),\quad u_t(x,0)=\psi(x).
		\end{cases}
	\end{align*}
	进行奇延拓
	\begin{align*}
		\varphi_\odd(x)=
		\begin{cases}
			\varphi(x),   & x\geqslant 0, \\
			-\varphi(-x), & x<0.
		\end{cases}
		\quad
		\psi_\odd(x)=
		\begin{cases}
			\psi(x),   & x\geqslant 0, \\
			-\psi(-x), & x<0.
		\end{cases}
	\end{align*}
	问题便适用d'Alembert公式,解
	\begin{align*}
		u(x,t) & =\frac12\bigfkh{\varphi_\odd(x-at)+\varphi_\odd(x+at)}+\frac1{2a}\int_{x-at}^{x+at}\psi_\odd(\xi)\d\xi \\
			   & = 
		\begin{cases}
			\frac12\bigfkh{\varphi(x-at)+\varphi(x+at)}+\frac1{2a}\int_{x-at}^{x+at}\psi(\xi)\d\xi, & x\geqslant at, \\
			\frac12\bigfkh{\varphi(x+at)-\varphi(at-x)}+\frac1{2a}\int_{at-x}^{x+at}\psi(\xi)\d\xi, & x<at.
		\end{cases}
	\end{align*}
	\tcblower
	端点自由半无限弦($u_t(0,t)=0$)采用偶延拓;有界弦:两端均延拓。
\end{example}
\begin{example}{中心对称球面波}{}
	中心对称的球面波$u=u(r)$,采用球坐标
	\[
		\lap=\frac1{r^2}\pp r\kh{r^2\pp r}+\frac1{r^2\sin\theta}\pp \theta\kh{\sin\theta\pp \theta}+\frac1{r^2\sin^2\theta}\pp[2]\phi.
	\]
	波动方程变为
	\[
		u_{tt}=a^2\biggkh{u_{rr}+\frac2ru_r}=\frac{a^2}r(ru)_{rr}.
	\]
	设$v=ru$,则可解得
	\[
		u(r)=\frac1r\bigfkh{f(r-at)+g(r+at)}.
	\]
	掺入边界条件$\varphi(r),\psi(r)$后,作奇延拓即可。
\end{example}
\begin{example}{三维波动方程与球面平均法}{}
	一般的三维波动方程
	\begin{align*}
		\begin{cases}
			u_{tt}=a^2(u_{xx}+u_{yy}+u_{zz}),                     \\
			u|_{t=0}=\varphi(x,y,z),\quad u_t|_{t=0}=\psi(x,y,z).
		\end{cases}
	\end{align*}
	采用球面平均法,定义$u$在以$M$为球心、$r$为半径的球面$S_r(M)$上的平均
	\[
		\overline u(r,t;M)=\frac1{4\pi r^2}\oint_{S_r(M)}u(x,y,z,t)\d S,
	\]
	
	略去证明过程,问题转化为中心对称球面波问题
	\begin{align*}
		\begin{cases}
			\overline u_{tt}=\frac1r\kh{r\overline u}_{rr},                                \\
			(r\overline u)_{r=0}=0,                                                        \\
			\overline u(r,0)=\overline\varphi(r),\quad\overline u_t(r,0)=\overline\psi(r).
		\end{cases}
	\end{align*}
	奇延拓解,注意:$(r\varphi)_\odd=r\varphi_\even,$
	\begin{align*}
		r\overline u(r,t)= & \frac12\bigfkh{(r-at)\overline\varphi_\even(r-at)+(r+at)\overline\varphi_\even(r+at)}+\frac1{2a}\int_{r-at}^{r+at}\rho\overline\psi_\even(\rho)\d\rho.
	\end{align*}
	为得到$u(x,y,z,t),$取极限
	\begin{align*}
		u(M,t)=\lim_{r\to 0^+}\overline u(r,t)
		=\dd t\kh{t\overline\varphi(at)}+t\overline\psi(at),
	\end{align*}
	上式称为Possion公式。
	\tcblower
	求解三维波动方程的关键在于计算球面平均
	\begin{align}\label{3D-Wave}
		&u(M,t)  =\pp t\kh{\frac1{4\pi a^2t}\oint_{S_{at}}\varphi\d S}+\frac1{4\pi a^2t}\oint_{S_{at}}\psi\d S\\\notag
		& =\pp t\kh{\frac1{4\pi}\int_0^{2\pi}\int_0^\pi t\varphi\sin\theta\d\theta\nd\phi}+\frac1{4\pi}\int_0^{2\pi}\int_0^\pi t\psi\sin\theta\d\theta\nd\phi,
	\end{align}
	%其中采用球坐标。
\end{example}

\iffalse
\begin{align*}
	\varphi\begin{pmatrix}
		\xi \\\eta\\\zeta
	\end{pmatrix}=\varphi\begin{pmatrix}
		x+at\sin\theta\cos\phi \\
		y+at\sin\theta\sin\phi \\
		z+at\cos\theta
	\end{pmatrix}.
\end{align*}
\fi
\begin{example}{二维波动方程与升维法}{}
	二维波动方程
	\begin{align*}
		\begin{cases}
			u_{tt}=a^2(u_{xx}+u_{yy}), \\
			u|_{t=0}=\varphi(x,y),\quad u_t|_{t=0}=\psi(x,y).
		\end{cases}
	\end{align*}
	采用升维法
	\begin{gather*}
		U(x,y,z,t)=u(x,y,t),\quad\varPhi(x,y,z)=\varphi(x,y),\\
		U(x,y,z,t)=\pp t\kh{\frac1{4\pi a^2t}\oint_{S_{at}}\varPhi\d S}+\frac1{4\pi a^2t}\oint_{S_{at}}\varPsi\d S
	\end{gather*}
	由于$U,\varPhi,\varPsi$取值与$z$无关,故积分区域可投影到$xy$平面上。
	记$D_r$是以$(x,y)$为圆心,$r$为半径的圆内区域,在$D_r$上
	\[
		\d S=\sqrt{1+\kh{\pv\zeta\xi}^2+\kh{\pv\zeta\eta}^2}\d A=\frac{r\d A}{\sqrt{r^2-(\xi-x)^2-(\eta-y)^2}}.
	\]
	故
	\begin{align}\label{2D-Wave}
		\begin{aligned}
			u(x,y,t)&= \frac1{2\pi a}\pp t\int_{D_{at}}\frac{\varphi(\xi,\eta)\d\xi\nd\eta}{\sqrt{a^2t^2-(\xi-x)^2-(\eta-y)^2}} \\
			&\qquad +\frac1{2\pi a}\int_{D_{at}}\frac{\psi(\xi,\eta)\d\xi\nd\eta}{\sqrt{a^2t^2-(\xi-x)^2-(\eta-y)^2}}.
		\end{aligned}
	\end{align}
\end{example}
\begin{remark}
	\eqref{3D-Wave}是在球面上积分,而\eqref{2D-Wave}是在圆域上积分。这个差别在物理上产生了截然不同的效果:对三维情况,波的传播既有清晰的前阵面,也有清晰的后阵面,可用于传播信号,这称为Huygens原理;对二维情况,波的传播有清晰的前阵面,但没有后阵面,这称为波的弥漫,或说这种波具有后效现象,不适合于传播信号。
\end{remark}
\begin{example}{一维波动方程}{}
	对一维弦振动方程,也可升二维,再将积分投影到$\bigfkh{x-at,x+at}$上。
	记$C_r$是以$(x,y)$为圆心,$r$为半径的圆,在$C_r$上
	\begin{gather*}
		\d\ell=\sqrt{1+\kh{\dv\eta\xi}^2}\d\xi=\frac{r\d\xi}{\sqrt{r^2-(\xi-x)^2}}.
	\end{gather*}
	故
	\begin{align*}
		&\quad\int_{D_{at}}\frac{\varPhi(\xi,\eta)\d\xi\nd\eta}{\sqrt{a^2t^2-(\xi-x)^2-(\eta-y)^2}}=\int_0^{at}\oint_{C_r}\frac{\varphi(\xi)}{\sqrt{a^2t^2-r^2}}\d\ell\d r\\
		&=2\int_0^{at}\int_{x-r}^{x+r}\frac{\varphi(\xi)}{\sqrt{a^2t^2-r^2}}\frac{r\d\xi}{\sqrt{r^2-(\xi-x)^2}}\d r\tag{交换次序}\\
		&=\kh{\int_x^{x+at}\int_{\xi-x}^{at}+\int_{x-at}^x\int_{x-\xi}^{at}}\frac{\varphi(\xi)}{\sqrt{a^2t^2-r^2}}\frac{2r\d r}{\sqrt{r^2-(\xi-x)^2}}\d\xi
	\end{align*}
	推出d'Alembert公式:
	\[
		u(x,t)=\pp t\kh{\frac1{2a}\int_{x-at}^{x+at}\varphi(\xi)\d\xi}+\frac1{2a}\int_{x-at}^{x+at}\psi(\xi)\d\xi.
	\]
\end{example}
\section{二阶线性偏微分方程及标准型}
两个自变量的二阶其次线性偏微分方程
\[
	a_{11}u_{xx}+2a_{12}u_{xy}+a_{22}u_{yy}+b_1u_x+b_2u_y+cu=0.
\]
进行变量代换
\begin{align*}
	\begin{cases}
		\xi=\xi(x,y) \\
		\eta=\eta(x,y)
	\end{cases}
	\quad\text{其中}~
	\det J(\xi,\eta)\neq 0.
\end{align*}
变成
\[
	A_{11}u_{\xi\xi}+2A_{12}u_{\xi\eta}+A_{22}u_{\eta\eta}+B_1u_\xi+B_2u_\eta+cu=0.
\]
其中
\begin{align*}
	A_{11} & =a_{11}\xi_x^2+2a_{12}\xi_x\xi_y+a_{22}\xi_y^2,                          \\
	A_{12} & =a_{11}\xi_x\eta_x+a_{12}\kh{\xi_x\eta_y+\xi_y\eta_x}+a_{22}\xi_y\eta_y, \\
	A_{22} & =a_{11}\eta_x^2+2 a_{12}\eta_x\eta_y+a_{22}\eta_y^2;                     \\
	B_1    & =a_{11}\xi_{xx}+2a_{12}\xi_{xy}+a_{22}\xi_{yy}+b_1\xi_x+b_2\xi_y,        \\
	B_2    & =a_{11}\eta_{xx}+2a_{12}\eta_{xy}+a_{22}\eta_{yy}+b_1\eta_x+b_2\eta_y,
\end{align*}
用矩阵表达即
\begin{align*}
	\begin{bmatrix}
		A_{11} & A_{12} \\
		A_{12} & A_{22}
	\end{bmatrix}=\begin{bmatrix}
		\xi_x  & \xi_x  \\
		\eta_x & \eta_y
	\end{bmatrix}\begin{bmatrix}
		a_{11} & a_{12} \\
		a_{12} & a_{22}
	\end{bmatrix}\begin{bmatrix}
		\xi_x  & \xi_x  \\
		\eta_x & \eta_y
	\end{bmatrix}\tp.
\end{align*}

考虑使$A_{11}=0$或$A_{22}=0$则有
\[
	a_{11}\kh{\dv yx}^2-2a_{12}\dv yx+a_{22}=0.
\]
其判别式
\[
	\Delta=a_{12}^2-a_{11}a_{22}.
\]
\paragraph{$\Delta>0$,双曲型}不妨设$a_{11}\neq 0$。
\begin{align}
	\dv yx=\frac{a_{12}\pm\sqrt\Delta}{a_{11}}\implies
	\begin{cases}
		\xi(x,y)=\const \\
		\eta(x,y)=\const'
	\end{cases}
\end{align}
继而$A_{11}=A_{22}=0,$
\begin{gather}\notag
	A_{12}=-\frac{2\Delta}{a_{11}}\xi_y\eta_y\neq 0,\\
	u_{\xi\eta}+\frac1{2A_{12}}\kh{B_1u_\xi+B_2u_\eta+cu}=0.
\end{gather}
上式就是\textbf{双曲型方程的标准型}。

若再作变量替换$p=(\xi+\eta)/2,\;q=(\xi-\eta)/2,$方程可化为
\begin{align}
	u_{pp}-u_{qq}+\frac1{A_{12}}\bigfkh{\kh{B_1+B_2}u_q+\kh{B_1-B_2}u_p+cu}=0.
\end{align}
\paragraph{$\Delta=0$,抛物型}只有一个线性ODE,不妨设$a_{11}\neq 0$。
\begin{align}
	\dv yx=\frac{a_{12}}{a_{11}}\implies\xi(x,y)=\const.
\end{align}
任取$\eta(x,y)$,可得$A_{11}=A_{12}=0$
\begin{align}
	u_{\eta\eta}+\frac1{A_{22}}\kh{B_1u_\xi+B_2u_\eta+cu}=0.
\end{align}
\paragraph{$\Delta<0$,椭圆型}ODE解为复函数,不妨设$a_{11}\neq 0,$
\begin{align}
	\dv yx=\frac{a_{12}\pm\i\sqrt{-\Delta}}{a_{11}}\implies \xi(x,y)\pm\i\eta(x,y)=\const.
\end{align}
其中$\xi,\eta$为实函数,$A_{12}=0,A_{11}=A_{22}\neq 0,$
\begin{align}
	u_{\xi\xi}+u_{\eta\eta}+\frac1{A_{11}}\kh{B_1u_\xi+B_2u_\eta+cu}=0.
\end{align}
\section{叠加原理和齐次化原理}
定义算子为从函数类到函数类的映射$\mathcal T$。
%一般的,二阶线性PDE中
%\[\Par\equiv\sum_{i,j=1}^na_{ij}(X)\pw{}{x_i}{x_j}+\sum_{i=1}^nb_i(X)\pp{x_i}+c(X).\]
%是$\sC^2\to\sC$的线性算子。
\begin{theorem}{叠加原理}{Superposition Principle}
	%所谓叠加原理,即
	线性算子$\cL$可与$\textstyle\lim,\sum,\int$等运算符交换。% 在课程范围内,这是适当的。
\end{theorem}

齐次化原理也称冲量原理,源于求解有外力的弦振动方程
\begin{align*}
	\begin{cases}
		u_{tt}=a^2u_{xx}+f(x,t),\\
		u(x,0)=\varphi(x),\quad u_t(x,0)=\psi(x).
	\end{cases}
\end{align*}
利用叠加原理,解$u=u_0+w$;其中$u_0$表示$f\equiv 0$的齐次化解;$w$表示$\varphi,\psi\equiv 0$的解,即纯受迫振动。

考虑时间段$(\tau,\tau+\vd\tau)$内,位移分布$w(x,t)=:v(x,t;\tau)\vd\tau$,外力冲量$f(x,\tau)\vd\tau$,有
\begin{align*}
	%w(x,t)=\int_0^tv(x,t;\tau)\d\tau,\quad
	\begin{cases}
			v_{tt}=a^2v_{xx},\\
			v|_{t=\tau}=0\quad v_t|_{t=\tau}=f(x,\tau).
	\end{cases}\implies v(x,t;\tau)=\frac1{2a}\int_{x-a(t-\tau)}^{x+a(t-\tau)}f(\xi,\tau)\d\xi.
\end{align*}
%因此
%\[=\frac1{2a}\int_0^t\bs3\int_{x-a(t-\tau)}^{t+a(t-\tau)}f(\xi,\tau)\d\xi\nd\tau.\]
叠加得
\begin{align}
	\begin{aligned}
		u(x,t)= {}&\frac12\bigfkh{\varphi(x-at)+\varphi(x+at)}+\frac1{2a}\int_{x-at}^{x+at}\psi(\xi)\d\xi+\frac1{2a}\int_0^t\int_{x-a(t-\tau)}^{x+a(t-\tau)}f(\xi,\tau)\d\xi\nd\tau.
	\end{aligned}
\end{align}
\begin{theorem}{齐次化原理}{Homogeneity Principle}
	定解问题
	\begin{align*}
		\begin{cases}
			\frac{\p^mu}{\p t^m}=\Par u+f(\bm x,t),&\bm x\in\mathbb R^n,\\
			\edg u_{t=0}=\edg{\pv ut}_{t=0}=\cdots=\edg{\frac{\p^{m-1}u}{\p t^{m-1}}}_{t=0}=0.
		\end{cases}
	\end{align*}
	其中$\Par$为常系数线性偏微分算子,有解
	\[
		u(\bm x,t)=\int_0^tv(\bm x,t;\tau)\d\tau,
	\]
	$v$满足
	\begin{align*}
		\begin{cases}
			\pv[m]vt=\Par v,&\bm x\in\mathbb R^n,\\
			\edg v_{t=\tau}=\edg{\pv v{t}}_{t=\tau}=\cdots=\edg{\pv[m-2]vt}_{t=\tau}=0, \\
			\edg{\pv[m-1]vt}_{t=\tau}=f(\bm x,\tau).
		\end{cases}
	\end{align*}

	当$\bm x\in V\subsetneq\mathbb R^n$时,还应加上其次边界条件
	\[
		\edg{\Par u}_{\p V}=0,\quad\edg{\Par v}_{\p V}=0.
	\]
\end{theorem}

