\documentclass{../thunote}

\theoremstyle{definition}
\newtheorem*{corollary}{推论}
\newtheorem*{remark}{注}
\newtheorem*{lemma}{引理}

%% text
\newcommand*{\vs}{~\text{-}~}
\newcommand*{\const}{\text{const}}
\newcommand*{\otherwise}{\text{otherwise}}
\newcommand*{\plusc}{{\color{lightgray}\,+\,\const}}

%% roman
\newcommand*{\e}{\mathop{}\!\mathrm{e}^}  % exp
\let\accenti\i
\renewcommand*{\i}{\mathrm{i}}

\usepackage{bm}
\newcommand{\hatbm}[1]{\hat{\bm{#1}}}  % bm with hat, maybe Fourier Transform
\newcommand{\nvec}[1]{\hat{\bm{#1}}}  % normalized vector, without space
\newcommand{\uvec}[1]{\mathop{}\!\nvec{#1}}  % unit vector, with possible space
\newcommand{\dotbm}[1]{\dot{\bm{#1}}}

\usepackage{mathrsfs}  % \mathscr

%% blackboard bold: number sets
\newcommand*{\NN}{\mathbb N}    % natural
\newcommand*{\ZZ}{\mathbb Z}    % integer
\newcommand*{\QQ}{\mathbb Q}    % rational
\newcommand*{\RR}{\mathbb R}    % real
\newcommand*{\CC}{\mathbb C}    % complex
\newcommand*{\FF}{\mathbb F}    % any number field
\newcommand*{\PP}{\mathbb P}    % prime

\usepackage{cancel}

%% vector operator
\let\divides\div
\newcommand*{\grad}{\nabla}         % gradient
\renewcommand*{\div}{\nabla\cdot}   % divergence
\newcommand*{\curl}{\nabla\times}   % curl
\newcommand*{\lap}{\nabla^2}        % Laplacian

%% differential
\let\accentd\d
\renewcommand*{\d}{\mathop{}\!\mathrm{d}}   % differential
\newcommand*{\nd}{\mathrm{d}}               % differential without space
\newcommand*{\vd}{\mathop{}\!\delta}    % delta: δ
\newcommand*{\D}{\Delta}                % Delta: Δ
\newcommand*{\p}{\partial}              % partial: ∂
\newcommand{\dd}[2][{}]{\frac{\nd^{#1}}{\nd{#2}^{#1}}}      % d/dx
\newcommand{\dv}[3][{}]{\frac{\nd^{#1}#2}{\nd{#3}^{#1}}}    % df/dx
\newcommand{\pp}[2][{}]{\frac{\p^{#1}}{\p{#2}^{#1}}}    % ∂/∂x
\newcommand{\pv}[3][{}]{\frac{\p^{#1}#2}{\p{#3}^{#1}}}  % ∂f/∂x
\newcommand{\pw}[3]{\frac{\p^2{#1}}{\p{#2}\p{#3}}}      % ∂^2f/∂x∂y

%% integral limits
\newcommand*{\zti}{_0^{+\infty}}            % zero to infinity
\newcommand*{\iti}{_{-\infty}^{+\infty}}    % -infinity to +infinity

% brackets with auto size
\newcommand{\abs}[1]{\left\lvert#1\right\rvert}     % absolute value: |x|
\newcommand{\norm}[1]{\left\lVert#1\right\rVert}    % norm: ||x||
\newcommand{\edg}[1]{\left.#1\right\rvert}          % edge line: f|
\newcommand{\kh}[1]{\left(#1\right)}                % parentheses: (x)
\newcommand{\fkh}[1]{\left[#1\right]}               % square brackets: [x]
\newcommand{\hkh}[1]{\left\{#1\right\}}             % braces: {x}
% \newcommand{\ang}[1]{\left\langle #1\right\rangle}  % angle brackets: <x>
\newcommand{\floor}[1]{\left\lfloor#1\right\rfloor} % floor
\newcommand{\ceil}[1]{\left\lceil#1\right\rceil}    % ceil
\newcommand{\ave}[1]{\left\langle #1\right\rangle}  % average: <x>
\newcommand{\set}[2]{\left\{#1\,\middle\vert\,#2\right\}}   % set: {x|x1,x2,...}
\newcommand{\bra}[1]{\left\langle #1\right\vert}    % bra: <ψ|
\newcommand{\ket}[1]{\left\vert #1\right\rangle}    % ket: |ψ>
\newcommand{\brkt}[2]{\left\langle #1\middle\vert #2\right\rangle}  % inner product of bra-ket: <φ|ψ>
\newcommand{\ktbr}[2]{\left\vert#1\right\rangle\hspace{-3pt}\left\langle #2\right\vert} % ket-bra: |ψ><φ|
\newcommand{\division}[2]{\left.{#1}\middle/{#2}\right.}    % division: A/B
\newcommand{\inp}[2]{\left\langle #1,#2\right\rangle}       % inner product: <u,v>

% brackets with fixed size
\newcommand{\nnorm}[1]{\lVert#1\rVert}
\newcommand{\nset}[2]{\{#1\,|\,#2\}}
\newcommand{\bigkh}[1]{\bigl(#1\bigr)}
\newcommand{\Bigkh}[1]{\Bigl(#1\Bigr)}
\newcommand{\biggkh}[1]{\biggl(#1\biggr)}
\newcommand{\bigfkh}[1]{\bigl[#1\bigr]}
\newcommand{\Bigfkh}[1]{\Bigl[#1\Bigr]}
\newcommand{\biggfkh}[1]{\biggl[#1\biggr]}

%% math operator
\let\Real\Re
\let\Imaginary\Im
\let\Re\relax
\let\Im\relax
\DeclareMathOperator{\Re}{Re}  % real part
\DeclareMathOperator{\Im}{Im}  % imaginary part
\DeclareMathOperator{\sech}{sech}
\DeclareMathOperator{\csch}{csch} 
\DeclareMathOperator{\arcsec}{arcsec}
\DeclareMathOperator{\arccot}{arccot} 
\DeclareMathOperator{\arccsc}{arccsc} 
\DeclareMathOperator{\arsinh}{arsinh} 
\DeclareMathOperator{\arcosh}{arcosh} 
\DeclareMathOperator{\artanh}{artanh} 
\DeclareMathOperator{\sinc}{sinc}
\DeclareMathOperator{\sgn}{sgn}     % sign function
\DeclareMathOperator{\id}{id}       % identity mapping
\DeclareMathOperator{\Res}{Res}     % residue
\DeclareMathOperator{\supp}{supp}   % support set

%% linear algebra
\DeclareMathOperator{\rank}{rank}   % rank
\DeclareMathOperator{\diag}{diag}   % diagonal
\DeclareMathOperator{\tr}{tr}       % trace
\newcommand*{\tp}{^\top}    % transpose: A^T
\newcommand*{\cj}{^\ast}    % conjugate: A*
\newcommand*{\dg}{^\dagger} % conjugate transpose/Hermite: A†
\newcommand*{\iv}{^{-1}}    % inverse: A^-1

%% physicists
\newcommand*{\Schr}{Schrödinger}
\newcommand*{\Legd}{Legendre}
\newcommand*{\deB}{de Broglie}
\newcommand*{\Rayl}{Rayleigh}
\newcommand*{\Lande}{Landé}

%% particles
\newcommand*{\elc}{\mathrm e}
\newcommand*{\pton}{\mathrm p}
\newcommand*{\nton}{\mathrm n}
\newcommand*{\mol}{\mathrm m}

%% physical constants/notation
\newcommand*{\NA}{N_{\mathrm A}}    % Avogadro constant
\newcommand*{\kB}{k_{\mathrm B}}    % Boltzmann constant
\newcommand*{\muB}{\mu_\mathrm B}   % Bohr magne
\newcommand*{\Ek}{E_{\mathrm k}}    % kinetic energy
\newcommand*{\FWHM}{\mathrm{FWHM}}  % full width at half maximum

%% subscript/superscript 
\newcommand*{\eff}{_\mathrm{eff}}   % effective
\newcommand*{\tot}{_\mathrm{tot}}   % total
\newcommand*{\maxi}{_\mathrm{max}}  % maximum
\newcommand*{\mini}{_\mathrm{min}}  % minimum

%% unit tag
\newcommand*{\lSI}{\tag{SI}}    % le SI
\newcommand*{\CGS}{\tag{CGS}}   % cm, g, s system

%% other
\newcommand*{\qqquad}{\qquad\quad}
\newcommand*{\qqqquad}{\qquad\qquad}

\newcommand{\notimplies}{\hspace{1ex}\not\hspace{-1ex}\implies}

\let\geq\geqslant
\let\leq\leqslant

\newcommand*{\avg}[1]{\overline{#1}}

\newcommand*{\bigo}{\mathcal O}     % big O notation
\newcommand*{\degree}{^\circ}       % degree

\newcommand{\fracdisp}[2]{\frac{\displaystyle #1}{\displaystyle #2}}

\newcommand{\lhkh}[1]{\left\{#1\right.} % left brace: {x


\renewcommand*{\lap}{\Delta}
\newcommand*{\Dif}{\mathcal D}
\newcommand*{\Par}{\mathcal P}

\DeclareMathOperator{\Heaviside}{H}
\DeclareMathOperator{\Si}{Si}

\newcommand*{\sC}{\mathscr C}
\newcommand*{\cL}{\mathcal L}  % linear operator
\newcommand{\cl}[1]{\mathcal L\fkh{#1}}
\newcommand{\cli}[1]{\mathcal L^{-1}\!\fkh{#1}}
\newcommand{\cf}[2][{}]{\mathcal F_{\mathrm{#1}}\fkh{#2}}
\newcommand{\cfi}[2][{}]{\mathcal F_{\mathrm{#1}}^{-1}\!\fkh{#2}}

\newcommand*{\even}{\mathrm e}
\newcommand*{\odd}{\mathrm o}
\newcommand*{\sine}{\mathrm s}
\newcommand*{\cosi}{\mathrm c}
\newcommand*{\Cooo}{\mathscr D}
\newcommand*{\Schwsp}{\mathscr S}
\DeclareMathOperator{\FS}{FS}

\begin{document}

\title{数学物理方法\\Methods of Mathematics and Physics}
\maketitle

\frontmatter
\tableofcontents

\mainmatter
% \setcounter{section}{-1}

\chapter{热力学基本定律}

%\section*{宏观}

宏观物质可以用很少的量表征。这种特性源于:
\textit{宏观测量与原子时间尺度相比极其缓慢,与原子空间尺度相比十分粗糙。}
宏观体系忽略了系统内部每个粒子的具体运动,正如Anderson所说:\textbf{\textit{More is different.}}
而热力学便是\textit{唯象}地描述多粒子行为的宏观理论。

\section{热力学第零定律}

\begin{definition}{热力学系统}{thermal system}
	热力学系统(thermal system)是大量微观粒子组成的有限宏观体系。
\end{definition}

平衡态指宏观性质不随时间改变的状态。

\begin{theorem}{热力学第零定律:热平衡定律}{thermal equilibrium}
	若系统A和系统B热平衡,且系统A和系统C也热平衡,则B和C热平衡。
\end{theorem}

\begin{corollary}
	互为热平衡的体系有一共同的物理性质,称为温度$T$。
\end{corollary}

\begin{definition}{物态方程}{state equation}
	物态方程(state equation)是温度$T$与其它状态参量间的关系。
\end{definition}

\begin{example}
	{理想气体物态方程}{}
	理想气体(ideal gas)的压强$p$、体积$V$、温度$T$和物质的量$n$之间的关系为:
	\begin{equation}
		pV=nRT.
	\end{equation}
	其中$R=\SI{8.314}{J/K.mol}$是理想气体常数。
\end{example}

\begin{example}
	{Van der Waals气体物态方程}{}
	Van der Waals气体考虑了分子间的相互作用和分子体积:
	\begin{equation}
		\biggkh{p+a\frac{n^2}{V^2}}(V-nb)=nRT.
	\end{equation}
	其中$a$与分子间的相互作用有关,$b$与分子体积有关。
\end{example}

\section{热力学第一定律}

热力学系统在外界影响下,会从一个平衡态过渡到另一个平衡态,在这个过程中的任一时刻,系统的状态都不是平衡态。
但是如果这个过程中的变化速度足够慢,每一瞬时都可以无限接近平衡态,我们就可以当做平衡态去处理这个过程。

\begin{definition}{准静态过程}{quasistatic process}
	准静态过程(quasistatic process)指每一瞬时,系统状态都无限接近平衡态的过程。
\end{definition}

系统的能量包括内能$U$和整体运动能量。对于封闭系统,能量交换有功$W$和热量$Q$两种方式。准静态过程中,
\begin{align}
	\vd W=\sum_iY_i\d y_i,
\end{align}
其中$(Y_i,y_i)$分别是广义力和广义坐标,如$(-p,V),(\mu_0H,M)$等。

\begin{theorem}{热力学第一定律:能量守恒定律}{Energy Conservation Law}
	一个热力学系统的内能增量$\d U$等于外界对它所做的功$\vd W$与外界向它传递的热量$\vd Q$的和:
	\begin{align}
		\d U=\vd W+\vd Q.
	\end{align}
\end{theorem}

\begin{remark}
	如果系统是\textbf{绝热}($\vd Q\equiv 0$)的,我们便可以用机械功$\vd W$测量内能的变化$\D U$,通过指定基准态的内能$U_0$就可以得出任意状态的内能$U$。进而我们可以测量导热系统的传热$\vd Q$。
\end{remark}

\begin{definition}{热容}{heat capacity}
	定义热容(heat capacity)是物质在单位温度变化下所吸收或放出的热量:
	\begin{equation}
		C:=\lim_{\D T\to0}\frac{\D Q}{\D T}.
	\end{equation}
	比热容(specific heat capacity)是单位质量的热容。
\end{definition}

\begin{remark}
	显然,热容与过程相关,可定义等容热容$C_V$和等压热容$C_p$。
\end{remark}

内能标准全微分式:将$U$全微分式中各变量微分前的系数用可测量表达出来。
\begin{example}{静流体系统}{static fluid system}
	以$T,V$为变量
	\[
		\d U=\underset{C_V}{\underline{\kh{\pv UT}_V}}\d T+\kh{\pv UV}_T\d V.
	\]
	已知 
	\begin{align*}
		C_p&=\kh{\frac{\vd Q}{\d T}}_p=\kh{\frac{\d U+p\d V}{\d T}}_p=\kh{\pv UT}_p+p\kh{\pv VT}_p\\
		&=\underset{C_V}{\underline{\kh{\pv UT}_V}}+\underset{\text{target}}{\underline{\kh{\pv UV}_T}}\kh{\pv VT}_p+p\kh{\pv VT}_p.
	\end{align*}
	因此
	\begin{align}
		\d U=C_V\d T+\fkh{(C_p-C_V)\kh{\pv TV}_p-p}\d V.
	\end{align}
\end{example}
\section{热力学第二定律}
\begin{theorem}{热力学第二定律}{Second Law of Thermodynamics}
	宏观的自发过程是不过逆的。
	\begin{compactitem}
		\item Clausius表述:不可能把热量从低温物体传到高温物体,而不引起其它变化。
		\item Kelvin表述:不可能从单一热源吸热,使之完全变成有用功,而不引起其它变化。
	\end{compactitem}
\end{theorem}
\begin{theorem}{Carnot定理}{Carnot's Theorem}
	在相同高、低温热源之间工作的热机中,可逆机的效率最高:
	\begin{align}
		\eta=1-\frac{Q_2}{Q_1}=1-\frac{T_2}{T_1}.
	\end{align}
	可逆机效率只与热源温度有关,与工作物质无关。
\end{theorem}
\paragraph{热力学温标}借助Carnot机可实现绝对温标。
\begin{theorem}{Clausius不等式}{Clausius inequality}
	在热力学循环中,系统热的变化及温度之间的关系:
	\begin{equation}
		\oint\frac{\vd Q}T\leqslant 0.
	\end{equation}
	当且仅当为可逆热机时取等号,此过程定义为可逆过程。
\end{theorem}
进而定义可逆过程中的熵
\begin{align}
	\d S:=\frac{\vd Q}T.
\end{align}
热力学第二定律的熵表述:孤立系统的熵不减,熵是热运动混乱程度的量度。
\begin{example}{熵的计算}{Calculating Entropy}
	将质量相同而温度分别为$T_1$和$T_2$的两杯水在等压下绝热的混合,求熵变。

	\textbf{解:}终态温度$T=(T_1+T_2)/2$,第一杯水的熵变为
	\[
		\D S_1=\int_{T_1}^T\frac{C_p\d T}T=C_p\ln\frac{T_1+T_2}{2T_1},
	\]
	第二杯水的熵变$\D S_2$同理可求,
	总熵增
	\[
		\D S_1+\D S_2=C_p\ln\frac{(T_1+T_2)^2}{4T_1T_2}\geqslant 0.
	\]
	取等号当且仅当$T_1=T_2$。
\end{example}
\section{热力学第三定律}
\begin{theorem}{热力学第三定律}{Third Law of Thermodynamics}
	% Nernst定理:
	$T\to 0$时,等温过程的熵变$\D_TS\to 0$

	Nernst原理:不可能使一个物体冷却到绝对温度的零度。
\end{theorem}
\chapter{随机变量及其分布}

\begin{definition}{(一维)随机变量}{random variable}
	定义随机变量(random variable)~$X:\Omega\to\RR$是样本空间上的实值函数。有
	\begin{compactitem}
		\item 离散型(discrete):至多可数个取值;
		\item 连续型(continuous):区间型取值(不严格);
		\item 其他
	\end{compactitem}
\end{definition}
\begin{definition}
	{随机变量的概率}{}
	$\forall I\subset\RR$可测,记原像集$X\inv(I)\in\cF$,定义\footnote{这种定义属于好看的鱼(形式简洁),而不属于好吃的鱼(实用)。}
	\[
		\P_X(X\in I):=\P(X\inv(I)),\quad\forall I\subset\RR~\text{可测}
	\]
	一般记$\P_X$为$\P$。
\end{definition}
\begin{definition}{累计分布函数}{cumulative distribution function}
	记$X$的累计分布函数(cumulative distribution function, CDF)\index{CDF, 累计分布函数}
	\[
		\CDF(x):=\P(X\leqslant x),\quad\forall x\in\RR
	\]
	则$\P(a<X\leqslant b)\equiv\CDF(b)-\CDF(a)$。
\end{definition}

\begin{corollary}
	CDF的性质:
	\begin{itemize}
		\item $\CDF(x)$单调递增(不严格单调);
		\item $\lim_{x\to+\infty}\CDF(x)=1,\lim_{x\to-\infty}\CDF(x)=0;$
		\item $\CDF(x)$右连续,不一定左连续。
	\end{itemize}
\end{corollary}
\begin{remark}~
	\begin{compactenum}
		\item 随机要素来自样本点$\omega$的随机选择;
		\item $X,Y$同样本空间时,一般地,$aX+bY$等$X,Y$的函数也是随机变量;
		\item 随机变量同分布$\iff$ CDF相同;但不代表变量相同。
	\end{compactenum}
\end{remark}

\section{离散分布}

\begin{definition}{离散分布}{discrete distribution}
	离散分布可由分布列(probability distribution)表示概率在样本空间中的分布
	\begin{center}
		\begin{tabular}{cccccc}
			\toprule
			$X$&$x_1$&$x_2$&$\cdots$&$x_i$&$\cdots$\\
			\midrule
			$p$&$p_1$&$p_2$&$\cdots$&$p_i$&$\cdots$\\
			\bottomrule
		\end{tabular}
	\end{center}
	% 概率质量函数(probability mass function, PMF)\index{PMF, 概率质量函数}
	% \[
	% 	f(x)=\P(X=x),\quad\forall x\in\RR
	% \]
\end{definition}

\begin{corollary}
	离散分布的CDF为阶梯函数。
\end{corollary}

\begin{definition}{期望和方差}{expectation and variance}
	期望(expectation)即均值
	\begin{equation}
		\E(X):=\sum\nolimits_{i\in I}x_ip_i,
	\end{equation}
	方差(variance)
	\begin{equation}
		\Var(X):=\sum\nolimits_{i\in I}\bigkh{x_i-\E(X)}^2p_i\equiv\E(X^2)-\E(X)^2.
	\end{equation}
\end{definition}

\begin{corollary}
	随机变量$X$的函数$g(X)$的期望
	\[
		\E(g(X))=\sum\nolimits_{i\in I}g(x_i)p_i.
	\]
\end{corollary}

\begin{remark}
	期望存在要求级数绝对收敛。
\end{remark}

\paragraph{二项分布}

~

\begin{definition}{\Bern 分布}{Bernoulli distribution}
	\Bern 分布也称0-1分布,$p$为成功概率,记作$X\sim\Bino(p)$,其分布列为
	\begin{equation}
		\P(X=k)=\begin{cases}
			1-p,&k=0\\
			p,&k=1
		\end{cases}
	\end{equation}
	% \begin{center}
	% 	\begin{tabular}{ccc}
	% 		\toprule
	% 		$X$&0&1\\
	% 		\midrule
	% 		$p$&$1-p$&$p$\\
	% 		\bottomrule
	% 	\end{tabular}
	% \end{center}
\end{definition}

\begin{definition}{二项分布}{binominal distribution}
	$n$次独立\Bern 试验的成功次数$X$服从二项分布(binominal distribution),记作$X\sim\Bino(n,p)$,
	\begin{equation}
		\P(X=k)=\binom nk p^k(1-p)^{n-k},\quad k=0,\ldots,n
	\end{equation}
	\begin{center}
		\includegraphics[width=.95\textwidth]{figures/pdf_bin.pdf}
		\captionof{figure}{不同$n,p$下的二项分布}
	\end{center}
\end{definition}

\begin{corollary}
	% 显然\Bern 分布就是二项分布的特例。
	二项分布的期望和方差为
	\begin{subequations}
		\begin{align}
			\E(X)&=np,\\
			\Var(X)&=np(1-p).
		\end{align}
	\end{subequations}
\end{corollary}

\paragraph{Poisson分布}

我们考虑一个时间段内的事件次数$X\sim\Bino(n,p)$,其中$n$是时间段的切片份数。
保持其期望$\E(X)=np=:\lambda$不变,令$n\to\infty$,即事件可能发生在任一时刻,则
\[
	\lim_{n\to\infty}\binom nk\kh{\frac\lambda{n}}^k\kh{1-\frac\lambda{n}}^{n-k}=\lim_{n\to\infty}\frac{\lambda^k}{k!}\kh{1-\frac\lambda{n}}^n\cancel{\frac{n!}{n^k(n-k)!}}\cancel{\kh{1-\frac\lambda{n}}^{-k}}=\frac{\lambda^k}{k!}\e{-\lambda}.
\]
% 我们便得到了一个新的分布,称为Poisson分布。

\begin{definition}{Poisson分布}{Poisson distribution}
	小概率事件在一段时间内发生的次数$X$服从Poisson分布,记作$X\sim\Pois(\lambda)$,
	\[
		\P(X=k)=\frac{\lambda^k}{k!}\e{-\lambda},\quad k=0,1,2,\ldots
	\]
	\begin{center}
		\includegraphics[width=.9\textwidth]{figures/pdf_poission.pdf}
		\captionof{figure}{不同$\lambda$下的Poisson分布}
	\end{center}
\end{definition}
\begin{corollary}
	Poisson分布的期望和方差为
	\begin{subequations}
		\begin{align}
			\E(X)&=\lambda,\\
			\Var(X)&=\lambda.
		\end{align}
	\end{subequations}
\end{corollary}

\begin{theorem}{Poisson分布近似二项分布}{}
	用$\Pois(np)$近似$\Bino(n,p)$的误差最多为$\min(p,np^2)$。
\end{theorem}

\begin{remark}
	若试验不独立,但满足弱相依条件下,Poisson分布仍为较好近似。
\end{remark}

\begin{example}{弱相依条件举例:配对问题}{weak dependence condition}
	在\exmref{exm:derangement} 中,尽管$A_i$和$A_j$并不独立,但弱相依
	\[
		\P(A_i)=\frac1n\approx\P(A_i|A_j)=\frac1{n-1}.
	\]
	记$X=$拿到自己帽子的人数,当$n\to\infty$时,$X\sim\Pois(1)$
	\[
		\P(X=k)=\frac{\e{-1}}{k!},\quad k=0,1,2,\ldots
	\]
	\tcblower
	下面用常规做法检验:指定$k$个人,记
	\begin{itemize}
		\item $E=$这$k$个人都拿到自己的帽子;
		\item $F=$余下$n-k$个人都未拿到自己的帽子。
	\end{itemize}
	则$\P(F|E)$就是$n-k$个人的错排问题,根据\exmref{exm:derangement} 中的结果:
	\[
		\P(E\cap F)=\P(E)\P(F|E)=\frac{(n-k)!}{n!}\cdot\frac{!(n-k)}{(n-k)!},
	\]
	对于$\P(X=k)$来说,由于这$k$个人是任选的,故
	\[
		\P(X=k)=\binom nk\P(E\cap F)=\frac1{k!}\cdot\frac{!(n-k)}{(n-k)!}\to\frac{1}{k!}\cdot\frac1{\e{}}.
	\]
\end{example}

\section{连续分布}

\begin{definition}{连续分布}{continuous distribution}
	若存在$f(x)\geqslant 0$,使得$\forall I\subset\RR$可测,都有
	\[
		\P(X\in I)=\int_If(x)\d x
	\]
	则称$X$为连续型随机变量,服从连续分布,$f(x)$为概率密度函数(probability density function, PDF)。\index{PDF, 概率密度函数}
\end{definition}

\begin{corollary}
	连续分布的性质:
	\begin{itemize}
		\item $\forall x\in\RR,\enspace\P(X=x)\equiv 0;$
		\item 归一性:
			\begin{equation}
				\int\iti f(x)\d x\equiv 1;
			\end{equation}
		\item 期望和方差(要求积分\textbf{绝对收敛})
			\begin{subequations}
				\begin{align}
					\E(X)&=\int\iti xf(x)\d x,\\
					\Var(X)&=\int\iti(x-\E(X))^2f(x)\d x.
				\end{align}
			\end{subequations}
		\item 连续分布的CDF $\CDF(x)$连续且可导,且
			\begin{equation}
				\CDF'(x)=f(x).
			\end{equation}
			若$\CDF(x)$严格递增,则$\CDF^{-1}(y)$存在;但若其不严格递增,也可well define
			\begin{equation}
				\label{eq:CDF-1}
				\CDF^{-1}(y):=\inf\set x{\CDF(x)\geqslant y}.
			\end{equation}
	\end{itemize}
\end{corollary}
\iffalse
\begin{example}{常见连续分布}{}
	\begin{compactenum}
		\item 均匀分布$f(x;a,b)=\frac1{b-a},a\leqslant x\leqslant b;$
		\item 指数分布$f(x;\lambda)=\lambda\e{-\lambda x},x\geqslant 0;$
		\item 正态分布$f(x;\mu,\sigma^2)=\frac1{\sqrt{2\pi}\sigma}\e{-(x-\mu)^2/2\sigma^2},-\infty<x<\infty;$
		\item 伽玛分布$f(x;\alpha,\lambda)=\frac{\lambda^\alpha}{\Gamma(\alpha)}x^{\alpha-1}\e{-\lambda x}.x\geqslant 0;$
		\item 卡方分布$f(x;n)=\frac{x^{n/2-1}}{2^{n/2}\Gamma(n/2)}\e{-x/2},x\geqslant 0;$
	\end{compactenum}
\end{example}
\fi
\begin{definition}{均匀分布}{uniform distribution}
	均匀分布(uniform distribution)记作$X\sim\Unif(a,b)$,%随机数$\Unif(0,1)$
	\begin{equation}
		f(x)=\begin{cases}
			\frac1{b-a},&x\in(a,b)\\
			0,&\text{elsewhere}
		\end{cases}
	\end{equation}
	特别地,服从标准均匀分布的随机变量$X\sim\Unif(0,1)$也称为随机数。
\end{definition}
\begin{corollary}
	均匀分布的期望和方差为
	\begin{subequations}
		\begin{align}
			\E(X)&=\frac{a+b}2,\\
			\Var(X)&=\frac{(b-a)^2}{12}.
		\end{align}
	\end{subequations}
\end{corollary}
\begin{definition}{正态分布}{normal distribution}
	正态分布(normal distribution)记作$X\sim\Norm(\mu,\sigma^2)$,%标准正态$\Norm(0,1)$
	\begin{equation}
		f(x)=\frac1{\sqrt{2\pi}\sigma}\exp\biggfkh{-\frac{(x-\mu)^2}{2\sigma^2}},\quad x\in\RR
	\end{equation}
	特别地,标准正态分布$\Norm(0,1)$的CDF记作$\Phi(x)$。
	\begin{center}
		\includegraphics[width=.9\textwidth]{figures/pdf_normal.pdf}
		\captionof{figure}{不同$\mu,\sigma$下的正态分布}
	\end{center}
\end{definition}
\begin{corollary}
	正态分布的期望和方差为
	\begin{subequations}
		\begin{align}
			\E(X)&=\mu,\\
			\Var(X)&=\sigma^2.
		\end{align}
	\end{subequations}
\end{corollary}
\begin{definition}{指数分布}{exponential distribution}
	指数分布(exponential distribution)通常刻画寿命、等待时间,记作$X\sim\Expo(\lambda)$,
	\begin{equation}
		f(x)=\begin{cases}
			\lambda\e{-\lambda x},&x>0\\
			0,&x\leqslant 0
		\end{cases}
	\end{equation}
	\begin{center}
		\includegraphics[width=.9\textwidth]{figures/pdf_exp.pdf}
		\captionof{figure}{不同$\lambda$下的指数分布}
	\end{center}
\end{definition}

\begin{corollary}
	指数分布的期望和方差为
	\begin{subequations}
		\begin{align}
			\E(X)&=\frac1\lambda,\\
			\Var(X)&=\frac1{\lambda^2}.
		\end{align}
	\end{subequations}
	其期望往往被定义为平均寿命$\beta:=\E(X)\equiv 1/\lambda$,因此也有教材将指数分布记作$\Expo(\beta)$。
\end{corollary}

\begin{corollary}
	指数分布的尾概率与Poisson分布有关:
	\[
		\P_{\Expo(\lambda)}(T>\tau)=\e{-\lambda\tau}=\P_{\Pois(\lambda\tau)}(N=0).
	\]
	这说明:如果时间$(0,\tau)$内事件的发生次数$N\sim\Pois(\lambda\tau)$,则相邻事件的时间间隔$\D T\sim\Expo(\lambda)$。
\end{corollary}

\begin{example}{指数分布的导出}{}
	产品在$(x,x+\d x)$时间内的失效率为
	\[
		\P(x<X<x+\d x|X>x)=\frac{\P(x<X<x+\d x)}{\P(X>x)}=\frac{\CDF(x+\d x)-\CDF(x)}{1-\CDF(x)}=\frac{\CDF'(x)}{1-\CDF(x)}\d x
	\]
	令失效率为$\lambda(x)\d x$
	\[
		\frac{\CDF'(x)}{1-\CDF(x)}=\lambda(x),\implies\CDF(x)=1-\exp\biggfkh{-\int_0^x\lambda(t)\d t}.
	\]
	若假设无老化:$\lambda(t)\equiv\lambda$,则分布为指数分布:
	\begin{equation}
		\CDF(x)=1-\e{-\lambda x},\quad x>0,
	\end{equation}

	这体现出指数分布的无记忆性:$\forall\tau>0$
	\[
		\P(X>t+\tau|X>\tau)=\frac{\e{-\lambda(t+\tau)}}{\e{-\lambda\tau}}=\e{-\lambda t}=\P(X>t)
	\]

	\tcblower
	
	改进$\lambda(x)=\alpha x^{\alpha+1}/\beta^\alpha$,得到Weibull分布
	\begin{equation}
		\label{eq:Weibull CDF}
		\CDF(x)=1-\e{-(x/\beta)^\alpha},\quad x>0.
	\end{equation}
\end{example}

\section{随机变量的函数}

\begin{theorem}
	{离散型随机变量的函数}{}
	$X$离散$\implies Y:=g(X)$离散。
\end{theorem}

\begin{theorem}{连续型随机变量的函数}{}
	若$g$处处可导且严格单调,则$Y=g(X)$的PDF为
	\begin{equation}
		f_Y(y)=f_X\bigkh{g^{-1}(y)}\abs{\fkh{g^{-1}(y)}'}.
	\end{equation}
	其本质是
	\begin{equation}
		\CDF_Y(y)=\CDF_X\bigkh{g^{-1}(y)}.
	\end{equation}
\end{theorem}
\begin{example}{生成随机变量}{}
	服从CDF $\CDF(y)$的随机变量$Y$可由随机数$X\sim\Unif(0,1)$生成:
	\begin{equation}
		Y=\CDF\inv(X).
	\end{equation}
\end{example}
\section*{Review}
\begin{compactenum}
	\item PMF/PDF, CDF
	\item 期望$\mu$、标准差$\sigma$;标准化$\frac{X-\mu}\sigma$
	\item 参数的意义:位置、尺度、形状
	\item $Y=g(X)$
\end{compactenum}
\chapter{Bose系统和Fermi系统}
理想气体满足非简并条件$\e\alpha\gg 1$,即
\[
	\e\alpha=\frac1n\kh{\frac{2\pi m\kB T}{h^2}}^{3/2}=\frac1{n\lambda_T^3}\gg 1,
\]
对应$\kB T$时的de Broglie波长
\begin{align}\label{T-deB}
	\lambda_T=\frac{h}{\sqrt{2\pi m\kB T}}.
\end{align}
而气体不满足非简并条件,$n\lambda_T^3\not\ll 1$,其分布就是Bose分布和Fermi分布%,对应平衡分布
\begin{align}
	a_i=\frac{\omega_i}{\e{\alpha+\beta\varepsilon_i}\pm 1},
\end{align}
$\pm$号的($+$)对应Fermi分布,($-$)对应Bose分布。
\begin{definition}{巨配分函数}{Grand Partition Function}
	定义巨配分函数的对数
	\begin{align}
		\ln\Xi(\alpha,\beta,y):=\pm\sum\omega_i\ln\kh{1\pm\e{-\alpha-\beta\varepsilon_i}},
	\end{align}
\end{definition}
可得
\begin{gather}
	N=-\pv{\ln\Xi}{\alpha};\\ % =\sum\frac{\omega_i}{\e{\alpha+\beta\varepsilon_i}\pm 1}
	E=-\pv{\ln\Xi}{\beta}.
\end{gather}
物态方程
\begin{align}
	Y_k=-\frac1\beta\pv{\ln\Xi}{y_k}.
\end{align}

再来确定熵,由
\[
	\d E=T\d S+\sum Y_k\d y_k+\mu\d N,
\]
于是
\[
	T\d S=-\d\kh{\pv{\ln\Xi}\beta}+\frac1\beta\sum\pv{\ln\Xi}{y_k}\d y_k-\mu\d N.
\]
利用
\[
	\d\ln\Xi=\pv{\ln\Xi}\alpha\d\alpha+\pv{\ln\Xi}\beta\d\beta+\sum\pv{\ln\Xi}{y_k}\d y_k
\]
消去求和项,可得
\[
	T\d S=\frac1\beta\d\kh{\ln\Xi-\alpha\pv{\ln\Xi}\alpha-\beta\pv{\ln\Xi}\beta}-\kh{\mu+\frac\alpha\beta}\d N.
\]
对封闭系统,$\d N\equiv 0$
\[
	\d S=\frac1{\beta T}\d\kh{\ln\Xi-\alpha\pv{\ln\Xi}\alpha-\beta\pv{\ln\Xi}\beta},
\]
系数对应Boltzmann常数$\kB$,
因而对开放系统
\begin{align}\label{Bose-mu}
	\alpha=-\beta\mu=-\frac\mu{\kB T}.
\end{align}
因此熵
\begin{align}
	S=\kB\kh{\ln\Xi-\alpha\pv{\ln\Xi}\alpha-\beta\pv{\ln\Xi}\beta}{\color[gray]{.8}-\,S'},
\end{align}

另一方面,由Boltzmann关系
\begin{align*}
	S & =\kB\ln\Om[F;B]\simeq\kB\sum\fkh{a_i\ln\kh{\frac{\omega_i}{a_i}\mp 1}\mp\omega_i\ln\kh{1\mp\frac{a_i}{\omega_i}}} \\
	  & =\kB\sum\fkh{a_i\kh{\alpha_\beta\varepsilon_i}\pm\omega_i\ln\kh{1\pm\e{-\alpha-\beta\varepsilon_i}}}                             \\
	  & =\kB\kh{\alpha N+\beta E+\ln\Xi}=\kB\kh{\ln\Xi-\alpha\pv{\ln\Xi}\alpha-\beta\pv{\ln\Xi}\beta}.
\end{align*}
因此$S'=0$。
\section{弱简并理想Bose气体和Fermi气体}
弱简并条件$n\lambda_T^3<1$,宏观量可对$n\lambda_T^3\equiv\e{-\alpha}$展开。\exmref{exm:Monatomic Molecule} 已给出单原子气体平动:
\begin{align}
	g(\varepsilon)\d\varepsilon=g_s\frac{2\pi V}{h^3}(2m)^{3/2}\sqrt\varepsilon\d\varepsilon,
\end{align}
因此弱简并单原子Bose气体和Fermi气体的巨配分函数的对数
\begin{align*}
	\ln\Xi(\alpha,\beta,V) & =\pm\int\zti g(\varepsilon)\ln\kh{1\pm\e{-\alpha-\beta\varepsilon}}\d\varepsilon\\
	&=\pm 2\pi g_sV\kh{\frac{2m}{h^2}}^{3/2}\int\zti\sqrt\varepsilon\ln\kh{1\pm\e{-\alpha-\beta\varepsilon}}\d\varepsilon,
\end{align*}

由于$\e{-\alpha}<1$,用展开式
\[
	\ln(1+ x)=\sum_{n=1}^\infty\frac{x^n}n,\quad\abs x<1,
\]
展开
\begin{align*}
	\int\zti\sqrt\varepsilon\ln\kh{1\pm\e{-\alpha-\beta\varepsilon}}\d\varepsilon=\pm\sum_{n=1}^\infty\frac{(\mp)^{n-1}}n\int\zti\sqrt\varepsilon\e{-n(\alpha+\beta\varepsilon)}\d\varepsilon \\
	=\pm\sum_{n=1}^\infty\frac{(\mp )^{n-1}}n\e{-n\alpha}\frac{\sqrt\pi}{2(n\beta)^{3/2}}=\pm\frac{\sqrt\pi}{2\beta^{3/2}}f(\alpha).
\end{align*}
其中
\[
	f(\alpha)=\sum_{n=1}^\infty(\mp)^{n-1}n^{-5/2}\e{-n\alpha}=\e{-\alpha}\mp 2^{-5/2}\e{-2\alpha}+\cdots.
\]
故
\begin{align}
	\ln\Xi=\pm g_sV\kh{\frac{2\pi m}{h^2\beta}}^{3/2}f(\alpha)=\frac{g_sV}{\lambda_T^3}f(\alpha).
\end{align}

反解出$\alpha$
\[
	N=-\pv{\ln\Xi}\alpha=-\frac{g_sV}{\lambda_T^3}f'(\alpha),
\]
因此
\[
	\xi:=\frac{n\lambda_T^3}{g_s}=-f'(\alpha)=\e{-\alpha}\mp 2^{-3/2}\e{-2\alpha}+\cdots
\]
进而
\(\e{-\alpha}=\xi\pm{2^{-3/2}}\xi^2+\cdots,\)
%\[\alpha=-\ln\frac{n\lambda_T^3}{g_s}\mp\e{-3/2}\frac{n\lambda_T^3}{g_s}+\cdots\]
\[
	f(\alpha)=\xi\pm 2^{-5/2}\xi^2+\cdots.
\]
宏观量
\begin{align}\notag
	E & =-\pv{\ln\Xi}\beta=-\ln\Xi\pv{\ln\ln\Xi}\beta=\frac32\frac{\ln\Xi}\beta \\
	  & =\frac32N\kB T\kh{1\pm 2^{-5/2}\xi+\cdots}.
\end{align}
比热
\begin{align}
	C_V=\kh{\pv ET}_V=\frac32N\kB\kh{1\mp 2^{-7/2}\xi+\cdots}.
\end{align}
物态方程
\begin{align}
	p=\frac1\beta\pv{\ln\Xi}V=\frac{\ln\Xi}{\beta V}=\frac23\frac EV=\frac{N\kB T}V\kh{1\pm 2^{-5/2}\xi+\cdots}.
\end{align}
熵
\begin{align}\notag
	S&=\kB\kh{\ln\Xi-\alpha\pv{\ln\Xi}\alpha-\beta\pv{\ln\Xi}\beta}=\kB\kh{\frac53\beta E+N\alpha}\\
	&=N\kB\fkh{\kh{\frac52-\ln\xi}\pm 2^{-7/2}\xi+\cdots}.
\end{align}
\paragraph{讨论}弱简并条件($n\lambda_T^3<1$)下,$E,p,S$:Fermi $>$半经典$>$ Bose;$C_V$反之。而强简并条件下,Bose气体和Fermi气体性质完全不同。
\section{Bose-Einstein凝聚}
Bose气体的化学势满足
\[
	a_i=\frac{\omega_i}{\e{\beta(\varepsilon_i-\mu)}-1}.
\]
$a_i\geqslant 0$,故$\varepsilon_i\geqslant \mu$。
取$\varepsilon_0=0$,则$\mu\leqslant 0$
\[
	N=\sum_i\frac{\omega_i}{\e{\beta(\varepsilon_i-\mu)}-1}.
\]
随着温度的降低,化学势增加。直到相变点$T_\crt$,$\mu=0$.

计算$T_\crt$,单原子分子能量准连续
\begin{align*}
	N&=\int\zti\frac{g(\varepsilon)\d\varepsilon}{\e{\beta_\crt\varepsilon}-1}=2\pi g_sV\kh{\frac{2m}{h^2}}^{3/2}\int\zti\frac{\sqrt\varepsilon\d\varepsilon}{\e{\beta_\crt\varepsilon}-1}\\
	&=2\pi g_sV\kh{\frac{2m}{h^2\beta_\crt}}^{3/2}\cdot\Gamma\kh{\frac32}\zeta\kh{\frac32}.
\end{align*}
由
% \[\int\zti\frac{x^{n-1}}{\e x-1}\d x=\Gamma(n)\zeta(n).\]
$\Gamma(3/2)=\sqrt\pi/2,\;\zeta(3/2)=2.612$可得
\begin{align}\label{Bose-TC}
	T_\crt=\frac{h^2}{2\pi m\kB}\kh{\frac n{2.612g_s}}^{2/3}.
\end{align}
$T\to T_\crt$时,$\mu\to0$,基态上的粒子数显著增加;另一方面,准连续近似时$g(\varepsilon)\propto\sqrt\varepsilon$忽略了$\varepsilon=0$态。

故激发态中应将$N$分为基态$N_0$和激发态$N_+$两部分,$N_+$部分推导与之前相同
\[
	\frac{N_+}N=\kh{\frac{\beta_\crt}\beta}^{3/2}=\kh{\frac T{T_\crt}}^{3/2}.
\]
因此基态
\begin{align}
	N_0=N\fkh{1-\kh{\frac T{T_\crt}}^{3/2}}.
\end{align}
当$T<T_\crt$降低时,$N_0$不断增多;$T\to 0$时$N_0\to N$,越来越多的粒子处于基态,称为\textbf{Bose-Einstein凝聚}。这个凝聚可看做动量空间的凝聚。

\paragraph{凝聚后的宏观现象}$\varepsilon=0$粒子
\[
	E=0,\;p=0,\;G=N\mu=E+pV-TS=0.
\]
对$E$等无贡献,起粒子源作用;宏观量子态。

$\varepsilon>0$粒子的贡献,注意$\alpha=0$
\begin{align}\notag
	\ln\Xi&=-\int\zti g(\varepsilon)\ln\kh{1-\e{-\beta\varepsilon}}\d\varepsilon\\\notag
	%=\frac23CV\beta^{-3/2}\int\zti\frac{x^{3/2}\d x}{\e{-x}-1}
	&=2\pi g_sV\kh{\frac{2m}{h^2\beta}}^{3/2}\cdot\frac23\int\zti\frac{x^{3/2}\d x}{\e x-1}\\
	&=\zeta\kh{\frac52}\cdot g_sV\kh{\frac{2\pi m}{h^2\beta}}^{3/2}.% \cdot\frac{3\sqrt\pi}4\cdot 1.341.
\end{align}
故
\begin{gather}
	E=-\pv{\ln\Xi}\beta=0.770N\kB T\kh{\frac T{T_\crt}}^{3/2},\\
	p=\frac1\beta\pv{\ln\Xi}V\propto m^{3/2}g_sT^{5/2},\\
	S=\kB\kh{\ln\Xi+N\alpha+\beta E}\propto m^{3/2}g_sVT^{3/2},\\
	C_V=\kh{\pv ET}_V=1.926N\kB\kh{\frac T{T_\crt}}^{3/2}.
\end{gather}
\paragraph{讨论:}
\begin{compactenum}
	\item $T\to 0$时,$E,p,S\to0$
	\item $C_V$在相变点前后的变化
	\begin{center}
		\begin{tikzpicture}
			\coor 5{4.5}{T/T_\crt}{C_V/N\kB};
			\draw[thick,dashed](0,2*1.926)node[left]{1.926}--(1,2*1.926)--(1,0)node[below]{1};
			\draw[thick,dashed](0,3)node[left]{1.5}--(5,3);
			\draw[thick,domain=0:1]plot(\x,{2*1.926*\x^1.5});
			\draw[thick,domain=1:5]plot(\x,{3+(2*1.926-3)/(\x^1.5)});
		\end{tikzpicture}
		\captionof{figure}{热容$C_V$随温度的变化}
	\end{center}
	\item $p\vs  V$
		\subitem 半经典极限$pV=N\kB T$;
		\subitem 凝聚时$p\propto T^{3/5}$与$V$无关。
	\item 凝聚体积$V_\crt$,由式\eqref{Bose-TC}知
	\[
	T=\frac{h^2}{2\pi m\kB}\kh{\frac{N/V_\crt}{2.612g_s}}^{2/3}.
\]
	因此 
	\begin{align}
		V_\crt=\frac{N}{2.612g_s}\kh{\frac{h^2}{2\pi m\kB T}}^{3/2}=\frac{N\lambda_T^3}{2.612g_s}.
	\end{align}
\end{compactenum}
由于历史条件,当时还不知道全同多粒子系存在(量子起源的)统计关联:对Bose子是有效吸
引;而Fermi子是有效排斥。因此,即使没有动力学相互作用,仍可在一定条件下由于有效相互
作用而发生凝聚现象。这是一种纯粹量子起源的相变。

实现Bose-Einstein凝聚极其困难,原则上要使气体冷却至
$\lambda_T\geqslant\avg d$,
但大多数情况下,在远高于BEC的$T_\crt$到达以前,已发生液化甚至固化的相变。为了实现原子气体的BEC,必须用极稀薄的气体,且要求
\begin{center}
	二体弹性碰撞的弛豫时间$\ll$形成分子集团的非弹性碰撞的弛豫时间
\end{center}
对于碱金属原子气体,前者$\sim\SI{10}\ms$,而后者有几秒至几分钟。%$\tau_\mathrm{elas}\ll$$\tau_\mathrm{inelas}$

BEC-BCS Crosssover Fermionic condensation.
\paragraph{液He}$T_\crt=\SI{2.17}\K$,$T<T_\crt$时的液He II具有超流性。

$T=T_\crt$时,比热趋于无穷,$C_T\vs T$曲线形似$\lambda$,故称$\lambda$相变。
\section{光子气体}
光子是一种特殊的Bose子,严格来说,光子没有Bose-Einstein凝聚\footnote{广义上来说,赋予光子以质量是可以发生BEC的。}。讨论黑体辐射,$T,V$给定,满足相对论关系
\begin{align}
	\varepsilon=h\nu=cp.
\end{align}
光子间无相互作用,符合理想气体。%光子自旋$s=1$,简并度$g_s=2$;
光子质量为0,因此$\lambda_T\to\infty$,且光子数不守恒,没有$\alpha$
\[
	a_i=\frac{\omega_i}{\e{\beta\varepsilon_i}-1}.
\]
\paragraph{黑体辐射公式}能完全吸收照射到它上面的各种波长的电磁波的物体,称为黑体。当$V$很大时,能量准连续,$(\nu,\nu+\d\nu)$内状态数
\[
	g(\nu)\d\nu=\frac{g_sV}{h^3}4\pi\kh{\frac{h\nu}c}^2\frac{h\d\nu}{c}=\frac{4\pi g_sV}{c^3}\nu^2\d\nu;
\]
光子数 
\[
	n(\nu)\d\nu=\frac{g(\nu)\d\nu}{\e{\beta h\nu}-1},
\]
光子$g_s=2$,能量 
\begin{align}
	u(\nu)\d\nu=\frac{n(\nu)}Vh\nu\d\nu=\frac{8\pi\nu^2}{c^3}\frac{h\nu}{\e{\beta h\nu}-1}\d\nu.
\end{align}
上式即Planck定律。

低频高温下,$h\nu\ll\kB T$,变为经典的\Rayl-Jeans定律
\[
	u(\nu)\d\nu\simeq\frac{8\pi\nu^2}{c^3}\kB T\d\nu.
\]
高频低温极限,变成Wein定律
\[
	u(\nu)\d\nu\simeq\frac{8\pi h\nu^3}{c^3}\e{-\beta h\nu}\d\nu.
\]

辐射场总能量
\[
	u=\int\zti u(\nu)\d\nu=\frac{8\pi\kB^4}{h^3c^3}\int\zti\frac{x^3\d x}{\e x-1}=\frac{8\pi^5\kB^4}{15h^3c^3}T^4.
\]
辐射通量密度
\begin{align}
	J=\frac c4u=\sigma T^4.
\end{align}
其中$\sigma=\SI{5.6704e-8}{\W\per\m\squared\per\K\squared}.$及Stefan-Boltzmann定律。

若将能量密度按波长分布
\[
	u(\lambda)\d\lambda=\frac{8\pi h}{\lambda^3}\frac{1}{\e{\beta hc/\lambda}-1}\frac c{\lambda^2}\d\lambda.
\]
其极大值满足Wein位移定律
\begin{align}
	\lambda_\mathrm mT=\frac{hc}{4.96\kB}=\SI{2.89777}{\mm\K}.
\end{align}
\paragraph{热力学}
\[
	g(\varepsilon)\d\varepsilon=2\cdot\frac{4\pi V}{h^3}\frac{\varepsilon^2\d\varepsilon}{c^3}.
\]
配分函数
\begin{align*}
	\ln\Xi(\beta,V)&=-\int\zti g(\varepsilon)\ln\kh{1-\e{-\beta\varepsilon}}\d\varepsilon\\
	&=-\frac{8\pi V}{h^3c^3\beta^3}\int\zti x^2\ln(1-\e x)\d x=\frac{8\pi^5V}{45h^3c^3\beta^3}.
\end{align*}

能量
\[
	E=-\pv{\ln\Xi}\beta=\frac{8\pi^5V}{15h^3c^3\beta^4}=:bVT^4.
\]
与前面一致。而比热
\[
	C_V=\kh{\pv ET}_V=4bVT^3,
\]
随着温度上升而增加,因为光子数不守恒。

压强
\[
	p=\frac1\beta\pv{\ln\Xi}V=\frac13\frac EV=\frac13bT^4.
\]
熵等热力学量
\begin{gather*}
	S=\kB\kh{\ln\Xi-\beta E}=4\kB\ln\Xi=\frac43bVT^3;\\
	F=U-TS=-\frac13U;\\
	G=F+pV=0,\implies\mu=0.
\end{gather*}
与光子数不守恒对应。
\section{声子气体}
在Einstein模型中,我们将固体晶格振动简谐近似为独立的简谐振子,频率$\nu$,量子数为$n$的振子激发态相当于产生了$n$个能量为$h\nu$的粒子,称为声子。

声子气体不可分,符合Bose分布,且声子数不守恒
\[
	a_i=\frac{\omega_i}{\e{\beta h\nu_i}-1}.
\]

Einstein模型定量不符,因为忽略了低频振动,而低温下的热激发主要在低频(长波)部分,当波长$\gg$原子间距时,可看做$0-\omega_\Db$的连续谱。


声波分为横波(transverse)和纵波(longitudinal),速度分别为$v_\tv$和$v_\lt$;横波有两种振动方式,纵波只有一种。
\[
	\varepsilon=\hbar\omega,\quad p=\hbar k;\quad \omega=kv.
\]
纵波声子状态数
\[
	\frac V{h^3}\cdot 4\pi p_\lt^2\d p_\lt=\frac V{2\pi^2v_\lt^3}\omega^2\d\omega.
\]
横波同理,故总状态数
\[
	g(\omega)\d\omega=\frac V{2\pi^2}\kh{2v_\tv^{-3}+v_\lt^{-3}}\omega^2\d\omega=:B\omega^2\d\omega.
\]
由
\[
	3	N=\int_0^{\omega_\Db}g(\omega)\d\omega=\frac B3\omega_\Db^3,\implies B=\frac{9N}{\omega_\Db^3}.
\]
可得
\begin{align*}
	g(\omega)=\begin{cases}
		9N\omega^2/\omega_\Db^3,&0\leqslant\omega\leqslant\omega_\Db\\
		0,&\omega>\omega_\Db
	\end{cases}
\end{align*}
能量 
\begin{align*}
	E&=E_0+\int\zti\frac{ g(\omega)\hbar\omega}{\e{\beta\hbar\omega}-1}\d\omega=E_0+\frac{9N\hbar}{\omega_\Db^3}\int_0^{\omega_\Db}\frac{\omega^3\d\omega}{\e{\beta\hbar\omega}-1}
\end{align*}
取Debye温度
\[
	\theta_\Db:=\frac{\hbar\omega_\Db}\kB\sim\SI{200}\K
\]
并取$y=\theta_\Db/T=\beta\hbar\omega_\Db$
\begin{gather}
	E=E_0+3N\kB T\Debye(y).\\
	C_V=3N\kB\fkh{4\Debye(y)-\frac{3y}{\e{y}-1}}.
\end{gather}
其中Debye函数
\[
	\Debye(y)=\frac 3{y^3}\int_0^{y}\frac{x^3\d x}{\e x-1}.
\]

高温极限$y\ll 1$
\begin{align*}
	\Debye(y)=\frac3{y^3}\int_0^yx^2-\frac{x^3}2+\bigo(x^4)\d x=1-\frac38y+\bigo(y^2).
\end{align*}
\[
	E\simeq E_0+3N\kB T,\quad C_V\simeq 3N\kB.
\]
低温极限$y\gg 1$,可认为
\begin{gather}
	\notag
	\Debye(y)=\frac3{y^3}\int\zti\frac{x^3\d x}{\e x-1}=\frac{\pi^4}{5y^3}.\\
	\label{eqn:CV-Debye}
	C_V=3N\kB\frac{4\pi^4}5\kh{\frac T{\theta_\Db}}^3\propto T^3.
\end{gather}
与试验符合。
\begin{compactenum}
	\item 固体中原子作用强,不能直接用近独立粒子统计。$T$较低时,简谐近似成立——原子集体振动的简正模式。
	相互独立:近独立的理想声子气体。
	\item 声子是准粒子,与振动激发态等效的粒子,有能量、动量等,
	%但不同于电子等,
	只存在于固体中,$\varepsilon$与$p$的关系(色散关系)可不同于普通粒子。
	\item 实际固体比热:金属、自由电子气贡献。

	化合物的分子间振动为声频,适用Debye模型;分子内振动为光频,适用Einstein模型。
\end{compactenum}
\section{Fermi气体}
讨论简并费米气体的低温性质,$n\lambda_T^3\geqslant 1$,相互作用弱。
\[
	a_i=\frac{\omega_i}{\e{\alpha+\beta\varepsilon_i}+1}.
\]
能级$\varepsilon_i$的每个量子态上的平均粒子数
\[
	f_i:=\frac{a_i}{\omega_i}=\frac1{\e{\alpha+\beta\varepsilon_i}+1}.
\]
\paragraph{完全Fermi气}由Pauli原理,粒子不能都处于$\varepsilon=0$态,但尽可能低,即存在$\varepsilon_\Fm$:当$\varepsilon<\varepsilon_\Fm$时,各量子态各有一个粒子;而$\varepsilon>\varepsilon_\Fm$时,态无粒子
\[
	\lim_{T\to0}f_i=\lim_{T\to0}\frac1{\e{(\varepsilon-\mu)/\kB T}+1}=\begin{cases}
	1,&\varepsilon<\mu(T=0)\equiv\varepsilon_\Fm\\0,&\varepsilon>\varepsilon_\Fm
\end{cases}
\]

单原子为例,能量准连续
\[
	g(\varepsilon)\d\varepsilon=2\pi g_s\kh{\frac{2m}{h^2}}^{3/2}V\sqrt\varepsilon\d\varepsilon=:CV\sqrt\varepsilon\d\varepsilon.
\]
有
\[
	N=\int_0^{\varepsilon_\Fm}g(\varepsilon)\d\varepsilon=\frac23CV\varepsilon_\Fm^{3/2}.
\]
故
\begin{align}
	\varepsilon_\Fm=\frac{h^2}{2m}\kh{\frac{3N}{4\pi g_sV}}^{2/3}.
\end{align}
零点能
\begin{align}
	U_0=\int_0^{\varepsilon_\Fm}g(\varepsilon)\varepsilon\d\varepsilon=\frac35N\varepsilon_\Fm.
\end{align}
零点压强
\[
	p_0=-\kh{\pv FV}_T=-\pv{U_0}V=-\dv{\varepsilon_\Fm}V\dv{U_0}{\varepsilon_\Fm}=\frac23\frac{U_0}V.
\]
熵
\[
	S=\kB\ln\Om[F]=0.
\]
\begin{example}{金属中的电子气}{Electron Gas in Metals}
	电子$m_\elc\sim\SI{e-30}\kg$,数密度$\sim\SI{e28}{\per\m\cubed}$, % \tothe{3}
	自旋$g_s=2$,故$\varepsilon_\Fm\sim\SI1\eV$,
	\[
	v_\Fm\sim\sqrt{\frac{2\varepsilon_\Fm}m}\sim\SI[per-mode=symbol]{e6}{\m\per\s}
\]
	压强$p_0\sim\SI{e4}\atm$,这是纯粹的量子效应。
\end{example}
\paragraph{强简并Fermi气}Fermi温度
\[
	T_\Fm:=\frac{\varepsilon_\Fm}\kB.
\]
对于金属电子气,$T_\Fm\sim\SI{e4}\K$。

低温情形$T\ll T_\Fm$,热运动能量小,粒子分布基本不变,只有$\varepsilon_\Fm$附近的粒子可能是跳到高能级态上:
\begin{center}
	\begin{tikzpicture}
		\coor 43{\varepsilon}{f_i};
		\draw[thick,dashed](2,1.25)--(2,0); % node[below]{$\mu$};
		\draw[thick,dashed](1,2.5)--(1,0)node[below]{$\mu-\kB T$};
		\node[left]at(0,2.5){1};
		\draw[thick,domain=0:3]plot(\x,{2.5/(e^(6*\x-12)+1)})node[below]{$\mu+\kB T$};
	\end{tikzpicture}
	\captionof{figure}{强简并Fermi气粒子分布}
\end{center}
定性估计比热$C_V$:相对$T=0$时,能量增量
\[
	\D E\simeq N\frac{\kB T}{\varepsilon_\Fm}\D\varepsilon,\quad\D\varepsilon=\kB T.
\]
比热
\[
	C_V\simeq 2\kB N\frac{\kB T}{\varepsilon_\Fm}\sim T.
\]

单原子,能量准连续,需计算积分
\[
	Q_\ell:=\int\zti f(\varepsilon)\varepsilon^\ell\d\varepsilon.
\]
注意到$f$的特点,可在$\varepsilon=\mu$展开
\begin{align}\notag
	Q_\ell&=\cancel{\edg{\frac{\varepsilon^\ell}{\ell+1}f(\varepsilon)}\zti}-\frac1{\ell+1}\int\zti f'(\varepsilon)\varepsilon^{\ell+1}\d\varepsilon % =:\int\zti f'(\varepsilon)\upsilon(\varepsilon)
	\d\varepsilon\\
	&=\sum_{n=0}^\infty\frac{\upsilon^{(n)}(\mu)}{n!}\int\zti f'(\varepsilon)(\varepsilon-\mu)^n\d\varepsilon,\quad\upsilon(\varepsilon):=-\frac{\varepsilon^{\ell+1}}{\ell+1}.
\end{align}
令$\eta:=\beta(\varepsilon-\mu)$,则
\[
	f(\varepsilon)=\frac1{\e\eta+1},\quad f'(\varepsilon)=-\frac{\beta\e\eta}{(\e\eta+1)^2}
\]
故
\[
	Q_\ell=-\sum_{n=0}^\infty\frac{\upsilon^{(n)}(\mu)}{n!\beta^n}\int_{-\beta\mu}^{+\infty}\frac{\eta^n\e\eta}{(\e\eta+1)^2}\d\eta.
\]

低温下,积分下限$-\beta\mu\to-\infty$
\begin{align*}
	Q_\ell&\simeq-\sum_{n=0}^\infty\frac{\upsilon^{(n)}(\mu)}{n!\beta^n}\int\iti\frac{\eta^n\e\eta}{(\e\eta+1)^2}\d\eta\\
	&=-\fkh{\upsilon(\mu)+\frac{\upsilon''(\mu)}{2\beta^2}\frac{\pi^2}3+\cdots}.
\end{align*}
故
\begin{align}
	N=CVQ_{1/2}&=\frac23CV\mu^{3/2}\fkh{1+\frac{\pi^2}8\alpha^{-2}+\bigo\!\kh{\alpha^{-4}}};\\
	U=CVQ_{3/2}&=\frac25CV\mu^{5/2}\fkh{1+\frac{5\pi^2}8\alpha^{-2}+\bigo\!\kh{\alpha^{-4}}}.
\end{align}
其中$\alpha^{-1}(T)=-\frac1{\beta\mu}=\frac{\kB T}{\mu}$。

巨配分函数$\ln\Xi=\frac23\beta U$,压强$p=\frac1\beta\pv{\ln\Xi}V$,
熵
\begin{align}\notag
	S&=\kB\kh{\ln\Xi-\alpha\pv{\ln\Xi}\alpha-\beta\pv{\ln\Xi}\beta}\\\notag
	&=\kB\frac4{15}CV\beta^{-3/2}(-\alpha)^{5/2}\fkh{0+\frac{5\pi^2}4\alpha^{-2}+\bigo\!\kh{\alpha^{-4}}}\\
	&=\frac{\pi^2}3CV\mu^{1/2}\kB^2T\fkh{1+\bigo\!\kh{\alpha^{-2}}}.
\end{align}

利用
\[
	N=\frac23CV\varepsilon_\Fm^{3/2}.
\]
结合$\varepsilon_\Fm=\mu_0$反解出$\mu$
\[
	\mu=\mu_0\fkh{1-\frac{\pi^2}{12}\alpha^{-2}+\bigo\!\kh{\alpha^{-4}}},
\]
不同于Bose气体,$\mu$可正可负。

宏观量用可观测量表示
\begin{gather}
	U=U_0\fkh{1+\frac{5\pi^2}{12}\kh{\frac T{T_\Fm}}^2+\bigo(T^4)},\\
	\label{eqn:CV-Fermi}
	C_V=N\kB\cdot\frac{\pi^2}{2}\frac T{T_\Fm}\fkh{1+\bigo(T^2)}.
\end{gather}
电子气对金属热容量的贡献首先由Sommerfeld解决。

因此低温下金属比热的实验值是电子气和晶格振动(Debye模型)共同贡献
\[
	C_V\sim \underset{\text{Fermi}}{c_\elc T}+\underset{\text{Debye}}{c_\vb T^3}.
\]
与实验符合得很好。
\begin{example}{电子比热vs.晶格比热}{}
	低温下,式\eqref{eqn:CV-Debye}给出晶格比热和式\eqref{eqn:CV-Fermi}给出电子气比热分别为
	\[
	C_V^\vb=N\kB\frac{12\pi^4}5\kh{\frac T{\theta_\Db}}^3,\quad C_V^\elc=N\kB\frac{\pi^2}2\frac T{T_\Fm}.
	\]
	对铜,$\theta_\Db\sim\SI{300}\K,\;T_\Fm\sim\SI{8e4}\K$,二者比值
	\[
	\frac{C_V^\elc}{C_V^\vb}=\frac5{24\pi^2}\frac T{T_\Fm}\kh{\frac{\theta_\Db}T}^3\sim\frac8{T^2}.
	\]
\end{example}
\chapter{原子核反应}
通过研究核衰变来认识原子核具有相当大的局限性:
\begin{compactitem}
	\item 只涉及不稳定核素向稳定核素的转变,大量稳定核素并不发生衰变;
	\item 是自发过程,不涉及核与核、核与其它粒子的相互作用;
	\item 核衰变仅限于几个MeV的低激发能级,而在高能量范围的核现象会更加丰富多彩……
\end{compactitem} 

\begin{definition}{原子核反应}{nuclear reaction}
	用具有一定能量的粒子(核子、原子核、$\gamma$射线或电子)轰击靶核,使其组成或能量状态发生变化,成为不稳定核素,并放出粒子的过程。
\end{definition}
与自发的衰变不同,核反应是被诱发的过程。

考虑一个典型的核反应
\begin{align}\label{AabB}
	a+\nuc A\to b+\nuc B
\end{align}
其中$a$是入射粒子,$\nuc A$是靶核,$\nuc B$是余核,$b$是出射粒子,这个反应我们可以记作
\[
	\reac AabB,
\]
其中$(a,b)$可以代表一类反应:$(\alpha,\nton),(\nton,\gamma)$等。

实现核反应的粒子来源:
\begin{compactitem}
	\item 放射源:$\alpha,\beta,\gamma,\nton$
	\item 宇宙射线:高能质子、$\nucli4{He}$、中子;
	\item 加速器:质子、中子、重离子、X/$\gamma$射线;
	\item 反应堆:中子、$\gamma$、中微子等。
\end{compactitem}
\section{原子核反应概况}
1919年,Rutherford实现了历史上第一个人工核反应
\[
	\alpha+\nucli{14}N\to\nucli{17}O+\pton.
\]

1932年,John. Cockcroft和Ernest Walton加速质子轰击锂靶,实现了第一个在加速器上实现的核反应
\[
	\pton+\nucli 7{Li}\to\alpha+\alpha.
\]

1934年,Curie夫妇产生第一个人工放射性核素
\begin{gather*}
	\alpha+\nucli{27}{Al}\to\nucli{30}P+\nton,\\
	\nucli{30}P\decayto{\SI{2.5}{min}}\nucli{30}{Si}+\elc^++\nu_\elc,\\
	\elc^++\elc^-\to\gamma+\gamma.
\end{gather*}

1932年,Chadwick发现中子
\[
	\alpha+\nucli9{Be}\to\nucli{12}C+\nton.
\]
\paragraph{核反应分类}
按出射粒子分类:
\begin{compactitem}
	\item 核散射:出射粒子和入射粒子是同种粒子;
	
	分为弹性散射$\reac AaaA$和非弹性散射$\reac Aa{a'}{A^\ast}$。
	\item 核转变:出射粒子和入射粒子不同。
	
	若出射粒子有$\gamma$,也叫辐射俘获;若入射粒子有$\gamma$,则称之为光核反应。
\end{compactitem}
按入射粒子分类:中子核反应、带电粒子核反应、光核反应。

按能量分类……一般的原子核物理只涉及低能核反应。
\paragraph{反应道}核反应的入射道和出射道统称为反应道。

各反应道的发生几率是不同的:随入射粒子能量变化,与核反应机制、核结构有关,受守恒条件约束。

\paragraph{核反应中的守恒}电荷、核子数、动量、能量、角动量、宇称守恒。
\section{核反应能和\textit{Q}方程}
核反应\eqref{AabB}中能量守恒
\[
	(m_a+m_{\nuc A})c^2+T_a+T_{\nuc A}=(m_b+m_{\nuc B})c^2+T_b+T_{\nuc B}
\]
核反应过程释放出的能量,称为反应能$Q$
\begin{align}\notag
	Q&=(T_b+T_{\nuc B})-(T_a+T_{\nuc A})\\
	&=(m_a+m_{\nuc A})c^2-(m_b+m_{\nuc B})c^2.
\end{align}
\paragraph{$Q$方程}假设靶核静止$T_{\nuc A}=0$,由动量守恒
\[
	\bm P_a=\bm P_b+\bm P_{\nuc B}.
\]
\begin{center}
	\includegraphics[page=8]{figures/tikz/layouts.pdf}
	\captionof{figure}{核反应动量守恒}
\end{center}
$b$的出射角$\theta$,由余弦定理,不易测量的$T_\nuc B$可被$P_a,P_b,\theta$表示
\[
	P_\nuc B^2=P_a^2+P_b^2-2P_aP_b\cos\theta.
\]
对于低能核反应,用非相对论公式$P^2=2mT$
%Q=\kh{1+\frac{m_b}{m_{\nuc B}}}T_b-\kh{1-\frac{m_a}{m_{\nuc B}}}T_a-\frac{2\sqrt{m_am_bT_aT_b}}{m_{\nuc B}}\cos\theta
\begin{align}\notag
	Q&=T_b+T_{\nuc B}-(T_a+0)\\
	&=\kh{1+\frac{m_b}{m_{\nuc B}}}T_b-\kh{1-\frac{m_a}{m_{\nuc B}}}T_a-\frac{2\sqrt{m_am_bT_aT_b}}{m_{\nuc B}}\cos\theta
\end{align}
进而可得到出射粒子$b$的能量-角度关系
\begin{align}
	T_b=\hkh{\frac{\sqrt{m_am_bT_a}}{m_{\nuc B}+m_b}\cos\theta\pm\sqrt{\kh{\frac{m_{\nuc B}-m_a}{m_{\nuc B}+m_b}+\frac{m_am_b\cos^2\theta}{(m_{\nuc B}+m_b)^2}}T_a+\frac{m_{\nuc B}}{m_{\nuc B}+m_b}Q}}^2.
\end{align}
这就是$Q$方程,\index{$Q$方程}将$T_a,T_b,\theta,Q$四个变量联系起来。

当余核B处于激发态的时候,反应能$Q$也会发生相应的改变。

\section{核反应的阈值}
\paragraph{实验室系与质心系}
入射粒子$a$和靶核A相对于质心C运动的动能之和,称为$a$的相对运动动能$T'$
\begin{align}
	T'=\frac{m_{\nuc A}}{m_a+m_{\nuc A}}T_a
\end{align}
由于A相对实验室L是静止的,$a$在L系下的动能$T_a$,只有一部分构成了内能项$T'$,能够参与核反应,而整个质心系的动能不能参与核反应。

\paragraph{核反应的阈值}
在L系中能够引起核反应的$a$的最低能量$T_\thres$
\begin{align}
	T_\thres=\frac{m_a+m_{\nuc A}}{m_{\nuc A}}\abs Q.
\end{align}
由于质心系要带走动能,$T_\thres$必然比$\abs Q$大。
\paragraph{出射角转换}
出射粒子在L系和C系中速度的关系
\[
	\bm v_b=\bm v'_b+\bm v\CM
\]
\begin{center}
	\includegraphics[page=9]{figures/tikz/layouts.pdf}
	\captionof{figure}{L系和C系速度关系}
\end{center}
由正弦定理、余弦定理等关系
\[
	\begin{cases}
		\frac{v_b'}{\sin\theta\LAB}=\frac{v\CM}{\sin\theta\CM-\sin\theta\LAB}\\
		v_b\cos\theta\LAB=v\CM+v_b'\sin\theta\CM\\
		v_b^2=v\CM^2+v_b'^2+2v\CM v_b'\cos\theta\CM
	\end{cases}
\]
定义
\begin{align}
	\gamma:=\frac{v\CM}{v_b'}.
\end{align}
则
\begin{align}
	\begin{cases}
		\theta\CM=\theta\LAB+\arcsin(\gamma\sin\theta\LAB)\\
		\cos\theta\LAB=\frac{\gamma+\cos\theta\CM}{(1+\gamma^2+2\gamma\cos\theta\CM)^{1/2}}
	\end{cases}
\end{align}
不难用$T'$和$Q$解出$\gamma$
\[
	\gamma^2=\frac{m_am_b}{m_\nuc Am_\nuc B}\frac{m_b+m_\nuc B}{m_a+m_\nuc A}\frac{T'}{T'+Q}
\]
由于$m_a+m_\nuc A\doteq m_b+m_\nuc B$
\begin{align}
	\gamma\doteq\sqrt{\frac{m_am_b}{m_\nuc Am_\nuc B}\frac{T'}{T'+Q}}.
\end{align}
\subparagraph{弹性散射}$Q=0,a=b,\nuc A=\nuc B$,故 
\[
	\gamma=\frac{m_a}{m_\nuc A}.
\]
当$m_\nuc A\gg m_a$时,$\gamma\doteq 0$,$\theta\CM=\theta\LAB$;当$m_\nuc A= m_a$时,$\gamma=1$,$\theta\CM=2\theta\LAB$。
\subparagraph{一般情况}当$\gamma<1$时,$v_b'>v\CM$

当$\theta\CM=\theta\LAB=0$时,$v_b$取最大值$v_{b\max{}}=v_b'+v\CM$;当$\theta\CM=\theta\LAB=\pi$时,$v_b$取最小值$v_{b\min{}}=v_b'-v\CM$。
\begin{center}
	\includegraphics[page=10]{figures/tikz/layouts.pdf}
	\captionof{figure}{$\gamma<1$}
\end{center}

当$\gamma>1$时,$v_b'<v\CM$。会出现圆锥效应:
一个$\theta\LAB$对应两个$\theta\CM$
\[
	\theta_{\mathrm L\max{}}=\arcsin\gamma\iv.
\]
\begin{center}
	\includegraphics[page=11]{figures/tikz/layouts.pdf}
	\captionof{figure}{$\gamma>1$圆锥效应}
\end{center}

\section{核反应截面和产额}
核反应截面$\sigma$的物理意义为一个入射粒子与单位面积上的靶核发生反应的概率。截面的大小与$a,\nuc A$和$T_a$有关。\index{反应截面}

截面的量纲为面积,单位为$\si{barn}$
\begin{align}
	\SI{1}{barn}=\SI{e-24}{cm^2}
\end{align}
与原子核的截面大小相当。

核反应产物的出射可能各向异性,这就会定义微分截面,
\begin{align}
	\sigma(\theta,\phi):=\dv\sigma\Omega(\theta,\phi)
\end{align}
实验测量微分截面,积分可得到总截面。
\[
	\sigma=\int_0^{2\pi}\int_0^\pi\sigma(\theta,\phi)\sin\theta\d\theta\nd\phi=2\pi\int_0^\pi\sigma(\theta)\sin\theta\d\theta.
\]
\paragraph{反应截面转换}出射粒子数不随坐标系的选择而改变
\[
	\sigma\CM(\theta\CM)\d\Omega\CM=\sigma\LAB(\theta\LAB)\d\Omega\LAB
\]
可得
\[
	\sigma\LAB(\theta\LAB)=\frac{(1+\gamma^2+2\gamma\cos\theta\CM)^{3/2}}{1+\gamma\cos\theta\CM}\sigma\CM(\theta\CM).
\]
\paragraph{核反应产额}定义:入射粒子在靶中引起的核反应数$N'$与入射粒子数$I_0$之比,称为核反应产额(yield) $Y$。\index{反应产额}
\begin{align}
	Y:=\frac{N'}{I_0}.
\end{align}
与反应截面$\sigma$和数密度$N$有关。

通过厚度为$D$的靶后,未发生反应的入射粒子数
\[
	-\frac{\d I}{I}=\sigma N\d x,\implies I=I_0\e{-\sigma ND}
\]
故透射率$T=\e{-\sigma ND},$
\begin{align}
	Y=1-\e{-\sigma ND}.
\end{align}
宏观截面($\si{1/cm}$) $\Sigma:=N\sigma$,平均自由程$\lambda=1/\Sigma$,厚靶$D\gg\lambda,Y=1.$

\sectionstar{核反应中的分波分析}
尽管核反应截面的单位$\si{barn}$与原子核截面相近,但是有时候二者也可以差异巨大。入射粒子带来的轨道角动量有不同的组成,可以根据不同的轨道角动量来分析核反应截面。

\paragraph{半经典的分波分析}入射粒子$a$速度$v_a$,相对质心的运动动能
\[
	T'=\frac12\mu v_a^2,\quad \mu:=\frac{m_am_{\nuc A}}{m_a+m_{\nuc A}}.
\]
相对运动动量
\[
	p=\mu v_a=\frac\hbar\barlambda,\quad\lambdabar:=\frac\hbar{p}.
\]
式中$\lambdabar$是约化de Broglie波长。\\
相对运动角动量
\[
	L=\rho p=\frac\rho\barlambda\hbar=\ell\hbar,\implies\rho=\ell\lambdabar,\quad\ell=0,1,\ldots
\]%ħ

$(a,\nuc A)$的碰撞过程,可以被分解为对应于轨道角动量$L=0,\hbar,2\hbar,\ldots$的部分。轨道角动量为$\ell\hbar$的入射粒子与靶核作用的截面为
\[
	S_\ell=\pi(\rho_{\ell+1}^2-\rho_\ell^2)=(2\ell+1)\pi\lambdabar^2.
\]
最大半径$\rho\maxi=R=R_a+R_{\nuc A}$,进而得到反应截面
\begin{align}
	\sigma=\sum_{\ell=0}^{R/\barlambda}(2\ell+1)\pi\lambdabar^2=\pi(R^2+\lambdabar^2),
\end{align}
核的尺寸和粒子的波动性,都对截面有贡献。
\paragraph{量子力学的分波分析}
向$x$方向入射的粒子束可用平面波表示,在有心力场中,球面波分解更合适
\[
	\psi_\i=\e{\i kx}=\e{\i kr\cos\theta}=\sum_{\ell=0}^\infty(2\ell+1)\i^\ell j_\ell(kr)P_\ell(\cos\theta),
\]
波函数远离原子核时,$kr\gg\ell$,
\[
	j_\ell(kr)\doteq\frac{\sin(kr-\pi\ell/2)}{kr}=\i\frac{\e{-\i(kr-\pi\ell/2)}-\e{\i(kr-\pi\ell/2)}}{kr}.
\]
故
\[
	\psi_\i=\frac1{2kr}\sum_{\ell=0}^\infty\i^{\ell+1}(2\ell+1)\fkh{\e{-\i(kr-\pi\ell/2)}-\e{\i(kr-\pi\ell/2)}}P_\ell(\cos\theta).
\]
其中$\e{-\i(kr-\pi\ell/2)}$代表入射球面波,$\e{\i(kr-\pi\ell/2)}$代表出射球面波。
由于原点上有靶核,散射会导致出射波函数的变化,
\[
	\psi=\frac1{2kr}\sum_{\ell=0}^\infty\i^{\ell+1}(2\ell+1)\fkh{\e{-\i(kr-\pi\ell/2)}-\eta_\ell\e{\i(kr-\pi\ell/2)}}P_\ell(\cos\theta).
\]
其中$\eta_\ell$是出射波系数。
故靶核导致的散射
\begin{align}
	\psi_\sca=\psi-\psi_\i=\frac1{2kr}\sum_{\ell=0}^\infty\i^{\ell+1}(2\ell+1)(1-\eta_\ell)\e{\i(kr-\pi\ell/2)}P_\ell(\cos\theta).
\end{align}

散射截面等于散射粒子数比上入射粒子注量率,因此
\[
	\d\sigma_\sca=\frac{j_\sca r^2\d\Omega}{j_\i},
\]
量子力学中
\[
	j=\frac{\hbar}{2m\i}\kh{\psi\cj\pv\psi{r}-\psi\pv{\psi\cj}r}.
\]
以$\ell=0$为例,计算得到
\[
	j_\sca=\frac{\hbar k}m\frac{\abs{1-\eta_0^2}}{4k^2r^2},\quad j_\i=\frac{\hbar k}m.
\]
故
\[
	\dv{\sigma_{\mathrm{sc},0}}\Omega=\frac{\abs{1-\eta_0^2}}{4k^2}=\frac{\barlambda^2}4\abs{1-\eta_0^2},
\]
把所有角动量均考虑进去
\[
	\dv{\sigma_\sca}\Omega=\frac{\barlambda^2}4\abs{\sum_{\ell=0}^\infty\i(2\ell+1)(1-\eta_\ell)P_\ell(\cos\theta)}^2.
\]
对$4\pi$立体角积分得散射截面
\begin{align}
	\sigma_\sca=\pi\lambdabar^2\sum_{\ell=0}^\infty(2\ell+1)\abs{1-\eta_\ell}^2.
\end{align}
对于反应截面,可以认为$a$消失了,类似的推导得出
\[
	\sigma_{\mathrm{r}}=\pi\lambdabar^2\sum_{\ell=0}^\infty(2\ell+1)(1-\abs{\eta_\ell}^2).
\]

\paragraph{低能中子的散射截面}对于低能入射中子,$\ell=0$,波函数简化为 
\[
	u_{\mathrm o}(r)=\frac\i{2k}\kh{\e{-\i kr}-\eta_0\e{\i kr}}.
\]
若入射粒子与核的作用已知,则核内波函数可知,继而可知核边界处的对数导数
\[
	f:=\edg{\frac r{u_{\mathrm o}}\dv{u_{\mathrm o}}r}_{r=R}\implies\eta_0=\frac{f+\i kR}{f-\i kR}\e{-\i2kR}
\]
当$f\to\infty$时,$\eta=\e{-\i2kR}$;当$f\to0$时,$\eta=-\e{-\i2kR}$,慢中子$kR\ll 1$,故 
\begin{align}
	\sigma_{\mathrm{sc},0}=\begin{cases}
		4\pi R^2,&f\to\infty\\
		4\pi\lambdabar^2,&f\to0
	\end{cases}
\end{align}
$4\pi R^2$对应势(形状)弹性散射截面,$4\pi\lambdabar^2$对应共振散射截面。
\paragraph{散射截面vs反应截面}
\begin{center}
	\includegraphics[page=12]{figures/tikz/layouts.pdf}
	\captionof{figure}{散射截面$\sigma_\sca$与反应截面$\sigma_\mathrm r$的允许范围}
\end{center}
允许有纯的散射过程($|\eta_\ell|=1$),但不允许有纯的反应过程。入射道对出射道总是开放的——入射粒子可以沿入射道返回,因此一定存在弹性散射。
\section{核反应机制及核反应模型}
\paragraph{核反应的三阶段描述}Weisskopf将核反应分为了三个阶段:
\begin{compactenum}
	\item \textbf{独立粒子阶段}:
	
	部分入射粒子被吸收,引起核反应;\\
	部分入射粒子被散射,形成弹性散射
	\item \textbf{复合系统阶段}:
	
	入射粒子与靶核交换能量:
	直接作用;形成复合核;中间过程
	\item \textbf{最后阶段}:
	
	复合系统分解为出射粒子和剩余核。
\end{compactenum}
各种截面有关系,间讲义。
\paragraph{复合核模型}核反应被分成相互独立的两个阶段:
\[
	a+\nuc A\to\nuc C^\ast\to\nuc B+b
\]
\begin{compactenum}
	\item 入射粒子射入靶核,与之形成一个复合核,该核处于激发态;
	\item 激发态的复合核可沿入射道弹性散射,也可能开放其它反应道。
\end{compactenum}
复合核的激发能$E^\ast$是相对动能$T'$和结合能$B_{a\nuc A}$的和
\begin{align}
	E^\ast=T'+B_{a\nuc A}=\frac{m_\nuc A}{m_a+m_\nuc A}T_a+B_{a\nuc A}.
\end{align}
复合核发射核子一般需要$\sim\SI{8}{MeV}$的分离能。尽管$E^\ast\sim\SI{20}{MeV}$,但平均到每个核子的能量只有$\SI{0.2}{MeV}$,故复合核的分离需要经过多次碰撞,其寿命$\sim\SIrange{e-18}{e-14}{s}$,对核反应而言,这是较长的时间。

反应截面
\begin{align}
	\sigma_{ab}=\sigma_{\mathrm{CN}}(T_a)W_b(E^\ast),
\end{align}
故复合核如何衰变是与它如何形成无关,只取决于系统现在的能量状态。
\paragraph{共振}只考虑s波,即$\ell=0$的情况,$\eta_0=\e{\i2\delta_0}$
\[
	\sigma_{\mathrm{sc},0}=\pi\lambdabar^2|1-\eta_0|^2=4\pi\lambdabar^2\sin^2\delta_0(T').
\]
当入射中子的能量达到某些值时,$\delta_0(T')=\pi/2$,散射截面达到极大,就出现了共振。

在$\delta_0(T')=\pi/2$处($T'=E_\mathrm R$)做Talyor展开,略去高阶项
\[
	\sigma_{\mathrm{sc},0}=\pi\lambdabar^2\frac{\Gamma^2}{(T'-E_\mathrm R)^2+\Gamma^2/4}.
\]
势弹性散射和共振弹性散射是复数和的关系,因此存在干涉。
\paragraph{慢中子反应}对于慢中子与一个$A>100$的靶核而言,最可几的复合核退激机制是发射$\gamma$射线,$(\nton,\gamma)$反应是个比较慢的过程
\[
	\sigma_{\nton,\gamma}=\pi\lambdabar^2\frac{\Gamma_\nton\Gamma_\gamma}{(T'-E_\mathrm R)^2+\Gamma^2/4}.
\]
由于约化de Broglie波长
\[
	\lambdabar^2=\frac{\hbar^2}{2\mu T'^2}\propto\frac1{v^2}
\]
低能中子进入势阱,$E_\nton\ll V_0$,并不容易
\[
	P\propto k\propto v,\implies\Gamma_\nton\propto v
\]

慢中子的动能,对衰变行为没什么影响,$T'\ll B_{\nton\nuc A}$,故
\[
	E^\ast=\const,\enspace\Gamma_\gamma=\const,
\]

非共振中子,能量的变化对复合核的衰变不再有什么影响了$\Gamma=\Gamma_\nton+\Gamma_\gamma$,$\Gamma_\nton\ll\Gamma_\gamma$,$\Gamma=\const$,故\index{$1/v$规律}
\begin{align}
	\label{slow-neutron}
	\sigma_{\nton,\gamma}\propto\frac1v.
\end{align}
当我们用慢中子来照射靶核时,中子能量的降低将会有助于$(\nton,\gamma)$反应的发生,该反应的截面反比于中子的速度。

这里的讨论是在“中子能量远离共振能量”这个前提下开展的,如果中子的动能碰巧在共振能量附近,则公式中分母的第一项会起作用,使的截面或大或小的偏离$1/v$规律
\paragraph{连续区理论}连续区——$T'$增加,复合核处于高激发态,能级重叠。能级密度加大,能级宽度加大,不再有单能级共振。
\paragraph{总结}截面测量



\appendix
\section{典型分布}

\subsection{离散分布}

\paragraph{二项分布}$\Bern(n,p)$:$n$次独立Bernuoll实验中的成功次数$X$
\begin{align}
	\P(X=k)={n\choose k}p^k(1-p)^{n-k},\quad k=0,\ldots,n
\end{align}
\paragraph{Pascal分布}$\mathrm{Pa}(r,p)$:独立Bernuoll实验成功$r$次所需要的次数$X$
\begin{align}
	\P(X=k)={k-1\choose r-1}p^r(1-p)^{k-r},\quad k=r,r+1,\ldots
\end{align}
\paragraph{负二项分布}$\mathrm{NB}(r,p)$:独立Bernuoll实验成功$r$次前失败的次数$X$\footnote{显然,Pascal分布和负二项分布只相差一个常数$r$。负二项分布名称中负的来源是\begin{align}
	{k+r-1\choose r-1}=(-)^k{-r\choose k}.
\end{align}}
\begin{align}
	\P(X=k)={k+r-1\choose r-1}p^r(1-p)^k,\quad k=0,1,\ldots
\end{align}
\paragraph{几何分布}$\mathrm{Ge}(p)=\mathrm{Pa}(1,p)$
\begin{align}
	\P(X=k)=p(1-p)^{k-1},\quad k=1,2,\ldots
\end{align}
\paragraph{超几何分布}$\mathrm H(n,N,M)$:$N$件产品中有$M$件不合格品,从中不放回地随机抽取$n$件,则其中含有$X$件不合格品
\begin{align}
	\P(X=k)=\frac{{M\choose k}{N-M\choose n-k}}{{N\choose n}},\quad k=0,1,\ldots,\min(M,n).
\end{align}
\paragraph{负超几何分布}$\mathrm{NH}(r,N,M)$:直到取到$k$个,所需抽样次数$X$
\begin{align}
	\P(X=k)=\frac{{k-1\choose r-1}{N-k\choose M-r}}{{N\choose M}},\quad k=r,\ldots,r+n-M.
\end{align}
\paragraph{Poisson分布}$\Pois(\lambda)$:$\Bern(n,\lambda/n)$当$n\to\infty$的极限
\begin{align}
	\P(X=k)=\frac{\lambda^k}{k!}\e{-\lambda},\quad k=0,1,2,\ldots
\end{align}
\subsubsection{期望、方差}
\begin{center}
	\begin{tabular}{c|c|c}
		\toprule
		Distribution&$\E(X)$&$\Var(X)$\\
		\midrule
		$\Bern(n,p)$&$np$&$np(1-p)$\\
		\midrule
		$\mathrm{Pa}(r,p)$&$\frac rp$&$\frac rp\kh{\frac1p-1}$\\
		\midrule
		$\mathrm{Ge}(p)$&$\frac1p$&$\frac1p\kh{\frac1p-1}$\\
		\midrule
		$\mathrm H(n,N,M)$&$\frac{nM}N$&$\frac{n(N-n)M(N-M)}{N^2(N-1)}$\\
		\midrule
		$\mathrm{NH}(r,N,M)$&$\frac{r(N+1)}{M+1}$&$\frac{r(N+1)(N-M)(M-r+1)}{(M+1)^2(M+2)}$\\
		\midrule
		$\Pois(\lambda)$&$\lambda$&$\lambda$\\
		\bottomrule
	\end{tabular}
\end{center}
\paragraph{二项分布}众数
\begin{gather*}
	\frac{\P(X=k)}{\P(X=k-1)}=\frac{{n\choose k}p}{{n\choose k-1}(1-p)}=\frac{n-k+1}{k}\frac p{1-p}>1,\\
	\implies k<(n+1)p.
\end{gather*}
故
\begin{align*}
	\text{Mode}=\begin{cases}
		n,&p>\frac{n}{n+1}\\
		(n+1)p-1\text{~or~}(n+1)p,&(n+1)p\in\NN\\
		\floor{(n+1)p},&(n+1)p\notin\NN
	\end{cases}
\end{align*}

期望 
\begin{align*}
	\E(X)&=\sum_{k=0}^nk{n\choose k}p^k(1-p)^{n-k}\\
	&=\sum_{k=1}^{n}n{n-1\choose k-1}p^k(1-p)^{n-k}=np.
\end{align*}
\paragraph{负二项分布}众数
\begin{gather*}
	\frac{\P(X=k)}{\P(X=k-1)}=\frac{k-1}{k-r}(1-p)>1,\\
	\implies k<\frac{r-1}{p}-1.
\end{gather*}
故
\begin{align*}
	\text{Mode}=\begin{cases}
		r,&p>\frac{r-1}{r+1}\\
		(n+1)p-1\text{~or~}(n+1)p,&(n+1)p\in\NN\\
		\floor{(n+1)p},&(n+1)p\notin\NN
	\end{cases}
\end{align*}
\paragraph{几何分布}
\paragraph{超几何分布}
\paragraph{Poisson分布}
\subsubsection{PGF, MGF和CF}
\begin{center}
	\begin{tabular}{c|c|c}
		\toprule
		Distribution&$\MGF(X)$&$\CF(X)$\\
		\midrule
		$\Bern(n,p)$&$(q+p\e t)^n$&$(q+p\e{\i t})^n$\\
		\midrule
		$\mathrm{NB}(r,p)$&$\kh{\frac p{q+p\e{t}}}^r$&$\kh{\frac p{q+p\e{\i t}}}^r$\\
		\midrule
		$\mathrm{Ge}(p)$&-&-\\
		\midrule
		$\mathrm H(n,N,M)$&-&-\\
		\midrule
		$\Pois(\lambda)$&$\e{\lambda(\e t-1)}$&$\e{\lambda(\e{\i t}-1)}$\\
		\bottomrule
	\end{tabular}
\end{center}

\subsection{连续分布}
\paragraph{均匀分布}$\Unif(a,b)$
\begin{align}
	f(x)=\frac1{b-a},\quad x\in(a,b).
\end{align}
\paragraph{正态分布}$\Norm(\mu,\sigma^2)$
\begin{align}
	f(x)=\frac1{\sqrt{2\pi}\sigma}\exp\fkh{-\frac{(x-\mu)^2}{2\sigma^2}},\quad x\in\RR.
\end{align}
\paragraph{指数分布}$\Expo(\lambda)$
\begin{align}
	f(x)=\lambda\e{-\lambda x},\quad x>0.
\end{align}
\paragraph{Gamma分布}$\mathrm{Ga}(\alpha,\lambda)$
\begin{align}
	f(x)=\frac{\lambda^\alpha}{\Gamma(\alpha)}x^{\alpha-1}\e{-\lambda x},\quad x>0.
\end{align}
其中$\alpha,\lambda>0$,Gamma函数
\[
	\Gamma(\alpha)=\int\zti x^{\alpha-1}\e{-x}\d x,
\]
特别地,$\Gamma(x+1)=x\Gamma(x),\,\Gamma(1)=1,\,\Gamma(1/2)=\sqrt\pi,\,\Gamma(n)=(n-1)!$,且
\begin{gather*}
	\mathrm{Ga}(1,\lambda)=\Expo(\lambda),\,\mathrm{Ga}(n/2,1/2)=\chi^2(n)\\
	f(x;\alpha,\lambda)\d x=f(\lambda x;\alpha,1)\d\lambda x
\end{gather*}
\paragraph{卡方分布}$\chi^2(n)$
\begin{align}
	f(x)=\frac{x^{n/2-1}}{2^{n/2}\Gamma(n/2)}\e{-x/2},\quad x>0.
\end{align}
其中$n\in\mathbb N_+$,特别地,$\chi^2(1)$
\[
	f(x)=\frac1{\sqrt{2\pi x}}\e{-x/2},\quad\lim_{x\to 0^+}f(x)=+\infty.
\]
\paragraph{Beta分布}$\mathrm{Be}(a,b)$
\begin{align}
	f(x)=\frac1{\mathrm B(a,b)}x^{a-1}(1-x)^{b-1},\quad x\in(0,1).
\end{align}
其中$a,b>0$,Beta函数
\[
	\mathrm B(a,b)=\int_0^1x^{a-1}(1-x)^{b-1}\d x\equiv\frac{\Gamma(a)\Gamma(b)}{\Gamma(a+b)},
\]
\paragraph{Cauchy分布}
\begin{align}
	f(x)=\frac1\pi\frac1{1+x^2},\quad x\in\RR
\end{align}
\paragraph{Landau分布}略
\subsubsection{众数、期望、方差}
\begin{center}
	\begin{tabular}{c|c|c|c}
		\toprule
		Distribution&Mode&$\E(X)$&$\Var(X)$\\
		\midrule
		$\Unif(a,b)$&-&$\frac{a+b}2$&$\frac{(b-a)^2}{12}$\\
		\midrule
		$\Norm(\mu,\sigma^2)$&$\mu$&$\mu$&$\sigma^2$\\
		\midrule
		$\Expo(\lambda)$&0&$\frac1\lambda$&$\frac1\lambda$\\
		\midrule
		$\mathrm{Ga}(\alpha,\lambda)$&-&&\\
		\midrule
		$\Unif(a,b)$&-&&\\
		\midrule
		$\Unif(a,b)$&-&&\\
		\bottomrule
	\end{tabular}
\end{center}
\section{简单随机抽样}
总体$Y_1,\ldots,Y_N$均值和方差为$\mu,\sigma^2$,简单随机样本为$X_1,\ldots,X_n$。

$A:=$样本中含有$Y_j$
\[
	\P(X_i=Y_j)=\P(X_i=Y_j|A)\P(A)+0=\frac1n\cdot\frac{{N-1\choose n-1}}{{N\choose n}}=\frac1N.
\]
故
\begin{align*}
	\E(X_i)&=\sum_{j=1}^N\P(X_i=Y_j)Y_j=\frac1N\sum_{j=1}^NY_j=\mu;\\
	\Var(X_i)&=\sum_{j=1}^N\P(X_i=Y_j)\bigkh{Y_j-\E(Y_j)}^2=\sigma^2.\\
	\E(\avg X)&=\frac1n\sum_{i=1}^n\E(X_i)=\mu.
\end{align*}
$B:=$样本中同时含有$Y_k,Y_\ell(k\neq\ell)$
\[
	\P(X_i=Y_k,X_j=Y_\ell)=\frac1{n(n-1)}\cdot\frac{{N-2\choose n-2}}{{N\choose n}}=\frac1{N(N-1)}.
\]
故
\begin{align*}
	\E(X_iX_j)&=\frac1{N(N-1)}\sum_{k=1}^N\sum_{\ell\neq k}Y_kY_\ell=\frac1{N(N-1)}\sum_{k=1}^NY_k(N\mu-Y_k)\\
	&=\frac{N^2\mu^2-N(\sigma^2+\mu^2)}{N(N-1)}=\mu^2-\frac{\sigma^2}{N-1}.\\
	\Cov(X_i,X_j)&=\E(X_iX_j)-\E(X_i)\E(X_j)=-\frac{\sigma^2}{N-1}.
\end{align*}
故
\begin{align*}
	\Var(\avg X)&=\frac1{n^2}\kh{\sum_{i=1}^n\Var(X_i)+\sum_{i\neq j}\Cov(X_i,X_j)}\\
	&=\frac1{n^2}\kh{n\sigma^2-n(n-1)\cdot\frac{\sigma^2}{N-1}}=\frac{\sigma^2}n\frac{N-n}{N-1}.\\
	\E(S^2)&=\frac n{n-1}\kh{\frac1n\sum_{i=1}^n\E(X_i^2)-\E(\avg X^2)}\\
	&=\frac n{n-1}\fkh{\sigma^2+\mu^2-\kh{\frac{\sigma^2}n\frac{N-n}{N-1}+\mu^2}}=\frac{N\sigma^2}{N-1}.
\end{align*}
\fi

\end{document}