% 手动排版,尽量杜绝使用

\newcommand{\bs}[1]{\hspace{-#1 pt}}		% 手动减间距	backspace
\newcommand{\bv}[1]{\vspace{-#1 pt}}		% 手动缩行距	backvspace
\def\directlisteqn{\vspace{-1ex}}
\newcommand*{\qqquad}{\qquad\quad}
\newcommand*{\qqqquad}{\qquad\qquad}
\iffalse									% 尽量避免孤行
	\widowpenalty=4000
	\clubpenalty=4000
\fi

% 杂项符号
\let\geq\geqslant
\def\avg{\overline}
\let\ifaoif\iff
\let\iff\relax
\newcommand*{\rqed}{\tag*{$\square$}}								% 靠右 QED
\newcommand*{\halfqed}{\tag*{$\boxdot$}}
\newcommand*{\thus}{\quad\Rightarrow\quad}							% =>
\newcommand*{\iff}{\enspace\Leftrightarrow\enspace}						% <=>	if and only if
\newcommand*{\ifnf}{\quad\Leftrightarrow\quad}						% <=>	if and only if
\newcommand*{\turnto}{\quad\to\quad}
\newcommand*{\normalize}{\quad\overset{\mathrm{normalize}}{-\!\!\!-\!\!\!-\!-\!\!\!\longrightarrow}\quad}
\newcommand*{\vthus}{\\$\Downarrow$\\}
\newcommand*{\viff}{\\$\Updownarrow$\\}
\newcommand*{\vs}{~\text{-}~}
\newcommand{\eg}[1][]{\subparagraph*{例#1:}}
\newcommand*{\prf}{\noindent\textbf{证明:}\quad}
\newcommand{\dpfr}[2]{\displaystyle\frac{#1}{#2}}					% 大分数
\newcommand{\frdp}[2]{\frac{\displaystyle #1}{\displaystyle #2}}
\newcommand{\spark}[1]{\;\textcolor{red}{#1}}
\newcommand*{\verylongrightarrow}{ -\bs4-\bs4-\bs4-\bs4-\bs5\longrightarrow}
\newcommand*{\semilongrightarrow}{-\bs4-\bs5\longrightarrow}

% 简化更常用的希腊字母
\newcommand*{\vf}{\varphi}
\newcommand*{\vF}{\varPhi}
\newcommand*{\vp}{\varPsi}
\newcommand*{\ve}{\varepsilon}
\newcommand*{\vC}{\varTheta}
\newcommand*{\ct}{\theta}			% 还是建议用 @ + Tab 快捷键

% 正体符号
\newcommand*{\cns}{\mathrm{const}}
\newcommand*{\plusc}{{\color{lightgray}\,+\,\cns}}
\newcommand*{\e}{\mathop{}\!\mathrm{e}^}	% e
\let\accenti\i
\renewcommand*{\i}{\mathrm{i}}
\newcommand*{\D}{\Delta}
\newcommand*{\p}{\partial}

\usepackage{bm}											% 粗体 \bm
\newcommand{\hbm}[1][r]{\hat{\bm #1}}	% 应该不会有两个字母的
\newcommand{\ibm}[1]{\,\bm #1}
\newcommand{\uvec}[1]{\mathop{}\!\hat{\bm #1}}

% Using EnglischeSchT script font style
%\newfontfamily{\calti}{EnglischeSchT}
%\newcommand{\mathcalti}[1]{\mbox{\calti{#1}}}
%\newcommand{\mathcaltibf}[1]{\mbox{\bf\calti{#1}}}

\usepackage{mathrsfs}									% 花体 \mathscr
% \usepackage{boondox-cal}								% 小写花体 \mathcal
\newcommand*{\RR}{\mathbb R}
\newcommand*{\CC}{\mathbb C}
\newcommand*{\ZZ}{\mathbb Z}
\newcommand*{\NN}{\mathbb N}
\newcommand*{\sC}{\mathscr C}			% n 阶连续可导函数
\newcommand*{\sR}{\mathscr R}			% 黎曼可积
% 算符用 \mathcal
\newcommand*{\cL}{\mathcal L}			% 表示一般算子
\newcommand{\cl}[1]{\mathcal L\fkh{#1}}
\newcommand{\cli}[1]{\mathcal L^{-1}\!\fkh{#1}}
\newcommand{\cf}[2][\!\,]{\mathcal F_\mathrm{#1}\fkh{#2}}
\newcommand{\cfi}[2][\!\,]{\mathcal F_\mathrm{#1}^{-1}\!\fkh{#2}}
% \newcommand{\cl}[2][0]{\mathcal L\ikh[#1]{#2}}
% \newcommand{\cli}[2][0]{\mathcal L^{-1}\ikh[#1]{#2}}
% \newcommand{\cf}[2][0]{\mathcal F\ikh[#1]{#2}}
% \newcommand{\cfi}[2][0]{\mathcal F^{-1}\ikh[#1]{#2}}

\usepackage{cancel}										% 删除线

\usepackage{xfrac}

% \usepackage{emoji}	需要 LuaTeX

% 导数等
\let\divides\div
\renewcommand*{\div}{\nabla\cdot}
\newcommand*{\curl}{\nabla\times}
\newcommand*{\lap}{\Delta}
\let\accentd\d
\renewcommand*{\d}{\mathop{}\!\mathrm{d}}
\newcommand*{\nd}{\mathrm{d}}
\newcommand*{\vd}{\mathop{}\!\delta}											% δ
\newcommand{\dd}[2][\;\!\!]{\frac{\nd^{#1}}{\nd #2^{#1}}}						% d/dx			我知道 \,\! 很愚蠢,但是 {} 无法在 Math Preview 上预览
\newcommand{\dn}[2]{\frac{\nd^{#1}}{\nd #2^{#1}}}								% d^n/dx^n		\dn2x≡\dd[2]x
\newcommand{\dv}[3][\;\!\!]{\frac{\nd^{#1}#2}{\nd #3^{#1}}}						% df/dx
\newcommand{\du}[3]{\frac{\nd^{#1}#2}{\nd #3^{#1}}}								% d^nf/dx^n		\du2fx≡\dv[2]fx
\newcommand{\pp}[2][\;\!\!]{\frac{\p^{#1}}{\p #2^{#1}}}							% ∂/∂x
\newcommand{\pn}[2]{\frac{\p^{#1}}{\p #2^{#1}}}									% ∂^n/∂x^n		\pn2x≡\pp[2]x
\newcommand{\pv}[3][\;\!\!]{\frac{\p^{#1}#2}{\p #3^{#1}}}						% ∂f/∂x
\newcommand{\pu}[3]{\frac{\p^{#1}#2}{\p #3^{#1}}}								% ∂^nf/∂x^n		\pu2x≡\pv[2]x
\newcommand{\pw}[3]{\frac{\p^2 #1}{\p #2\p #3}}									% ∂^2f/∂x∂y
\newcommand{\pvv}[6]{															% ∂^(m+n)f/∂x^m∂y^n
	\ifnum#4=1
		\ifnum#6=1
			\frac{\p^{#1}#2}{\p #3\p #5}
		\else
			\frac{\p^{#1}#2}{\p #3\p #5^{#6}}
		\fi
	\else
		\ifnum#6=1
			\frac{\p^{#1}#2}{\p #3^{#4}\p #5}
		\else
			\frac{\p^{#1}#2}{\p #3^{#4}\p #5^{#6}}
		\fi
	\fi}
\newcommand{\dvd}[2]{\left.#1\middle\slash #2\right.}							% 斜除

% 积分
\newcommand*{\intt}{\bs2\int\bs8\int}											% ∫∫
\newcommand*{\inttt}{\int\bs8\int\bs8\int}										% ∫∫∫
\newcommand*{\intdt}{\int\bs3\cdot\bs2\cdot\bs2\cdot\bs4\int}					% ∫...∫
\newcommand*{\zti}{_0^{+\infty}}												% _0^+∞
\newcommand*{\iti}{_{-\infty}^{+\infty}}										% _-∞^+∞
\newcommand{\fmto}[3][\infty]{_{#2=#3}^{#1}}

% 括号
\newcommand{\abs}[1]{\left\lvert#1\right\rvert}									% |x| 绝对值
\newcommand{\norm}[1]{\left\lVert#1\right\rVert}								% ||x|| 模
\newcommand{\edg}[1]{\left.#1\right\rvert}										% f|  竖线
\newcommand{\kh}[1]{\left(#1\right)}											% (x) 括号
\newcommand{\bigkh}[1]{\bigl(#1\bigr)}
\newcommand{\Bigkh}[1]{\Bigl(#1\Bigr)}
\newcommand{\biggkh}[1]{\biggl(#1\biggr)}
\newcommand{\fkh}[1]{\left[#1\right]}											% [x] 方括号
\newcommand{\bigfkh}[1]{\bigl[#1\bigr]}
\newcommand{\Bigfkh}[1]{\Bigl[#1\Bigr]}
\newcommand{\biggfkh}[1]{\biggl[#1\biggr]}
\newcommand{\hkh}[1]{\left\{#1\right\}}											% {x} 花括号
\newcommand{\zkh}[1]{\lfloor\bs{4.7}\lceil #1\rceil\bs{4.7}\rfloor}				% [x] 中括号
\newcommand{\floor}[1]{\left\lfloor#1\right\rfloor}
\newcommand{\ceil}[1]{\left\lceil#1\right\rceil}
\newcommand{\set}[2]{\left\{#1\,\middle\vert\,#2\right\}}						% {x|x1,x2,...} 集合
\newcommand{\ave}[1]{\left\langle #1\right\rangle}								% <x> 平均值
\newcommand{\bra}[1]{\left\langle #1\right\vert}								% <ψ| 左矢
\newcommand{\ket}[1]{\left\vert #1\right\rangle}								% |ψ> 右矢
\newcommand{\brkt}[2]{\left\langle #1\middle\vert #2\right\rangle}				% <φ|ψ> 内积
\newcommand{\ktbr}[2]{\left\vert#1\right\rangle \bs3\left\langle #2\right\vert}	% |ψ><φ|
\newcommand{\inp}[2]{\left\langle #1,#2\right\rangle}							% <f,g> 内积

% 数学运算符
\let\Real\Re
\let\Imagine\Im
\let\Re\relax
\let\Im\relax
\DeclareMathOperator{\Re}{Re}					% 
\DeclareMathOperator{\Im}{Im}					% 
\DeclareMathOperator{\sech}{sech}				% 
\DeclareMathOperator{\csch}{csch}				% 
\DeclareMathOperator{\arcsec}{arcsec}			% 
\DeclareMathOperator{\arccot}{arccot}			% 
\DeclareMathOperator{\arccsc}{arccsc}			% 
\DeclareMathOperator{\arsinh}{arsinh}			% 
\DeclareMathOperator{\arcosh}{arcosh}			% 
\DeclareMathOperator{\artanh}{artanh}			% 
\DeclareMathOperator{\sgn}{sgn}					% 符号函数
\DeclareMathOperator{\Li}{Li}					% 
\DeclareMathOperator{\Si}{Si}
\DeclareMathOperator{\Ci}{Ci}
\DeclareMathOperator{\sinc}{sinc}
\DeclareMathOperator{\Heaviside}{H}
\DeclareMathOperator{\arr}{A}					% 排列数
\DeclareMathOperator{\com}{C}					% 组合数
\DeclareMathOperator{\Res}{Res}					% 留数
\DeclareMathOperator{\supp}{supp}				% 支撑集
\DeclareMathOperator{\Int}{Int}					% 内部
\DeclareMathOperator{\Ext}{Ext}					% 外部
\newcommand*{\bigo}{\mathcal O}
\newcommand{\degree}{^\circ}

% 线性代数
% \newif\ifLinearAlgebra\LinearAlgebratrue
% \ifLinearAlgebra
\DeclareMathOperator{\rank}{rank}
\DeclareMathOperator{\id}{id}
\newcommand*{\tp}{^\top}					% AT 转置
\newcommand*{\cj}{^\ast}					% A* 共轭
\newcommand*{\dg}{^\dagger}					% A† 共轭转置
\newcommand*{\iv}{^{-1}}					% A-1
% \fi

% 物理学家
\newcommand*{\Schr}{Schrödinger}
\newcommand*{\Legd}{Legendre}
\newcommand*{\deB}{de Broglie}
\newcommand*{\Rayl}{Rayleigh}
\newcommand*{\Lande}{Landé}

% 粒子
\newcommand*{\elc}{\mathrm e}
\newcommand*{\pton}{\mathrm p}
\newcommand*{\nton}{\mathrm n}
\newcommand*{\mol}{\mathrm m}

% 物理常数
\newcommand*{\NA}{N_{\bs1\mathrm A}}						% Avogadro 常数
\newcommand*{\kB}{k_{\mathrm B}}							% Boltzmann 常数
\newcommand*{\muB}{\mu_\mathrm B}							% Bohr 磁矩

% 
\newcommand*{\Ek}{E_{\mathrm k}}							% 动能
\newcommand*{\eff}{_\mathrm{eff}}							% 有效下标
\newcommand*{\tot}{_\mathrm{tot}}
\newcommand*{\maxi}{_\mathrm{max}}
\newcommand*{\mini}{_\mathrm{min}}
\newcommand*{\lSI}{\tag{SI}}
\newcommand*{\CGS}{\tag{CGS}}								% cm, g, s 制
\newcommand*{\FWHM}{\mathrm{FWHM}}


\newenvironment{equationset}{\left\{\begin{aligned}}{\end{aligned}\right.}