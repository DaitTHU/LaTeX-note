\def\coursename{微积分}
\def\coursefullname{多元微积分与级数}
\def\courseEnglishname{Multivariate Calculus and Series}
\def\teachername{姚家燕}
\def\beginday{2021/3/31}

\documentclass[a4paper, 11pt]{article}

\usepackage[UTF8]{ctex}

\usepackage[T1]{fontenc}								% 字体
\catcode`\。=\active
\newcommand{。}{.} % {\ifmmode\text{.}\else .\fi}
\catcode`\(=\active
\catcode`\)=\active
\newcommand{(}{(}
\newcommand{)}{)}

% \usepackage{zhlineskip}

\usepackage{nicematrix}
% \usepackage{setspace}
% \linespread{1}						% 一倍行距
\setlength{\headheight}{14pt}			% 页眉高度
% \setlength{\lineskip}{0ex}			% 行距
\renewcommand\arraystretch{.82}		% 表格

\usepackage{amssymb, amsmath, amsfonts, amsthm}			% 数学符号,公式,字体,定理环境
\everymath{\displaystyle}			% \textstyle \scriptstyle \scriptscriptstyle
\allowdisplaybreaks[4]      		% 使用行间公式格式
% \makeatletter
% \renewcommand{\maketag@@@}[1]{\hbox{\m@th\normalsize\normalfont#1}}
% \makeatother
\newif\ifcontent\contenttrue		% if 显示目录
\newif\ifparskip\parskipfalse		% if 增加目录后的行距
\newif\ifshowemail\showemailfalse	% if 显示 email
\def\firstandforemost{
	\maketitle
	%\thispagestyle{empty}\clearpage
	\ifcontent
		\renewcommand{\contentsname}{目录}
		\tableofcontents
		\thispagestyle{empty}
		\clearpage
	\fi
	\ifparskip
		\setlength{\parskip}{.8ex}	% 设置额外的段距,目录后
	\fi								% 在 \firstandforemost 前设置 \parskiptrue
	\makenomenclature
	\printnomenclature
	\setcounter{page}{1}
}

\usepackage{mathtools}									% \rcase 环境等

% \usepackage{physics}

\usepackage[]{siunitx}									% 国际制单位
\sisetup{
	inter-unit-product = \ensuremath{{}\cdot{}},
	per-mode = symbol,
	per-mode = reciprocal-positive-first,
	range-units = single,
	separate-uncertainty = true,
	range-phrase = \ifmmode\text{\;-\;}\else\;-\;\fi
}
\DeclareSIUnit\angstrom{\text{Å}}
\DeclareSIUnit\atm{\text{atm}}
% SIunits 额外定义了一个 \square 表示平方,
% 还会把 \cdot 空格加大,真有够无语的 😅

\usepackage{authblk}									% 作者介绍
\ifx \coursefullname\undefined
	\ifx \coursename\undefined
		\def\coursename{笔记}
	\fi
	\def\coursefullname{\coursename}
\fi
\ifx \authorname\undefined
	\def\authorname{Dait}
\fi
\ifx \departmentname\undefined
	\def\departmentname{DEP 00, THU}
\fi
\ifx \emailaddress\undefined
	\def\emailaddress{daiyj20@mails.tsinghua.edu.cn}
\fi
\ifx \beginday\undefined
	\def\beginday{2021}
\fi
\ifx \endday\undefined
	\def\endday{\number\year/\number\month/\number\day}
\fi
\ifx \titleannotation\undefined
	\ifx \teachername\undefined
		\title{\textbf{\coursefullname}}
	\else
		\title{\textbf{\coursefullname}\\\small\textit{主要整理自\teachername 老师讲义}}
	\fi
\else
	\title{\textbf{\coursefullname}\\\small\textit{(\titleannotation)}}
\fi
\newif\ifdefaultauthor\defaultauthortrue
\ifdefaultauthor
	\author{by~\authorname~at~\departmentname}
	\ifshowemail
		\affil{\emailaddress}
	\fi
\fi
\ifx \endday\beginday
	\date{\beginday}
\else
	\date{\beginday~-~\endday}
\fi

\usepackage{hyperref}									% 链接
\ifx \courseEnglishname\undefined
	\def\courseEnglishname{Note}
\fi
\ifx \authorEnglishname\undefined
	\def\authorEnglishname{Dait}
\fi
\hypersetup{
	% dvipdfm								% 表示用 dvipdfm 生成 pdf
	pdftitle={\coursename},
	pdfauthor={\authorname},
	colorlinks=true, breaklinks=true,		% 超链接设置
	linkcolor=black, citecolor=black, urlcolor=blue
}

\usepackage[british]{babel}								% 长单词自动连字符换行
\hyphenation{long-sen-ten-ce}				% 自定义拆分方式

\usepackage{tikz}
\usetikzlibrary{quotes, angles}
\usepackage{pgfplots}
\pgfplotsset{compat=1.17}								% TikZ
\newcommand{\coor}[5][0]{
	\draw[thick,latex-latex](#1,#3)node[left]{$#5$}--(#1,0)node[shift={(-135:7pt)}]{$O$}--(#1+#2,0)node[right]{$#4$}
}			% 坐标轴

\usepackage{enumerate}									% 编号
\usepackage{paralist}
\setlength{\pltopsep}{1ex}
\setlength{\plitemsep}{1ex}
\ifx \eqnrange\undefined
	\numberwithin{equation}{section}
\else
	\numberwithin{equation}{\eqnrange}
\fi

\renewcommand{\thempfootnote}{\Roman{mpfootnote}}
\renewcommand{\thefootnote}{\Roman{footnote}}		% 注释上标 I, II,...
\newcommand{\sectionstar}[1]{
	\section[\hspace{-.8em}*\hspace{.3em}#1]{\hspace{-1em}*\hspace{.5em}#1}
}
\newcommand{\subsectionstar}[1]{					% 带星号的 section 和 subsection
	\subsection{\hspace{-1em}*\hspace{.5em}#1}
}
\newcommand{\subsubsectionstar}[1]{					% 带星号的 section 和 subsection
	\subsubsection{\hspace{-1em}*\hspace{.5em}#1}
}
\newcommand*{\appendiks}{
	\appendix
	\part*{附录}
	\addcontentsline{toc}{part}{附录}
}
\iffalse			% 不清楚
	\newcommand{\varsection}[1]{
		\refstepcounter{section}
		\section*{\thesection\quad #1}
		\addcontentsline{toc}{section}{\makebox[0pt][r]{*}\thesection\quad #1}
	}
\fi

\usepackage{fancyhdr}									% 页眉页脚
\ifx \coursename\undefined
	\def\coursename{笔记}
\fi
\fancyhf{}\pagestyle{fancy}
\fancyhead[L]{\coursename\rightmark}
\fancyhead[R]{by~\authorname}
\fancyfoot[C]{-~\thepage~-} 			%页码

\usepackage{colortbl, booktabs}							% 表

\usepackage{graphicx}
\usepackage{float}
\usepackage{caption}									% 图
\captionsetup{
	margin=20pt, format=hang,
	justification=justified
}
\newcounter{tikzpic}
\def\tikzchap{
	\stepcounter{tikzpic}\\
	\small 图~\thetikzpic\quad
}
\newcounter{linetable}
\newcommand{\tablechap}[1]{
	\stepcounter{linetable}
	{\small 表~\thelinetable\quad #1}\\[1em]
}

\usepackage{tcolorbox}									% 盒子
\tcbuselibrary{theorems, skins, breakable}
\definecolor{MatchaGreen}{HTML}{73C088}		% 抹茶绿B7C6B3
\newtcbtheorem[number within = subsection]{example}{例}{
	enhanced, breakable, sharp corners,
	attach boxed title to top left = {yshifttext = -1mm},
	before skip = 2ex,
	colback = MatchaGreen!5,				% 文本框内的底色
	colframe = MatchaGreen,					% 文本框框沿的颜色
	fonttitle = \bfseries,					% 标题字体用粗体	coltitle 默认 white,
	boxed title style = {
			sharp corners, size = small, colback = MatchaGreen,
		}
}{exm}
\definecolor{MelancholyBlue}{HTML}{9EAABA}	% melancholy: 沮丧
\newcounter{pslt}
\setcounter{pslt}{-1}
\newtcbtheorem[use counter = pslt]{posulate}{假设}{
	enhanced, breakable, sharp corners,
	attach boxed title to top left = {yshifttext = -1mm}, before skip = 2ex,
	colback = MelancholyBlue!5, colframe = MelancholyBlue, fonttitle = \bfseries,
	boxed title style = {
			sharp corners, size = small, colback = MelancholyBlue,
		}
}{psl}
\definecolor{PureBlue}{HTML}{80A3D0}
\newtcbtheorem[number within = subsection]{definition}{定义}{
	enhanced, breakable, sharp corners,
	attach boxed title to top left = {yshifttext = -1mm}, before skip = 2ex,
	colback = PureBlue!5, colframe = PureBlue, fonttitle = \bfseries,
	boxed title style = {
			sharp corners, size = small, colback = PureBlue,
		}
}{dfn}
\definecolor{PeachRed}{HTML}{EA868F}
\newtcbtheorem[number within = subsection]{theorem}{定理}{
	enhanced, breakable, sharp corners,
	attach boxed title to top left = {yshifttext = -1mm}, before skip = 2ex,
	colback = PeachRed!5, colframe = PeachRed, fonttitle = \bfseries,
	boxed title style = {
			sharp corners, size = small, colback = PeachRed,
		}
}{thm}
\definecolor{SchembriumYellow}{HTML}{fbd26a}	% 申博太阳城黄
\newtcbtheorem[number within = section]{method}{方法}{
	enhanced, breakable, sharp corners,
	attach boxed title to top left = {yshifttext = -1mm}, before skip = 2ex,
	colback = SchembriumYellow!5, colframe = SchembriumYellow, fonttitle = \bfseries,
	boxed title style = {
			sharp corners, size = small, colback = SchembriumYellow,
		}
}{mtd}
% 保留颜色
\definecolor{fadedgold}{HTML}{D9CBB0}		% 褪色金
\definecolor{saturatedgold}{HTML}{F0E0C2}	% staurated: 饱和
\definecolor{elegantblue}{HTML}{C4CCD7}		% elegant: 优雅
\definecolor{ivory}{HTML}{F1ECE6}			% 象牙
\definecolor{gloomypruple}{HTML}{CCC1D2}	% 阴沉紫
% \textcolor[HTML]{FFC23A}					% 石板灰

\definecolor{Green}{rgb}{0,.8,0}

\usepackage{imakeidx}								% 索引

\usepackage{nomencl}								% 关键词
%\setlength{\nomitemsep}{0.2cm}							% 设置术语之间的间距
\renewcommand{\nomentryend}{.}							% 设置打印出术语的结尾的字符
\renewcommand{\eqdeclaration}[1]{见公式:(#1)}			% 设置打印见公式的样式
\renewcommand{\pagedeclaration}[1]{见第 (#1) 页}		% 设置打印页的样式
\renewcommand{\nomname}{术语表} 						% 修改术语表标题的名称。

\usepackage{array}
\usepackage{booktabs} % 三线表
\usepackage{multirow}
% 手动排版,尽量杜绝使用

\newcommand{\bs}[1]{\hspace{-#1 pt}}		% 手动减间距	backspace
\newcommand{\bv}[1]{\vspace{-#1 pt}}		% 手动缩行距	backvspace
\def\directlisteqn{\vspace{-1ex}}
\newcommand*{\qqquad}{\qquad\quad}
\newcommand*{\qqqquad}{\qquad\qquad}
\iffalse									% 尽量避免孤行
	\widowpenalty=4000
	\clubpenalty=4000
\fi

% 杂项符号
\let\geq\geqslant
\def\avg{\overline}
\let\ifaoif\iff
\let\iff\relax
\newcommand*{\rqed}{\tag*{$\square$}}								% 靠右 QED
\newcommand*{\halfqed}{\tag*{$\boxdot$}}
\newcommand*{\thus}{\quad\Rightarrow\quad}							% =>
\newcommand*{\iff}{\enspace\Leftrightarrow\enspace}						% <=>	if and only if
\newcommand*{\ifnf}{\quad\Leftrightarrow\quad}						% <=>	if and only if
\newcommand*{\turnto}{\quad\to\quad}
\newcommand*{\normalize}{\quad\overset{\mathrm{normalize}}{-\!\!\!-\!\!\!-\!-\!\!\!\longrightarrow}\quad}
\newcommand*{\vthus}{\\$\Downarrow$\\}
\newcommand*{\viff}{\\$\Updownarrow$\\}
\newcommand*{\vs}{~\text{-}~}
\newcommand{\eg}[1][]{\subparagraph*{例#1:}}
\newcommand*{\prf}{\noindent\textbf{证明:}\quad}
\newcommand{\dpfr}[2]{\displaystyle\frac{#1}{#2}}					% 大分数
\newcommand{\frdp}[2]{\frac{\displaystyle #1}{\displaystyle #2}}
\newcommand{\spark}[1]{\;\textcolor{red}{#1}}
\newcommand*{\verylongrightarrow}{ -\bs4-\bs4-\bs4-\bs4-\bs5\longrightarrow}
\newcommand*{\semilongrightarrow}{-\bs4-\bs5\longrightarrow}

% 简化更常用的希腊字母
\newcommand*{\vf}{\varphi}
\newcommand*{\vF}{\varPhi}
\newcommand*{\vp}{\varPsi}
\newcommand*{\ve}{\varepsilon}
\newcommand*{\vC}{\varTheta}
\newcommand*{\ct}{\theta}			% 还是建议用 @ + Tab 快捷键

% 正体符号
\newcommand*{\cns}{\mathrm{const}}
\newcommand*{\plusc}{{\color{lightgray}\,+\,\cns}}
\newcommand*{\e}{\mathop{}\!\mathrm{e}^}	% e
\let\accenti\i
\renewcommand*{\i}{\mathrm{i}}
\newcommand*{\D}{\Delta}
\newcommand*{\p}{\partial}

\usepackage{bm}											% 粗体 \bm
\newcommand{\hbm}[1][r]{\hat{\bm #1}}	% 应该不会有两个字母的
\newcommand{\ibm}[1]{\,\bm #1}
\newcommand{\uvec}[1]{\mathop{}\!\hat{\bm #1}}

% Using EnglischeSchT script font style
%\newfontfamily{\calti}{EnglischeSchT}
%\newcommand{\mathcalti}[1]{\mbox{\calti{#1}}}
%\newcommand{\mathcaltibf}[1]{\mbox{\bf\calti{#1}}}

\usepackage{mathrsfs}									% 花体 \mathscr
% \usepackage{boondox-cal}								% 小写花体 \mathcal
\newcommand*{\RR}{\mathbb R}
\newcommand*{\CC}{\mathbb C}
\newcommand*{\ZZ}{\mathbb Z}
\newcommand*{\NN}{\mathbb N}
\newcommand*{\sC}{\mathscr C}			% n 阶连续可导函数
\newcommand*{\sR}{\mathscr R}			% 黎曼可积
% 算符用 \mathcal
\newcommand*{\cL}{\mathcal L}			% 表示一般算子
\newcommand{\cl}[1]{\mathcal L\fkh{#1}}
\newcommand{\cli}[1]{\mathcal L^{-1}\!\fkh{#1}}
\newcommand{\cf}[2][\!\,]{\mathcal F_\mathrm{#1}\fkh{#2}}
\newcommand{\cfi}[2][\!\,]{\mathcal F_\mathrm{#1}^{-1}\!\fkh{#2}}
% \newcommand{\cl}[2][0]{\mathcal L\ikh[#1]{#2}}
% \newcommand{\cli}[2][0]{\mathcal L^{-1}\ikh[#1]{#2}}
% \newcommand{\cf}[2][0]{\mathcal F\ikh[#1]{#2}}
% \newcommand{\cfi}[2][0]{\mathcal F^{-1}\ikh[#1]{#2}}

\usepackage{cancel}										% 删除线

\usepackage{xfrac}

% \usepackage{emoji}	需要 LuaTeX

% 导数等
\let\divides\div
\renewcommand*{\div}{\nabla\cdot}
\newcommand*{\curl}{\nabla\times}
\newcommand*{\lap}{\Delta}
\let\accentd\d
\renewcommand*{\d}{\mathop{}\!\mathrm{d}}
\newcommand*{\nd}{\mathrm{d}}
\newcommand*{\vd}{\mathop{}\!\delta}											% δ
\newcommand{\dd}[2][\;\!\!]{\frac{\nd^{#1}}{\nd #2^{#1}}}						% d/dx			我知道 \,\! 很愚蠢,但是 {} 无法在 Math Preview 上预览
\newcommand{\dn}[2]{\frac{\nd^{#1}}{\nd #2^{#1}}}								% d^n/dx^n		\dn2x≡\dd[2]x
\newcommand{\dv}[3][\;\!\!]{\frac{\nd^{#1}#2}{\nd #3^{#1}}}						% df/dx
\newcommand{\du}[3]{\frac{\nd^{#1}#2}{\nd #3^{#1}}}								% d^nf/dx^n		\du2fx≡\dv[2]fx
\newcommand{\pp}[2][\;\!\!]{\frac{\p^{#1}}{\p #2^{#1}}}							% ∂/∂x
\newcommand{\pn}[2]{\frac{\p^{#1}}{\p #2^{#1}}}									% ∂^n/∂x^n		\pn2x≡\pp[2]x
\newcommand{\pv}[3][\;\!\!]{\frac{\p^{#1}#2}{\p #3^{#1}}}						% ∂f/∂x
\newcommand{\pu}[3]{\frac{\p^{#1}#2}{\p #3^{#1}}}								% ∂^nf/∂x^n		\pu2x≡\pv[2]x
\newcommand{\pw}[3]{\frac{\p^2 #1}{\p #2\p #3}}									% ∂^2f/∂x∂y
\newcommand{\pvv}[6]{															% ∂^(m+n)f/∂x^m∂y^n
	\ifnum#4=1
		\ifnum#6=1
			\frac{\p^{#1}#2}{\p #3\p #5}
		\else
			\frac{\p^{#1}#2}{\p #3\p #5^{#6}}
		\fi
	\else
		\ifnum#6=1
			\frac{\p^{#1}#2}{\p #3^{#4}\p #5}
		\else
			\frac{\p^{#1}#2}{\p #3^{#4}\p #5^{#6}}
		\fi
	\fi}
\newcommand{\dvd}[2]{\left.#1\middle\slash #2\right.}							% 斜除

% 积分
\newcommand*{\intt}{\bs2\int\bs8\int}											% ∫∫
\newcommand*{\inttt}{\int\bs8\int\bs8\int}										% ∫∫∫
\newcommand*{\intdt}{\int\bs3\cdot\bs2\cdot\bs2\cdot\bs4\int}					% ∫...∫
\newcommand*{\zti}{_0^{+\infty}}												% _0^+∞
\newcommand*{\iti}{_{-\infty}^{+\infty}}										% _-∞^+∞
\newcommand{\fmto}[3][\infty]{_{#2=#3}^{#1}}

% 括号
\newcommand{\abs}[1]{\left\lvert#1\right\rvert}									% |x| 绝对值
\newcommand{\norm}[1]{\left\lVert#1\right\rVert}								% ||x|| 模
\newcommand{\edg}[1]{\left.#1\right\rvert}										% f|  竖线
\newcommand{\kh}[1]{\left(#1\right)}											% (x) 括号
\newcommand{\bigkh}[1]{\bigl(#1\bigr)}
\newcommand{\Bigkh}[1]{\Bigl(#1\Bigr)}
\newcommand{\biggkh}[1]{\biggl(#1\biggr)}
\newcommand{\fkh}[1]{\left[#1\right]}											% [x] 方括号
\newcommand{\bigfkh}[1]{\bigl[#1\bigr]}
\newcommand{\Bigfkh}[1]{\Bigl[#1\Bigr]}
\newcommand{\biggfkh}[1]{\biggl[#1\biggr]}
\newcommand{\hkh}[1]{\left\{#1\right\}}											% {x} 花括号
\newcommand{\zkh}[1]{\lfloor\bs{4.7}\lceil #1\rceil\bs{4.7}\rfloor}				% [x] 中括号
\newcommand{\floor}[1]{\left\lfloor#1\right\rfloor}
\newcommand{\ceil}[1]{\left\lceil#1\right\rceil}
\newcommand{\set}[2]{\left\{#1\,\middle\vert\,#2\right\}}						% {x|x1,x2,...} 集合
\newcommand{\ave}[1]{\left\langle #1\right\rangle}								% <x> 平均值
\newcommand{\bra}[1]{\left\langle #1\right\vert}								% <ψ| 左矢
\newcommand{\ket}[1]{\left\vert #1\right\rangle}								% |ψ> 右矢
\newcommand{\brkt}[2]{\left\langle #1\middle\vert #2\right\rangle}				% <φ|ψ> 内积
\newcommand{\ktbr}[2]{\left\vert#1\right\rangle \bs3\left\langle #2\right\vert}	% |ψ><φ|
\newcommand{\inp}[2]{\left\langle #1,#2\right\rangle}							% <f,g> 内积

% 数学运算符
\let\Real\Re
\let\Imagine\Im
\let\Re\relax
\let\Im\relax
\DeclareMathOperator{\Re}{Re}					% 
\DeclareMathOperator{\Im}{Im}					% 
\DeclareMathOperator{\sech}{sech}				% 
\DeclareMathOperator{\csch}{csch}				% 
\DeclareMathOperator{\arcsec}{arcsec}			% 
\DeclareMathOperator{\arccot}{arccot}			% 
\DeclareMathOperator{\arccsc}{arccsc}			% 
\DeclareMathOperator{\arsinh}{arsinh}			% 
\DeclareMathOperator{\arcosh}{arcosh}			% 
\DeclareMathOperator{\artanh}{artanh}			% 
\DeclareMathOperator{\sgn}{sgn}					% 符号函数
\DeclareMathOperator{\Li}{Li}					% 
\DeclareMathOperator{\Si}{Si}
\DeclareMathOperator{\Ci}{Ci}
\DeclareMathOperator{\sinc}{sinc}
\DeclareMathOperator{\Heaviside}{H}
\DeclareMathOperator{\arr}{A}					% 排列数
\DeclareMathOperator{\com}{C}					% 组合数
\DeclareMathOperator{\Res}{Res}					% 留数
\DeclareMathOperator{\supp}{supp}				% 支撑集
\DeclareMathOperator{\Int}{Int}					% 内部
\DeclareMathOperator{\Ext}{Ext}					% 外部
\newcommand*{\bigo}{\mathcal O}
\newcommand{\degree}{^\circ}

% 线性代数
% \newif\ifLinearAlgebra\LinearAlgebratrue
% \ifLinearAlgebra
\DeclareMathOperator{\rank}{rank}
\DeclareMathOperator{\id}{id}
\newcommand*{\tp}{^\top}					% AT 转置
\newcommand*{\cj}{^\ast}					% A* 共轭
\newcommand*{\dg}{^\dagger}					% A† 共轭转置
\newcommand*{\iv}{^{-1}}					% A-1
% \fi

% 物理学家
\newcommand*{\Schr}{Schrödinger}
\newcommand*{\Legd}{Legendre}
\newcommand*{\deB}{de Broglie}
\newcommand*{\Rayl}{Rayleigh}
\newcommand*{\Lande}{Landé}

% 粒子
\newcommand*{\elc}{\mathrm e}
\newcommand*{\pton}{\mathrm p}
\newcommand*{\nton}{\mathrm n}
\newcommand*{\mol}{\mathrm m}

% 物理常数
\newcommand*{\NA}{N_{\bs1\mathrm A}}						% Avogadro 常数
\newcommand*{\kB}{k_{\mathrm B}}							% Boltzmann 常数
\newcommand*{\muB}{\mu_\mathrm B}							% Bohr 磁矩

% 
\newcommand*{\Ek}{E_{\mathrm k}}							% 动能
\newcommand*{\eff}{_\mathrm{eff}}							% 有效下标
\newcommand*{\tot}{_\mathrm{tot}}
\newcommand*{\maxi}{_\mathrm{max}}
\newcommand*{\mini}{_\mathrm{min}}
\newcommand*{\lSI}{\tag{SI}}
\newcommand*{\CGS}{\tag{CGS}}								% cm, g, s 制
\newcommand*{\FWHM}{\mathrm{FWHM}}


\newenvironment{equationset}{\left\{\begin{aligned}}{\end{aligned}\right.}

\begin{document}
\firstandforemost

\section{多元函数微分}
\begin{definition}{向量和多元函数}{}
	$x_1,x_2,\ldots,x_n\in\RR$向量
	\[
		X:=\zkh{x_1,x_2,\ldots,x_n}\tp,
	\]
	定义在$\Omega\subset\RR^n$上的$n$元函数$f:\Omega\to\RR.$
\end{definition}
$f_1,f_2,\ldots,f_m:\Omega\to\RR$构成向量值函数$F=\zkh{f_1,f_2,\ldots,f_m}\tp.$
\subsection{偏导数}
\begin{definition}{}{}
	$\hat e_1,\hat e_2,\ldots,\hat e_n$ 是$\RR^n$的基底,则$f$在$X_0$点关于$x_i$的偏导数
	\[
		\pv f{x_i}(X_0):=\lim_{t\to 0}\frac{f(X_0+t\hat e_i)-f(X_0)}t,
	\]
	$f$的全微分
	\begin{align}
		\d f(X_0)=\sum_{i=1}^n\pv f{x_i}(X_0)\d x_i.
	\end{align}
\end{definition}
有时用$\partial_if$比$\p f/\p x_i$更好.
\begin{theorem}{可微性判定}{}
	对二元函数$f(x,y)$
	\begin{center}
		$\pv fx$在$X_0$连续,$\pv fy(X_0)$存在
		\vthus
		$f$在$X_0$可微性
		\viff
		$f(X)-\fkh{f(X_0)+\pv fx(X_0)\Delta x+\pv fy(X_0)\Delta y}=o\kh{\norm{\Delta X}}$
	\end{center}
	其中$\D X\equiv\zkh{\D x,\D y}\tp.$
\end{theorem}
\subsection{方向导数和梯度}
\begin{definition}{}{}
	$f$可微,其沿单位向量$\hat \ell=\zkh{\cos\alpha_1,\cos\alpha_2,\ldots,\cos\alpha_n}$方向的方向导数 % \vec \ell=\zkh{x_1,x_2,\ldots,x_n}, \vec \ell/|\ell|=
	\[
		\pv f{\vec \ell}(X_0):=\lim_{t\to 0}\frac{f(X_0+t\hat \ell)-f(X_0)}t;
	\]
	梯度
	\[
		\nabla f:=\left[\pv f{x_1},\pv f{x_2},\ldots,\pv f{x_n}\right]\tp.
	\]
\end{definition}
因此方向导数也可以看做
\begin{align}
	\pv f{\vec \ell}(X_0)=\sum_{i=1}^n\pv f{x_i}(X_0)\cos\alpha_i=\nabla f(X_0)\cdot\hat \ell.
\end{align}
\subsection{高阶偏导数}
\begin{definition}{}{}
	若$\dvd f{x_i}$对$x_j$的偏导存在,则记二阶偏导数
	\[
		\pw f{x_j}{x_i}:=\pp{x_j}\pv f{x_j};\quad \pu2 f{x_i}\equiv\pw f{x_i}{x_i}
	\]
	也可记作$\partial_{ji}f.$
\end{definition}
\begin{theorem}{二阶偏导的存在性判定}{}
	\begin{center}
		$\pw f{x_i}{x_j}$和$\pw f{x_j}{x_i}$中一个在$X_0$连续
		\vthus
		$\pw f{x_i}{x_j}=\pw f{x_j}{x_i}$
	\end{center}
\end{theorem}
\subsection{向量值函数微分}
\begin{definition}{Jacobi矩阵}{}
	向量值函数$F:\RR^m\to\RR^n$%,于是$\vec f:\Omega\to\RR^m$
	\begin{gather*}
		\begin{bmatrix}
			\text df_1 \\
			\vdots     \\
			\text df_m
		\end{bmatrix}=
		%\begin{bmatrix}
		%    \frac{\p f_1}{\p x_1}\d x_1+\cdots+\frac{\p f_1}{\p x_n}\d x_n \\
		%    \vdots                                                         \\
		%    \frac{\p f_m}{\p x_1}\d x_1+\cdots+\frac{\p f_m}{\p x_n}\d x_n
		%\end{bmatrix}=
		\begin{bmatrix}
			\pv{f_1}{x_1} & \cdots & \pv{f_1}{x_n} \\
			\vdots                & \ddots & \vdots                \\
			\pv{f_m}{x_1} & \cdots & \pv{f_m}{x_n}
		\end{bmatrix}
		\begin{bmatrix}
			\d x_1 \\
			\vdots \\
			\d x_n
		\end{bmatrix};\\
		\d F=J_F\d X.
	\end{gather*}
	式中的矩阵称为Jacobi矩阵
	\[
		J_F\equiv\frac{\p(f_1,\ldots,f_m)}{\p(x_1,\ldots,x_n)}.
	\]
	$m=n$时,对应有Jacobi行列式
	\[
		\frac{D(f_1,\ldots,f_n)}{D(x_1,\ldots,x_n)}:=\det J_F
	\]
\end{definition}
\subsection{复合函数微分}
\begin{theorem}{}{}
	$G:\RR^n\to \RR^m,F:\RR^m\to \RR^k$, $Y_0=G(X_0)$,则下面三个结论是等价的
	\begin{compactenum}[(1)]
		\item $\d(F\circ G)(X_0)=\d F(Y_0)\d G(X_0);$
		\item $J_{F\circ G}(X_0)=J_F(Y_0)J_G(X_0);$
		\item $\frac{\p(f_1,\ldots,f_k)}{\p(x_1,\ldots,x_n)}(X_0)=\frac{\p(f_1,\ldots,f_k)}{\p(y_1,\ldots,y_m)}(Y_0)\frac{\p(g_1,\ldots,g_m)}{\p(x_1,\ldots,x_n)}(X_0).$
	\end{compactenum}
\end{theorem}
特别的,当$k=1$即$F=f$为单值函数时
\begin{align}
	\pv{f\circ G}{x_i}(X_0)=\sum_{j=1}^m\pv f{y_j}(Y_0)\pv{g_j}{x_i}(X_0).
\end{align}
$\color{gray}\text{例:}\pv{f(x,y,x^2)}x=\p_1f(x,y,x^2)+2x\,\p_3f(x,y,x^2)$
\subsection{隐函数}
\begin{theorem}{二元隐函数}{}
	给定$f(x,y)\in\sC(\RR^2)$\footnote{表示$f$在$\RR^2$上连续.}与点$P(x_0,y_0)$,
	\begin{center}
		$f(P)=0$且$\pv fy(P)\neq 0$
		\vthus\bv6
		$y=y(x)$存在,且$\dv yx=-\frdp{\pv fx(x,y)}{\pv fy(x,y)}.$
	\end{center}
\end{theorem}
证明:$\d f=\pv fx\d x+\pv fy\d y=0.$
\begin{theorem}{多元隐函数}{}
	给定$f(X,y)\in\sC(\RR^{n+1})$与点$P(X_0,y_0)$,
	\begin{center}
		$f(P)=0$且$\pv fy(P)\neq 0$
		\vthus\bv6
		$y=y(X)$存在,且$\dv y{x_i}=-\frdp{\pv f{x_i}(X,y)}{\pv fy(X,y)}.$
	\end{center}
\end{theorem}
\begin{theorem}{向量值隐函数}{}
	给定$f_i(X,Y)\in\sC(\RR^{n+m}),(i=1,2,\ldots,m)$与点$P(X_0,Y_0)$,
	\begin{center}
		$F(P)=\vec 0$且$\frac{D(f_1,\ldots,f_m)}{D(y_1,\ldots,y_m)}(P)\neq 0$
		\vthus
		$Y=Y(X)$存在,且$\frac{\p(y_1,\ldots,y_m)}{\p(x_1,\ldots,x_n)}=-\left(\frac{\p(f_1,\ldots,f_m)}{\p(y_1,\ldots,y_m)}\right)^{-1}\frac{\p(f_1,\ldots,f_m)}{\p(x_1,\ldots,x_n)}.$
	\end{center}
\end{theorem}
\begin{theorem}{反函数}{}
	给定$Y=F(X)$,则反函数$X=F^{-1}(Y)$满足
	\[
		J_{F^{-1}}(X)=\left(J_F(X)\right)^{-1}.
	\]
\end{theorem}
\begin{example}{二阶偏导数举例}{}
	给定$f(x,y,z)=0$,求$\pw zyx.$
	\begin{align*}
		\pw zyx & =-\pp y\kh{\dvd{\pv fx}{\pv fz}}                                  \\
				 & =-\dvd{\kh{\pp y\pv fx\cdot\pv fy-\pv fx\cdot\pp y\pv fz}}{\kh{\pv fz}^2}
	\end{align*}
	其中
	\begin{align*}
		\pv{}y\pv fx & =\pw fyx+\pw fzx\pv zy,\quad \pv zy=-\dvd{\pv fy}{\pv fz}; \\
		\pv{}y\pv fz & =\pw fyz+\pu2 fz\pv zy.
	\end{align*}
	故
	{\scriptsize\begin{align*}
		\pw zyx=-\dvd{\left[\kh{\pv fz}^2\pw fyx-\pv fy\pv fz\pw fzx-\pv fx\pv fz\pw fyz+\pv fx\pv fy\pu2 fz\right]}{\kh{\pv fz}^3}.
	\end{align*}}
\end{example}
\subsection{法与切}
给定向量$\vec v=\zkh{a,b,c}\tp$与点$P_0(x_0,y_0,z_0)$,可以确定
\begin{align*}
	\text{线}:   & \qquad \frac{x-x_0}a=\frac{y-y_0}b=\frac{z-z_0}c \\
	\text{平面}: & \qquad a(x-x_0)+b(y-y_0)+c(z-z_0)=0
\end{align*}
参数方程可以得到显函数表达式
\begin{gather*}
	\text{线}:~
	\begin{cases}
		x=x_0+at \\[-1ex]
		y=y_0+bt \\[-1ex]
		z=z_0+ct
	\end{cases}
	\quad
	\text{平面}:~
	\begin{cases}
		x=x_0+a_1u+b_1v \\[-1ex]
		y=y_0+a_2u+b_2v \\[-1ex]
		z=z_0+a_3u+b_3v
	\end{cases}
	\\
	a=
	\begin{vmatrix}
		a_2 & b_2 \\
		a_3 & b_3
	\end{vmatrix}\quad b=
	\begin{vmatrix}
		a_3 & b_3 \\
		a_1 & b_1
	\end{vmatrix}\quad c=
	\begin{vmatrix}
		a_1 & b_1 \\
		a_2 & b_2
	\end{vmatrix}.
\end{gather*}
\paragraph{曲面的法线和切平面}
\subparagraph{显函数}
\[
	z=f(x,y)\thus\vec v=\left[\pv fx,\pv fy,-1\right]_{(x_0,y_0)}.
\]
\subparagraph{参数方程}
\begin{align*}
	\begin{cases}
		x=f_1(u,v) \\
		y=f_2(u,v) \\
		z=f_3(u,v)
	\end{cases}
	\thus\vec v=\left[\frac{D(f_2,f_3)}{D(u,v)},\frac{D(f_3,f_1)}{D(u,v)},\frac{D(f_1,f_2)}{D(u,v)}\right]_{(u_0,v_0)}.
\end{align*}
\subparagraph{隐函数}
\[
	f(x,y,z)=0\thus\vec v=\fkh{\pv fx,\pv fy,\pv fz}_{P_0}=\nabla f(P_0).
\]
\paragraph{曲线的法平面和切线}
\subparagraph{参数}
\begin{align*}
	\begin{cases}
		x=x(t) \\[-1ex]
		y=y(t) \\[-1ex]
		z=z(t)
	\end{cases}
	\thus\vec v=\fkh{x'(t),y'(t),z'(t)}.
\end{align*}
\subparagraph{隐函数}
\begin{align*}
	\begin{cases}
		F_1(x,y,z)=0 \\
		F_2(x,y,z)=0
	\end{cases}
	\thus\vec v=\left[\frac{D(F_1,F_2)}{D(y,z)},\frac{D(F_1,F_2)}{D(z,x)},\frac{D(F_1,F_2)}{D(x,y)}
		\right]_{P_0}
\end{align*}
\newpage
\subsection{Talyor公式}
\begin{definition}{Hesse矩阵}{}
	$f:\Omega\to\RR$的Jacobi矩阵(行向量)为
	\begin{align*}
		J_f=
		\begin{bmatrix}
			\pv f{x_1} & \pv f{x_2} & \cdots & \pv f{x_n}
		\end{bmatrix}.
	\end{align*}
	定义Hesse矩阵
	\begin{align*}
		H_f:=
		\begin{bmatrix}
			\frac{\p^2f}{\p x_1^2}     & \cdots & \frac{\p^2f}{\p x_1\p x_n} \\
			\vdots                     & \ddots & \vdots                     \\
			\frac{\p^2f}{\p x_n\p x_1} & \cdots & \frac{\p^2f}{\p x_n^2}
		\end{bmatrix}.
	\end{align*}
\end{definition}
\paragraph{带Lagrange余项的1阶Talyor公式}
$\exists\theta\in(0,1),X_\theta:=X_0+\theta\Delta X$
\begin{align}
	f(X)=f(X_0)+J_f(X_0)\Delta X+\frac12\Delta X\tp H_f(X_\theta)\Delta X.
\end{align}
\paragraph{带Peano余项的2阶Talyor公式}
\begin{align}
	f(X)=f(X_0)+J_f(X_0)\Delta X+\frac12\Delta X\tp H_f(X_0)\Delta X+\alpha(\Delta X).
\end{align}
此处$\alpha(\Delta X)=\frac12\Delta X\tp\tilde{H_f}(X_0+\theta\Delta X)\Delta X=o(\norm{\Delta X}^2),\quad\Delta X\to0$
\paragraph{$m$阶Talyor公式} % $\Delta X=[\Delta x_1,\Delta x_2,\ldots,\Delta x_n]\tp,$
\begin{align*}
	f(X) & =\sum_{k=0}^{m}\frac1{k!}\left(\sum_{i=1}^{n}\Delta x_i\pp{x_i}\right)^kf(X_0)+\frac1{(m+1)!}\left(\sum_{i=1}^{n}\Delta x_i\pp{x_i}\right)^{m+1}f(X_\theta) \\
		 & =\sum_{k=0}^{m}\frac1{k!}\left(\sum_{i=1}^{n}\Delta x_i\pp{x_i}\right)^kf(X_0)+o(\norm{\Delta X}^m)
\end{align*}
$\color{gray}\text{例:}f(x,y)=$
\begin{align*}
	\color{gray}f(x_0,y_0)+\pv fx\Delta x+\pv fy\Delta y+\frac12\left(\frac{\p^2f}{\p x^2}\Delta x^2+2\pw fxy\Delta x\Delta y+\pv[2]fy\Delta y^2\right) \\
	\color{gray}+\frac16\left(\pv[3]fx\Delta x^3+3\frac{\p^3f}{\p x^2\p y}\Delta x^2\Delta y+3\frac{\p^3f}{\p x\p y^2}\Delta x\Delta y^2+\pv[3]fy\Delta y^3\right)+\cdots.
\end{align*}
$\color{gray}\text{最后换回来:}\Delta x=x-x_0,\Delta y=y-y_0$
\subsection{极值与条件极值}
\begin{theorem}{Fermat定理}{}
	\begin{center}
		$f$在$X_0$可微,且$X_0$是$f$极值点
		\vthus
		$X_0$是$f$驻点,即$\pv f{x_i}(X_0)=0,\quad i=1,2,\ldots,n.$
	\end{center}
\end{theorem}
\begin{theorem}{极值点的充分条件}{}
	$f$在$X_0$二阶连续可微,且$X_0$是$f$驻点
	\begin{compactenum}[(1)]
		\item $H_f(X_0)$正定$\thus$极小;
		\item $H_f(X_0)$负定$\thus$极大;
		\item $H_f(X_0)$不定$\thus$不是极值点.
	\end{compactenum}
	判断正负定方法:(1)左上行列式; (2)特征值.
\end{theorem}
\begin{theorem}{Lagrange乘数法}{}
	给定$k$维曲面
	\[
		S=\set X{\varphi_i(X)=0,i=1,2,\ldots,n-k}.
	\]
	定义Lagrange函数
	\begin{align}
		L(X,\varLambda):=f(X)+\sum_{i=1}^{n-k}\lambda_i\varphi_i(X).
	\end{align}
	若$X_0\in S$为$f$在$S$上的条件极值点,则存在$\varLambda$使得$(X_0,\Lambda)$为$L$的驻点.
\end{theorem}
\paragraph{计算有界闭曲面上的最值}
\begin{compactenum}[(1)]
	\item 算出$f$在$\RR^2$上的驻点(求Hesse矩阵判定正负定得出是否为极值点)
	\item 选取在曲面内的驻点并求值
	\item 固定曲面边界
\end{compactenum}
\newpage
\section{含参积分及广义含参积分}
\begin{definition}{含参积分}{}
	含参积分
	\[
		I(y)=\int_a^b\bs3f(x,y)\d x,\quad y\in\zkh{c,d}.
	\]
	广义含参积分包括无穷积分和瑕积分,有统一形式
	\[
		I(y)=\int_a^\omega\bs3f(x,y)\d x\equiv\lim_{A\to\omega^-}\int_a^A\bs3f(x,y)\d x.
	\]
	无穷积分$\omega=+\infty$;瑕积分$\omega\in\RR$但$f$在$\omega$邻域内无界(奇点).
\end{definition}
\begin{definition}{一致连续}{}
	$\forall\varepsilon>0,\exists\delta>0$ 使 $\forall X,Y$ 满足 $\vert X-Y\vert<\delta$,有 $|f(X)-f(Y)|<\varepsilon.$
	\tcblower
	\paragraph*{否定形式:}存在两点列$\{X_k\}\{Y_k\}$使得$\forall k\geqslant 1,|f(X_k)-f(Y_k)|\geqslant\varepsilon_0>0.$
\end{definition}
\begin{theorem}{一致连续的判定}{}
	函数$f\in\sC(\Omega)$在有界闭集$\Omega$上连续$\thus$一致连续.
\end{theorem}
\begin{definition}{一致收敛}{}
	$\forall\varepsilon>0,\exists N>a$ 使 $\forall A>N,\forall y\in[c,d]$,有 $\left|\int_a^A\bs3f(x,y)\d x-I(y)\right|<\varepsilon.$
\end{definition}
\begin{theorem}{一致收敛的Weierstrass判别法}{}
	\begin{center}
		$|f(x,y)|\leqslant F(x),\int_a^\omega\bs3F(x)\d x$收敛.
		\vthus
		$\int_a^\omega\bs3f(x,y)\d x$对$y$一致收敛.
	\end{center}
\end{theorem}
\begin{theorem}{一致收敛的Dirichlet判别法}{}
	\begin{center}
		$\int_a^A\bs3f(x,y)\d x$有界;\\
		$g(x,y)$关于$x$单调且$\lim_{x\to\omega^-}g(x,y)=0$
		\vthus
		$\int_a^\omega\bs3f(x,y)g(x,y)\d x$对$y$一致收敛.
	\end{center}
\end{theorem}
\begin{theorem}{一致收敛的Abel判别法}{}
	\begin{center}
		$\int_a^\omega\bs3f(x,y)\d x$对$y$一致收敛;\\
		$g(x,y)$关于$x$单调且有界
		\vthus
		$\int_a^\omega\bs3f(x,y)g(x,y)\d x$对$y$一致收敛.
	\end{center}
\end{theorem}
\subsection{积分号的可交换性}
\begin{theorem}{连续性}{}
	$f(x,y)$连续$\thus I(y)$连续,即
	\[
		\lim_{y\to y_0}\int_a^bf(x,y)\d x=\int_a^b\lim_{y\to y_0}f(x,y)\d x.
	\]
	\tcblower
	$f(x,y)$连续且$\int_a^\omega f(x,y)\d x$关于$y$一致收敛.$\thus I(y)$连续,即
	\[
		\lim_{y\to y_0}\int_a^\omega f(x,y)\d x=\int_a^\omega\lim_{y\to y_0}f(x,y)\d x.
	\]
\end{theorem}
\begin{theorem}{可微性}{}
	$f(x,y),\pv fy(x,y)$连续$\thus I(y)$可微,且
	\[
		I'(y)=\frac{\d}{\d y}\int_a^b\bs3f(x,y)\d x=\int_a^b\bs3\pp yf(x,y)\d x.
	\]
	\tcblower
	$f(x,y),\pv{f}{y}(x,y)$连续且$\int_a^\omega\pv{f}{y}(x,y)\d x$关于$y$一致收敛$\thus I(y)$可微,且
	\[
		I'(y)=\frac{\d}{\d y}\int_a^\omega\bs3f(x,y)\d x=\int_a^\omega\bs3\pp yf(x,y)\d x.
	\]
\end{theorem}
公式
\[
	\dv{}y\int_{\alpha(y)}^{\beta(y)}\bs3f(x,y)\d x=\int_{\alpha(y)}^{\beta(y)}\pv fy(x,y)\d x+f(\beta(y),y)\beta'(y)-f(\alpha(y),y)\alpha'(y).
\]
\begin{theorem}{积分性}{}
	$f(x,y)$连续$\thus I(y)$可积,且
	\[
		\int_c^d\bs3I(y)\d y=\int_c^d\bs5\int_a^b f(x,y)\d x\d y=\int_a^b\bs5\int_c^d f(x,y)\d y\d x.
	\]
	\tcblower
	$f(x,y)$连续且$\int_a^\omega f(x,y)\d x$关于$y$一致收敛$\thus I(y)$可积,且
	\[
		\int_c^d\bs3I(y)\d y=\int_c^d\bs5\int_a^\omega f(x,y)\d x\d y=\int_c^\omega\bs5\int_a^b f(x,y)\d x\d y
	\]
\end{theorem}
*对于含两个广义积分的交换条件更严格
\begin{compactenum}
	\item $f(x,y)$连续
	\item $\int_a^{+\infty}\bs{10}f(x,y)\d x,\int_c^{+\infty}\bs{10}f(x,y)\d y$分别关于$y\in[c,C],x\in[a,A]$一致收敛
	\item $\int_c^{+\infty}\bs7\int_a^{+\infty}\bs{10}|f(x,y)|\d x\d y,\int_a^{+\infty}\bs7\int_c^{+\infty}\bs{10}|f(x,y)|\d y\d x$至少一个存在
\end{compactenum}
则
\[
	\int_c^{+\infty}\bs7\int_a^{+\infty}\bs{10}f(x,y)\d x\d y=\int_a^{+\infty}\bs7\int_c^{+\infty}\bs{10}f(x,y)\d y\d x
\]
\clearpage
\begin{example}{Gamma函数}{}
	\begin{center}
		$\Gamma(x):=\int\zti\bs5t^{x-1}\e{-t}\d t.$
	\end{center}
	递推公式
	\begin{align*}
		\Gamma(1)   & =\int\zti\bs5\e{-t}\d t=1.                                              \\
		\Gamma(x+1) & =\int\zti\bs5 t^x\cdot\e{-t}\d t                                        \\
					& =\edg{-t^x\e{-t}}\zti+\int\zti\bs5\e{-t}\cdot xt^{x-1}\d t=x\Gamma(x).
	\end{align*}
	对于$n\in\mathbb N,\Gamma(n)=(n-1)!.$
	% $\Gamma(x+1)=x\Gamma(x),\Gamma(1)=1$,对正整数$n$,
	% \[
	% 	\Gamma(n)=\int_0^{+\infty}\bs5t^{n-1}\e{-t}\d t=(n-1)!
	% \]

	\textbf{余元公式}
	\[
		\Gamma(x)\Gamma(1-x)=\frac{\pi}{\sin\pi x},\quad x\in(0,1).
	\]
	证明略,有$\Gamma(1/2)=\sqrt\pi$
\end{example}
\begin{example}{Beta函数}{}
	\begin{center}
		$\text B(p,q):=\int_0^1t^{p-1}(1-t)^{q-1}\d t.$ % =2\int_0^{\pi/2}\cos^{2p-1}\theta\sin^{2q-1}\theta\d\theta.$
	\end{center}
	与Gamma函数的关系为$\text B(p,q)=\frac{\Gamma(p)\Gamma(q)}{\Gamma(p+q)}.$
\end{example}
\begin{example}{Possion积分}{}
	\begin{center}
		$\int\zti\bs5\e{-t^2}\d t=\frac{\sqrt\pi}2.$
	\end{center}
	\textbf{证明:}考虑积分的平方
	\begin{align*}
		\int\zti\bs5\e{-x^2}\d x\int\zti\bs5\e{-y^2}\d y=\int\zti\bs7\int\zti\e{-x^2-y^2}\d x\nd y \\
		=\int_0^{\pi/2}\bs5\int\zti\e{-r^2}\cdot r\d r\nd\theta=\frac\pi{2}\cdot\frac12\int\zti\e{-s}\d s=\frac\pi{4}.
	\end{align*}
	\iffalse
		由$\ex>0,\forall x\in\RR$
		\[
			\intt_A\e{-x^2-y^2}\d x\nd y\leqslant\intt_B\e{-x^2-y^2}\d x\nd y\leqslant\intt_C\e{-x^2-y^2}\d x\nd y
		\]
		其中
		\begin{align*}
			B & =\{(x,y)~|~0\leqslant x\leqslant R,0\leqslant y\leqslant R\}, \\
			A & =\{(x,y)~|~x^2+y^2\leqslant R^2,x\geqslant 0,y\geqslant 0\},  \\
			C & =\{(x,y)~|~x^2+y^2\leqslant 2R^2,x\geqslant 0,y\geqslant 0\}.
		\end{align*}
		又
		\begin{align*}
			\intt_A\e{-x^2-y^2}\d x\nd y & =\int_0^{\pi/2}\bs7\int_0^R\e{-\rho^2}\rho\d\rho\d\varphi=\frac{\pi}4\left(1-\e{-R^2}\right), \\
			\intt_B\e{-x^2-y^2}\d x\nd y & =\int_0^R\bs7\int_0^R\e{-x^2}\e{-y^2}\d x\nd y=\left(\int_0^R\e{-t^2}\d t\right)^2,          \\
			\intt_C\e{-x^2-y^2}\d x\nd y & =\frac{\pi}4\left(1-\e{-2R^2}\right).
		\end{align*}
		当$R\to+\infty$,即得$\int\zti\bs5\e{-t^2}\d t=\frac{\sqrt\pi}2$.
	\fi
\end{example}
\begin{example}{Dirichlet积分}{}
	\begin{center}
		$\int\zti\bs3\frac{\sin x}x\d x=\frac{\pi}2.$
	\end{center}
	\textbf{证明:}令
	\[
		F(y)=\int\zti\bs3\frac{\sin x}x\e{-xy}\d x,\quad y\geqslant 0.
	\]
	由$\frac{\sin x}x\e{-xy}$可被延拓为$\RR_{\geqslant 0}^2$上的连续函数,故$F(y)$在$\RR_{\geqslant 0}$上连续.

	注意到,积分
	\[
		\int\zti\bs3\frac{\sin x}x\d x
	\]
	收敛;$\forall x,y\geqslant 0,|\e{-xy}|\leqslant 1$且$\e{-xy}$关于$x$递减,由Abel判别准则可知$F(y)$关于$y\in[0,+\infty)$一致收敛.

	任取$a>0,\forall x\geqslant 0,y\geqslant a,$可知$\left|-\sin x\e{-xy}\right|\leqslant \e{-xy}\leqslant \e{-ax}$有界,由Weierstrass判别准则知,积分
	\[
		\int\zti\bs5\pv {}y\frac{\sin x}x\e{-xy}\d x=-\int\zti\bs7\sin x\e{-xy}\d x
	\]
	一致收敛,因此$F(y)$在$[a,+\infty)$上可导
	\begin{align*}
		F'(y) & =-\int\zti\bs7\sin x\e{-xy}\d x=\int\zti\bs7\Im\e{-xy-\i x}\d x                                                \\
			  & =-\Im\left.\frac{\e{-(y+\i)x}}{y+\i}\right|\zti\bs7=-\left.\frac{\e{-yx}}{y^2+1}\Im(y-\i)\e{-\i x}\right|\zti \\
			  & =\left.\frac{\e{-yx}}{y^2+1}\left(y\sin x+\cos x\right)\right|\zti=-\frac1{1+y^2}.
	\end{align*}
	因为$a>0$任意,故$\forall y>0,$
	\[
		F'(y)=-\frac1{1+y^2},\quad F(y)=-\arctan y+C.
	\]
	又$y\to+\infty,$
	\[
		|F(y)|\leqslant\int\zti\bs3\frac{|\sin x|}x\e{-xy}\d x\leqslant\int\zti\bs8\e{-xy}\d x=\frac1y\to0.
	\]
	故$\lim_{y\to+\infty}\bs3F(y)=C-\frac{\pi}2=0$,即$C=\frac{\pi}2.$

	又$F(y)$在$[0,+\infty)$上连续,故$\forall y\geqslant 0,F(y)=\frac{\pi}2-\arctan y.$特别的
	\[
		F(0)=\int\zti\bs3\frac{\sin x}x\d x=\frac{\pi}2.
	\]
\end{example}
\newpage
\section{重积分}
\subsection{二重积分}
\begin{theorem}{累次积分法}{}
	$D=\{(x,y)~|~\alpha(x)\leqslant y\leqslant\beta(x),a\leqslant x\leqslant b\}$,$\alpha(x),\beta(x)$在$[a,b]$上连续,则$f\in\mathscr R(D)$,且
	\[
		\intt_Df(x,y)\d x\nd y=\int_a^b\bs5\int_{\alpha(x)}^{\beta(x)}\bs5f(x,y)\d y\d x.
	\]
\end{theorem}
\begin{theorem}{变量代换法}{}
	以$x,y$为变量的区域$D$经过变量代换及其逆变换
	\begin{align*}
		\begin{cases}
			u=u(x,y), \\
			v=v(x,y).
		\end{cases}
		\quad\left(\frac{D(x,y)}{D(u,v)}\neq 0\right).
	\end{align*}
	就会变为以$u,v$为变量的区域$D'$,积分也会变换为
	\[
		\intt_Df(x,y)\d x\nd y=\intt_{D'}\bs2f(x(u,v),y(u,v))\left|\frac{D(x,y)}{D(u,v)}\right|\d u\nd v.
	\]
\end{theorem}
\paragraph{例:极坐标}$x=\rho\cos\theta,y=\rho\sin\theta.$
\[
	\intt_Df(x,y)\d x\nd y=\intt_{D'}\bs2f(\rho\cos\theta,\rho\sin\theta)\rho\d\rho\nd\theta.
\]
\paragraph{例:旋转}$x=u\cos\theta-v\sin\theta,y=u\sin\theta+v\cos\theta.$
\[
	\intt_Df(x,y)\d x\nd y=\intt_{D'}\bs2f(u\cos\theta-v\sin\theta,u\sin\theta+v\cos\theta)\d u\nd v.
\]
\subsubsection*{曲面面积问题}
空间$O-xyz$中曲面的方程为
\begin{align*}
	\begin{cases}
		x=x(u,v), \\
		y=y(u,v), \\
		z=z(u,v).
	\end{cases}
	\quad\left(\pv{(x,y,z)}{(u,v)}\text{列满秩}\right).
\end{align*}
转化为$O-uv$曲面中的二重积分问题,其中两条曲线切向量
\[
	\vec u=\left[\pv xu,\pv yu,\pv zu\right]\tp,\qquad\vec v=\left[\pv xv,\pv yv,\pv zv\right]\tp
\]
则
\[
	S=\intt_D\norm{\vec u\times\vec v}\d u\nd v
\]
若设
\begin{align}
	A=\frac{D(y,z)}{D(u,v)},\qquad B=\frac{D(z,x)}{D(u,v)},\qquad C=\frac{D(x,y)}{D(u,v)}.
\end{align}
则$\vec u\times\vec v=[A,B,C]\tp,$
\begin{align}
	S=\intt_D\sqrt{A^2+B^2+C^2}\d u\nd v
\end{align}
另一种方式是利用$(\vec u\times\vec v)^2=u^2v^2-(\vec u\cdot\vec v)^2,$
\begin{align}
	E & =\vec u\cdot\vec u=\left(\pv xu\right)^2+\left(\pv yu\right)^2+\left(\pv zu\right)^2, \\
	G & =\vec v\cdot\vec v=\left(\pv xv\right)^2+\left(\pv yv\right)^2+\left(\pv zv\right)^2, \\
	F & =\vec u\cdot\vec v=\pv xu\pv xv+\pv yu\pv yv+\pv zu\pv zv.
\end{align}
\[
	S=\intt_D\sqrt{EG-F^2}\d u\nd v
\]
特别的,当曲线是显式的,即$z=z(x,y)$
\[
	S=\intt_D\sqrt{1+\left(\pv zx\right)^2+\left(\pv zy\right)^2}\d x\nd y.
\]
\subsection{三重积分}
\begin{theorem}
	{累次积分法}{}
	$\Omega=\{(x,y,z)~|~\alpha(x,y)\leqslant z\leqslant\beta(x,y),(x,y)\in D\}$,则
	\[
		\inttt_\Omega f(x,y,z)\d x\nd y\nd z=\intt_D\int_{\alpha(x,y)}^{\beta(x,y)}f(x,y,z)\d z\d x\nd y.
	\]
\end{theorem}
\begin{theorem}
	{变量代换法}{}
	同样的,对于变量替换
	\begin{align*}
		\begin{cases}
			x=x(r,s,t), \\
			y=y(r,s,t), \\
			z=z(r,s,t).
		\end{cases}
		\quad\left(\frac{D(x,y,z)}{D(r,s,t)}\neq 0\right).
	\end{align*}
	有
	\[
		\inttt_\Omega f(x,y,z)\d x\nd y\nd z=\inttt_{\Omega'}\tilde f(r,s,t)\abs{\frac{D(x,y,z)}{D(r,s,t)}}\d r\nd s\nd t.
	\]
	其中$\tilde f(r,s,t)=f(x(r,s,t),y(r,s,t),z(r,s,t)).$
\end{theorem}
\paragraph{例:柱坐标}$x=\rho\cos\theta,y=\rho\sin\theta,z=z.$
\[
	\inttt_\Omega f(x,y,z)\d x\nd y\nd z=\inttt_{\Omega'}f(\rho\cos\theta,\rho\sin\theta,z)\rho\d\rho\nd\theta\nd z.
\]
\paragraph{例:球坐标}$x=r\sin\theta\cos\phi,y=r\sin\theta\sin\phi,z=r\cos\theta.$
\begin{align*}
	\inttt_\Omega & f(x,y,z)\d x\nd y\nd z                                                                                   \\
				  & =\inttt_{\Omega'}f(\sin\theta\cos\phi,r\sin\theta\sin\phi,r\cos\theta)r^2\sin\theta\d r\nd\theta\nd\phi.
\end{align*}
\begin{example}{$n$重积分}{}
	定义$\RR^n$中单位球
	\[
		\Omega_n=\set{(x_1,x_2,\ldots,x_n)}{\sum_{i=1}^nx_i^2\leqslant 1}
	\]
	的体积
	\[
		V_n=\int\bs3\cdot\bs2\cdot\bs2\cdot\bs4\int_{\Omega_n}\bs5\d x_1\bs3\cdot\bs2\cdot\bs2\cdot\bs1\d x_n=\int_{-1}^1\bs3\int\bs3\cdot\bs2\cdot\bs2\cdot\bs4\int_{D_{n-1}}\bs8\d x_1\bs3\cdot\bs2\cdot\bs2\cdot\bs1\d x_{n-1}\d x_n.
	\]
	其中
	\[
		D_{n-1}=\set{(x_1,\ldots,x_{n-1})}{\sum_{i=1}^{n-1}x_i^2\leqslant 1-x_n^2}.
	\]
	作变换$x_i=\sqrt{1-x_n^2}\,u_i,(i=1,\ldots,n-1)$,则
	\[
		\frac{D(x_1,\ldots,x_{n-1})}{D(u_1,\ldots,u_{n-1})}=\left(1-x_n^2\right)^{\kh{n-1}/2}.
	\]
	可得
	\begin{align*}
		V_n & =\int_{-1}^1\bs3\int\bs3\cdot\bs2\cdot\bs2\cdot\bs4\int_{\Omega_{n-1}}\bs8\left(1-x_n^2\right)^{\kh{n-1}/2}\d u_1\bs3\cdot\bs2\cdot\bs2\cdot\bs1\d u_{n-1}\d x_n \\
			& =\int_{-1}^1\left(1-x_n^2\right)^{\kh{n-1}/2}V_{n-1}\d x_n=2V_{n-1}\int_0^1\left(1-x^2\right)^{\kh{n-1}/2}\d x.
	\end{align*}
	令$x=\sqrt t$,则
	\begin{align*}
		V_n & =V_{n-1}\int_0^1(1-t)^{\kh{n-1}/2}t^{-1/2}\d t=V_{n-1}\text B\left(\frac12,\frac{n+1}2\right)                                                                  \\
			& =V_{n-1}\cdot\frdp{\Gamma\kh{\frac12}\Gamma\kh{\frac{n+1}2}}{\Gamma\kh{\frac{n+2}2}}=V_{n-1}\cdot\frdp{\Gamma\kh{\frac{n+1}2}}{\Gamma\kh{\frac{n+2}2}}\sqrt\pi \\
		%& =V_1\cdot\frdp{\Gamma\kh{\frac{n+1}2}}{\Gamma\kh{\frac{n+2}2}}\frdp{\Gamma\kh{\frac{n}2}}{\Gamma\kh{\frac{n+1}2}}\cdots\frdp{\Gamma\kh{\frac32}}{\Gamma\kh{\frac52}}\pi^{\kh{n-1}/2} \\
			& =V_1\cdot\frdp{\Gamma\kh{\frac32}}{\Gamma\kh{\frac{n+2}2}}\pi^{\kh{n-1}/2}=\frdp{\pi^{n/2}}{\Gamma\kh{\frac n2+1}}.
	\end{align*}
	我们就得到了“体积”公式
	\begin{align}
		V_n=
		\begin{cases}
			\frac{\pi^k}{k!},        & n=2k,   \\[
		2ex]
			\frac{2^k\pi^{k-1}}{n!}, & n=2k-1.
		\end{cases}
	\end{align}
	例如$V_1=2,\quad V_2=\pi,\quad V_3=\frac 43\pi,\quad V_4=\frac 12\pi^2,\quad V_5=\frac 8{15}\pi^2,\ldots.$
\end{example}
\clearpage
\section{曲线积分和曲面积分}
\begin{definition}{第一类、第二类曲线积分和曲面积分}{}
	第一类曲线积分
	\[
		\int_Lf(X)\d\ell=\int_\alpha^\beta f(x(t),y(t))\sqrt{x'(t)^2+y'(t)^2}\d t.
	\]
	第一类曲面积分
	\[
		\int_Sf(X)\d\sigma=\int_Df(x(u,v),y(u,v),z(u,v))\norm{\vec u\times\vec v}\d u\nd v.
	\]
	第二类曲线积分
	\[
		\int_A^B\vec F\cdot\d\vec\ell=\int_A^B X(x,y)\d x+Y(x,y)\d y.
	\]
	第二类曲面积分
	\[
		\int_{S^+}\vec V\cdot\d\vec S=\int_{S^+}X\d y\wedge\d z+Y\d z\wedge\d x+Z\d x\wedge\d y.
	\]
\end{definition}
第二类曲面积分可以化为第一类曲面积分
\[
	\int_{S^+}\vec V\cdot\d\vec S=\int_S\vec V\cdot\hat n\d\sigma.
\]
从计算方面来说
\[
	\int_{S^+}X\d y\wedge\d z+Y\d z\wedge\d x+Z\d x\wedge\d y=\pm\intt_DXA+YB+ZC\d u\nd v.
\]
\subsection{Green, Gauss, Stokes公式}
\begin{theorem}{Green公式}{}
	$X(x,y),Y(x,y)$在有界单连通闭区域$D$上连续可微
	\begin{align*}
		\color{Green}\intt_D\pv Xx+\pv Yy\d x\nd y=\oint_{\p D}X\d y-Y\d x, \\
		\color{Green}\intt_D\pv Yx-\pv Xy\d x\nd y=\oint_{\p D}X\d x+Y\d y.
	\end{align*}
\end{theorem}
\begin{theorem}{Gauss公式}{}
	$X(x,y),Y(x,y)$在有界单连通闭区域$D$上连续可微
	\begin{align*}
		\inttt_\Omega\nabla\cdot\vec V\d x\nd y\nd z=\intt_{\p\Omega}\vec V\cdot\d\vec S.
	\end{align*}
\end{theorem}
\end{document}