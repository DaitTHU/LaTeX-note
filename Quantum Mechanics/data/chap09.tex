\chapter{跃迁理论}
若体系的Hamilton量$\hat H_0$不显含时间,能量本征值问题的解为
\[
	\hat H_0\ket n=E_n\ket n.
\]
若$\hat H(t)=\hat H_0+\hat H'(t)$显含时间,体系将有一定的概率离开初态$\ket{k}$而处于其它定态$\ket{k'}$,这就是\textbf{量子跃迁}。
\section{量子态随时间的变化}
状态随时间的演化由\Schr 方程决定
\begin{align}
	\begin{cases}
		\i\hbar\pp t\ket{\psi(t)}=\hat H(t)\ket{\psi(t)},\\
		\ket{\psi(0)}=\ket{k}.
	\end{cases}
\end{align}
采用能量$\hat H_0$表象 
\begin{align}
	\ket{\psi(0)}=\sum_n C_{nk}(t)\ket n\e{-\i E_nt/\hbar},\quad C_{nk}(0)=\vd_{nk}.
\end{align}
体系$t$时刻跃迁到定态$\ket{k'}$的概率为$\abs{C_{k'k}(t)}^2$。

为求出跃迁概率,将表象表达式带入\Schr 方程,假定$\hat H'$中不包括$\p/\p t$作用
\begin{align}\notag
	\i\hbar\sum_n\fkh{\dot C_{nk}(t)-\cancel{\frac{\i E_nC_{nk}(t)}\hbar}}\ket n\e{-\i E_nt/\hbar}&=\sum_nC_{nk}(t)\fkh{\cancel{E_n}+\hat H'(t)}\ket n\e{-\i E_nt/\hbar}.\\
	\i\hbar\sum_n\dot C_{nk}(t)\ket n\e{-\i E_nt/\hbar}&=\sum_nC_{nk}(t)\hat H'(t)\ket n\e{-\i E_nt/\hbar}.
\end{align}
与$\ket{k'}$内积
\begin{align}
	\i\hbar\dot C_{k'k}(t)\e{-\i E_{k'}t/\hbar}=\sum_nC_{nk}(t)\bra{k'}\hat H'(t)\ket n\e{-\i E_nt/\hbar}
\end{align}
定义 
\begin{align}
	H_{k'n}'(t):=\bra{k'}\hat H'(t)\ket n,\quad\omega_{k'n}=\frac{E_{k'}-E_n}\hbar.
\end{align}
得到跃迁振幅$C_{k'k}(t)$满足 
\begin{align}
	\begin{cases}
		\i\hbar\dd tC_{k'k}(t)=\sum_nH'_{k'n}(t)\e{\i\omega_{k'n}t}C_{nk}(t),\\
		C_{k'k}(0)=\vd_{k'k}.
	\end{cases}
\end{align}
两边对$t$积分
\begin{align}
	C_{k'k}(t)=C_{k'k}(0)+\frac1{\i\hbar}\int_0^t\sum_nH'_{k'n}(\tau)\e{\i\omega_{k'n}\tau}C_{nk}(\tau)\d\tau
\end{align}
这是一个积分方程,可采用迭代求解。

迭代一次
\begin{align*}
	C_{k'k}(t)&=\vd_{k'k}+\frac1{\i\hbar}\int_0^t\sum_nH'_{k'n}(\tau)\e{\i\omega_{k'n}\tau}\fkh{\vd_{nk}+\frac1{\i\hbar}\int_0^\tau\sum_nH'_{k'n}(\pi)\e{\i\omega_{k'n}\pi}C_{nk}(\pi)\d\pi}\d\tau\\
	&=\vd_{k'k}+\frac1{\i\hbar}\int_0^tH'_{k'k}(\tau)\e{\i\omega_{k'k}\tau}\d\tau+\cdots
\end{align*}
零级近似
\[
C_{k'k}^{(0)}=\vd_{k'k};
\]
一级近似 
\[
C_{k'k}^{(1)}=\frac1{\i\hbar}\int_0^tH'_{k'k}(\tau)\e{\i\omega_{k'k}\tau}\d\tau,
\]
代表直接从初态$\ket k$跃迁到末态$\ket{k'}$,故跃迁概率
\begin{align}
	P_{k'k}(t)=\frac1{\hbar^2}\abs{\int_0^tH'_{k'k}(\tau)\e{\i\omega_{k'k}\tau}\d\tau}^2.
\end{align}
由$\hat H'$是Hermite的。故$P_{k'k}=P_{kk'}$,即从初态到末态的跃迁概率等于从末态到初态的跃迁概率。

对于初态$\ket k$和末态$\ket{k'}$都有简并的情况,计算跃迁概率应对$\ket k$能级各简并态求平均,而对$\ket{k'}$能级各简并态求和,此时跃迁概率不一定相等。

二级近似
\[
C_{k'k}^{(2)}=\frac1{(\i\hbar)^2}\sum_n\int_0^tH'_{k'n}(\tau)\e{\i\omega_{k'n}\tau}\int_0^\tau H'_{nk}(\pi)\e{\i\omega_{nk}\pi}\d\pi\d\tau,
\]
代表从初态$\ket k$经中间态$\ket n$跃迁到末态$\ket{k'}$。
\sectionstar{周期微扰和常微扰}
没讲
\section{光的吸收与辐射}
光与原子的相互作用包括受激吸收、受激辐射和自发辐射。其中自发辐射是前面的理论无法解释的,Einstein基于热力学和统计物理中的平衡概念给出过半唯象的理论,巧妙地导出了自发辐射系数。
\paragraph{电偶极跃迁}若入射光为理想单色偏振光,
\[
	\bm E=\bm E_0\cos(\omega t-\bm k\cdot\bm r),\quad\bm B=\frac{\bm k}{\abs k}\times\bm E.\CGS
\]
对电子的作用 
\[
	\bm f=-e\kh{\bm E+\frac{\bm v}c\times\bm B},\CGS
\]
原子中电子的速度$v\ll c$,故可仅考虑电场的作用。

对于可见光和紫外光\footnote{X光并不满足。}波长$\lambda\gg a$\,(Bohr半径),故在原子范围内电场可视为均匀场
\[
	\bm E\doteq\bm E_0\cos\omega t.
\]
光对原子的作用可近似表示成电子的电偶极矩与电场的相互作用
\[
	\hat H'(t)=-\bm D\cdot\bm E=W\cos\omega t.
\]
其中电子的电偶极矩$\bm D=-e\r$,电偶极矩与电场作用引起的跃迁称为电偶极跃迁。
\begin{align*}
	C_{k'k}^{(1)}(t)&=\frac1{\i\hbar}\int_0^tH_{k'k}'(\tau)\e{-\i\omega_{k'k}\tau}\d\tau=\frac{W_{k'k}}{\i\hbar}\int_0^t\cos\omega\tau\e{-i\omega_{k'k}\tau}\d\tau\\
	&=-\frac{W_{k'k}}{2\hbar}\fkh{\frac{\e{\i(\omega_{k'k}+\omega)t}-1}{\omega_{k'k}+\omega}+\frac{\e{\i(\omega_{k'k}-\omega)t}-1}{\omega_{k'k}-\omega}}.
\end{align*}
下面讨论原子吸收光的跃迁,$E_{k'}>E_k$,只有当入射光$\omega\doteq\omega_{k'k}$时,才会引起$E_k\to E_{k'}$跃迁,此时
\begin{align*}
	C_{k'k}^{(1)}(t)=-\frac{W_{k'k}}{2\hbar}\frac{\e{\i(\omega_{k'k}-\omega)t}-1}{\omega_{k'k}-\omega}.
\end{align*}
从$k\to k'$的概率为
\begin{align}
	P_{k'k}(t)=\abs{C_{k'k}^{(1)}(t)}^2=\frac{\abs{W_{k'k}}^2}{4\hbar^2}\fkh{\frac{\sin(\omega_{k'k}-\omega)t/2}{(\omega_{k'k}-\omega)/2}}^2
\end{align}
$t\to\infty$时,有 
\begin{align}
	P_{k'k}(t)=\frac{\pi t}{2\hbar^2}\abs{W_{k'k}}^2\vd\kh{\omega_{k'k}-\omega};
\end{align}
跃迁速率
\begin{align}
	w_{k'k}&=\dd tP_{k'k}=\frac\pi{2\hbar^2}\abs{W_{k'k}}^2\vd\kh{\omega_{k'k}-\omega}\\
	&=\frac\pi{2\hbar^2}\abs{\bm D_{k'k}}^2E_0^2\cos^2\theta\vd\kh{\omega_{k'k}-\omega},
\end{align}
$\theta$为电子的电偶极矩与电场的夹角。

而对于非偏振光,应对$\cos^2\theta$取平均
\[
	\ave{\cos^2\theta}=\frac1{4\pi}\int_0^{2\pi}\int_0^\pi\cos^2\theta\sin\theta\d\theta\nd\phi=\frac13.
\]
跃迁速率 
\begin{align}
	w_{k'k}=\frac\pi{6\hbar^2}\abs{\bm D_{k'k}}^2E_0^2\vd\kh{\omega_{k'k}-\omega}.
\end{align}

对于非单色光,总跃迁速率是对各种频率求和
\begin{align}
	w\tot=\int\iti\omega_{k'k}\d\omega=\frac\pi{6\hbar^2}\abs{\bm D_{k'k}}^2E_0^2(\omega_{k'k}).
\end{align}

频率为$\omega$的电磁波能量密度的时间平均值
\begin{align*}
	\rho(\omega)=\frac1{8\pi}\ave{E^2+B^2}=\frac1{8\pi}E_0^2(\omega).\CGS
\end{align*}
故非偏振自然光引起的跃迁速率
\begin{align}
	w_{k'k}=\frac{4\pi^2}{3\hbar^2}\abs{\bm D_{k'k}}^2\rho(\omega_{k'k})=\frac{4\pi^2e^2}{3\hbar^2}\abs{\r_{k'k}}^2\rho(\omega_{k'k}).
\end{align}
由$\r$为奇宇称算符,只有$\ket k,\ket{k'}$宇称相反时,$\abs{\r_{k'k}}^2$才不为0。又
\begin{align*}
	\r=r[\sin\theta\cos\varphi,\sin\theta\sin\varphi,\cos\theta],
\end{align*}
由第 \pageref{costhetaYlm} 页的球谐函数递推式知,跃迁态间需满足$\D\ell=\pm 1,\,\D m=0,\pm 1$。
\paragraph{Einstein跃迁理论}