\chapter{基本假设}

\section{态矢量}

在量子力学中,一个物理态(physical state)由复Hilbert空间(Hilbert space)中的态矢量(state vector)表示。

\begin{definition}{Dirac符号}{Dirac notation}
	态矢量可以用一个列向量$\ket\psi$表示,称之为右矢(ket)。
	
	左矢(bra)是右矢的共轭转置$\bra\psi\equiv\ket\psi^\dagger$。
\end{definition}
\begin{definition}{Hilbert空间}{Hilbert space}
	Hilbert空间$H$是一个定义了内积(inner product)的向量空间。$\forall\ket a,\ket b\in H$,其内积$\brkt ab$满足以下性质:
\end{definition}

\begin{definition}{波函数}{Wave Function}
	波函数$\psi(\r)$刻画粒子在空间的概率分布.约定内积记号
	\[
		(\psi,\varphi)\equiv\int\psi^*(\r)\varphi(\r)\d\r,
	\]	
	波函数应是归一化的,即$(\psi,\psi)=1.$
\end{definition}
由Fourier展开,粒子的动量分布
\begin{align}
	\psi(\r)       & =\frac1{(2\pi\hbar)^{3/2}}\int\varphi(\bm p)\e{\i\bm p\cdot\r/\hbar}\d\bm p; \\
	\varphi(\bm p) & =\frac1{(2\pi\hbar)^{3/2}}\int\psi(\r)\e{-\i\bm p\cdot\r/\hbar}\d\bm r.
\end{align}
\begin{theorem}{不确定度关系}{Uncertainty Relation}
	% Heisenberg提出\textbf{不确定度关系}:
	不可能同时准确测量位移$\r$和动量$\bm p$,即
	\begin{align}
		\delta x\vd p\geqslant\hbar/2.
	\end{align}
	证明见第 \pageref{Proof of UR} 页。
\end{theorem}
由不确定度关系,不存在函数$\bm p(\r)$,需要以\textit{算符}表示$\bm p$和$\r$的关系,由动量的性质知动量算符的形式为
\[
	\qo p=-\i\hbar\nabla,
\]
\begin{definition}{平均值}{Mean Value}
	力学量的平均值,比如势能$V(\r)$
	\[
		\avg V\equiv(\psi,V\psi)=\int\abs{\psi(\r)}^2V(\r)\d\bm r,
	\]
	动量的平均值
	\[
		\avg{\bm p}=(\psi,\qo p\psi)=\int\psi^*(\r)\qo p\psi(\r)\d\bm r.
	\]
	推导见第 \pageref{Derivation of Mean Value} 页。
\end{definition}
\begin{theorem}{Schrödinger方程}{Schrödinger Equation}
	波函数随时间的演化满足
	\begin{align}
		\i\hbar\pv{\psi}t=\hat H\psi.
	\end{align}
	其中Hamilton算符
	\[
		\hat H=\frac{\hat p^2}{2m}+V.
	\] % =-\frac{\hbar^2}{2m}\nabla^2+V.\]
\end{theorem}
\begin{definition}{概率密度和概率流密度}{Probility Density}
	概率密度
	\[
		\rho(\r,t):=\psi^*(\r,t)\psi(\r,t),
	\]
	概率流密度
	\[
		\bm j(\r,t):=-\frac{\i\hbar}{2m}\kh{\psi^*\nabla\psi-\psi\nabla\psi^*}.
	\]
	由Schrödinger方程,二者有关系
	\[
		\pv\rho{t}=\psi^*\pv\psi{t}+\psi\pv{\psi^*}t=\nabla\cdot\bm j.
	\]
\end{definition}
考虑$\hat H$的本征值$E_n$和本征函数$\psi_n(\r)$
\begin{align}
	\hat H\psi_n(\r)=E_n\psi_n(\r).
\end{align}
称为定态Schrödinger方程,代表能量有确定值,而方程的通解可以写成本征方程的线性组合
\[
	\psi(\r,t)=\sum c_n\psi_n(\r)\e{\i E_nt/\hbar}.
\]