\chapter{微扰理论}
可以精确求解的量子力学问题很少,在处理各种实际问题时,除了采用适当的模型以简化问题外,往往还需要采用合适的近似解法。
\section{非简并定态微扰理论}
当$\hat H$比较复杂时,定态\Schr 方程不能精确求解,若$\hat H$具有以下形式
\[
	\hat H=\hat H_0+\hat H'
\]
其中$\hat H_0$是可解的,$\hat H'\ll\hat H_0$是小的修正,可令
\begin{align*}
	E_n&=E_n^{(0)}+E_n^{(1)}+E_n^{(2)}+\cdots\\
	\psi_n&=\psi_n^{(0)}+\psi_n^{(1)}+\psi_n^{(2)}+\cdots
\end{align*}
其中$E_n^{(k)}$和$\psi_n^{(k)}$与$\hat H'$的$k$次方成正比。代入原方程有
\[
	(\hat H_0+\hat H')(\psi_n^{(0)}+\psi_n^{(1)}+\cdots)=(E_n^{(0)}+E_n^{(1)}+\cdots)(\psi_n^{(0)}+\psi_n^{(1)}+\cdots)
\]
逐阶比较得到
\begin{align}
	&\text{零级方程:}&(\hat H_0-E_n^{(0)})\psi_n^{(0)}&=0;\\
	&\text{一级方程:}&(\hat H_0-E_n^{(0)})\psi_n^{(1)}&=-(\hat H'-E_n^{(1)})\psi_n^{(0)};\\
	&\text{二级方程:}&(\hat H_0-E_n^{(0)})\psi_n^{(2)}&=-(\hat H'-E_n^{(1)})\psi_n^{(1)}+E_n^{(2)}\psi_n^{(0)};\\\notag
	&\text{……}
\end{align}
一般说来,越高次的项越小,所以可以只保留最低的几阶,便有足够的精度。

以下约定:波函数的各级高级近似解与零级近似解都正交,即
\[
	\inp{\psi_n^{(0)}}{\psi_n^{(k)}}=0.
\]
对于非简并情形,即$\hat H_0$属于$E_n^{(0)}$的本征态只有一个
\[
	\psi_n^{(1)}=\sum_ma_{nm}^{(1)}\psi_m^{(0)}
\]
代入一级方程有
\[
	\sum_ma_{nm}^{(1)}(\hat H_0-E_n^{(0)})\psi_m^{(0)}=-(\hat H'-E_n^{(1)})\psi_n^{(0)}
\]
等式的两端与$\psi_k^{(0)}$内积
\[
a_{nk}^{(1)}(E_k^{(0)}-E_n^{(0)})=-\inp{\psi_k^{(0)}}{\hat H'\psi_n^{(0)}}+E_n^{(1)}\vd_{kn}
\]
取$k=n$得到一级微扰能
\begin{align}
	E_n^{(1)}=\inp{\psi_n^{(0)}}{\hat H'\psi_n^{(0)}}=:H_{nn}'
\end{align}
若取$k\neq n$,得到
\begin{align}
	a_{nk}^{(1)}=\frac{H_{kn}'}{E_k^{(0)}-E_n^{(0)}},\quad H_{kn}':=\inp{\psi_k^{(0)}}{\hat H'\psi_n^{(0)}}.
\end{align}
故一阶微扰波函数
\begin{align}
	\psi_n^{(1)}=\sum_{m\neq n}\frac{H_{mn}'}{E_m^{(0)}-E_n^{(0)}}\psi_m^{(0)}
\end{align}
微扰适用条件为
\begin{align}
	\abs{\frac{H_{mn}'}{E_m^{(0)}-E_n^{(0)}}}\ll 1.
\end{align}
\paragraph{二阶微扰能}二级微扰方程
\[
	(\hat H_0-E_n^{(0)})\psi_n^{(2)}=-(\hat H'-E_n^{(1)})\sum_{m\neq n}\frac{H_{kn}'}{E_k^{(0)}-E_n^{(0)}}\psi_m^{(0)}+E_n^{(2)}\psi_n^{(0)}
\]
两端与$\psi_n^{(0)}$内积,得到
\begin{gather}\notag
	0=-\sum_{m\neq n}\frac{H_{mn}'}{E_k^{(0)}-E_n^{(0)}}\inp{\psi_n^{(0)}}{\hat H'\psi_m^{(0)}}+0+E_n^{(2)},\\
	\implies E_n^{(2)}=\sum_{m\neq n}\frac{\abs{H_{mn}'}^2}{E_n^{(0)}-E_m^{(0)}}.
\end{gather}

准确到二级近似下,能量的本征值为
\begin{align}
	E_n=E_n^{(0)}+H_{nn}'+\sum_{m\neq n}\frac{\abs{H_{mn}'}^2}{E_n^{(0)}-E_m^{(0)}}
\end{align}
上述理论成立需要$\hat H_0$为分离谱,无简并。
\begin{example}{电介质的极化率}{}
	各向同性的非极性分子电介质在外电场作用下极化,求感生电偶极矩。

	考虑正离子运动,无外场时为简谐运动
	\begin{gather*}
		\hat H_0=-\frac{\hbar^2}{2\mu}\dd[2]x+\frac12\mu\omega^2x^2,\\
		\hat H_n\ket n=E_n^{(0)}\ket n,\quad E_n^{(0)}=\kh{n+\frac12}\hbar\omega.
	\end{gather*}
	沿$x$向加恒定电场相当于施加微扰$\hat H'=-q\varepsilon x$
	\begin{align*}
		H_{nk}'&=\bra n\hat H'\ket k=-q\varepsilon\bra nx\ket k\\
		&=-q\varepsilon\sqrt{\frac\hbar{\mu\omega}}\kh{\sqrt{\frac{k+1}2}\vd_{n,k+1}+\sqrt{\frac k2}\vd_{n,k-1}}
	\end{align*}
	一级能量修正$H_{kk}'=0$,二阶近似能量
	\begin{align*}
		E_k&=E_k^{(0)}+0+\sum_{n\neq k}\frac{\abs{H_{nk}'}^2}{E_k^{(0)}-E_n^{(0)}}=E_k^{(0)}-\frac{q^2\varepsilon^2}{2\mu\omega^2}.
	\end{align*}
	实际上,能量是有精确解的
	\[
V=\frac12\mu\omega^2x^2-q\varepsilon x=\frac12\mu\omega^2\kh{x-\frac{q\varepsilon}{\mu\omega^2}}^2-\frac{q^2\varepsilon^2}{2\mu\omega^2}.
\]
	它的第一项只不过是把原来的谐振子势能平移了一段距离,这个移动不会影响谐振子的能级,而它的第二项正是前面求出的常数项。

	一级近似态
	\begin{align*}
		\ket{\psi_k}&=\ket k+\sum_{n\neq k}\frac{H_{nk}'}{E_k^{(0)}-E_n^{(0)}}\ket n\\
		&=\ket k+\frac{q\varepsilon}{\sqrt{\hbar\mu\omega^3}}\kh{\sqrt{\frac{k+1}2}\ket{k+1}-\sqrt{\frac k2}\ket{k-1}},
	\end{align*}
	无外加场时,非极性分子正(负)离子的位置平均值$\bra kx\ket k=0$,即固有电偶极矩为零,而加外电场后正离子位移
	\[
		\bra{\psi_k}x\ket{\psi_k}=\frac{2q\varepsilon}{\sqrt{\hbar\mu\omega^3}}\kh{\sqrt{\frac{k+1}2}\brkt k{k+1}-\sqrt{\frac k2}\brkt k{k-1}}=\frac{q\varepsilon}{\mu\omega^2}.
\]
	故感生电偶极矩$D$和极化率$\kappa$
	\[
D=\abs q\frac{2\abs q\varepsilon}{\mu\omega^2}=\frac{2q^2\varepsilon}{\mu\omega^2},\quad \kappa=\frac D\varepsilon=\frac{2q^2}{\mu\omega^2}.
\]
\end{example}
\begin{example}{氦原子}{}
	氦原子在原子核外有两个电子,Hamilton量包括两个电子在原子核的Column引力场中的运动
	\[
		\hat H_0=\kh{-\frac12\nabla_1^2-\frac Z{r_1}}+\kh{-\frac12\nabla_1^2-\frac Z{r_2}};
\]
	以及两个电子之间的Column排斥能(微扰项)
	\[
		\hat H'=\frac1{r_{12}}.
\]

	它对于两个电子空间坐标的交换是对称的。由于电子是Fermi子,两个电子相应的自旋态只能反对称的自旋单态$\chi_{00}$
	\[
		\psi(\r_1,\r_2)=\psi_{100}(\r_1)\psi_{100}(\r_2)\chi_{00}(s_{1z},s_{2z}).
\]
	相应的本征值$E_1^0=-Z^2$,一级修正
	\[
		\ave{\frac1{r_{12}}}=\int\frac{\abs{\psi_{100}(\r_1)\psi_{100}(\r_2)}^2}{r_{12}}\d\r_1\nd\r_2.
\]
	由
	\[
		\psi_{100}(\r)=\frac{Z^{3/2}}{\sqrt\pi}\e{-Zr};\quad \int\frac{\e{-Z(r_1+r_2)}}{r_{12}}\d\r_1\nd\r_2=\frac{5\pi^2}{8Z^5}.
\]
	得
	\[
E=-Z^2+\frac58Z.
\]
\end{example}
\section{简并定态微扰理论}
实际问题中,特别是处理体系的激发态时,常常碰到简并态或近似简并态。此时,非简并态微扰论是不适用的。

这里首先碰到的困难是:零级能量给定后,对应的零级波函数并未确定,这是简并态微扰论首先要解决的问题。

体系能级的简并性与体系的对称性密切相关。当考虑微扰之后,如体系的某种对称性受到破坏,则能级可能分裂,简并将部分或全部解除。因而在简并态微扰中,充分考虑体系的对称性及其破缺是至关重要的。
\paragraph{一级微扰能和零级波函数}$E_n^{(0)}$简并时,
\[
	\hat H^{(0)}\psi_{ni}^{(0)}=E_n^{(0)}\psi_{ni}^{(0)},\quad (i=1,2,\ldots,k).
\]
简并度$f_n=k$,引入微扰后假设
\[
	\psi_n^{(0)}=\sum_{i=1}^kc_i^{(0)}\psi_{ni}^{(0)}.
\]
代入一级微扰方程
\[
	(\hat H_0-E_n^{(0)})\psi_n^{(1)}=-(\hat H'-E_n^{(1)})\sum_{i=1}^kc_i^{(0)}\psi_{ni}^{(0)}.
\]
两端与$\psi_{nj}^{(0)}$内积
\[
0=-\sum_{k=1}^nc_i^{(0)}\fkh{\inp{\psi_{nj}^{(0)}}{\hat H'\psi_{ni}^{(0)}}-E_n^{(1)}\vd_{ij}}.
\]
记
\begin{align}
	H_{ji}':=\inp{\psi_{nj}^{(0)}}{\hat H'\psi_{ni}^{(0)}}
\end{align}
得到久期(secular)方程
\begin{align}
	\det(H'-E_n^{(1)}I)=\begin{vmatrix}
		H_{11}'-E_n^{(1)}&H_{12}'&\cdots&H_{1k}'\\
		H_{21}'&H_{22}'-E_n^{(1)}&\cdots&H_{2k}'\\
		\vdots&\vdots&\ddots&\vdots\\
		H_{k1}'&H_{k2}'&\cdots&H_{kk}'-E_n^{(1)}
	\end{vmatrix}=0.
\end{align}
从中可以解出$E_n^{(1)}$以及它们对应的$c_i^{(0)}$,这就决定了一级微扰能和零级波函数。
\paragraph{Stark效应}Stark效应就是原子或分子在外电场作用下能级和光谱发生分裂的现象。
\sectionstar{变分法}没讲