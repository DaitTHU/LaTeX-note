\def\coursename{微扰论作业}
\documentclass[a4paper, 11pt]{article}

\usepackage[UTF8]{ctex}

\usepackage[T1]{fontenc}								% 字体
\catcode`\。=\active
\newcommand{。}{.} % {\ifmmode\text{.}\else .\fi}
\catcode`\(=\active
\catcode`\)=\active
\newcommand{(}{(}
\newcommand{)}{)}

% \usepackage{zhlineskip}

\usepackage{nicematrix}
% \usepackage{setspace}
% \linespread{1}						% 一倍行距
\setlength{\headheight}{14pt}			% 页眉高度
% \setlength{\lineskip}{0ex}			% 行距
\renewcommand\arraystretch{.82}		% 表格

\usepackage{amssymb, amsmath, amsfonts, amsthm}			% 数学符号,公式,字体,定理环境
\everymath{\displaystyle}			% \textstyle \scriptstyle \scriptscriptstyle
\allowdisplaybreaks[4]      		% 使用行间公式格式
% \makeatletter
% \renewcommand{\maketag@@@}[1]{\hbox{\m@th\normalsize\normalfont#1}}
% \makeatother
\newif\ifcontent\contenttrue		% if 显示目录
\newif\ifparskip\parskipfalse		% if 增加目录后的行距
\newif\ifshowemail\showemailfalse	% if 显示 email
\def\firstandforemost{
	\maketitle
	%\thispagestyle{empty}\clearpage
	\ifcontent
		\renewcommand{\contentsname}{目录}
		\tableofcontents
		\thispagestyle{empty}
		\clearpage
	\fi
	\ifparskip
		\setlength{\parskip}{.8ex}	% 设置额外的段距,目录后
	\fi								% 在 \firstandforemost 前设置 \parskiptrue
	\makenomenclature
	\printnomenclature
	\setcounter{page}{1}
}

\usepackage{mathtools}									% \rcase 环境等

% \usepackage{physics}

\usepackage[]{siunitx}									% 国际制单位
\sisetup{
	inter-unit-product = \ensuremath{{}\cdot{}},
	per-mode = symbol,
	per-mode = reciprocal-positive-first,
	range-units = single,
	separate-uncertainty = true,
	range-phrase = \ifmmode\text{\;-\;}\else\;-\;\fi
}
\DeclareSIUnit\angstrom{\text{Å}}
\DeclareSIUnit\atm{\text{atm}}
% SIunits 额外定义了一个 \square 表示平方,
% 还会把 \cdot 空格加大,真有够无语的 😅

\usepackage{authblk}									% 作者介绍
\ifx \coursefullname\undefined
	\ifx \coursename\undefined
		\def\coursename{笔记}
	\fi
	\def\coursefullname{\coursename}
\fi
\ifx \authorname\undefined
	\def\authorname{Dait}
\fi
\ifx \departmentname\undefined
	\def\departmentname{THU}
\fi
\ifx \emailaddress\undefined
	\def\emailaddress{daiyj20@mails.tsinghua.edu.cn}
\fi
\ifx \beginday\undefined
	\def\beginday{2021}
\fi
\ifx \endday\undefined
	\def\endday{\number\year/\number\month/\number\day}
\fi
\ifx \titleannotation\undefined
	\ifx \teachername\undefined
		\title{\textbf{\coursefullname}}
	\else
		\title{\textbf{\coursefullname}\\\small\textit{主要整理自\teachername 老师讲义}}
	\fi
\else
	\title{\textbf{\coursefullname}\\\small\textit{(\titleannotation)}}
\fi
\newif\ifdefaultauthor\defaultauthortrue
\ifdefaultauthor
	\author{by~\authorname~at~\departmentname}
	\ifshowemail
		\affil{\emailaddress}
	\fi
\fi
\ifx \endday\beginday
	\date{\beginday}
\else
	\date{\beginday~-~\endday}
\fi

\usepackage{hyperref}									% 链接
\ifx \courseEnglishname\undefined
	\def\courseEnglishname{Note}
\fi
\ifx \authorEnglishname\undefined
	\def\authorEnglishname{Dait}
\fi
\hypersetup{
	% dvipdfm								% 表示用 dvipdfm 生成 pdf
	pdftitle={\coursename},
	pdfauthor={\authorname},
	colorlinks=true, breaklinks=true,		% 超链接设置
	linkcolor=black, citecolor=black, urlcolor=blue
}

\usepackage[british]{babel}								% 长单词自动连字符换行
\hyphenation{long-sen-ten-ce}				% 自定义拆分方式

\usepackage{tikz}
\usetikzlibrary{quotes, angles}
\usepackage{pgfplots}
\pgfplotsset{compat=1.17}								% TikZ
\newcommand{\coor}[5][0]{
	\draw[thick,latex-latex](#1,#3)node[left]{$#5$}--(#1,0)node[shift={(-135:7pt)}]{$O$}--(#1+#2,0)node[right]{$#4$}
}			% 坐标轴

\usepackage{enumerate}									% 编号
\usepackage{paralist}
\setlength{\pltopsep}{1ex}
\setlength{\plitemsep}{1ex}
\ifx \eqnrange\undefined
	\numberwithin{equation}{section}
\else
	\numberwithin{equation}{\eqnrange}
\fi

\renewcommand{\thempfootnote}{\Roman{mpfootnote}}
\renewcommand{\thefootnote}{\Roman{footnote}}		% 注释上标 I, II,...
\newcommand{\sectionstar}[1]{
	\section[\hspace{-.8em}*\hspace{.3em}#1]{\hspace{-1em}*\hspace{.5em}#1}
}
\newcommand{\subsectionstar}[1]{					% 带星号的 section 和 subsection
	\subsection{\hspace{-1em}*\hspace{.5em}#1}
}
\newcommand{\subsubsectionstar}[1]{					% 带星号的 section 和 subsection
	\subsubsection{\hspace{-1em}*\hspace{.5em}#1}
}
\newcommand*{\appendiks}{
	\appendix
	\part*{附录}
	\addcontentsline{toc}{part}{附录}
}
\iffalse			% 不清楚
	\newcommand{\varsection}[1]{
		\refstepcounter{section}
		\section*{\thesection\quad #1}
		\addcontentsline{toc}{section}{\makebox[0pt][r]{*}\thesection\quad #1}
	}
\fi

\usepackage{fancyhdr}									% 页眉页脚
\ifx \coursename\undefined
	\def\coursename{笔记}
\fi
\fancyhf{}\pagestyle{fancy}
\fancyhead[L]{\coursename\rightmark}
\fancyhead[R]{by~\authorname}
\fancyfoot[C]{-~\thepage~-} 			%页码

\usepackage{colortbl, booktabs}							% 表

\usepackage{graphicx}
\usepackage{float}
\usepackage{caption}									% 图
\captionsetup{
	margin=20pt, format=hang,
	justification=justified
}
\newcounter{tikzpic}
\def\tikzchap{
	\stepcounter{tikzpic}\\
	\small 图~\thetikzpic\quad
}
\newcounter{linetable}
\newcommand{\tablechap}[1]{
	\stepcounter{linetable}
	{\small 表~\thelinetable\quad #1}\\[1em]
}

\usepackage{tcolorbox}									% 盒子
\tcbuselibrary{theorems, skins, breakable}
\definecolor{MatchaGreen}{HTML}{73C088}		% 抹茶绿B7C6B3
\newtcbtheorem[number within = subsection]{example}{例}{
	enhanced, breakable, sharp corners,
	attach boxed title to top left = {yshifttext = -1mm},
	before skip = 2ex,
	colback = MatchaGreen!5,				% 文本框内的底色
	colframe = MatchaGreen,					% 文本框框沿的颜色
	fonttitle = \bfseries,					% 标题字体用粗体	coltitle 默认 white,
	boxed title style = {
			sharp corners, size = small, colback = MatchaGreen,
		}
}{exm}
\definecolor{MelancholyBlue}{HTML}{9EAABA}	% melancholy: 沮丧
\newcounter{pslt}
\setcounter{pslt}{-1}
\newtcbtheorem[use counter = pslt]{posulate}{假设}{
	enhanced, breakable, sharp corners,
	attach boxed title to top left = {yshifttext = -1mm}, before skip = 2ex,
	colback = MelancholyBlue!5, colframe = MelancholyBlue, fonttitle = \bfseries,
	boxed title style = {
			sharp corners, size = small, colback = MelancholyBlue,
		}
}{psl}
\definecolor{PureBlue}{HTML}{80A3D0}
\newtcbtheorem[number within = subsection]{definition}{定义}{
	enhanced, breakable, sharp corners,
	attach boxed title to top left = {yshifttext = -1mm}, before skip = 2ex,
	colback = PureBlue!5, colframe = PureBlue, fonttitle = \bfseries,
	boxed title style = {
			sharp corners, size = small, colback = PureBlue,
		}
}{dfn}
\definecolor{PeachRed}{HTML}{EA868F}
\newtcbtheorem[number within = subsection]{theorem}{定理}{
	enhanced, breakable, sharp corners,
	attach boxed title to top left = {yshifttext = -1mm}, before skip = 2ex,
	colback = PeachRed!5, colframe = PeachRed, fonttitle = \bfseries,
	boxed title style = {
			sharp corners, size = small, colback = PeachRed,
		}
}{thm}
\definecolor{SchembriumYellow}{HTML}{fbd26a}	% 申博太阳城黄
\newtcbtheorem[number within = section]{method}{方法}{
	enhanced, breakable, sharp corners,
	attach boxed title to top left = {yshifttext = -1mm}, before skip = 2ex,
	colback = SchembriumYellow!5, colframe = SchembriumYellow, fonttitle = \bfseries,
	boxed title style = {
			sharp corners, size = small, colback = SchembriumYellow,
		}
}{mtd}
% 保留颜色
\definecolor{fadedgold}{HTML}{D9CBB0}		% 褪色金
\definecolor{saturatedgold}{HTML}{F0E0C2}	% staurated: 饱和
\definecolor{elegantblue}{HTML}{C4CCD7}		% elegant: 优雅
\definecolor{ivory}{HTML}{F1ECE6}			% 象牙
\definecolor{gloomypruple}{HTML}{CCC1D2}	% 阴沉紫
% \textcolor[HTML]{FFC23A}					% 石板灰

\definecolor{Green}{rgb}{0,.8,0}

\usepackage{imakeidx}								% 索引

\usepackage{nomencl}								% 关键词
%\setlength{\nomitemsep}{0.2cm}							% 设置术语之间的间距
\renewcommand{\nomentryend}{.}							% 设置打印出术语的结尾的字符
\renewcommand{\eqdeclaration}[1]{见公式:(#1)}			% 设置打印见公式的样式
\renewcommand{\pagedeclaration}[1]{见第 (#1) 页}		% 设置打印页的样式
\renewcommand{\nomname}{术语表} 						% 修改术语表标题的名称。

\usepackage{array}
\usepackage{booktabs} % 三线表
\usepackage{multirow}
% 手动排版,尽量杜绝使用

\newcommand{\bs}[1]{\hspace{-#1 pt}}		% 手动减间距	backspace
\newcommand{\bv}[1]{\vspace{-#1 pt}}		% 手动缩行距	backvspace
\def\directlisteqn{\vspace{-1ex}}
\iffalse									% 尽量避免孤行
	\widowpenalty=4000
	\clubpenalty=4000
\fi

% 杂项符号
\def\avg{\overline}
\newcommand*{\rqed}{\tag*{$\square$}}								% 靠右 QED
\newcommand*{\halfqed}{\tag*{$\boxdot$}}
\newcommand*{\thus}{\quad\Rightarrow\quad}							% =>
\newcommand*{\ifnf}{\quad\Leftrightarrow\quad}						% <=>	if and only if
\newcommand*{\turnto}{\quad\to\quad}
\newcommand*{\normalize}{\quad\overset{\mathrm{normalize}}{-\!\!\!-\!\!\!-\!-\!\!\!\longrightarrow}\quad}
\newcommand*{\vthus}{\\$\Downarrow$\\}
\newcommand*{\viff}{\\$\Updownarrow$\\}
\newcommand*{\vs}{~\text{-}~}
\newcommand{\eg}[1][]{\subparagraph*{例#1:}}
\newcommand*{\prf}{\noindent\textbf{证明:}\quad}
\newcommand{\dpfr}[2]{\displaystyle\frac{#1}{#2}}					% 大分数
\newcommand{\frdp}[2]{\frac{\displaystyle #1}{\displaystyle #2}}
\newcommand{\spark}[1]{\;\textcolor{red}{#1}}

% 简化更常用的希腊字母
\newcommand*{\vf}{\varphi}
\newcommand*{\vF}{\varPhi}
\newcommand*{\vp}{\varPsi}
\newcommand*{\ve}{\varepsilon}
\newcommand*{\vC}{\varTheta}
\newcommand*{\ct}{\theta}			% 还是建议用 @ + Tab 快捷键

% 正体符号
\newcommand*{\cns}{\mathrm{const}}
\newcommand*{\plusc}{{\color{lightgray}\,+\,\cns}}
\newcommand*{\e}{\mathop{}\!\mathrm{e}^}	% e
\let\accenti\i
\renewcommand*{\i}{\mathrm{i}}
\newcommand*{\D}{\Delta}
\newcommand*{\p}{\partial}

\usepackage{bm}											% 粗体 \bm
\newcommand{\hbm}[1][r]{\hat{\bm #1}}	% 应该不会有两个字母的
\newcommand{\ibm}[1]{\,\bm #1}

% Using EnglischeSchT script font style
\newfontfamily{\calti}{EnglischeSchT}
\newcommand{\mathcalti}[1]{\mbox{\calti{#1}}}
\newcommand{\mathcaltibf}[1]{\mbox{\bf\calti{#1}}}

\usepackage{mathrsfs}									% 花体 \mathscr
% \usepackage{boondox-cal}								% 小写花体 \mathcal
\newcommand*{\RR}{\mathbb R}
\newcommand*{\CC}{\mathbb C}
\newcommand*{\ZZ}{\mathbb Z}
\newcommand*{\sC}{\mathscr C}			% n 阶连续可导函数
\newcommand*{\sR}{\mathscr R}			% 黎曼可积
% 算符用 \mathcal
\newcommand*{\cL}{\mathcal L}			% 表示一般算子
\newcommand{\cl}[1]{\mathcal L\fkh{#1}}
\newcommand{\cli}[1]{\mathcal L^{-1}\!\fkh{#1}}
\newcommand{\cf}[2][\!\,]{\mathcal F_\mathrm{#1}\fkh{#2}}
\newcommand{\cfi}[2][\!\,]{\mathcal F_\mathrm{#1}^{-1}\!\fkh{#2}}
% \newcommand{\cl}[2][0]{\mathcal L\ikh[#1]{#2}}
% \newcommand{\cli}[2][0]{\mathcal L^{-1}\ikh[#1]{#2}}
% \newcommand{\cf}[2][0]{\mathcal F\ikh[#1]{#2}}
% \newcommand{\cfi}[2][0]{\mathcal F^{-1}\ikh[#1]{#2}}

\usepackage{cancel}										% 删除线

\usepackage{xfrac}

% \usepackage{emoji}	需要 LuaTeX

% 导数等
\let\divides\div
\renewcommand*{\div}{\nabla\cdot}
\newcommand*{\curl}{\nabla\times}
\newcommand*{\lap}{\Delta}
\let\accentd\d
\renewcommand*{\d}{\mathop{}\!\mathrm{d}}
\newcommand*{\nd}{\mathrm{d}}
\newcommand*{\vd}{\mathop{}\!\delta}											% δ
\newcommand{\dd}[2][\;\!\!]{\frac{\nd^{#1}}{\nd #2^{#1}}}						% d/dx			我知道 \,\! 很愚蠢,但是 {} 无法在 Math Preview 上预览
\newcommand{\dn}[2]{\frac{\nd^{#1}}{\nd #2^{#1}}}								% d^n/dx^n		\dn2x≡\dd[2]x
\newcommand{\dv}[3][\;\!\!]{\frac{\nd^{#1}#2}{\nd #3^{#1}}}						% df/dx
\newcommand{\du}[3]{\frac{\nd^{#1}#2}{\nd #3^{#1}}}								% d^nf/dx^n		\du2fx≡\dv[2]fx
\newcommand{\pp}[2][\;\!\!]{\frac{\p^{#1}}{\p #2^{#1}}}							% ∂/∂x
\newcommand{\pn}[2]{\frac{\p^{#1}}{\p #2^{#1}}}									% ∂^n/∂x^n		\pn2x≡\pp[2]x
\newcommand{\pv}[3][\;\!\!]{\frac{\p^{#1}#2}{\p #3^{#1}}}						% ∂f/∂x
\newcommand{\pu}[3]{\frac{\p^{#1}#2}{\p #3^{#1}}}								% ∂^nf/∂x^n		\pu2x≡\pv[2]x
\newcommand{\pw}[3]{\frac{\p^2 #1}{\p #2\p #3}}									% ∂^2f/∂x∂y
\newcommand{\pvv}[6]{															% ∂^(m+n)f/∂x^m∂y^n
	\ifnum#4=1
		\ifnum#6=1
			\frac{\p^{#1}#2}{\p #3\p #5}
		\else
			\frac{\p^{#1}#2}{\p #3\p #5^{#6}}
		\fi
	\else
		\ifnum#6=1
			\frac{\p^{#1}#2}{\p #3^{#4}\p #5}
		\else
			\frac{\p^{#1}#2}{\p #3^{#4}\p #5^{#6}}
		\fi
	\fi}
\newcommand{\dvd}[2]{\left.#1\middle\slash #2\right.}							% 斜除

% 积分
\newcommand*{\intt}{\bs2\int\bs8\int}											% ∫∫
\newcommand*{\inttt}{\int\bs8\int\bs8\int}										% ∫∫∫
\newcommand*{\intdt}{\int\bs3\cdot\bs2\cdot\bs2\cdot\bs4\int}					% ∫...∫
\newcommand*{\zti}{_0^{+\infty}}												% _0^+∞
\newcommand*{\iti}{_{-\infty}^{+\infty}}										% _-∞^+∞
\newcommand{\fmto}[3][\infty]{_{#2=#3}^{#1}}

% 括号
\newcommand{\abs}[1]{\left\lvert#1\right\rvert}									% |x| 绝对值
\newcommand{\norm}[1]{\left\lVert#1\right\rVert}								% ||x|| 模
\newcommand{\edg}[1]{\left.#1\right\rvert}										% f|  竖线
\newcommand{\kh}[1]{\left(#1\right)}											% (x) 括号
\newcommand{\fkh}[1]{\left[#1\right]}											% [x] 方括号
\newcommand{\hkh}[1]{\left\{#1\right\}}											% {x} 花括号
\newcommand{\zkh}[1]{\lfloor\bs{4.7}\lceil #1\rceil\bs{4.7}\rfloor}				% [x] 中括号
\newcommand{\ikh}[2][0]{\ifnum#1=0 \zkh{#2}\else \fkh{#2}\fi}					% [x] 可调大小的中括号
\newcommand{\set}[2]{\left\{#1\,\middle\vert\,#2\right\}}						% {x|x1,x2,...} 集合
\newcommand{\ave}[1]{\left\langle #1\right\rangle}								% <x> 平均值
\newcommand{\bra}[1]{\left\langle #1\right\vert}								% <ψ| 左矢
\newcommand{\ket}[1]{\left\vert #1\right\rangle}								% |ψ> 右矢
\newcommand{\brkt}[2]{\left\langle #1\middle\vert #2\right\rangle}				% <φ|ψ> 内积
\newcommand{\ktbr}[2]{\left\vert#1\right\rangle \bs3\left\langle #2\right\vert}	% |ψ><φ|
\newcommand{\inp}[2]{\left\langle #1,#2\right\rangle}							% <f,g> 内积

% 数学运算符
\let\Real\Re
\let\Imagin\Im
\let\Re\relax
\let\Im\relax
\DeclareMathOperator{\Re}{Re}					% 
\DeclareMathOperator{\Im}{Im}					% 
\DeclareMathOperator{\sech}{sech}				% 
\DeclareMathOperator{\csch}{csch}				% 
\DeclareMathOperator{\arcsec}{arcsec}			% 
\DeclareMathOperator{\arccot}{arccot}			% 
\DeclareMathOperator{\arccsc}{arccsc}			% 
\DeclareMathOperator{\arsinh}{arsinh}			% 
\DeclareMathOperator{\arcosh}{arcosh}			% 
\DeclareMathOperator{\artanh}{artanh}			% 
\DeclareMathOperator{\sgn}{sgn}					% 符号函数
\DeclareMathOperator{\Li}{Li}					% 
\DeclareMathOperator{\Si}{Si}
\DeclareMathOperator{\Ci}{Ci}
\DeclareMathOperator{\sinc}{sinc}
\DeclareMathOperator{\Heaviside}{H}
\DeclareMathOperator{\arr}{A}					% 排列数
\DeclareMathOperator{\com}{C}					% 组合数
\DeclareMathOperator{\Res}{Res}					% 留数
\DeclareMathOperator{\supp}{supp}
\newcommand*{\bigo}{\mathcal O}

% 线性代数
\newif\ifLinearAlgebra\LinearAlgebratrue
\ifLinearAlgebra
\DeclareMathOperator{\rank}{rank}
\DeclareMathOperator{\id}{id}
\newcommand*{\tp}{^{\mathrm T}}				% AT 转置
\newcommand*{\cj}{^\ast}					% A* 共轭
\newcommand*{\dg}{^\dagger}					% A† 共轭转置
\newcommand*{\iv}{^{-1}}					% A-1
\fi

% 物理学家
\newcommand*{\Schr}{Schrödinger}
\newcommand*{\Legd}{Legendre}
\newcommand*{\deB}{de Broglie}
\newcommand*{\Rayl}{Rayleigh}
\newcommand*{\Lande}{Landé}

% 粒子
\newcommand*{\elc}{\mathrm e}
\newcommand*{\pton}{\mathrm p}
\newcommand*{\mol}{\mathrm m}

% 物理常数
\newcommand*{\NA}{N_{\bs1\mathrm A}}						% Avogadro 常数
\newcommand*{\kB}{k_{\mathrm B}}							% Boltzmann 常数
\newcommand*{\muB}{\mu_\mathrm B}							% Bohr 磁矩

% 
\newcommand*{\Ek}{E_{\mathrm k}}							% 动能
\newcommand*{\eff}{_\mathrm{eff}}							% 有效下标
\newcommand*{\tot}{_\mathrm{tot}}
\newcommand*{\lSI}{\tag{SI}}
\newcommand*{\CGS}{\tag{CGS}}								% cm, g, s 制


\renewcommand*{\r}{\bm r}
\newcommand{\qo}[1]{\bm{\hat #1}\!\mathop{}}

\begin{document}
	\part{作业}
	\section{10.1}
	零级方程 
	\begin{align}
		(\hat H_0-E_n^{(0)})\ket n=0.
	\end{align}
	由谐振子结论,$E_n^{(0)}=\kh{n+1/2}\hbar\omega$
	\begin{align}
		\brkt xn=\psi_n^{(0)}(x)=\sqrt{\frac\alpha{\sqrt\pi 2^nn!}}H_n(\alpha x)\e{-\alpha^2x^2/2},\quad\alpha=\sqrt{\frac{\mu\omega}\hbar}.
	\end{align}
	由Hermite多项式的结论
	\begin{align}
		2z H_n(z)=2nH_{n-1}(z)+H_{n+1}(z)
	\end{align}
	故
	{\small\begin{align}
		x\ket n&=\frac1{\sqrt2\alpha}\kh{\sqrt n\ket{n-1}+\sqrt{n+1}\ket{n+1}};\\
		x^2\ket n&%=\frac1{\alpha^2}\fkh{\sqrt{\frac n2}\kh{\sqrt{\frac{n-1}2}\ket{n-2}+\sqrt{\frac n2}\ket n}+\sqrt{\frac{n+1}2}\kh{\sqrt{\frac{n+1}2}\ket n+\sqrt{\frac{n+2}2}\ket{n+2}}}\\
		%&
		=\frac1{2\alpha^2}\fkh{\sqrt{n(n-1)}\ket{n-2}+(2n+1)\ket n+\sqrt{(n+1)(n+2)}\ket{n+2}};\\\notag
		x^3\ket n&=\frac1{2\sqrt2\alpha^3}\left[\sqrt{n(n-1)(n-2)}\ket{n-3}+3n\sqrt n\ket{n-1}+\right.\\
		&\qquad\left. 3(n+1)\sqrt{n+1}\ket{n+1}+\sqrt{(n+1)(n+2)(n+3)}\ket{n+3}\right].
	\end{align}}
	因此 
	\begin{align}\notag
		H_{nk}'&=\bra n\hat H'\ket k=\beta\bra nx^3\ket k\\
		&=\beta\kh{\frac\hbar{2\mu\omega}}^{3/2}\fkh{\sqrt{n(n-1)(n-2)}\vd_{n,k-3}+\cdots}.\label{Hnk}
	\end{align}

	一级微扰能$E_n^{(1)}=H_{nn}'=0$,二级微扰能
	\begin{align}
		E_n^{(2)}=\sum_{m\neq n}\frac{\abs{H_{mn}'}^2}{E_n^{(0)}-E_m^{(0)}}=\sum_{m=\cdots}\frac{\abs{H_{mn}'}^2}{(n-m)\hbar\omega}
	\end{align}
	由式(\ref{Hnk})知,$m=n\pm 1,n\pm 3$,代入有
	{\footnotesize\begin{align}\notag
		E_n^{(2)}&=\frac{\beta^2}{\hbar\omega}\kh{\frac\hbar{2\mu\omega}}^3\fkh{\frac{n(n-1)(n-2)}3+9n^3-9(n+1)^3-\frac{(n+1)(n+2)(n+3)}3}\\
		&=-(30n^2+30n+11)\frac{\beta^2\hbar^2}{8\mu^3\omega^4}.
	\end{align}}
	故能量本征值 
	\begin{align}
		E_n=\kh{2n+1}\frac{\hbar\omega}2-(30n^2+30n+11)\frac{\beta^2\hbar^2}{8\mu^3\omega^4}.
	\end{align}

	一阶微扰波函数
	\begin{align}
		\psi_n^{(1)}&=\sum_{m\neq n}\frac{H_{mn}'}{E_m^{(0)}-E_n^{(0)}}\psi_m^{(0)}\\\notag
		&=\frac\beta{\hbar\omega}\kh{\frac\hbar{2\mu\omega}}^{3/2}\left[\frac{\sqrt{n(n-1)(n-2)}}3\psi_{n-3}^{(0)}+3n\sqrt n\psi_{n-1}^{(0)}\right.
		\\
		&\left. \qquad -3(n-1)\sqrt{n-1}\psi_{n+1}^{(0)}-\frac{\sqrt{(n+1)(n+2)(n+3)}}3\psi_{n+3}^{(0)}\right].
	\end{align}
	波函数 
	\begin{align}
		\psi_n(x)=\psi_n^{(0)}(x)+\psi_n^{(1)}(x).
	\end{align}

	\section{10.2}
	\subsection*{(a)}
	分离$\varPsi(x_1,x_2)=\psi_1(x_1)\psi_2(x_2)$
	\begin{align}
		H_0=\kh{-\frac{\hbar^2}{2\mu}\pp[2]{x_1}+\frac12\mu\omega^2x_1^2}+\kh{-\frac{\hbar^2}{2\mu}\pp[2]{x_2}+\frac12\mu\omega^2x_2^2}.
	\end{align}
	易知,$\psi_1,\psi_2$分别满足谐振子。故能量本征值
	\begin{align}
		E_N^{(0)}=(n_1+n_2+1)\hbar\omega=:(N+1)\hbar\omega.
	\end{align}
	能级简并度$f=N+1$。
	\subsection*{(b)}
	第一激发态$N=1$,有
	\begin{align}
		\varPsi_{11}^{(0)}=\psi_{1}(x_1)\psi_{0}(x_2),\quad\varPsi_{12}^{(0)}=\psi_{0}(x_1)\psi_{1}(x_2).
	\end{align}
	则
	\begin{align}
		H_{11}'&=\inp{\varPsi_{11}^{(0)}}{\hat H'\varPsi_{11}^{(0)}}=-\lambda\bra 1x_1\ket 1\bra 0x_2\ket 0=0,\\
		H_{12}'=H_{21}'&=\inp{\varPsi_{11}^{(0)}}{\hat H'\varPsi_{12}^{(0)}}=-\lambda\bra 1x_1\ket 0\bra 0x_2\ket 1=-\frac\lambda{2\alpha^2},\\
		H_{22}'&=\inp{\varPsi_{12}^{(0)}}{\hat H'\varPsi_{12}^{(0)}}=-\lambda\bra 0x_1\ket 0\bra 1x_2\ket 1=0,
	\end{align}
	久期方程
	\begin{align}
		\begin{vmatrix}
			H_{11}'-E_1^{(1)}&H_{12}'\\
			H_{21}'&H_{22}'-E_1^{(1)}
		\end{vmatrix}=0,\thus E_1^{(1)}=\pm\frac\lambda{2\alpha^2}.
	\end{align}
	故一级近似
	\begin{align}
		E_1=2\hbar\omega\pm\frac{\lambda\hbar}{2\mu\omega}.\label{EN}
	\end{align}
	\subsection*{(c)}
	作坐标变换
	\begin{align}
		\begin{cases}
			x_1=\frac1{\sqrt2}(\xi+\eta)\\
			x_2=\frac1{\sqrt2}(\xi-\eta)
		\end{cases}\ifnf
		\begin{cases}
			\xi=\frac1{\sqrt2}(x_1+x_2)\\
			\eta=\frac1{\sqrt2}(x_1-x_2)
		\end{cases}
	\end{align}
	则
	\begin{align}
		\pp{x_1}&=\pv\xi{x_1}\pp\xi+\pv\eta{x_1}\pp\eta=\frac1{\sqrt2}\kh{\pp\xi+\pp\eta};\\
		\pp[2]{x_1}&=\frac12\kh{\pp[2]\xi+\pp\xi\pp\eta+\pp[2]\eta}.
	\end{align}
	故
	\begin{align}\notag
		H_0&=-\frac{\hbar^2}{2\mu}\kh{\pp[2]{x_1}+\pp[2]{x_2}}+\frac12\mu\omega^2\kh{x_1^2+x_2^2}\\
		&=-\frac{\hbar^2}{2\mu}\kh{\pp[2]\xi+\pp[2]\eta}+\frac12\mu\omega^2(\xi^2+\eta^2);\\
		H'&=-\lambda x_1x_2=-\frac\lambda{2}(\xi^2-\eta^2).
	\end{align}
	故
	\begin{align}\notag
		H&=-\frac{\hbar^2}{2\mu}\kh{\pp[2]\xi+\pp[2]\eta}+\frac12\mu\omega^2(\xi^2+\eta^2)-\frac\lambda{2}(\xi^2-\eta^2)\\
		&=\fkh{-\frac{\hbar^2}{2\mu}\pp[2]\xi+\frac12(\mu\omega^2-\lambda)\xi^2}+\fkh{-\frac{\hbar^2}{2\mu}\pp[2]\eta+\frac12(\mu\omega^2+\lambda)\eta^2}.
	\end{align}
	可得能量本征值
	\begin{align}
		E_{n_\xi,n_\eta}=\kh{n_\xi+\frac12}\hbar\sqrt{\omega^2-\frac\lambda\mu}+\kh{n_\eta+\frac12}\hbar\sqrt{\omega^2+\frac\lambda\mu}.
	\end{align}
	取$n_\xi=1,n_\eta=0$,则
	\begin{align}
		E_{1,0}=\frac{\hbar\omega}{2}\kh{3\sqrt{1-\frac\lambda{\mu\omega^2}}+\sqrt{1+\frac\lambda{\mu\omega^2}}}
	\end{align}
	与式(\ref{EN})比较,当$\lambda\ll\mu\omega^2$时,有
	\begin{align}
		E_{1,0}\doteq\frac{\hbar\omega}{2}\fkh{3\kh{1-\frac\lambda{2\mu\omega^2}}+\kh{1+\frac\lambda{2\mu\omega^2}}}=2\hbar\omega-\frac{\lambda\hbar}{2\mu\omega}.
	\end{align}
	而取$n_\xi=0,n_\eta=1$对应式(\ref{EN})取$+$。微扰项需$\lambda\ll\mu\omega^2$。

	\section{10.8}
	\subsection*{(a)}
	设转子能量本征函数为$\psi(\vf)$,
	\begin{align}
		H\psi=-\frac{\hbar^2}{2I}\pv[2]\psi\vf=E\psi,
	\end{align}
	解得
	\begin{align}
		\psi_m(\vf)=\frac1{\sqrt{2\pi}}\e{\i m\vf},\quad m=0,\pm 1,\pm 2,\ldots
	\end{align}
	能量本征值 
	\begin{align}
		E_m=\frac{m^2\hbar^2}{2I}.
	\end{align}
	\subsection*{(b)}
	由于一个$E_m$对应$\pm m$,故简并度$f=2$。
	\begin{align}
		H_{++}'&=\inp{\psi_m}{\hat H'\psi_m}=-\frac{D\mathscr E}{2\pi}\int_0^{2\pi}\cos\vf\d\vf=0;\\
		H_{+-}'&=\inp{\psi_m}{\hat H'\psi_{-m}}=-\frac{D\mathscr E}{2\pi}\int_0^{2\pi}\cos\vf\e{-2\i m\vf}\d\vf=0;\\
		H_{--}'&=\inp{\psi_{-m}}{\hat H'\psi_{-m}}=-\frac{D\mathscr E}{2\pi}\int_0^{2\pi}\cos\vf\d\vf=0.
	\end{align}
	一级微扰能$E_m^{(1)}=0$,故能级仍为$E_m$。

	设
	\begin{align}
		\psi_m^{(1)}=\sum_nc_{mn}\psi_n^{(0)}
	\end{align}
	代入一级微扰方程$(\hat H_0-E_m^{(0)})\psi_m^{(1)}=-(\hat H'-E_m^{(1)})\psi_m^{(0)}$
	\begin{align}
		\sum_nc_{mn}(E_n-E_m)\psi_n^{(0)}&=-\hat H'\psi_m^{(0)}
	\end{align}
	与$\psi_k^{(0)}$内积
	\begin{align}\notag
		(E_k-E_m)c_{mk}&=-\inp{\psi_k^{(0)}}{\hat H'\psi_m^{(0)}}\\
		\frac{\hbar^2}{2I}(k^2-m^2)c_{mk}&=\frac{D\mathscr E}{2\pi}\int_0^{2\pi}\cos\vf\e{\i(m-k)\vf}\d\vf
	\end{align}
	积分只有在$m-k=\pm 1$时不为0,此时
	\begin{align}
		c_{m,m-1}=\frac{ID\mathscr E}{(-2m+1)\hbar^2},\quad c_{m,m+1}=\frac{ID\mathscr E}{(2m+1)\hbar^2}
	\end{align}
	故
	\begin{align}
		\psi_m^{(1)}=\frac{ID\mathscr E}{\hbar^2}\fkh{\frac1{2m+1}\psi_{m+1}^{(0)}-\frac1{2m-1}\psi_{m-1}^{(0)}}.
	\end{align}
	\subsection*{(c)}
	\Schr 方程变为 
	\begin{align}\notag
		-\frac{\hbar^2}{2I}\pv[2]\psi\vf-D\mathscr E\kh{1-\frac{\vf^2}2}\psi&=E\psi.\\
		-\frac{\hbar^2}{2I}\pv[2]\psi\vf+\frac12D\mathscr E\vf^2\psi&=(E+D\mathscr E)\psi.
	\end{align}
	是谐振子的形式,故振动能级
	\begin{align}
		E_n=\kh{n+\frac12}\hbar\sqrt{\frac{D\mathscr E}I}-D\mathscr E,\quad n=0,1,2,\ldots
	\end{align}
	本征函数 
	\begin{align}
		\psi_n(\vf)=\sqrt{\frac\alpha{\sqrt\pi 2^nn!}}H_n(\alpha\vf)\e{-\alpha^2\vf^2/2},\quad\alpha=\sqrt[4]{\frac{ID\mathscr E}{\hbar^2}}.
	\end{align}
	\section{补充题}
	\subsection*{(1)}
	氢原子1s基态归一化波函数
	\begin{align}
		\psi_{100}(r,\theta,\phi)=\frac1{\sqrt{\pi a_0^3}}\e{-r/a_0}.
	\end{align}
	故,只需将Bohr半径换为
	\begin{align}
		a=\frac{\hbar^2}{\mu e^2}=\frac{2\hbar^2}{m_\elc e^2}=2a_0.
	\end{align}
	即得电子偶素的波函数
	\begin{align}
		\psi_{100}(r)=\frac1{2\sqrt{2\pi a_0^3}}\e{-r/2a_0}.
	\end{align}
	\subsection*{(2)}
	取$a_0=1$,则 
	\begin{align}
		\avg{r^2}=4\pi\int\zti r^2\abs{\psi_{100}}^2r^2\d r=\frac12\int\zti\e{-r}r^4\d r=12.
	\end{align}
	故均方根值
	\begin{align}
		\sqrt{\avg{r^2}}=2\sqrt3.
	\end{align}
	这应该是电子偶素半径或直径的物理估计。
	\subsection*{(3)}
	超精细相互作用
	\begin{align}
		\hat H'=-\frac{8\pi}3\qo{\mu_\mathrm p}\cdot\,\qo{\mu_\elc}\vd(\bm r)=-\frac{8\pi}3\kh{\frac e{m_\elc c}}^2\hat S_\mathrm p\cdot\hat S_\elc\vd(\bm r).
	\end{align}
	总角动量$\hat J=\hat S_\mathrm p+\hat S_\elc$,则
	\begin{align}
		\hat H'=-\frac{4\pi}3\kh{\frac e{m_\elc c}}^2\kh{\hat J^2-\hat S_\mathrm p^2-\hat S_\elc^2}\vd(\bm r).
	\end{align}
	1s态中,$S_\mathrm p=S_\elc=1/2$,单态$\chi_{00}$下,
	\begin{align}
		\inp{\psi_{100}\chi_{00}}{\hat H'\psi_{100}\chi_{00}}&=-\frac1{8\pi a_0^3}\frac{4\pi}3\kh{\frac e{m_\elc c}}^2\kh{0-\frac34-\frac34}\hbar^2
	\end{align}
	三重态$\chi_1$下,
	\begin{align}
		\inp{\psi_{100}\chi_1}{\hat H'\psi_{100}\chi_1}=-\frac1{8\pi a_0^3}\frac{4\pi}3\kh{\frac e{m_\elc c}}^2\kh{2-\frac34-\frac34}\hbar^2.
	\end{align}
	故单态能级最低,能级差
	\begin{align}
		\D E=\frac1{3a_0^3}\kh{\frac{e\hbar}{m_\elc c}}^2.
	\end{align}
	能移
	\begin{align}
		\D\nu=\frac{\D E}h=\SI{1.16}\GHz.
	\end{align}
	\clearpage
	\part{笔记}
	\section{微扰理论}
可以精确求解的量子力学问题很少,在处理各种实际问题时,除了采用适当的模型以简化问题外,往往还需要采用合适的近似解法。
\subsection{非简并定态微扰理论}
当$\hat H$比较复杂时,定态\Schr 方程不能精确求解,若$\hat H$具有以下形式
\[\hat H=\hat H_0+\hat H'\]
其中$\hat H_0$是可解的,$\hat H'\ll\hat H_0$是小的修正,可令
\begin{align*}
	E_n&=E_n^{(0)}+E_n^{(1)}+E_n^{(2)}+\cdots\\
	\psi_n&=\psi_n^{(0)}+\psi_n^{(1)}+\psi_n^{(2)}+\cdots
\end{align*}
其中$E_n^{(k)}$和$\psi_n^{(k)}$与$\hat H'$的$k$次方成正比。代入原方程有
\[(\hat H_0+\hat H')(\psi_n^{(0)}+\psi_n^{(1)}+\cdots)=(E_n^{(0)}+E_n^{(1)}+\cdots)(\psi_n^{(0)}+\psi_n^{(1)}+\cdots)\]
逐阶比较得到
\begin{align}
	&\text{零级方程:}&(\hat H_0-E_n^{(0)})\psi_n^{(0)}&=0;\\
	&\text{一级方程:}&(\hat H_0-E_n^{(0)})\psi_n^{(1)}&=-(\hat H'-E_n^{(1)})\psi_n^{(0)};\\
	&\text{二级方程:}&(\hat H_0-E_n^{(0)})\psi_n^{(2)}&=-(\hat H'-E_n^{(1)})\psi_n^{(1)}+E_n^{(2)}\psi_n^{(0)};\\\notag
	&\text{……}
\end{align}
一般说来,越高次的项越小,所以可以只保留最低的几阶,便有足够的精度。

以下约定:波函数的各级高级近似解与零级近似解都正交,即
\[\inp{\psi_n^{(0)}}{\psi_n^{(k)}}=0.\]
对于非简并情形,即$\hat H_0$属于$E_n^{(0)}$的本征态只有一个
\[\psi_n^{(1)}=\sum_ma_{nm}^{(1)}\psi_m^{(0)}\]
代入一级方程有
\[\sum_ma_{nm}^{(1)}(\hat H_0-E_n^{(0)})\psi_m^{(0)}=-(\hat H'-E_n^{(1)})\psi_n^{(0)}\]
等式的两端与$\psi_k^{(0)}$内积
\[a_{nk}^{(1)}(E_k^{(0)}-E_n^{(0)})=-\inp{\psi_k^{(0)}}{\hat H'\psi_n^{(0)}}+E_n^{(1)}\vd_{kn}\]
取$k=n$得到一级微扰能
\begin{align}
	E_n^{(1)}=\inp{\psi_n^{(0)}}{\hat H'\psi_n^{(0)}}=:H_{nn}'
\end{align}
若取$k\neq n$,得到
\begin{align}
	a_{nk}^{(1)}=\frac{H_{kn}'}{E_k^{(0)}-E_n^{(0)}},\quad H_{kn}':=\inp{\psi_k^{(0)}}{\hat H'\psi_n^{(0)}}.
\end{align}
故一阶微扰波函数
\begin{align}
	\psi_n^{(1)}=\sum_{m\neq n}\frac{H_{mn}'}{E_m^{(0)}-E_n^{(0)}}\psi_m^{(0)}
\end{align}
微扰适用条件为
\begin{align}
	\abs{\frac{H_{mn}'}{E_m^{(0)}-E_n^{(0)}}}\ll 1.
\end{align}
\paragraph*{二阶微扰能}二级微扰方程
\[(\hat H_0-E_n^{(0)})\psi_n^{(2)}=-(\hat H'-E_n^{(1)})\sum_{m\neq n}\frac{H_{kn}'}{E_k^{(0)}-E_n^{(0)}}\psi_m^{(0)}+E_n^{(2)}\psi_n^{(0)}\]
两端与$\psi_n^{(0)}$内积,得到
\begin{gather}\notag
	0=-\sum_{m\neq n}\frac{H_{mn}'}{E_k^{(0)}-E_n^{(0)}}\inp{\psi_n^{(0)}}{\hat H'\psi_m^{(0)}}+0+E_n^{(2)},\\
	\thus E_n^{(2)}=\sum_{m\neq n}\frac{\abs{H_{mn}'}^2}{E_n^{(0)}-E_m^{(0)}}.
\end{gather}

准确到二级近似下,能量的本征值为
\begin{align}
	E_n=E_n^{(0)}+H_{nn}'+\sum_{m\neq n}\frac{\abs{H_{mn}'}^2}{E_n^{(0)}-E_m^{(0)}}
\end{align}
上述理论成立需要$\hat H_0$为分离谱,无简并。
\begin{example}{电介质的极化率}{}
	各向同性的非极性分子电介质在外电场作用下极化,求感生电偶极矩。

	考虑正离子运动,无外场时为简谐运动
	\begin{gather*}
		\hat H_0=-\frac{\hbar^2}{2\mu}\dd[2]x+\frac12\mu\omega^2x^2,\\
		\hat H_n\ket n=E_n^{(0)}\ket n,\quad E_n^{(0)}=\kh{n+\frac12}\hbar\omega.
	\end{gather*}
	沿$x$向加恒定电场相当于施加微扰$\hat H'=-q\varepsilon x$
	\begin{align*}
		H_{nk}'&=\bra n\hat H'\ket k=-q\varepsilon\bra nx\ket k\\
		&=-q\varepsilon\sqrt{\frac\hbar{\mu\omega}}\kh{\sqrt{\frac{k+1}2}\vd_{n,k+1}+\sqrt{\frac k2}\vd_{n,k-1}}
	\end{align*}
	一级能量修正$H_{kk}'=0$,二阶近似能量
	\begin{align*}
		E_k&=E_k^{(0)}+0+\sum_{n\neq k}\frac{\abs{H_{nk}'}^2}{E_k^{(0)}-E_n^{(0)}}=E_k^{(0)}-\frac{q^2\varepsilon^2}{2\mu\omega^2}.
	\end{align*}
	实际上,能量是有精确解的
	\[V=\frac12\mu\omega^2x^2-q\varepsilon x=\frac12\mu\omega^2\kh{x-\frac{q\varepsilon}{\mu\omega^2}}^2-\frac{q^2\varepsilon^2}{2\mu\omega^2}.\]
	它的第一项只不过是把原来的谐振子势能平移了一段距离,这个移动不会影响谐振子的能级,而它的第二项正是前面求出的常数项。

	一级近似态
	\begin{align*}
		\ket{\psi_k}&=\ket k+\sum_{n\neq k}\frac{H_{nk}'}{E_k^{(0)}-E_n^{(0)}}\ket n\\
		&=\ket k+\frac{q\varepsilon}{\sqrt{\hbar\mu\omega^3}}\kh{\sqrt{\frac{k+1}2}\ket{k+1}-\sqrt{\frac k2}\ket{k-1}},
	\end{align*}
	无外加场时,非极性分子正(负)离子的位置平均值$\bra kx\ket k=0$,即固有电偶极矩为零,而加外电场后正离子位移
	\[\bra{\psi_k}x\ket{\psi_k}=\frac{2q\varepsilon}{\sqrt{\hbar\mu\omega^3}}\kh{\sqrt{\frac{k+1}2}\brkt k{k+1}-\sqrt{\frac k2}\brkt k{k-1}}=\frac{q\varepsilon}{\mu\omega^2}.\]
	故感生电偶极矩$D$和极化率$\kappa$
	\[D=\abs q\frac{2\abs q\varepsilon}{\mu\omega^2}=\frac{2q^2\varepsilon}{\mu\omega^2},\quad \kappa=\frac D\varepsilon=\frac{2q^2}{\mu\omega^2}.\]
\end{example}
\begin{example}{氦原子}{}
	氦原子在原子核外有两个电子,Hamilton量包括两个电子在原子核的Column引力场中的运动
	\[\hat H_0=\kh{-\frac12\nabla_1^2-\frac Z{r_1}}+\kh{-\frac12\nabla_1^2-\frac Z{r_2}};\]
	以及两个电子之间的Column排斥能(微扰项)
	\[\hat H'=\frac1{r_{12}}.\]

	它对于两个电子空间坐标的交换是对称的。由于电子是Fermi子,两个电子相应的自旋态只能反对称的自旋单态$\chi_{00}$
	\[\psi(\r_1,\r_2)=\psi_{100}(\r_1)\psi_{100}(\r_2)\chi_{00}(s_{1z},s_{2z}).\]
	相应的本征值$E_1^0=-Z^2$,一级修正
	\[\ave{\frac1{r_{12}}}=\int\frac{\abs{\psi_{100}(\r_1)\psi_{100}(\r_2)}^2}{r_{12}}\d\r_1\nd\r_2.\]
	由
	\[\psi_{100}(\r)=\frac{Z^{3/2}}{\sqrt\pi}\e{-Zr};\quad \int\frac{\e{-Z(r_1+r_2)}}{r_{12}}\d\r_1\nd\r_2=\frac{5\pi^2}{8Z^5}.\]
	得
	\[E=-Z^2+\frac58Z.\]
\end{example}
\subsection{简并定态微扰理论}
实际问题中,特别是处理体系的激发态时,常常碰到简并态或近似简并态。此时,非简并态微扰论是不适用的。

这里首先碰到的困难是:零级能量给定后,对应的零级波函数并未确定,这是简并态微扰论首先要解决的问题。

体系能级的简并性与体系的对称性密切相关。当考虑微扰之后,如体系的某种对称性受到破坏,则能级可能分裂,简并将部分或全部解除。因而在简并态微扰中,充分考虑体系的对称性及其破缺是至关重要的。
\paragraph*{一级微扰能和零级波函数}$E_n^{(0)}$简并时,
\[\hat H^{(0)}\psi_{ni}^{(0)}=E_n^{(0)}\psi_{ni}^{(0)},\quad (i=1,2,\ldots,k).\]
简并度$f_n=k$,引入微扰后假设
\[\psi_n^{(0)}=\sum_{i=1}^kc_i^{(0)}\psi_{ni}^{(0)}.\]
代入一级微扰方程
\[(\hat H_0-E_n^{(0)})\psi_n^{(1)}=-(\hat H'-E_n^{(1)})\sum_{i=1}^kc_i^{(0)}\psi_{ni}^{(0)}.\]
两端与$\psi_{nj}^{(0)}$内积
\[0=-\sum_{k=1}^nc_i^{(0)}\fkh{\inp{\psi_{nj}^{(0)}}{\hat H'\psi_{ni}^{(0)}}-E_n^{(1)}\vd_{ij}}.\]
记
\begin{align}
	H_{ji}':=\inp{\psi_{nj}^{(0)}}{\hat H'\psi_{ni}^{(0)}}
\end{align}
得到久期(secular)方程
\begin{align}
	\det(H'-E_n^{(1)}I)=\begin{vmatrix}
		H_{11}'-E_n^{(1)}&H_{12}'&\cdots&H_{1k}'\\
		H_{21}'&H_{22}'-E_n^{(1)}&\cdots&H_{2k}'\\
		\vdots&\vdots&\ddots&\vdots\\
		H_{k1}'&H_{k2}'&\cdots&H_{kk}'-E_n^{(1)}
	\end{vmatrix}=0.
\end{align}
从中可以解出$E_n^{(1)}$以及它们对应的$c_i^{(0)}$,这就决定了一级微扰能和零级波函数。
\paragraph*{Stark效应}Stark效应就是原子或分子在外电场作用下能级和光谱发生分裂的现象。
\end{document}