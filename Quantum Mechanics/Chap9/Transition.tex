\def\coursename{跃迁理论}
\def\courseEnglishname{Quantum Mechanics}
\def\teachername{陈新、郭永}
\def\beginday{2022/3/25}
\def\endday{2022/3/26}

\documentclass[a4paper, 11pt]{article}

\usepackage[UTF8]{ctex}

\usepackage[T1]{fontenc}								% 字体
\catcode`\。=\active
\newcommand{。}{.} % {\ifmmode\text{.}\else .\fi}
\catcode`\(=\active
\catcode`\)=\active
\newcommand{(}{(}
\newcommand{)}{)}

% \usepackage{zhlineskip}

\usepackage{nicematrix}
% \usepackage{setspace}
% \linespread{1}						% 一倍行距
\setlength{\headheight}{14pt}			% 页眉高度
% \setlength{\lineskip}{0ex}			% 行距
\renewcommand\arraystretch{.82}		% 表格

\usepackage{amssymb, amsmath, amsfonts, amsthm}			% 数学符号,公式,字体,定理环境
\everymath{\displaystyle}			% \textstyle \scriptstyle \scriptscriptstyle
\allowdisplaybreaks[4]      		% 使用行间公式格式
% \makeatletter
% \renewcommand{\maketag@@@}[1]{\hbox{\m@th\normalsize\normalfont#1}}
% \makeatother
\newif\ifcontent\contenttrue		% if 显示目录
\newif\ifparskip\parskipfalse		% if 增加目录后的行距
\newif\ifshowemail\showemailfalse	% if 显示 email
\def\firstandforemost{
	\maketitle
	%\thispagestyle{empty}\clearpage
	\ifcontent
		\renewcommand{\contentsname}{目录}
		\tableofcontents
		\thispagestyle{empty}
		\clearpage
	\fi
	\ifparskip
		\setlength{\parskip}{.8ex}	% 设置额外的段距,目录后
	\fi								% 在 \firstandforemost 前设置 \parskiptrue
	\makenomenclature
	\printnomenclature
	\setcounter{page}{1}
}

\usepackage{mathtools}									% \rcase 环境等

% \usepackage{physics}

\usepackage[]{siunitx}									% 国际制单位
\sisetup{
	inter-unit-product = \ensuremath{{}\cdot{}},
	per-mode = symbol,
	per-mode = reciprocal-positive-first,
	range-units = single,
	separate-uncertainty = true,
	range-phrase = \ifmmode\text{\;-\;}\else\;-\;\fi
}
\DeclareSIUnit\angstrom{\text{Å}}
\DeclareSIUnit\atm{\text{atm}}
% SIunits 额外定义了一个 \square 表示平方,
% 还会把 \cdot 空格加大,真有够无语的 😅

\usepackage{authblk}									% 作者介绍
\ifx \coursefullname\undefined
	\ifx \coursename\undefined
		\def\coursename{笔记}
	\fi
	\def\coursefullname{\coursename}
\fi
\ifx \authorname\undefined
	\def\authorname{Dait}
\fi
\ifx \departmentname\undefined
	\def\departmentname{DEP 00, THU}
\fi
\ifx \emailaddress\undefined
	\def\emailaddress{daiyj20@mails.tsinghua.edu.cn}
\fi
\ifx \beginday\undefined
	\def\beginday{2021}
\fi
\ifx \endday\undefined
	\def\endday{\number\year/\number\month/\number\day}
\fi
\ifx \titleannotation\undefined
	\ifx \teachername\undefined
		\title{\textbf{\coursefullname}}
	\else
		\title{\textbf{\coursefullname}\\\small\textit{主要整理自\teachername 老师讲义}}
	\fi
\else
	\title{\textbf{\coursefullname}\\\small\textit{(\titleannotation)}}
\fi
\newif\ifdefaultauthor\defaultauthortrue
\ifdefaultauthor
	\author{by~\authorname~at~\departmentname}
	\ifshowemail
		\affil{\emailaddress}
	\fi
\fi
\ifx \endday\beginday
	\date{\beginday}
\else
	\date{\beginday~-~\endday}
\fi

\usepackage{hyperref}									% 链接
\ifx \courseEnglishname\undefined
	\def\courseEnglishname{Note}
\fi
\ifx \authorEnglishname\undefined
	\def\authorEnglishname{Dait}
\fi
\hypersetup{
	% dvipdfm								% 表示用 dvipdfm 生成 pdf
	pdftitle={\coursename},
	pdfauthor={\authorname},
	colorlinks=true, breaklinks=true,		% 超链接设置
	linkcolor=black, citecolor=black, urlcolor=blue
}

\usepackage[british]{babel}								% 长单词自动连字符换行
\hyphenation{long-sen-ten-ce}				% 自定义拆分方式

\usepackage{tikz}
\usetikzlibrary{quotes, angles}
\usepackage{pgfplots}
\pgfplotsset{compat=1.17}								% TikZ
\newcommand{\coor}[5][0]{
	\draw[thick,latex-latex](#1,#3)node[left]{$#5$}--(#1,0)node[shift={(-135:7pt)}]{$O$}--(#1+#2,0)node[right]{$#4$}
}			% 坐标轴

\usepackage{enumerate}									% 编号
\usepackage{paralist}
\setlength{\pltopsep}{1ex}
\setlength{\plitemsep}{1ex}
\ifx \eqnrange\undefined
	\numberwithin{equation}{section}
\else
	\numberwithin{equation}{\eqnrange}
\fi

\renewcommand{\thempfootnote}{\Roman{mpfootnote}}
\renewcommand{\thefootnote}{\Roman{footnote}}		% 注释上标 I, II,...
\newcommand{\sectionstar}[1]{
	\section[\hspace{-.8em}*\hspace{.3em}#1]{\hspace{-1em}*\hspace{.5em}#1}
}
\newcommand{\subsectionstar}[1]{					% 带星号的 section 和 subsection
	\subsection{\hspace{-1em}*\hspace{.5em}#1}
}
\newcommand{\subsubsectionstar}[1]{					% 带星号的 section 和 subsection
	\subsubsection{\hspace{-1em}*\hspace{.5em}#1}
}
\newcommand*{\appendiks}{
	\appendix
	\part*{附录}
	\addcontentsline{toc}{part}{附录}
}
\iffalse			% 不清楚
	\newcommand{\varsection}[1]{
		\refstepcounter{section}
		\section*{\thesection\quad #1}
		\addcontentsline{toc}{section}{\makebox[0pt][r]{*}\thesection\quad #1}
	}
\fi

\usepackage{fancyhdr}									% 页眉页脚
\ifx \coursename\undefined
	\def\coursename{笔记}
\fi
\fancyhf{}\pagestyle{fancy}
\fancyhead[L]{\coursename\rightmark}
\fancyhead[R]{by~\authorname}
\fancyfoot[C]{-~\thepage~-} 			%页码

\usepackage{colortbl, booktabs}							% 表

\usepackage{graphicx}
\usepackage{float}
\usepackage{caption}									% 图
\captionsetup{
	margin=20pt, format=hang,
	justification=justified
}
\newcounter{tikzpic}
\def\tikzchap{
	\stepcounter{tikzpic}\\
	\small 图~\thetikzpic\quad
}
\newcounter{linetable}
\newcommand{\tablechap}[1]{
	\stepcounter{linetable}
	{\small 表~\thelinetable\quad #1}\\[1em]
}

\usepackage{tcolorbox}									% 盒子
\tcbuselibrary{theorems, skins, breakable}
\definecolor{MatchaGreen}{HTML}{73C088}		% 抹茶绿B7C6B3
\newtcbtheorem[number within = subsection]{example}{例}{
	enhanced, breakable, sharp corners,
	attach boxed title to top left = {yshifttext = -1mm},
	before skip = 2ex,
	colback = MatchaGreen!5,				% 文本框内的底色
	colframe = MatchaGreen,					% 文本框框沿的颜色
	fonttitle = \bfseries,					% 标题字体用粗体	coltitle 默认 white,
	boxed title style = {
			sharp corners, size = small, colback = MatchaGreen,
		}
}{exm}
\definecolor{MelancholyBlue}{HTML}{9EAABA}	% melancholy: 沮丧
\newcounter{pslt}
\setcounter{pslt}{-1}
\newtcbtheorem[use counter = pslt]{posulate}{假设}{
	enhanced, breakable, sharp corners,
	attach boxed title to top left = {yshifttext = -1mm}, before skip = 2ex,
	colback = MelancholyBlue!5, colframe = MelancholyBlue, fonttitle = \bfseries,
	boxed title style = {
			sharp corners, size = small, colback = MelancholyBlue,
		}
}{psl}
\definecolor{PureBlue}{HTML}{80A3D0}
\newtcbtheorem[number within = subsection]{definition}{定义}{
	enhanced, breakable, sharp corners,
	attach boxed title to top left = {yshifttext = -1mm}, before skip = 2ex,
	colback = PureBlue!5, colframe = PureBlue, fonttitle = \bfseries,
	boxed title style = {
			sharp corners, size = small, colback = PureBlue,
		}
}{dfn}
\definecolor{PeachRed}{HTML}{EA868F}
\newtcbtheorem[number within = subsection]{theorem}{定理}{
	enhanced, breakable, sharp corners,
	attach boxed title to top left = {yshifttext = -1mm}, before skip = 2ex,
	colback = PeachRed!5, colframe = PeachRed, fonttitle = \bfseries,
	boxed title style = {
			sharp corners, size = small, colback = PeachRed,
		}
}{thm}
\definecolor{SchembriumYellow}{HTML}{fbd26a}	% 申博太阳城黄
\newtcbtheorem[number within = section]{method}{方法}{
	enhanced, breakable, sharp corners,
	attach boxed title to top left = {yshifttext = -1mm}, before skip = 2ex,
	colback = SchembriumYellow!5, colframe = SchembriumYellow, fonttitle = \bfseries,
	boxed title style = {
			sharp corners, size = small, colback = SchembriumYellow,
		}
}{mtd}
% 保留颜色
\definecolor{fadedgold}{HTML}{D9CBB0}		% 褪色金
\definecolor{saturatedgold}{HTML}{F0E0C2}	% staurated: 饱和
\definecolor{elegantblue}{HTML}{C4CCD7}		% elegant: 优雅
\definecolor{ivory}{HTML}{F1ECE6}			% 象牙
\definecolor{gloomypruple}{HTML}{CCC1D2}	% 阴沉紫
% \textcolor[HTML]{FFC23A}					% 石板灰

\definecolor{Green}{rgb}{0,.8,0}

\usepackage{imakeidx}								% 索引

\usepackage{nomencl}								% 关键词
%\setlength{\nomitemsep}{0.2cm}							% 设置术语之间的间距
\renewcommand{\nomentryend}{.}							% 设置打印出术语的结尾的字符
\renewcommand{\eqdeclaration}[1]{见公式:(#1)}			% 设置打印见公式的样式
\renewcommand{\pagedeclaration}[1]{见第 (#1) 页}		% 设置打印页的样式
\renewcommand{\nomname}{术语表} 						% 修改术语表标题的名称。

\usepackage{array}
\usepackage{booktabs} % 三线表
\usepackage{multirow}
% 手动排版,尽量杜绝使用

\newcommand{\bs}[1]{\hspace{-#1 pt}}		% 手动减间距	backspace
\newcommand{\bv}[1]{\vspace{-#1 pt}}		% 手动缩行距	backvspace
\def\directlisteqn{\vspace{-1ex}}
\newcommand*{\qqquad}{\qquad\quad}
\newcommand*{\qqqquad}{\qquad\qquad}
\iffalse									% 尽量避免孤行
	\widowpenalty=4000
	\clubpenalty=4000
\fi

% 杂项符号
\let\geq\geqslant
\def\avg{\overline}
\let\ifaoif\iff
\let\iff\relax
\newcommand*{\rqed}{\tag*{$\square$}}								% 靠右 QED
\newcommand*{\halfqed}{\tag*{$\boxdot$}}
\newcommand*{\thus}{\quad\Rightarrow\quad}							% =>
\newcommand*{\iff}{\enspace\Leftrightarrow\enspace}						% <=>	if and only if
\newcommand*{\ifnf}{\quad\Leftrightarrow\quad}						% <=>	if and only if
\newcommand*{\turnto}{\quad\to\quad}
\newcommand*{\normalize}{\quad\overset{\mathrm{normalize}}{-\!\!\!-\!\!\!-\!-\!\!\!\longrightarrow}\quad}
\newcommand*{\vthus}{\\$\Downarrow$\\}
\newcommand*{\viff}{\\$\Updownarrow$\\}
\newcommand*{\vs}{~\text{-}~}
\newcommand{\eg}[1][]{\subparagraph*{例#1:}}
\newcommand*{\prf}{\noindent\textbf{证明:}\quad}
\newcommand{\dpfr}[2]{\displaystyle\frac{#1}{#2}}					% 大分数
\newcommand{\frdp}[2]{\frac{\displaystyle #1}{\displaystyle #2}}
\newcommand{\spark}[1]{\;\textcolor{red}{#1}}
\newcommand*{\verylongrightarrow}{ -\bs4-\bs4-\bs4-\bs4-\bs5\longrightarrow}
\newcommand*{\semilongrightarrow}{-\bs4-\bs5\longrightarrow}

% 简化更常用的希腊字母
\newcommand*{\vf}{\varphi}
\newcommand*{\vF}{\varPhi}
\newcommand*{\vp}{\varPsi}
\newcommand*{\ve}{\varepsilon}
\newcommand*{\vC}{\varTheta}
\newcommand*{\ct}{\theta}			% 还是建议用 @ + Tab 快捷键

% 正体符号
\newcommand*{\cns}{\mathrm{const}}
\newcommand*{\plusc}{{\color{lightgray}\,+\,\cns}}
\newcommand*{\e}{\mathop{}\!\mathrm{e}^}	% e
\let\accenti\i
\renewcommand*{\i}{\mathrm{i}}
\newcommand*{\D}{\Delta}
\newcommand*{\p}{\partial}

\usepackage{bm}											% 粗体 \bm
\newcommand{\hbm}[1][r]{\hat{\bm #1}}	% 应该不会有两个字母的
\newcommand{\ibm}[1]{\,\bm #1}
\newcommand{\uvec}[1]{\mathop{}\!\hat{\bm #1}}

% Using EnglischeSchT script font style
%\newfontfamily{\calti}{EnglischeSchT}
%\newcommand{\mathcalti}[1]{\mbox{\calti{#1}}}
%\newcommand{\mathcaltibf}[1]{\mbox{\bf\calti{#1}}}

\usepackage{mathrsfs}									% 花体 \mathscr
% \usepackage{boondox-cal}								% 小写花体 \mathcal
\newcommand*{\RR}{\mathbb R}
\newcommand*{\CC}{\mathbb C}
\newcommand*{\ZZ}{\mathbb Z}
\newcommand*{\NN}{\mathbb N}
\newcommand*{\sC}{\mathscr C}			% n 阶连续可导函数
\newcommand*{\sR}{\mathscr R}			% 黎曼可积
% 算符用 \mathcal
\newcommand*{\cL}{\mathcal L}			% 表示一般算子
\newcommand{\cl}[1]{\mathcal L\fkh{#1}}
\newcommand{\cli}[1]{\mathcal L^{-1}\!\fkh{#1}}
\newcommand{\cf}[2][\!\,]{\mathcal F_\mathrm{#1}\fkh{#2}}
\newcommand{\cfi}[2][\!\,]{\mathcal F_\mathrm{#1}^{-1}\!\fkh{#2}}
% \newcommand{\cl}[2][0]{\mathcal L\ikh[#1]{#2}}
% \newcommand{\cli}[2][0]{\mathcal L^{-1}\ikh[#1]{#2}}
% \newcommand{\cf}[2][0]{\mathcal F\ikh[#1]{#2}}
% \newcommand{\cfi}[2][0]{\mathcal F^{-1}\ikh[#1]{#2}}

\usepackage{cancel}										% 删除线

\usepackage{xfrac}

% \usepackage{emoji}	需要 LuaTeX

% 导数等
\let\divides\div
\renewcommand*{\div}{\nabla\cdot}
\newcommand*{\curl}{\nabla\times}
\newcommand*{\lap}{\Delta}
\let\accentd\d
\renewcommand*{\d}{\mathop{}\!\mathrm{d}}
\newcommand*{\nd}{\mathrm{d}}
\newcommand*{\vd}{\mathop{}\!\delta}											% δ
\newcommand{\dd}[2][\;\!\!]{\frac{\nd^{#1}}{\nd #2^{#1}}}						% d/dx			我知道 \,\! 很愚蠢,但是 {} 无法在 Math Preview 上预览
\newcommand{\dn}[2]{\frac{\nd^{#1}}{\nd #2^{#1}}}								% d^n/dx^n		\dn2x≡\dd[2]x
\newcommand{\dv}[3][\;\!\!]{\frac{\nd^{#1}#2}{\nd #3^{#1}}}						% df/dx
\newcommand{\du}[3]{\frac{\nd^{#1}#2}{\nd #3^{#1}}}								% d^nf/dx^n		\du2fx≡\dv[2]fx
\newcommand{\pp}[2][\;\!\!]{\frac{\p^{#1}}{\p #2^{#1}}}							% ∂/∂x
\newcommand{\pn}[2]{\frac{\p^{#1}}{\p #2^{#1}}}									% ∂^n/∂x^n		\pn2x≡\pp[2]x
\newcommand{\pv}[3][\;\!\!]{\frac{\p^{#1}#2}{\p #3^{#1}}}						% ∂f/∂x
\newcommand{\pu}[3]{\frac{\p^{#1}#2}{\p #3^{#1}}}								% ∂^nf/∂x^n		\pu2x≡\pv[2]x
\newcommand{\pw}[3]{\frac{\p^2 #1}{\p #2\p #3}}									% ∂^2f/∂x∂y
\newcommand{\pvv}[6]{															% ∂^(m+n)f/∂x^m∂y^n
	\ifnum#4=1
		\ifnum#6=1
			\frac{\p^{#1}#2}{\p #3\p #5}
		\else
			\frac{\p^{#1}#2}{\p #3\p #5^{#6}}
		\fi
	\else
		\ifnum#6=1
			\frac{\p^{#1}#2}{\p #3^{#4}\p #5}
		\else
			\frac{\p^{#1}#2}{\p #3^{#4}\p #5^{#6}}
		\fi
	\fi}
\newcommand{\dvd}[2]{\left.#1\middle\slash #2\right.}							% 斜除

% 积分
\newcommand*{\intt}{\bs2\int\bs8\int}											% ∫∫
\newcommand*{\inttt}{\int\bs8\int\bs8\int}										% ∫∫∫
\newcommand*{\intdt}{\int\bs3\cdot\bs2\cdot\bs2\cdot\bs4\int}					% ∫...∫
\newcommand*{\zti}{_0^{+\infty}}												% _0^+∞
\newcommand*{\iti}{_{-\infty}^{+\infty}}										% _-∞^+∞
\newcommand{\fmto}[3][\infty]{_{#2=#3}^{#1}}

% 括号
\newcommand{\abs}[1]{\left\lvert#1\right\rvert}									% |x| 绝对值
\newcommand{\norm}[1]{\left\lVert#1\right\rVert}								% ||x|| 模
\newcommand{\edg}[1]{\left.#1\right\rvert}										% f|  竖线
\newcommand{\kh}[1]{\left(#1\right)}											% (x) 括号
\newcommand{\bigkh}[1]{\bigl(#1\bigr)}
\newcommand{\Bigkh}[1]{\Bigl(#1\Bigr)}
\newcommand{\biggkh}[1]{\biggl(#1\biggr)}
\newcommand{\fkh}[1]{\left[#1\right]}											% [x] 方括号
\newcommand{\bigfkh}[1]{\bigl[#1\bigr]}
\newcommand{\Bigfkh}[1]{\Bigl[#1\Bigr]}
\newcommand{\biggfkh}[1]{\biggl[#1\biggr]}
\newcommand{\hkh}[1]{\left\{#1\right\}}											% {x} 花括号
\newcommand{\zkh}[1]{\lfloor\bs{4.7}\lceil #1\rceil\bs{4.7}\rfloor}				% [x] 中括号
\newcommand{\floor}[1]{\left\lfloor#1\right\rfloor}
\newcommand{\ceil}[1]{\left\lceil#1\right\rceil}
\newcommand{\set}[2]{\left\{#1\,\middle\vert\,#2\right\}}						% {x|x1,x2,...} 集合
\newcommand{\ave}[1]{\left\langle #1\right\rangle}								% <x> 平均值
\newcommand{\bra}[1]{\left\langle #1\right\vert}								% <ψ| 左矢
\newcommand{\ket}[1]{\left\vert #1\right\rangle}								% |ψ> 右矢
\newcommand{\brkt}[2]{\left\langle #1\middle\vert #2\right\rangle}				% <φ|ψ> 内积
\newcommand{\ktbr}[2]{\left\vert#1\right\rangle \bs3\left\langle #2\right\vert}	% |ψ><φ|
\newcommand{\inp}[2]{\left\langle #1,#2\right\rangle}							% <f,g> 内积

% 数学运算符
\let\Real\Re
\let\Imagine\Im
\let\Re\relax
\let\Im\relax
\DeclareMathOperator{\Re}{Re}					% 
\DeclareMathOperator{\Im}{Im}					% 
\DeclareMathOperator{\sech}{sech}				% 
\DeclareMathOperator{\csch}{csch}				% 
\DeclareMathOperator{\arcsec}{arcsec}			% 
\DeclareMathOperator{\arccot}{arccot}			% 
\DeclareMathOperator{\arccsc}{arccsc}			% 
\DeclareMathOperator{\arsinh}{arsinh}			% 
\DeclareMathOperator{\arcosh}{arcosh}			% 
\DeclareMathOperator{\artanh}{artanh}			% 
\DeclareMathOperator{\sgn}{sgn}					% 符号函数
\DeclareMathOperator{\Li}{Li}					% 
\DeclareMathOperator{\Si}{Si}
\DeclareMathOperator{\Ci}{Ci}
\DeclareMathOperator{\sinc}{sinc}
\DeclareMathOperator{\Heaviside}{H}
\DeclareMathOperator{\arr}{A}					% 排列数
\DeclareMathOperator{\com}{C}					% 组合数
\DeclareMathOperator{\Res}{Res}					% 留数
\DeclareMathOperator{\supp}{supp}				% 支撑集
\DeclareMathOperator{\Int}{Int}					% 内部
\DeclareMathOperator{\Ext}{Ext}					% 外部
\newcommand*{\bigo}{\mathcal O}
\newcommand{\degree}{^\circ}

% 线性代数
% \newif\ifLinearAlgebra\LinearAlgebratrue
% \ifLinearAlgebra
\DeclareMathOperator{\rank}{rank}
\DeclareMathOperator{\id}{id}
\newcommand*{\tp}{^\top}					% AT 转置
\newcommand*{\cj}{^\ast}					% A* 共轭
\newcommand*{\dg}{^\dagger}					% A† 共轭转置
\newcommand*{\iv}{^{-1}}					% A-1
% \fi

% 物理学家
\newcommand*{\Schr}{Schrödinger}
\newcommand*{\Legd}{Legendre}
\newcommand*{\deB}{de Broglie}
\newcommand*{\Rayl}{Rayleigh}
\newcommand*{\Lande}{Landé}

% 粒子
\newcommand*{\elc}{\mathrm e}
\newcommand*{\pton}{\mathrm p}
\newcommand*{\nton}{\mathrm n}
\newcommand*{\mol}{\mathrm m}

% 物理常数
\newcommand*{\NA}{N_{\bs1\mathrm A}}						% Avogadro 常数
\newcommand*{\kB}{k_{\mathrm B}}							% Boltzmann 常数
\newcommand*{\muB}{\mu_\mathrm B}							% Bohr 磁矩

% 
\newcommand*{\Ek}{E_{\mathrm k}}							% 动能
\newcommand*{\eff}{_\mathrm{eff}}							% 有效下标
\newcommand*{\tot}{_\mathrm{tot}}
\newcommand*{\maxi}{_\mathrm{max}}
\newcommand*{\mini}{_\mathrm{min}}
\newcommand*{\lSI}{\tag{SI}}
\newcommand*{\CGS}{\tag{CGS}}								% cm, g, s 制
\newcommand*{\FWHM}{\mathrm{FWHM}}


\newenvironment{equationset}{\left\{\begin{aligned}}{\end{aligned}\right.}
\contentfalse

\let\oldr\r
\renewcommand*{\r}{\bm r}						% 坐标向量
\newcommand{\qo}[1]{\bm{\hat #1}\!\mathop{}}	% 量子力学中的向量算符
\newcommand{\cmm}[2]{\zkh{\hat #1,\hat #2}}		% 对易子
\newcommand{\acmm}[2]{\{\hat #1,\hat #2\}}

\newcommand{\Larmor}{\mathrm L}
\newcommand{\Hall}{\mathrm H}

\newcommand{\cyc}{\mathrm c}
\DeclareMathOperator{\Cle}{C}

\begin{document}

\firstandforemost

\subsection{线性谐振子}
势能
\[
	V(x)=\frac12m\omega^2x^2.
\]
其中$\omega$为谐振子固有圆频率.
\[
	\pu 2\psi{x}+\kh{\frac{2mE}{\hbar^2}-\frac{m^2\omega^2}{\hbar^2}x^2}\psi=0.
\]
做变换$\xi=\sqrt{\frac{m\omega}\hbar}x,~\lambda=\frac{2E}{\hbar\omega}$
\[
	\dv[2]\psi\xi+(\lambda-\xi^2)\psi=0.
\]
当$x\to\infty,$方程近似为
\[
	\dv[2]\psi\xi-\xi^2\psi=0,\thus\psi\sim\e{\pm\xi^2/2}.
\]
由束缚态要求,解应有形式$\psi(\xi)=\e{-\xi^2/2}H(\xi),$
\[
	\dv[2] H\xi-2\xi\dv H\xi+(\lambda-1)H=0.
\]
此即Hermite方程,用级数解得系数
\[
	c_{k+2}=\frac{2k+1-\lambda}{(k+1)(k+2)}c_k,
\]
注意到$k\to\infty$
\[
	\frac{c_{k+2}}{c_k}\to\frac{2}k,\quad H(\xi)\sim\sum_{i=i_0}^\infty\frac{\xi^{2i}}{i!}=\e{\xi^2}.
\]
仍使$\psi(\xi)$发散.除非$H(\xi)$的项有限$\lambda=2k+1$且只能出现奇或偶次幂.

故能量本征值
\begin{align}
	E_n=\kh{n+\frac12}\hbar\omega,\quad n=0,1,\ldots
\end{align}
基态能量不为0。

对应Hermite方程
\[
	H_n''-2\xi H_n'+2nH_n=0.
\]
解称为Hermite多项式\index{Hermite Polynomial}
\[
	H_n(x)=(-1)^n\e{x^2}\dd[n]x\e{-x^2}.
\]
\begin{example}{Hermite多项式}{Hermite Polynomial}
	前几项 %($\psi e^{\xi^2/2}$)
	\begin{align*}
		H_0 & =1,  & H_2 & =4x^2-2,   & H_4 & =16x^4-48x^2+12,    \\
		H_1 & =2x, & H_3 & =8x^3-12x, & H_5 & =32x^5-160x^3+120x.
	\end{align*}
	\iffalse
		\begin{align*}
			\psi_0 & \propto e^{-\xi^2/2}            & \psi_3 & \propto\frac1{\sqrt3}(2\xi^3-3\xi),
			\psi_1 & \propto\sqrt2\xi e^{-\xi^2/2}   & \psi_4 & \propto\frac1{2\sqrt6}(4\xi^4-12\xi^3+3),     \\
			\psi_2 & \propto\frac1{\sqrt2}(2\xi^2-1) & \psi_6 & \propto\frac1{2\sqrt{30}}(4\xi^5-20\xi^3+15).
		\end{align*}
	\fi
\end{example}
Hermite多项式的内积
\[
	\int\iti H_nH_{n'}\e{-x^2}\d x=2^nn!\,\delta_{nn'}.
\]
因此本征态
\begin{align}
	\psi_n=\sqrt[4]{\frac{m\omega}{\pi\hbar}}\frac1{\sqrt{2^nn!}}H_n(\xi)\e{-\xi^2/2},\quad\xi=\sqrt{\frac{m\omega}\hbar}x.
\end{align}
宇称为$\kh{-1}^n.$
\paragraph*{递推关系}
{\small\begin{align}
	x\ket n&=\frac1{\sqrt2\alpha}\fkh{\sqrt n\ket{n-1}+\sqrt{n+1}\ket{n+1}};\\
	x^2\ket n&%=\frac1{\alpha^2}\fkh{\sqrt{\frac n2}\kh{\sqrt{\frac{n-1}2}\ket{n-2}+\sqrt{\frac n2}\ket n}+\sqrt{\frac{n+1}2}\kh{\sqrt{\frac{n+1}2}\ket n+\sqrt{\frac{n+2}2}\ket{n+2}}}\\
	%&
	=\frac1{2\alpha^2}\fkh{\sqrt{n(n-1)}\ket{n-2}+(2n+1)\ket n+\sqrt{(n+1)(n+2)}\ket{n+2}}.
\end{align}}
\subsection{角动量算符}
顺承经典力学中的定义,角动量算符
\[
	\qo L:=\qo r\times\,\qo p.
\]
直角坐标表象下
\begin{align*}
	\qo L & =\zkh{\hat x,\hat y,\hat z}\tp\times\zkh{\hat p_x,\hat p_y,\hat p_z}\tp                             \\
	      & =\fkh{\hat y\hat p_z-\hat z\hat p_y,\hat z\hat p_x-\hat x\hat p_z,\hat x\hat p_y-\hat y\hat p_x}\tp % \\ &
	=:\zkh{\hat L_x,\hat L_y,\hat L_z}\tp.
\end{align*}
易证$\hat L_x,\hat L_y,\hat L_z$是Hermite的。对易子
\[
	\cmm{L_x}{L_y}=(\hat p_z\hat z-\hat z\hat p_z)(\hat y\hat p_x-\hat x\hat p_y)=\i\hbar\,\hat L_z.
\]
$\cmm{L_y}{L_z}$等同理,因此
\begin{align}
	\qo L\times\,\qo L=\i\hbar\qo L.
\end{align}

另一方面,角动量平方算符$\hat L^2\equiv\hat L_x^2+\hat L_y^2+\hat L_z^2$与$\hat L_z$等分量对易
\begin{align}\notag
	\cmm{L^2}{L_z}&=\cmm{L^2_x}{L_z}+\cmm{L^2_y}{L_z}+0\\\notag
	&=\hat L_x\cmm{L_x}{L_z}+\cmm{L_x}{L_z}\hat L_x+\hat L_y\cmm{L_y}{L_z}+\cmm{L_y}{L_z}\hat L_y\\
	&=-\i\hbar(\hat L_x\hat L_y+\hat L_y\hat L_x)+\i\hbar(\hat L_y\hat L_x+\hat L_x\hat L_y)=0.
\end{align}

联级Stern-Gerlach实验证明,在确定银原子的$\hat L_z$时,其$\hat L_x,\hat L_y$没有确定值,即$\hat L_x,\hat L_y,\hat L_z$不能同时有确定值。

回顾\textit{球坐标系}下
\[
	\nabla=\fkh{\pp r,\frac1r\pp\ct,\frac1{r\sin\ct}\pp\vf}\tp_\mathrm{Sp}.
\]
Laplace算符
\[
	\nabla^2=\frac1{r^2}\pp r\kh{r^2\pp r}+\frac1{r^2\sin\theta}\pp\theta\kh{\sin\theta\pp\theta}+\frac1{r^2\sin^2\theta}\pp[2]\vf.
\]
定义$\qo L=-\i\hbar\,\qo\Lambda$
\[
	\qo\Lambda:=\qo r\times\,\nabla=\fkh{0,-\frac1{\sin\theta}\pp\vf,\pp\theta}\tp_\mathrm{Sp}.
\]
特别的,代回直角坐标系后,有
\iffalse
	\begin{align*}
		\hat L_x & =\i\hbar\kh{+\sin\vf\pp\ct+\cot\ct\cos\vf\pp\vf} \\
		\hat L_y & =\i\hbar\kh{-\cos\vf\pp\ct+\cot\ct\sin\vf\pp\vf} \\
		\hat L_z & =-\i\hbar\pp\vf.
	\end{align*}
\fi
\begin{gather}
	\hat L_z=-\i\hbar\pp\vf;\\
	\hat L^2=-\hbar^2\fkh{\frac1{\sin\theta}\pp\theta\kh{\sin\theta\pp\theta}+\frac1{\sin^2\ct}\pp[2]\vf}.
\end{gather}
由$\hat L_z,\hat L^2$对易,二者有共同本征函数。
\paragraph*{$\hat L_z$的本征态}设本征值为$m\hbar$,本征函数$\psi_m(\vf)$
\[
	\hat L_z\psi_m=-\i\hbar\pv{\psi_m}\varphi=m\hbar\psi_m\thus\psi_m=C\e{\i m\vf}.
\]
且本征态应具有周期性$\psi_m(\vf+2\pi)=\psi_m(\vf)$,故$m=0,\pm 1,\pm 2,\ldots$
\[
	(\psi_m,\psi_m)=\abs C^2\int_0^{2\pi}\d\vf=1,\thus C=\frac1{\sqrt{2\pi}}.
\]
\paragraph*{$\hat L^2$的本征态}设本征值为$\lambda\hbar^2$,本征函数$Y(\ct,\vf)$
\[
	\hat\Lambda^2Y=\frac1{\sin\theta}\pp\theta\kh{\sin\theta\pv Y\theta}+\frac1{\sin^2\ct}\pv[2]Y\vf=-\lambda Y.
\]
分离变量$Y(\ct,\vf)=\varTheta(\ct)\varPhi(\vf)$
\[
	\frac\vF{\sin\ct}\dd\ct\kh{\sin\ct\dv\vC\ct}+\frac\vC{\sin^2\ct}\dv[2]\vF\vf=-\lambda\vC\vF.
\]
即
\[
	\frac{\sin\theta}\vC\dd\ct\kh{\sin\theta\dv\vC\ct}+\lambda\sin^2\ct=-\frac1\vF\pv[2]\vF\vf=m^2.
\]

引入$w:=\cos\ct,\;P(w):=\vC(\arccos w)=\vC(\ct)$
\begin{equation}
	\dd w\fkh{\kh{1-w^2}\dv Pw}+\kh{\lambda-\frac{m^2}{1-w^2}}P=0.
\end{equation}
这是缔合Legendre方程,$\pm 1$是方程的奇点,只有
\[
	\lambda=\ell(\ell+1),\quad\ell=\abs m,\abs m+1,\ldots
\]
时,方程才有收敛解$P_\ell^m(w)$。

特别的,当$m=0$时,
\[
	\dd w\fkh{\kh{1-w^2}\dv Pw}+\ell(\ell+1)P=0.
\]
便是Legendre方程,其解是Legendre多项式
\[
P_\ell(w)=\frac1{2^\ell\ell!}\dd[\ell]w\kh{w^2-1}^\ell.
\]
对应的,当$m>0$时,缔合\Legd 函数
\[
P_\ell^m(w)=\kh{1-w^2}^{m/2}\dd[m]wP_\ell(w).
\]
而对于$m<0$的情形,其实应当与$\abs m$相同;若对正负均沿用原定义,则
\[
P_\ell^{-m}=(-1)^m\frac{(\ell-m)!}{(\ell+m)!}P_\ell^m,\quad m>0.
\]

\Legd 多项式的奇偶性由$\ell$决定。
\begin{example}{\Legd 函数表}{Table of Legendre}
	$\ell=0,1,2,3$的Legendre多项式$P_\ell$和缔合Legendre函数$P_\ell^m(\cos\theta)$%\setlength\abovedisplayskip{10pt}
	\small{\begin{align*}
			P_0 & =1, & P_0^0 & =1;\\
			P_1 & =x, & P_1^0 & =\cos\theta, & P_1^1 & =\sin\theta;\\
			P_2 & =\frac12(3x^2-1), & P_2^0 & =\frac12(3\cos^2\theta-1), & P_2^1 & =3\sin\theta\cos\theta,\\
			    && P_2^2 &=3\sin^2\theta;\\
			P_3 & =\frac12(5x^3-3x), & P_3^0 & =\frac12(5\cos^3\theta-3\cos\theta), & P_3^1 & =\frac32\sin\theta(5\cos^2\theta-1),\\
			    && P_3^2 & =15\sin^2\theta\cos\theta, & P_3^3 & =15\sin^3\theta;
			%\\P_4&=\frac18(35x^4-30&x^2&+3),&P_5&=\frac18(63x^5-70x^3+15x).
		\end{align*}}
\end{example}
轨道角动量本征函数最后为
\[
	Y_{\ell m}(\ct,\vf)=N_{\ell m}P_\ell^m(\cos\ct)\e{\i m\vf},
\]
%正交归一
%\[\int\abs{Y_{\ell m}(\ct,\vf)}^2\d\Omega=1,\quad\d\Omega=\sin\ct\d\ct\nd\vf.\]
由于
\[
	\int_{-1}^1P_\ell^m(x)P_{\ell'}^{m'}(x)\d x=\frac2{2\ell+1}\frac{(\ell+m)!}{(\ell-m)!}\vd_{\ell\ell'}\vd_{mm'}.
\]
得
\[
N_{\ell m}=(-1)^m\sqrt{\frac{2\ell+1}{4\pi}\frac{(\ell-m)!}{(\ell+m)!}}
\]
称$Y_{\ell m(\ct,\vf)}$为球谐函数,$\ell$为角量子数,$m$为磁量子数。原子物理中将$\ell=0,1,2,3,\ldots$的状态分别称为$\mathrm{s,p,d,f}$态。

$\hat L^2$的本征值$\ell$下有$2\ell+1$个可能的$m$,简并度为$2\ell+1$。
\begin{example}{球谐函数表}{Table of Spherical Harmonics}
	$\ell=0,1,2,3$已归一化后的$Y_\ell^m$
	\small{\begin{align*}
			Y_0^0~\;    & =\frac1{2\sqrt\pi},                                           & Y_2^{\pm 2} & =\sqrt{\frac{15}{32\pi}}\sin^2\theta\e{\pm 2\i\phi},                 \\
			Y_1^0~\;    & =\sqrt{\frac3{4\pi}}\cos\theta,                               & Y_3^0~\;    & =\sqrt{\frac7{16\pi}}(5\cos^3\theta-3\cos\theta),                    \\
			Y_1^{\pm 1} & =\mp\sqrt{\frac3{8\pi}}\sin\theta\e{\pm\i\phi},               & Y_3^{\pm 1} & =\mp\sqrt{\frac{21}{64\pi}}\sin\theta(5\cos^2\theta-1)\e{\pm\i\phi}, \\
			Y_2^0~\;    & =\sqrt{\frac5{16\pi}}(3\cos^2\theta-1),                       & Y_3^{\pm 2} & =\sqrt{\frac{105}{32\pi}}\sin^2\theta\cos\theta\e{\pm 2\i\phi},      \\
			Y_2^{\pm 1} & =\mp\sqrt{\frac{15}{8\pi}}\sin\theta\cos\theta\e{\pm \i\phi}, & Y_3^{\pm 3} & =\mp\sqrt{\frac{35}{64\pi}}\sin^3\theta\e{\pm 3\i\phi}.
		\end{align*}}
\end{example}
球谐函数是$\hat L^2$和$\hat L_z$的共同本征函数
\begin{align*}
	\begin{cases}
		\hat L^2Y_{\ell m}=\ell(\ell+1)\hbar^2Y_{\ell m}, & \ell=0,1,2,\ldots           \\
		\hat L_zY_{\ell m}=m\hbar Y_{\ell m},             & m=-\ell,-\ell+1,\ldots,\ell
	\end{cases}
\end{align*}
\paragraph*{宇称}作空间反射变换$\r\to-\r$,对应球坐标中$(r,\ct,\vf)\to(r,\pi-\ct,\pi+\vf)$
\begin{gather*}
	P_{\ell}^{m}\kh{\cos (\pi-\theta)}=P_{\ell}^{m}(-\cos \theta)=(-1)^{\ell-m} P_{\ell}^{m}(\cos \theta) \\
	\e{\i m(\pi+\varphi)}=(-1)^{m}\e{\i m \varphi}                                                   \\
	Y_{\ell m}(\pi-\theta, \pi+\varphi)=(-1)^{\ell} Y_{\ell m}(\theta, \varphi)
\end{gather*}
因此$Y_{\ell m}(\ct,\vf)$的宇称为$(-1)^\ell$.
\paragraph*{递推关系}
{\small\begin{align}\label{costhetaYlm}
	\cos\theta Y_\ell^m&=\sqrt{\frac{(\ell+1)^2-m^2}{(2\ell+1)(2\ell+3)}}Y_{\ell+1}^m+\sqrt{\frac{\ell^2-m^2}{(2\ell-1)(2\ell+1)}}Y_{\ell-1}^m.\\
	\sin\theta\e{\pm\i\vf}Y_\ell^m&=\pm\sqrt{\frac{(\ell\pm m+1)(\ell\pm m+2)}{(2\ell+1)(2\ell+3)}}Y_{\ell+1}^{m+1}+\sqrt{\frac{(\ell\mp m)(\ell\mp m+1)}{(2\ell-1)(2\ell+1)}}Y_{\ell-1}^{m\pm 1}.
\end{align}}
\clearpage
\section{跃迁理论}
若体系的Hamilton量$\hat H_0$不显含时间,能量本征值问题的解为
\[
	\hat H_0\ket n=E_n\ket n.
\]
若$\hat H(t)=\hat H_0+\hat H'(t)$显含时间,体系将有一定的概率离开初态$\ket{k}$而处于其它定态$\ket{k'}$,这就是\textbf{量子跃迁}。
\subsection{量子态随时间的变化}
状态随时间的演化由\Schr 方程决定
\begin{align}
	\begin{cases}
		\i\hbar\pp t\ket{\psi(t)}=\hat H(t)\ket{\psi(t)},\\
		\ket{\psi(0)}=\ket{k}.
	\end{cases}
\end{align}
采用能量$\hat H_0$表象 
\begin{align}
	\ket{\psi(0)}=\sum_n C_{nk}(t)\ket n\e{-\i E_nt/\hbar},\quad C_{nk}(0)=\vd_{nk}.
\end{align}
体系$t$时刻跃迁到定态$\ket{k'}$的概率为$\abs{C_{k'k}(t)}^2$。

为求出跃迁概率,将表象表达式带入\Schr 方程,假定$\hat H'$中不包括$\p/\p t$作用
{\small\begin{align}\notag
	\i\hbar\sum_n\fkh{\dot C_{nk}(t)-\cancel{\frac{\i E_nC_{nk}(t)}\hbar}}\ket n\e{-\i E_nt/\hbar}&=\sum_nC_{nk}(t)\fkh{\cancel{E_n}+\hat H'(t)}\ket n\e{-\i E_nt/\hbar}.\\
	\i\hbar\sum_n\dot C_{nk}(t)\ket n\e{-\i E_nt/\hbar}&=\sum_nC_{nk}(t)\hat H'(t)\ket n\e{-\i E_nt/\hbar}.
\end{align}}
与$\ket{k'}$内积
\begin{align}
	\i\hbar\dot C_{k'k}(t)\e{-\i E_{k'}t/\hbar}=\sum_nC_{nk}(t)\bra{k'}\hat H'(t)\ket n\e{-\i E_nt/\hbar}
\end{align}
定义 
\begin{align}
	H_{k'n}'(t):=\bra{k'}\hat H'(t)\ket n,\quad\omega_{k'n}=\frac{E_{k'}-E_n}\hbar.
\end{align}
得到跃迁振幅$C_{k'k}(t)$满足 
\begin{align}
	\begin{cases}
		\i\hbar\dd tC_{k'k}(t)=\sum_nH'_{k'n}(t)\e{\i\omega_{k'n}t}C_{nk}(t),\\
		C_{k'k}(0)=\vd_{k'k}.
	\end{cases}
\end{align}
两边对$t$积分
\begin{align}
	C_{k'k}(t)=C_{k'k}(0)+\frac1{\i\hbar}\int_0^t\sum_nH'_{k'n}(\tau)\e{\i\omega_{k'n}\tau}C_{nk}(\tau)\d\tau
\end{align}
这是一个积分方程,可采用迭代求解。

迭代一次
\begin{align*}
	C_{k'k}(t)&=\vd_{k'k}+\frac1{\i\hbar}\int_0^t\sum_nH'_{k'n}(\tau)\e{\i\omega_{k'n}\tau}\cdot\\
	&\quad\fkh{\vd_{nk}+\frac1{\i\hbar}\int_0^\tau\sum_nH'_{k'n}(\pi)\e{\i\omega_{k'n}\pi}C_{nk}(\pi)\d\pi}\d\tau\\
	&=\vd_{k'k}+\frac1{\i\hbar}\int_0^tH'_{k'k}(\tau)\e{\i\omega_{k'k}\tau}\d\tau+\cdots
\end{align*}
零级近似
\[
C_{k'k}^{(0)}=\vd_{k'k};
\]
一级近似 
\[
C_{k'k}^{(1)}=\frac1{\i\hbar}\int_0^tH'_{k'k}(\tau)\e{\i\omega_{k'k}\tau}\d\tau,
\]
代表直接从初态$\ket k$跃迁到末态$\ket{k'}$,故跃迁概率
\begin{align}
	P_{k'k}(t)=\frac1{\hbar^2}\abs{\int_0^tH'_{k'k}(\tau)\e{\i\omega_{k'k}\tau}\d\tau}^2.
\end{align}
由$\hat H'$是Hermite的。故$P_{k'k}=P_{kk'}$,即从初态到末态的跃迁概率等于从末态到初态的跃迁概率。

对于初态$\ket k$和末态$\ket{k'}$都有简并的情况,计算跃迁概率应对$\ket k$能级各简并态求平均,而对$\ket{k'}$能级各简并态求和,此时跃迁概率不一定相等。

二级近似
\[
C_{k'k}^{(2)}=\frac1{(\i\hbar)^2}\sum_n\int_0^tH'_{k'n}(\tau)\e{\i\omega_{k'n}\tau}\int_0^\tau H'_{nk}(\pi)\e{\i\omega_{nk}\pi}\d\pi\d\tau,
\]
代表从初态$\ket k$经中间态$\ket n$跃迁到末态$\ket{k'}$。
\subsection{周期微扰和常微扰}
没讲
\subsection{光的吸收与辐射}
光与原子的相互作用包括受激吸收、受激辐射和自发辐射。其中自发辐射是前面的理论无法解释的,Einstein基于热力学和统计物理中的平衡概念给出过半唯象的理论,巧妙地导出了自发辐射系数。
\paragraph*{电偶极跃迁}若入射光为理想单色偏振光,
\[
	\bm E=\bm E_0\cos(\omega t-\bm k\cdot\bm r),\quad\bm B=\frac{\bm k}{\abs k}\times\bm E.\CGS
\]
对电子的作用 
\[
	\bm f=-e\kh{\bm E+\frac{\bm v}c\times\bm B},\CGS
\]
原子中电子的速度$v\ll c$,故可仅考虑电场的作用。

对于可见光和紫外光\footnote{X光并不满足。}波长$\lambda\gg a$\,(Bohr半径),故在原子范围内电场可视为均匀场
\[
	\bm E\doteq\bm E_0\cos\omega t.
\]
光对原子的作用可近似表示成电子的电偶极矩与电场的相互作用
\[
	\hat H'(t)=-\bm D\cdot\bm E=W\cos\omega t.
\]
其中电子的电偶极矩$\bm D=-e\r$,电偶极矩与电场作用引起的跃迁称为电偶极跃迁。
\begin{align*}
	C_{k'k}^{(1)}(t)&=\frac1{\i\hbar}\int_0^tH_{k'k}'(\tau)\e{-\i\omega_{k'k}\tau}\d\tau=\frac{W_{k'k}}{\i\hbar}\int_0^t\cos\omega\tau\e{-i\omega_{k'k}\tau}\d\tau\\
	&=-\frac{W_{k'k}}{2\hbar}\fkh{\frac{\e{\i(\omega_{k'k}+\omega)t}-1}{\omega_{k'k}+\omega}+\frac{\e{\i(\omega_{k'k}-\omega)t}-1}{\omega_{k'k}-\omega}}.
\end{align*}
下面讨论原子吸收光的跃迁,$E_{k'}>E_k$,只有当入射光$\omega\doteq\omega_{k'k}$时,才会引起$E_k\to E_{k'}$跃迁,此时
\begin{align*}
	C_{k'k}^{(1)}(t)=-\frac{W_{k'k}}{2\hbar}\frac{\e{\i(\omega_{k'k}-\omega)t}-1}{\omega_{k'k}-\omega}.
\end{align*}
从$k\to k'$的概率为
\begin{align}
	P_{k'k}(t)=\abs{C_{k'k}^{(1)}(t)}^2=\frac{\abs{W_{k'k}}^2}{4\hbar^2}\fkh{\frac{\sin(\omega_{k'k}-\omega)t/2}{(\omega_{k'k}-\omega)/2}}^2
\end{align}
$t\to\infty$时,有 
\begin{align}
	P_{k'k}(t)=\frac{\pi t}{2\hbar^2}\abs{W_{k'k}}^2\vd\kh{\omega_{k'k}-\omega};
\end{align}
跃迁速率
\begin{align}
	w_{k'k}&=\dd tP_{k'k}=\frac\pi{2\hbar^2}\abs{W_{k'k}}^2\vd\kh{\omega_{k'k}-\omega}\\
	&=\frac\pi{2\hbar^2}\abs{\bm D_{k'k}}^2E_0^2\cos^2\theta\vd\kh{\omega_{k'k}-\omega},
\end{align}
$\theta$为电子的电偶极矩与电场的夹角。

而对于非偏振光,应对$\cos^2\theta$取平均
\[
	\ave{\cos^2\theta}=\frac1{4\pi}\int_0^{2\pi}\bs5\int_0^\pi\cos^2\theta\sin\theta\d\theta\nd\phi=\frac13.
\]
跃迁速率 
\begin{align}
	w_{k'k}=\frac\pi{6\hbar^2}\abs{\bm D_{k'k}}^2E_0^2\vd\kh{\omega_{k'k}-\omega}.
\end{align}

对于非单色光,总跃迁速率是对各种频率求和
\begin{align}
	w\tot=\int\iti\omega_{k'k}\d\omega=\frac\pi{6\hbar^2}\abs{\bm D_{k'k}}^2E_0^2(\omega_{k'k}).
\end{align}

频率为$\omega$的电磁波能量密度的时间平均值
\begin{align*}
	\rho(\omega)=\frac1{8\pi}\ave{E^2+B^2}=\frac1{8\pi}E_0^2(\omega).\CGS
\end{align*}
故非偏振自然光引起的跃迁速率
\begin{align}
	w_{k'k}=\frac{4\pi^2}{3\hbar^2}\abs{\bm D_{k'k}}^2\rho(\omega_{k'k})=\frac{4\pi^2e^2}{3\hbar^2}\abs{\r_{k'k}}^2\rho(\omega_{k'k}).
\end{align}
由$\r$为奇宇称算符,只有$\ket k,\ket{k'}$宇称相反时,$\abs{\r_{k'k}}^2$才不为0。又
\begin{align*}
	\r=r\zkh{\sin\theta\cos\vf,\sin\theta\sin\vf,\cos\ct}
\end{align*}
由第 \pageref{costhetaYlm} 页的球谐函数递推式知,跃迁态间需满足$\D\ell=\pm 1,\,\D m=0,\pm 1$。
\paragraph*{Einstein跃迁理论}
\end{document}