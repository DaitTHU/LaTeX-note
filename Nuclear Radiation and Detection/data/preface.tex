\preface

\paragraph{课程简介}
工物系一级学科\textbf{核科学与技术}下属5个二级学科:
\begin{compactitem}
	\item 核技术及应用
	\item 核能科学与工程
	\item 核燃料循环与材料
	\item 辐射防护和环境保护
	\item 医学物理与工程
\end{compactitem}
本课是进入核科学领域的第一个桥梁课程,构成了本学科其它课程的基础,为工物系本科生的必修专业基础课。

\paragraph{课程特色}
\begin{itemize}
	\item 一 个体系;
	\item 二 个关键(衰变纲图、$\gamma$能谱);
	\item 三 种探测器(气体、闪烁、半导体探测器);
	\item 四 个基本内容:
	\begin{compactenum}
		\item 辐射与物质相互作用的物理、数学规律;
		\item 探测器输出信号形成的物理过程;
		\item 探测器输出回路与其工作状态的关系;
		\item 统计涨落对探测器性能的影响。
	\end{compactenum}
\end{itemize}

\paragraph{教参}
\href{http://reserves.lib.tsinghua.edu.cn/Search/BookDetail?bookId=bcac7c78-bf4c-4bfd-9ecc-f1b6f47e6b10}{《核辐射物理及探测学》哈尔滨工程大学出版社,第2版。}

