\chapter{半导体探测器}
气体探测器
能量分辨率较好,但探测效率太低;
闪烁体探测器
探测效率很好,但能量分辨率不好;% 载流子的形成环节太多,不断损失
半导体探测器(semiconductor detectors)
能量分辨率很好,且探测效率较高。

\section{半导体基本性质}

本征(intrinsic)半导体是理想的、纯净的半导体。在本征半导体中,电子密度$n$和空穴密度$p$相同
\[
	n=p=CT^{3/2}\e{-E_G/2\kB T},
\]
式中$E_G$是禁带宽度。
室温(\SI{300}{K})下本征Si $n=p=\SI{1.45e10}{/cm^3}\ll$金属中的电子密度$\sim\SI{E22}{/cm^3}$。

在半导体材料中有选择地掺入一些杂质,使原子在半导体禁带中产生局部能级,影响半导体的性质。
\begin{compactitem}
	\item 掺有施主杂质\footnote{V族元素,能级接近导带底端能量,室温下热运动使杂质原子离化。}的半导体中多数载流子是电子,叫做N型半导体。
	\item 掺有受主杂质\footnote{III族元素,能级接近价带顶端能量,室温下价带中电子容易跃迁到这些能级上。}的半导体中多数载流子是空穴,叫做P型半导体。
\end{compactitem}
\paragraph{半导体作为探测介质的物理性能}
\subparagraph{载流子密度}
%在半导体中,
电子与空穴的密度的乘积是(与温度相关的)常数
\[
	np\propto T^3\e{-E_G/\kB T},
\]
\subparagraph{补偿效应}
若在N型半导体中加入受主杂质,可使$p$增大同时$n$减小,当$p>n$时,N型半导体转化为P型半导体,叫做补偿效应;

当$p=n$时,称为完全补偿。

\subparagraph{平均电离能}
半导体的平均电离能很小($W\sim\SI{3}{eV}$),近似与入射粒子种类和能量无关。入射粒子电离产生的电子和空穴的数目是相同的(不受掺杂影响),且服从Fano分布。
\subparagraph{载流子的迁移率}
载流子产生之后会发生扩散和漂移,通常扩散可以忽略不计(若对位置精度要求不高),电场强度不高($E<\SI{E3}{V/cm}$)时,载流子迁移率$\mu$正比于场强$E$
\[
	u_n=\mu_nE,\enspace u_p=\mu_pE,
\]
迁移率$\mu$随温度$T$下降而上升%,低温时漂移速度快
。%空穴迁移率小于电子迁移率
$\mu_p<\mu_n$,但不过相差$\numrange{2}{3}$倍。

场强大时,漂移速度逐渐饱和;电子和空穴的最大速度$\sim\SI{E7}{cm/s}$%量级,比气体中电子的$\SI{E6}{cm/s}$要快。由于半导体探测器的尺寸并不大
,因此半导体探测器的电流持续时间是较短的。
\subparagraph{载流子的寿命}
载流子在产生之后,除了会发生扩散或在电场下漂移并形成信号,还有可能发生:
\begin{compactitem}
	\item 陷落(trap):Au, Zn, Cd等的存在,使载流子陷落,不能移动,最终会释放,但是对信号没有贡献……
	\item 复合(recombination):deep impurities也可以捕获电子和空穴,导致复合%(4b5a),比直接复合(1b)要容易
\end{compactitem}
载流子寿命$\tau$:从产生到重新陷落(复合)的平均时间间隔。
载流子的漂移长度(trapping length)
\[
	L=\mu E\tau
\]
%半导体探测器要求:载流子的漂移长度 >> 灵敏体积的长度
理想晶体:$\tau\sim\si{s}$;高纯度的Si和Ge:$\tau\sim\si{ms}\gg$信号收集时间($\SI{e-7}{s}$)
\subparagraph{半导体的电阻率}
电阻率与电子、空穴浓度及其迁移率有关
\[
	\rho=\frac1{e(n\mu_n+p\mu_p)}.
\]
本征或完全补偿时($n=p$),电阻率$\rho$最大。
%室温下:
\subparagraph{探测器对半导体的要求}
\begin{compactenum}
	\item 长载流子寿命:保证载流子能够被收集
	\item 高电阻率:漏电流小、结电容小
\end{compactenum}

\section{均匀型半导体探测器}

\paragraph{带电粒子与半导体晶体的相互作用}
带电粒子与晶体中的电子相互作用,迅速损失能量,产生的电子-空穴对数服从Fano分布。

电子由价带(满带)进入导带,
可从最高(第一)价带进入最低(第一)导带,
也可从更深的满带激发到更高的导带中。
ps时间后,电子降至第一导带,空穴上升至第一价带。
或者是产生$\delta$电子,继续电离。
\paragraph{均匀型半导体探测器的工作原理及性能}
均匀型半导体探测器相当于固体电离室,电子-空穴分别向正负极漂移,在外电路形成电流信号,电子-空穴的收集时间$\sim\SI{e-7}{s}$,探测效率$\gg$气体探测器

金刚石、CdZnTe、HgI、TlBr
禁带宽度大,电阻率高,可以室温下工作,但载流子寿命很短($\SI{e-8}{s}$)。

半导体具有长的载流子寿命(ms),能够避免上述问题,但如果半导体电阻率太小,则电子热运动的噪声会很大,能量分辨率低。
\paragraph{小结}
半导体探测器平均电离能$W$小,是能量分辨率好的基础。
但$W$小的本质原因是禁带宽度$E_G$小,这意味着热运动可以激发电子跨越禁带,在价带和导带上分别形成数量可观的空穴和电子,减小了电阻率。
小的电阻率使热噪声变得不可忽略,该噪声将会严重影响能量分辨率。
$E_G$大的材料虽然电阻大,但载流子的寿命很短,漂移长度短。

我们先选择了Si, Ge载流子寿命长、漂移长度大的优点。
再着手解决它电阻率低的缺点。

\section{P-N结型半导体探测器}

\subsection{工作原理}

把P型半导体和N型半导体结合在一起,多数载流子的扩散形成了P-N结,不均匀的空间电荷自建电场
阻止了多数载流子的进一步扩散,形成耗尽区(depeleted zone),因此P-N结具有高的电阻率。

气体和闪烁体探测器都有探测器所对应的电容$C_1$——气体探测器是由两个极板构成的,闪烁探测器则是由D$_n$和阳极构成的。%由于它们的面积不可能为0,则必然存在。
气体和闪烁体探测器极板的几何关系是完全确定的,因此具有确定的$C_1$。

%对P-N结探测器,这个$C_1$也存在(结区横截面积不可能为0),但P-N结探测器的两个“极板”距离却是变化的,这使得它的$C_1$不会是确定的。
但对P-N结探测器,其结区宽度与工作电压有关,即$C_1$不确定。
这是个麻烦,我们需要用\textbf{电荷灵敏前置放大器}来解决。
\paragraph{击穿电压}
增加结区厚度$W$的好处:
\begin{compactitem}
	\item 增大灵敏体积,使带电粒子能量能够全部沉积在其中
	\item 减小探测器电容
\end{compactitem}
坏处:
\begin{compactenum}
	\item 结区内反向电流$I_{G'}$变大
	\item 结区的电场不均匀,在交界区场强最大,有可能发生击穿(Zener击穿或雪崩击穿)
\end{compactenum}
应对方法:电阻率越高,则耗尽层越厚,电场越弱,不易击穿——
加保护电阻,限制电流,可防止探测器的击穿损坏

\subsection{输出信号}

探测器中电子的漂移速度
\[
	\dv xt=-\mu_nE=\mu_n\frac{N_de}\varepsilon(W-x),
\]
解得
\begin{align}
	x=W-(W-x_0)\e{-t/\tau},\quad\tau:=\frac\varepsilon{e\mu_nN_d}=\varepsilon\rho.
\end{align}
电子漂移引起的感应电流
\[
	I_n=\frac{2e}{W^2\tau}(W-x_0)^2\e{-2t/\tau};
\]
空穴漂移引起的感应电流
\[
	I_p=\frac{2e}{W^2\tau'}(W-x_0)^2\e{2t/\tau'}.%\enspace\tau':=\frac\varepsilon{e\mu_pN_d}
\]
由于电离发生的位置不同,电子和空穴产生的位置也将会不同,%因此每个电子、空穴运动各自所扫过的电位差占V0的份额就是不同的,其运动时间也是不同的。
所以,P-N结探测器的电流形状不可能是确定的。
由于弹道亏损必然存在,那么对于具有这种特点的电流,就要求成型电路的$R_0C_0\gg$电流持续时间,才能保证$V\maxi\propto Q\propto E\dep$。

通常,电子和空穴的最大收集时间为$t_{cn}\sim\si{ns},\enspace t_{cp}\sim\SI{10}{ns}$。
当$R_0C_0\gg t_c$时,输出电压脉冲前沿由电流脉冲形状决定,
后沿以输出回路时间常数$R_0C_0$指数规律下降。
探测器输出电压脉冲幅度为
\[
	V\maxi=\frac{Ne}{C_0},
\]
输出回路等效电容$C_0=C_d+C'+C_i$,而探测器结区电容$C_d$随反向工作电压变化$C_d\propto V^{-1/2}$。为避免输出信号幅度的变化,就需要采用电荷灵敏前置放大器,此时
\[
	C_0=C_d+C'+C_i+(1+A)C_f\doteq(1+A)C_f.
\]
电荷灵敏前置放大器的输出脉冲幅度
\begin{align}
	V\maxi=\frac{Ne}{C_0}A\doteq\frac{Ne}{C_f}.
\end{align}
输出回路等效电阻
\[
	R=R_d\parallel R_a\parallel R_i\parallel\frac{R_f}{1+A}\doteq\frac{R_f}{1+A}.
\]
等效输出回路$R_0C_0=R_fC_f$,通常选择$C_f\sim\si{pF},\enspace R_f\sim\si{G\ohm}$,从而$R_fC_f\sim\si{ms}\gg\SI{10}{ns}$。

但是由于$C_f$很小,对噪声的负反馈很弱。%一个$\vd v_i$的噪声会被运放放大
%\vd v_o=\frac{C_d+C'+C_i+C_f}{C_f}\abs{\vd v_i}.
增大工作偏压使结区变宽,$C_1$减小,则噪声也小,对能量分辨率是有利的。

\subsection{性能}

电子空穴对服从Fano分布,其统计涨落
\[
	\FWHM_1=2.355\sqrt{FwE}.
\]
探测器和电子学噪声的影响
\[
	\FWHM_2=2.355w\mathrm{ENC}.
\]
等效噪声电荷(equivalent noise charge, ENC):放大器输出端噪声电压均方根值等效到输入端的电荷数。

%电荷灵敏前置放大器的噪声参数:零电容噪声(keV)、噪声斜率(keV/pF)。
\begin{compactitem}
	\item 分辨时间:受制于探测器输出电流脉冲的宽度。P-N结探测器载流子的收集时间$\sim\si{ns}$,这是分辨时间的极限。考虑到电荷灵敏前置放大器的时间常数,分辨时间可达ms。经过主放成型后,可在$\SI{e-5}s$量级。
	\item 时间分辨本领(ns):脉冲信号的上升时间。电压放大器$\SIrange{1}{10}{ns}$,电流放大器更小。
	\item 时滞:基本为0,电子空穴对一旦产生,就有电流了。
\end{compactitem}
\iffalse
\begin{compactenum}
	\item 
\end{compactenum}
\begin{compactitem}
	\item 
\end{compactitem}
\fi

\paragraph{P-N结型半导体探测器的应用}
重带电粒子的测量:优异的能量分辨率和线性。

\section{P-I-N型半导体探测器}

基体用P型半导体(因为极高纯度的材料多为P型),例如掺B的Si或Ge单晶。
一端表面蒸Li,Li离子化为Li$^+$,形成P-N结。

外加电场,使Li$^+$漂移。
Li$^+$与受主杂质(如Ga$^-$)中和,并可实现自动补偿,形成I区。
I区是完全补偿区,
呈电中性,电场均匀;
耗尽时电阻率可达$\SI{E10}{\ohm.cm}$;
灵敏体积厚度可达$\SIrange{10}{20}{cm}$。

\section{HPGe半导体探测器}

Li漂移探测器需要低温保存与使用、生产周期(Li漂移时间)长,现常使用高纯锗(high purity germanium, HPGe)半导体探测器。%Ge的纯度可以达到PPT($10^{-12}$)。

耗尽层的宽度
\[
	W=\sqrt{\frac{2\varepsilon V_0}{eN_i}}.
\]
HPGe杂质密度$N_i\sim\SI{e10}{/cm^3}$,可使$W>\SI{1}{cm}$。HPGe探测器是P-N结型探测器,仍须低温使用,但可常温保存。

\section{锂漂移和HPGe半导体探测器的性能与应用}

高分辨率可用作能谱分析(而非计数器),关心全能峰。

光电效应、Compton效应、电子对效应($>\SI{1.022}{MeV}$)是探测$\gamma$的基本相互作用。
形成全能峰的最后一步必然是光电效应。

分析复杂$\gamma$能谱时,希望有高的峰康比
%峰顶计数与康普顿坪平均计数之比
($\numrange{20}{90}$):
\begin{compactenum}
	\item 增大探测器灵敏体积
	\item 改善几何形状:长度=直径
	\item 通过Compton反符合技术可进一步提高峰康比一个量级
\end{compactenum}

