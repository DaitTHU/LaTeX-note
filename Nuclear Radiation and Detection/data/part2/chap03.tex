\chapter{闪烁探测器}

闪烁探测器(scintillation detectors)是利用辐射在某些物质中产生的闪光来探测电离辐射的探测器。
闪烁探测器的优点:
\begin{compactenum}
	\item 密度、体积大,对中性射线($\gamma,\nton$)的探测效率高,易于沉积能量,因此能够测量能谱。
	\item 时间特性好,有的探测器(如塑料闪烁体、BaF$_2$)可实现ns的时间分辨能力。
\end{compactenum}

\section{闪烁探测器的基本原理}

闪烁探测器是利用辐射在闪烁体内闪光来探测电离辐射的探测器。
由
闪烁体、
光电倍增管(photomultiplier tube, PMT)、
高压电源、
低压电源、
分压器、
前置放大器
构成。
其工作过程为:
\begin{compactenum}
	\item 射线沉积能量,电离能损产生带电粒子
	%电离能损:γ射线制造三种次级电子,中子核反应后的带电粒子,或者带电粒子直接入射。
	\item 带电粒子使闪烁体电离或激发,退激后发出大量荧光光子(可见光);
	\item 荧光光子被PMT的光阴极转换为光电子,并被第一打拿极收集;
	\item 电子在PMT的打拿极间运动并倍增(\numrange{e7}{e10});
	\item 流经外回路
	%决定工作状态:电流脉冲型、电压脉冲型?
\end{compactenum}

\subsection{闪烁体}

高探测效率(高$Z$、高密度、大尺寸)、
高发光效率、
能量线性好、
自吸收小、
发光时间短、
可加工性好、
易于耦合(合适的折射率)
\paragraph{闪烁体的分类}
\begin{compactitem}
	\item 无机闪烁体:探测效率高、光输出产额高、线性好;但发光时间较长
	\begin{compactitem}
		\item 纯晶体:BGO, BaF$_2$
		\item 无机盐晶体:NaI(Tl), ZnS(Ag)
		\item 玻璃体:Li$_2$O$·$2SiO$_2$(Ce)
	\end{compactitem}
	\item 有机闪烁体:发光时间短;但光输出产额低、光电截面低
	\begin{compactitem}
		\item 有机晶体:蒽,萘,芪
		\item 有机液体闪烁体
		\item 塑料闪烁体
	\end{compactitem}
	\item 气体闪烁体:Ar, Xe
\end{compactitem}
\paragraph{无机闪烁体的发光机制}
以NaI(Tl), CsI(Tl),CsI(Na)等为最典型,又称卤素碱金属晶体(Alkali Halide Scintillator)。

入射带电粒子在闪烁体内电离,产生电子-空穴对或激子(exciton);
导带上电子和价带上空穴可以复合成激子,
激子吸收热运动能量也可变成电子-空穴。
退激可能发出光子,也可能使晶格振动而不发光。

纯离子晶体退激发出的光子容易被晶体自吸收,传输出的光子少;禁带宽度大,退激发光在紫外范围,光阴极不响应。

晶体掺杂$10^{-3}$量级的激活剂(activator),形成特殊晶格点,在禁带中形成局部能级。
电子-空穴迁移到杂质原子处,使杂质原子处于激发态,形成发光或复合中心(luminescence/ recombination centers)。杂质原子退激:
\begin{compactitem}
	\item 荧光(fluorescence):1$\SIrange{10}{500}{ns}$。
	\item 磷光(phosphorescence):亚稳态,发光时间较长,是afterglow的主要部分。
	\item 猝灭(quenching):转换为晶格的热运动。
\end{compactitem}
\paragraph{有机闪烁体的发光机制}
有机闪烁体都是苯环化合物,分子之间仅有松散的范德瓦尔斯力。其激发与发光是由分子自身的激发和跃迁产生
的。荧光在ns级,磷光在ms级或更长。

发射、吸收光谱的峰值是分开的,因此有机闪烁体对其所发射的荧光是透明的。
在溶剂中加入高效闪烁物质,可提高闪烁效率,构成二元有机闪烁体。
发射谱的短波部分与吸收谱的长波部分有重叠,可在有的有机闪烁体中加入移波剂,以减少自吸收。

\subsection{闪烁体的物理特性}

\paragraph{发光光谱}
闪烁体发射光子数与光子波长(能量)的关系曲线,与闪烁体、激活剂(移波剂)、温度有关。
\paragraph{发光效率}
希望发光效率较高,且对不同能量保持常数。
\begin{compactenum}
	\item 光能产额(光输出,光产额)
	\[
		Y_{\mathrm{ph}}:=\frac{\avg n_{\mathrm{ph}}}{E\dep}.
	\]
	NaI(Tl)晶体$Y_{\mathrm{ph}}=\SI{4.3e4}{/MeV}$,每荧光光子$\SI{23}{eV}$。
	\item 闪烁效率(发光效率,能量转换效率)
	\[
		C_{\mathrm{np}}:=\frac{E_{\mathrm{ph}}}{E\dep}=\frac{h\avg\nu}{E\dep}.
	\]
	\item 相对闪烁效率(相对发光效率)
\end{compactenum}
\paragraph{闪烁发光时间}
由上升时间与衰减时间决定

可能有快、慢成分。
\paragraph{其它特性}
探测效率等

\subsection{光的收集与光导}

闪烁光的收集需要:
\begin{compactitem}
	\item 光学反射层:让光向光阴极的方向传播。
	反射方式
	\begin{compactitem}
		\item 镜面(specular)反射:铝箔、镀铝塑料薄膜
		\item 漫反射(diffuse):MgO、TiO$_2$、聚四氟乙烯塑料带等
	\end{compactitem}
	\item 光学耦合剂:让光可以射出闪烁体。
	
	光子由光密物质射向光疏物质时,存在发生全反射的临界角,在闪烁体与PMT间填充光学耦合剂,可以让更多角度的光射出闪烁体。
	\item 光导:连接闪烁体与光电转换器件。
	
	利用大折射系数材料的全反射。
\end{compactitem}

\section{光电倍增管}

\paragraph{光电倍增管的类型}略
\paragraph{光电倍增管的结构与工作原理}
光电倍增管结构:
\begin{compactenum}
	\item 光学窗:光阴极所附着的结构
	\item 光阴极(photocathode):通过光电效应将荧光光子转换为光电子。%所有光阴极都有截止波长,因此并非所有光子都会被转换为电子。
	\item 打拿极(dynode):电子倍增
	\item 阳极(anode):收集电子
\end{compactenum}

打拿极要求:
次级电子产额大;
热电子与光电子发射小;
大电流工作时性能稳定;
快速响应。

打拿极的次级电子产额$\delta$
\[
	\delta:=\frac{\text{发射的次级电子数}}{\text{入射的初级电子数}}.
\]

\paragraph{光电倍增管的供电回路}
\begin{compactenum}
	\item K-D$_1$电压较高(几倍于其它打拿极间的电压),可有效收集光电子,减少电子飞行时间的离散;
	\item 中间各打拿极均匀分压;
	\item 最后几个打拿极间高电压、大电流,需要电容稳压;
	\item 最后打拿极与阳极间电压较小。
\end{compactenum}
分压电路:提供静态工作点,直流电流应显著大于脉冲电流。

分压器所用电阻的温度系数应当小,稳定性高。
总功率不要太大,以免PMT因为温度升高而漂移

正高压供电方式阴极接地即可;但
需要隔直电容:
不适合于累计状态、
易受高压纹波的影响。

负高压供电方式无需隔直电容,适合高计数率、定时特性好、适合累计状态;但阴极处于高压,需防止与周围接地材料之间的高压漏电信号、场致发光(玻璃)。
\paragraph{光电倍增管的主要性能}
\begin{compactitem}
	\item 光阴极的光谱响应:光阴极发射光电子的几率与光子波长的关系,定义量子效率
	\begin{align}
		Q_\mathrm k(\lambda):=\frac{\text{发射电子数}}{\text{入射光子数}}.
	\end{align}
	NaI(Tl):若$Q=25$ - 30\%,则$\SIrange{100}{120}{eV/\text{光电子}}$
	\item 光阴极、阳极的光照灵敏度:
	\[
		S_\mathrm k=\frac{i_\mathrm k}F,\quad S_\mathrm a=\frac{i_\mathrm a}F.
	\]
	$i_\mathrm k,i_\mathrm a$是阴极电流和阳极电流(A),$F$是入射到光阴极的光通量(lm)
	\item 第一打拿极(D1)的电子收集系数
	\[
		g_\mathrm c:=\frac{\text{第一打拿极收集到的光电子数}}{\text{光阴极发出的光电子数}}.
	\]
	$g_\mathrm c$对PMT的幅度分辨率影响较大,在有聚焦极的光电倍增管中,$g_\mathrm c>95\%$
	\item 光电倍增管的电流放大倍数
	\[
		M=\frac{\text{阳极收集到的电子数}}{\text{第一打拿极收集到的光电子数}}=(g\delta)^n.
	\]
	$g$为电子传输效率,$\delta$为各级电子倍增系数,$n$为倍增级数。
\end{compactitem}
则%荧光光子数目$\avg n_{\mathrm{ph}}=E\dep Y_{\mathrm{ph}}$,
光电转换因子
\[
	\avg T:=F_{\mathrm{ph}}\avg Q_\mathrm kg_\mathrm c,
\]
$F_{\mathrm{ph}}$为光子传输系数,则D1光电子数$n_\elc=Tn_{\mathrm{ph}}$。

PMT输出脉冲中的电荷量
%Q=E\dep Y_{\mathrm{ph}}F_{\mathrm{ph}}\avg Q_\mathrm kg_\mathrm c\avg Me,
\begin{align}
	Q=\avg n_{\mathrm{ph}}\avg T\,\avg Me.
\end{align}
\begin{compactitem}
	\item PMT的时间特性:
	\begin{compactitem}
		\item 渡越时间$t_\elc\sim\SIrange{20}{80}{ns}$:从光电子离开光阴极算起各电子到达阳极的时间。影响时滞。
		\item 渡越时间离散$\D t_\elc\sim\si{ns}$:到达阳极的每个电子都经历了不同的倍增过程和飞行距离,导致了飞行时间的涨落。
		
		影响系统的时间分辨能力;对于有机闪烁体,也影响探测器的分辨时间。
	\end{compactitem}
\end{compactitem}

\section{闪烁探测器的输出信号}

\paragraph{闪烁探测器输出信号的物理过程}
倍增电子向后级漂移,
感应电流从外回路流过。
\paragraph{闪烁探测器的输出回路}
信号可从回路取,也可从前级打拿极(D$_{n-1}$)取信号。

\subsection{电流脉冲信号}

光电倍增管的单电子响应函数$p(t)$:
\begin{compactitem}
	\item 对于大部分无机闪烁体,$p(t)$可以近似看做一个延迟的$\delta$函数;
	\item 对于有机闪烁体,$p(t)$的展宽和发光时间相当,不能看做$\delta$函数。
\end{compactitem}
无机闪烁体,一次闪烁光引起的输出电流
\[
	I(t)=\frac{n_{\mathrm{ph}}TMe}{\tau_0}\e{-(t-t_\elc)/\tau_0},\quad t\geqslant t_\elc.
\]
$\tau_0\sim\SI{250}{ns}$为发光衰减时间。

\subsection{电压脉冲信号}

输出电压信号的一般形式
\[
	V(t)=\frac1{C_0}\e{-t/R_0C_0}\int_0^tI(\tau)\e{\tau/R_0C_0}\d\tau,
\]
代入上式得
\[
	V(t)=\frac Q{C_0}\frac{R_0C_0}{R_0C_0-\tau_0}\kh{\e{-t/R_0C_0}-\e{-t/\tau_0}},
\]
\begin{compactitem}
	\item $R_0C_0\gg\tau_0$,
	\begin{align*}
		V(t)&\doteq\frac Q{C_0}\kh{\e{-t/R_0C_0}-\e{-t/\tau_0}}\\
		&=\begin{cases}
			\frac Q{C_0}\kh{1-\e{-t/\tau_0}},&t\ll\tau_0\\
			\frac Q{C_0},&\tau_0\ll t\ll R_0C_0\\
			\frac Q{C_0}\e{-t/R_0C_0},&t>R_0C_0
		\end{cases}
	\end{align*}
	\item $R_0C_0\ll\tau_0$,
	\begin{align*}
		V(t)&\doteq\frac{R_0Q}{\tau_0}\kh{\e{-t/R_0C_0}-\e{-t/\tau_0}}\\
		&=\begin{cases}
			\frac{R_0Q}{\tau_0}\kh{1-\e{-t/R_0C_0}},&t\ll\tau_0\\
			\frac {R_0Q}{\tau_0},&R_0C_0\ll t\ll \tau_0\\
			\frac {R_0Q}{\tau_0}\e{-t/\tau_0},&t>\tau_0
		\end{cases}
	\end{align*}
	\item $R_0C_0=\tau_0$,
	\[
		V(t)=\frac{Q}{C_0}\frac{t}{R_0C_0}\e{-t/R_0C_0},
	\]
	
	\item $C_0$不变,随着$R_0$的增大,幅度、脉宽都增大;
	\item $R_0$不变,随着$C_0$的增大,幅度降低、脉宽增大。
\end{compactitem}

闪烁探测器电流信号的形状总是确定的\footnote{本课程仅二例是电流形状确定的:正比计数器、闪烁体探测器。},因此弹道亏损的程度是确定的。无论$R_0C_0$怎样改变,$V\maxi\propto Q\propto E\dep$的关系总存在,可以测量射线的沉积能量。

实际中常选取$R_0C_0=\tau_0$,幅度足够大,脉宽足够窄。同时,$C_0$应尽可能小,以获得大的脉冲幅度。
\paragraph{电压脉冲型 vs 电流脉冲型工作状态}
所谓电压型脉冲信号和电流型脉冲信号,说的都是电流信号流经外回路后的电压信号。$R_0C_0$和$\tau_0$之间的大小关系,决定了是哪种脉冲信号。
\begin{compactitem}
	\item 电压型脉冲信号($R_0C_0\gg\tau_0$)的重点在于(通过脉冲幅度)分析电流信号的面积,希望电压信号的幅度够大,能抵抗后续电路的噪声,获得好的能量分辨率。
	\item 电流型脉冲信号($R_0C_0\ll\tau_0$)的重点则在于如实地反映电流信号的形状。这在PSD (Pulse Shape Discrimination)技术识别粒子种类的应用中非常重要(例如$\nton/\gamma$识别)。
\end{compactitem}

\subsection{输出信号的涨落}

光电倍增管输出电荷数$n_a$是个串级型随机变量:
\begin{compactitem}
	\item 闪烁体发出的光子数$n_{\mathrm{ph}}$,近似服从Poisson分布;
	\item 光子能否转换为第一打拿极收集到的光电子,是Bernoulli事件,期望值为$T$;
	\item 第一打拿极收集到的光电子数$n_\elc$为泊松分布;
	\item PMT的电子倍增系数$M$。
	\[
		\avg M=\avg\delta_1\avg\delta^{n-1},\implies\nu_M^2=\frac1{\avg\delta_1}\cdot\frac{\avg\delta}{\avg\delta-1},
	\]
\end{compactitem}
由$\avg n_a=\avg n_{\mathrm{ph}}\avg T\,\avg M$,可得 
\begin{align}
	\nu_{n_a}^2=\frac1{\avg n_{\mathrm{ph}}\avg T}\biggkh{1+\frac1{\avg\delta_1}\cdot\frac{\avg\delta}{\avg\delta-1}},
\end{align}
第一打拿极的倍增系数$\delta_1$大些为好,比如负电子亲和(NEA)材料。

\section{闪烁探测器的主要性能}

\paragraph{$\gamma$闪烁谱仪的组成与工作原理}
探测次级电子能谱:光电效应、Compton效应、电子对效应。

\subsection{单能\textgamma 射线的次级电子能谱}

\begin{compactitem}
	\item 小尺寸闪烁体:仅吸收次级电子的能量
	\item 大尺寸闪烁体:吸收全部次级电子、次级电磁辐射能量
	\item 中等尺寸闪烁体(实际情况):吸收次级电子能量,以一定几率吸收次级电磁辐射能量。
\end{compactitem}
还要考虑“不速之客”:来自环境的“散射”。

\textbf{务必看讲义里的图}。

\subsection{\textgamma 射线的输出脉冲幅度谱}

以上只考虑了沉积能量,尚未考虑电离过程的随机性及噪声因素:
\begin{compactitem}
	\item 沉积能量$\to$荧光光子$\to$光电子$\to$电子倍增$\to$电子学噪声,输出幅度涨落使峰宽度、边界展宽。
	\item PMT噪声与暗电流形成小幅度连续谱
\end{compactitem}

\subsection{NaI(Tl)单晶闪烁谱仪的性能}

\paragraph{能量分辨率}
\[
	\eta=2.355\sqrt{\frac1{EY_{\mathrm{ph}}T}\biggkh{1+\frac1{\avg\delta_1}\cdot\frac{\avg\delta}{\avg\delta-1}}},
\]
改善分辨率:
\begin{compactenum}
	\item 射线沉积的能量$E$增多(但此时半高宽是越大的)
	\item 选择发光效率高($Y_{\mathrm{ph}}$)的闪烁体,提升$n_{\mathrm{ph}}$;
	\item 改善光电转换效率$T$,提升$n_{\mathrm{ph}}$;
	\item 提升$\delta_1$。
\end{compactenum}
此外,高压稳定性和多道的道宽也会有所影响。
\paragraph{能量线性}单位能量输出幅度与入射粒子能量的关系
\[
	E=G\cdot Ch+E_0,
\]
理想情况:发光效率$C_{\mathrm{np}}$与入射粒子的能量无关,这样全能峰的幅度就与入射$\gamma$光子的能量成正比。

实际上:
\begin{compactitem}
	\item 由于发光效率与入射粒子种类和能量有关,因此并非线性。
	\item $\gamma$能谱只涉及电子引起的闪光,因此$\gamma$谱仪的非线性是由发光效率随电子能量不同而产生的(对NaI(Tl),在$\SIrange{0.1}{1}{MeV}$变化约15\% )。
\end{compactitem}
\paragraph{探测效率、峰总比、峰康比}平行$\gamma$束
\[
	\varepsilon=1-\e{-N\sigma D}=1-\exp\biggkh{-\NA\frac\rho{A}\sigma_\gamma D},
\]
提升探测器的探测效率:高密度($\rho$)、大($D$)、原子序数高($\sigma_\gamma/A$)。

提高本征峰效率,略
\paragraph{时间特性}
\begin{compactitem}
	\item 分辨时间:取决于输出电压脉冲信号的宽度,电压$R_0C_0$,电流$\tau_0$
	\item 时滞:主要取决于光电倍增管的渡越时间$\avg t_\elc$
	\item 时间分辨本领:主要取决于渡越时间的离散$\D\avg t_\elc$,为获得好的时间	分辨本领,须选用快速光电倍增管。
\end{compactitem}



