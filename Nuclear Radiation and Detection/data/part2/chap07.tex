\chapter{中子及中子的探测}
中子的性质:
\begin{compactitem}
	\item 质量$m_\nton=\SI{939.565300}{MeV}/c^2$
	\item 电荷为0,中性
	\item 自旋$s_\nton=\hbar/2$,Fermi子; 
	\item 磁矩$\mu_\nton=-1.913042\mu_N$
	\item 半衰期$T_{1/2}=\SI{10.183}{min}$
\end{compactitem}
中子越快,粒子性越明显。

\section{中子源}

自由中子会衰变,中子只能稳定地存在于原子核中,它必然是被束缚的。

要想把它从原子核中“解放”出来,就必须向靶核馈入能量。
\paragraph{同位素中子源}放射性核素产生的射线与轻核发生反应放出中子
\begin{compactitem}
	\item ($\alpha$,n)型:结合能+相对动能
	\item ($\gamma$,n)型:$\gamma$光子能量 
\end{compactitem}
通过自发裂变产生中子,靶核自带能量。

特点:体积小、产额有限、伴随较强的$\gamma$射线、无法开关。
\paragraph{反应堆中子源}自带能量、诱发重核裂变放出。

注量率很高,可达$\SI{E16}{/s.cm^2}$
\paragraph{加速器中子源}结合能+相对动能

利用加速器加速重带电粒子(p,d等),通过核反应来获得中子。

\section{中子与物质的相互作用}

\begin{compactitem}
	\item 引力:(除了超冷中子,$\SI{5}{m/s}$)很弱;
	\item 电磁相互作用:具有磁矩,可以与磁场发生相互作用
	\item 弱相互作用:衰变;
	\item 强相互作用:测量中子
\end{compactitem}
宏观截面(cm$^{-1}$)
\[
	\Sigma=N\sigma=\frac\rho{A}\NA\sum\nu_i\sigma_i,
\]
平均自由程$\lambda=1/\Sigma$。

中子的慢化(moderation,即减速,deceleration)对于中子的测量与应用具有重要的意义。
\begin{compactitem}
	\item 能量较高时,非弹性散射可迅速降低中子能量。但具有阈值
	\item 也可通过弹性碰撞损失能量
	\[
		\frac{E_{\nton'}}{E_\nton}=\frac{A^2+1+2A\cos\theta\CM}{(A+1)^2},
	\]
\end{compactitem}
对能量$<\SI{10}{MeV}$的中子,在质心系中可近似看作S波入射和散射,因此(单位立体角内的)出射概率和角度$\theta\CM$无关

对于能量不太高的中子,散射中子在质心系中近似是各向同性分布的,实验室系下散射中子的能量分布也是均匀的;

一次弹性散射之后,中子动能损失份额
\[
	\frac{\D E_\nton}{E_\nton}=\frac{2A}{(A+1)^2}(1-\cos\theta\CM),
\]
中子动能的平均对数能量损失
\[
	\xi=\ave{\D(\ln E_\nton)}=1+\frac{(A-1)^2}{2A}\ln\frac{A-1}{A+1}.
\]
从$E_\ini$慢化为$E_\fin$所需的平均次数
\[
	\avg N=\frac{\ln E_\ini/E_\fin}{\xi},
\]
水,聚乙烯,石蜡是很好的慢化体材料!

当中子能量降低到热运动的能量时,弹性散射无法使之能量继续降低,形成热中子峰。

\section{中子的探测}

中子是间接致电离粒子,不能直接引起探测介质的电离、激发

在探测器或探测介质内必须具备能同中子发生相互作用并产生可被探测的次级带电粒子的物质——辐射体,中子在辐射体上发生核反应、核反冲、裂变、活化等过程来产生带电粒子

中子与光子不同,它敏感于核素,而不是元素
\paragraph{核反应法}通过测量核反应后的重带电粒子,就可以实现对中子的测量。

由于反应能$\gg$中子能量,对于低能中子能量的测量是很困难的。
\paragraph{核反冲法}弹性散射中,通过记录反冲核的电离效果来探测中子。主要用于快中子探测。

选用轻核(塑料/液体闪烁体)作为中子探测器材料
\[
	E_\nuc{R}=E_\nton\frac{4m_\nton M}{(M+m_\nton)^2}\cos^2\varphi\LAB,\quad 0\leqslant\varphi\LAB\leqslant 90^\circ.
\]
质子是常用的中子反冲核素
\[
	E_\pton=E_\nton\cos^2\varphi\LAB.
\]
能量$<\SI{10}{MeV}$的中子入射时,反冲质子能谱分布可认为是均匀分布
\paragraph{核裂变法}
\[
	\nton+X\to{}^\ast X\to Y+Z+x\nton+Q,
\]
聚变能$Q\sim\SI{200}{MeV}\gg$入射中子的能量。
\begin{compactitem}
	\item 不能测量中子能量,只能测量中子注量
	\item 由于能量很大,可以很好的甄别$\gamma$射线
\end{compactitem}
许多重核只有当中子能量大于“阈值”时“才”发生裂变(严格地讲,不是阈能反应)。可以用一系列不同阈值的裂变核素来判断中子能谱,叫做阈探测器。
\paragraph{活化法}
在母核吸收中子后,余核具有放射性。

选用一些具有较高活化截面的核素,活化后的放射性核素具有较易测量的放射性。

比如
\[
	\nton+\nucli{115}{In}\to\nucli{116}{In}+\gamma,
\]
$\nucli{116}{In}$会发生$\beta$衰变,可测量$\beta$粒子的发射率
\[
	A_{\nucli{116}{In}}(t)=N_{\nucli{115}{In}}\sigma_{(\nton,\gamma)}\Phi_\nton(1-\e{-\lambda_{\nucli{116}{In}}t}),
\]
可确定中子的注量率。
\paragraph{中子灵敏度}
\begin{align}
	\eta:=\frac R\Phi,
\end{align}
$R$为反应发生率(1/s),$\Phi$为中子注量率(\si{1/cm^2.s})

$R=N_\mathrm t\sigma_0v_0n$,$\Phi=n\avg v$,故 
\[
	\eta=\frac{N_\mathrm t\sigma_0v_0}{\avg v}
\]
$\SI{20}{\celsius}$热中子
\[
	\frac{\avg v}v=1.128,
\]
故
\[
	\eta=\frac{N_\mathrm t\sigma_0}{1.128}.
\]
\iffalse
\begin{compactenum}
	\item 
\end{compactenum}
\begin{compactitem}
	\item 
\end{compactitem}
\fi

% \appendiks
