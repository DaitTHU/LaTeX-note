\chapter{辐射测量方法}
\begin{compactitem}
	\item 放射性活度测量:放射性活度、发射率
	\item 辐射粒子能量测量:粒子能量、能谱
	\item 粒子鉴别:鉴别未知粒子、区分不同粒子
	\item 辐射场测量:空间分布、注量率
	\begin{compactitem}
		\item 位置测量:入射位置、其它物理量
		\item 时间测量:入射时间、半衰期、飞行时间
		\item 辐射剂量测量:辐射能量吸收
	\end{compactitem}
\end{compactitem}

\section{放射性样品的活度测量}

相对法测量:需要一个已知活度$A_0$的标准源,在同样条件下测量标准源和被测样品的计数率$n_0,n$, 根据计数率与活度成正比,可求出样品的活度:
\[
	A=A_0\frac n{n_0},
\]
相对法测量简便,但条件苛刻:必须有一个与被测样品相同的已知活度的标准源,且测量条件必须相同。

绝对法测量:复杂,需要考虑很多影响测量的因素,但绝对测量法是活度测量的基本方法。

\subsection{影响活度测量的几个因素}

\paragraph{几何因素$f_g$}
点源情况:源的线度$r\ll$源探距离$H$,几何因素
\[
	f_g=\frac\Omega{4\pi}=\frac12\biggkh{1-\frac H{\sqrt{R^2+H^2}}}.
\]
非点源情况:$r\sim H$或$r>H$,
\[
	f_g=\frac12\biggfkh{1-\frac1{(1+a^2)^{1/2}}-\frac{3ab}{8(1+a^2)^{5/2}}+\frac{5ab^2}{16(1+a^2)^{7/2}}-\frac{35a^2b^2}{64(1+a^2)^{9/2}}}.
\]
\paragraph{本征探测效率$\varepsilon_{\mathrm{int}}$}
每个进入探测器灵敏体积的粒子被观察到的概率。

本征探测效率和探测器的种类、
%NaI(Tl) vs BGO
大小、形状、窗、
入射粒子的种类、能量、
%100keV光子 vs 1MeV光子,α粒子
入射束的形状、
记录仪器的阈值等有关。
\paragraph{吸收因素$f_a$}
\paragraph{散射因素$f_b$}
对于$\beta,\gamma$来说,散射问题很重要。
对$\alpha$来说,除了在径迹末端,散射并不重要。
\paragraph{分辨时间$f_\tau$}
死时间校正因子
\[
	f_\tau=\frac nm=1-n\tau_D.
\]
分辨时间$\tau_D$与信号脉宽、阈值、信号幅度、计数率(G-M)等有关。
\paragraph{本底计数率$n_b$}
狭义的本底计数率:无样品时测量装置的计数率;
干扰计数率:样品中其它射线的计数率。二者之和为总本底。

\subsection{\textalpha 放射性样品活度的测量}

\paragraph{小立体角法测量薄$\alpha$放射性样品}
\begin{compactitem}
	\item 探测器:塑料闪烁体,ZnS(Ag),CsI(Tl),金硅面垒探测器,薄窗正比管
	\item 源发射$\alpha$:各向同性
	\item 点源:要求源探距离远,管子长度$\sim\SI{10}{cm}$
	\item 避免吸收和散射:抽真空
\end{compactitem}
\[
	A=\frac{n-n_\mathrm b}{f_\tau f_g}.
\]
\paragraph{厚样品$\alpha$活度的相对测量法}
实际情况(如铀矿砂中$\alpha$放射性测量)中,$\alpha$的自吸收不可避免!

\subsection{低水平活度样品测量问题}

在低水平放射性测量(如环境监测、辐射防护、考古、地质学及有关生命科学的研究)中,样品的活度可能很低;相比之下,本底的活度较高,对样品的活度测量构成了干扰。

$n_0\ll n_\mathrm b$,则$n_\mathrm s\doteq n_b$探测源装置的优质因子
\begin{align}
	Q=\frac1T=\frac{\nu_{n_0}^2n_0^2}{4n_\mathrm b}.
\end{align}
由于样品的活度$\ll$本底活度,放置样品和不放置样品的测量时间应选为一样$t_\mathrm s=t_\mathrm b$。
\[
	N_0=N_\mathrm s-N_\mathrm b,
\]
如果样品没有放射性,$N_0\sim\mathrm N(0,\sqrt{2N_\mathrm b})$,为将误认为有放射性的概率控制在5\%
\[
	L_\mathrm C=1.645\sqrt{2 N_\mathrm b}
\]
如果样品有放射性,$N_0\sim\mathrm N(N_\mathrm D,\sqrt{2N_\mathrm b+N_\mathrm D})$,为将误认为没有放射性的概率控制在5\%
\[
	N_\mathrm D=L_\mathrm C+1.645\sqrt{2N_\mathrm b+N_\mathrm D}\doteq 4.653\sqrt{N_\mathrm b}+2.706,
\]
即源最小计数(minimum detectable amount, MDA)。

减小本底放射性的方法:方向准直、能量分辨率、脉冲形状甄别、磁场光子极化方向选择、时间关系、符合。

\section{符合方法}

\begin{definition}{符合}{coincidence}
	符合(coincidence)事件是同时发生的两个(或多个)事件。

	符合方法:用不同的探测器来判定(测量)两个事件的时间相关性的方法。

	真符合:同时发生、完全相关。

	偶然符合:同时发生但完全无关。(噪声、本底、不同核)

	反符合:事件同时发生、完全相关,但某个探测器起否决作用。

	延迟符合:完全相关,但不一定同时发生。
\end{definition}
粒子事件在探测器的信号有一个分辨时间$\tau_\mathrm s$,称为符合分辨时间。两个探测器在同一个$\tau_\mathrm s$内均发生的事件便可认为是符合事件。

两个事件的偶然符合计数率
\[
	n_{\mathrm{rc}}=2\tau_\mathrm sn_1n_2,
\]
以此类推,$k$重符合时的偶然符合计数率$k\tau_\mathrm s^{k-1}n_1n_2\cdots n_{k}$。
\paragraph{反符合}
举例:记录入射$\gamma$射线在探测器中能量全吸收的事件;
去除发生Compton散射且散射光子又发生逃逸的事件。
\paragraph{延时符合}
级联衰变行为,必然要求延时符合。
\paragraph{符合曲线}
信号间的时间延迟$t_\mathrm d$变化时,符合计数率$n(t_\mathrm d)$也将变化;此时得到的$n(t_\mathrm d)\vs t_\mathrm d$曲线被称为符合曲线。决定该曲线特性(形状)的因素有:
\begin{compactitem}
	\item 符合电路的工作特性(电子学分辨时间$\tau_\mathrm s$)
	\item 信号形成过程的时间离散,反映了系统的时间分辨能力
	\begin{compactitem}
		\item 快符合($\tau_\mathrm s<\SI{10}{ns}$):符合曲线宽度主要由“同步”信号的时间离散来决定;
		\item 慢符合($\tau_\mathrm s>\SI{10}{ns}$):符合曲线宽度主
		要由电子学分辨时间决定
	\end{compactitem}
	\item 信号间的时间关系($\beta,\gamma$的时间差)
\end{compactitem}
假设成形脉冲是理想的矩形波$\tau_\mathrm s$,$\FWHM=2\tau_\mathrm s$。
\paragraph{真偶符合比}
偶然符合总是存在的,而真符合却是未必存在的,它要在做了适当的延迟后才能出现。在符合测量中,我们希望看到较大的真偶符合比。

比如,真符合计数率为
\[
	n_{\mathrm{co}}=A\cdot\Omega_\beta\varepsilon_\beta\cdot\Omega_\gamma\varepsilon_\gamma,
\]
偶然符合计数率为
\[
	n_{\mathrm{rc}}=2\tau_\mathrm s n_\beta n_\gamma=2\tau_\mathrm s\cdot A\Omega_\beta\varepsilon_\beta\cdot A\Omega_\gamma\varepsilon_\gamma,
\]
真偶符合比
\[
	\frac{n_{\mathrm{co}}}{n_{\mathrm{rc}}}=\frac1{2A\tau_\mathrm s}>1,
\]
便要求$A,\tau_\mathrm s$不能太大。

\section{能量测量}

\subsection{\textgamma 能谱分析}

\paragraph{探测器的能量分辨率}见前文。
\paragraph{$\gamma$射线在探测器中沉积能量}
$\gamma\to$次级电子(光电子、Compton反冲电子、电子对效应后的正负电子) $\to$次级电子沉积能量,形成载流子。

$\gamma$射线经一次反应后,产物有且仅有三种情况:
\begin{compactitem}
	\item 光电效应:光电子+电离原子(Auger电子/X射线)
	\item Compton散射:反冲电子+散射光子
	\item 电子对效应:电子+正电子(两个湮没光子)
\end{compactitem}
产生的X射线/散射光子/湮没光子会接着发生下一次反应,这一连串反应是无法被探测器区分的,会一起沉积。

光电效应后,电离原子产生的X射线和Auger电子都有可能导致别的原子或自身电离,但并不值得担心,这会引发$Z$的减小,或空位更趋向产生于外层电子处,于是Auger电子的发射概率越来越大,即使同时有X射线产生,其能量较小,逃逸也将变得越来越难。通过多次X射线或者Auger电子,$\gamma$光子的能量就被全部沉积下来了。

次级电子携带了$\gamma$/X光子的能量,将在探测器内发生如下的反应:
\begin{compactitem}
	\item 电离与激发(产生载流子)
	\begin{compactitem}
		\item 气体:电子-离子对
		\item 闪烁体:电子-空穴对(激发的原子/分子) $\to$荧光光子%-(光电转换) 
		$\to$ D1收集到的光电子
		\item 半导体:电子-空穴对
	\end{compactitem}
	\item 轫致辐射(X射线)有可能被抓住,
	光电效应最为可能;
	\item 表面效应
	\begin{compactitem}
		\item X射线可能会从探测器中跑掉
		\item 次级电子若是在探测器的表面产生,也可能会跑掉
	\end{compactitem}
\end{compactitem}
\paragraph{$\gamma$射线探测效率}由$N,\sigma,D$共同决定。
\begin{compactitem}
	\item 高能$\gamma$射线:选择高$Z$材料,使$\sigma$变大,再辅以适当的$N$和$D$;
	\item 低能$\gamma$射线,%不论对什么$Z$,
	$\sigma$已经不小,%因此只要$N$和$D$也不是很小,则NσD就相当可观了,能够实现高的效率,此时
	不必刻意选择高$Z$和大体积探测器。
\end{compactitem}
低能X/$\gamma$能谱特点:
\begin{compactitem}
	\item 减少窗吸收:用铍窗正比计数器、铍窗Si(Li)探测器。
	\item 主要是光电效应。
	\item 有明显的X射线逃逸峰(光电峰),由探测器介质决定:I/Ge/Si逃逸。
	%I逃逸$\SI{28.6}{keV}$、Ge逃逸$\SI{9.89}{keV}$、Si逃逸$\SI{1.74}{keV}$。
\end{compactitem}
\paragraph{射线来源\& 堆积问题}
\begin{center}
	\captionof{table}{能量特征峰}
	\begin{tabular}{ccc}
		\toprule
		峰&能量&来源\\
		\midrule
		和峰&最可几峰叠加&计数率、探测器分辨时间\\
		\midrule
		全能峰&$h\nu$&\multirow{5}*{源、探测器}\\
		光电峰&$h\nu-\varepsilon_{\mathrm K}$\\%[1ex]
		Compton边缘&$h\nu-E_{\gamma'}(180^\circ)$\\
		单逃逸峰&$h\nu-m_\elc c^2$\\
		双逃逸峰&$h\nu-2m_\elc c^2$\\
		\midrule
		湮没峰&$m_\elc c^2$&\multirow{3}*{源、环境}\\
		反散射峰&$E_{\gamma'}(>\!150^\circ)$\\
		特征X射线峰&$\varepsilon_{\mathrm{K,b}}$\\
		\bottomrule
	\end{tabular}
\end{center}
\begin{remark}
	$m_\elc c^2=\SI{511}{keV},\enspace E_{\gamma'}\sim\SI{200}{keV},\enspace\varepsilon_{\mathrm{K,I/Ge/Si}}\sim\SI{10}{keV}$,
	\[
		E_{\gamma'}(\theta)=h\nu'=\frac{h\nu}{1+\alpha(1-\cos\theta)},\quad\alpha:=\frac{h\nu}{m_\elc c^2}.
	\]
\end{remark}
能量未全部沉积:
\begin{compactitem}
	\item 光电峰:光子与探测器(I/Ge/Si)发生光电效应后特征X射线跑了
	\item (多次) Compton散射坪:Compton散射光子跑了
	\item 单(双)逃逸峰:1(2)个$\SI{511}{keV}$光子跑了
	\item 次级电子轫致辐射逃逸:电子减速时,辐射光子跑了
	\item 边缘效应(次级电子逃逸):电子能量没损失完就跑了
\end{compactitem}
周围介质(屏蔽材料等)影响:
\begin{compactitem}
	\item 正电子湮没辐射:($>\SI{1.022}{MeV}$的)光子在探测器外的材料中发生电子对效应,正电子湮没后产生的$\SI{511}{keV}$光子
	\item 轫致辐射
	\item 散射光子 
	\item 反散射峰:光子在探测器外发生Compton反散射,产生了$\sim\SI{200}{keV}$的光子
	\item 特征X射线
\end{compactitem}

\subsection{\textgamma 能谱装置}

\paragraph{单晶$\gamma$能谱仪}

\paragraph{全吸收反Compton $\gamma$能谱仪}

\paragraph{Compton谱仪(双晶谱仪)}
全能峰+ Compton反冲电子能量
\paragraph{电子对谱仪(三晶谱仪)}
双逃逸峰

